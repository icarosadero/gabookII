%
% Copyright � 2012 Peeter Joot.  All Rights Reserved.
% Licenced as described in the file LICENSE under the root directory of this GIT repository.
%

%
%
\mychapter{Lorentz force rotor formulation}
\index{Lorentz force!rotor}
\label{chap:electronRotor}
%\date{March 18, 2009.  electronRotor.tex}

\section{Motivation}

Both \citep{baylis-2007} and \citep{doran2003gap} cover rotor formulations
of the Lorentz force equation.  Work through some of this on my own to
better understand it.

\section{In terms of GA}

An active Lorentz transformation can be used to translate from the rest frame of a particle with worldline \(x\) to
an observer frame, as in

\begin{equation}\label{eqn:eRotor:LorentzTx}
\begin{aligned}
y &= \Lambda x \reverse{\Lambda}
\end{aligned}
\end{equation}

Here Lorentz transformation is used in the general sense, and can include both spatial rotation and boost effects, but satisfies \(\Lambda\reverse{\Lambda} = 1\).  Taking proper time derivatives we have

\begin{equation}\label{eqn:electronRotor:20}
\begin{aligned}
\ydot
&= \Lambdadot x \reverse{\Lambda} + \Lambda x \reverse{\Lambdadot} \\
&= \Lambda \left(\reverse{\Lambda} \Lambdadot\right) x \reverse{\Lambda} + \Lambda x \left(\reverse{\Lambdadot} \Lambda \right) \reverse{\Lambda} \\
\end{aligned}
\end{equation}

Since \(\reverse{\Lambda}\Lambda = \Lambda\reverse{\Lambda} = 1\) we also have

\begin{equation}\label{eqn:electronRotor:40}
\begin{aligned}
0 &= \Lambdadot\reverse{\Lambda} + \Lambda\reverse{\Lambdadot}  \\
0 &= \reverse{\Lambda}\Lambdadot + \reverse{\Lambdadot}\Lambda
\end{aligned}
\end{equation}

Here is where a bivector variable

\begin{equation}\label{eqn:electronRotor:60}
\begin{aligned}
\Omega/2 = \reverse{\Lambda} \Lambdadot
\end{aligned}
\end{equation}

is introduced, from which we have \(\reverse{\Lambdadot} \Lambda = -\Omega/2\), and

\begin{equation}\label{eqn:electronRotor:80}
\begin{aligned}
\ydot &= \inv{2} \left( \Lambda \Omega x \reverse{\Lambda} - \Lambda x \Omega \reverse{\Lambda} \right) \\
\end{aligned}
\end{equation}

Or
\begin{equation}\label{eqn:electronRotor:100}
\begin{aligned}
\reverse{\Lambda} \ydot \Lambda &= \inv{2} \left( \Omega x - x \Omega \right) \\
\end{aligned}
\end{equation}

The inclusion of the factor of two in the definition of \(\Omega\) was cheating, so that we get the bivector vector dot product above.  Presuming \(\Omega\) is really a bivector (return to this in a bit), we then have

\begin{equation}\label{eqn:electronRotor:120}
\begin{aligned}
\reverse{\Lambda} \ydot \Lambda &= \Omega \cdot x
\end{aligned}
\end{equation}

We can express the time evolution of \(y\) using this as a stepping stone, since we have

\begin{equation}\label{eqn:electronRotor:140}
\begin{aligned}
\reverse{\Lambda} y \Lambda &= x
\end{aligned}
\end{equation}

Using this we have
\begin{equation}\label{eqn:electronRotor:160}
\begin{aligned}
0
&= \gpgradeone{ \reverse{\Lambda} \ydot \Lambda - \Omega \cdot x } \\
&= \gpgradeone{ \reverse{\Lambda} \ydot \Lambda - \Omega x } \\
&= \gpgradeone{ \reverse{\Lambda} \ydot \Lambda - \Omega \reverse{\Lambda} y \Lambda } \\
&= \gpgradeone{ \left( \reverse{\Lambda} \ydot - \reverse{\Lambda} \Lambda \Omega \reverse{\Lambda} y \right) \Lambda } \\
&= \gpgradeone{ \reverse{\Lambda} \left( \ydot - \Lambda \Omega \reverse{\Lambda} y \right) \Lambda } \\
\end{aligned}
\end{equation}

So we have the complete time evolution of our observer frame worldline for the particle, as a sort of an eigenvalue
equation for the proper time differential operator

\begin{equation}\label{eqn:electronRotor:180}
\begin{aligned}
\ydot
&= \left( \Lambda \Omega \reverse{\Lambda} \right) \cdot y = \left( 2 \Lambdadot \reverse{\Lambda} \right) \cdot y
\end{aligned}
\end{equation}

Now, what
\href{http://www.ime.unicamp.br/%7Eicca8/videos/baylis.avi}{Baylis did in his lecture}, and what Doran/Lasenby did as
well in the text (but I did not understand it then when I read it the first time) was to identify this time evolution
in terms of Lorentz transform change with the Lorentz force.

Recall that the Lorentz force equation is

\begin{equation}\label{eqn:eRotor:LorentzForce}
\begin{aligned}
\vdot = \frac{e}{m c} F \cdot v
\end{aligned}
\end{equation}

where \(F = \BE + i c \BB\), like \(\Lambdadot\reverse{\Lambda}\) is also a bivector.  If we write the velocity worldline
of the particle in the lab frame in terms of the rest frame particle worldline as

\begin{equation}\label{eqn:electronRotor:200}
\begin{aligned}
v = \Lambda c t \gamma_0 \reverse{\Lambda}
\end{aligned}
\end{equation}

Then for the field \(F\) observed in the lab frame we are left with a differential equation
\(2 \Lambdadot \reverse{\Lambda} = e F / mc\)
for the Lorentz transformation
that produces the observed motion of the particle given the field that acts on it

\begin{equation}\label{eqn:eRotor:LorentzTxEvolution}
\begin{aligned}
\Lambdadot = \frac{e}{2 m c} F \Lambda
\end{aligned}
\end{equation}

Okay, good.  I understand now well enough what they have done to reproduce the end result (with the exception of my
result including a factor of \(c\) since they have worked with \(c=1\)).

\subsection{Omega bivector}

It has been assumed above that \(\Omega = 2 \reverse{\Lambda} \Lambdadot\) is a bivector.  One way to confirm this is by examining the grades of this product.  Two bivectors, not necessarily related can only have grades 0, 2, and 4.  Because \(\Omega = -\reverse{\Omega}\), as seen above, it can have no grade 0 or grade 4 parts.

While this is a powerful way to verify the bivector nature of this object it is fairly abstract.  To get a better feel for this, let us
consider this object in detail for a purely spatial rotation, such as
\begin{equation}\label{eqn:electronRotor:220}
\begin{aligned}
R_\theta(x) &= \Lambda x \reverse{\Lambda} \\
\Lambda &= \exp( -i n \theta/ 2 ) = \cos( \theta/ 2 ) - i n \sin( \theta/ 2 ),
\end{aligned}
\end{equation}
where \(n\) is a spatial unit bivector, \(n^2 = 1\), in the span of \(\{\sigma_k = \gamma_k \gamma_0\}\).

\subsubsection{Verify rotation form}
To verify that this has the appropriate action, by linearity two cases must be considered.
First is the action on \(n\) or the components of any vector in this direction.
\begin{equation}\label{eqn:electronRotor:240}
\begin{aligned}
R_\theta(n)
&= \Lambda n \reverse{\Lambda} \\
&= \left( \cos( \theta/ 2 ) - i n \sin( \theta/ 2 ) \right) n \reverse{\Lambda} \\
&= n \left( \cos( \theta/ 2 ) - i n \sin( \theta/ 2 ) \right) \reverse{\Lambda} \\
&= n \Lambda \reverse{\Lambda} \\
&= n.
\end{aligned}
\end{equation}

The rotation operator does not change any vector colinear with the axis of rotation (the normal).  For a
vector \(m\) that is perpendicular to axis of rotation \(n\) (ie: \(2 ( m \cdot n ) = mn + nm = 0 \)), we have
\begin{equation}\label{eqn:electronRotor:260}
\begin{aligned}
R_\theta(m)
&= \Lambda m \reverse{\Lambda} \\
&= \left( \cos( \theta/ 2 ) - i n \sin( \theta/ 2 ) \right) m \reverse{\Lambda} \\
&= \left( m \cos( \theta/ 2 ) - i (n m) \sin( \theta/ 2 ) \right) \reverse{\Lambda} \\
&= \left( m \cos( \theta/ 2 ) + i (m n) \sin( \theta/ 2 ) \right) \reverse{\Lambda} \\
&= m (\reverse{\Lambda})^2 \\
&= m \exp( i n \theta )
\end{aligned}
\end{equation}

This is a rotation of the vector \(m\) that lies in the \(i n\) plane by \(\theta\) as desired.

\subsubsection{The rotation bivector}

We want derivatives of the \(\Lambda\) object.

\begin{equation}\label{eqn:electronRotor:280}
\begin{aligned}
\Lambdadot
&= \frac{\thetadot}{2} \left( -\sin( \theta/ 2 ) - i n \cos( \theta/ 2 ) \right) - i \ndot \cos(\theta/2) \\
&= \frac{i n \thetadot}{2} \left( i n \sin( \theta/ 2 ) - \cos( \theta/ 2 ) \right) - i \ndot \cos(\theta/2) \\
&= -\inv{2} \exp( -i n \theta/2 ) {i n \thetadot} - i \ndot \cos(\theta/2) \\
\end{aligned}
\end{equation}

So we have

\begin{equation}\label{eqn:electronRotor:300}
\begin{aligned}
\Omega
&= 2 \reverse{\Lambda} \Lambdadot \\
&= -{i n \thetadot} - 2 \exp(i n \theta/2) i \ndot \cos(\theta/2) \\
&= -{i n \thetadot} - 2 \cos(\theta/2) \left( \cos(\theta/2) - i n \sin(\theta/2)  \right) i \ndot \\
&= -{i n \thetadot} - 2 \cos(\theta/2) \left( \cos(\theta/2) i \ndot + n \ndot \sin(\theta/2)  \right) \\
\end{aligned}
\end{equation}

Since \(n \cdot \ndot = 0\), we have \(n \ndot = n \wedge \ndot\), and sure enough all the terms are bivectors.  Specifically
we have

\begin{equation}\label{eqn:electronRotor:320}
\begin{aligned}
\Omega
&= -\thetadot(i n) - (1 + \cos\theta ) (i \ndot) - \sin\theta (n \wedge \ndot)
\end{aligned}
\end{equation}

\subsection{Omega bivector for boost}

TODO.

\section{Tensor variation of the Rotor Lorentz force result}

There is not anything in the initial Lorentz force rotor result that intrinsically requires geometric algebra.  At least until
one actually
wants to express the Lorentz transformation concisely in terms of half angle or boost rapidity exponentials.

In fact
the logic above is not much different than the approach used in \citep{TongDynamics} for rigid body motion.  Let us try this in matrix or tensor
form and see how it looks.

\subsection{Tensor setup}
\index{Lorentz force!tensor}

Before anything else some notation for the tensor work must be established.  Similar to \eqnref{eqn:eRotor:LorentzTx} write a Lorentz transformed vector as a
linear transformation.  Since we want only the matrix of this linear transformation with respect to a specific observer frame, the details
of the transformation can be omitted for now.  Write

\begin{equation}\label{eqn:electronRotor:340}
\begin{aligned}
y = \LL(x)
\end{aligned}
\end{equation}

and introduce an orthonormal frame \(\{\gamma_\mu\}\), and the corresponding reciprocal frame
\(\{\gamma^\mu\}\), where \(\gamma_\mu \cdot \gamma^\nu = {\delta_\mu}^\nu\).
In this basis, the relationship between the vectors becomes

\begin{equation}\label{eqn:electronRotor:360}
\begin{aligned}
y^\mu \gamma_\mu
&= \LL(x^\nu \gamma_\nu) \\
&= x^\nu \LL(\gamma_\nu) \\
\end{aligned}
\end{equation}

Or
\begin{equation}\label{eqn:electronRotor:380}
\begin{aligned}
y^\mu &= x^\nu \LL(\gamma_\nu) \cdot \gamma^\mu \\
\end{aligned}
\end{equation}

The matrix of the linear transformation can now be written as

\begin{equation}\label{eqn:electronRotor:400}
\begin{aligned}
{\Lambda_\nu}^\mu &= \LL(\gamma_\nu) \cdot \gamma^\mu
\end{aligned}
\end{equation}

and this can now be used to express the coordinate transformation in abstract index notation

\begin{equation}\label{eqn:electronRotor:420}
\begin{aligned}
y^\mu &= x^\nu {\Lambda_\nu}^\mu
\end{aligned}
\end{equation}

Similarly, for the inverse transformation, we can write

\begin{equation}\label{eqn:electronRotor:440}
\begin{aligned}
x &= \LL^{-1}(y) \\
{\ILambda_\nu}^\mu &= \LL^{-1}(\gamma_\nu) \cdot \gamma^\mu \\
x^\mu &= y^\nu {\ILambda_\nu}^\mu
\end{aligned}
\end{equation}

I have seen this expressed using primed indices and the same symbol \(\Lambda\) used for both the forward and inverse
transformation ... lacking skill in tricky index manipulation I have avoided such a notation because I will probably get it
wrong.  Instead different symbols for the two different matrices will be used here and \(\Pi\) was picked for the inverse
rather arbitrarily.

With substitution

\begin{equation}\label{eqn:electronRotor:460}
\begin{aligned}
y^\mu &= x^\nu {\Lambda_\nu}^\mu = (y^\alpha {\ILambda_\alpha}^\nu) {\Lambda_\nu}^\mu  \\
x^\mu &= y^\nu {\ILambda_\nu}^\mu = (x^\alpha {\Lambda_\alpha}^\nu) {\ILambda_\nu}^\mu
\end{aligned}
\end{equation}

the pair of explicit inverse relationships between the two matrices can be read off as

\begin{equation}\label{eqn:electronRotor:480}
\begin{aligned}
{\delta_\alpha}^\mu &= {\ILambda_\alpha}^\nu {\Lambda_\nu}^\mu = {\Lambda_\alpha}^\nu {\ILambda_\nu}^\mu
\end{aligned}
\end{equation}

\subsection{Lab frame velocity of particle in tensor form}

In tensor form we want to express the worldline of the particle in the lab frame coordinates.  That is

\begin{equation}\label{eqn:electronRotor:500}
\begin{aligned}
v
&= \LL(c t \gamma_0) \\
&= \LL(x^0 \gamma_0) \\
&= x^0 \LL(\gamma_0) \\
\end{aligned}
\end{equation}

Or
\begin{equation}\label{eqn:electronRotor:520}
\begin{aligned}
v^\mu
&= x^0 \LL(\gamma_0) \cdot \gamma^\mu \\
&= x^0 {\Lambda_0}^\mu
\end{aligned}
\end{equation}

\subsection{Lorentz force in tensor form}

The Lorentz force equation \eqnref{eqn:eRotor:LorentzForce} in tensor form will also be needed.  The bivector \(F\) is

\begin{equation}\label{eqn:electronRotor:540}
\begin{aligned}
F = \inv{2} F_{\mu\nu} \gamma^\mu \wedge \gamma^\nu
\end{aligned}
\end{equation}

So we can write

\begin{equation}\label{eqn:electronRotor:560}
\begin{aligned}
F \cdot v
&= \inv{2} F_{\mu\nu} (\gamma^\mu \wedge \gamma^\nu) \cdot \gamma_\alpha v^\alpha \\
&= \inv{2} F_{\mu\nu} (\gamma^\mu {\delta^\nu}_\alpha - \gamma^\nu {\delta^\mu}_\alpha) v^\alpha \\
&= \inv{2} (v^\alpha F_{\mu\alpha} \gamma^\mu -v^\alpha F_{\alpha\nu} \gamma^\nu )
\end{aligned}
\end{equation}

And
\begin{equation}\label{eqn:electronRotor:580}
\begin{aligned}
\vdot_\sigma
&= \frac{e}{m c} ( F \cdot v ) \cdot \gamma_\sigma \\
&= \frac{e}{2 m c} (v^\alpha F_{\mu\alpha} \gamma^\mu -v^\alpha F_{\alpha\nu} \gamma^\nu ) \cdot \gamma_\sigma \\
&= \frac{e}{2 m c} v^\alpha ( F_{\sigma\alpha} - F_{\alpha\sigma} ) \\
&= \frac{e}{m c} v^\alpha F_{\sigma\alpha} \\
\end{aligned}
\end{equation}

Or

\begin{equation}\label{eqn:electronRotor:600}
\begin{aligned}
\vdot^\sigma &= \frac{e}{m c} v^\alpha {F^\sigma}_\alpha
\end{aligned}
\end{equation}

\subsection{Evolution of Lab frame vector}

Given a lab frame vector with all the (proper) time evolution expressed via the Lorentz transformation

\begin{equation}\label{eqn:electronRotor:620}
\begin{aligned}
y^\mu
&= x^\nu {\Lambda_\nu}^\mu \\
\end{aligned}
\end{equation}

we want to calculate the derivatives as in the GA procedure

\begin{equation}\label{eqn:electronRotor:640}
\begin{aligned}
\ydot^\mu
&= x^\nu {\Lambdadot_\nu}^\mu \\
&= x^\alpha {\delta_\alpha}^\nu {\Lambdadot_\nu}^\mu \\
&= x^\alpha {\Lambda_\alpha}^\beta {\ILambda_\beta}^\nu {\Lambdadot_\nu}^\mu \\
\end{aligned}
\end{equation}

With \(y = v\), this is

\begin{equation}\label{eqn:electronRotor:660}
\begin{aligned}
\vdot^\sigma
&= v^\alpha {\ILambda_\alpha}^\nu {\Lambdadot_\nu}^\sigma \\
&= v^\alpha \frac{e}{m c} {F^\sigma}_\alpha
\end{aligned}
\end{equation}

So we can make the identification of the bivector field with the Lorentz transformation matrix

\begin{equation}\label{eqn:electronRotor:680}
\begin{aligned}
{\ILambda_\alpha}^\nu {\Lambdadot_\nu}^\sigma &= \frac{e}{m c} {F^\sigma}_\alpha
\end{aligned}
\end{equation}

With an additional summation to invert we have
\begin{equation}\label{eqn:electronRotor:700}
\begin{aligned}
{\Lambda_\beta}^\alpha {\ILambda_\alpha}^\nu {\Lambdadot_\nu}^\sigma &= {\Lambda_\beta}^\alpha \frac{e}{m c} {F^\sigma}_\alpha
\end{aligned}
\end{equation}
%{\delta_\beta}^\nu &= {\ILambda_\beta}^\alpha {\Lambda_\alpha}^\nu = {\Lambda_\beta}^\alpha {\ILambda_\alpha}^\nu

This leaves a tensor differential equation that will provide the complete time evolution of the lab frame worldline for the particle in the field

\begin{equation}\label{eqn:electronRotor:720}
\begin{aligned}
{{\Lambdadot}_\mu}^\nu &= \frac{e}{m c} {\Lambda_\mu}^\alpha {F^\nu}_\alpha
\end{aligned}
\end{equation}

This is the equivalent of the GA equation \eqnref{eqn:eRotor:LorentzTxEvolution}.  However, while the GA equation is directly integrable for constant \(F\), how to do this in the equivalent tensor formulation is not so clear.

Want to revisit this, and try to perform this integral in both forms, ideally
for both the simpler constant field case, as well as for a more general field.
Even better would be to be able to express \(F\) in terms of the current
density vector, and then treat the proper interaction of two charged particles.

\section{Gauge transformation for spin}

In the Baylis article \eqnref{eqn:eRotor:LorentzTxEvolution} is transformed as
\(\Lambda \rightarrow \Lambda_{\omega_0} \exp( -i \Be_3 \omega_0 \tau)\).

Using this we have

\begin{equation}\label{eqn:electronRotor:740}
\begin{aligned}
\Lambdadot
&\rightarrow \frac{d}{d\tau}\left(\Lambda_{\omega_0} \exp( -i \Be_3 \omega_0 \tau) \right) \\
&= \Lambdadot_{\omega_0} \exp( -i \Be_3 \omega_0 \tau)
- \Lambda_{\omega_0} ( i \Be_3 \omega_0 ) \exp( -i \Be_3 \omega_0 \tau)
\end{aligned}
\end{equation}

For the transformed \eqnref{eqn:eRotor:LorentzTxEvolution} this gives

\begin{equation}\label{eqn:electronRotor:760}
\begin{aligned}
\Lambdadot_{\omega_0} \exp( -i \Be_3 \omega_0 \tau)
- \Lambda_{\omega_0} ( i \Be_3 \omega_0 ) \exp( -i \Be_3 \omega_0 \tau)
&= \frac{e}{2 m c} F \Lambda_{\omega_0} \exp( -i \Be_3 \omega_0 \tau)
\end{aligned}
\end{equation}

Canceling the exponentials, and shuffling

\begin{equation}\label{eqn:eRotor:firstTry}
\begin{aligned}
\Lambdadot_{\omega_0} &= \frac{e}{2 m c} F \Lambda_{\omega_0} + \Lambda_{\omega_0} ( i \Be_3 \omega_0 )
\end{aligned}
\end{equation}

How does he commute the \(i\Be_3\) term with the Lorentz transform?  How about instead
transforming as
\(\Lambda \rightarrow \exp( -i \Be_3 \omega_0 \tau) \Lambda_{\omega_0}\).

Using this we have

\begin{equation}\label{eqn:electronRotor:780}
\begin{aligned}
\Lambdadot
&\rightarrow \frac{d}{d\tau}\left(
\exp( -i \Be_3 \omega_0 \tau)
\Lambda_{\omega_0}
\right) \\
&=
\exp( -i \Be_3 \omega_0 \tau)
\Lambdadot_{\omega_0}
-
( i \Be_3 \omega_0 ) \exp( -i \Be_3 \omega_0 \tau)
\Lambda_{\omega_0}
\end{aligned}
\end{equation}

then, the transformed \eqnref{eqn:eRotor:LorentzTxEvolution} gives

\begin{equation}\label{eqn:electronRotor:800}
\begin{aligned}
\exp( -i \Be_3 \omega_0 \tau)
\Lambdadot_{\omega_0}
-
( i \Be_3 \omega_0 ) \exp( -i \Be_3 \omega_0 \tau)
\Lambda_{\omega_0}
&= \frac{e}{2 m c} F
\exp( -i \Be_3 \omega_0 \tau)
\Lambda_{\omega_0}
\end{aligned}
\end{equation}

Multiplying by the inverse exponential, and shuffling, noting that \(\exp(i\Be_3\alpha)\) commutes with \(i\Be_3\), we have

\begin{equation}\label{eqn:electronRotor:820}
\begin{aligned}
\Lambdadot_{\omega_0}
&=
( i \Be_3 \omega_0 ) \Lambda_{\omega_0}
+ \frac{e}{2 m c}
\exp( i \Be_3 \omega_0 \tau)
F
\exp( -i \Be_3 \omega_0 \tau)
\Lambda_{\omega_0}  \\
&=
\frac{e}{2 m c} \left(
\frac{2 m c}{e} ( i \Be_3 \omega_0 )
+
\exp( i \Be_3 \omega_0 \tau)
F
\exp( -i \Be_3 \omega_0 \tau)
\right)
\Lambda_{\omega_0}
\end{aligned}
\end{equation}

So, if one writes \(F_{\omega_0} = \exp( i \Be_3 \omega_0 \tau) F \exp( -i \Be_3 \omega_0 \tau)\), then
the transformed differential equation for the Lorentz transformation takes the form

\begin{equation}\label{eqn:electronRotor:840}
\begin{aligned}
\Lambdadot_{\omega_0}
&=
\frac{e}{2 m c} \left(
\frac{2 m c}{e} ( i \Be_3 \omega_0 )
+
F_{\omega_0}
\right)
\Lambda_{\omega_0}
\end{aligned}
\end{equation}

This is closer to Baylis's equation 31.
Dropping \(\omega_0\) subscripts this is

\begin{equation}\label{eqn:electronRotor:860}
\begin{aligned}
\Lambdadot
&=
\frac{e}{2 m c} \left(
\frac{2 m c}{e} ( i \Be_3 \omega_0 )
+
F
\right)
\Lambda
\end{aligned}
\end{equation}

A phase change in the Lorentz transformation rotor has introduced an additional term, one that
Baylis appears to identify with the spin vector \(\BS\).  My way of getting there seems fishy, so I think that
I am missing something.

Ah, I see.  If we go back to \eqnref{eqn:eRotor:firstTry}, then with
\(\BS = \Lambda_{\omega_0} ( i \Be_3 ) \reverse{\Lambda}_{\omega_0}\) (an application of a Lorentz transform to the unit bivector for the \(\Be_2 \Be_3\) plane), one has

\begin{equation}\label{eqn:electronRotor:880}
\begin{aligned}
\Lambdadot_{\omega_0}
&= \inv{2} \left( \frac{e}{m c} F + 2 \omega_0 \BS \right) \Lambda_{\omega_0}
\end{aligned}
\end{equation}
