%
% Copyright � 2012 Peeter Joot.  All Rights Reserved.
% Licenced as described in the file LICENSE under the root directory of this GIT repository.
%

%
%
%\input{../peeter_prologue.tex}

\mychapter{(INCOMPLETE) Geometry of Maxwell radiation solutions}
\label{chap:radiationGeometry}

%\blogpage{http://sites.google.com/site/peeterjoot/math2009/radiationGeometry.pdf}
%\date{Aug 14, 2009}
%\revisionInfo{\(RCSfile: radiationGeometry.tex,v \) Last \(Revision: 1.10 \) \(Date: 2009/10/22 02:07:20 \)}

\beginArtWithToc
%\beginArtNoToc

\section{Motivation}

We have in GA multiple possible ways to parametrize an oscillatory time dependence for a radiation field.

This was going to be an attempt to systematically solve the resulting eigen-multivector problem, starting with the a \(I\zcap \omega t\) exponential time parametrization, but I got stuck part way.  Perhaps using a plain old \(I \omega t\) would work out better, but I have spent more time on this than I want for now.

\section{Setup.  The eigenvalue problem}

Again following Jackson \citep{jackson1975cew}, we use CGS units.  Maxwell's equation in these units, with \(F = \BE + I\BB/\sqrt{\mu\epsilon}\) is
%
\begin{equation}\label{eqn:radiationGeometry:foo1}
\begin{aligned}
0 &= (\spacegrad + \sqrt{\mu\epsilon} \partial_0) F
\end{aligned}
\end{equation}
%
With an assumed oscillatory time dependence
%
\begin{equation}\label{eqn:radiationGeometry:foo2}
\begin{aligned}
F =
\calF
e^{i\omega t}
\end{aligned}
\end{equation}
%
Maxwell's equation reduces to a multivariable eigenvalue problem
%
\begin{equation}\label{eqn:radiationGeometry:foo3}
\begin{aligned}
\spacegrad \calF &= - \calF i \lambda \\
\lambda &= \sqrt{\mu\epsilon} \frac{\omega}{c}
\end{aligned}
\end{equation}
%
We have some flexibility in picking the imaginary.  As well as a non-geometric imaginary \(i\) typically used for a phasor representation where we take real parts of the field, we have additional possibilities, two of which are
%
\begin{equation}\label{eqn:radiationGeometry:foo4}
\begin{aligned}
i &= \xcap\ycap\zcap = I \\
i &= \xcap \ycap = I \zcap
\end{aligned}
\end{equation}
%
The first is the spatial pseudoscalar, which commutes with all vectors and bivectors.  The second is the unit bivector for the transverse plane, here parametrized by duality using the perpendicular to the plane direction \(\zcap\).

Let us examine the geometry required of the object \(\calF\) for each of these two geometric modeling choices.

\section{Using the transverse plane bivector for the imaginary}

Assuming no prior assumptions about \(\calF\) let us allow for the possibility of scalar, vector, bivector and pseudoscalar components
%
\begin{equation}\label{eqn:radiationGeometry:foo5}
\begin{aligned}
F = e^{-I\zcap \omega t} ( F_0 + F_1 + F_2 + F_3 )
\end{aligned}
\end{equation}
%
Writing \(e^{-I\zcap \omega t} = \cos(\omega t) -I \zcap \sin(\omega t) = C_\omega -I \zcap S_\omega\), an expansion of this product separated into grades is
%
\begin{equation}\label{eqn:radiationGeometry:29}
\begin{aligned}
F &=
  C_\omega F_0 - I S_\omega (\zcap \wedge F_2) \\
&+ C_\omega F_1 - \zcap S_\omega (I F_3) + S_\omega (\zcap \cross F_1)  \\
&+ C_\omega F_2 - I \zcap S_\omega F_0 - I S_\omega (\zcap \cdot F_2) \\
&+ C_\omega F_3 - I S_\omega (\zcap \cdot F_1)
\end{aligned}
\end{equation}
%
By construction \(F\) has only vector and bivector grades, so a requirement for zero scalar and pseudoscalar for all \(t\) means that we have four immediate constraints (with \(\Bn \perp \zcap\).)
%
\begin{equation}\label{eqn:radiationGeometry:49}
\begin{aligned}
F_0 &= 0 & \\
F_3 &= 0 & \\
F_2 &= \zcap \wedge \Bm \\
F_1 &= \Bn
\end{aligned}
\end{equation}
%
Since we have the flexibility to add or subtract any scalar multiple of \(\zcap\) to \(\Bm\) we can write \(F_2 = \zcap \Bm\) where \(\Bm \perp \zcap\).  Our field can now be written as just
%
\begin{equation}\label{eqn:radiationGeometry:69}
\begin{aligned}
F &=
 C_\omega \Bn - I S_\omega (\zcap \wedge \Bn)  \\
&+ C_\omega \zcap \Bm - I S_\omega (\zcap \cdot (\zcap \Bm)) \\
\end{aligned}
\end{equation}
%
We can similarly require \(\Bn \perp \zcap\), leaving
%
\begin{equation}\label{eqn:radiationGeometry:foo7}
\begin{aligned}
F &= (C_\omega - I \zcap S_\omega ) \Bn  + (C_\omega - I \zcap S_\omega) \Bm \zcap
\end{aligned}
\end{equation}
%
So, just the geometrical constraints give us
%
\begin{equation}\label{eqn:radiationGeometry:foo6}
\begin{aligned}
F &= e^{-I\zcap \omega t}(\Bn + \Bm \zcap)
\end{aligned}
\end{equation}
%
The first thing to be noted is that this phasor representation utilizing for the imaginary the transverse plane bivector \(I\zcap\) cannot be the most general.  This representation allows for only transverse fields!  This can be seen two ways.  Computing the transverse and propagation field components we have
%
\begin{equation}\label{eqn:radiationGeometry:89}
\begin{aligned}
F_z
&= \inv{2}(F + \zcap F \zcap) \\
&=
\inv{2} e^{-I\zcap \omega t}( \Bn + \Bm \zcap + \zcap \Bn \zcap + \zcap \Bm \zcap \zcap) \\
&=
\inv{2} e^{-I\zcap \omega t}( \Bn + \Bm \zcap - \Bn - \Bm \zcap ) \\
&= 0
\end{aligned}
\end{equation}
%
The computation for the transverse field \(F_t = (F - \zcap F \zcap)/2\) shows that \(F = F_t\) as expected since the propagation component is zero.

Another way to observe this is from the split of \(F\) into electric and magnetic field components.  From \eqnref{eqn:radiationGeometry:foo7} we have
%
\begin{equation}\label{eqn:radiationGeometry:foo8}
\begin{aligned}
\BE &= \cos(\omega t) \Bm + \sin(\omega t) (\zcap \cross \Bm) \\
\BB &= \cos(\omega t) (\zcap \cross \Bn) - \sin(\omega t) \Bn
\end{aligned}
\end{equation}
%
The space containing each of the \(\BE\) and \(\BB\) vectors lies in the span of the transverse plane.  We also see that there is some potential redundancy in the representation visible here since we have four vectors describing this span \(\Bm\), \(\Bn\), \(\zcap \cross \Bm\), and \(\zcap \cross \Bn\), instead of just two.

\subsection{General wave packet}

If \eqnref{eqn:radiationGeometry:foo1} were a scalar equation for \(F(\Bx,t)\) it can be readily shown using Fourier transforms the field propagation in time given initial time description of the field is
%
\begin{equation}\label{eqn:radiationGeometry:foob1}
\begin{aligned}
F(\Bx, t) = \int \left( \inv{(2\pi)^3} \int F(\Bx', 0) e^{i\Bk \cdot (\Bx' -\Bx)} d^3 x \right) e^{i c \Bk t/ \sqrt{\mu\epsilon}} d^3 k
\end{aligned}
\end{equation}
%
In traditional complex algebra the vector exponentials would not be well formed.  We do not have the problem in the GA formalism, but this does lead to a contraction since the resulting \(F(\Bx,t)\) cannot be scalar valued.  However, by using this as a motivational tool, and
also using assumed structure for the discrete frequency infinite wavetrain phasor, we can guess that a transverse only (to \(z\)-axis) wave packet may be described by a single direction variant of the Fourier result above.  That is
%
\begin{equation}\label{eqn:radiationGeometry:foob2}
\begin{aligned}
F(\Bx, t) =
\inv{\sqrt{2\pi}} \int
e^{-I \zcap \omega t}
\calF(\Bx, \omega)
d\omega
\end{aligned}
\end{equation}
%
Since \eqnref{eqn:radiationGeometry:foob2} has the same form as the earlier single frequency phasor test solution, we now know that \(\calF\) is required to anticommute with \(\zcap\).  Application of Maxwell's equation to this test solution gives us
%
\begin{equation}\label{eqn:radiationGeometry:109}
\begin{aligned}
%\begin{align}\label{eqn:radiationGeometry:foob3}
(\spacegrad +\sqrt{\mu\epsilon} \partial_0) F(\Bx,t) &=
(\spacegrad +\sqrt{\mu\epsilon} \partial_0)
\inv{\sqrt{2\pi}} \int
\calF(\Bx, \omega)
e^{I \zcap \omega t}
d\omega \\
&=
\inv{\sqrt{2\pi}}\int
\left(\spacegrad \calF + \calF I \zcap \sqrt{\mu\epsilon} \frac{\omega}{c}\right)
e^{I \zcap \omega t}
d\omega
\end{aligned}
\end{equation}
%\end{align}
%
This means that \(\calF\) must satisfy the gradient eigenvalue equation for all \(\omega\)
%
\begin{equation}\label{eqn:radiationGeometry:foob4}
\begin{aligned}
\spacegrad \calF = -\calF I \zcap \sqrt{\mu\epsilon} \frac{\omega}{c}
\end{aligned}
\end{equation}
%
Observe that this is the single frequency problem of equation \eqnref{eqn:radiationGeometry:foo3}, so for mono-directional light we can consider the infinite wave train instead of a wave packet with no loss of generality.

\subsection{Applying separation of variables}

While this may not lead to the most general solution to the radiation problem, the transverse only propagation problem is still one of interest.  Let us see where this leads.  In order to reduce the scope of the problem by one degree of freedom, let us split out the \(\zcap\) component of the gradient, writing
%
\begin{equation}\label{eqn:radiationGeometry:fooc1}
\begin{aligned}
\spacegrad = \spacegrad_t + \zcap \partial_z
\end{aligned}
\end{equation}
%
Also introduce a product split for separation of variables for the \(z\) dependence.  That is
%
\begin{equation}\label{eqn:radiationGeometry:fooc2}
\begin{aligned}
\calF = G(x,y) Z(z)
\end{aligned}
\end{equation}
%
Again we are faced with the problem of too many choices for the grades of each of these factors.  We can pick one of these, say \(Z\), to have only scalar and pseudoscalar grades so that the two factors commute.  Then we have
%
\begin{equation}\label{eqn:radiationGeometry:129}
\begin{aligned}
(\spacegrad_t + \spacegrad_z) \calF = (\spacegrad_t G) Z + \zcap G \partial_z Z = -G Z I \zcap \lambda
\end{aligned}
\end{equation}
%
With \(Z\) in an algebra isomorphic to the complex numbers, it is necessarily invertible (and commutes with it is derivative).  Similar arguments to the grade fixing for \(\calF\) show that \(G\) has only vector and bivector grades, but does \(G\) have the inverse required to do the separation of variables?  Let us blindly suppose that we can do this (and if we can not we can probably fudge it since we multiply again soon after).  With some rearranging we have
%
\begin{equation}\label{eqn:radiationGeometry:fooc3}
\begin{aligned}
-\inv{G} \zcap (\spacegrad_t G + G I \zcap \lambda) = (\partial_z Z)\inv{Z} = \text{constant}
\end{aligned}
\end{equation}
%
We want to separately equate these to a constant.  In order to commute these factors we have only required that \(Z\) have only scalar and pseudoscalar grades, so for the constant let us pick an arbitrary element in this subspace.  That is
%
\begin{equation}\label{eqn:radiationGeometry:fooc4}
\begin{aligned}
(\partial_z Z)\inv{Z} = \alpha + k I
\end{aligned}
\end{equation}
%
The solution for the \(Z\) factor in the separation of variables is thus
%
\begin{equation}\label{eqn:radiationGeometry:fooc5}
\begin{aligned}
Z \propto e^{(\alpha + k I)z}
\end{aligned}
\end{equation}
%
For \(G\) the separation of variables gives us
%
\begin{equation}\label{eqn:radiationGeometry:fooc6}
\begin{aligned}
\spacegrad_t G + (G \zcap \lambda + \zcap G k) I + \zcap G \alpha = 0
\end{aligned}
\end{equation}
%
We have now reduced the problem to something like a two variable eigenvalue problem, where the differential operator to find eigenvectors for is the transverse gradient \(\spacegrad_t\).  We unfortunately have an untidy split of the eigenvalue into left and right hand factors.
%Can we relate \eqnref{eqn:radiationGeometry:fooc6} to anything we know?  The very simplest case is for constant \(G\) and no hyperbolic angle (\(\alpha=0\)).  A solution then requires \(\lambda = \sqrt{\mu\epsilon} \omega/c = k\).  This wave number and angular frequency dependency is familiar since we also had that when we started with the assumption of oscillatory \(z\) and \(t\) dependence.
%
%How do we solve this when \(G\) is non constant?

While the product \(GZ\) was transverse only, we have now potentially lost that nice property for \(G\) itself, and do not know if \(G\) is strictly commuting or anticommuting with \(\zcap\).  Assuming either possibility for now, we can split this multivector into transverse and propagation direction fields \(G = G_t + G_z\)
%
\begin{equation}\label{eqn:radiationGeometry:fooc7}
\begin{aligned}
G_t &= \inv{2}(G - \zcap G \zcap) \\
G_z &= \inv{2}(G + \zcap G \zcap)
\end{aligned}
\end{equation}
%
With this split, noting that \(\zcap G_t = -G_t \zcap\), and \(\zcap G_z = G_z \zcap\) a rearrangement of \eqnref{eqn:radiationGeometry:fooc6} produces
%
\begin{equation}\label{eqn:radiationGeometry:fooc8}
\begin{aligned}
(\grad_t + \zcap ((k-\lambda) I + \alpha)) G_t = -(\grad_t + \zcap ((k+\lambda) I + \alpha)) G_z
\end{aligned}
\end{equation}
%
How do we find the eigen multivectors \(G_t\) and \(G_z\)?  A couple possibilities come to mind (perhaps not encompassing all solutions).  One is for one of \(G_t\) or \(G_z\) to be zero, and the other to separately require both halves of \eqnref{eqn:radiationGeometry:fooc8} equal a constant, very much like separation of variables despite the fact that both of these functions \(G_t\) and \(G_z\) are functions of \(x\) and \(y\).  The easiest non-trivial path is probably letting both sides of \eqnref{eqn:radiationGeometry:fooc8} separately equal zero, so that we are left with two independent eigen-multivector problems to solve
%
\begin{equation}\label{eqn:radiationGeometry:fooc9}
\begin{aligned}
\grad_t G_t &= -\zcap ((k-\lambda) I + \alpha)) G_t \\
\grad_t G_z &= -\zcap ((k+\lambda) I + \alpha)) G_z
\end{aligned}
\end{equation}
%
Damn.  have to mull this over.  Do not know where to go with it.

%\EndArticle
%%\EndNoBibArticle
