%
% Copyright � 2012 Peeter Joot.  All Rights Reserved.
% Licenced as described in the file LICENSE under the root directory of this GIT repository.
%

%
%
\mychapter{Lorentz boost of Lorentz force equations}
\index{Lorentz force!boost}
\label{chap:lorentzForceTx}
%\date{May 23, 2009.  lorentzForceTx.tex}

\section{Motivation}

Reading of \citep{bohm1996str} is a treatment of the Lorentz transform
properties of the Lorentz force equation.  This is not clear to me
without working through it myself, so do this.

I also have the urge to
try this with the GA formulation of the Lorentz transformation.  That may not end up being simpler
if one works with the non-covariant form of the Lorentz force equation, but only trying it will tell.

\section{Compare forms of the Lorentz Boost}

Working from the Geometric Algebra form of the Lorentz boost, show equivalence to the standard
coordinate matrix form and the vector form from Bohm.

\subsection{Exponential form}

Write the Lorentz boost of a four vector \(x = x^\mu \gamma_\mu = ct \gamma_0 + x^k \gamma_k\) as
%
\begin{equation}\label{eqn:lorentzFboost:LorentzBoost}
\begin{aligned}
L(x) &=
e^{-\alpha \vcap/2}
x
e^{\alpha \vcap/2}
\end{aligned}
\end{equation}
%
\subsection{Invariance property}

A Lorentz transformation (boost or rotation) can be defined as those transformation that leave the four vector square unchanged.

Following \citep{doran2003gap}, work with a \(+---\) metric signature (\(1 = \gamma_0^2 = -\gamma_k^2\)), and \(\sigma_k = \gamma_k \gamma_0\).  Our four vector square in this representation has the familiar invariant form
%
\begin{equation}\label{eqn:lorentzForceTx:20}
\begin{aligned}
x^2
&=
\lr{ ct \gamma_0 + x^m \gamma_m }
\lr{ ct \gamma_0 + x^k \gamma_k } \\
&=
\lr{ ct \gamma_0 + x^m \gamma_m }
\gamma_0^2
\lr{ ct \gamma_0 + x^k \gamma_k } \\
&=
\lr{ ct + x^m \sigma_m }
\lr{ ct - x^k \sigma_k } \\
&= (ct + \Bx) (ct - \Bx) \\
&= (ct)^2 - \Bx^2
\end{aligned}
\end{equation}
%
and we expect this of the Lorentz boost of \eqnref{eqn:lorentzFboost:LorentzBoost}.  To verify we have
%
\begin{equation}\label{eqn:lorentzForceTx:40}
\begin{aligned}
L(x)^2
&=
e^{-\alpha \vcap/2}
x
e^{\alpha \vcap/2}
e^{-\alpha \vcap/2}
x
e^{\alpha \vcap/2} \\
&=
e^{-\alpha \vcap/2}
x
x
e^{\alpha \vcap/2} \\
&=
x^2
e^{-\alpha \vcap/2}
e^{\alpha \vcap/2} \\
&=
x^2
\end{aligned}
\end{equation}
%
\subsection{Sign of the rapidity angle}
\index{rapidity}

The factor \(\alpha\) will be the rapidity angle, but what sign do we want for a boost along the positive \(\vcap\) direction?

Dropping to coordinates is an easy way to determine the sign convention in effect.  Write \(\vcap = \sigma_1\)
%
\begin{equation}\label{eqn:lorentzForceTx:60}
\begin{aligned}
L(x) &=
e^{-\alpha \vcap/2}
x
e^{\alpha \vcap/2} \\
&=
(\cosh(\alpha/2) - \sigma_1 \sinh(\alpha/2))
(
x^0 \gamma_0
+x^1 \gamma_1
+x^2 \gamma_2
+x^3 \gamma_3
)
(\cosh(\alpha/2) + \sigma_1 \sinh(\alpha/2))
\end{aligned}
\end{equation}
%
\(\sigma_1\) commutes with \(\gamma_2\) and \(\gamma_3\) and anticommutes otherwise, so we have
%
\begin{equation}\label{eqn:lorentzForceTx:80}
\begin{aligned}
L(x) &=
\lr{
x^2 \gamma_2
+x^3 \gamma_3
}
e^{-\alpha \vcap/2}
e^{\alpha \vcap/2}
+
\lr{
x^0 \gamma_0
+x^1 \gamma_1
}
e^{\alpha \vcap} \\
&=
x^2 \gamma_2
+x^3 \gamma_3
+\lr{
x^0 \gamma_0
+x^1 \gamma_1
}
e^{\alpha \vcap} \\
&=
x^2 \gamma_2
+x^3 \gamma_3
+\lr{
x^0 \gamma_0
+x^1 \gamma_1
}
(\cosh(\alpha) + \sigma_1 \sinh(\alpha))
\end{aligned}
\end{equation}
%
Expanding out just the \(0,1\) terms changed by the transformation we have
%
\begin{equation*}
\begin{aligned}
\lr{
x^0 \gamma_0
+x^1 \gamma_1
}
&
(\cosh(\alpha) + \sigma_1 \sinh(\alpha)) \\
=
x^0 \gamma_0 \cosh(\alpha)
+x^1 \gamma_1 \cosh(\alpha)
+x^0 \gamma_0 \sigma_1 \sinh(\alpha)
+x^1 \gamma_1 \sigma_1 \sinh(\alpha) \\
&=
x^0 \gamma_0 \cosh(\alpha)
+x^1 \gamma_1 \cosh(\alpha)
+x^0 \gamma_0 \gamma_1 \gamma_0 \sinh(\alpha)
+x^1 \gamma_1 \gamma_1 \gamma_0 \sinh(\alpha) \\
&=
x^0 \gamma_0 \cosh(\alpha)
+x^1 \gamma_1 \cosh(\alpha)
-x^0 \gamma_1 \sinh(\alpha)
-x^1 \gamma_0 \sinh(\alpha) \\
&=
\gamma_0 (x^0 \cosh(\alpha) -x^1 \sinh(\alpha) )
+\gamma_1 (x^1 \cosh(\alpha) -x^0 \sinh(\alpha) )
\end{aligned}
\end{equation*}
%
Writing \({x^\mu}' = L(x) \cdot \gamma^\mu\), and \(x^\mu = x \cdot \gamma^\mu\),
and a substitution of \(\cosh(\alpha) = 1/\sqrt{1 - \Bv^2/c^2}\), and \(\alpha \vcap = \tanh^{-1}(\Bv/c)\),
we have the traditional coordinate
expression for the one directional Lorentz boost
%
\begin{equation}\label{eqn:lorentzForceTx:100}
\begin{aligned}
\begin{bmatrix}
{x^0}' \\
{x^1}' \\
{x^2}' \\
{x^3}'
\end{bmatrix}
&=
\begin{bmatrix}
\cosh\alpha & -\sinh\alpha & 0 & 0 \\
-\sinh\alpha & \cosh\alpha & 0 & 0 \\
0 & 0 & 1 & 0 \\
0 & 0 & 0 & 1 \\
\end{bmatrix}
\begin{bmatrix}
x^0 \\
x^1 \\
x^2 \\
x^3
\end{bmatrix}
\end{aligned}
\end{equation}
%
Performing this expansion showed initially showed that I had the wrong sign for \(\alpha\) in the exponentials and I went back and
adjusted it all accordingly.

\subsection{Expanding out the Lorentz boost for projective and rejective directions}

Two forms of Lorentz boost representations have been compared above.  An additional one is used in the Bohm text (a
vector form of the Lorentz transformation not using coordinates).  Let us see
if we can derive that from the exponential form.

Start with computation of components of a four vector relative to an observer timelike unit vector \(\gamma_0\).
%
\begin{equation}\label{eqn:lorentzForceTx:120}
\begin{aligned}
x
&= x \gamma_0 \gamma_0 \\
&= (x \gamma_0) \gamma_0 \\
&=
\lr{ x \cdot \gamma_0 + x \wedge \gamma_0 }
\gamma_0 \\
\end{aligned}
\end{equation}
%
For the spatial vector factor above write \(\Bx = x \wedge \gamma_0\), for
%
\begin{equation}\label{eqn:lorentzForceTx:140}
\begin{aligned}
x
&= \lr{ x \cdot \gamma_0 } \gamma_0 + \Bx \gamma_0 \\
&=
\lr{ x \cdot \gamma_0 }
\gamma_0 + \Bx \vcap \vcap \gamma_0 \\
&=
\lr{ x \cdot \gamma_0 }
\gamma_0 + (\Bx \cdot \vcap) \vcap \gamma_0 + (\Bx \wedge \vcap) \vcap \gamma_0 \\
\end{aligned}
\end{equation}
%
We have the following commutation relations for the various components
\begin{equation}\label{eqn:lorentzForceTx:160}
\begin{aligned}
\vcap (\gamma_0) &= - \gamma_0 \vcap \\
\vcap (\vcap \gamma_0) &= - (\vcap \gamma_0) \vcap \\
\vcap ((\Bx \wedge \vcap) \vcap \gamma_0 )
%&= -(\Bx \wedge \vcap) \vcap \vcap \gamma_0
&= ((\Bx \wedge \vcap) \vcap \gamma_0) \vcap
\end{aligned}
\end{equation}
%
For a four vector \(u\) that commutes with \(\vcap\) we have \(e^{-\alpha \vcap/2} u = u e^{-\alpha \vcap/2}\), and if it anticommutes
we have the conjugate relation
\(e^{-\alpha \vcap/2} u = u e^{\alpha \vcap/2}\).  This gives us
%
\begin{equation}\label{eqn:lorentzForceTx:180}
\begin{aligned}
L(x)
&=
(\Bx \wedge \vcap) \vcap \gamma_0 +
\left( (x \cdot \gamma_0) \gamma_0 + (\Bx \cdot \vcap) \vcap \gamma_0 \right) e^{\alpha \vcap} \\
\end{aligned}
\end{equation}
%
Now write the exponential as a scalar and spatial vector sum
\begin{equation}\label{eqn:lorentzForceTx:200}
\begin{aligned}
e^{\alpha \vcap}
&=
\cosh\alpha
+\vcap \sinh\alpha
\\
&=
\gamma (1 +\vcap \tanh\alpha )
\\
&=
\gamma (1 +\vcap \beta)
\\
&=
\gamma (1 + \Bv/c )
\\
\end{aligned}
\end{equation}
%
Expanding out the exponential product above, also writing \(x^0 = ct = x \cdot \gamma_0\), we have
%
\begin{equation}\label{eqn:lorentzForceTx:220}
\begin{aligned}
( &x^0 \gamma_0 + (\Bx \cdot \vcap) \vcap \gamma_0 ) e^{\alpha \vcap} \\
&=
\gamma ( x^0 \gamma_0 + (\Bx \cdot \vcap) \vcap \gamma_0 ) ( 1 + \Bv/c ) \\
&=
\gamma (
x^0 \gamma_0
+ (\Bx \cdot \vcap) \vcap \gamma_0
+x^0 \gamma_0 \Bv/c
+ (\Bx \cdot \vcap) \vcap \gamma_0 \Bv/c
) \\
\end{aligned}
\end{equation}
%
So for the total Lorentz boost in vector form we have
%
\begin{equation}\label{eqn:lorentzFboost:fourVectorExpanded}
\begin{aligned}
L(x)
&=
(\Bx \wedge \vcap) \vcap \gamma_0 +
\gamma \left(x^0 - \Bx \cdot \frac{\Bv}{c} \right) \gamma_0
+ \gamma \left( \Bx \cdot \inv{\Bv/c} - x^0 \right) \frac{\Bv}{c} \gamma_0
\end{aligned}
\end{equation}
%
Now a visual inspection shows that this does match
equation (15-12) from the text:
%
\begin{equation}\label{eqn:lorentzFboost:explicitSpaceAndTimeTransformed}
\begin{aligned}
\Bx'
&= \Bx - (\vcap \cdot \Bx) \vcap + \frac{ (\vcap \cdot \Bx )\vcap - \Bv t }{ \sqrt{1 - (v^2/c^2)} } \\
t'
&= \frac{ t - (\Bv \cdot \Bx )/c^2 }{ \sqrt{1 - (v^2/c^2)} }
\end{aligned}
\end{equation}
%
but the equivalence of these is perhaps not so obvious without familiarity
with the GA constructs.

\subsection{differential form}

Bohm utilizes a vector differential form of the Lorentz transformation
for both the spacetime and energy-momentum vectors.  From equation
\eqnref{eqn:lorentzFboost:explicitSpaceAndTimeTransformed}
we can derive the expressions used.  In particular for the transformed
spatial component we have
%
\begin{equation}\label{eqn:lorentzForceTx:240}
\begin{aligned}
\Bx'
&= \Bx + \gamma \left( -(\vcap \cdot \Bx) \vcap \inv{\gamma} + (\vcap \cdot \Bx )\vcap - \Bv t \right) \\
&= \Bx + \gamma \left( (\vcap \cdot \Bx) \vcap \left(1-\inv{\gamma} \right) - \Bv t \right) \\
&= \Bx + (\gamma-1)(\vcap \cdot \Bx) \vcap - \gamma \Bv t \\
\end{aligned}
\end{equation}
%
So in differential vector form we have
\begin{equation}\label{eqn:lorentzFboost:differentialSpaceTime}
\begin{aligned}
d\Bx'
&= d\Bx + (\gamma-1)(\vcap \cdot d\Bx) \vcap - \gamma \Bv dt \\
dt'
&= \gamma ( dt -
\lr{ \Bv \cdot d\Bx }/c^2 )
\end{aligned}
\end{equation}
%
and by analogy with \(dx^0 = cdt \rightarrow dE/c\), and \(d\Bx \rightarrow d\Bp\), we also have the energy momentum transformation
%
\begin{equation}\label{eqn:lorentzFboost:differentialEnergyMomentum}
\begin{aligned}
d\Bp'
&= d\Bp + (\gamma-1)(\vcap \cdot d\Bp) \vcap - \gamma \Bv dE/c^2 \\
dE'
&= \gamma ( dE - \Bv \cdot d\Bp )
\end{aligned}
\end{equation}
%
Reflecting on these forms of the Lorentz transformation, they are quite
natural ways to express the vector results.  The terms with \(\gamma\) factors
are exactly what we are used to in the coordinate representation (transformation
of only the time component and the projection of the spatial vector in the
velocity direction), while the \(-1\) part of the \((\gamma-1)\) term just
subtracts off the projection unaltered, leaving
\(d\Bx - (d\Bx \cdot \vcap) \vcap = (d\Bx \wedge \vcap) \vcap\), the rejection
from the \(\vcap\) direction.

\section{Lorentz force transformation}

Preliminaries out of the way, now we want to examine the transform of the electric and magnetic field as used in the Lorentz force equation.  In
CGS units as in the text we have
%
\begin{equation}\label{eqn:lorentzForceTx:260}
\begin{aligned}
\frac{d\Bp}{dt} &= q \left( \bcE + \frac{\Bv}{c} \cross \bcH \right) \\
\frac{dE}{dt} &= q \bcE \cdot \Bv
\end{aligned}
\end{equation}
%
After writing this
in differential form
\begin{equation}\label{eqn:lorentzFboost:untransformedLorentzForce}
\begin{aligned}
d\Bp &= q \left( \bcE dt + \frac{d\Bx}{c} \cross \bcH \right) \\
dE &= q \bcE \cdot d\Bx
\end{aligned}
\end{equation}
%
and the transformed variation of this equation, also in differential form
\begin{equation}\label{eqn:lorentzFboost:transformedLorentzForce}
\begin{aligned}
d\Bp' &= q \left( \bcE' dt' + \frac{d\Bx'}{c} \cross \bcH' \right) \\
dE' &= q \bcE' \cdot d\Bx'
\end{aligned}
\end{equation}
%
A brute force insertion of the transform results of equations
\eqnref{eqn:lorentzFboost:differentialSpaceTime}, and \eqnref{eqn:lorentzFboost:differentialEnergyMomentum} into these is performed.  This is mostly
a mess of algebra.
%If it was not for mechanical algebra and calculus physics would be much easier!

While the Bohm book covers some of this, other parts are left for the reader.  Do the whole thing here as an exercise.

\subsection{Transforming the Lorentz power equation}

Let us start with the energy rate equation in its entirety without interleaving the momentum calculation.
%
\begin{equation}\label{eqn:lorentzForceTx:280}
\begin{aligned}
\inv{q} dE'
&= \bcE' \cdot d\Bx' \\
&= \bcE' \cdot \left( d\Bx + (\gamma-1)(\Vcap \cdot d\Bx) \Vcap - \gamma \BV dt \right) \\
&= \bcE' \cdot d\Bx + (\gamma-1)(\Vcap \cdot d\Bx) \bcE' \cdot \Vcap - \gamma \bcE' \cdot \BV dt \\
\inv{q}\gamma ( dE - \BV \cdot d\Bp ) &= \\
\gamma \bcE \cdot d\Bx - \gamma \BV \cdot \left(\bcE dt + \frac{d\Bx}{c} \cross \bcH \right) &= \\
\gamma \bcE \cdot d\Bx
- \gamma \BV \cdot \bcE dt
- \gamma \inv{c} d\Bx \cdot (\bcH \cross \BV) &= \\
\end{aligned}
\end{equation}
%
Grouping \(dt\) and \(d\Bx\) terms we have
\begin{equation}\label{eqn:lorentzForceTx:300}
\begin{aligned}
0 &=
d\Bx \cdot \left(
\bcE' + (\gamma-1) \Vcap (\bcE' \cdot \Vcap)
-\gamma \bcE
+ \gamma (\bcH \cross \BV/c)
\right)
+ dt \gamma \BV \cdot (\bcE - \bcE')
\end{aligned}
\end{equation}
%
Now the argument is that both the \(dt\) and \(d\Bx\) factors must separately equal zero.
Assuming
that for now (but come back to this and think it through), and writing \(\bcE = \bcE_\parallel + \bcE_\perp\) for the projective
and rejective components of the field relative to the boost direction \(\BV\) (same for \(\bcH\) and the transformed fields) we have from the \(dt\) term
%
\begin{equation}\label{eqn:lorentzForceTx:320}
\begin{aligned}
0
&= \BV \cdot ( \bcE_\parallel + \bcE_\perp -\bcE_\parallel' -\bcE_\perp' ) \\
&= \BV \cdot ( \bcE_\parallel -\bcE_\parallel' ) \\
\end{aligned}
\end{equation}
%
So we can conclude
\begin{equation}\label{eqn:lorentzForceTx:340}
\begin{aligned}
\bcE_\parallel' =\bcE_\parallel
\end{aligned}
\end{equation}
%
Now from the \(d\Bx\) coefficient, we have
%
\begin{equation}\label{eqn:lorentzForceTx:360}
\begin{aligned}
0
&=
\bcE_\parallel'
+\bcE_\perp'
 + (\gamma-1) \Vcap (\bcE_\parallel' \cdot \Vcap)
-\gamma \bcE_\parallel
-\gamma \bcE_\perp
+ \gamma (\bcH_\perp \cross \BV/c) \\
&=
\mathLabelBox
[
   labelstyle={below of=m\themathLableNode, below of=m\themathLableNode}
]
{
\left(
\bcE_\parallel' -\Vcap (\bcE_\parallel' \cdot \Vcap)
\right)
}{\(\bcE_\parallel' - \bcE_\parallel'\)}
+\bcE_\perp'
-\gamma
\mathLabelBox
[
   labelstyle={below of=m\themathLableNode, below of=m\themathLableNode}
]
{
\left(
\bcE_\parallel - \Vcap (\bcE_\parallel' \cdot \Vcap)
\right)
}{\(\bcE_\parallel - \bcE_\parallel\)}
-\gamma \bcE_\perp
+ \gamma (\bcH_\perp \cross \BV/c) \\
\end{aligned}
\end{equation}
%
This now completely specifies the transformation properties of the electric field under a \(\BV\) boost
%
\begin{equation}\label{eqn:lorentzFboost:finalResultsForPower}
\begin{aligned}
\bcE_\perp' &= \gamma \left( \bcE_\perp + \frac{\BV}{c} \cross \bcH_\perp \right) \\
\bcE_\parallel' &= \bcE_\parallel
\end{aligned}
\end{equation}
%
(it also confirms the typos in the text).

\subsection{Transforming the Lorentz momentum equation}

Now we do the exercise for the reader part, and express the transformed momentum differential of equation
\eqnref{eqn:lorentzFboost:transformedLorentzForce} in terms of \eqnref{eqn:lorentzFboost:differentialSpaceTime}
%
\begin{equation}\label{eqn:lorentzForceTx:380}
\begin{aligned}
\inv{q} d\Bp'
&= \bcE' dt' + \frac{d\Bx'}{c} \cross \bcH' \\
&= \gamma \bcE' dt - \gamma \bcE' (\BV \cdot d\Bx )/c^2
+ d\Bx \cross \bcH'/c + (\gamma-1)(\Vcap \cdot d\Bx) \Vcap \cross \bcH'/c - \gamma \BV \cross \bcH'/c dt \\
\end{aligned}
\end{equation}
%
Now for the LHS using \eqnref{eqn:lorentzFboost:differentialEnergyMomentum} and \eqnref{eqn:lorentzFboost:untransformedLorentzForce} we have
\begin{equation}\label{eqn:lorentzForceTx:400}
\begin{aligned}
\inv{q} d\Bp'
&= d\Bp/q + (\gamma-1)(\Vcap \cdot d\Bp/q) \Vcap - \gamma \BV dE/qc^2 \\
&= \bcE dt + \frac{d\Bx}{c} \cross \bcH
+ (\gamma-1)( \Vcap \cdot \bcE dt + \Vcap \cdot (d\Bx \cross \bcH/c) ) \Vcap - \gamma \BV (
\bcE \cdot d\Bx
)/c^2 \\
&= \bcE dt + \frac{d\Bx}{c} \cross \bcH
+ (\gamma-1) (\Vcap \cdot \bcE) \Vcap dt
+ (\gamma-1)( d\Bx \cdot (\bcH \cross \Vcap/c) ) \Vcap
- \gamma \BV ( \bcE \cdot d\Bx )/c^2 \\
\end{aligned}
\end{equation}
%
Combining these and grouping by \(dt\) and \(d\Bx\) we have
\begin{equation}\label{eqn:lorentzForceTx:420}
\begin{aligned}
dt &\left(
-( \bcE - (\Vcap \cdot \bcE) \Vcap )
+ \gamma (\bcE' - (\Vcap \cdot \bcE) \Vcap )
- \gamma \BV \cross \bcH'/c
\right) \\
&=
 \frac{\gamma}{c^2} \left( \bcE' (\BV \cdot d\Bx ) - \BV ( \bcE \cdot d\Bx ) \right)
+ \frac{d\Bx}{c} \cross (\bcH -\bcH')  \\
&\qquad + \frac{\gamma-1}{c}
\left(
( d\Bx \cdot (\bcH \cross \Vcap) ) \Vcap - (\Vcap \cdot d\Bx) (\Vcap \cross \bcH')
\right) \\
\end{aligned}
\end{equation}
%
What a mess, and this is after some initial grouping!  From the power result we have \(\Vcap \cdot \bcE = \Vcap \cdot \bcE'\) so we can write the LHS of this mess as
%
\begin{equation}\label{eqn:lorentzForceTx:440}
\begin{aligned}
dt &\left(
-( \bcE - (\Vcap \cdot \bcE) \Vcap )
+ \gamma (\bcE' - (\Vcap \cdot \bcE) \Vcap )
- \gamma \BV \cross \bcH'/c
\right) \\
&=
dt \left(
-( \bcE - (\Vcap \cdot \bcE) \Vcap )
+ \gamma (\bcE' - (\Vcap \cdot \bcE') \Vcap )
- \gamma \BV \cross \bcH'/c
\right)
\\
&=
dt \left(
-\bcE_\perp
+ \gamma \bcE_\perp'
- \gamma \BV \cross \bcH'/c
\right)
\\
&=
dt \left(
-\bcE_\perp
+ \gamma \bcE_\perp'
- \gamma \BV \cross \bcH_\perp'/c
\right)
\\
\end{aligned}
\end{equation}
%
If this can separately equal zero independent of the \(d\Bx\) terms we have
%
\begin{equation}\label{eqn:lorentzForceTx:460}
\begin{aligned}
\bcE_\perp = \gamma \left( \bcE_\perp' - \frac{\BV}{c} \cross \bcH_\perp' \right)
\end{aligned}
\end{equation}
%
Contrast this to the result for \(\bcE_\perp'\) in the first of \eqnref{eqn:lorentzFboost:finalResultsForPower}.  It differs only by
a sign which has an intuitive relativistic (anti)symmetry that is not entirely unsurprising.  If a boost along \(\BV\)
takes \(\bcE\) to \(\bcE'\), then an boost with opposing direction makes sense for the reverse.

Despite being reasonable seeming, a relation like \(\bcH_\parallel = \bcH_\parallel'\) was expected ... does that follow from this somehow?
Perhaps things will become more clear after examining the mess on the RHS involving all the \(d\Bx\) terms?

The first part of this looks amenable to some algebraic manipulation.  Using
\((\bcE' \wedge \BV) \cdot d\Bx = \bcE' (\BV \cdot d\Bx) - \BV (\bcE' \cdot d\Bx)\), we have
%
\begin{equation}\label{eqn:lorentzForceTx:480}
\begin{aligned}
\bcE' (\BV \cdot d\Bx ) - \BV ( \bcE \cdot d\Bx )
&=
(\bcE' \wedge \BV) \cdot d\Bx + \BV (\bcE' \cdot d\Bx) - \BV ( \bcE \cdot d\Bx ) \\
&=
(\bcE' \wedge \BV) \cdot d\Bx + \BV ((\bcE' - \bcE) \cdot d\Bx) \\
\end{aligned}
\end{equation}
%
and
\begin{equation}\label{eqn:lorentzForceTx:500}
\begin{aligned}
(\bcE' \wedge \BV) \cdot d\Bx
&=
\gpgradeone{ (\bcE' \wedge \BV) d\Bx } \\
&=
\gpgradeone{ i(\bcE' \cross \BV) d\Bx } \\
&=
\gpgradeone{ i ((\bcE' \cross \BV) \wedge d\Bx) } \\
&=
\gpgradeone{ i^2 ((\bcE' \cross \BV) \cross d\Bx) } \\
&=
d\Bx \cross (\bcE' \cross \BV)
\end{aligned}
\end{equation}
%
Putting things back together, does it improve things?
%
\begin{equation}\label{eqn:lorentzForceTx:520}
\begin{aligned}
0 &=
% \frac{\gamma}{c^2} \left( d\Bx \cross (\bcE' \cross \BV) + \BV ((\bcE' - \bcE) \cdot d\Bx) \right)
%+ \frac{d\Bx}{c} \cross (\bcH -\bcH')
{d\Bx} \cross \left(
{\gamma} \left(\bcE' \cross \frac{\BV}{c} \right)
+ (\bcH -\bcH')
\right) \\
&+\frac{\gamma}{c} \BV ((\bcE' - \bcE) \cdot d\Bx)  \\
%
&+ (\gamma-1)
\left(
( d\Bx \cdot (\bcH \cross \Vcap) ) \Vcap - (\Vcap \cdot d\Bx) (\Vcap \cross \bcH')
\right) \\
\end{aligned}
\end{equation}
%
Perhaps the last bit can be factored into \(d\Bx\) crossed with some function of \(\bcH - \bcH'\)?
