%
% Copyright � 2012 Peeter Joot.  All Rights Reserved.
% Licenced as described in the file LICENSE under the root directory of this GIT repository.
%

%
%
%\input{../peeter_prologue.tex}

\chapter{Graphical representation of Spherical Harmonics for \texorpdfstring{\(l=1\)}{l equal 1}}
\index{spherical harmonics}
\label{chap:L1Associated}

%\blogpage{http://sites.google.com/site/peeterjoot/math2009/L1Associated.pdf}
%\date{Aug 16, 2009}
%\revisionInfo{\(RCSfile: L1Associated.tex,v \) Last \(Revision: 1.6 \) \(Date: 2009/10/22 02:07:20 \)}

%\beginArtWithToc
\beginArtNoToc

\section{First observations}

In Bohm's QT \citep{bohm1989qt}, 14.17), the properties of \(l=1\) associated Legendre polynomials are examined under rotation.  Wikipedia (\citep{wiki:sphericalHarm} calls these eigen functions the spherical harmonics.

The unnormalized eigenfunctions are given (eqn (14.47) in Bohm) for \(s \in [0,l]\), with \(\cos\theta = \zeta\) by

\begin{equation}\label{eqn:L1Associated:foo0}
\begin{aligned}
\psi_l^{l-s} = \frac{e^{i(l-s)\phi}}{(1-\zeta^2)^{(l-s)/2}} \frac{\partial^s}{\partial \zeta^s} (1-\zeta^2)^l
\end{aligned}
\end{equation}

The normalization is provided by a surface area inner product

\begin{equation}\label{eqn:L1Associated:foo2}
\begin{aligned}
(u,v) &= \int_{\theta=0}^{\pi} \int_{\phi=0}^{2\pi} u v^\conj \sin\theta d\theta d\phi
\end{aligned}
\end{equation}

Computing these for \(l=1\), and disregarding any normalization these eigenfunctions can be found to be

\begin{equation}\label{eqn:L1Associated:foo1}
\begin{aligned}
\psi_1^1 &= \sin\theta e^{i\phi} \\
\psi_1^0 &= \cos\theta \\
\psi_1^{-1} &= \sin\theta e^{-i\phi}
\end{aligned}
\end{equation}

There is a direct relationship between these eigenfunctions with a triple of vectors associated with a point on the unit sphere.  Referring to \cref{fig:L1Associated}, observe the three doubled arrow vectors, all associated with a point on the unit sphere \(\Bx = (x,y,z) = (\sin\theta \cos\phi, \sin\theta \cos\phi, \cos\theta)\).
\imageFigure{../figures/gabook/L1Associated}{Vectoring the \(l=1\) associated Legendre polynomials}{fig:L1Associated}{0.4}

The normal to the \(x,y\) plane from \(\Bx\), designated \(\Bn\) has the vectorial value
\begin{equation}\label{eqn:L1Associated:foo3}
\Bn = \cos\theta \Be_3.
\end{equation}

From the origin to the point of the \(x,y\) plane intersection to the normal we have
\begin{equation}\label{eqn:L1Associated:foo4}
\begin{aligned}
\Brho = \sin\theta (\cos\phi \Be_1 + \sin\phi \Be_2) = \Be_1 \sin\theta e^{\Be_1 \Be_2 \phi}
\end{aligned}
\end{equation}

and finally in the opposite direction also in the plane and mirroring \(\Brho\) we have the last of this triplet of vectors
\begin{equation}\label{eqn:L1Associated:foo5}
\Brho_{-} = \sin\theta (\cos\phi \Be_1 - \sin\phi \Be_2) = \Be_1 \sin\theta e^{-\Be_1 \Be_2 \phi}.
\end{equation}

So, if we choose to use \(i=\Be_1 \Be_2\) (the bivector for the plane normal to the z-axis), then we can in fact vectorize these eigenfunctions.  The vectors \(\Brho\) (i.e. \(\psi_1^1)\), and \(\Brho_{-}\) (i.e. \(\psi_1^{-1}\)) are both normal to \(\Bn\) (i.e. \(\psi_1^0\)), but while the vectors \(\Brho\) and \(\Brho_{-}\) are both in the plane one is produced with a counterclockwise rotation of \(\Be_1\) by \(\phi\) in the plane and the other with an opposing rotation.

Summarizing, we can write the unnormalized vectors the relations
\begin{equation*}
\begin{array}{l l l}%\label{eqn:L1Associated:foo7}
\psi_1^1 &= \Be_1 \Brho &= \sin\theta e^{\Be_1\Be_2 \phi} \\
\psi_1^0 &= \Be_3 \Bn &= \cos\theta \\
\psi_1^{-1} &= \Be_1 \Brho_{-} &= \sin\theta e^{-\Be_1\Be_2 \phi}
\end{array}
\end{equation*}

%\begin{equation*}\label{eqn:L1Associated:foo7}
%\begin{array}{l l l}
%\Brho &= \Be_1 \psi_1^1 &= \sin\theta e^{\Be_1\Be_2 \phi} = \sin\theta (\cos\phi \Be_1 + \sin\phi \Be_2) \\
%\Bn &= \Be_3 \psi_1^0 &= \cos\theta \Be_3 \\
%\Brho_{-} &= \Be_1 \psi_1^{-1} &= \sin\theta e^{-\Be_1\Be_2 \phi} = \sin\theta (\cos\phi \Be_1 - \sin\phi \Be_2)
%\end{array}
%\end{equation*}

I have no familiarity yet with the \(l=2\) or higher Legendre eigenfunctions.  Do they also admit a geometric representation?

\section{Expressing Legendre eigenfunctions using rotations}

We can express a point on a sphere with a pair of rotation operators.  First rotating \(\Be_3\) towards \(\Be_1\) in the \(z,x\) plane by \(\theta\), then in the \(x,y\) plane by \(\phi\) we have the point \(\Bx\) in \cref{fig:L1Associated}

Writing the result of the first rotation as \(\Be_3'\) we have

\begin{equation}\label{eqn:L1Associated:foob1}
\begin{aligned}
\Be_3' = \Be_3 e^{\Be_{31}\theta} = e^{-\Be_{31}\theta/2} \Be_3 e^{\Be_{31}\theta/2}
\end{aligned}
\end{equation}

One more rotation takes \(\Be_3'\) to \(\Bx\).  That is

\begin{equation}\label{eqn:L1Associated:foob2}
\begin{aligned}
\Bx = e^{-\Be_{12}\phi/2} \Be_3' e^{\Be_{12}\phi/2}
\end{aligned}
\end{equation}

All together, writing \(R_\theta = e^{\Be_{31}\theta/2}\), and \(R_\phi = e^{\Be_{12}\phi/2}\), we have

\begin{equation}\label{eqn:L1Associated:foob3}
\begin{aligned}
\Bx = \tilde{R_\phi} \tilde{R_\theta} \Be_3 R_\theta R_\phi
\end{aligned}
\end{equation}

It is worth a quick verification that this produces the desired result.

\begin{equation}\label{eqn:L1Associated:27}
\begin{aligned}
\tilde{R_\phi} \tilde{R_\theta} \Be_3 R_\theta R_\phi
&= \tilde{R_\phi} \Be_3 e^{\Be_{31}\theta} R_\phi \\
&= e^{-\Be_{12}\phi/2} (\Be_3 \cos\theta + \Be_1 \sin\theta) e^{\Be_{12}\phi/2} \\
&=
\Be_3 \cos\theta + \Be_1 \sin\theta e^{\Be_{12}\phi} \\
\end{aligned}
\end{equation}

This is the expected result

\begin{equation}\label{eqn:L1Associated:foob4}
\begin{aligned}
\Bx = \Be_3 \cos\theta + \sin\theta (\Be_1 \sin\theta + \Be_2 \cos\theta)
\end{aligned}
\end{equation}

The projections onto the \(\Be_3\) and the \(x,y\) plane are then, respectively,

\begin{equation}\label{eqn:L1Associated:foob5}
\begin{aligned}
\Bx_z &= \Be_3 (\Be_3 \cdot \Bx) = \Be_3 \cos\theta  \\
\Bx_{x,y} &= \Be_3 (\Be_3 \wedge \Bx) = \sin\theta (\Be_1 \sin\theta + \Be_2 \cos\theta)
\end{aligned}
\end{equation}

So if \(\Bx_{\pm}\) is the point on the unit sphere associated with the rotation angles \(\theta,\pm\phi\), then we have for the \(l=1\) associated Legendre polynomials

\begin{equation}\label{eqn:L1Associated:foob6}
\begin{aligned}
\psi_1^0 &= \Be_3 \cdot \Bx \\
\psi_1^{\pm 1} &= \Be_1 \Be_3 (\Be_3 \wedge \Bx_{\pm})
\end{aligned}
\end{equation}

Note that the \(\pm\) was omitted from \(\Bx\) for \(\psi_1^0\) since either produces the same \(\Be_3\) component.  This gives us a nice geometric interpretation of these eigenfunctions.  We see that \(\psi_1^0\) is the biggest when \(\Bx\) is close to straight up, and when this occurs \(\psi_1^{\pm 1}\) are correspondingly reduced, but when \(\Bx\) is close to the \(x,y\) plane where \(\psi_1^{\pm 1}\) will be greatest the \(z\)-axis component is reduced.

%\EndArticle
