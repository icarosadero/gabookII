%
% Copyright � 2012 Peeter Joot.  All Rights Reserved.
% Licenced as described in the file LICENSE under the root directory of this GIT repository.
%

%
%
\mychapter{Gamma Matrices}
\label{chap:PJDiracGamma}
\index{gamma matrices}
%\date{Dec 13, 2008.  gamma.tex}

\section{Dirac matrices}

\index{Dirac!matrix}
\index{gamma matrix}
The Dirac matrices \(\gamma^\mu\) can be used as a Minkowski basis.  The basic defining relationship is the Minkowski metric, where the dot products satisfy
%
\begin{equation}\label{eqn:gamma:20}
\begin{aligned}
\scalarProduct{\gamma^\mu}{\gamma^\nu} &= \pm \delta_{\mu\nu} \\
(\scalarProduct{\gamma^0}{\gamma^0})(\scalarProduct{\gamma^a}{\gamma^a}) &= -1 \quad \text{where \(a \in \{1,2,3\}\)}
\end{aligned}
\end{equation}
%
There is freedom to pick the positive square for either \(\gamma^0\) or \(\gamma^a\), and both conventions are common.

One of the matrix representations for these vectors listed in the
\href{https://en.wikipedia.org/wiki/Gamma_matrices}{Dirac matrix wikipedia article}
is
%
\begin{equation}\label{eqn:gamma:basis}
\begin{aligned}
\gamma^0 &= \begin{bmatrix}
 1  &  0  &  0  &  0  \\
 0  &  1  &  0  &  0  \\
 0  &  0  &  -1  &  0  \\
 0  &  0  &  0  &  -1  \\
\end{bmatrix} \quad
\gamma^1 = \begin{bmatrix}
 0  &  0  &  0  &  1  \\
 0  &  0  &  1  &  0  \\
 0  &  -1  &  0  &  0  \\
 -1  &  0  &  0  &  0  \\
\end{bmatrix} \\
\gamma^2 &= \begin{bmatrix}
 0  &  0  &  0  &  -i  \\
 0  &  0  &  i  &  0  \\
 0  &  i  &  0  &  0  \\
 -i  &  0  &  0  &  0  \\
\end{bmatrix}
\quad \gamma^3 = \begin{bmatrix}
 0  &  0  &  1  &  0  \\
 0  &  0  &  0  &  -1  \\
 -1  &  0  &  0  &  0  \\
 0  &  1  &  0  &  0  \\
\end{bmatrix}
\end{aligned}
\end{equation}
%
For this particular basis we have a \(+---\) metric signature.  In the matrix form this takes the specific meaning that \((\gamma^0)^2 = I\), and \((\gamma^a)^2 = -I\).

A table of all the possible product variants of \eqnref{eqn:gamma:basis} can be found below in the appendix.

\subsection{anticommutator product}
\index{anticommutator}

Noting that the matrices square in the fashion just described and that they reverse sign when multiplication order is reversed allows for summarizing the dot products relationships as follows
%
\begin{equation}\label{eqn:gamma:symmetric}
\begin{aligned}
\symmetric{\gamma^\mu}{\gamma^\nu}
&= {\gamma^\mu}{\gamma^\nu} + {\gamma^\nu}{\gamma^\mu} \\
%&= 2 (\scalarProduct{\gamma^\mu}{\gamma^\nu}) I \\
&= 2 \eta^{\mu\nu} I,
\end{aligned}
\end{equation}
%
where the metric tensor \(\eta^{\mu\nu} = \scalarProduct{\gamma^\mu}{\gamma^\nu}\) is commonly summarized as coordinates of a matrix as in
%
\begin{equation}\label{eqn:gamma:40}
\begin{aligned}
\begin{bmatrix}
\eta^{\mu\nu}
\end{bmatrix}
&=
\begin{bmatrix}
1 & 0 & 0 & 0 \\
0 & -1 & 0 & 0 \\
0 & 0 & -1 & 0 \\
0 & 0 & 0 & -1 \\
\end{bmatrix}
\end{aligned}
\end{equation}
%
The relationship \eqnref{eqn:gamma:symmetric} is taken as the defining relationship for the Dirac matrices, but can be seen to be just a matricized statement of the Clifford vector dot product.

\subsection{Written as Pauli matrices}
\index{Pauli matrices}

Using the Pauli matrices
%
\begin{equation}\label{eqn:gamma:60}
\begin{aligned}
\sigma_1 = \PauliX \quad \sigma_2 = \PauliY \quad \sigma_3 = \PauliZ
\end{aligned}
\end{equation}
%
one can write the Dirac matrices and all their products (reading from the multiplication table) more concisely as
%
\begin{equation}\label{eqn:gamma:80}
\begin{aligned}
\gamma^0 &=
\begin{bmatrix}
I & 0 \\
0 & -I
\end{bmatrix} \\
\gamma^a &=
\begin{bmatrix}
0 & \sigma_a \\
-\sigma_a & 0 \\
\end{bmatrix} \\
\gamma^0 \gamma^a &=
\begin{bmatrix}
0 & \sigma_a \\
\sigma_a & 0 \\
\end{bmatrix} \\
\gamma^a \gamma^b &=
- i \epsilon_{a b c}
\begin{bmatrix}
\sigma_c & 0 \\
0 & \sigma_c \\
\end{bmatrix} \\
\gamma^1 \gamma^2 \gamma^3 &= i
\begin{bmatrix}
0 & -I \\
I & 0
\end{bmatrix} \\
\gamma^0 \gamma^1 \gamma^2 &= i
\begin{bmatrix}
-\sigma_1 & 0 \\
0 & \sigma_1 \\
\end{bmatrix} \\
\gamma^3 \gamma^0 \gamma^1 &= i
\begin{bmatrix}
\sigma_2 & 0 \\
0 & -\sigma_2 \\
\end{bmatrix} \\
\gamma^0 \gamma^1 \gamma^2 &= i
\begin{bmatrix}
-\sigma_3 & 0 \\
0 & \sigma_3 \\
\end{bmatrix}
\end{aligned}
\end{equation}
%
\subsection{Deriving properties using the Pauli matrices}

From the multiplication table a number of properties can be observed.  Using the Pauli matrices one can arrive at these more directly using the multiplication identity for those
matrices
%
\begin{equation}\label{eqn:gamma:100}
\begin{aligned}
\sigma_a \sigma_b = 2 i \epsilon_{abc} \sigma_c
\end{aligned}
\end{equation}
%
Actually taking the time to type this out in full does not seem worthwhile and is a fairly straightforward exercise.

\subsection{Conjugation behavior}
\index{conjugation}

Unlike the Pauli matrices, the Dirac matrices do not split nicely via conjugation.  Instead we have the time basis vector and its dual are Hermitian
%
\begin{equation}\label{eqn:gamma:120}
\begin{aligned}
(\gamma^0)^\conj &= \gamma^0 \\
(\gamma^1 \gamma^2 \gamma^3)^\conj &= \gamma^1 \gamma^2 \gamma^3
\end{aligned}
\end{equation}
%
whereas the spacelike basis vectors and their duals are all anti-Hermitian
%
\begin{equation}\label{eqn:gamma:140}
\begin{aligned}
(\gamma^a)^\conj &= -\gamma^a \\
(\gamma^a \gamma^b \gamma^c)^\conj &= - \gamma^a \gamma^b \gamma^c.
\end{aligned}
\end{equation}
%
For the scalar and the pseudoscalar parts we have a Hermitian split
%
\begin{equation}\label{eqn:gamma:160}
\begin{aligned}
I^\conj &= I \\
(\gamma^0 \gamma^1 \gamma^2 \gamma^3)^\conj &= -(\gamma^0 \gamma^1 \gamma^2 \gamma^3)^\conj
\end{aligned}
\end{equation}
%
and finally, also have a Hermitian split of the bivector parts into spacetime (relative vectors), and the purely spatial bivectors
%
\begin{equation}\label{eqn:gamma:180}
\begin{aligned}
(\gamma^0 \gamma^a)^\conj &= \gamma^0 \gamma^a \\
(\gamma^a \gamma^b)^\conj &= -\gamma^a \gamma^b
\end{aligned}
\end{equation}
%
Is there a logical and simple set of matrix operations that splits things nicely into scalar, vector, bivector, trivector, and pseudoscalar parts as there was with the Pauli
matrices?

\section{Appendix.  Table of all generated products}

A small C++ program using boost::numeric::ublas and std::complex,
plus some perl to generate part of that, was
written to generate the multiplication table for the gamma matrix products
for this particular basis.  The metric tensor and the antisymmetry of
the wedge products can be seen from these.

%% <GENERATED>


%
%
\begin{equation}\label{eqn:gamma:200}
\begin{aligned}
\gamma^0 \gamma^0 = \begin{bmatrix}
 1  &  0  &  0  &  0  \\
 0  &  1  &  0  &  0  \\
 0  &  0  &  1  &  0  \\
 0  &  0  &  0  &  1  \\
\end{bmatrix} \quad
\gamma^1 \gamma^1 = \begin{bmatrix}
 -1  &  0  &  0  &  0  \\
 0  &  -1  &  0  &  0  \\
 0  &  0  &  -1  &  0  \\
 0  &  0  &  0  &  -1  \\
\end{bmatrix}
\end{aligned}
\end{equation}
%
\begin{equation}\label{eqn:gamma:220}
\begin{aligned}
\gamma^2 \gamma^2 = \begin{bmatrix}
 -1  &  0  &  0  &  0  \\
 0  &  -1  &  0  &  0  \\
 0  &  0  &  -1  &  0  \\
 0  &  0  &  0  &  -1  \\
\end{bmatrix} \quad
\gamma^3 \gamma^3 = \begin{bmatrix}
 -1  &  0  &  0  &  0  \\
 0  &  -1  &  0  &  0  \\
 0  &  0  &  -1  &  0  \\
 0  &  0  &  0  &  -1  \\
\end{bmatrix}
\end{aligned}
\end{equation}
%
\begin{equation}\label{eqn:gamma:240}
\begin{aligned}
\gamma^0 \gamma^1 = \begin{bmatrix}
 0  &  0  &  0  &  1  \\
 0  &  0  &  1  &  0  \\
 0  &  1  &  0  &  0  \\
 1  &  0  &  0  &  0  \\
\end{bmatrix} \quad
\gamma^1 \gamma^0 = \begin{bmatrix}
 0  &  0  &  0  &  -1  \\
 0  &  0  &  -1  &  0  \\
 0  &  -1  &  0  &  0  \\
 -1  &  0  &  0  &  0  \\
\end{bmatrix}
\end{aligned}
\end{equation}
%
\begin{equation}\label{eqn:gamma:260}
\begin{aligned}
\gamma^0 \gamma^2 = \begin{bmatrix}
 0  &  0  &  0  &  -i  \\
 0  &  0  &  i  &  0  \\
 0  &  -i  &  0  &  0  \\
 i  &  0  &  0  &  0  \\
\end{bmatrix} \quad
\gamma^2 \gamma^0 = \begin{bmatrix}
 0  &  0  &  0  &  i  \\
 0  &  0  &  -i  &  0  \\
 0  &  i  &  0  &  0  \\
 -i  &  0  &  0  &  0  \\
\end{bmatrix}
\end{aligned}
\end{equation}
%
\begin{equation}\label{eqn:gamma:280}
\begin{aligned}
\gamma^0 \gamma^3 = \begin{bmatrix}
 0  &  0  &  1  &  0  \\
 0  &  0  &  0  &  -1  \\
 1  &  0  &  0  &  0  \\
 0  &  -1  &  0  &  0  \\
\end{bmatrix} \quad
\gamma^3 \gamma^0 = \begin{bmatrix}
 0  &  0  &  -1  &  0  \\
 0  &  0  &  0  &  1  \\
 -1  &  0  &  0  &  0  \\
 0  &  1  &  0  &  0  \\
\end{bmatrix}
\end{aligned}
\end{equation}
%
\begin{equation}\label{eqn:gamma:300}
\begin{aligned}
\gamma^1 \gamma^2 = \begin{bmatrix}
 -i  &  0  &  0  &  0  \\
 0  &  i  &  0  &  0  \\
 0  &  0  &  -i  &  0  \\
 0  &  0  &  0  &  i  \\
\end{bmatrix} \quad
\gamma^2 \gamma^1 = \begin{bmatrix}
 i  &  0  &  0  &  0  \\
 0  &  -i  &  0  &  0  \\
 0  &  0  &  i  &  0  \\
 0  &  0  &  0  &  -i  \\
\end{bmatrix}
\end{aligned}
\end{equation}
%
\begin{equation}\label{eqn:gamma:320}
\begin{aligned}
\gamma^1 \gamma^3 = \begin{bmatrix}
 0  &  1  &  0  &  0  \\
 -1  &  0  &  0  &  0  \\
 0  &  0  &  0  &  1  \\
 0  &  0  &  -1  &  0  \\
\end{bmatrix} \quad
\gamma^3 \gamma^1 = \begin{bmatrix}
 0  &  -1  &  0  &  0  \\
 1  &  0  &  0  &  0  \\
 0  &  0  &  0  &  -1  \\
 0  &  0  &  1  &  0  \\
\end{bmatrix}
\end{aligned}
\end{equation}
%
\begin{equation}\label{eqn:gamma:340}
\begin{aligned}
\gamma^2 \gamma^3 = \begin{bmatrix}
 0  &  -i  &  0  &  0  \\
 -i  &  0  &  0  &  0  \\
 0  &  0  &  0  &  -i  \\
 0  &  0  &  -i  &  0  \\
\end{bmatrix} \quad
\gamma^3 \gamma^2 = \begin{bmatrix}
 0  &  i  &  0  &  0  \\
 i  &  0  &  0  &  0  \\
 0  &  0  &  0  &  i  \\
 0  &  0  &  i  &  0  \\
\end{bmatrix}
\end{aligned}
\end{equation}
%
\begin{equation}\label{eqn:gamma:360}
\begin{aligned}
\gamma^1 \gamma^2 \gamma^3 = \begin{bmatrix}
 0  &  0  &  -i  &  0  \\
 0  &  0  &  0  &  -i  \\
 i  &  0  &  0  &  0  \\
 0  &  i  &  0  &  0  \\
\end{bmatrix} \quad
\gamma^2 \gamma^3 \gamma^0 = \begin{bmatrix}
 0  &  -i  &  0  &  0  \\
 -i  &  0  &  0  &  0  \\
 0  &  0  &  0  &  i  \\
 0  &  0  &  i  &  0  \\
\end{bmatrix}
\end{aligned}
\end{equation}
%
\begin{equation}\label{eqn:gamma:380}
\begin{aligned}
\gamma^3 \gamma^0 \gamma^1 = \begin{bmatrix}
 0  &  1  &  0  &  0  \\
 -1  &  0  &  0  &  0  \\
 0  &  0  &  0  &  -1  \\
 0  &  0  &  1  &  0  \\
\end{bmatrix} \quad
\gamma^0 \gamma^1 \gamma^2 = \begin{bmatrix}
 -i  &  0  &  0  &  0  \\
 0  &  i  &  0  &  0  \\
 0  &  0  &  i  &  0  \\
 0  &  0  &  0  &  -i  \\
\end{bmatrix}
\end{aligned}
\end{equation}
%
\begin{equation}\label{eqn:gamma:400}
\begin{aligned}
\gamma^0 \gamma^1 \gamma^2 \gamma^3 = \begin{bmatrix}
 0  &  0  &  -i  &  0  \\
 0  &  0  &  0  &  -i  \\
 -i  &  0  &  0  &  0  \\
 0  &  -i  &  0  &  0  \\
\end{bmatrix}
\end{aligned}
\end{equation}
%
%
%% </GENERATED>
