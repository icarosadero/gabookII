%
% Copyright � 2012 Peeter Joot.  All Rights Reserved.
% Licenced as described in the file LICENSE under the root directory of this GIT repository.
%

%
%
%\input{../peeter_prologue.tex}

\mychapter{Bivector form of quantum angular momentum operator}
\index{angular momentum operator}
\label{chap:qmAngularMom}

%\blogpage{http://sites.google.com/site/peeterjoot/math2009/qmAngularMom.pdf}
%%\date{July 27, 2009}
%%\revisionInfo{\(RCSfile: qmAngularMom.tex,v \) Last \(Revision: 1.15 \) \(Date: 2009/10/22 02:07:20 \)}

%\date{July 27, 2009.  \(RCSfile: qmAngularMom.tex,v \) Last \(Revision: 1.15 \) \(Date: 2009/10/22 02:07:20 \)}

\beginArtWithToc

\section{Spatial bivector representation of the angular momentum operator}

Reading \citep{bohm1989qt} on the angular momentum operator, the form of the operator is suggested by analogy where components of \(\Bx \cross \Bp\) with
the position representation \(\Bp \sim -i \Hbar \spacegrad\) used to expand the coordinate representation of the operator.

The result is the following coordinate representation of the operator

%\begin{align*}
%L_x &= -i \Hbar( y \partial_z - z \partial_y ) \\
%L_y &= -i \Hbar( z \partial_x - x \partial_z ) \\
%L_z &= -i \Hbar( x \partial_y - y \partial_x ) \\
%\end{align*}
\begin{equation}\label{eqn:qmAngularMom:37}
\begin{aligned}
L_1 &= -i \Hbar( x_2 \partial_3 - x_3 \partial_2 ) \\
L_2 &= -i \Hbar( x_3 \partial_1 - x_1 \partial_3 ) \\
L_3 &= -i \Hbar( x_1 \partial_2 - x_2 \partial_1 ) \\
\end{aligned}
\end{equation}

It is interesting to put these in vector form, and then employ the freedom to use for \(i = \sigma_1 \sigma_2 \sigma_3\) the spatial pseudoscalar.

\begin{equation}\label{eqn:qmAngularMom:57}
\begin{aligned}
\BL
&=
-\sigma_1 (\sigma_1 \sigma_2 \sigma_3) \Hbar( x_2 \partial_3 - x_3 \partial_2 )
-\sigma_2 (\sigma_2 \sigma_3 \sigma_1) \Hbar( x_3 \partial_1 - x_1 \partial_3 )
-\sigma_3 (\sigma_3 \sigma_1 \sigma_2) \Hbar( x_1 \partial_2 - x_2 \partial_1 ) \\
&=
-\sigma_2 \sigma_3 \Hbar( x_2 \partial_3 - x_3 \partial_2 )
-\sigma_3 \sigma_1 \Hbar( x_3 \partial_1 - x_1 \partial_3 )
-\sigma_1 \sigma_2 \Hbar( x_1 \partial_2 - x_2 \partial_1 ) \\
&=
-\Hbar ( \sigma_1 x_1 +\sigma_2 x_2 +\sigma_3 x_3 ) \wedge ( \sigma_1 \partial_1 +\sigma_2 \partial_2 +\sigma_3 \partial_3 ) \\
\end{aligned}
\end{equation}

The choice to use the pseudoscalar for this imaginary seems a logical one and the end result is a pure bivector representation of angular momentum operator

\begin{equation}\label{eqn:qmAngularMom:ang}
\BL = - \Hbar \Bx \wedge \spacegrad
\end{equation}

The choice to represent angular momentum as a bivector \(\Bx \wedge \Bp\) is also natural in classical mechanics (encoding the orientation of the plane and the magnitude of the momentum in the bivector), although its dual form the axial vector \(\Bx \cross \Bp\) is more common, at least in introductory mechanics.  Observe that there is no longer any explicit imaginary in \eqnref{eqn:qmAngularMom:ang}, since the bivector itself has an implicit complex structure.

\section{Factoring the gradient and Laplacian}

The form of \eqnref{eqn:qmAngularMom:ang} suggests a more direct way to extract the angular momentum operator from the Hamiltonian (i.e. from the Laplacian).  Bohm uses the spherical polar representation of the Laplacian as the starting point.  Instead let us project the gradient itself in a specific constant direction \(\Ba\), much as we can do to find the polar form angular velocity and acceleration components.

Write

\begin{equation}\label{eqn:qmAngularMom:77}
\begin{aligned}
\spacegrad
&=
\inv{\Ba} \Ba \spacegrad \\
&=
\inv{\Ba} (\Ba \cdot \spacegrad + \Ba \wedge \spacegrad) \\
\end{aligned}
\end{equation}

Or
\begin{equation}\label{eqn:qmAngularMom:97}
\begin{aligned}
\spacegrad
&=
\spacegrad \Ba \inv{\Ba} \\
&=
(\spacegrad \cdot \Ba + \spacegrad \wedge \Ba) \inv{\Ba} \\
&=
(\Ba \cdot \spacegrad - \Ba \wedge \spacegrad) \inv{\Ba} \\
\end{aligned}
\end{equation}

The Laplacian is therefore

\begin{equation}\label{eqn:qmAngularMom:117}
\begin{aligned}
\spacegrad^2
&=
\gpgradezero{ \spacegrad^2 } \\
&=
\gpgradezero{
(\Ba \cdot \spacegrad - \Ba \wedge \spacegrad) \inv{\Ba} \inv{\Ba} (\Ba \cdot \spacegrad + \Ba \wedge \spacegrad)
} \\
&=
\inv{\Ba^2} \gpgradezero{
(\Ba \cdot \spacegrad - \Ba \wedge \spacegrad) (\Ba \cdot \spacegrad + \Ba \wedge \spacegrad)
} \\
&=
\inv{\Ba^2} ((\Ba \cdot \spacegrad)^2 - (\Ba \wedge \spacegrad)^2 ) \\
\end{aligned}
\end{equation}

So we have for the Laplacian a representation in terms of projection and rejection components

\begin{equation}\label{eqn:qmAngularMom:137}
\begin{aligned}
\spacegrad^2
&=
(\acap \cdot \spacegrad)^2 - \inv{\Ba^2} (\Ba \wedge \spacegrad)^2 \\
&=
(\acap \cdot \spacegrad)^2 - (\acap \wedge \spacegrad)^2 \\
\end{aligned}
\end{equation}

The vector \(\Ba\) was arbitrary, and just needed to be constant with respect to the factorization operations.  Setting \(\Ba = \Bx\), the radial position from the origin one may guess that we have

\begin{equation}\label{eqn:qmAngularMom:wrong}
\spacegrad^2 = \frac{\partial^2 }{\partial r^2} - \inv{\Bx^2} (\Bx \wedge \spacegrad)^2
\end{equation}

however, with the switch to a non-constant position vector \(\Bx\), this cannot possibly be right.

\section{The Coriolis term}

The radial factorization of the gradient relied on the direction vector \(\Ba\) being constant.  If we evaluate \eqnref{eqn:qmAngularMom:wrong}, then there should be a non-zero remainder compared to the Laplacian.  Evaluation by coordinate expansion is one way to verify this, and should produce the difference.  Let us do this in two parts, starting with the scalar part of \((x \wedge \grad)^2\).  Summation will be implied by mixed indices, and for generality a general basis and associated reciprocal frame will be used.

\begin{equation}\label{eqn:qmAngularMom:157}
\begin{aligned}
\gpgradezero{ (x \wedge \grad)^2 } f
&=
((x^\mu \gamma_\mu) \wedge (\gamma_\nu \partial^\nu)) \cdot
((x_\alpha \gamma^\alpha) \wedge (\gamma^\beta \partial_\beta)) f \\
&=
(\gamma_\mu \wedge \gamma_\nu) \cdot (\gamma^\alpha \wedge \gamma^\beta) x^\mu \partial^\nu (x_\alpha \partial_\beta) f \\
&=
({\delta_\mu}^\beta {\delta_\nu}^\alpha -{\delta_\mu}^\alpha {\delta_\nu}^\beta) x^\mu \partial^\nu (x_\alpha \partial_\beta) f \\
&=
x^\mu \partial^\nu ((x_\nu \partial_\mu) - x_\mu \partial_\nu) f \\
&=
x^\mu (\partial^\nu x_\nu) \partial_\mu f - x^\mu (\partial^\nu x_\mu) \partial_\nu f \\
&+x^\mu x_\nu \partial^\nu \partial_\mu f - x^\mu x_\mu \partial^\nu \partial_\nu f \\
&=
(n-1) x \cdot \grad f +x^\mu x_\nu \partial^\nu \partial_\mu f - x^2 \grad^2 f \\
\end{aligned}
\end{equation}

For the dot product we have
\begin{equation}\label{eqn:qmAngularMom:177}
\begin{aligned}
\gpgradezero{ (x \cdot \grad)^2 } f
&=
x^\mu \partial_\mu( x^\nu \partial_\nu ) f \\
&=
x^\mu (\partial_\mu x^\nu) \partial_\nu  f + x^\mu x^\nu \partial_\mu \partial_\nu f \\
&=
x^\mu \partial_\mu f + x^\mu x_\nu \partial^\nu \partial_\mu f \\
&=
x \cdot \grad f + x^\mu x_\nu \partial^\nu \partial_\mu f \\
\end{aligned}
\end{equation}

So, forming the difference we have

\begin{equation}\label{eqn:qmAngularMom:197}
(x \cdot \grad)^2 f - \gpgradezero{(x \wedge \grad)^2} f =
-(n - 2) x \cdot \grad f + x^2 \grad^2 f \\
\end{equation}

Or
\begin{equation}\label{eqn:qmAngularMom:withScalar}
\grad^2 = \inv{x^2} (x \cdot \grad)^2 - \inv{x^2} \gpgradezero{(x \wedge \grad)^2} + (n - 2) \inv{x} \cdot \grad
\end{equation}

\section{On the bivector and quadvector components of the squared angular momentum operator}

The requirement for a scalar selection on all the \((x \wedge \grad)^2\) terms is a bit ugly, but omitting it would be incorrect for two reasons.  One reason is that this is a bivector operator and not a bivector (where the squaring operates on itself).  The other is that we derived a result for arbitrary dimension, and the product of two bivectors in a general space has grade 2 and grade 4 terms in addition to the scalar terms.  Without taking only the scalar parts, lets expand this product a bit more carefully, starting with

\begin{equation}\label{eqn:qmAngularMom:217}
(x \wedge \grad)^2
=
(\gamma_\mu \wedge \gamma_\nu) (\gamma^\alpha \wedge \gamma^\beta)
x^\mu \partial^\nu x_\alpha \partial_\beta
\end{equation}

Just expanding the multivector factor for now, we have

\begin{equation}\label{eqn:qmAngularMom:237}
\begin{aligned}
&2 (\gamma_\mu \wedge \gamma_\nu) (\gamma^\alpha \wedge \gamma^\beta) \\
&=
\gamma_\mu \gamma_\nu (\gamma^\alpha \wedge \gamma^\beta)
- \gamma_\nu \gamma_\mu (\gamma^\alpha \wedge \gamma^\beta) \\
&=
\gamma_\mu
\left(
{\delta_\nu}^\alpha \gamma^\beta - {\delta_\nu}^\beta \gamma^\alpha
+ \gamma_\nu \wedge \gamma^\alpha \wedge \gamma^\beta \right)
-
\gamma_\nu \left(
{\delta_\mu}^\alpha \gamma^\beta - {\delta_\mu}^\beta \gamma^\alpha
+ \gamma_\mu \wedge \gamma^\alpha \wedge \gamma^\beta \right)
\\
&=
{\delta_\nu}^\alpha {\delta_\mu}^\beta - {\delta_\nu}^\beta {\delta_\mu}^\alpha
-{\delta_\mu}^\alpha {\delta_\nu}^\beta + {\delta_\mu}^\beta {\delta_\nu}^\alpha \\
&+ \gamma_\mu \wedge \gamma_\nu \wedge \gamma^\alpha \wedge \gamma^\beta
- \gamma_\nu \wedge \gamma_\mu \wedge \gamma^\alpha \wedge \gamma^\beta \\
&+ \gamma_\mu \cdot (\gamma_\nu \wedge \gamma^\alpha \wedge \gamma^\beta )
-\gamma_\nu \cdot (\gamma_\mu \wedge \gamma^\alpha \wedge \gamma^\beta ) \\
\end{aligned}
\end{equation}

Our split into grades for this operator is then, the scalar

\begin{equation}\label{eqn:qmAngularMom:257}
\begin{aligned}
\gpgradezero{(x \wedge \grad)^2 }
&= (x \wedge \grad) \cdot (x \wedge \grad) \\
&= \lr{ {\delta_\nu}^\alpha {\delta_\mu}^\beta - {\delta_\nu}^\beta {\delta_\mu}^\alpha  }
x^\mu \partial^\nu x_\alpha \partial_\beta \\
\end{aligned}
\end{equation}

the pseudoscalar (or grade 4 term in higher than 4D spaces).

\begin{equation}\label{eqn:qmAngularMom:277}
\begin{aligned}
\gpgradefour{(x \wedge \grad)^2 }
&= (x \wedge \grad) \wedge (x \wedge \grad) \\
&= \lr{ \gamma_\mu \wedge \gamma_\nu \wedge \gamma^\alpha \wedge \gamma^\beta   }
x^\mu \partial^\nu x_\alpha \partial_\beta \\
\end{aligned}
\end{equation}

If we work in dimensions less than or equal to three, we will have no grade four term since this wedge product is zero (irrespective of the operator action), so in 3D we have only a bivector term in excess of the scalar part of this operator.

The bivector term deserves some reduction, but is messy to do so.  This has been done separately in (\chapcite{bivectorSelect})

We can now write for the squared operator

\begin{equation}\label{eqn:qmAngularMom:onGradeTwo3}
(x \wedge \grad)^2 =
(n-2)(x \wedge \grad)
+
(x \wedge \grad) \wedge (x \wedge \grad)
+(x \wedge \grad) \cdot (x \wedge \grad)
\end{equation}

and then eliminate the scalar selection from the \eqnref{eqn:qmAngularMom:withScalar}

\begin{equation}\label{eqn:qmAngularMom:onGradeTwo4}
\grad^2 = \inv{x^2} (x \cdot \grad)^2 + (n - 2) \inv{x} \cdot \grad
- \inv{x^2}
\left(
(x \wedge \grad)^2 - (n-2) (x \wedge \grad) - (x \wedge \grad) \wedge (x \wedge \grad)
\right)
\end{equation}

In 3D this is

\begin{equation}\label{eqn:qmAngularMom:onGradeTwo5}
\spacegrad^2 = \inv{\Bx^2} (\Bx \cdot \spacegrad)^2 + \inv{\Bx} \cdot \spacegrad
- \inv{\Bx^2} \lr{ \Bx \wedge \spacegrad - 1  } (\Bx \wedge \spacegrad)
\end{equation}

Wow, that was an ugly mess of algebra.  The worst of it for the bivector grades was initially incorrect and the correct handling omitted.  There is likely a more clever coordinate free way to do the same expansion.  We will see later that at least a partial verification of \eqnref{eqn:qmAngularMom:onGradeTwo5} can be obtained by considering of the Quantum eigenvalue problem, examining simultaneous eigenvalues of \(\Bx \wedge \spacegrad\), and \(\gpgradezero{\Bx \wedge \spacegrad)^2}\).  However, lets revisit this after examining the radial terms in more detail, and also after verifying that at least in the scalar selection form, this factorized Laplacian form has the same structure as the Laplacian in scalar \(r\), \(\theta\), and \(\phi\) operator form.

FIXME: the reduction of the scalar selection term doesn't look right: and appears to leave a bivector term in an otherwise scalar equation.  With that term in place, this doens't match the same identity \citep{sakurai2014modern} \texteqnref{6.16}, whereas \cref{eqn:qmSum:goo6} does.  Does that cancel out when \( \lr{ \Bx \wedge \spacegrad }^2 \) is expanded?

\section{Correspondence with explicit radial form}

We have seen above that we can factor the 3D Laplacian as

\begin{equation}\label{eqn:qmAngularMom:3dLaplacian}
\spacegrad^2 \psi = \inv{\Bx^2}( (\Bx \cdot \spacegrad)^2 + \Bx \cdot \spacegrad - \gpgradezero{ (\Bx \wedge \spacegrad)^2 } ) \psi
\end{equation}

Contrast this to the explicit \(r,\theta,\phi\) form as given in (Bohm's \citep{bohm1989qt}, 14.2)

\begin{equation}\label{eqn:qmAngularMom:LaplacianRTP}
\spacegrad^2 \psi = \inv{r} \frac{\partial^2}{\partial r^2} (r\psi) + \inv{r^2} \left(
\inv{\sin\theta} \partial_\theta \sin\theta \partial_\theta + \inv{\sin^2\theta} + \partial_{\phi \phi} \right) \psi
\end{equation}

Let us expand out the non-angular momentum operator terms explicitly as a partial verification of this factorization.  The radial term in Bohm's Laplacian formula expands out to

\begin{equation}\label{eqn:qmAngularMom:297}
\begin{aligned}
\inv{r} \frac{\partial^2}{\partial r^2} (r\psi)
&=
\inv{r} \partial_r (\partial_r r \psi) \\
&=
\inv{r} \partial_r (\psi + r\partial_r \psi) \\
&=
\inv{r} \partial_r \psi + \inv{r}( \partial_r \psi + r \partial_{rr} \psi) \\
&=
\frac{2}{r} \partial_r \psi + \partial_{rr} \psi \\
\end{aligned}
\end{equation}

On the other hand, with \(\Bx = r\rcap\), what we expect to correspond to the radial term in the vector factorization is

\begin{equation}\label{eqn:qmAngularMom:317}
\begin{aligned}
\inv{\Bx^2}( (\Bx \cdot \spacegrad)^2 + \Bx \cdot \spacegrad ) \psi
&=
\inv{r^2}( (r \rcap \cdot \spacegrad)^2 + r \rcap \cdot \spacegrad  ) \psi \\
&=
\inv{r^2}( (r \partial_r )^2 + r \partial_r  ) \psi \\
&=
\inv{r^2}( r \partial_r \psi + r^2 \partial_{rr} \psi + r \partial_r \psi ) \\
&=
\frac{2}{r} \partial_r \psi + \partial_{rr} \psi
\end{aligned}
\end{equation}

Okay, good.  It is a brute force way to verify things, but it works.  With \(\Bx \wedge \spacegrad = I (\Bx \cross \spacegrad)\) we can eliminate the wedge product from the factorization expression \eqnref{eqn:qmAngularMom:3dLaplacian} and express things completely in quantities that can be understood without any resort to Geometric Algebra.  That is

\begin{equation}\label{eqn:qmAngularMom:LaplacianRadialCross}
\spacegrad^2 \psi = \inv{r} \frac{\partial^2}{\partial r^2} (r\psi) + \inv{r^2} \gpgradezero{ (\Bx \cross \spacegrad)^2 } \psi
\end{equation}

Bohm resorts to analogy and an operatorization of \(L_c = \epsilon_{abc} (x_a p_b - x_b p_a)\), then later a spherical polar change of coordinates to match exactly the \(L^2\) expression with \eqnref{eqn:qmAngularMom:LaplacianRTP}.  With the GA formalism we see this a bit more directly, although it is not the least bit obvious that the operator \(\Bx \cross \spacegrad\) has no radial dependence.  Without resorting to a comparison with the explicit \(r,\theta,\phi\) form that would not be so easy to see.

\section{Raising and Lowering operators in GA form}

Having seen in (\chapcite{L1Associated}) that we have a natural GA form for the \(l=1\) spherical harmonic eigenfunctions \(\psi_1^{m}\), and that we have the vector angular momentum operator \(\Bx \cross \spacegrad\) showing up directly in a sort-of-radial factorization of the Laplacian, it is natural to wonder what the GA form of the raising and lowering operators are.  At least for the \(l=1\) harmonics use of \(i = I \Be_3\) (unit bivector for the \(x-y\) plane) for the imaginary ended up providing a nice geometric interpretation.

Let us see what that provides for the raising and lowering operators.  First we need to express \(L_x\) and \(L_y\) in terms of our bivector angular momentum operator.  Let us switch notations and drop the \(-i \Hbar\) factor from \eqnref{eqn:qmAngularMom:ang} writing just

\begin{equation}\label{eqn:qmAngularMom:Ang}
\BL = \Bx \wedge \spacegrad
\end{equation}

We can now write this in terms of components with respect to the basis bivectors \(I \Be_k\).  That is

\begin{equation}\label{eqn:qmAngularMom:BLprojected}
\BL = \sum_k \lr{ (\Bx \wedge \spacegrad) \cdot \inv{I \Be_k} } I \Be_k
\end{equation}

These scalar product results are expected to match the \(L_x\), \(L_y\), and \(L_z\) components at least up to a sign.  Let us check, picking \(L_z\) as representative

\begin{equation}\label{eqn:qmAngularMom:337}
\begin{aligned}
(\Bx \wedge \spacegrad) \cdot \inv{I \Be_3}
&=
(\sigma_m \wedge \sigma^k) \cdot {-\sigma_1 \sigma_2 \sigma_3 \sigma_3} x^m \partial_k \\
&=
(\sigma_m \wedge \sigma^k) \cdot {-\sigma_1 \sigma_2} x^m \partial_k \\
&=
-( x^2 \partial_1 - x^1 \partial_2 )
\end{aligned}
\end{equation}

With the \(-i\Hbar\) factors dropped this is \(L_z = L_3 = x^1 \partial_2 - x^2 \partial_1\), the projection of \(\BL\) onto the \(x-y\) plane \(I \Be_k\).  So, now how about the raising and lowering operators

\begin{equation}\label{eqn:qmAngularMom:357}
\begin{aligned}
L_x \pm i L_y
&=
L_x \pm I \Be_3 L_y \\
&=
\BL \cdot \inv{I\Be_1} \pm I \Be_3 \BL \cdot \inv{I\Be_2} \\
&=
-\Be_1 I \lr{ I \Be_1 \BL \cdot \inv{I\Be_1} \pm I \Be_2 \BL \cdot \inv{I\Be_2}  } \\
\end{aligned}
\end{equation}

Or
\begin{equation}\label{eqn:qmAngularMom:raisingLoweringBivector}
(I \Be_1) L_x \pm i L_y = I \Be_1 \BL \cdot \inv{I\Be_1} \pm I \Be_2 \BL \cdot \inv{I\Be_2}
\end{equation}

Compare this to the projective split of \(\BL\) \eqnref{eqn:qmAngularMom:BLprojected}.  We have projections of the bivector angular momentum operator onto the bivector directions \(I\Be_1\) and \(I\Be_2\) (really the bivectors for the planes perpendicular to the \(\xcap\) and \(\ycap\) directions).

We have the Laplacian in explicit vector form and have a clue how to vectorize (really bivectorize) the raising and lowering operators.  We have also seen how to geometrize the first spherical harmonics.  The next logical step is to try to apply this vector form of the raising and lowering operators to the vector form of the spherical harmonics.

\section{Explicit expansion of the angular momentum operator}

There is a couple of things to explore before going forward.  One is an explicit verification that \(\Bx \wedge \spacegrad\) has no radial dependence (something not obvious).  Another is that we should be able to compare the \(\Bx^{-2} (\Bx \wedge \spacegrad)^2\) (as done for the \(\Bx \cdot \spacegrad\) terms) the explicit \(r,\theta,\phi\) expression for the Laplacian to verify consistency and correctness.

For the spherical polar rotation we use the rotor

\begin{equation}\label{eqn:qmAngularMom:angExp1}
R = e^{\Be_{31}\theta/2} e^{\Be_{12}\phi/2}
\end{equation}

Our position vector and gradient in spherical polar coordinates are

\begin{equation}\label{eqn:qmAngularMom:angExp2}
\Bx = r \tilde{R} \Be_3 R
\end{equation}

\begin{equation}\label{eqn:qmAngularMom:angExp3}
\spacegrad = \rcap \partial_r + \thetacap \inv{r} \partial_\theta + \phicap \inv{r \sin\theta} \partial_\phi
\end{equation}

with the unit vectors translate from the standard basis as

\begin{equation}\label{eqn:qmAngularMom:angExp4}
\begin{pmatrix}
\rcap \\
\thetacap \\
\phicap \\
\end{pmatrix}
=
\tilde{R}
\begin{pmatrix}
\Be_3 \\
\Be_1 \\
\Be_2 \\
\end{pmatrix}
R
\end{equation}

This last mapping can be used to express the gradient unit vectors in terms of the standard basis, as we did for the position vector \(\Bx\).  That is

\begin{equation}\label{eqn:qmAngularMom:angExp5}
\spacegrad = \tilde{R} \lr{ \Be_3 R \partial_r + \Be_1 R \inv{r} \partial_\theta + \Be_2 R \inv{r \sin\theta} \partial_\phi  }
\end{equation}

Okay, we have now got all the pieces collected, ready to evaluate \(\Bx \wedge \spacegrad\)

\begin{equation}\label{eqn:qmAngularMom:377}
\begin{aligned}
\Bx \wedge \spacegrad
&=
r \gpgradetwo{\tilde{R} \Be_3 R
\tilde{R} \lr{ \Be_3 R \partial_r + \Be_1 R \inv{r} \partial_\theta + \Be_2 R \inv{r \sin\theta} \partial_\phi  } } \\
&=
r \gpgradetwo{\tilde{R} \lr{ R \partial_r + \Be_3 \Be_1 R \inv{r} \partial_\theta + \Be_3 \Be_2 R \inv{r \sin\theta} \partial_\phi  } } \\
\end{aligned}
\end{equation}

Observe that the \({\Be_3}^2\) contribution is only a scalar, so bivector selection of that is zero.  In the remainder we have cancellation of \(r/r\) factors, leaving just

\begin{equation}\label{eqn:qmAngularMom:angExp6}
\Bx \wedge \spacegrad
=
\tilde{R} \lr{ \Be_3 \Be_1 R \partial_\theta + \Be_3 \Be_2 R \inv{\sin\theta} \partial_\phi  }
\end{equation}

Using \eqnref{eqn:qmAngularMom:angExp4} this is

\begin{equation}\label{eqn:qmAngularMom:angExp7}
\Bx \wedge \spacegrad
=
\rcap \lr{ \thetacap \partial_\theta + \phicap \inv{\sin\theta} \partial_\phi  }
\end{equation}

As hoped, there is no explicit radial dependence here, taking care of the first of the desired verifications.

Next we want to square this operator.  It should be noted that in the original derivation where we ``factored'' the gradient operator with respect to the reference vector \(\Bx\) our Laplacian really followed by considering \((\Bx \wedge \spacegrad)^2 \equiv \gpgradezero{(\Bx \wedge \spacegrad)^2}\).  That is worth noting since a regular bivector would square to a negative constant, whereas the operator factors of the vectors in this expression do not intrinsically commute.

An additional complication for evaluating the square of \(\Bx \wedge \spacegrad\) using the result of \eqnref{eqn:qmAngularMom:angExp7} is that \(\thetacap\) and \(\rcap\) are functions of \(\theta\) and \(\phi\), so we would have to operate on those too.  Without that operator subtlety we get the wrong answer

\begin{equation}\label{eqn:qmAngularMom:397}
\begin{aligned}
-\gpgradezero{ (\Bx \wedge \spacegrad)^2 }
&=
\gpgradezero{
\tilde{R} \lr{ \Be_1 R \partial_\theta + \frac{\Be_2 R}{\sin\theta}\partial_\phi  }
\tilde{R} \lr{ \Be_1 R \partial_\theta + \frac{\Be_2 R}{\sin\theta}\partial_\phi  }
 } \\
&\ne
\partial_{\theta\theta} + \inv{\sin^2\theta} \partial_{\phi\phi}
\end{aligned}
\end{equation}

Equality above would only be if the unit vectors were fixed.  By comparison we also see that this is missing a \(\cot\theta \partial_\theta\) term.  That must come from the variation of the unit vectors with position in the second application of \(\Bx \wedge \spacegrad\).

\section{Derivatives of the unit vectors}

To properly evaluate the angular momentum square we will need to examine the \(\partial_\theta\) and \(\partial_\phi\) variation of the unit vectors \(\rcap\), \(\thetacap\), and \(\phicap\).  Some part of this question can be evaluated without reference to the specific vector or even which derivative is being evaluated.  Writing \(e\) for one of \(\Be_1\), \(\Be_2\), or \(\Be_k\), and \(\sigma = \tilde{R} e R\) for the mapping of this vector under rotation, and \(\partial\) for the desired \(\theta\) or \(\phi\) partial derivative, we have

\begin{equation}\label{eqn:qmAngularMom:rotationGen1}
\partial (\tilde{R} e R)
=
(\partial \tilde{R}) e R + \tilde{R} e (\partial R)
\end{equation}

Since \(\tilde{R} R = 1\), we have

\begin{equation}\label{eqn:qmAngularMom:417}
\begin{aligned}
0
&= \partial (\tilde{R} R) \\
&=
(\partial \tilde{R}) R + \tilde{R} (\partial R)
\end{aligned}
\end{equation}

So substitution of \((\partial \tilde{R}) = -\tilde{R} (\partial R) \tilde{R}\), back into \eqnref{eqn:qmAngularMom:rotationGen1} supplies

\begin{equation}\label{eqn:qmAngularMom:437}
\begin{aligned}
\partial (\tilde{R} e R)
&=
-\tilde{R} (\partial R) \tilde{R} e R + \tilde{R} e (\partial R) \\
&=
-\tilde{R} (\partial R) (\tilde{R} e R) + (\tilde{R} e R) \tilde{R} (\partial R) \\
&=
-\tilde{R} (\partial R) \sigma + \sigma \tilde{R} (\partial R) \\
\end{aligned}
\end{equation}

Writing the bivector term as

\begin{equation}\label{eqn:qmAngularMom:rotationGen2}
\Omega = \tilde{R} (\partial R)
\end{equation}

The change in the rotated vector is seen to be entirely described by the commutator of that vectors image under rotation with \(\Omega\).  That is

\begin{equation}\label{eqn:qmAngularMom:rotationGen3}
\partial \sigma = \antisymmetric{\sigma}{\Omega}
\end{equation}

Our spherical polar rotor was given by

\begin{equation}\label{eqn:qmAngularMom:rotationGen4}
R = e^{\Be_{31}\theta/2} e^{\Be_{12}\phi/2}
\end{equation}

Lets calculate the \(\Omega\) bivector for each of the \(\theta\) and \(\phi\) partials.  For \(\theta\) we have

\begin{equation}\label{eqn:qmAngularMom:457}
\begin{aligned}
\Omega_\theta &= \tilde{R} \partial_\theta R \\
&= \frac{1}{2} e^{-\Be_{12}\phi/2} e^{-\Be_{31}\theta/2} \Be_{31} e^{\Be_{31}\theta/2} e^{\Be_{12}\phi/2} \\
&= \frac{1}{2} e^{-\Be_{12}\phi/2} \Be_{31} e^{\Be_{12}\phi/2} \\
&= \frac{1}{2} \Be_3 e^{-\Be_{12}\phi/2} \Be_{1} e^{\Be_{12}\phi/2} \\
&= \frac{1}{2} \Be_{31} e^{\Be_{12}\phi} \\
\end{aligned}
\end{equation}

Explicitly, this is the bivector \(\Omega_\theta = (\Be_{31} \cos\theta + \Be_{32} \sin\theta)/2\), a wedge product of a vectors in \(\zcap\) direction with one in the perpendicular \(x-y\) plane (curiously a vector in the \(x-y\) plane rotated by polar angle \(\theta\), not the equatorial angle \(\phi\)).

FIXME: picture.  Draw this plane cutting through the sphere.

For the \(\phi\) partial variation of any of our unit vectors our bivector rotation generator is

\begin{equation}\label{eqn:qmAngularMom:477}
\begin{aligned}
\Omega_\phi &= \tilde{R} \partial_\phi R \\
&= \frac{1}{2} e^{-\Be_{12}\phi/2} e^{-\Be_{31}\theta/2} e^{\Be_{31}\theta/2} \Be_{12} e^{\Be_{12}\phi/2} \\
&= \frac{1}{2} \Be_{12} \\
\end{aligned}
\end{equation}

This one has no variation at all with angle whatsoever.  If this is all correct so far perhaps that is not surprising given the fact that we expect an extra \(\cot\theta\) in the angular momentum operator square, so a lack of \(\phi\) dependence in the rotation generator likely means that any additional \(\phi\) dependence will cancel out.  Next step is to take these rotation generator bivectors, apply them via commutator products to the \(\rcap\), \(\thetacap\), and \(\phicap\) vectors, and see what we get.

\section{Applying the vector derivative commutator (or not)}

Let us express the \(\thetacap\) and \(\phicap\) unit vectors explicitly in terms of the standard basis.  Starting with \(\thetacap\) we have

\begin{equation}\label{eqn:qmAngularMom:497}
\begin{aligned}
\thetacap
&= \tilde{R} \Be_1 R \\
&= e^{-\Be_{12}\phi/2} e^{-\Be_{31}\theta/2} \Be_1 e^{\Be_{31}\theta/2} e^{\Be_{12}\phi/2} \\
&= e^{-\Be_{12}\phi/2} \Be_1 e^{\Be_{31}\theta} e^{\Be_{12}\phi/2} \\
&= e^{-\Be_{12}\phi/2} (\Be_1 \cos\theta -\Be_3 \sin\theta) e^{\Be_{12}\phi/2} \\
&= \Be_1 \cos\theta e^{\Be_{12}\phi} -\Be_3 \sin\theta
\end{aligned}
\end{equation}

Explicitly in vector form, eliminating the exponential, this is \(\thetacap = \Be_1 \cos\theta \cos\phi + \Be_2 \cos\theta\sin\phi - \Be_3\sin\theta\), but it is more convenient to keep the exponential as is.

For \(\phicap\) we have
\begin{equation}\label{eqn:qmAngularMom:517}
\begin{aligned}
\phicap
&= \tilde{R} \Be_2 R \\
&= e^{-\Be_{12}\phi/2} e^{-\Be_{31}\theta/2} \Be_2 e^{\Be_{31}\theta/2} e^{\Be_{12}\phi/2} \\
&= e^{-\Be_{12}\phi/2} \Be_2 e^{\Be_{12}\phi/2} \\
&= \Be_2 e^{\Be_{12}\phi} \\
\end{aligned}
\end{equation}

Again, explicitly this is \(\phicap = \Be_2 \cos\phi - \Be_1 \sin\phi\), but we will use the exponential form above.  Last we want \(\rcap\)

\begin{equation}\label{eqn:qmAngularMom:537}
\begin{aligned}
\rcap
&= \tilde{R} \Be_3 R \\
&= e^{-\Be_{12}\phi/2} e^{-\Be_{31}\theta/2} \Be_3 e^{\Be_{31}\theta/2} e^{\Be_{12}\phi/2} \\
&= e^{-\Be_{12}\phi/2} \Be_3 e^{\Be_{31}\theta} e^{\Be_{12}\phi/2} \\
&= e^{-\Be_{12}\phi/2} (\Be_3 \cos\theta + \Be_1 \sin\theta) e^{\Be_{12}\phi/2} \\
&= \Be_3 \cos\theta + \Be_1 \sin\theta e^{\Be_{12}\phi} \\
\end{aligned}
\end{equation}

Summarizing we have
\begin{equation}\label{eqn:qmAngularMom:vecBivecDerivatives1}
\begin{aligned}
\thetacap &= \Be_1 \cos\theta e^{\Be_{12}\phi} -\Be_3 \sin\theta \\
\phicap &= \Be_2 e^{\Be_{12}\phi} \\
\rcap &= \Be_3 \cos\theta + \Be_1 \sin\theta e^{\Be_{12}\phi}
\end{aligned}
\end{equation}

Or without exponentials
\begin{equation}\label{eqn:qmAngularMom:vecBivecDerivatives2}
\begin{aligned}
\thetacap &= \Be_1 \cos\theta \cos\phi + \Be_2 \cos\theta\sin\phi - \Be_3\sin\theta \\
\phicap &= \Be_2 \cos\phi - \Be_1 \sin\phi \\
\rcap &= \Be_3 \cos\theta + \Be_1 \sin\theta \cos\phi + \Be_2 \sin\theta \sin\phi
\end{aligned}
\end{equation}

Now, having worked out the cool commutator result, it appears that it will actually be harder to use it, then to just calculate the derivatives directly (at least for the \(\phicap\) derivatives).  For those we have

\begin{equation}\label{eqn:qmAngularMom:557}
\begin{aligned}
\partial_\theta \phicap
&= \partial_\theta \Be_2 e^{\Be_{12}\phi} \\
&= 0
\end{aligned}
\end{equation}

and

\begin{equation}\label{eqn:qmAngularMom:577}
\begin{aligned}
\partial_\phi \phicap
&= \partial_\phi \Be_2 e^{\Be_{12}\phi} \\
&= \Be_2 \Be_{12} e^{\Be_{12}\phi} \\
&= -\Be_{12} \phicap
\end{aligned}
\end{equation}

This multiplication takes \(\phicap\) a vector in the \(x,y\) plane and rotates it 90 degrees, leaving an inwards facing radial unit vector in the x,y plane.

Now, having worked out the commutator method, lets at least verify that it works.

\begin{equation}\label{eqn:qmAngularMom:597}
\begin{aligned}
\partial_\theta \phicap
&= \antisymmetric{\phicap}{\Omega_\theta} \\
&= \phicap \Omega_\theta - \Omega_\theta \phicap \\
&= \inv{2} (\phicap \Be_{31} e^{\Be_{12}\phi} - \Be_{31} e^{\Be_{12}\phi} \phicap)  \\
&= \inv{2} (\Be_2 \Be_3 \Be_1 e^{-\Be_{12}\phi} e^{\Be_{12}\phi} - \Be_{3} \Be_{1} \Be_{2} e^{-\Be_{12}\phi} e^{\Be_{12}\phi})  \\
&= \inv{2} (-\Be_3 \Be_2 \Be_1 - \Be_{3} \Be_{1} \Be_{2} )  \\
&= 0
\end{aligned}
\end{equation}

Much harder this way compared to taking the derivative directly, but we at least get the right answer.  For the \(\phi\) derivative using the commutator we have

\begin{equation}\label{eqn:qmAngularMom:617}
\begin{aligned}
\partial_\phi \phicap
&= \antisymmetric{\phicap}{\Omega_\phi} \\
&= \phicap \Omega_\phi - \Omega_\phi \phicap \\
&= \inv{2} (\phicap \Be_{12} - \Be_{12} \phicap)  \\
&= \inv{2} (\Be_2 e^{\Be_{12}\phi} \Be_{12} - \Be_{12} \Be_2 e^{\Be_{12}\phi})  \\
&= \inv{2} (-\Be_{12} \Be_2 e^{\Be_{12}\phi} - \Be_{12} \Be_2 e^{\Be_{12}\phi})  \\
&= -\Be_{12} \phicap
\end{aligned}
\end{equation}

Good, also consistent with direct calculation.  How about our \(\thetacap\) derivatives?  Lets just calculate these directly without bothering at all with the commutator.  This is

\begin{equation}\label{eqn:qmAngularMom:637}
\begin{aligned}
\partial_\phi \thetacap
&= \Be_1 \cos\theta \Be_12 e^{\Be_{12}\phi}  \\
&= \Be_2 \cos\theta e^{\Be_{12}\phi}  \\
&= \cos\theta \phicap
\end{aligned}
\end{equation}

and
\begin{equation}\label{eqn:qmAngularMom:657}
\begin{aligned}
\partial_\theta \thetacap
&= -\Be_1 \sin\theta e^{\Be_{12}\phi} -\Be_3 \cos\theta \\
&= -\Be_{12} \sin\theta \phicap -\Be_3 \cos\theta \\
\end{aligned}
\end{equation}

Finally, last we have the derivatives of \(\rcap\).  Those are

\begin{equation}\label{eqn:qmAngularMom:677}
\begin{aligned}
\partial_\phi \rcap
&= \Be_2 \sin\theta e^{\Be_{12}\phi} \\
&= \sin\theta \phicap
\end{aligned}
\end{equation}

and
\begin{equation}\label{eqn:qmAngularMom:697}
\begin{aligned}
\partial_\theta \rcap
&= -\Be_3 \sin\theta + \Be_1 \cos\theta e^{\Be_{12}\phi} \\
&= -\Be_3 \sin\theta + \Be_{12} \cos\theta \phicap \\
\end{aligned}
\end{equation}

Summarizing, all the derivatives we need to evaluate the square of the angular momentum operator are
\begin{equation}\label{eqn:qmAngularMom:vecBivecDerivatives3}
\begin{aligned}
\partial_\theta \phicap &= 0 \\
\partial_\phi \phicap &= -\Be_{12} \phicap \\
\partial_\theta \thetacap &= -\Be_{12} \sin\theta \phicap -\Be_3 \cos\theta  \\
\partial_\phi \thetacap &= \cos\theta \phicap \\
\partial_\theta \rcap &= -\Be_3 \sin\theta + \Be_{12} \cos\theta \phicap \\
\partial_\phi \rcap &= \sin\theta \phicap
\end{aligned}
\end{equation}

Bugger.  We actually want the derivatives of the bivectors \(\rcap\thetacap\) and \(\rcap\phicap\) so we are not ready to evaluate the squared angular momentum.  There is three choices, one is to use these results and apply the chain rule, or start over and directly take the derivatives of these bivectors, or use the commutator result (which did not actually assume vectors and we can apply it to bivectors too if we really wanted to).

An attempt to use the chain rule get messy, but it looks like the bivectors reduce nicely, making it pointless to even think about the commutator method.  Introducing some notational conveniences, first write \(i = \Be_{12}\).  We will have to be a bit careful with this since it commutes with \(\Be_3\), but anticommutes with \(\Be_1\) or \(\Be_2\) (and therefore \(\phicap\)).  As usual we also write \(I = \Be_1 \Be_2 \Be_3\) for the Euclidean pseudoscalar (which commutes with all vectors and bivectors).

\begin{equation}\label{eqn:qmAngularMom:717}
\begin{aligned}
\rcap \thetacap
&= (\Be_3 \cos\theta + i \sin\theta \phicap)(\cos\theta i \phicap - \Be_3 \sin\theta) \\
&=
\Be_3 \cos^2\theta i \phicap -i \sin^2\theta \phicap \Be_3 +(i \phicap i \phicap -\Be_3 \Be_3 ) \cos\theta \sin\theta \\
&=
i \Be_3 (\cos^2\theta + \sin^2\theta) \phicap +(-\phicap i^2 \phicap - 1) \cos\theta \sin\theta \\
\end{aligned}
\end{equation}

This gives us just

\begin{equation}\label{eqn:qmAngularMom:vecBivecDerivatives4}
\rcap \thetacap = I \phicap
\end{equation}

and calculation of the bivector partials will follow exclusively from the \(\phicap\) partials tabulated above.

Our other bivector does not reduce quite as cleanly.  We have
\begin{equation}\label{eqn:qmAngularMom:737}
\rcap \phicap
=
(\Be_3 \cos\theta + i \sin\theta \phicap) \phicap
\end{equation}

So for this one we have

\begin{equation}\label{eqn:qmAngularMom:vecBivecDerivatives5}
\rcap \phicap = \Be_3 \phicap \cos\theta + i \sin\theta
\end{equation}

Tabulating all the bivector derivatives (details omitted) we have

\begin{equation}\label{eqn:qmAngularMom:vecBivecDerivatives6}
\begin{aligned}
\partial_\theta (\rcap \thetacap) &= 0 \\
\partial_\phi (\rcap \thetacap) &= \Be_3 \phicap \\
\partial_\theta (\rcap \phicap) &= -\Be_3 \phicap \sin\theta + i \cos\theta
= i e^{I\phicap\theta} \\
\partial_\phi (\rcap \phicap) &= -I \phicap \cos\theta
\end{aligned}
\end{equation}

Okay, we should now be armed to do the squaring of the angular momentum.

\section{Squaring the angular momentum operator}

It is expected that we have the equivalence of the squared bivector form of angular momentum with the classical scalar form in terms of spherical angles \(\phi\), and \(\theta\).  Specifically, if no math errors have been made playing around with this GA representation, we should have the following identity for the scalar part of the squared angular momentum operator

\begin{equation}\label{eqn:qmAngularMom:squaring1}
-\gpgradezero{ (\Bx \wedge \spacegrad)^2 } =
\inv{\sin\theta} \PD{\theta}{} \sin\theta \PD{\theta}{} + \inv{\sin^2\theta} \frac{\partial^2}{\partial \phi^2}
\end{equation}

To finally attempt to verify this we write the angular momentum operator in polar form, using \(i = \Be_1 \Be_2\) as

\begin{equation}\label{eqn:qmAngularMom:squaring0}
\Bx \wedge \spacegrad
=
\rcap \lr{ \thetacap \partial_\theta + \phicap \inv{\sin\theta} \partial_\phi  }
\end{equation}

Expressing the unit vectors in terms of \(\phicap\) and after some rearranging we have

\begin{equation}\label{eqn:qmAngularMom:squaring2}
\Bx \wedge \spacegrad
=
I \phicap \lr{ \partial_\theta + i e^{I\phicap \theta} \inv{\sin\theta} \partial_\phi  }
\end{equation}

Using this lets now compute the partials.  First for the \(\theta\) partials we have

\begin{equation}\label{eqn:qmAngularMom:757}
\begin{aligned}
\partial_\theta (\Bx \wedge \spacegrad)
&=
I \phicap \left(
\partial_{\theta\theta}
+ i I \phicap e^{I\phicap \theta} \inv{\sin\theta} \partial_\phi
+ i e^{I\phicap \theta} \frac{\cos\theta}{\sin^2\theta} \partial_\phi
+ i e^{I\phicap \theta} \inv{\sin\theta} \partial_{\theta\phi}
\right) \\
&=
I \phicap \left(
\partial_{\theta\theta}
+ i ( I \phicap e^{I\phicap \theta} \sin\theta
+  e^{I\phicap \theta} \cos\theta ) \inv{\sin^2\theta} \partial_\phi
+ i e^{I\phicap \theta} \inv{\sin\theta} \partial_{\theta\phi}
\right) \\
&=
I \phicap \left(
\partial_{\theta\theta}
+ i e^{2 I\phicap \theta} \inv{\sin^2\theta} \partial_\phi
+ i e^{I\phicap \theta} \inv{\sin\theta} \partial_{\theta\phi}
\right) \\
\end{aligned}
\end{equation}

Premultiplying by \(I\phicap\) and taking scalar parts we have the first part of the application of \eqnref{eqn:qmAngularMom:squaring2} on itself,

\begin{equation}\label{eqn:qmAngularMom:squaring3}
\gpgradezero{ I \phicap \partial_\theta (\Bx \wedge \spacegrad) } = -\partial_{\theta\theta}
\end{equation}

For the \(\phi\) partials it looks like the simplest option is using the computed bivector \(\phi\) partials \(\partial_\phi (\rcap \thetacap) = \Be_3 \phicap\), \(\partial_\phi (\rcap \phicap) = -I \phicap \cos\theta\).  Doing so we have

\begin{equation}\label{eqn:qmAngularMom:777}
\begin{aligned}
\partial_\phi (\Bx \wedge \spacegrad)
&=
\partial_\phi \lr{ \rcap \thetacap \partial_\theta + \rcap \phicap \inv{\sin\theta} \partial_\phi  }  \\
&=
\Be_3 \phicap \partial_\theta +
+\rcap \thetacap \partial_{\phi\theta}
-I \phicap \cot\theta \partial_\phi
+ \rcap \phicap \inv{\sin\theta} \partial_{\phi\phi}
\end{aligned}
\end{equation}

So the remaining terms of the squared angular momentum operator follow by premultiplying by \(\rcap \phicap/\sin\theta\), and taking scalar parts.  This is

\begin{equation}\label{eqn:qmAngularMom:797}
\begin{aligned}
\gpgradezero{ \rcap\phicap \inv{\sin\theta} \partial_\phi (\Bx \wedge \spacegrad) }
&=
\inv{\sin\theta} \gpgradezero{
-\rcap \Be_3 \partial_\theta +
-\phicap \thetacap \partial_{\phi\theta}
-\rcap I \cot\theta \partial_\phi
}
- \inv{\sin^2\theta} \partial_{\phi\phi}
\end{aligned}
\end{equation}

The second and third terms in the scalar selection have only bivector parts, but since \(\rcap = \Be_3 \cos\theta + \Be_1 \sin\theta e^{\Be_{12}\phi}\) has component in the \(\Be_3\) direction, we have

\begin{equation}\label{eqn:qmAngularMom:squaring4}
\gpgradezero{ \rcap\phicap \inv{\sin\theta} \partial_\phi (\Bx \wedge \spacegrad) }
=
-\cot\theta \partial_\theta - \inv{\sin^2\theta} \partial_{\phi\phi}
\end{equation}

Adding results from \eqnref{eqn:qmAngularMom:squaring3}, and \eqnref{eqn:qmAngularMom:squaring4} we have

\begin{equation}\label{eqn:qmAngularMom:squaring5}
-\gpgradezero{ (\Bx \wedge \spacegrad)^2 }
=
\partial_{\theta\theta} +\cot\theta \partial_\theta + \inv{\sin^2\theta} \partial_{\phi\phi}
\end{equation}

A final verification of \eqnref{eqn:qmAngularMom:squaring1} now only requires a simple calculus expansion

\begin{equation}\label{eqn:qmAngularMom:817}
\begin{aligned}
\inv{\sin\theta} \PD{\theta}{} \sin\theta \PD{\theta}{} \psi
&=
\inv{\sin\theta} \PD{\theta}{} \sin\theta \partial_\theta \psi \\
&=
\inv{\sin\theta} (\cos\theta \partial_\theta \psi + \sin\theta \partial_{\theta\theta} \psi) \\
&=
\cot\theta \partial_\theta \psi + \partial_{\theta\theta} \psi
\end{aligned}
\end{equation}

Voila.  This exercise demonstrating that what was known to have to be true, is in fact explicitly true, is now done.  There is no new or interesting results in this in and of itself, but we get some additional confidence in the new methods being experimented with.

\section{3D Quantum Hamiltonian}

Going back to the quantum Hamiltonian we do still have the angular momentum operator as one of the distinct factors of the Laplacian.  As operators we have something akin to the projection of the gradient onto the radial direction, as well as terms that project the gradient onto the tangential plane to the sphere at the radial point

\begin{equation}\label{eqn:qmAngularMom:837}
\begin{aligned}
-\frac{\Hbar^2}{2m} \spacegrad^2 + V
&=
-\frac{\Hbar^2}{2m} \lr{ \inv{\Bx^2} (\Bx \cdot \spacegrad)^2 - \inv{\Bx^2} \gpgradezero{(\Bx \wedge \spacegrad)^2} + \inv{\Bx} \cdot \spacegrad  } + V
\end{aligned}
\end{equation}

Using the result of \eqnref{eqn:qmAngularMom:onGradeTwo5} and the radial formulation for the rest, we can write this

\begin{equation}\label{eqn:qmAngularMom:857}
\begin{aligned}
0
&= \lr{ \spacegrad^2 - \frac{2m}{\Hbar^2} (V - E)  } \psi \\
&=
\inv{r}\frac{\partial}{\partial r} r \frac{\partial \psi}{\partial r}
- \inv{r^2} \lr{ \Bx \wedge \spacegrad - 1  } (\Bx \wedge \spacegrad) \psi
- \frac{2m}{\Hbar^2} (V - E) \psi \\
\end{aligned}
\end{equation}

If \(V = V(r)\), then a radial split by separation of variables is possible.  Writing \(\psi = R(r) Y\), we get

\begin{equation}\label{eqn:qmAngularMom:threeDqm1}
\begin{aligned}
\frac{r}{R} \frac{\partial}{\partial r} r \frac{\partial R}{\partial r} - \frac{2m r^2}{\Hbar^2} (V(r) - E)
= \inv{Y} \lr{ \Bx \wedge \spacegrad - 1  } (\Bx \wedge \spacegrad) Y = \text{constant}
\end{aligned}
\end{equation}

For the constant, lets use \(c\), and split this into a pair of equations

\begin{equation}\label{eqn:qmAngularMom:threeDqm2}
r \frac{\partial}{\partial r} r \frac{\partial R}{\partial r} - \frac{2m r^2 R}{\Hbar^2} (V(r) - E) = c R
\end{equation}

\begin{equation}\label{eqn:qmAngularMom:threeDqm3}
\lr{ \Bx \wedge \spacegrad - 1  } (\Bx \wedge \spacegrad) Y = c Y
\end{equation}

In this last we can examine simultaneous eigenvalues of \(\Bx \wedge \spacegrad\), and \(\gpgradezero{(\Bx \wedge \spacegrad)^2}\).  Suppose that
\(Y_\lambda\) is an eigenfunction of \(\Bx \wedge \spacegrad\) with eigenvalue \(\lambda\).  We then have

\begin{equation}\label{eqn:qmAngularMom:877}
\begin{aligned}
\gpgradezero{(\Bx \wedge \spacegrad)^2} Y_\lambda
&= \lr{ \Bx \wedge \spacegrad - 1  } (\Bx \wedge \spacegrad) Y_\lambda \\
&= \lr{ \Bx \wedge \spacegrad - 1  } \lambda Y_\lambda  \\
&= \lambda \lr{ \lambda - 1  } Y_\lambda
\end{aligned}
\end{equation}

We see immediately that \(Y_\lambda\) is then also an eigenfunction of \(\gpgradezero{(\Bx \wedge \spacegrad)^2}\), with eigenvalue

\begin{equation}\label{eqn:qmAngularMom:threeDqm4p}
\lambda \lr{ \lambda - 1  }
\end{equation}

Bohm gives results for simultaneous eigenfunctions of \(L_x, L_y\), or \(L_z\) with \(L^2\), in which case the eigenvalues match.  He also shows that eigenfunctions of raising and lowering operators, \(L_x \pm iL_y\) are also simultaneous eigenfunctions of \(L^2\), but having \(m(m \pm 1)\) eigenvalues.  This is something slightly different since we are not considering any specific components, but we still see that eigenfunctions of the bivector angular momentum operator \(\Bx \wedge \spacegrad\) are simultaneous eigenfunctions of the scalar squared angular momentum operator \(\gpgradezero{\Bx \wedge \spacegrad}\) (Q: is that identical to the scalar operator \(L^2\)).

Moving on, the next order of business is figuring out how to solve the multivector eigenvalue problem

\begin{equation}\label{eqn:qmAngularMom:threeDqm4}
(\Bx \wedge \spacegrad) Y_\lambda = \lambda Y_\lambda
\end{equation}

\section{Angular momentum polar form, factoring out the raising and lowering operators, and simultaneous eigenvalues}

After a bit more manipulation we find that the angular momentum operator polar form representation, again using \(i = \Be_1 \Be_2\), is

\begin{equation}\label{eqn:qmAngularMom:polar1}
\Bx \wedge \spacegrad = I \phicap ( \partial_\theta + i \cot\theta \partial_\phi + \Be_{23} e^{i\phi} \partial_\phi )
\end{equation}

Observe how similar the exponential free terms within the braces are to the raising operator as given in Bohm's equation (14.40)

\begin{equation}\label{eqn:qmAngularMom:polar2}
\begin{aligned}
L_x + i L_y &= e^{i\phi} (\partial_\theta + i \cot\theta \partial_\phi ) \\
L_z &= \inv{i} \partial_\phi
\end{aligned}
\end{equation}

In fact since \(\Be_{23}e^{i\phi} = e^{-i\phi} \Be_{23}\), the match can be made even closer

\begin{equation}\label{eqn:qmAngularMom:polar3}
\Bx \wedge \spacegrad = I \phicap e^{-i\phi} ( \mathLabelBox{e^{i\phi} (\partial_\theta + i \cot\theta \partial_\phi)}{\(= L_x + i L_y\)} + \Be_{13} \mathLabelBox{\inv{i} \partial_\phi}{\(=L_z\)} )
\end{equation}

This is a surprising factorization, but noting that \(\phicap = \Be_2 e^{i\phi}\) we have

\begin{equation}\label{eqn:qmAngularMom:polar4}
\Bx \wedge \spacegrad = \Be_{31} \lr{ e^{i\phi} (\partial_\theta + i \cot\theta \partial_\phi) + \Be_{13} \inv{i} \partial_\phi  }
\end{equation}

It appears that the factoring out from the left of a unit bivector (in this case \(\Be_{31}\)) from the bivector angular momentum operator, leaves as one of the remainders the raising operator.

Similarly, noting that \(\Be_{13}\) anticommutes with \(i = \Be_{12}\), we have the right factorization

\begin{equation}\label{eqn:qmAngularMom:polar5}
\Bx \wedge \spacegrad =
\lr{ e^{-i\phi} (\partial_\theta - i \cot\theta \partial_\phi) - \Be_{13} \inv{i} \partial_\phi  }
\Be_{31}
\end{equation}

Now in the remainder, we see the polar form representation of the lowering operator \(L_x - i L_y = e^{-i\phi}(\partial_\theta - i\cot\theta \partial_\phi)\).

I was not expecting the raising and lowering operators ``to fall out'' as they did by simply expressing the complete bivector operator in polar form.  This is actually fortuitous since it shows why this peculiar combination is of interest.

If we find a zero solution to the raising or lowering operator, that is also a solution of the eigenproblem \((\partial_\phi - \lambda) \psi = 0\), then this is necessarily also an eigensolution of \(\Bx \wedge \spacegrad\).  A secondary implication is that this is then also an eigensolution of \(\gpgradezero{(\Bx \wedge \spacegrad)^2} \psi = \lambda' \psi\).  This was the starting point in Bohm's quest for the spherical harmonics, but why he started there was not clear to me.

Saying this without the words, let us look for eigenfunctions for the non-raising portion of \eqnref{eqn:qmAngularMom:polar4}.  That is

\begin{equation}\label{eqn:qmAngularMom:polar6}
\Be_{31} \Be_{13} \inv{i} \partial_\phi f = \lambda f
\end{equation}

Since \(\Be_{31} \Be_{13} = 1\) we want solutions of

\begin{equation}\label{eqn:qmAngularMom:polar7}
\partial_\phi f = i \lambda f
\end{equation}

Solutions are

\begin{equation}\label{eqn:qmAngularMom:polar12}
f = \kappa(\theta) e^{i\lambda \phi}
\end{equation}

A demand that this is a zero eigenfunction for the raising operator, means we are looking for solutions of

\begin{equation}\label{eqn:qmAngularMom:polar8}
\Be_{31} e^{i\phi} (\partial_\theta + i \cot\theta \partial_\phi) \kappa(\theta) e^{i\lambda \phi} = 0
\end{equation}

It is sufficient to find zero eigenfunctions of

\begin{equation}\label{eqn:qmAngularMom:polar9}
(\partial_\theta + i \cot\theta \partial_\phi) \kappa(\theta) e^{i\lambda \phi} = 0
\end{equation}

Evaluation of the \(\phi\) partials and rearrangement leaves us with an equation in \(\theta\) only

\begin{equation}\label{eqn:qmAngularMom:polar10}
\frac{\partial \kappa }{\partial \theta} = \lambda \cot\theta \kappa
\end{equation}

This has solutions \(\kappa = A(\phi) (\sin\theta)^\lambda\), where because of the partial derivatives in \eqnref{eqn:qmAngularMom:polar10} we are free to make the integration constant a function of \(\phi\).  Since this is the functional dependence that is a zero of the raising operator, including this at the \(\theta\) dependence of \eqnref{eqn:qmAngularMom:polar12} means that we have a simultaneous zero of the raising operator, and an eigenfunction of eigenvalue \(\lambda\) for the remainder of the angular momentum operator.

\begin{equation}\label{eqn:qmAngularMom:polar11}
f(\theta,\phi) = (\sin\theta)^\lambda e^{i\lambda \phi}
\end{equation}

This is very similar seeming to the process of adding homogeneous solutions to specific ones, since we augment the specific eigenvalued solutions for one part of the operator by ones that produce zeros for the rest.

As a check lets apply the angular momentum operator to this as a test and see if the results match our expectations.

\begin{equation}\label{eqn:qmAngularMom:897}
\begin{aligned}
(\Bx \wedge \spacegrad ) (\sin\theta)^\lambda e^{i\lambda \phi}
&=
\rcap \lr{ \thetacap \partial_\theta + \phicap \inv{\sin\theta} \partial_\phi  }  (\sin\theta)^\lambda e^{i\lambda \phi} \\
&=
\rcap \lr{ \thetacap \lambda (\sin\theta)^{\lambda-1} \cos\theta + \phicap \inv{\sin\theta} (\sin\theta)^\lambda (i\lambda) } e^{i\lambda \phi} \\
&=
\lambda \rcap \lr{ \thetacap \cos\theta + \phicap i  } e^{i\lambda \phi} (\sin\theta)^{\lambda-1}  \\
\end{aligned}
\end{equation}

From \eqnref{eqn:qmAngularMom:vecBivecDerivatives5} we have

\begin{equation}\label{eqn:qmAngularMom:917}
\begin{aligned}
\rcap \phicap i
&= \Be_3 \phicap i \cos\theta - \sin\theta \\
&= \Be_{32} i e^{i\phi} \cos\theta - \sin\theta \\
&= \Be_{13} e^{i\phi} \cos\theta - \sin\theta \\
\end{aligned}
\end{equation}

and from \eqnref{eqn:qmAngularMom:vecBivecDerivatives4} we have

\begin{equation}\label{eqn:qmAngularMom:937}
\begin{aligned}
\rcap \thetacap
&= I \phicap  \\
&= \Be_{31} e^{i\phi}
\end{aligned}
\end{equation}

Putting these together shows that \((\sin\theta)^\lambda e^{i\lambda \phi}\) is an eigenfunction of \(\Bx \wedge \spacegrad\),

\begin{equation}\label{eqn:qmAngularMom:polar13}
(\Bx \wedge \spacegrad ) (\sin\theta)^\lambda e^{i\lambda \phi} = -\lambda (\sin\theta)^\lambda e^{i\lambda \phi}
\end{equation}

This negation surprised me at first, but I do not see any errors here in the arithmetic.  Observe that this provides a verification of messy algebra that led to \eqnref{eqn:qmAngularMom:onGradeTwo5}.  That was

\begin{equation}\label{eqn:qmAngularMom:polar14}
\gpgradezero{(\Bx \wedge \spacegrad)^2} \questionEquals \lr{ \Bx \wedge \spacegrad - 1  } (\Bx \wedge \spacegrad)
\end{equation}

Using this and \eqnref{eqn:qmAngularMom:polar13} the operator effect of \(\gpgradezero{(\Bx \wedge \spacegrad)^2}\) for the eigenvalue we have is

\begin{equation}\label{eqn:qmAngularMom:957}
\begin{aligned}
\gpgradezero{(\Bx \wedge \spacegrad)^2}
(\sin\theta)^\lambda e^{i\lambda \phi}
&=
\lr{ \Bx \wedge \spacegrad - 1  } (\Bx \wedge \spacegrad)
(\sin\theta)^\lambda e^{i\lambda \phi} \\
&=
((-\lambda)^2 - (-\lambda)) (\sin\theta)^\lambda e^{i\lambda \phi} \\
\end{aligned}
\end{equation}

So the eigenvalue is \(\lambda(\lambda + 1)\), consistent with results obtained with coordinate and scalar polar form tools.

\section{Summary}

Having covered a fairly wide range in the preceding Geometric Algebra exploration of the angular momentum operator, it seems worthwhile to attempt to summarize what was learned.

The exploration started with a simple observation that the use of the spatial pseudoscalar for the imaginary of the angular momentum operator in its coordinate form

\begin{equation}\label{eqn:qmSum:goo1}
\begin{aligned}
L_1 &= -i \Hbar( x_2 \partial_3 - x_3 \partial_2 ) \\
L_2 &= -i \Hbar( x_3 \partial_1 - x_1 \partial_3 ) \\
L_3 &= -i \Hbar( x_1 \partial_2 - x_2 \partial_1 )
\end{aligned}
\end{equation}

allowed for expressing the angular momentum operator in its entirety as a bivector valued operator

\begin{equation}\label{eqn:qmSum:goo2}
\BL = - \Hbar \Bx \wedge \spacegrad
\end{equation}

The bivector representation has an intrinsic complex behavior, eliminating the requirement for an explicitly imaginary \(i\) as is the case in the coordinate representation.

It was then assumed that the Laplacian can be expressed directly in terms of \(\Bx \wedge \spacegrad\).  This is not an unreasonable thought since we can factor the gradient with components projected onto and perpendicular to a constant reference vector \(\acap\) as

\begin{equation}\label{eqn:qmSum:goo3}
\spacegrad = \acap (\acap \cdot \spacegrad) + \acap (\acap \wedge \spacegrad)
\end{equation}

and this squares to a Laplacian expressed in terms of these constant reference directions

\begin{equation}\label{eqn:qmSum:goo4}
\spacegrad^2 = (\acap \cdot \spacegrad)^2 - (\acap \cdot \spacegrad)^2
\end{equation}

a quantity that has an angular momentum like operator with respect to a constant direction.  It was then assumed that we could find an operator representation of the form

\begin{equation}\label{eqn:qmSum:goo5}
\spacegrad^2 = \inv{\Bx^2} \lr{ (\Bx \cdot \spacegrad)^2 - \gpgradezero{(\Bx \cdot \spacegrad)^2} + f(\Bx, \spacegrad)  }
\end{equation}

Where \(f(\Bx, \spacegrad)\) was to be determined, and was found by subtraction.  Thinking ahead to relativistic applications this result was obtained for the n-dimensional Laplacian and was found to be

\begin{equation}\label{eqn:qmSum:goo6}
\grad^2 = \inv{x^2} \lr{ (n-2 + x \cdot \grad) (x \cdot \grad) - \gpgradezero{(x \wedge \grad)^2}  }
\end{equation}

For the 3D case specifically this is

\begin{equation}\label{eqn:qmSum:goo7}
\spacegrad^2 = \inv{\Bx^2} \lr{ (1 + \Bx \cdot \spacegrad) (\Bx \cdot \spacegrad) - \gpgradezero{(\Bx \wedge \spacegrad)^2}  }
\end{equation}

While the scalar selection above is good for some purposes, it interferes with observations about simultaneous eigenfunctions for the angular momentum operator and the scalar part of its square as seen in the Laplacian.  With some difficulty and tedium, by subtracting the bivector and quadvector grades from the squared angular momentum operator \((x \wedge \grad)^2\) it was eventually found in \cref{eqn:qmAngularMom:onGradeTwo5} that \eqnref{eqn:qmSum:goo6} can be written as

\begin{equation}\label{eqn:qmSum:goo8}
\grad^2 =
\inv{x^2} \left(
  (n-2 + x \cdot \grad) (x \cdot \grad)
+ (n-2 - x \wedge \grad) (x \wedge \grad)
+ (x \wedge \grad) \wedge (x \wedge \grad)
\right)
\end{equation}

In the 3D case the quadvector vanishes and \eqnref{eqn:qmSum:goo7} with the scalar selection removed is reduced to

\begin{equation}\label{eqn:qmSum:goo9}
\spacegrad^2 =
\inv{\Bx^2} \lr{ (1 + \Bx \cdot \spacegrad) (\Bx \cdot \spacegrad) + (1 - \Bx \wedge \spacegrad) (\Bx \wedge \spacegrad)  }
\end{equation}

FIXME: This doesn't look right, since we have a bivector \( \inv{\Bx^2} \Bx \wedge \spacegrad \) on the RHS and everything else is a scalar.

In 3D we also have the option of using the duality relation between the cross and the wedge \(\Ba \wedge \Bb = i (\Ba \cross \Bb)\) to express the Laplacian

\begin{equation}\label{eqn:qmSum:goo10}
\spacegrad^2 =
\inv{\Bx^2} \lr{ (1 + \Bx \cdot \spacegrad) (\Bx \cdot \spacegrad) + (1 - i (\Bx \cross \spacegrad)) i(\Bx \cross \spacegrad)  }
\end{equation}

Since it is customary to express angular momentum as \(\BL = -i \Hbar (\Bx \cross \spacegrad)\), we see here that the imaginary in this context should perhaps necessarily be viewed as the spatial pseudoscalar.  It was that guess that led down this path, and we come full circle back to this considering how to factor the Laplacian in vector quantities.  Curiously this factorization is in no way specific to Quantum Theory.

A few verifications of the Laplacian in \eqnref{eqn:qmSum:goo10} were made.  First it was shown that the directional derivative terms containing \(\Bx \cdot \spacegrad\), are equivalent to the radial terms of the Laplacian in spherical polar coordinates.   That is

\begin{equation}\label{eqn:qmSum:goo11}
\inv{\Bx^2} (1 + \Bx \cdot \spacegrad) (\Bx \cdot \spacegrad) \psi = \inv{r} \frac{\partial^2}{\partial r^2} (r\psi)
\end{equation}

Employing the quaternion operator for the spherical polar rotation

\begin{equation}\label{eqn:qmSum:goo12}
\begin{aligned}
R &= e^{\Be_{31}\theta/2} e^{\Be_{12}\phi/2} \\
\Bx &= r \tilde{R} \Be_3 R
\end{aligned}
\end{equation}

it was also shown that there was explicitly no radial dependence in the angular momentum operator which takes the form

\begin{equation}\label{eqn:qmSum:goo13}
\begin{aligned}
\Bx \wedge \spacegrad
&= \tilde{R} \lr{ \Be_3 \Be_1 R \partial_\theta + \Be_3 \Be_2 R \inv{\sin\theta} \partial_\phi  } \\
&= \rcap \lr{ \thetacap \partial_\theta + \phicap \inv{\sin\theta} \partial_\phi  }
\end{aligned}
\end{equation}

Because there is a \(\theta\), and \(\phi\) dependence in the unit vectors \(\rcap\), \(\thetacap\), and \(\phicap\), squaring the angular momentum operator in this form means that the unit vectors are also operated on.  Those vectors were given by the triplet

\begin{equation}\label{eqn:qmSum:goo14}
\begin{pmatrix}
\rcap \\
\thetacap \\
\phicap \\
\end{pmatrix}
=
\tilde{R}
\begin{pmatrix}
\Be_3 \\
\Be_1 \\
\Be_2 \\
\end{pmatrix}
R
\end{equation}

Using \(I = \Be_1 \Be_2 \Be_3\) for the spatial pseudoscalar, and \(i = \Be_1 \Be_2\) (a possibly confusing switch of notation) for the bivector of the x-y plane we can write the spherical polar unit vectors in exponential form as

\begin{equation}\label{eqn:qmSum:goo15}
\begin{pmatrix}
\phicap \\
\rcap \\
\thetacap \\
\end{pmatrix}
=
\begin{pmatrix}
\Be_2 e^{i\phi} \\
\Be_3 e^{I \phicap \theta} \\
i \phicap e^{I \phicap \theta} \\
\end{pmatrix}
\end{equation}

These or related expansions were used to verify (with some difficulty) that the scalar squared bivector operator is identical to the expected scalar spherical polar coordinates parts of the Laplacian

\begin{equation}\label{eqn:qmSum:goo16}
-\gpgradezero{ (\Bx \wedge \spacegrad)^2 } =
\inv{\sin\theta} \PD{\theta}{} \sin\theta \PD{\theta}{} + \inv{\sin^2\theta} \frac{\partial^2}{\partial \phi^2}
\end{equation}

Additionally, by left or right dividing a unit bivector from the angular momentum operator, we are able to find that the raising and lowering operators are left as one of the factors

\begin{equation}\label{eqn:qmSum:goo17}
\begin{aligned}
\Bx \wedge \spacegrad &= \Be_{31} \lr{ e^{i\phi} (\partial_\theta + i \cot\theta \partial_\phi) + \Be_{13} \inv{i} \partial_\phi  } \\
\Bx \wedge \spacegrad &= \lr{ e^{-i\phi} (\partial_\theta - i \cot\theta \partial_\phi) - \Be_{13} \inv{i} \partial_\phi  } \Be_{31}
\end{aligned}
\end{equation}

Both of these use \(i = \Be_1 \Be_2\), the bivector for the plane, and not the spatial pseudoscalar.  We are then able to see that in the context of the raising and lowering operator for the radial equation the interpretation of the imaginary should be one of a plane.

Using the raising operator factorization, it was calculated that \((\sin\theta)^\lambda e^{i\lambda \phi}\) was an eigenfunction of the bivector operator \(\Bx \wedge \spacegrad\) with eigenvalue \(-\lambda\).  This results in the simultaneous eigenvalue of \(\lambda(\lambda + 1)\) for this eigenfunction with the scalar squared angular momentum operator.

There are a few things here that have not been explored to their logical conclusion.

The bivector Fourier projections \(I \Be_k (\Bx \wedge \spacegrad ) \cdot (-I \Be_k)\) do not obey the commutation relations of the scalar angular momentum components, so an attempt to directly use these to construct raising and lowering operators does not produce anything useful.  The raising and lowering operators in a form that could be used to find eigensolutions were found by factoring out \(\Be_{13}\) from the bivector operator.  Making this particular factorization was a fluke and only because it was desirable to express the bivector operator entirely in spherical polar form.  It is curious that this results in raising and lowering operators for the x,y plane, and understanding this further would be nice.

In the eigen solutions for the bivector operator, no quantization condition was imposed.  I do not understand the argument that Bohm used to do so in the traditional treatment, and revisiting this once that is done is in order.

I am also unsure exactly how Bohm knows that the inner product for the eigenfunctions should be a surface integral.  This choice works, but what drives it.  Can that be related to anything here?

%\EndArticle
