%
% Copyright � 2012 Peeter Joot.  All Rights Reserved.
% Licenced as described in the file LICENSE under the root directory of this GIT repository.
%

%
%
\mychapter{Lorentz force relation to the energy momentum tensor}
\label{chap:PJstressEnergyLorentz}
\index{Lorentz force!energy momentum tensor}
%\date{Feb 13, 2009.  stressEnergyLorentz.tex}

\section{Motivation}

Have now made a few excursions related to the concepts of electrodynamic
field energy and momentum.

In \chapcite{PJpoynting} the energy density rate and Poynting divergence
relationship was demonstrated using Maxwell's equation.  That was:
%
\begin{equation}\label{eqn:stressEnergyLorentz:20}
\begin{aligned}
\PD{t}{}\frac{\epsilon_0}{2} \left(\BE^2 + c^2 \BB^2\right) + \spacegrad \cdot \inv{\mu_0}(\BE \cross \BB) &= -\BE \cdot \Bj
\end{aligned}
\end{equation}
%
In terms of the field energy density \(U\), and Poynting vector \(\BP\), this is
\begin{equation}\label{eqn:seLorentz:energyDensityPoyntingDefined}
\begin{aligned}
U &= \frac{\epsilon_0}{2} \left(\BE^2 + c^2 \BB^2\right) \\
\BP &= \inv{\mu_0}(\BE \cross \BB) \\
\PD{t}{U} + \spacegrad \cdot \BP &= -\BE \cdot \Bj
\end{aligned}
\end{equation}
%
In \chapcite{PJemstresstensor} this was related to the
energy momentum four vectors
%
\begin{equation}\label{eqn:seLorentz:lorentzForceT}
\begin{aligned}
T(a) &= \frac{\epsilon_0}{2} F a \tilde{F}
\end{aligned}
\end{equation}
%
as defined
in \citep{doran2003gap}, but the big picture view
of things was missing.

Later in \chapcite{PJpoyntingRate} the rate of change of Poynting vector
was calculated, with an additional attempt to relate this to \(T(\gamma_\mu)\).

These relationships, and the operations required to factoring out the divergence were considerably messier.

Finally, in \chapcite{PJelectricFieldEnergy} the four vector \(T(\gamma_\mu)\)
was related to the Lorentz force and the work done moving a charge against
a field.  This provides the natural context for the energy momentum tensor,
since it appears that the spacetime divergence of each of the
\(T(\gamma_\mu)\) four vectors appears to be a component of the
four vector Lorentz force (density).

In these notes the divergences will be calculated to confirm the
connection between the Lorentz force and energy momentum tensor directly.
This is actually expected to be simpler than the previous calculations.

It is also
potentially of interest, as shown in \chapcite{PJFourierVacuum}, and
\chapcite{PJplaneWave}
that the energy density and Poynting vectors, and energy momentum four vector,
were seen to be naturally expressible as Hermitian conjugate operations
%
\begin{equation}\label{eqn:stressEnergyLorentz:40}
\begin{aligned}
F^\dagger &= \gamma_0 \tilde{F} \gamma_0
\end{aligned}
\end{equation}
\begin{equation}\label{eqn:stressEnergyLorentz:60}
\begin{aligned}
T(\gamma_0) &= \frac{\epsilon_0}{2} F F^\dagger \gamma_0
\end{aligned}
\end{equation}
\begin{equation}\label{eqn:stressEnergyLorentz:80}
\begin{aligned}
U &= T(\gamma_0) \cdot \gamma_0 = \frac{\epsilon_0}{4} \left(F F^\dagger + F^\dagger F \right) \\
\BP/c &= T(\gamma_0) \wedge \gamma_0 = \frac{\epsilon_0}{4} \left(F F^\dagger - F^\dagger F \right)
\end{aligned}
\end{equation}
%
It is conceivable that a generalization of Hermitian conjugation, where the spatial basis vectors are used instead of \(\gamma_0\), will
provide a mapping and driving structure from the Four vector quantities and the somewhat scrambled seeming set
of relationships observed in the split spatial and time domain.  That will also be explored here.

\section{Spacetime divergence of the energy momentum four vectors}

The spacetime divergence of the field energy momentum four vector \(T(\gamma_0)\) has been calculated previously.  Let us redo this
calculation for the other components.
%
\begin{equation}\label{eqn:stressEnergyLorentz:100}
\begin{aligned}
\grad \cdot T(\gamma_\mu)
&= \frac{\epsilon_0}{2} \gpgradezero{ \grad (F \gamma_\mu \tilde{F}) } \\
&= \frac{\epsilon_0}{2} \gpgradezero{ (\grad F) \gamma_\mu \tilde{F} + (\tilde{F} \grad) F \gamma_\mu } \\
&= \frac{\epsilon_0}{2} \gpgradezero{ (\grad F) \gamma_\mu \tilde{F} + \gamma_\mu \tilde{F} (\grad F) } \\
&= {\epsilon_0} \gpgradezero{ (\grad F) \gamma_\mu \tilde{F} } \\
&= \inv{c} \gpgradezero{ J \gamma_\mu \tilde{F} } \\
\end{aligned}
\end{equation}
%
The ability to perform cyclic reordering of terms in a scalar product has been used above.  Application of one more
reverse operation (which does not change a scalar), gives us
%
\begin{equation}\label{eqn:seLorentz:lorentzForceTdivergence}
\begin{aligned}
\grad \cdot T(\gamma_\mu) &= \inv{c} \gpgradezero{ F \gamma_\mu J }
\end{aligned}
\end{equation}
%
Let us expand the right hand size first.
%
\begin{equation}\label{eqn:stressEnergyLorentz:120}
\begin{aligned}
\inv{c} \gpgradezero{ F \gamma_\mu J } &= \inv{c} \gpgradezero{ (\BE + i c \BB) \gamma_\mu (c \rho \gamma_0 + \Bj \gamma_0) }
\end{aligned}
\end{equation}
%
The \(\mu = 0\) term looks the easiest, and for that one we have
%
\begin{equation}\label{eqn:stressEnergyLorentz:140}
\begin{aligned}
\inv{c} \gpgradezero{ (\BE + i c \BB) (c \rho - \Bj) }  = -\Bj \cdot \BE
\end{aligned}
\end{equation}
%
Now, for the other terms, say \(\mu = k\), we have
%
\begin{equation}\label{eqn:stressEnergyLorentz:160}
\begin{aligned}
\inv{c} \gpgradezero{ (\BE + i c \BB) (c \rho \sigma_k - \sigma_k \Bj ) }
&= E^k \rho - \gpgradezero{ i \BB \sigma_k \Bj }  \\
&= E^k \rho - J^a B^b \gpgradezero{ \sigma_1 \sigma_2 \sigma_3 \sigma_b \sigma_k \sigma_a }  \\
&= E^k \rho - J^a B^b \epsilon_{a k b} \\
&= E^k \rho + J^a B^b \epsilon_{k a b} \\
&= (\rho \BE + \Bj \cross \BB) \cdot \sigma_k
\end{aligned}
\end{equation}
%
Summarizing the two results we have
%
\begin{equation}\label{eqn:seLorentz:lorentzForcePair}
\begin{aligned}
\inv{c} \gpgradezero{ F \gamma_0 J } &= -\Bj \cdot \BE \\
\inv{c} \gpgradezero{ F \gamma_k J } &= (\rho \BE + \Bj \cross \BB) \cdot \sigma_k
\end{aligned}
\end{equation}
%
The second of these is easily recognizable as components of the Lorentz force for an element of charge (density).  The first
of these is actually the energy component of the four vector Lorentz force, so expanding that in terms of spacetime quantities
is the next order of business.

\section{Four vector Lorentz Force}

The Lorentz force in covariant form is

%\begin{align}
%m \ddot{x}_\mu &= q F_{} \cdot \frac{dx}{d\tau}
%\end{align}
%
%Or in vector/bivector form
%
\begin{equation}\label{eqn:seLorentz:lorentzForceGA}
\begin{aligned}
m \ddot{x} &= q F \cdot \dot{x}/c
\end{aligned}
\end{equation}
%
Two verifications of this are in order.  One is that we get the traditional vector form of the Lorentz force equation from
this and the other is that we can get the traditional tensor form from this equation.

\subsection{Lorentz force in tensor form}
\index{Lorentz force!tensor}

Recovering the tensor form is probably the easier of the two operations.  We have
%
\begin{equation}\label{eqn:stressEnergyLorentz:180}
\begin{aligned}
m \ddot{x}_\mu \gamma^\mu
&= \frac{q}{2} F_{\alpha\beta} \dot{x}_\sigma (\gamma^{\alpha} \wedge \gamma^\beta) \cdot \gamma^\sigma \\
&= \frac{q}{2} F_{\alpha\beta} \dot{x}^\sigma (\gamma^{\alpha} {\delta^\beta}_\sigma -\gamma^{\beta} {\delta^\alpha}_\sigma ) \\
&= \frac{q}{2} F_{\alpha\beta} \dot{x}^\beta \gamma^{\alpha} - \frac{q}{2} F_{\alpha\beta} \dot{x}^\alpha \gamma^{\beta} \\
\end{aligned}
\end{equation}
%
Dotting with \(\gamma_\mu\) the right hand side is
%
\begin{equation}\label{eqn:stressEnergyLorentz:200}
\begin{aligned}
\frac{q}{2} F_{\mu\beta} \dot{x}^\beta - \frac{q}{2} F_{\alpha\mu} \dot{x}^\alpha
&= {q} F_{\mu\alpha} \dot{x}^\alpha
\end{aligned}
\end{equation}
%
Which recovers the tensor form of the equation
\begin{equation}\label{eqn:seLorentz:lorentzForceTensor}
\begin{aligned}
m \ddot{x}_\mu &= {q} F_{\mu\alpha} \dot{x}^\alpha
\end{aligned}
\end{equation}
%
\subsection{Lorentz force components in vector form}
\index{Lorentz force!vector form}
%
\begin{equation}\label{eqn:stressEnergyLorentz:220}
\begin{aligned}
m \gamma \frac{d}{dt} \gamma \left(c + \sigma_k \frac{dx^k}{dt}\right) \gamma_0
&= \frac{q}{2c}(F v - v F) \\
&=
\frac{q \gamma}{2c}
(\BE + i c\BB)
\left(c + \sigma_k \frac{dx^k}{dt}\right) \gamma_0
\\
&\quad -
\frac{q \gamma}{2c}
\left(c + \sigma_k \frac{dx^k}{dt}\right) \gamma_0
(\BE + i c\BB) \\
\end{aligned}
\end{equation}
%
Right multiplication by \(\gamma_0/\gamma\) we have
\begin{equation}\label{eqn:stressEnergyLorentz:240}
\begin{aligned}
m \frac{d}{dt} \gamma \left(c + \Bv \right)
&= \frac{q }{2c} \left( (\BE + i c\BB) \left(c + \Bv \right) -\left(c + \Bv \right) (-\BE + i c\BB) \right) \\
&= \frac{q }{2c} \left(
%(\BE + i c\BB) (c + \Bv )
%-(c + \Bv ) (-\BE + i c\BB)
+ 2 \BE c
+ \BE \Bv  + \Bv \BE
+ i c (\BB \Bv  - \Bv \BB)
\right) \\
\end{aligned}
\end{equation}
%
After a last bit of reduction this is
\begin{equation}\label{eqn:stressEnergyLorentz:260}
\begin{aligned}
m \frac{d}{dt} \gamma \left(c + \Bv \right) &= q (\BE + \Bv \cross \BB) + q \BE \cdot \Bv/c
\end{aligned}
\end{equation}
%
In terms of four vector momentum this is
\begin{equation}\label{eqn:seLorentz:lorentzForceVec}
\begin{aligned}
\dot{p} = q ( \BE \cdot \Bv/c + \BE + \Bv \cross \BB ) \gamma_0
\end{aligned}
\end{equation}
%
\subsection{Relation to the energy momentum tensor}
\index{Lorentz force!energy momentum tensor}

It appears that to relate the energy momentum tensor to the Lorentz force we have
to work with the upper index quantities rather than the lower index stress tensor vectors.  Doing so
our four vector force per unit volume is
%
\begin{equation}\label{eqn:stressEnergyLorentz:280}
\begin{aligned}
\PD{V}{\dot{p}}
&= (\Bj \cdot \BE + \rho \BE + \Bj \cross \BB) \gamma_0 \\
%&= - \inv{c} \left(\gpgradezero{ F \gamma^0 J } \gamma_0 + \gpgradezero{ F \gamma^k J } \gamma_k \right) \\
&= - \inv{c} \gpgradezero{ F \gamma^\mu J } \gamma_\mu \\
&= - (\grad \cdot T(\gamma^\mu)) \gamma_\mu
\end{aligned}
\end{equation}
%
The term \(\gpgradezero{ F \gamma^\mu J } \gamma_\mu \) appears to be expressed simply has \(F \cdot J\) in
\citep{doran2003gap}.  Understanding that simple statement is now possible now that an exploration
of some of the underlying ideas has been made.  In retrospect having seen the bivector product form of the Lorentz force equation, it should have been
clear, but some of the associated trickiness in their treatment obscured this
fact ( Although their treatment is only two pages, I still only
understand half of what they are doing!)


\section{Expansion of the energy momentum tensor}

While all the components of the divergence of the energy momentum tensor have been expanded explicitly, this has not been
done here for the tensor itself.  A mechanical expansion of the tensor in terms of field tensor components \(F^{\mu\nu}\) has been
done previously and is not particularly enlightening.  Let us work it out here in terms of electric and magnetic field components.  In particular for the \(T^{0\mu}\) and \(T^{\mu0}\) components of the tensor in terms of energy density and the Poynting vector.

\subsection{In terms of electric and magnetic field components}

Here we want to expand
%
\begin{equation}\label{eqn:stressEnergyLorentz:300}
\begin{aligned}
T(\gamma^\mu) = \frac{-\epsilon_0}{2} (\BE + i c \BB) \gamma^\mu (\BE + i c \BB)
\end{aligned}
\end{equation}
%
It will be convenient here to temporarily work with \(\epsilon_0 = c = 1\), and put them back in afterward.
%T(\gamma^\mu) = \frac{-1}{2} (\BE + i \BB) \gamma^\mu (\BE + i \BB)

\subsubsection{First row}
%
First expanding \(T(\gamma^0)\) we have
\begin{equation}\label{eqn:stressEnergyLorentz:320}
\begin{aligned}
T(\gamma^0)
&= \frac{1}{2} (\BE + i \BB) (\BE - i \BB) \gamma^0 \\
&= \frac{1}{2} (\BE^2 + \BB^2 + i (\BB \BE - \BE \BB)) \gamma^0 \\
&= \frac{1}{2} (\BE^2 + \BB^2) \gamma^0 + i ( \BB \wedge \BE ) \gamma^0 \\
\end{aligned}
\end{equation}
%
Using the wedge product dual \(\Ba \wedge \Bb = i (\Ba \cross \Bb)\), and putting back in the units, we have our
first stress energy four vector,
%
\begin{equation}\label{eqn:stressEnergyLorentz:340}
\begin{aligned}
T(\gamma^0) &= \lr{ \frac{\epsilon_0}{2} \lr{\BE^2 + c^2 \BB^2} + \inv{\mu_0 c} (\BE \cross \BB) } \gamma^0
\end{aligned}
\end{equation}
%
In particular the energy density and the components of the Poynting vector can be picked off by dotting with each of the \(\gamma^\mu\) vectors.  That is
%
\begin{equation}\label{eqn:stressEnergyLorentz:360}
\begin{aligned}
U                    &= T(\gamma^0) \cdot \gamma^0 \\
\BP/c \cdot \sigma_k &= T(\gamma^0) \cdot \gamma^k
\end{aligned}
\end{equation}
%
\subsubsection{First column}

We have Poynting vector terms in the \(T^{0k}\) elements of the matrix.  Let us quickly verify that we have them
in the \(T^{k0}\) positions too.

To do so, again with \(c = \epsilon_0 = 1\) temporarily this is a computation of
%
\begin{equation}\label{eqn:stressEnergyLorentz:380}
\begin{aligned}
T(\gamma^k) \cdot \gamma^0
&= \inv{2} (T(\gamma^k) \gamma^0 + \gamma^0 T(\gamma^k)) \\
&= \frac{-1}{4} (F \gamma^k F \gamma^0 + \gamma^0 F \gamma^k F) \\
&= \frac{1}{4} (F \sigma_k \gamma_0 F \gamma^0 - \gamma^0 F \gamma_0 \sigma_k F) \\
&= \frac{1}{4} (F \sigma_k (-\BE + i \BB) - (-\BE + i\BB) \sigma_k F) \\
&= \frac{1}{4} \gpgradezero{\sigma_k (-\BE + i \BB)(\BE + i \BB) - \sigma_k (\BE + i \BB)(-\BE + i\BB) } \\
&= \frac{1}{4} \gpgradezero{\sigma_k (-\BE^2 -\BB^2 +2 (\BE \cross \BB)) - \sigma_k (-\BE^2 -\BB^2 - 2(\BE \cross \BB)) } \\
\end{aligned}
\end{equation}
%
Adding back in the units we have
%
\begin{equation}\label{eqn:stressEnergyLorentz:400}
\begin{aligned}
T(\gamma^k) \cdot \gamma^0 &= \epsilon_0 c (\BE \cross \BB) \cdot \sigma_k = \inv{c}\BP \cdot \sigma_k
\end{aligned}
\end{equation}
%
As expected, these are the components of the Poynting vector (scaled by \(1/c\) for units of energy density).

\subsubsection{Diagonal and remaining terms}
%
\begin{equation}\label{eqn:stressEnergyLorentz:420}
\begin{aligned}
T(\gamma^a) \cdot \gamma^b
&= \inv{2} (T(\gamma^a) \gamma^b + \gamma^b T(\gamma^a)) \\
&= \frac{-1}{4} (F \gamma^a F \gamma^b + \gamma^a F \gamma^b F) \\
&= \frac{1}{4} (F \sigma_a \gamma_0 F \gamma^b - \gamma^a F \gamma_0 \sigma_b F) \\
&= \frac{1}{4} (F \sigma_a (-\BE + i \BB) \sigma_b + \sigma_a (-\BE + i \BB) \sigma_b F) \\
&= \frac{1}{2} \gpgradezero{\sigma_a (-\BE + i \BB) \sigma_b (\BE + i\BB) } \\
\end{aligned}
\end{equation}
%
From this point is there any particularly good or clever way to do the remaining reduction?  Doing it with
coordinates looks like it would be easy, but also messy.  A decomposition of \(\BE\) and \(\BB\) that are parallel
and perpendicular to the spatial basis vectors also looks feasible.

Let us try the dumb way first
%
\begin{equation}\label{eqn:stressEnergyLorentz:440}
\begin{aligned}
T(\gamma^a) \cdot \gamma^b
&= \frac{1}{2} \gpgradezero{\sigma_a (-E^k \sigma_k + i B^k \sigma_k) \sigma_b (E^m \sigma_m + i B^m \sigma_m) } \\
&=
\inv{2} (B^k E^m - E^k B^m) \gpgradezero{ i \sigma_a \sigma_k \sigma_b \sigma_m }
- \inv{2} (E^k E^m + B^k B^m) \gpgradezero{ \sigma_a \sigma_k \sigma_b \sigma_m } \\
\end{aligned}
\end{equation}
%
Reducing the scalar operations is going to be much different for the \(a = b\), and \(a \ne b\) cases.  For the diagonal case
we have
%
\begin{equation}\label{eqn:stressEnergyLorentz:460}
\begin{aligned}
T(\gamma^a) \cdot \gamma^a
&=
\inv{2} (B^k E^m - E^k B^m) \gpgradezero{ i \sigma_a \sigma_k \sigma_a \sigma_m }
- \inv{2} (E^k E^m + B^k B^m) \gpgradezero{ \sigma_a \sigma_k \sigma_a \sigma_m } \\
&=
- \inv{2} \sum_{m, k \ne a} \inv{2} (B^k E^m - E^k B^m) \gpgradezero{ i \sigma_k \sigma_m }
+ \inv{2} \sum_{m, k \ne a} (E^k E^m + B^k B^m) \gpgradezero{ \sigma_k \sigma_m } \\
&+ \inv{2} \sum_{m} (B^a E^m - E^a B^m) \gpgradezero{ i \sigma_a \sigma_m }
- \inv{2} \sum_m (E^a E^m + B^a B^m) \gpgradezero{ \sigma_a \sigma_m } \\
\end{aligned}
\end{equation}
%
Inserting the units again we have
\begin{equation}\label{eqn:stressEnergyLorentz:480}
\begin{aligned}
T(\gamma^a) \cdot \gamma^a
&=
\frac{\epsilon_0}{2} \left( \sum_{k \ne a} \left( (E^k)^2 + c^2 (B^k)^2 \right) - \left( (E^a)^2  + c^2 (B^a)^2  \right) \right)
\end{aligned}
\end{equation}
%
Or, adding and subtracting, we have the diagonal in terms of energy density (minus a fudge)
%
\begin{equation}\label{eqn:stressEnergyLorentz:500}
\begin{aligned}
T(\gamma^a) \cdot \gamma^a &= U - \epsilon_0 \left( (E^a)^2  + c^2 (B^a)^2  \right)
\end{aligned}
\end{equation}
%
Now, for the off diagonal terms.  For \(a \ne b\) this is
\begin{equation}\label{eqn:stressEnergyLorentz:520}
\begin{aligned}
T(\gamma^a) \cdot \gamma^b
&=
\inv{2} \sum_m (B^a E^m - E^a B^m) \gpgradezero{ i \sigma_b \sigma_m }
+\inv{2} \sum_{m}(B^b E^m - E^b B^m) \gpgradezero{ i \sigma_a \sigma_m } \\
&- \inv{2} \sum_m (E^a E^m + B^a B^m) \gpgradezero{ \sigma_b \sigma_m }
- \inv{2} \sum_{m}(E^b E^m + B^b B^m) \gpgradezero{ \sigma_a \sigma_m } \\
&+\inv{2} \sum_{m, k \ne a,b}(B^k E^m - E^k B^m) \gpgradezero{ i \sigma_a \sigma_k \sigma_b \sigma_m }
- \inv{2} \sum_{m, k \ne a,b}(E^k E^m + B^k B^m) \gpgradezero{ \sigma_a \sigma_k \sigma_b \sigma_m } \\
\end{aligned}
\end{equation}
%
The first two scalar filters that include \(i\) will be zero, and we have deltas
\(\gpgradezero{ \sigma_b \sigma_m } = \delta_{bm}\) in the next two.
The remaining two terms have only vector and bivector terms, so we have zero scalar parts.
That leaves (restoring units)
%
\begin{equation}\label{eqn:stressEnergyLorentz:540}
\begin{aligned}
T(\gamma^a) \cdot \gamma^b
&= - \frac{\epsilon_0}{2} \left( E^a E^b + E^b E^a + c^2 (B^a B^b + B^b B^a) \right)
\end{aligned}
\end{equation}
%
\subsection{Summarizing}

Collecting all the results, with \(T^{\mu\nu} = T(\gamma^\mu) \cdot \gamma^\nu\), we have
%
\begin{equation}\label{eqn:stressEnergyLorentz:560}
\begin{aligned}
T^{00} &= \frac{\epsilon_0}{2} \left(\BE^2 + c^2 \BB^2\right) \\
T^{aa} &= \frac{\epsilon_0}{2} \left(\BE^2 + c^2 \BB^2\right) - \epsilon_0 \left( (E^a)^2  + c^2 (B^a)^2  \right) \\
T^{k0} = T^{0k} &= \inv{c} \left( \inv{\mu_0}(\BE \cross \BB) \right) \cdot \sigma_k \\
T^{ab} = T^{ba} &= - \frac{\epsilon_0}{2} \left( E^a E^b + E^b E^a + c^2 (B^a B^b + B^b B^a) \right)
\end{aligned}
\end{equation}
%
\subsection{Assembling a four vector}

Let us see what one of the \(T^{a\mu} \gamma_\mu\) rows of the tensor looks like in four vector form.  Let \(f \ne g \ne h\)
represent an even permutation of the integers \(1,2,3\).  Then we have
%
\begin{equation}\label{eqn:stressEnergyLorentz:580}
\begin{aligned}
T^f
&= T^{f\mu} \gamma_\mu \\
&=
\frac{\epsilon_0}{2} c (E^g B^h - E^h B^g) \gamma_0 \\
&+\frac{\epsilon_0}{2} \left( -(E^f)^2 +(E^g)^2 +(E^h)^2 + c^2 ( -(B^f)^2 +(B^g)^2 +(B^h)^2 ) \right) \gamma_f \\
&-\frac{\epsilon_0}{2} \left( E^f E^g + E^g E^f + c^2 (B^f B^g + B^g B^f) \right) \gamma_g \\
&-\frac{\epsilon_0}{2} \left( E^f E^h + E^h E^f + c^2 (B^f B^h + B^h B^f) \right) \gamma_h \\
\end{aligned}
\end{equation}
%
It is pretty amazing that the divergence of this produces the \(f\) component of the Lorentz force (density)
%
\begin{equation}\label{eqn:stressEnergyLorentz:600}
\begin{aligned}
\partial_\mu T^{f\mu} = (\rho \BE + \Bj \cross \BB) \cdot \sigma_f
\end{aligned}
\end{equation}
%
Demonstrating this directly without having STA as an available tool would be quite tedious, and looking at this
expression inspires no particular attempt to try!

%where \(U\), and \(\BP\) are as defined in \eqnref{eqn:seLorentz:energyDensityPoyntingDefined}.

% other way:
%
%To reduce this, let us write the fields in terms of projections and rejections onto the \(\sigma_k\) direction, as in \(\BE_\parallel = (\BE \cdot \sigma_k) \sigma_k\), and \(\BE_\perp = (\BE \wedge \sigma_k) \sigma_k\).  Then we have
%

\section{Conjugation?}
% from planewave.ltx
\subsection{Followup: energy momentum tensor}

This also suggests a relativistic generalization of conjugation, since the time basis vector should perhaps not have
a distinguishing role.  Something like this:
%
\begin{equation}\label{eqn:stressEnergyLorentz:620}
\begin{aligned}
F^{\dagger_\mu} &= \gamma_\mu \tilde{F} \gamma_\mu
\end{aligned}
\end{equation}
%
Or perhaps:
\begin{equation}\label{eqn:stressEnergyLorentz:640}
\begin{aligned}
F^{\dagger_\mu} &= \gamma_\mu \tilde{F} \gamma^\mu
\end{aligned}
\end{equation}
%
may make sense for consideration of the other components of the general energy momentum tensor, which had roughly the form:
%
\begin{equation}\label{eqn:stressEnergyLorentz:660}
\begin{aligned}
T^{\mu\nu} \propto T(\gamma_\mu) \cdot \gamma^\nu
\end{aligned}
\end{equation}
%
(with some probable adjustments to index positions).  Think this through later.
