%
% Copyright � 2012 Peeter Joot.  All Rights Reserved.
% Licenced as described in the file LICENSE under the root directory of this GIT repository.
%

%
%
\mychapter{Rayleigh-Jeans Law Notes}
\label{chap:PJrayleighJeans}
\index{Rayleigh-Jeans law}
%\date{Dec 27, 2008.  rayleighJeans.tex}

\section{Motivation}

Fill in the gaps for a reading of the
initial parts of the Rayleigh-Jeans discussion of \citep{bohm1989qt}.

\section{2. Electromagnetic energy}

Energy of the field given to be:
%
\begin{equation}\label{eqn:rayleighJeans:20}
\begin{aligned}
E = \inv{8\pi} \int (\bcE^2 + \bcH^2)
\end{aligned}
\end{equation}
%
I still do not really know where this comes from.
Could perhaps justify this with a Hamiltonian of a field (although this is
uncomfortably abstract).

With the particle Hamiltonian we have
%
\begin{equation}\label{eqn:rayleighJeans:40}
\begin{aligned}
H = \qdot_i p_i -\LL
\end{aligned}
\end{equation}
%
What is the field equivalent of this?  Try to get the feel for this with some simple fields (such as the one dimensional vibrating string), and the Coulomb field.  For the physical case, do this with both the Hamiltonian approach and a physical limiting argument.

\section{3. Electromagnetic Potentials}

Bohm writes Maxwell's equations in non-SI units, and also, naturally, not in STA form which would be somewhat more natural for a gauge
discussion.
%
\begin{equation}\label{eqn:rayleighJeans:60}
\begin{aligned}
\spacegrad \cross \bcE &= -\inv{c} \partial_t \bcH \\
\spacegrad \cdot \bcE &= 4 \pi \rho \\
\spacegrad \cross \bcH &= \inv{c} \partial_t \bcE + 4 \pi \Bj \\
\spacegrad \cdot \bcH &= 0
\end{aligned}
\end{equation}
%
In STA form this is
%
\begin{equation}\label{eqn:rayleighJeans:80}
\begin{aligned}
\spacegrad \bcE &= - \partial_0 i\bcH + 4 \pi \rho \\
\spacegrad i\bcH &= -\partial_0 \bcE - 4 \pi \Bj \\
\end{aligned}
\end{equation}
%
Or
\begin{equation}\label{eqn:rayleigh_jeans:maxwellNotQuiteCovariant}
\begin{aligned}
\spacegrad (\bcE + i\bcH) + \partial_0 (\bcE + i\bcH) &= 4 \pi (\rho - \Bj)
\end{aligned}
\end{equation}
%
Left multiplying by \(\gamma_0\) gives
%
\begin{equation}\label{eqn:rayleighJeans:100}
\begin{aligned}
\gamma_0 \spacegrad
&= \gamma_0 \sum_k \sigma_k \partial_k \\
&= \gamma_0 \sum_k \gamma_k \gamma_0 \partial_k \\
&= -\sum_k \gamma_k \partial_k \\
&= \gamma^k \partial_k \\
\end{aligned}
\end{equation}
%
and
%
\begin{equation}\label{eqn:rayleighJeans:120}
\begin{aligned}
\gamma_0 \Bj
&= \sum_k \gamma_0 \sigma_k j^k \\
&= -\sum_k \gamma_k j^k,
\end{aligned}
\end{equation}
%
so with \(J^0 = \rho\), \(J^k = j^k\) and \(J = \gamma_\mu J^\mu\), we have
%
\begin{equation}\label{eqn:rayleighJeans:140}
\begin{aligned}
\gamma^\mu \partial_\mu (\bcE + i\bcH) &= 4 \pi J
\end{aligned}
\end{equation}
%
and finally with \(F = \bcE + i\bcH\), we have Maxwell's equation in covariant form
%
\begin{equation}\label{eqn:rayleighJeans:160}
\begin{aligned}
\grad F &= 4 \pi J.
\end{aligned}
\end{equation}
%
Next it is stated that general solutions can be expressed as
%
\begin{equation}\label{eqn:rayleigh_jeans:fieldsFromPotentials}
\begin{aligned}
\bcH &= \spacegrad \cross \Ba \\
\bcE &= - \inv{c} \PD{t}{\Ba} - \spacegrad \phi
\end{aligned}
\end{equation}
%
Let us double check that this jives with the bivector potential solution \(F = \grad \wedge A = \bcE + i\bcH\).  Let us split our bivector
into spacetime and spatial components by the conjugate operation
%
\begin{equation}\label{eqn:rayleighJeans:180}
\begin{aligned}
F^\conj &=\gamma_0 F \gamma_0 \\
&= \gamma_0 \gamma^\mu \wedge \gamma^\nu \partial_\mu A_\mu \gamma_0 \\
&=
\left\{
\begin{array}{l l}
0 & \quad \mbox{if \(\mu = \nu\)} \\
\gamma^\mu \gamma^\nu \partial_\mu A_\nu & \quad \mbox{if \(\mu \in \{1,2,3\}\), and \(\nu \in \{1,2,3\}\)} \\
-\gamma^\mu \gamma^\nu \partial_\mu A_\nu & \quad \mbox{one of \(\mu = 0\) or \(\nu = 0\) } \\
\end{array} \right.
\end{aligned}
\end{equation}
%
\begin{equation}\label{eqn:rayleighJeans:200}
\begin{aligned}
F
&= \bcE + i\bcH \\
&= \inv{2}(F - F^\conj) + \inv{2}(F + F^\conj) \\
&= \left(\gamma^k \wedge \gamma^0 \partial_k A_0 +\gamma^0 \wedge \gamma^k \partial_0 A_k\right) + \left(\gamma^a \wedge \gamma^b \partial_a A_b\right) \\
&= -\left(\sum_k \sigma_k \partial_k A^0 + \partial_0 \sigma_k A^k\right) + i\left(\epsilon_{abc}\sigma_a \partial_b A^c\right) \\
\end{aligned}
\end{equation}
%
So, with \(\Ba = \sigma_k A^k\), and \(\phi = A^0\), we do have equations \eqnref{eqn:rayleigh_jeans:fieldsFromPotentials} as identical to \(F = \grad \wedge A\).

Now how about the gauge variations of the fields?  Bohm writes that we can alter the potentials by
%
\begin{equation}\label{eqn:rayleigh_jeans:gauge}
\begin{aligned}
\Ba' &= \Ba - \spacegrad \psi \\
\phi' &= \phi + \inv{c}\PD{t}{\psi}
\end{aligned}
\end{equation}
%
How does this translate to an alteration of the four potential?  For the vector potential we have
%
\begin{equation}\label{eqn:rayleighJeans:220}
\begin{aligned}
\sigma_k {A^k}' &= \sigma_k A^k - \sigma_k \partial \psi \\
\gamma_k \gamma_0 {A^k}' &= \gamma_k \gamma_0 A^k - \gamma_k \gamma_0 \partial_k \psi \\
-\gamma_0 \gamma_k {A^k}' &= -\gamma_0 \gamma_k A^k - \gamma_0 \gamma^k \partial_k \psi \\
\gamma_k {A^k}' &= \gamma_k A^k + \gamma^k \partial_k \psi \\
\end{aligned}
\end{equation}
%
with \(\phi = A^0\), add in the \(\phi\) term
%
\begin{equation}\label{eqn:rayleighJeans:240}
\begin{aligned}
\gamma_0 \phi' &= \gamma_0 \phi + \gamma_0 \PD{x^0}{\psi} \\
\gamma_0 \phi' &= \gamma_0 \phi + \gamma^0 \PD{x^0}{\psi}
\end{aligned}
\end{equation}
%
For
\begin{equation}\label{eqn:rayleighJeans:260}
\begin{aligned}
\gamma_\mu {A^\mu}' &= \gamma_\mu A^\mu + \gamma^\mu \partial_\mu \psi \\
\end{aligned}
\end{equation}
%
Which is just a statement that we can add a spacetime gradient to our vector potential without altering the field equation:
%
\begin{equation}\label{eqn:rayleighJeans:280}
\begin{aligned}
A' &= A + \grad \psi
\end{aligned}
\end{equation}
%
Let us verify that this does in fact not alter Maxwell's equation.
%
\begin{equation}\label{eqn:rayleighJeans:300}
\begin{aligned}
\grad (\grad \wedge (A + \grad \psi) &= 4 \pi J
\grad (\grad \wedge A) + \grad (\grad \wedge \grad \psi) &=
\end{aligned}
\end{equation}
%
Since \(\grad \wedge \grad = 0\) we have
%
\begin{equation}\label{eqn:rayleighJeans:320}
\begin{aligned}
\grad (\grad \wedge A') = \grad (\grad \wedge A)
\end{aligned}
\end{equation}
%
Now the statement that \(\grad \wedge \grad\) as an operator equals zero, just by virtue of \(\grad\) being a vector is worth explicit
confirmation.  Let us expand that to verify
%
\begin{equation}\label{eqn:rayleighJeans:340}
\begin{aligned}
\grad \wedge \grad \psi
&= \gamma^\mu \wedge \gamma^\nu \partial_\mu \partial_\nu \psi \\
&= \left(\sum_{\mu < \nu} + \sum_{\nu < \mu}\right) \gamma^\mu \wedge \gamma^\nu \partial_\mu \partial_\nu \psi \\
&= \sum_{\mu < \nu} \gamma^\mu \wedge \gamma^\nu (\partial_\mu \partial_\nu \psi - \partial_\nu \partial_\mu \psi) \\
\end{aligned}
\end{equation}
%
So, we see that we additionally need the field variable \(\psi\) to be sufficiently continuous for mixed partial equality for the
statement that \(\grad \wedge \grad = 0\) to be valid.  Assuming that continuity is taken as a given the confirmation of the invariance under this transformation is thus complete.

Now, Bohm says it is possible to pick \(\spacegrad \cdot \Ba' = 0\).  From \eqnref{eqn:rayleigh_jeans:gauge} that implies
%
\begin{equation}\label{eqn:rayleighJeans:360}
\begin{aligned}
\spacegrad \cdot \Ba'
&= \spacegrad \cdot \Ba - \spacegrad \cdot \spacegrad \psi \\
&= \spacegrad \cdot \Ba - \spacegrad^2 \psi = 0 \\
\end{aligned}
\end{equation}
%
So, provided we can find a solution to the Poisson equation
%
\begin{equation}\label{eqn:rayleigh_jeans:psiForDivAPrimeEqZero}
\begin{aligned}
\spacegrad^2 \psi = \spacegrad \cdot \Ba
\end{aligned}
\end{equation}
%
one can find a \(\psi, \Ba\) gauge transformation that has the particular quality that \(\spacegrad \cdot \Ba' = 0\).

That solution, from \eqnref{eqn:rayleigh_jeans:laplacianOfPoisson} is
%
\begin{equation}\label{eqn:rayleighJeans:380}
\begin{aligned}
\psi(\Br) = -\inv{4\pi}\int (\spacegrad' \cdot \Ba(\Br')) dV' \inv{\Abs{\Br-\Br'}}
\end{aligned}
\end{equation}
%
The corollary to this
is that one may similarly impose a requirement that \(\spacegrad \cdot \Ba = 0\), since if that is not the case, some \(\Ba'\) can be added to the vector potential to make that the case.

FIXME: handwaving description here.  Show with a math statement with \(\Ba \rightarrow \Ba'\).

\subsection{Free space solutions}

From \eqnref{eqn:rayleigh_jeans:maxwellNotQuiteCovariant} and \eqnref{eqn:rayleigh_jeans:fieldsFromPotentials}
the free space solution to Maxwell's
equation must satisfy
%
\begin{equation}\label{eqn:rayleighJeans:400}
\begin{aligned}
0
&= \left(\spacegrad + \partial_0\right) (\bcE + i\bcH) \\
&= \left(\spacegrad + \partial_0\right) \left(- \partial_0{\Ba} - \spacegrad \phi + \spacegrad \wedge \Ba \right) \\
&= - \spacegrad \partial_0{\Ba} - \spacegrad^2 \phi + \spacegrad (\spacegrad \wedge \Ba)
 - \partial_{00}{\Ba} - \partial_0 \spacegrad \phi + \partial_0 (\spacegrad \wedge \Ba) \\
&= - \spacegrad \cdot \partial_0{\Ba} - \spacegrad^2 \phi + \spacegrad \cdot (\spacegrad \wedge \Ba)
 - \partial_{00}{\Ba} - \partial_0 \spacegrad \phi  \\
\end{aligned}
\end{equation}
%
Since the scalar and vector parts of this equation must separately equal zero we have
%
\begin{equation}\label{eqn:rayleighJeans:420}
\begin{aligned}
0 &= - \partial_0 \spacegrad \cdot {\Ba} - \spacegrad^2 \phi \\
0 &= \spacegrad \cdot (\spacegrad \wedge \Ba) - \partial_{00}{\Ba} - \partial_0 \spacegrad \phi  \\
\end{aligned}
\end{equation}
%
If one picks a gauge transformation such that \(\spacegrad \cdot \Ba = 0\) we then have
\begin{equation}\label{eqn:rayleighJeans:440}
\begin{aligned}
0 &= \spacegrad^2 \phi \\
0 &= \spacegrad^2 \Ba - \partial_{00}{\Ba} - \partial_0 \spacegrad \phi  \\
\end{aligned}
\end{equation}
%
For the first Bohm argues that ``It is well known that the only solution of this equation that is regular over all space is \(\phi = 0\)'', and anything else implies charge in the region.  What does regular mean here?  I suppose this seems like a reasonable enough statement, but I think the proper way to think about this is really that one has picked the covariant gauge \(\grad \cdot A = 0\) (that is simpler anyhow).  With an acceptance of the \(\phi =0\) argument one is left with the vector potential wave equation which was the desired goal of that section.

Note: The following 
\href{https://www.physicsforums.com/threads/electrodynamic-vector-potential-wave-equations-in-free-space.281874/}{physicsforums thread} discusses some of the confusion I had in this bit of text.

\subsection{Doing this all directly}

Now, the whole point of the gauge transformation appears to be to show that one can find the four wave equation solutions for
Maxwell's equation by picking a specific gauge.  This is actually trivial to do from the STA Maxwell equation:
%
\begin{equation}\label{eqn:rayleighJeans:460}
\begin{aligned}
\grad (\grad \wedge A) = \grad( \grad A - \grad \cdot A ) = \grad^2 A - \grad (\grad \cdot A) = 4 \pi J
\end{aligned}
\end{equation}
%
So, if one picks a gauge transformation with \(\grad \cdot A = 0\), one has
%
\begin{equation}\label{eqn:rayleighJeans:480}
\begin{aligned}
\grad^2 A = 4 \pi J
\end{aligned}
\end{equation}
%
This is precisely the four wave equations desired
\begin{equation}\label{eqn:rayleighJeans:500}
\begin{aligned}
\partial_\nu\partial^\nu A^\mu = 4 \pi J^\mu
\end{aligned}
\end{equation}
%
FIXME: show the precise gauge transformation \(A \rightarrow A'\) that leads to \(\grad \cdot A = 0\).

\section{Energy density.  Get the units right with these CGS equations}

We will want to calculate the equivalent of
%
\begin{equation}\label{eqn:rayleighJeans:520}
\begin{aligned}
U = \frac{\epsilon_0}{2} (\BE^2 + c^2 \BB^2)
\end{aligned}
\end{equation}
%
but are faced with the alternate units of Bohm's text.  Let us repeat the
derivation of the electric field energy from \chapcite{PJelectricFieldEnergy}
in the CGS units directly from Maxwell's equation
%
\begin{equation}\label{eqn:rayleigh_jeans:maxwell}
\begin{aligned}
F &= \bcE + i\bcH \\
J &= (\rho + \Bj) \gamma_0 \\
\grad F &= 4 \pi J
\end{aligned}
\end{equation}
%
to ensure we get it right.

To start with we our spacetime split of \eqnref{eqn:rayleigh_jeans:maxwell} is
%
\begin{equation}\label{eqn:rayleighJeans:540}
\begin{aligned}
( \partial_0 + \spacegrad ) (\bcE + \bcH) = 4 \pi (\rho - \Bj)
\end{aligned}
\end{equation}
%
The scalar part gives us Coulomb's law
%
\begin{equation}\label{eqn:rayleighJeans:560}
\begin{aligned}
\spacegrad \cdot \bcE = 4 \pi \rho
\end{aligned}
\end{equation}
%
Gauss's theorem applied to a spherical constant density charge distribution
gives us
\begin{equation}\label{eqn:rayleighJeans:580}
\begin{aligned}
\int \spacegrad \cdot \bcE dV &= 4 \pi \int \rho dV \\
\implies \\
\int {\bcE} \cdot \ncap dA &= 4 \pi Q \\
\implies \\
\Abs{\bcE} 4 \pi r^2 &= 4 \pi Q \\
\end{aligned}
\end{equation}
%
so we have the expected ``unitless'' Coulomb law force equation
%
\begin{equation}\label{eqn:rayleighJeans:600}
\begin{aligned}
{\BF} = q\bcE = \frac{q Q }{r^2} \rcap
\end{aligned}
\end{equation}
%
So far so good.  Next introduction of a potential.  For statics we do not care
about the four vectors and stick with the old fashion definition of the potential \(\phi\) indirectly in terms of \(\bcE\).  That is
%
\begin{equation}\label{eqn:rayleighJeans:620}
\begin{aligned}
\bcE = -\spacegrad \phi
\end{aligned}
\end{equation}
%
A line integral of this gives us \(\phi\) in terms of \(\bcE\)
\begin{equation}\label{eqn:rayleighJeans:640}
\begin{aligned}
-\int \bcE \cdot \Br
&= \int \spacegrad \phi \cdot d\Br \\
&= \phi - \phi_0 \\
\end{aligned}
\end{equation}
%
With \(\phi(\infty) = 0\) this is
%
\begin{equation}\label{eqn:rayleighJeans:660}
\begin{aligned}
\phi(d)
&= -\int_{r=\infty}^d \bcE \cdot d\Br  \\
&= -\int_{r=\infty}^d \frac{Q}{r^2} \rcap \cdot d\Br  \\
&= -\int_{r=\infty}^d \frac{Q}{r^2} dr  \\
&= \frac{Q}{d} \\
%&= -\int \frac{\rho dV}{r^2} dr
\end{aligned}
\end{equation}
%
Okay.  Now onto the electrostatic energy.  The work done to move one charge from infinite to some separation \(d\) of another like sign charge
is
%
\begin{equation}\label{eqn:rayleighJeans:680}
\begin{aligned}
\int_{r=\infty}^{d} F \cdot d\Br
&= \int_{r= \infty}^d \frac{q Q}{r^2} \rcap \cdot (-d\Br)  \\
&= -\int_{r= \infty}^d \frac{qQ}{r^2} dr  \\
&= \frac{qQ}{d} \\
&= q_1 \phi_2(d) \\
\end{aligned}
\end{equation}
%
%
For a distribution of discrete charges we have to sum over all pairs
%
\begin{equation}\label{eqn:rayleighJeans:700}
\begin{aligned}
W
&= \sum_{i \ne j} \frac{q_i q_j}{d_{ij}} \\
&= \sum_{i,j} \inv{2} \frac{q_i q_j}{d_{ij}} \\
\end{aligned}
\end{equation}
%
In a similar fashion we can do a continuous variation, employing a double summation over all space.  Note first
that we can also write
one of the charge densities in terms of the potential
%
\begin{equation}\label{eqn:rayleighJeans:720}
\begin{aligned}
\bcE &= - \spacegrad \phi \\
\implies \\
\spacegrad \cdot \bcE
&= - \spacegrad \cdot \spacegrad \phi \\
&= - \spacegrad^2 \phi \\
&= 4 \pi \rho
\end{aligned}
\end{equation}
%
\begin{equation}\label{eqn:rayleighJeans:740}
\begin{aligned}
W
&= \inv{2} \int \rho \phi(r) dV \\
&= -\inv{8\pi} \int \phi \spacegrad^2 \phi dV \\
&= \inv{8\pi} \int ( (\spacegrad \phi)^2 - \spacegrad \cdot (\phi \spacegrad \phi)) dV \\
&= \inv{8\pi} \int (-\bcE)^2 - \inv{8\pi} \int (\phi \spacegrad \phi) \cdot \ncap dA
\end{aligned}
\end{equation}
%
Here the one and two subscripts could be dropped with a switch to the total charge density and the potential from this complete charge superposition.
For our final result we have an energy density of
%
\begin{equation}\label{eqn:rayleighJeans:760}
\begin{aligned}
\frac{dW}{dV} &= \inv{8\pi} {\bcE}^2
\end{aligned}
\end{equation}
%
\section{Auxiliary details}

\subsection{Confirm Poisson solution to Laplacian}
\index{Laplacian!Poisson solution}

Bohm lists the solution for \eqnref{eqn:rayleigh_jeans:psiForDivAPrimeEqZero} (a Poisson integral), but I forget how one shows this.  I can not figure out how to integrate this Laplacian, but it is simple enough to confirm this by back substitution.

Suppose one has
%
\begin{equation}\label{eqn:rayleighJeans:780}
\begin{aligned}
\psi = \int \frac{\rho(\Br')}{\Abs{\Br - \Br'}} dV'
\end{aligned}
\end{equation}
%
We can take the Laplacian by direct differentiation under the integration sign
%
\begin{equation}\label{eqn:rayleighJeans:800}
\begin{aligned}
\spacegrad^2 \psi = \int {\rho(\Br')} dV' \spacegrad^2 \inv{\Abs{\Br - \Br'}}
\end{aligned}
\end{equation}
%
To evaluate the Laplacian we need
%
\begin{equation}\label{eqn:rayleighJeans:820}
\begin{aligned}
\PD{x_i}{\Abs{\Br - \Br'}^k}
&= \PD{x_i}{} \left(\sum_j (x_j - x_j')^2 \right)^{k/2} \\
&= k { 2\Abs{\Br - \Br'}^{k-2}} \PD{x_i}{} \left(\sum_j (x_j - x_j')^2 \right) \\
%&= k { 2\Abs{\Br - \Br'}^{k-2}} 2 (x_i - x_i')
&= k { \Abs{\Br - \Br'}^{k-2}} (x_i - x_i')
\end{aligned}
\end{equation}
%
So we have
\begin{equation}\label{eqn:rayleighJeans:840}
\begin{aligned}
\PD{x_i}{} \PD{x_i}{} {\Abs{\Br - \Br'}^{-1}}
&=
- (x_i - x_i') \PD{x_i}{}{ \inv{ \Abs{\Br - \Br'}^3} }
- \inv{ \Abs{\Br - \Br'}^3} \PD{x_i}{ (x_i - x_i') } \\
&=
3 (x_i - x_i')^2 { \inv{ \Abs{\Br - \Br'}^5} }
- \inv{ \Abs{\Br - \Br'}^3} \\
\end{aligned}
\end{equation}
%
So, provided \(\Br \ne \Br'\) we have
%
\begin{equation}\label{eqn:rayleighJeans:860}
\begin{aligned}
\spacegrad^2 \psi &=
3 (\Br - \Br')^2 \inv{ \Abs{\Br - \Br'}^5}
- 3 \inv{ \Abs{\Br - \Br'}^3} \\
&= 0
\end{aligned}
\end{equation}
%
Observe that this is true only for \R{3}.  Now, one is left with only an integral around a neighborhood around the point \(\Br\) which can be made small enough that \(\rho(\Br') = \rho(\Br)\) in that volume can be taken as constant.
%
\begin{equation}\label{eqn:rayleighJeans:880}
\begin{aligned}
\spacegrad^2 \psi
&= \rho(\Br) \int dV' \spacegrad^2 \inv{\Abs{\Br - \Br'}} \\
&= \rho(\Br) \int dV' \spacegrad \cdot \spacegrad \inv{\Abs{\Br - \Br'}} \\
&= -\rho(\Br) \int dV' \spacegrad \cdot \frac{(\Br -\Br')}{\Abs{\Br - \Br'}^3} \\
\end{aligned}
\end{equation}
%
Now, if the divergence in this integral was with respect to the primed variable that ranges over the infinitesimal volume, then this could be converted to a surface integral.
Observe that a radial expansion of this divergence allows for convenient change of variables to the primed \(x_i'\) coordinates
%
\begin{equation}\label{eqn:rayleighJeans:900}
\begin{aligned}
\spacegrad \cdot \frac{(\Br -\Br')}{\Abs{\Br - \Br'}^3}
&=
\left(\frac{\Br - \Br'}{\Abs{\Br-\Br'}} \PD{\Abs{\Br-\Br'}}{}\right) \cdot
\left(\frac{\Br - \Br'}{\Abs{\Br-\Br'}} \inv{\Abs{\Br-\Br'}^2}\right) \\
&=
\PD{\Abs{\Br'-\Br}}{} {\Abs{\Br'-\Br}^{-2}} \\
&=
\left(\frac{\Br' - \Br}{\Abs{\Br'-\Br}} \PD{\Abs{\Br'-\Br}}{}\right) \cdot
\left(\frac{\Br' - \Br}{\Abs{\Br'-\Br}} \inv{\Abs{\Br'-\Br}^2}\right) \\
&= \spacegrad' \cdot \frac{(\Br'-\Br)}{\Abs{\Br' - \Br}^3}
\end{aligned}
\end{equation}
%
Now, since \(\Br'-\Br\) is in the direction of the outwards normal the divergence theorem can be used
%
\begin{equation}\label{eqn:rayleighJeans:920}
\begin{aligned}
\spacegrad^2 \psi
&= -\rho(\Br) \int dV' \spacegrad' \cdot \frac{(\Br' -\Br)}{\Abs{\Br' - \Br}^3} \\
&= -\rho(\Br) \int_{\partial V'} dA' \inv{\Abs{\Br' - \Br}^2} \\
\end{aligned}
\end{equation}
%
Picking a spherical integration volume, for which the radius is constant \(R = \Abs{\Br'-\Br}\), we have
%
\begin{equation}\label{eqn:rayleighJeans:940}
\begin{aligned}
\spacegrad^2 \psi
&= -\rho(\Br) 4 \pi R^2 \inv{R^2} \\
\end{aligned}
\end{equation}
%
In summary this is
%
\begin{equation}\label{eqn:rayleigh_jeans:laplacianOfPoisson}
\begin{aligned}
\psi &= \int \frac{\rho(\Br')}{\Abs{\Br - \Br'}} dV' \\
\spacegrad^2 \psi &= - 4 \pi \rho(\Br)
\end{aligned}
\end{equation}
%
Having written this out I recall that the same approach was used in
\citep{schwartz1987pe} (there it was to calculate \(\spacegrad \cdot \BE\) in terms of the charge density, but the ideas are all the same.)
