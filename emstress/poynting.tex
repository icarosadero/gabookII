%
% Copyright � 2012 Peeter Joot.  All Rights Reserved.
% Licenced as described in the file LICENSE under the root directory of this GIT repository.
%

%
%
\mychapter{Poynting vector and Electromagnetic Energy conservation}
\label{chap:PJpoynting}
\index{Poynting vector}
\index{energy conservation!electromagnetic}
%\date{Dec 29, 2008.  poynting.tex}

\section{Motivation}

Clarify Poynting discussion from \citep{doran2003gap}.

Equation 7.59 and 7.60 derives a \(\BE \cross \BB\) quantity, the Poynting vector, as a sort of energy flux through the surface of the containing volume.

There are a couple of magic steps here that were not at all obvious to me.  Go through this in enough detail that it makes sense to me.

\section{Charge free case}

In SI units the Energy density is given as

\begin{equation}\label{eqn:poynting:energy}
\begin{aligned}
U = \frac{\epsilon_0}{2}\left( \BE^2 + c^2 \BB^2 \right)
\end{aligned}
\end{equation}

In \chapcite{PJelectricFieldEnergy} the electrostatic energy portion of this
energy was observed.
FIXME: A magnetostatics derivation (ie: unchanging currents)
is possible for the \(\BB^2\) term, but I have not done this myself yet.

It is somewhat curious that the total field energy is just this
sum without any cross terms (all those cross terms show up in the
field momentum).  A logical confirmation of this in a general
non-electrostatics and non-magnetostatics context will not be done here.
Instead it will be assumed that \eqnref{eqn:poynting:energy} has been correctly identified
as the field energy (density), and a mechanical calculation of the time
rate of change of this
quantity (the power density) will be performed.  In doing so we can find the
analogue of the momentum.  How to truly identify this quantity with momentum
will hopefully become clear as we work with it.

Given this energy density the rate of change of energy in a volume is then

\begin{equation}\label{eqn:poynting:20}
\begin{aligned}
\frac{dU}{dt}
&=
\frac{d}{dt}
\frac{\epsilon_0}{2} \int dV \left( \BE^2 + c^2 \BB^2 \right) \\
&=
\epsilon_0 \int dV \left( \BE \cdot \PD{t}{\BE} + c^2 \BB \cdot \PD{t}{\BB} \right) \\
\end{aligned}
\end{equation}

The next (omitted in the text) step is to utilize Maxwell's equation to eliminate the time derivatives.  Since this is the
charge and current free case, we can write Maxwell's as

\begin{equation}\label{eqn:poynting:40}
\begin{aligned}
0
&= \gamma_0 \grad F \\
&= \gamma_0 (\gamma^0 \partial_0 + \gamma^k \partial_k) F \\
&= (\partial_0 + \gamma_k\gamma_0 \partial_k) F \\
&= (\partial_0 + \sigma_k \partial_k) F \\
&= (\partial_0 + \spacegrad)F \\
&= (\partial_0 + \spacegrad)(\BE + ic \BB) \\
&= \partial_0 \BE + ic \partial_0 \BB + \spacegrad \BE + ic \spacegrad \BB \\
\end{aligned}
\end{equation}

In the spatial (\(\sigma\)) basis we can separate this into even and odd grades, which are separately equal to zero

\begin{equation}\label{eqn:poynting:60}
\begin{aligned}
0 &= \partial_0 \BE + ic \spacegrad \BB \\
%   1                  3,1
0 &= ic \partial_0 \BB + \spacegrad \BE
%  2                    0,2
\end{aligned}
\end{equation}

A selection of just the vector parts is

\begin{equation}\label{eqn:poynting:80}
\begin{aligned}
\partial_t \BE &= - ic^2 \spacegrad \wedge \BB \\
\partial_t \BB &= i\spacegrad \wedge \BE
\end{aligned}
\end{equation}

Which can be back substituted into the energy flux
\begin{equation}\label{eqn:poynting:100}
\begin{aligned}
\frac{dU}{dt}
&= \epsilon_0 \int dV \left( \BE \cdot (-i c^2 \spacegrad \wedge \BB) + c^2 \BB \cdot (i \spacegrad \wedge \BE) \right) \\
&= \epsilon_0 c^2 \int dV \gpgradezero{ \BB i \spacegrad \wedge \BE -\BE i \spacegrad \wedge \BB } \\
\end{aligned}
\end{equation}

Since the two divergence terms are zero we can drop the wedges here for

\begin{equation}\label{eqn:poynting:120}
\begin{aligned}
\frac{dU}{dt}
&= \epsilon_0 c^2 \int dV \gpgradezero{ \BB i \spacegrad \BE -\BE i \spacegrad \BB } \\
&= \epsilon_0 c^2 \int dV \gpgradezero{ (i \BB) \spacegrad \BE -\BE \spacegrad (i\BB) } \\
&= \epsilon_0 c^2 \int dV \spacegrad \cdot ( (i \BB) \cdot \BE ) \\
\end{aligned}
\end{equation}

Justification for this last step can be found below in the derivation of \eqnref{eqn:poynting:poyntingDivergence}.

We can now use Stokes theorem to change this into a surface integral for a final energy flux

\begin{equation}\label{eqn:poynting:140}
\begin{aligned}
\frac{dU}{dt}
&= \epsilon_0 c^2 \int d\BA \cdot ( (i \BB) \cdot \BE ) \\
\end{aligned}
\end{equation}

This last bivector/vector dot product is the Poynting vector

\begin{equation}\label{eqn:poynting:160}
\begin{aligned}
(i \BB) \cdot \BE
&= \gpgradeone{ (i \BB) \cdot \BE } \\
&= \gpgradeone{ i \BB \BE } \\
&= \gpgradeone{ i (\BB \wedge \BE) } \\
&= i (\BB \wedge \BE) \\
&= i^2(\BB \cross \BE) \\
&= \BE \cross \BB \\
\end{aligned}
\end{equation}

So, we can identity the quantity

\begin{equation}\label{eqn:poynting:poynting}
\begin{aligned}
\BP = \epsilon_0 c^2 \BE \cross \BB = \epsilon_0 c (i c \BB) \cdot \BE
\end{aligned}
\end{equation}

as a directed energy density flux through the surface of a containing volume.

\section{With charges and currents}

To calculate time derivatives we want to take Maxwell's equation and put into a form with explicit time derivatives, as was done before, but this time be more careful with the handling of the four vector current term.  Starting with left factoring out of a \(\gamma_0\) from the spacetime gradient.

\begin{equation}\label{eqn:poynting:180}
\begin{aligned}
\grad &= \gamma^0 \partial_0 + \gamma^k \partial_k \\
&= \gamma^0 (\partial_0 - \gamma^k \gamma_0 \partial_k) \\
&= \gamma^0 (\partial_0 + \sigma_k \partial_k) \\
\end{aligned}
\end{equation}

Similarly, the \(\gamma_0\) can be factored from the current density

\begin{equation}\label{eqn:poynting:200}
\begin{aligned}
J
&= \gamma_0 c \rho + \gamma_k J^k \\
&= \gamma_0 (c \rho - \gamma_k \gamma_0 J^k) \\
&= \gamma_0 (c \rho - \sigma_k J^k) \\
&= \gamma_0 (c \rho - \Bj )
\end{aligned}
\end{equation}

With this Maxwell's equation becomes

\begin{equation}\label{eqn:poynting:220}
\begin{aligned}
\gamma_0 \grad F &= \gamma_0 J / \epsilon_0 c \\
(\partial_0 + \spacegrad) ( \BE + i c \BB ) &= \rho/\epsilon_0 - \Bj/\epsilon_0 c \\
\end{aligned}
\end{equation}

A split into even and odd grades including current and charge density is thus

\begin{equation}\label{eqn:poynting:240}
\begin{aligned}
\spacegrad \BE + \partial_t (i \BB) &= \rho/\epsilon_0 \\
\spacegrad (i \BB) c^2 + \partial_t \BE &= -\Bj/\epsilon_0
\end{aligned}
\end{equation}

Now, taking time derivatives of the energy density gives

\begin{equation}\label{eqn:poynting:260}
\begin{aligned}
\PD{t}{U}
&= \PD{t}{}\inv{2} \epsilon_0 \left( \BE^2 - (ic \BB)^2 \right) \\
&= \epsilon_0 \left( \BE \cdot \partial_t \BE - c^2 (i\BB) \cdot \partial_t (i\BB) \right) \\
&= \epsilon_0 \gpgradezero{ \BE ( -\Bj/\epsilon_0 -\spacegrad (i \BB) c^2 ) - c^2 (i\BB) ( -\spacegrad \BE + \rho/\epsilon_0 ) } \\
&= -\BE \cdot \Bj + c^2 \epsilon_0 \gpgradezero{ i\BB \spacegrad \BE -\BE \spacegrad (i \BB) } \\
&= -\BE \cdot \Bj + c^2 \epsilon_0 \left( (i\BB) \cdot (\spacegrad \wedge \BE) - \BE \cdot (\spacegrad \cdot (i \BB)) \right) \\
\end{aligned}
\end{equation}

Using \eqnref{eqn:poynting:poyntingDivergence}, we now have the rate of change of
field energy for the general case including currents.  That is

\begin{equation}\label{eqn:poynting:280}
\begin{aligned}
\PD{t}{U} &= -\BE \cdot \Bj + c^2 \epsilon_0 \spacegrad \cdot (\BE \cdot (i\BB))
\end{aligned}
\end{equation}

Written out in full, and in terms of the Poynting vector this is

\begin{equation}\label{eqn:poynting:300}
\begin{aligned}
\PD{t}{}\frac{\epsilon_0}{2} \left(\BE^2 + c^2 \BB^2\right) + c^2 \epsilon_0 \spacegrad \cdot (\BE \cross \BB) &= -\BE \cdot \Bj
\end{aligned}
\end{equation}

\section{Poynting vector in terms of complete field}

In \eqnref{eqn:poynting:poynting} the individual parts of the complete Faraday
bivector \(F = \BE + i c \BB\) stand out.  How would the Poynting vector be
expressed in terms of \(F\) or in tensor form?

One possibility is to write \(\BE \cross \BB\) in terms of F using
a conjugate split of the Maxwell bivector

\begin{equation}\label{eqn:poynting:320}
\begin{aligned}
F \gamma_0 = - \gamma_0(\BE - i c \BB)
\end{aligned}
\end{equation}

we have
\begin{equation}\label{eqn:poynting:340}
\begin{aligned}
\gamma^0 F \gamma_0 = - (\BE - i c \BB)
\end{aligned}
\end{equation}

and
\begin{equation}\label{eqn:poynting:360}
\begin{aligned}
i c \BB &= \inv{2}(F + \gamma^0 F \gamma_0) \\
\BE &= \inv{2}(F - \gamma^0 F \gamma_0) \\
\end{aligned}
\end{equation}

However \citep{doran2003gap} has the answer more directly in terms of the
electrodynamic stress tensor.

\begin{equation}\label{eqn:poynting:380}
\begin{aligned}
T(a) = -\frac{\epsilon_0}{2} F a F
\end{aligned}
\end{equation}

In particular for \(a = \gamma_0\), this is

\begin{equation}\label{eqn:poynting:400}
\begin{aligned}
T(\gamma_0)
&= -\frac{\epsilon_0}{2} F \gamma_0 F \\
&= \frac{\epsilon_0}{2} (\BE + i c\BB) (\BE - i c \BB) \gamma_0 \\
&= \frac{\epsilon_0}{2} (\BE^2 + c^2 \BB^2 + i c (\BB \BE -\BB \BE)) \gamma_0 \\
&= \frac{\epsilon_0}{2} (\BE^2 + c^2 \BB^2) \gamma_0 + i c \epsilon_0 (\BB \wedge \BE) \gamma_0 \\
&= \frac{\epsilon_0}{2} (\BE^2 + c^2 \BB^2) \gamma_0 + c \epsilon_0 (\BE \cross \BB) \gamma_0 \\
\end{aligned}
\end{equation}

So one sees that the energy and the Poynting vector are components of an energy density momentum four vector

\begin{equation}\label{eqn:poynting:420}
\begin{aligned}
T(\gamma_0)
&= U \gamma_0 + \inv{c} \BP \gamma_0 \\
\end{aligned}
\end{equation}

Writing \(U^0 = U\) and \(U^k = P^k/c\), this is \(T(\gamma_0) = U^\mu \gamma_\mu\).

(inventing such a four vector is how Doran/Lasenby started, so this is not be too surprising).  This
relativistic context helps justify the Poynting vector as a momentum like quantity, but is not quite
satisfactory.  It would make sense to do some classical comparisons, perhaps of interacting wave functions
or something like that, to see how exactly this quantity is momentum like.  Also how exactly is this energy
momentum tensor used, how does it transform, ...

\section{Energy Density from Lagrangian?}

I did not get too far trying to calculate the electrodynamic Hamiltonian density for the general case, so I tried it for a very
simple special case, with just an electric field component in one direction:

\begin{equation}\label{eqn:poynting:440}
\begin{aligned}
\calL
&= \frac{1}{2}(E_x)^2 \\
&= \frac{1}{2}(F_{01})^2 \\
&= \frac{1}{2}(\partial_0 A_1 - \partial_1 A_0)^2 \\
\end{aligned}
\end{equation}

\citep{goldstein1951cm} gives the Hamiltonian density as

\begin{equation}\label{eqn:poynting:460}
\begin{aligned}
\pi &= \frac{\partial \calL}{\partial \dot{n}} \\
\calH &= \dot{n} \pi - \calL
\end{aligned}
\end{equation}

If I try calculating this I get

\begin{equation}\label{eqn:poynting:480}
\begin{aligned}
\pi
&= \frac{\partial}{\partial (\partial_0 A_1)} \left( \frac{1}{2}(\partial_0 A_1 - \partial_1 A_0)^2 \right) \\
&= \partial_0 A_1 - \partial_1 A_0 \\
&= F_{01} \\
\end{aligned}
\end{equation}

So this gives a Hamiltonian of
\begin{equation}\label{eqn:poynting:500}
\begin{aligned}
\calH
&= \partial_0 A_1 F_{01} - \frac{1}{2}(\partial_0 A_1 - \partial_1 A_0)F_{01} \\
&= \frac{1}{2} (\partial_0 A_1 + \partial_1 A_0 )F_{01} \\
&= \frac{1}{2} ((\partial_0 A_1)^2 - (\partial_1 A_0)^2 )
\end{aligned}
\end{equation}

For a Lagrangian density of \(E^2 - B^2\) we have an energy density of \(E^2 + B^2\), so I had have expected the Hamiltonian density here to stay equal to \(E_x^2/2\), but it
does not look like that is what I get (what I calculated is not at all familiar seeming).

If I have not made a mistake here, perhaps I am incorrect in assuming that the Hamiltonian density of the electrodynamic Lagrangian should be the energy density?

\section{Appendix.  Messy details}

For both the charge and the charge free case, we need a proof of

\begin{equation}\label{eqn:poynting:520}
\begin{aligned}
(i\BB) \cdot (\spacegrad \wedge \BE) - \BE \cdot (\spacegrad \cdot (i \BB))
&= \spacegrad \cdot (\BE \cdot (i\BB))
\end{aligned}
\end{equation}

This is relativity straightforward, albeit tedious, to do backwards.

\begin{equation}\label{eqn:poynting:540}
\begin{aligned}
\spacegrad \cdot ((i \BB) \cdot \BE)
&= \gpgradezero{ \spacegrad ((i \BB) \cdot \BE)} \\
&= \inv{2} \gpgradezero{ \spacegrad ( i \BB \BE - \BE i \BB ) } \\
&= \inv{2} \gpgradezero{
  \dot{\spacegrad} i \dot{\BB} \BE
+ \dot{\spacegrad} i \BB \dot{\BE}
- \dot{\spacegrad} \dot{\BE} i \BB
- \dot{\spacegrad} \BE i \dot{\BB}
} \\
&= \inv{2} \gpgradezero{
  \BE \spacegrad (i \BB) - (i\dot{\BB}) \dot{\spacegrad} \BE
+ \dot{\BE} \dot{\spacegrad} i \BB - i \BB \spacegrad \BE
} \\
&= \inv{2} \left(
  \BE \cdot (\spacegrad \cdot (i \BB)) - ((i\dot{\BB}) \cdot \dot{\spacegrad}) \cdot \BE
+ (\dot{\BE} \wedge \dot{\spacegrad}) \cdot (i \BB) - (i \BB) \cdot (\spacegrad \wedge \BE)
\right)
\\
\end{aligned}
\end{equation}

Grouping the two sets of repeated terms after reordering and the associated sign adjustments we have

\begin{equation}\label{eqn:poynting:poyntingDivergence}
\begin{aligned}
\spacegrad \cdot ((i \BB) \cdot \BE) &= \BE \cdot (\spacegrad \cdot (i \BB)) - (i \BB) \cdot (\spacegrad \wedge \BE)
\end{aligned}
\end{equation}

which is the desired identity (in negated form) that was to be proved.

There is likely some theorem that could be used to avoid some of this algebra.

\section{References for followup study}

Some of the content available in the
article \href{http://farside.ph.utexas.edu/teaching/em/lectures/node89.html}{Energy Conservation} looks like it will also be useful to study (in particular
it goes through some examples that convert this from a math treatment to
a physics story).
