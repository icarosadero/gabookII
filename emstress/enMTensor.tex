%
% Copyright � 2012 Peeter Joot.  All Rights Reserved.
% Licenced as described in the file LICENSE under the root directory of this GIT repository.
%

%
%
\mychapter{Energy momentum tensor relation to Lorentz force}
\label{chap:PJenMtensor}
%\date{Feb 17, 2009.  enMTensor.tex}

\section{Motivation}
In \chapcite{PJstressEnergyLorentz} the energy momentum tensor was related
to the Lorentz force in STA form.  Work the same calculation strictly in
tensor form, to
%compare the difficulty of the algebra and
develop more comfort with tensor manipulation.  This should also serve
as a translation aid to compare signs due to metric tensor differences
in other reading.

\subsection{Definitions}

The energy momentum ``tensor'', really a four vector, is defined
in \citep{doran2003gap}
as
%
\begin{equation}\label{eqn:enMTensor:20}
\begin{aligned}
T(a) &=
\frac{\epsilon_0}{2} F a \tilde{F} = -\frac{\epsilon_0}{2} F a {F}
\end{aligned}
\end{equation}
%
We have seen that the divergence of the \(T(\gamma^\mu)\) vectors generate
the Lorentz force relations.

Let us expand this with respect to index lower basis vectors for use in the
divergence calculation.
%
\begin{equation}\label{eqn:enMTensor:40}
\begin{aligned}
T(\gamma^\mu) &=
\lr{ T(\gamma^\mu) \cdot \gamma^\nu } \gamma_\nu
\end{aligned}
\end{equation}
%
So we define
\begin{equation}\label{eqn:enMTensor:60}
\begin{aligned}
T^{\mu \nu}
&= T(\gamma^\mu) \cdot \gamma^\nu
\end{aligned}
\end{equation}
%
and can write these four vectors in tensor form as
\begin{equation}\label{eqn:enMTensor:80}
\begin{aligned}
T(\gamma^\mu) &= T^{\mu \nu} \gamma_\nu
\end{aligned}
\end{equation}
%
\subsection{Expanding out the tensor}

An expansion of \(T^{\mu\nu}\) was done in \chapcite{PJemstresstensor}, but looking
back that seems a peculiar way, using the four vector potential.

Let us try again in terms of \(F^{\mu\nu}\) instead.  Our field is
%
\begin{equation}\label{eqn:enMTensor:100}
\begin{aligned}
F
&= \inv{2} F^{\mu\nu} \gamma_{\mu} \wedge \gamma_\nu
%\\
%&= \inv{2} F_{\mu\nu} \gamma^{\mu} \wedge \gamma^\nu
\end{aligned}
\end{equation}
%
So our tensor components are
\begin{equation}\label{eqn:enMTensor:120}
\begin{aligned}
T^{\mu\nu}
&= T(\gamma^\mu) \cdot \gamma^\nu \\
&= -\frac{\epsilon_0}{8} F^{\lambda\sigma} F^{\alpha\beta}
\gpgradezero{ (\gamma_\lambda \wedge \gamma_\sigma) \gamma^\mu (\gamma_\alpha \wedge \gamma_\beta) \gamma^\nu } \\
\end{aligned}
\end{equation}
%
Or
\begin{equation}\label{eqn:enMTensor:140}
\begin{aligned}
-8{\inv{\epsilon_0}} T^{\mu\nu}
&=
F^{\lambda\sigma} F^{\alpha\beta}
\gpgradezero{
(\gamma_\lambda {\delta_\sigma}^\mu
-\gamma_\sigma {\delta_\lambda}^\mu)
(\gamma_\alpha {\delta_\beta}^\nu
-\gamma_\beta {\delta_\alpha}^\nu)
} \\
&+
F^{\lambda\sigma} F^{\alpha\beta}
\gpgradezero{ (\gamma_\lambda \wedge \gamma_\sigma \wedge \gamma^\mu) (\gamma_\alpha \wedge \gamma_\beta \wedge \gamma^\nu) } \\
\end{aligned}
\end{equation}
%
Expanding only the first term to start with
\begin{equation}\label{eqn:enMTensor:160}
\begin{aligned}
&
F^{\lambda\sigma} F^{\alpha\beta} (\gamma_\lambda {\delta_\sigma}^\mu) \cdot  (\gamma_\alpha {\delta_\beta}^\nu)
+F^{\lambda\sigma} F^{\alpha\beta} (\gamma_\sigma {\delta_\lambda}^\mu) \cdot (\gamma_\beta {\delta_\alpha}^\nu)  \\
&-F^{\lambda\sigma} F^{\alpha\beta} (\gamma_\lambda {\delta_\sigma}^\mu) \cdot (\gamma_\beta {\delta_\alpha}^\nu)
-F^{\lambda\sigma} F^{\alpha\beta} (\gamma_\sigma {\delta_\lambda}^\mu) \cdot (\gamma_\alpha {\delta_\beta}^\nu)  \\
&=
F^{\lambda\mu} F^{\alpha\nu} \gamma_\lambda \cdot \gamma_\alpha
+F^{\mu\sigma} F^{\nu\beta} \gamma_\sigma \cdot \gamma_\beta
-F^{\lambda\mu} F^{\nu\beta} \gamma_\lambda \cdot \gamma_\beta
-F^{\mu\sigma} F^{\alpha\nu} \gamma_\sigma \cdot \gamma_\alpha \\
&=
\eta_{\alpha\beta}
(
F^{\lambda\mu} F^{\alpha\nu} \gamma_\lambda \cdot \gamma^\beta
+
%\eta_{\alpha\beta}
F^{\mu\sigma} F^{\nu\alpha} \gamma_\sigma \cdot \gamma^\beta
-
%\eta_{\alpha\beta}
F^{\lambda\mu} F^{\nu\alpha} \gamma_\lambda \cdot \gamma^\beta
-
%\eta_{\alpha\beta}
F^{\mu\sigma} F^{\alpha\nu} \gamma_\sigma \cdot \gamma^\beta )
\\
&=
\eta_{\alpha\lambda} F^{\lambda\mu} F^{\alpha\nu} + \eta_{\alpha\sigma} F^{\mu\sigma} F^{\nu\alpha}
- \eta_{\alpha\lambda} F^{\lambda\mu} F^{\nu\alpha} - \eta_{\alpha\sigma} F^{\mu\sigma} F^{\alpha\nu}  \\
&=
2( \eta_{\alpha\lambda} F^{\lambda\mu} F^{\alpha\nu} + \eta_{\alpha\sigma} F^{\mu\sigma} F^{\nu\alpha} ) \\
&=
2( \eta_{\alpha\beta} F^{\beta\mu} F^{\alpha\nu} +
\eta_{\alpha\beta} F^{\mu\beta} F^{\nu\alpha} ) \\
&=
4 \eta_{\alpha\beta} F^{\beta\mu} F^{\alpha\nu}  \\
&= 4 F^{\beta\mu} {F_{\beta}}^{\nu} \\
&= 4 F^{\alpha\mu} {F_{\alpha}}^{\nu} \\
\end{aligned}
\end{equation}
%
For the second term after a shuffle of indices we have
\begin{equation}\label{eqn:enMTensor:180}
\begin{aligned}
F^{\lambda\sigma} F_{\alpha\beta}
\eta^{\mu\mu'} \gpgradezero{ (\gamma_\lambda \wedge \gamma_\sigma \wedge \gamma_\mu) (\gamma^\alpha \wedge \gamma^\beta \wedge \gamma^\nu) } \\
\end{aligned}
\end{equation}
%
This dot product is reducible with the identity
\begin{equation}\label{eqn:enMTensor:200}
\begin{aligned}
(a \wedge b \wedge c) \cdot (d \wedge e \wedge f) &=
(((a \wedge b \wedge c) \cdot d) \cdot e) \cdot f
\end{aligned}
\end{equation}
%
leaving a completely antisymmetized sum
%
\begin{equation}\label{eqn:enMTensor:220}
\begin{aligned}
&
F^{\lambda\sigma} F_{\alpha\beta}
\eta^{\mu\mu'}
(
{\delta_\lambda}^\nu {\delta_\sigma}^\beta {\delta_{\mu'}}^\alpha
-{\delta_\lambda}^\nu {\delta_\sigma}^\alpha {\delta_{\mu'}}^\beta
-{\delta_\lambda}^\beta {\delta_\sigma}^\nu {\delta_{\mu'}}^\alpha
+{\delta_\lambda}^\alpha {\delta_\sigma}^\nu {\delta_{\mu'}}^\beta
+{\delta_\lambda}^\beta {\delta_\sigma}^\alpha {\delta_{\mu'}}^\nu
-{\delta_\lambda}^\alpha {\delta_\sigma}^\beta {\delta_{\mu'}}^\nu
) \\
&=
  F^{\nu\beta} F_{{\mu'}\beta} \eta^{\mu\mu'}
- F^{\nu\alpha} F_{\alpha{\mu'}} \eta^{\mu\mu'}
- F^{\beta\nu} F_{{\mu'}\beta} \eta^{\mu\mu'}
+ F^{\alpha\nu} F_{\alpha{\mu'}} \eta^{\mu\mu'}
+ F^{\beta\alpha} F_{\alpha\beta} \eta^{\mu\mu'} {\delta_{\mu'}}^\nu
- F^{\alpha\beta} F_{\alpha\beta} \eta^{\mu\mu'} {\delta_{\mu'}}^\nu
 \\
&=
4 F^{\nu\alpha} F_{{\mu'}\alpha} \eta^{\mu\mu'}
+ 2 F^{\beta\alpha} F_{\alpha\beta} \eta^{\mu\mu'} {\delta_{\mu'}}^\nu
 \\
&=
4 F^{\nu\alpha} {F^{\mu}}_{\alpha}
+ 2 F^{\beta\alpha} F_{\alpha\beta} \eta^{\mu\nu}
 \\
\end{aligned}
\end{equation}
%
Combining these we have
\begin{equation}\label{eqn:enMTensor:240}
\begin{aligned}
T^{\mu\nu}
&=
-\frac{\epsilon_0}{8} \left(
 4 F^{\alpha\mu} {F_{\alpha}}^{\nu}
+ 4 F^{\nu\alpha} {F^{\mu}}_{\alpha}
+ 2 F^{\beta\alpha} F_{\alpha\beta} \eta^{\mu\nu}
\right) \\
&=
\frac{\epsilon_0}{8} \left(
- 4 F^{\alpha\mu} {F_{\alpha}}^{\nu}
+ 4 F^{\alpha\mu} {F^{\nu}}_{\alpha}
+ 2 F^{\alpha\beta} F_{\alpha\beta} \eta^{\mu\nu}
\right) \\
\end{aligned}
\end{equation}
%
%
If by some miracle all the index manipulation worked out, we have
\begin{equation}\label{eqn:stressEnTen:miracle}
\begin{aligned}
T^{\mu\nu} &= {\epsilon_0} \left( F^{\alpha\mu} {F^{\nu}}_{\alpha} + \inv{4} F^{\alpha\beta} F_{\alpha\beta} \eta^{\mu\nu} \right)
\end{aligned}
\end{equation}
%
\subsubsection{Justifying some of the steps}

For justification of some of the
index manipulations of the \(F\) tensor components it is
helpful to think back to the definitions in terms of four vector potentials
%
\begin{equation}\label{eqn:enMTensor:260}
\begin{aligned}
F &= \grad \wedge A \\
&= \partial^\mu A^\nu \gamma_\mu \wedge \gamma_\nu \\
&= \partial_\mu A_\nu \gamma^\mu \wedge \gamma^\nu \\
&= \partial_\mu A^\nu \gamma^\mu \wedge \gamma_\nu \\
&= \partial^\mu A_\nu \gamma_\mu \wedge \gamma^\nu \\
&= \inv{2}(\partial^\mu A^\nu -\partial^\nu A^\mu ) \gamma_\mu \wedge \gamma_\nu \\
&= \inv{2}(\partial_\mu A_\nu -\partial_\nu A_\mu ) \gamma^\mu \wedge \gamma^\nu \\
&= \inv{2}(\partial_\mu A^\nu -\partial^\nu A_\mu ) \gamma^\mu \wedge \gamma_\nu \\
&= \inv{2}(\partial^\mu A_\nu -\partial_\nu A^\mu ) \gamma_\mu \wedge \gamma^\nu
\end{aligned}
\end{equation}
%
So with the shorthand
\begin{equation}\label{eqn:enMTensor:280}
\begin{aligned}
F^{\mu\nu} &= \partial^\mu A^\nu -\partial^\nu A^\mu \\
F_{\mu\nu} &= \partial_\mu A_\nu -\partial_\nu A_\mu \\
{F_{\mu}}^{\nu} &= \partial_\mu A^\nu -\partial^\nu A_\mu \\
{F^{\mu}}_{\nu} &= \partial^\mu A_\nu -\partial_\nu A^\mu
\end{aligned}
\end{equation}
%
We have
\begin{equation}\label{eqn:enMTensor:300}
\begin{aligned}
F
&= \inv{2}F^{\mu\nu} \gamma_\mu \wedge \gamma_\nu \\
&= \inv{2}F_{\mu\nu} \gamma^\mu \wedge \gamma^\nu \\
&= \inv{2}{F_\mu}^\nu \gamma^\mu \wedge \gamma_\nu \\
&= \inv{2}{F^\mu}_\nu \gamma_\mu \wedge \gamma^\nu
\end{aligned}
\end{equation}
%
In particular, and perhaps not obvious without the definitions handy, the following was used above
%
\begin{equation}\label{eqn:enMTensor:320}
\begin{aligned}
{F^{\mu}}_{\nu} &= -{F_{\nu}}^{\mu}
\end{aligned}
\end{equation}
%
\subsection{The divergence}

What is our divergence in tensor form?  This would be
%
\begin{equation}\label{eqn:enMTensor:340}
\begin{aligned}
\grad \cdot T(\gamma^\mu)
&= (\gamma^\alpha \partial_\alpha ) \cdot (T^{\mu\nu} \gamma_\nu) \\
\end{aligned}
\end{equation}
%
So we have
\begin{equation}\label{eqn:enMTensor:360}
\begin{aligned}
\grad \cdot T(\gamma^\mu)
&= \partial_\nu T^{\mu\nu}
\end{aligned}
\end{equation}
%
Ignoring the \(\epsilon_0\) factor for now, chain rule gives us
%
\begin{equation}\label{eqn:enMTensor:380}
\begin{aligned}
(\partial_\nu &F^{\alpha\mu}) {F^{\nu}}_{\alpha} +
F^{\alpha\mu} (\partial_\nu {F^{\nu}}_{\alpha}) +
\inv{2} (\partial_\nu F^{\alpha\beta}) F_{\alpha\beta} \eta^{\mu\nu} \\
&=
(\partial_\nu F^{\alpha\mu}) {F^{\nu}}_{\alpha} +
{F_{\alpha}}^{\mu}
(\partial_\nu F^{\nu\alpha}) +
\inv{2} (\partial_\nu F^{\alpha\beta}) F_{\alpha\beta} \eta^{\mu\nu}
\end{aligned}
\end{equation}
%
Only this center term is recognizable in terms of current since we
have
\begin{equation}\label{eqn:enMTensor:400}
\begin{aligned}
\grad \cdot F &= J/\epsilon_0 c
\end{aligned}
\end{equation}
%
Where the LHS is
\begin{equation}\label{eqn:enMTensor:420}
\begin{aligned}
\grad \cdot F
&= \gamma^\alpha \partial_\alpha \cdot \left( \inv{2} F^{\mu\nu} \gamma_\mu \wedge \gamma_\nu \right) \\
&= \inv{2} \partial_\alpha F^{\mu\nu} ( {\delta^\alpha}_\mu \gamma_\nu -{\delta^\alpha}_\nu \gamma_\mu ) \\
&= \partial_\mu F^{\mu\nu} \gamma_\nu
\end{aligned}
\end{equation}
%
So we have
%
\begin{equation}\label{eqn:enMTensor:440}
\begin{aligned}
\partial_\mu F^{\mu\nu}
&= (J \cdot \gamma^\nu)/\epsilon_0 c \\
&= ((J^\alpha \gamma_\alpha) \cdot \gamma^\nu)/\epsilon_0 c \\
&= J^\nu/\epsilon_0 c
\end{aligned}
\end{equation}
%
Or
\begin{equation}\label{eqn:enMTensor:460}
\begin{aligned}
\partial_\mu F^{\mu\nu} &= J^\nu/\epsilon_0 c
\end{aligned}
\end{equation}
%\partial_\nu F^{\nu\alpha} &= J^\alpha/\epsilon_0 c
%
So we have
%
\begin{equation}\label{eqn:enMTensor:480}
\begin{aligned}
\grad \cdot T(\gamma^\mu)
&= \epsilon_0\left(
(\partial_\nu F^{\alpha\mu}) {F^{\nu}}_{\alpha} +
\inv{2} (\partial_\nu F^{\alpha\beta}) F_{\alpha\beta} \eta^{\mu\nu}
\right)
+
{F_{\alpha}}^{\mu} J^\alpha/c
\end{aligned}
\end{equation}
%
So, to get the expected result the remaining two derivative terms must somehow cancel.  How to reduce these?  Let us look at twice that
%
\begin{equation}\label{eqn:enMTensor:500}
\begin{aligned}
2 (\partial_\nu &F^{\alpha\mu}) {F^{\nu}}_{\alpha} + (\partial_\nu F^{\alpha\beta}) F_{\alpha\beta} \eta^{\mu\nu} \\
&= 2 (\partial^\nu F^{\alpha\mu}) F_{\nu\alpha} + (\partial^\mu F^{\alpha\beta}) F_{\alpha\beta} \\
&= (\partial^\nu F^{\alpha\mu}) (F_{\nu\alpha} -F_{\alpha\nu}) + (\partial^\mu F^{\alpha\beta}) F_{\alpha\beta} \\
&=
(\partial^\alpha F^{\beta\mu}) F_{\alpha\beta}
+(\partial^\beta F^{\mu\alpha}) F_{\alpha\beta}
+ (\partial^\mu F^{\alpha\beta}) F_{\alpha\beta} \\
&=
(\partial^\alpha F^{\beta\mu} +\partial^\beta F^{\mu\alpha} + \partial^\mu F^{\alpha\beta}) F_{\alpha\beta} \\
\end{aligned}
\end{equation}
%
Ah, there is the trivector term of Maxwell's equation hiding in there.
%
\begin{equation}\label{eqn:enMTensor:520}
\begin{aligned}
0
&= \grad \wedge F \\
&= \gamma_\mu \partial^\mu \wedge \left(\inv{2} F^{\alpha\beta} (\gamma_\alpha \wedge \gamma_\beta) \right) \\
&= \inv{2} (\partial^\mu F^{\alpha\beta}) (\gamma_\mu \wedge \gamma_\alpha \wedge \gamma_\beta) \\
&= \inv{3!}
\left(
\partial^\mu F^{\alpha\beta}
+\partial^\alpha F^{\beta\mu}
+\partial^\beta F^{\mu\alpha}
\right)
(\gamma_\mu \wedge \gamma_\alpha \wedge \gamma_\beta)
\end{aligned}
\end{equation}
%
Since this is zero, each component of this trivector must separately equal zero, and we have
%
\begin{equation}\label{eqn:enMTensor:540}
\begin{aligned}
\partial^\mu F^{\alpha\beta} +\partial^\alpha F^{\beta\mu} +\partial^\beta F^{\mu\alpha} = 0
\end{aligned}
\end{equation}
%
So, where \(T^{\mu\nu}\) is defined by \eqnref{eqn:stressEnTen:miracle}, the final result is
%
\begin{equation}\label{eqn:stressEnTen:covariantTensor}
\begin{aligned}
\partial_\nu T^{\mu\nu} &= F^{\alpha\mu} J_\alpha/c
\end{aligned}
\end{equation}
