%
% Copyright � 2012 Peeter Joot.  All Rights Reserved.
% Licenced as described in the file LICENSE under the root directory of this GIT repository.
%

%
%
\chapter{Understanding four velocity transform from rest frame}
\index{four velocity}
\index{boost}
\index{rest frame}
\label{chap:velocityTx}
%\date{August 13, 2008}

\section{}

\citep{doran2003gap} writes \(v = R \gamma_0 R^\dagger\), as a proper velocity expressed in terms of a rest frame velocity and a Lorentz boost.
This was not clear to me, and would probably be a lot more
obvious to me if I had fully read chapter 5, but in my defense it is a hard read without first getting
more familiarity with basic relativity.

Let us just expand this out
to see how this works.  First thing to note is that there is an omitted factor
of \(c\), and I will add that back in here, since I am not comfortable enough
without it explicitly for now.

With:

\begin{equation}\label{eqn:velocityTx:20}
\begin{aligned}
\Bv/c &= \tanh\left(\alpha\right)\vcap \\
R &= \exp\left(\alpha \vcap/2\right)
\end{aligned}
\end{equation}

We want to expansion this Lorentz boost exponential (see details section) and apply it to the rest frame basis vector.  Writing
\(C = \cosh\left(\alpha/2\right)\), and \(S = \sinh\left(\alpha/2\right)\), we have:

\begin{equation}\label{eqn:velocityTx:40}
\begin{aligned}
v
&= R \left(c \gamma_0\right) R^\dagger \\
&= c \left(C + \vcap S\right) \gamma_0 \left(C - \vcap S\right) \\
&= c \left(C \gamma_0 + S \vcap \gamma_0\right) \left(C - \vcap S\right) \\
&= c \left( C^2 \gamma_0 + SC \vcap \gamma_0 -CS \gamma_0\vcap - S^2 \vcap \gamma_0 \vcap \right) \\
\end{aligned}
\end{equation}

Now, here things can start to get confusing since \(\vcap\) is a spatial quantity with vector-like spacetime basis bivectors \(\sigma_i = \gamma_i \gamma_0\).  Factoring out the \(\gamma_0\) term, utilizing the fact that \(\gamma_0\) and \(\sigma_i\) anticommute (see below).

\begin{equation}\label{eqn:velocityTx:60}
\begin{aligned}
v
&= c \left( C^2 + S^2 + 2 SC \vcap \right) \gamma_0 \\
&= c \left( \cosh\left(\alpha\right) + \vcap \sinh\left(\alpha\right) \right) \gamma_0 \\
&= c \cosh\left(\alpha\right) \left( 1 + \vcap \tanh\left(\alpha\right) \right) \gamma_0 \\
&= c \cosh\left(\alpha\right) \left( 1 + \Bv/c \right) \gamma_0 \\
&= c \gamma \left( 1 + \Bv/c \right) \gamma_0 \\
&= \gamma \left( c \gamma_0 + \sum v^i \gamma_i\right) \\
&= \frac{dt}{d\tau}\left( c \gamma_0 + \sum v^i \gamma_i\right) \\
&= \frac{dt}{d\tau} \frac{d}{dt}\left( c t \gamma_0 + \sum x^i \gamma_i\right) \\
&= \frac{dt}{d\tau} \frac{d}{dt} \sum x^{\mu} \gamma_{\mu} \\
&= \frac{d}{d\tau} \sum x^{\mu} \gamma_{\mu} \\
&= \frac{dx}{d\tau}
\end{aligned}
\end{equation}

So, we get the end result that demonstrates that a Lorentz boost applied to the rest event vector \(x = x^0 \gamma_0 = c t \gamma_0\) directly produces the four velocity for the motion from the new viewpoint.  This makes some intuitive sense, but
I do not feel this is necessarily obvious without demonstration.

This also explains how the text is able to use the wedge and dot product ratios with the \(\gamma_0\) basis vector
to produce the relative spatial velocity.  If one introduces a rest frame proper velocity of
\(w = \frac{d}{dt}\left(ct \gamma_0\right) = c \gamma_0\), then one has:

\begin{equation}\label{eqn:velocityTx:80}
\begin{aligned}
v \cdot w
&= \left(\sum \frac{d x^{\mu}}{d\tau} \gamma_{\mu}\right) \cdot \left(c\gamma_0\right) \\
&= c^2 \gamma
\end{aligned}
\end{equation}

\begin{equation}\label{eqn:velocityTx:100}
\begin{aligned}
v \wedge w
&= \left(\sum \frac{d x^{\mu}}{d\tau} \gamma_{\mu}\right) \wedge \left(c\gamma_0\right) \\
&= \left(\sum \frac{d x^{i}}{d\tau} \gamma_{i}\right) \wedge \left(c\gamma_0\right) \\
&= c \sum \frac{d x^{i}}{d\tau} \sigma_{i} \\
&= c \frac{dt}{d\tau} \sum \frac{d x^{i}}{dt} \sigma_{i} \\
&= c \gamma \sum \frac{d x^{i}}{dt} \sigma_{i} \\
\end{aligned}
\end{equation}

Combining these one has the spatial observer dependent relative velocity:

\begin{equation}
\frac{v \wedge w}{v \cdot w} = \inv{c} \sum \frac{d x^{i}}{dt} \sigma_{i} = \frac{\Bv}{c}
\end{equation}

\subsection{Invariance of relative velocity?}

What is not clear to me is whether this can be used to determine the relative velocity between two particles in the general case, when one of them is not a rest frame velocity (time progression only at a fixed point in space.)
The text seems
to imply this is the case, so perhaps it is obvious to them only and not me;)

This can be verified relatively easily for the extreme case, where one boosts both the \(w\), and \(v\) velocities to measure \(v\) in its rest frame.

Expressed mathematically this is:

\begin{equation}\label{eqn:velocityTx:120}
\begin{aligned}
w &= c \gamma_0 \\
v &= R w R^\dagger \\
v' &= R^\dagger v R = R^\dagger R c \gamma_0 R^\dagger R = c \gamma_0 \\
w' &= R^\dagger w R \\
\end{aligned}
\end{equation}

Now, this last expression for \(w'\) can be expanded brute force as was done initially to calculate \(v\) (and I in
fact did that initially without thinking).  The end result matches what should have been the intuitive expectation, with the velocity components all negated in a conjugate like fashion:

\begin{equation*}
w' = \gamma\left( c\gamma_0 - \sum v^i \gamma_i \right)
\end{equation*}

With this result we have:

\begin{equation*}
v' \cdot w' = c \gamma_0 \cdot \gamma\left( c\gamma_0 - \sum v^i \gamma_i \right) = \gamma c^2
\end{equation*}

\begin{equation}\label{eqn:velocityTx:140}
\begin{aligned}
v' \wedge w'
&= c \gamma_0 \wedge \gamma\left( c\gamma_0 - \sum v^i \gamma_i \right) \\
&= -c \gamma \sum v^i \gamma_0 \gamma_i \\
&= c \gamma \sum v^i \sigma_i \\
\end{aligned}
\end{equation}

Dividing the two we have the following relative velocity between the two proper velocities:

\begin{equation*}
\frac{v' \wedge w'}{v' \cdot w'} = \inv{c} \sum v^i \sigma_i = \Bv/c.
\end{equation*}

Lo and behold, this is the same as when the first event worldline was in its rest frame, so we have the same
relative velocity regardless of which of the two are observed at rest.  The remaining obvious question is
how to show that this is a general condition, assuming that it is.

\subsection{General invariance?}

Intuitively, I would guess that this is fact the case because when only two particles are considered, the result should be the same independent of which of the
two is considered at rest.

Mathematically, I would express this statement by saying that if one has
a Lorentz boost that takes \(v' = T v T^\dagger\) to its rest frame, then application of this to both proper velocities leaves both the wedge and dot product
parts of this ratio unchanged:

\begin{equation}\label{eqn:velocityTx:160}
\begin{aligned}
v \cdot w
&= \left(T^\dagger v' T\right) \cdot \left(T^\dagger w' T\right) \\
&= \gpgradezero{\left(T^\dagger v' T\right) \left(T^\dagger w' T\right)} \\
&= \gpgradezero{T^\dagger v' w' T} \\
&= \gpgradezero{T^\dagger v' \cdot w' T} +
\mathLabelBox
[
   labelstyle={xshift=2cm},
   linestyle={out=270,in=90, latex-}
]
{\gpgradezero{T^\dagger v' \wedge w' T}}{\(=0\)} \\
&= \left(v' \cdot w'\right)\gpgradezero{T^\dagger T} \\
&= v' \cdot w'
\end{aligned}
\end{equation}

\begin{equation}\label{eqn:velocityTx:180}
\begin{aligned}
v \wedge w
&= \left(T^\dagger v' T\right) \wedge \left(T^\dagger w' T\right) \\
&= \gpgradetwo{\left(T^\dagger v' T\right) \left(T^\dagger w' T\right)} \\
&= \gpgradetwo{T^\dagger v' w' T} \\
&=
\mathLabelBox
[
   labelstyle={xshift=2cm},
   linestyle={out=270,in=90, latex-}
]
{\gpgradetwo{T^\dagger v' \cdot w' T}}{\(=0\)} + \gpgradetwo{T^\dagger v' \wedge w' T} \\
&= T^\dagger \left(v' \wedge w'\right) T
\end{aligned}
\end{equation}

FIXME: can not those last \(T\) factors be removed somehow?

\section{Appendix. Omitted details from above}

\subsection{exponential of a vector}

Understanding the vector exponential is a prerequisite above.  This is defined
and interpreted by series expansion as with matrix exponentials.
Expanding
in series the exponential of a vector \(\Bx = x\xcap\), we have:

\begin{equation}\label{eqn:velocityTx:200}
\begin{aligned}
\exp\left(\Bx\right)
&= \sum \frac{\Bx^{2k}}{\left(2k\right)!} + \sum \frac{\Bx^{2k+1}}{\left(2k+1\right)!} \\
&= \sum \frac{x^{2k}}{\left(2k\right)!} + \xcap \sum \frac{x^{2k+1}}{\left(2k+1\right)!} \\
&= \cosh\left(x\right) + \xcap \sinh\left(x\right)
\end{aligned}
\end{equation}

Notationally this can also be written:

\begin{equation*}
\exp\left(\Bx\right) = \cosh\left(\Bx\right) + \sinh\left(\Bx\right)
\end{equation*}

But doing so will not really help.

\subsection{\texorpdfstring{\(\Bv\)}{v} anticommutes with \texorpdfstring{\(\gamma_0\)}{gamma 0}}

\begin{equation}\label{eqn:velocityTx:220}
\begin{aligned}
\Bv \gamma_0
&= \sum v^i \sigma_i \gamma_0 \\
&= \sum v^i \gamma_i \gamma_0 \gamma_0 \\
&= -\sum v^i \gamma_0 \gamma_i \gamma_0 \\
&= - \gamma_0 \sum v^i \gamma_i \gamma_0 \\
&= - \gamma_0 \sum v^i \sigma_0 \\
&= - \gamma_0 \Bv
\end{aligned}
\end{equation}
