%
% Copyright � 2012 Peeter Joot.  All Rights Reserved.
% Licenced as described in the file LICENSE under the root directory of this GIT repository.
%

%
%
\chapter{Lorentz transformation of spacetime gradient}
\index{spacetime gradient!Lorentz transformation}
\label{chap:spacetimegrad}
%\date{July 16, 2008}

\section{Motivation}

We have observed that the wave equation is Lorentz invariant, and conversely that invariance of the form of the wave equation under linear transformation for light can be used to calculate the Lorentz transformation.  Specifically, this means that we require the equations of light (wave equation) retain its form after a change of variables that includes
a (possibly scaled) translation.  The wave equation should have no mixed partial terms, and retain the form:

\begin{equation*}
\lr{
\spacegrad^2 - \partial_{ct}^2
} F =
\lr{ {\spacegrad'}^2 - \partial_{ct'}^2 }
F = 0
\end{equation*}

Having expressed the spacetime gradient with a (STA) Minkowski basis, and knowing that the Maxwell equation written using the spacetime gradient is Lorentz invariant:

\begin{equation*}
\grad F = J,
\end{equation*}

we therefore expect that the square root of the wave equation (Laplacian) operator is also Lorentz invariant.  Here this idea is explored, and we look at how the spacetime
gradient behaves under Lorentz transformation.

\subsection{Lets do it}

Our spacetime gradient is

\begin{equation*}
\grad = \sum \gamma^{\mu} \frac{\partial}{\partial x^{\mu}}
\end{equation*}

Under Lorentz transformation we can transform the \(x^1=x\), and \(x^0 = ct\) coordinates:

\begin{equation*}
\begin{bmatrix}
x' \\
ct' \\
\end{bmatrix}
=
\gamma
\begin{bmatrix}
1 & -\beta \\
-\beta & 1 \\
\end{bmatrix}
\begin{bmatrix}
x \\
ct \\
\end{bmatrix}
\end{equation*}

Set \(c=1\) for convenience, and use this to transform the partials:

\begin{equation}\label{eqn:spacetimegrad:20}
\begin{aligned}
\Partial{}{x}
&= \Partial{x'}{x} \Partial{}{x'} + \Partial{t'}{x} \Partial{}{t'} \\
&= \gamma
\left( \Partial{}{x'} -\beta \Partial{}{t'} \right) \\
\end{aligned}
\end{equation}

\begin{equation}\label{eqn:spacetimegrad:40}
\begin{aligned}
\Partial{}{t}
&= \Partial{x'}{t} \Partial{}{x'} + \Partial{t'}{t} \Partial{}{t'} \\
&= \gamma \left( -\beta \Partial{}{x'} + \Partial{}{t'} \right) \\
\end{aligned}
\end{equation}

Inserting this into our expression for the gradient we have

\begin{equation}\label{eqn:spacetimegrad:60}
\begin{aligned}
\grad
&= \gamma^0 \Partial{}{t}
 + \gamma^1 \Partial{}{x}
 + \gamma^2 \Partial{}{y}
 + \gamma^3 \Partial{}{z} \\
&= \gamma^0 \gamma \left( -\beta \Partial{}{x'} + \Partial{}{t'} \right)
 + \gamma^1 \gamma \left( \Partial{}{x'} -\beta \Partial{}{t'} \right)
 + \gamma^2 \Partial{}{y}
 + \gamma^3 \Partial{}{z}.
\end{aligned}
\end{equation}

Grouping by the primed partials this is:

\begin{equation}\label{eqn:spacetimegrad:80}
\begin{aligned}
\grad
&= \gamma \left(\gamma^0 - \beta\gamma^1\right) \Partial{}{t'}
 + \gamma \left(\gamma^1 - \beta\gamma^0\right) \Partial{}{x'}
 + \gamma^2 \Partial{}{y}
 + \gamma^3 \Partial{}{z}.
\end{aligned}
\end{equation}

Lo and behold, the basis vectors with respect to the new coordinates appear to themselves transform as a Lorentz pair.  Specifically:

\begin{equation*}
\begin{bmatrix}
{\gamma^1}' \\
{\gamma^0}' \\
\end{bmatrix}
=
\gamma
\begin{bmatrix}
1 & -\beta \\
-\beta & 1 \\
\end{bmatrix}
\begin{bmatrix}
{\gamma^1} \\
{\gamma^0} \\
\end{bmatrix}
\end{equation*}

Now this is a bit curious looking since these new basis vectors are a funny mix of the original time and space basis vectors.  Observe however that these linear combinations of the basis vectors \({\gamma^0}'\), and \({\gamma^1}'\) do behave just as adequately as timelike and spacelike basis vectors:

\begin{equation}\label{eqn:spacetimegrad:100}
\begin{aligned}
{\gamma^0}' {\gamma^0}'
&= \gamma^2
\lr{ -\beta \gamma^1 + \gamma^0 }
\lr{ -\beta \gamma^1 + \gamma^0 } \\
&= \gamma^2
\lr{ -\beta^2 + 1 -\beta \gamma^0\gamma^1 -\beta \gamma^1 \gamma^0 }
\\
&= \gamma^2 \Bigl(-\beta^2 + 1 +
\mathLabelBox
[
   labelstyle={xshift=2cm},
   linestyle={out=270,in=90, latex-}
]
{\beta \gamma^1\gamma^0 -\beta \gamma^1 \gamma^0}{\(=0\)} \Bigr) \\
&= 1
\end{aligned}
\end{equation}

and for the transformed ``spacelike'' vector, it squares like a spacelike vector:

\begin{equation}\label{eqn:spacetimegrad:120}
\begin{aligned}
{\gamma^1}' {\gamma^1}'
&= \gamma^2
\lr{ \gamma^1 -\beta \gamma^0 }
\lr{ \gamma^1 -\beta \gamma^0 }
\\
&= \gamma^2
\lr{ -1 + \beta^2 -\beta\gamma^0 \gamma^1 -\beta\gamma^1\gamma^0 }
\\
&= \gamma^2 \Bigl(-1 + \beta^2 +
\mathLabelBox
[
   labelstyle={xshift=2cm},
   linestyle={out=270,in=90, latex-}
]
{\beta \gamma^1\gamma^0 -\beta \gamma^1 \gamma^0}{\(=0\)} \Bigr) \\
&= -1
\end{aligned}
\end{equation}

The conclusion is that like the wave equation, its square root, the spacetime gradient is also Lorentz invariant, and to achieve this invariance we
transform both the coordinates and the basis vectors (there was no need to transform the basis vectors for the wave equation since it is a scalar
equation).

In fact, this gives a very interesting way to view the Lorentz transform.  It is not just notational that we can think of the spacetime gradient as one of the square roots of the wave equation.
Like the vector square root of a scalar
there are infinitely many such roots, all differing by an angle or rotation in the vector space:

\begin{equation*}
(R \Bn R^\dagger)^2 = 1
\end{equation*}

Requiring the mixed signature (Minkowski) metric for the space requires only that we need a slightly different meaning for any of the possible rotations
applied to the vector.

\subsection{transform the spacetime bivector}

I am not sure of the significance of the following yet, but it is interesting to note that the spacetime bivector for the transformed coordinate
pair is also invariant:

\begin{equation}\label{eqn:spacetimegrad:140}
\begin{aligned}
{\gamma^1}' {\gamma^0}'
&= \gamma^2
\lr{ \gamma^1 -\beta \gamma^0 }
\lr{ -\beta \gamma^1 + \gamma^0 } \\
&= \gamma^2
\lr{ \beta -\beta +\beta^2 \gamma^0 \gamma^1 + \gamma^1 \gamma^0 } \\
&= \gamma^2 \lr{ 1-\beta^2 } \gamma^1 \gamma^0 \\
&= \gamma^1 \gamma^0
\end{aligned}
\end{equation}

We can probably use this to figure out how to transform bivector quantities like the electromagnetic field \(F\).
