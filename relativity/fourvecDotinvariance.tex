%
% Copyright � 2012 Peeter Joot.  All Rights Reserved.
% Licenced as described in the file LICENSE under the root directory of this GIT repository.
%

%
%
\mychapter{Four vector dot product invariance and Lorentz rotors}
\index{Lorentz invariance}
\label{chap:fourvecDotinvariance}
%\date{August 1, 2008}        % Deleting this command produces today's date.
\section{}

Prof. Ramamurti Shankar's
In the relativity lectures of
\citep{ShankarPhy200} Prof. Shankar
indicates that the
four vector dot product
is a Lorentz invariant.  This makes some logical sense, but lets demonstrate it explicitly.

Start with a Lorentz transform matrix between coordinates for two four vectors (omitting the components perpendicular  to the motion) :

\begin{equation*}
{
\begin{bmatrix}
x^1 \\
x^0 \\
\end{bmatrix}
}'
=
\gamma
\begin{bmatrix}
1 & -\beta \\
-\beta & 1
\end{bmatrix}
\begin{bmatrix}
x^1 \\
x^0 \\
\end{bmatrix}
\end{equation*}

\begin{equation*}
{
\begin{bmatrix}
y^1 \\
y^0 \\
\end{bmatrix}
}'
=
\gamma
\begin{bmatrix}
1 & -\beta \\
-\beta & 1
\end{bmatrix}
\begin{bmatrix}
y^1 \\
y^0 \\
\end{bmatrix}
\end{equation*}

Now write out the dot product between the two vectors given the perceived length and time measurements for the same events in the moving frame:

\begin{equation}\label{eqn:fourvecDotinvariance:20}
\begin{aligned}
X' \cdot Y'
&= \gamma^2 \left( (-\beta x^1 + x^0)(-\beta y^1 + y^0) -(x^1 -\beta x^0) (y^1 -\beta y^0) \right) \\
&= \gamma^2 \left( (\beta^2 x^1 y^1 + x^0 y^0) + x^0 y^1( -\beta + \beta ) + x^1 y^0( -\beta + \beta ) -(x^1 y^1 + \beta^2 x^0 y^0) \right) \\
&= \gamma^2 \left( x^0 y^0 (1-\beta^2) - (1-\beta^2) x^1 y^1 \right) \\
&= x^0 y^0 - x^1 y^1 \\
&= X \cdot Y
\end{aligned}
\end{equation}

This completes the proof of dot product Lorentz invariance.  An automatic consequence of this is invariance
of the Minkowski length.

\subsection{Invariance shown with hyperbolic trig functions}

Dot product or length invariance can also be shown with the hyperbolic representation of the Lorentz transformation:

\begin{equation}\label{eqn:fVecDotInv:hyperbolicmatrix}
{
\begin{bmatrix}
x^1 \\
x^0 \\
\end{bmatrix}
}'
=
\begin{bmatrix}
\cosh(\alpha) & -\sinh(\alpha) \\
-\sinh(\alpha) & \cosh(\alpha)
\end{bmatrix}
\begin{bmatrix}
x^1 \\
x^0 \\
\end{bmatrix}
\end{equation}

Writing \(S=\sinh(\alpha)\), and \(C=\cosh(\alpha)\) for short, this gives:

\begin{equation}\label{eqn:fourvecDotinvariance:40}
\begin{aligned}
X' \cdot Y'
&= \left( (-S x^1 + C x^0)(-S y^1 + C y^0) -(C x^1 -S x^0) (C y^1 -S y^0) \right) \\
&= \left( (S^2  x^1 y^1 + C^2  x^0 y^0) + x^0 y^1( -SC + SC ) + x^1 y^0( -SC + SC ) -(C^2  x^1 y^1 + S^2  x^0 y^0) \right) \\
&= \left( x^0 y^0 (C^2  -S^2 ) - (C^2 -S^2 ) x^1 y^1 \right) \\
&= x^0 y^0 - x^1 y^1 \\
&= X \cdot Y
\end{aligned}
\end{equation}

This is not really any less work.

\section{Geometric product formulation of Lorentz transform}

We can show the above invariance almost trivially when we write the Lorentz boost in exponential form.  However we first have to show
how to do so.

Writing the spacetime bivector \(\gamma_{10} = \gamma_1 \wedge \gamma_0\) for short, lets calculate the exponential of this spacetime bivector, as scaled with a rapidity
angle \(\alpha\) :

\begin{equation}\label{eqn:fVecDotInv:bivecexponential}
\exp(\gamma_{10}\alpha) = \sum \frac{(\gamma_{10}\alpha)^k}{k!}
\end{equation}

Now, the spacetime bivector has a unit square:

\begin{equation*}
{\gamma_{10}}^2 = \gamma_{1010} = -\gamma_{1001} = -\gamma_{11} = 1
\end{equation*}

so, we can split the sum of \eqnref{eqn:fVecDotInv:bivecexponential} into even and odd parts, and pull out the common bivector factor:

\begin{equation}\label{eqn:fVecDotInv:bivechyper}
\exp(\gamma_{10}\alpha)
= \sum \frac{\alpha^{2k}}{(2k)!} + \gamma_{10}\sum \frac{\alpha^{2k+1}}{(2k+1)!}
= \cosh(\alpha) + \gamma_{10} \sinh(\alpha)
\end{equation}

\subsection{Spatial rotation}
\index{rotation invariance}

So, this quite a similar form as bivector exponential with a Euclidean metric.  For such a space the bivector had a negative square, just like the complex unit imaginary,
which allowed for the normal trigonometric split of the exponential:

\begin{equation}
\exp(\Be_{12}\theta)
= \sum (-1)^k\frac{\theta^{2k}}{(2k)!} + \Be_{12}\sum (-1)^k\frac{\theta^{2k+1}}{(2k+1)!}
= \cos(\theta) + \Be_{12} \sin(\theta)
\end{equation}

Now, with the Minkowski metric having a negative square for purely spatial components, how does a purely spacial bivector behave when squared?  Let us try it with

\begin{equation*}
{\gamma_{12}}^2
= \gamma_{1212}
= -\gamma_{1221}
= \gamma_{11}
= -1
\end{equation*}

This also has a square that behaves like the unit imaginary, so we can do spacial rotations with rotors like we can with Euclidean space.  However, we have to invert the sign of the angle when using a Minkowski metric.  Take a specific example of a 90 degree rotation in the x-y plane, expressed in complex form:

\begin{equation}\label{eqn:fourvecDotinvariance:60}
\begin{aligned}
R_{\pi/2}(\gamma_1)
&= \gamma_1 \exp({ \gamma_{12} \pi/2 }) \\
&= \gamma_1 (0 + \gamma_{12}) \\
&= -\gamma_2 \\
\end{aligned}
\end{equation}

In general our Rotor equation with a Minkowski \((+,-,-,-)\) metric will be thus be:

\begin{equation}\label{eqn:fVecDotInv:spacerot}
R_{\theta}(x) = \exp( i\theta/2) x \exp( -i\theta/2)
\end{equation}

Here \(i\) is a spatial bivector (a bivector with negative square), such as \(\gamma_{1}\wedge\gamma_{2}\), and the rotation sense is with increasing angle from \(\gamma_1\) towards \(\gamma_2\).

\subsection{Validity of the double sided spatial rotor formula}

To demonstrate the validity of \eqnref{eqn:fVecDotInv:spacerot} one has to observe how the unit vectors \(\gamma_{\mu}\) behave with respect to commutation, and how that behavior results in either commutation or conjugate commutation with the exponential rotor.  Without any loss of generality one can restrict attention to a specific example, such as bivector \(\gamma_{12}\).  By inspection, \(\gamma_0\), and \(\gamma_3\) both commute since an even number of exchanges in position is required for either:

\begin{equation}\label{eqn:fourvecDotinvariance:80}
\begin{aligned}
\gamma_{0} \gamma_{12}
&= \gamma_{0} \wedge \gamma_{1} \wedge \gamma_{2} \\
&= \gamma_{1} \wedge \gamma_{2} \wedge \gamma_{0} \\
&= \gamma_{12} \gamma_0
\end{aligned}
\end{equation}

For this reason, application of the double sided rotation does not change any such (perpendicular) vector that commutes with the rotor:

\begin{equation}\label{eqn:fourvecDotinvariance:100}
\begin{aligned}
R_{\theta}(x_{\perp})
&= \exp( i\theta/2) x_{\perp} \exp( -i\theta/2) \\
&= x_{\perp} \exp( i\theta/2) \exp( -i\theta/2) \\
&= x_{\perp}
\end{aligned}
\end{equation}

Now for the basis vectors that lie in the plane of the spatial rotation we have anticommutation:

\begin{equation}\label{eqn:fourvecDotinvariance:120}
\begin{aligned}
\gamma_{1} \gamma_{12}
&= -\gamma_{1} \gamma_{21}  \\
&= -\gamma_{121} \\
&= -\gamma_{12} \gamma_{1}
\end{aligned}
\end{equation}

\begin{equation}\label{eqn:fourvecDotinvariance:140}
\begin{aligned}
\gamma_{2} \gamma_{12}
&= \gamma_{21}\gamma_{2} \\
&= -\gamma_{12}\gamma_{2}
\end{aligned}
\end{equation}

Given an understanding of how the unit vectors either commute or anticommute with the bivector for the plane of rotation, one can now see how these behave when multiplied by a rotor expressed exponentially:

\begin{equation}\label{eqn:fVecDotInv:spatialcommutationrule}
\gamma_{\mu}\exp(i\theta)
= \gamma_{\mu}\left( \cos(\theta) + i\sin(\theta) \right)
=
\left\{
\begin{array}{l l}
\left( \cos(\theta) + i\sin(\theta) \right) \gamma_{\mu} & \quad \mbox{if \(\gamma_{\mu} \cdot i = 0\)} \\
\left( \cos(\theta) - i\sin(\theta) \right) \gamma_{\mu} & \quad \mbox{if \(\gamma_{\mu} \cdot i \ne 0\)} \\
\end{array} \right.
\end{equation}

The condition \(\gamma_{\mu} \cdot i = 0\) corresponds to a spacelike vector perpendicular to the plane of rotation, or a timelike vector, or any combination of the two, whereas
\(\gamma_{\mu} \cdot i \ne 0\) is true for any spacelike vector that lies completely in the plane of rotation.

Putting this information all together, we now complete the verification that the double sided rotor formula leaves the perpendicular spacelike or the timelike components untouched.  For for purely spacelike vectors in the plane of rotation we recover the single sided complex form rotation as illustrated by the following x-y plane rotation:

\begin{equation}\label{eqn:fourvecDotinvariance:160}
\begin{aligned}
R_{\theta}(x_{\parallel})
&= \exp( \gamma_{12}\theta/2) x_{\parallel} \exp( -\gamma_{12}\theta/2) \\
&= x_{\parallel} \exp( -\gamma_{12}\theta/2) \exp( -\gamma_{12}\theta/2) \\
&= x_{\parallel} \exp( -\gamma_{12}\theta) \\
\end{aligned}
\end{equation}

\subsection{Back to time space rotation}

Now, like we can express a spatial rotation in exponential form, we can do the same for the hyperbolic ``rotation'' matrix of \eqnref{eqn:fVecDotInv:hyperbolicmatrix}.  Direct expansion
\footnote{
The paper ``Generalized relativistic velocity addition with spacetime algebra'', http://arxiv.org/pdf/physics/0511247.pdf derives the bivector form of this Lorentz boost directly in an interesting fashion.  Simple relativistic arguments are used that are quite similar to those of Einstein in his ``Relativity, the special and general theory'' appendix.  This paper is written in a form that requires you to work out many of the details yourself (likely for brevity).  However, once that extra work is done, I found the first half of that paper quite readable.
}
of the product is the easiest way to see that this is the case:

\begin{equation}\label{eqn:fourvecDotinvariance:180}
\begin{aligned}
\left(\gamma_{1} x^1 + \gamma_{0} x^0 \right)\exp(\gamma_{10}\alpha)
&= \left(\gamma_{1} x^1 + \gamma_{0} x^0 \right) \left( \cosh(\alpha) +\gamma_{10}\sinh(\alpha) \right) \\
\end{aligned}
\end{equation}

\begin{equation}\label{eqn:fVecDotInv:lorentz}
\begin{aligned}
&\left(\gamma_{1} x^1 + \gamma_{0} x^0 \right)\exp(\gamma_{10}\alpha) \\
&\qquad = \gamma_1\left( x^1 \cosh(\alpha) - x^0 \sinh(\alpha)\right)
 + \gamma_0\left( x^0 \cosh(\alpha) - x^1 \sinh(\alpha)\right)
\end{aligned}
\end{equation}

As with the spatial rotation, full characterization of this exponential rotation operator, in both single and double sided form requires that one looks at how the various unit vectors commute with the unit bivector.  Without loss of generality one can restrict attention to a specific case, as done with the \(\gamma_{10}\) above.

As in the spatial case, \(\gamma_{2}\), and \(\gamma_{3}\) both commute with \(\gamma_{10} = \gamma_1 \wedge \gamma_0\).  Example:

\begin{equation*}
\gamma_{2} \gamma_{10}
= \gamma_2 \wedge \gamma_1 \wedge \gamma_0
= \gamma_1 \wedge \gamma_0 \wedge \gamma_2
= \gamma_{10} \gamma_{2}
\end{equation*}

Now, consider each of the basis vectors in the spacetime plane.

\begin{equation*}
\gamma_{0} \gamma_{10}
= \gamma_{010}
= \gamma_{01} \gamma_{0}
= -\gamma_{10} \gamma_{0}
\end{equation*}

\begin{equation*}
\gamma_{1} \gamma_{10}
= \gamma_{110}
= -\gamma_{101}
= -\gamma_{10} \gamma_{1}
\end{equation*}

Both of the basis vectors in the spacetime plane anticommute with the bivector that describes the plane, and as a result we have a conjugate change in the exponential comparing left and right multiplication as with a spatial rotor.  Summarizing for the general case by introducing a spacetime rapidity plane described by a bivector

\(\Balpha = \alphacap \alpha\), we have:

\begin{equation}\label{eqn:fVecDotInv:spacetimecommutationrule}
\begin{aligned}
\gamma_{\mu}\exp(\Balpha)
&= \gamma_{\mu}\left( \cosh(\alpha) + \alphacap\sinh(\alpha) \right) \\
&=
\left\{
\begin{array}{l l}
\left( \cosh(\alpha) + \alphacap\sinh(\alpha) \right) \gamma_{\mu} & \quad \mbox{if \(\gamma_{\mu} \cdot \alphacap = 0\)} \\
\left( \cosh(\alpha) - \alphacap\sinh(\alpha) \right) \gamma_{\mu} & \quad \mbox{if \(\gamma_{\mu} \cdot \alphacap \ne 0\)} \\
\end{array} \right.
\end{aligned}
\end{equation}

Observe the similarity between \eqnref{eqn:fVecDotInv:spatialcommutationrule}, and \eqnref{eqn:fVecDotInv:spacetimecommutationrule} for spatial
and spacetime rotors.  Regardless of whether the plane is spacelike, or a spacetime plane we have the same rule:

\begin{equation}\label{eqn:fVecDotInv:generalrule}
\gamma_{\mu}\exp(\BB)
=
\left\{
\begin{array}{l l}
\exp(\BB) \gamma_{\mu} & \quad \mbox{if \(\gamma_{\mu} \cdot \Bcap = 0\)} \\
\exp(-\BB) \gamma_{\mu} & \quad \mbox{if \(\gamma_{\mu} \cdot \Bcap \ne 0\)}
\end{array} \right.
\end{equation}

Here, if \(\BB\) is a spacelike bivector (\(\BB^2 < 0\)) we get trigonometric functions generated by the exponentials, and if it represents
the spacetime plane \(\BB^2 > 0\) we get the hyperbolic functions.  As with the spatial rotor formulation, we have the same result for the
general signature bivector, and can write the generalized spacetime or spatial rotation as:

\begin{equation}
R_{\BB}(x) = \exp(-\BB/2) x \exp(\BB/2)
\end{equation}

Some care is required assigning meaning to the bivector angle \(\BB\).  We have seen that this is an negatively oriented spatial rotation in the $
\Bcap$ plane when spacelike.  How about for the spacetime case?
Lets go back and rewrite \eqnref{eqn:fVecDotInv:lorentz} in terms of vector
relations, with \(\Bv = v \vcap\)

\begin{equation}
\begin{aligned}
&\left( x^1 \vcap + x^0 \gamma_0 \right)
\left(
\frac{1}{\sqrt{1 -{\abs{(\Bv/c)}}^2}} + \frac{(\Bv/c) \gamma_0}{\sqrt{1 -{\abs{(\Bv/c)}}^2}}
\right) \\
&\qquad =
\vcap \gamma
\left( x^1 - x^0 v/c \right)
+
\gamma_0 \gamma
\left( x^0 - x^1 v/c \right)
\end{aligned}
\end{equation}

This allows for the following identification:

\begin{equation*}
\cosh(\alpha) + \vcap \gamma_0 \sinh(\alpha) = \exp( \vcap \gamma_0 \alpha)
=
\frac{1 + (\Bv/c) \gamma_0}{\sqrt{1 -{\abs{\Bv/c}}^2}}
\end{equation*}

which gives us the rapidity bivector (\(\BB\) above) in terms of the values we are familiar with:

\begin{equation*}
\vcap \gamma_0 \alpha = \log\left(
\frac{1 + (\Bv/c) \gamma_0}{\sqrt{1 -{\abs{\Bv/c}}^2}} \right)
\end{equation*}

Or,

\begin{equation*}
\BB = \vcap \gamma_0 \alpha = \tanh^{-1}(v/c) \vcap \gamma_0
\end{equation*}

Now since \(\abs{v/c} < 1\), the hyperbolic inverse tangent here can be expanded in (the slowly convergent) power series:

\begin{equation*}
\tanh^{-1}(x) = \sum_{k=0} \frac{x^{2k+1}}{2k+1}
\end{equation*}

Observe that this has only odd powers, and \(((\Bv/c) \gamma_0)^{2k+1} = \vcap\gamma_0 (v/c)^{2k+1}\).  This allows for the notational nicety of working with the spacetime bivector directly instead of only its magnitude:

\begin{equation}
\BB = \tanh^{-1}((\Bv/c) \gamma_0)
\end{equation}

\subsection{FIXME}

Revisit the equivalence of the two identities above.  How can one get from the log
expression to the hyperbolic inverse tangent directly?

\subsection{Apply to dot product invariance}

With composition of rotation and boost rotors we can form a generalized Lorentz transformation.  For example application of a rotation with rotor \(R\), to a boost with spacetime rotor \(L_0\), we get a combined more general transformation:

\begin{equation*}
L(x) = R ( L_0 x {L_0}^\dagger ) R^\dagger
\end{equation*}

In both cases, the rotor and its reverse when multiplied are identity:

\begin{equation*}
1 = R R^\dagger = L L^\dagger
\end{equation*}

It is not hard to see one can also compose an arbitrary set of rotations and boosts in the same fashion.  The new rotor will also satisfy \(L L^\dagger = 1\).

Application of such a rotor to a four vector we have:

\begin{equation*}
X' = L X L^\dagger
\end{equation*}

\begin{equation*}
Y' = L Y L^\dagger
\end{equation*}

\begin{equation}\label{eqn:fourvecDotinvariance:200}
\begin{aligned}
X' \cdot Y' &= (L X L^\dagger) \cdot (L Y L^\dagger) \\
&= \gpgradezero{ L X L^\dagger L Y L^\dagger } \\
&= \gpgradezero{ L X Y L^\dagger } \\
&= \gpgradezero{ L (X \cdot Y) L^\dagger } + \gpgradezero{ L (X \wedge Y) L^\dagger } \\
&= (X \cdot Y) \gpgradezero{ L L^\dagger } \\
&= X \cdot Y
\end{aligned}
\end{equation}

It is also clear that the four bivector \(X \wedge Y\) will also be Lorentz invariant.  This also implies that the geometric product of two four vectors \(X Y\) will also be Lorentz invariant.

UPDATE (Aug 14): I
do not
recall my reasons for thinking that the bivector invariance was clear initially.  It does not seem so clear now after the fact so I should have written it down.
