%
% Copyright � 2012 Peeter Joot.  All Rights Reserved.
% Licenced as described in the file LICENSE under the root directory of this GIT repository.
%

%
%
%\input{../peeter_prologue.tex}

\mychapter{Relativistic Doppler formula}
\index{Doppler equation!relativistic}
\label{chap:frequencyTx}
%\blogpage{http://sites.google.com/site/peeterjoot/math2009/frequencyTx.pdf}
%%\date{June 27, 2009}
%%\revisionInfo{\(RCSfile: frequencyTx.tex,v \) Last \(Revision: 1.6 \) \(Date: 2009/10/22 02:07:20 \)}

%\date{June 27, 2009 \(RCSfile: frequencyTx.tex,v \) Last \(Revision: 1.6 \) \(Date: 2009/10/22 02:07:20 \)}

\beginArtWithToc

\section{Transform of angular velocity four vector}

It was possible to derive the Lorentz boost matrix by requiring that the wave equation operator
%
\begin{equation}\label{eqn:frequencyTx:20}
\begin{aligned}
\grad^2 = \inv{c^2}\frac{\partial^2}{\partial t^2} - \spacegrad^2
\end{aligned}
\end{equation}
%
retain its form under linear transformation (\chapcite{PJLorentzWave}).  Applying spatial Fourier transforms (\chapcite{PJwaveFourier}), one finds that solutions to the wave equation
%
\begin{equation}\label{eqn:frequencyTx:40}
\begin{aligned}
\grad^2 \psi(t,\Bx) = 0
\end{aligned}
\end{equation}
%
Have the form
%
\begin{equation}\label{eqn:frequencyTx:60}
\begin{aligned}
\psi(t, \Bx) = \int A(\Bk) e^{i(\Bk \cdot \Bx - \omega t)} d^3 k
\end{aligned}
\end{equation}
%
Provided that \(\omega = \pm c \Abs{\Bk}\).  Wave equation solutions can therefore be thought of as continuously weighted superpositions of constrained fundamental solutions
%
\begin{equation}\label{eqn:frequencyTx:80}
\begin{aligned}
\psi &= e^{i(\Bk \cdot \Bx - \omega t)} \\
c^2 \Bk^2 &= \omega^2
\end{aligned}
\end{equation}
%
The constraint on frequency and wave number has the look of a Lorentz square
%
\begin{equation}\label{eqn:frequencyTx:100}
\begin{aligned}
\omega^2 - c^2 \Bk^2 = 0
\end{aligned}
\end{equation}
%
Which suggests that in additional to the spacetime vector
%
\begin{equation}\label{eqn:frequencyTx:120}
\begin{aligned}
X = (ct, \Bx) = x^\mu \gamma_\mu
\end{aligned}
\end{equation}
%
evident in the wave equation fundamental solution, we also have a frequency-wavenumber four vector
%
\begin{equation}\label{eqn:frequencyTx:140}
\begin{aligned}
K = (\omega/c, \Bk) = k^\mu \gamma_\mu
\end{aligned}
\end{equation}
%
The pair of four vectors above allow the fundamental solutions to be put explicitly into covariant form
%
\begin{equation}\label{eqn:frequencyTx:160}
\begin{aligned}
K \cdot X = \omega t - \Bk \cdot \Bx = k_\mu x^\mu
\end{aligned}
\end{equation}
%
\begin{equation}\label{eqn:frequencyTx:180}
\begin{aligned}
\psi = e^{-i K \cdot X}
\end{aligned}
\end{equation}
%
Let us also examine the transformation properties of this fundamental solution, and see as a side effect that \(K\)
has transforms appropriately as a four vector.
%
\begin{equation}\label{eqn:frequencyTx:200}
\begin{aligned}
0 &= \grad^2 \psi(t,\Bx) \\
&= {\grad'}^2 \psi(t',\Bx') \\
&= {\grad'}^2 e^{i(\Bx' \cdot \Bk' - \omega' t')} \\
&= -\left(\frac{{\omega'}^2}{c^2} - {\Bk'}^2 \right) e^{i(\Bx' \cdot \Bk' - \omega' t')} \\
\end{aligned}
\end{equation}
%
We therefore have the same form of frequency wave number constraint in the transformed frame (if we require that
the wave function for light is unchanged under transformation)
%
\begin{equation}\label{eqn:frequencyTx:220}
\begin{aligned}
{\omega'}^2 = c^2 {\Bk'}^2
\end{aligned}
\end{equation}
%
Writing this as
%
\begin{equation}\label{eqn:frequencyTx:240}
\begin{aligned}
0 = {\omega}^2 - c^2 {\Bk}^2 = {\omega'}^2 - c^2 {\Bk'}^2
\end{aligned}
\end{equation}
%
singles out the Lorentz invariant nature of the \((\omega, \Bk)\) pairing, and we conclude that this pairing
does indeed transform as a four vector.

\section{Application of one dimensional boost}

Having attempted to justify the four vector nature of the wave number vector \(K\), now move on to application of a boost along the x-axis to this vector.
%
\begin{equation}\label{eqn:frequencyTx:260}
\begin{aligned}
\begin{bmatrix}
\omega' \\
c k' \\
\end{bmatrix}
&=
\gamma
\begin{bmatrix}
1 & -\beta \\
-\beta& 1 \\
\end{bmatrix}
\begin{bmatrix}
\omega \\
c k \\
\end{bmatrix}
\\
&=
\begin{bmatrix}
\omega - v k \\
c k - \beta \omega
\end{bmatrix}
\end{aligned}
\end{equation}
%
We can take ratios of the frequencies if we make use of the dependency between \(\omega\) and \(k\).  Namely, \(\omega = \pm c k\).  We then have
%
\begin{equation}\label{eqn:frequencyTx:280}
\begin{aligned}
\frac{\omega'}{\omega}
%&= \omega - (v/c) (\pm \omega)
&= \gamma(1 \mp \beta) \\
&= \frac{1 \mp \beta}{\sqrt{1 - \beta^2}} \\
&= \frac{1 \mp \beta}{\sqrt{1 - \beta}\sqrt{1 + \beta}} \\
\end{aligned}
\end{equation}
%
For the positive angular frequency this is
%
\begin{equation}\label{eqn:frequencyTx:300}
\begin{aligned}
\frac{\omega'}{\omega}
&= \frac{\sqrt{1 - \beta}}{\sqrt{1 + \beta}}
\\
\end{aligned}
\end{equation}
%
and for the negative frequency the reciprocal.

Deriving this with a Lorentz boost is much simpler than the time dilation argument in wikipedia doppler article \citep{wiki:relDoppler}.  EDIT: Later found exactly the above boost argument in the wiki k-vector article \citep{wiki:kvector}.

What is missing here is putting this in a physical context properly with source and reciever frequencies spelled out.  That would make this more than just math.

%\EndArticle
