%
% Copyright � 2012 Peeter Joot.  All Rights Reserved.
% Licenced as described in the file LICENSE under the root directory of this GIT repository.
%

%
%
\mychapter{GravitoElectroMagnetism}
\index{Gravito-electromagnetism}
\label{chap:gem}
%\date{October 26, 2008.  gem.tex}

\section{Some rough notes on reading of GravitoElectroMagnetism review}

I found the GEM equations interesting, and explored the surface of them slightly.  Here are some notes, mostly as a reference for myself ... looking at the
GEM equations mostly generates questions, especially since I do not have the GR
background to understand where the potentials (ie: what is that stress energy
tensor \(T_{\mu\nu}\)) nor the specifics of where the metric tensor
(perturbation of the Minkowski metric) came from.

\section{Definitions}

The article \citep{mashhoon2003gbr} outlines the GEM equations, which in short
are

Scalar and potential fields

\begin{equation}\label{eqn:gem:20}
\begin{aligned}
\Phi \approx \frac{GM}{r}, \quad \BA \approx \frac{G}{c} \frac{\BJ \cross \Bx}{r^3}
\end{aligned}
\end{equation}

Gauge condition

\begin{equation}\label{eqn:gem:40}
\begin{aligned}
\inv{c}\PD{t}{\Phi} + \spacegrad \cdot \left( \inv{2} \BA \right) = 0.
\end{aligned}
\end{equation}

GEM fields
\begin{equation}\label{eqn:gem:60}
\begin{aligned}
\BE = - \spacegrad \Phi -\inv{c} \PD{t}{}\left( \inv{2} \BB \right), \quad \BB = \spacegrad \cross \BA
\end{aligned}
\end{equation}

and finally the Maxwell-like equations are

\begin{equation}\label{eqn:gem:80}
\begin{aligned}
\spacegrad \cross \BE &= -\inv{c} \PD{t}{}\left(\inv{2}\BB\right) \\
\spacegrad \cdot \left( \inv{2} \BB \right) &= 0 \\
\spacegrad \cdot \BE &= 4 \pi G \rho \\
\spacegrad \cross \left( \inv{2} \BB \right) &= \inv{c} \PD{t}{\BE} + \frac{4\pi G}{c}\BJ
\end{aligned}
\end{equation}

\section{STA form}

As with Maxwell's equations a Clifford algebra representation should be possible to put this into a more symmetric form.  Combining the spatial div and grads, following conventions from \citep{doran2003gap} we have

\begin{equation}\label{eqn:gem:100}
\begin{aligned}
\spacegrad \BE &= 4 \pi G \rho + \inv{c} \PD{t}{}\left(\inv{2}I \BB\right) \\
\spacegrad \left( \inv{2} I \BB \right) &= \inv{c} \PD{t}{\BE} + \frac{4\pi G}{c}\BJ
\end{aligned}
\end{equation}

Or
\begin{equation}\label{eqn:gem:120}
\begin{aligned}
\left( \spacegrad -\inv{c} \PD{t}{}\right) \left( \BE + \inv{2} I \BB \right) &= \frac{4\pi G}{c} \left( c \rho + \BJ \right)
\end{aligned}
\end{equation}

Left multiplication with \(\gamma_0\), using a time positive metric signature (\((\gamma_0)^2=1\)),
\begin{equation}\label{eqn:gem:140}
\begin{aligned}
\left( \spacegrad -\inv{c} \PD{t}{}\right) \gamma_0 \left( -\BE + \inv{2} I \BB \right) &= \frac{4\pi G}{c} \left( c \rho \gamma_0 + J^i \gamma_i \right)
\end{aligned}
\end{equation}

But \(\left( \spacegrad -\inv{c} \PD{t}{}\right) \gamma_0 = \gamma_i \partial_i - \gamma_0 \partial_0 = -\gamma^\mu \partial_\mu = -\grad\).  Introduction of a four vector mass density \(J = c\rho \gamma_0 + J^i \gamma_i = J^\mu \gamma_\mu\), and a bivector field \(F = \BE -\inv{2} I \BB\) this is

\begin{equation}\label{eqn:gem:160}
\begin{aligned}
\grad F = -\frac{4\pi G}{c} J
\end{aligned}
\end{equation}

The gauge condition suggests a four potential \(V = \Phi \gamma_0 + \BA \gamma_0 = V^\mu \gamma_\mu\), where \(V^0 = \Phi\), and \(V^i = A^i/2\).  This merges the
space and time parts of the gauge condition

\begin{equation}\label{eqn:gem:180}
\begin{aligned}
\grad \cdot V = \gamma^\mu \partial_\mu \cdot \gamma_\nu V^\nu = \partial_\mu V^\mu = \inv{c}\PD{t}{\Phi} + \inv{2}\partial_i A^i.
\end{aligned}
\end{equation}

It is reasonable to assume that \(F = \grad \wedge V\) as in electromagnetism.  Let us see if this is the case

\begin{equation}\label{eqn:gem:200}
\begin{aligned}
\BE - I\BB/2
&= - \spacegrad \Phi -\inv{c} \PD{t}{}\left( \inv{2} \BB \right) - I\spacegrad \cross \BA/2 \\
&= - \gamma_i \partial_i \gamma_0 V^0 - \inv{2} \partial_0 A^i \gamma_i \gamma_0 + \spacegrad \wedge \BA/2 \\
&= \gamma^i \partial_i \gamma_0 V^0 + \gamma^0 \partial_0 \gamma_i A^i/2 - \gamma_i \partial_i \wedge \gamma_j V^j \\
&= \gamma^i \partial_i \gamma_0 V^0 + \gamma^0 \partial_0 \gamma_i V^i + \gamma^i \partial_i \wedge \gamma_j V^j \\
&= \gamma^\mu \partial_\mu \wedge \gamma_\nu V^\nu \\
&= \grad \wedge V
\end{aligned}
\end{equation}

Okay, so in terms of potential we have the form as Maxwell's equation

\begin{equation}\label{eqn:gem:field}
\begin{aligned}
\grad (\grad \wedge V) &= -\frac{4\pi G}{c} J.
\end{aligned}
\end{equation}

With the gauge condition \(\grad \cdot V = 0\), this produces the wave equation

\begin{equation}\label{eqn:gem:220}
\begin{aligned}
\grad^2 V &= -\frac{4\pi G}{c} J.
\end{aligned}
\end{equation}

In terms of the author's original equation 1.2 it appears that roughly
\(V^\mu = \barh_{0\mu}\), and \(J^\mu \propto T_{0\mu}\).

This is logically how he is able to go from that equation to the Maxwell
form since both have the same four-vector wave equation form (when \(T_{ij} \approx 0\)).  To give the potentials specific values in terms of mass and current
distribution appears to be where the retarded integrals are used.

The author expresses \(T^{\mu\nu}\) in terms of \(\rho\), and mass current \(j\), but
the field equations are in terms of \(T_{\mu\nu}\).  What metric tensor is
used to translate from upper to lower indices in this case.  ie: is it \(g_{\mu\nu}\), or \(\eta_{\mu\nu}\) ?

\section{Lagrangians}

\subsection{Field Lagrangian}

Since the electrodynamic equation and corresponding field Lagrangian is
\begin{equation}\label{eqn:gem:240}
\begin{aligned}
\grad (\grad \wedge A) &= \frac{J}{\epsilon_0 c} \\
\LL &= -\frac{\epsilon_0 c}{2} (\grad \wedge A)^2 + A \cdot J
\end{aligned}
\end{equation}

Then, from \eqnref{eqn:gem:field}, the GEM field Lagrangian in covariant form is

\begin{equation}\label{eqn:gem:260}
\begin{aligned}
\LL &= \frac{c}{8 \pi G} (\grad \wedge V)^2 + V \cdot J \\
\end{aligned}
\end{equation}

Writing \(F^{\mu\nu} = \partial^\mu V^\nu - \partial^\nu V^\mu\), the scalar part of this Lagrangian is:

\begin{equation}\label{eqn:gem:280}
\begin{aligned}
\LL &= -\frac{c}{16 \pi G} F^{\mu\nu} F_{\mu\nu} + V^\sigma J_\sigma \\
\end{aligned}
\end{equation}

Is this expression hiding in the Einstein field equations?

What is the Lagrangian for Newtonian gravity, and how do they compare?

\subsection{Interaction Lagrangian}

The metric (equation 1.4) in the article is given to be

\begin{equation}\label{eqn:gem:300}
\begin{aligned}
ds^2 &=
-c^2\left(1 - 2 \frac{\Phi}{c^2}\right) dt^2
+\frac{4}{c}\left(\BA \cdot d\Bx \right) dt
+\left(1 + 2 \frac{\Phi}{c^2}\right) \delta_{ij}dx^i dx^j \\
\implies
\Abs{ds^2} = c^2 (d\tau)^2 &= (dx^0)^2 - \sum_i (dx^i)^2
-2 \frac{V_0}{c^2} (dx^0)^2
-\frac{8}{c^2} V_i dx^i dx^0
- 2 \frac{V_0}{c^2} \delta_{ij}dx^i dx^j
\end{aligned}
\end{equation}

With \(v = \gamma_\mu dx^\mu/d\tau\), the Lagrangian for interaction is

\begin{equation}\label{eqn:gem:320}
\begin{aligned}
\LL
&= \inv{2} m \Abs{\frac{ds}{d\tau}}^2  \\
&= \inv{2} m c^2 \\
&= \inv{2} m v^2 -2 \frac{m V_0}{c^2} \sum_\mu (\xdot^\mu)^2 -\frac{8 m}{c^2} V_i \xdot^0 \xdot^i  \\
\end{aligned}
\end{equation}

\begin{equation}\label{eqn:gem:interactionlagrangian}
\begin{aligned}
\LL &= \inv{2} m v^2 - 2m \left( V_0 \sum_\mu (\xdot^\mu / c)^2 + 4 V_i (\xdot^0/c) (\xdot^i/c) \right)
\end{aligned}
\end{equation}

Now, unlike the Lorentz force Lagrangian
\begin{equation}\label{eqn:gem:340}
\begin{aligned}
\LL &= \inv{2} m v^2 + q A \cdot v/c,
\end{aligned}
\end{equation}

the Lagrangian of \eqnref{eqn:gem:interactionlagrangian} is quadratic in powers of \(\xdot^\mu\).
There are remarks in the article saying that the non-covariant Lagrangian used to arrive at the Lorentz force equivalent was a first order approximation.
Evaluation of this interaction Lagrangian does not produce anything like the
\(\pdot_\mu = \kappa F_{\mu\nu}\xdot^\nu\) that we see in electrodynamics.

The calculation is not interesting but the end result for reference is

\begin{equation}\label{eqn:gem:360}
\begin{aligned}
\pdot
%&= \frac{4m}{c^2} \frac{d}{d\tau}\left( V_0 \gamma^\mu v^\mu + 2V_i (v^i \gamma^0 + v^0 \gamma^i) \right) \\
%&- \frac{2m}{c^2} \left( \sum_\mu (v^\mu)^2 \grad V_0 + 4 v^0 v^i \grad V_i \right) \\
&= \frac{4m}{c^2} \left( (v \cdot \grad V_0) \gamma^\mu v^\mu + 2 (v \cdot \grad V_i) (v^i \gamma^0 + v^0 \gamma^i) \right) \\
&+ \frac{4m}{c^2} \left( V_0 \gamma^\mu a^\mu + 2V_i (a^i \gamma^0 + a^0 \gamma^i) \right) \\
&- \frac{2m}{c^2} \left( \sum_\mu (v^\mu)^2 \grad V_0 + 4 v^0 v^i \grad V_i \right)
\end{aligned}
\end{equation}

This can be simplified somewhat, but no matter what it will be quadratic in the velocity coordinates.

The article also says that the line element is approximate.
Has some of what
is required for a more symmetric covariant interaction proper force been
discarded?

\section{Conclusion}

The ideas here are interesting.  At a high level, roughly, as I see it, the equation

\begin{equation}\label{eqn:gem:380}
\begin{aligned}
\grad^2 h_{0\mu} = T_{0\mu}
\end{aligned}
\end{equation}

has exactly the same form as Maxwell's equations in covariant form, so you can define an antisymmetric field tensor equation in the same way, treating these elements of h, and the corresponding elements of T as a four vector potential and mass current.

That said, I do not have the GR background to know understand the introduction.  For example, how to actually arrive at 1.2
or how to calculated your metric tensor in equation 1.4.  I would have expected 1.4 to have a more symmetric form like the covariant Lorentz force Lagrangian (\(v^2 + kA.v\)), since you can get a Lorentz force like equation out of it.  Because of the quadratic velocity terms, no matter how one varies that metric with respect to s as a parameter, one cannot get anything at all close to the electrodynamics Lorentz force equation \(m\ddot{x}^\mu = q F_\mu\nu \dot{x}_\nu\), so the correspondence between electromagnetism and GR breaks down once one considers the interaction.
