%
% Copyright � 2012 Peeter Joot.  All Rights Reserved.
% Licenced as described in the file LICENSE under the root directory of this GIT repository.
%

%
%
%\chapter{Preface}
% this suppresses an explicit chapter number for the preface.
\chapter*{Preface}%\normalsize
  \thispagestyle{empty}
  \addcontentsline{toc}{chapter}{Preface}

This is an exploratory collection of notes containing worked examples of more advanced applications of Geometric Algebra (GA), also known as Clifford Algebra.

These notes are (dis)organized into the following chapters

\begin{itemize}
\item Relativity.
This is a fairly incoherent chapter, including an attempt to develop the Lorentz transformation by requiring wave equation invariance, Lorentz transformation of the four-vector (STA) gradient, and a look at the relativistic doppler equation.
\item Electrodynamics.
The GA formulation of Maxwell's equation (singular in GA) is developed here.  Various basic topics of electrodynamics are examined using the GA toolbox, including the Biot-Savart law, the covariant form for Maxwell's equation (Space Time Algebra, or STA), four vectors and potentials, gauge invariance, TEM waves, and some Lienard-Wiechert problems.
\item Lorentz Force.
Here the GA form of the Lorentz force equation and its relation to the usual vectorial representation is explored.  This includes some application of boosts to the force equation to examine how it transforms under observe dependent conditions.
\item Electrodynamic stress energy.
This chapter explores concepts of electrodynamic energy and momentum density and the GA representation of the Poynting vector and the stress-energy tensors.
\item Quantum Mechanics.
This chapter includes a look at the Dirac Lagrangian, and how this can be cast into GA form.  Properties of the Pauli and Dirac bases are explored, and how various matrix operations map onto their GA equivalents.  A bivector form for the angular momentum operator is examined.  A multivector form for the first few spherical harmonic eigenfunctions is developed.  A multivector factorization of the three and four dimensional Laplacian and the angular momentum operators are derived.
\item Fourier treatments.
Solutions to various PDE equations are attempted using Fourier series and transforms.  Much of this chapter was exploring Fourier solutions to the GA form of Maxwell's equation, but a few other non-geometric algebra Fourier problems were also tackled.
\end{itemize}

I can not promise that I have explained things in a way that is good for anybody else.  My audience was essentially myself as I existed at the time of writing (i.e. undergraduate engineering background), but the prerequisites, for both the mathematics and the physics, have evolved continually.

Peeter Joot  \quad peeterjoot@pm.me
