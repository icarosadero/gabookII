%
% Copyright © 2013 Peeter Joot.  All Rights Reserved.
% Licenced as described in the file LICENSE under the root directory of this GIT repository.
%
\part{Relativity}
   %
% Copyright © 2012 Peeter Joot.  All Rights Reserved.
% Licenced as described in the file LICENSE under the root directory of this GIT repository.
%
%
%
\mychapter{Wave equation based Lorentz transformation derivation}
\index{Lorentz transformation!wave equation}
\label{chap:PJLorentzWave}
%\date{June 25, 2008.  lorentz.tex}
\section{Intro}
My old electrodynamics book did a Lorentz transformation derivation using a requirement for invariance of a spherical light shell.  ie:
\begin{equation}\label{eqn:lorentz:20}
x^2 - c^2 t^2 = {x'}^2 - c^2 {t'}^2
\end{equation}
%
Such an approach does not require any sophisticated math, but I never understood why that invariance condition could be assumed.
To understand that intuitively, requires that you understand how the speed of light is constant.  There are some subtleties
involved in understanding that which are not necessarily obvious to me.  A good illustration of this is Feynman's question
about what speed to expect light to be
going from a rocket ship going 100000 miles per second is a good example (ref: book: Six not so easy parts).
Many people who would say "yes, the speed of light is constant" would still answer 280000 miles per second for that question.

I present below an alternate approach to deriving the Lorentz transformation.  This has a bit more math (ie: partial differentials for
change of variables in the wave equation).  However, compared to really understanding that the speed of light is constant,
I think it is easier to conceptualize the idea that light is wavelike regardless of the motion of the observer since it (ie: an electrodynamic field)
must satisfy the wave equation (ie: Maxwell's equations) regardless of the parametrization.  I am curious if somebody
else also new to the subject of relativity would agree?

The motivation for this is the fact that many introductory relativity texts mention how Lorentz observed that
while Maxwell's equations were not invariant with respect to Galilean
transformation, they were with his modified transformation.

I found it interesting to consider this statement with a bit of detail.  The result is what I think is an interesting approach
to introducing the Lorentz transformation.

\section{The wave equation for Electrodynamic fields (light)}

From Maxwell's equations one can show that in a charge and current free region
the electric field and magnetic field both satisfy the wave equation:
%
\begin{equation}
\laplacian - \inv{c^2}\frac{\partial^2}{\partial t^2} = 0
\end{equation}
%
I believe this is the specific case where there are the light contains enough
photons that the bulk (wavelike) phenomena dominate and quantum effects do not have to be considered.

The wikipedia article Electromagnetic radiation (under Derivation)

%\htmladdnormallink{<URL>} { https://en.wikipedia.org/wiki/Electromagnetic_radiation#Derivation }

goes over this nicely.

Although this can be solved separately for either \(\BE\) or \(\BB\) the two are not independent.
This dependence is nicely expressed by writing the electromagnetic field as a complete
bivector \(\BF = \BE + i c \BB\), and in that form the
general solution to this equation for the combined electromagnetic
field is:
%
\begin{equation}
\BF = (\BE_0 + \kcap \wedge \BE_0) f( \kcap \cdot \Br \pm c t)
\end{equation}
%
Here f is any function, and represents the amplitude of the waveform.

\section{Verifying Lorentz invariance}

The Lorentz transformation for a moving (primed) frame where the motion is
along the x axis is (\(\beta = v/c\), \(\gamma^{-2} = 1 -\beta^2\)).
%
\begin{equation*}
\begin{bmatrix}
x' \\
c t' \\
\end{bmatrix}
=
\gamma
\begin{bmatrix}
1 & -\beta \\
-\beta & 1 \\
\end{bmatrix}
\end{equation*}
%
Or,
\begin{equation*}
\begin{bmatrix}
x \\
c t \\
\end{bmatrix}
=
\gamma
\begin{bmatrix}
1 & \beta \\
\beta & 1 \\
\end{bmatrix}
\end{equation*}
%
Using this we can express the partials of the wave equation in the
primed frame.  Starting with the first derivatives:
%
\begin{equation}\label{eqn:lorentz:160}
\begin{aligned}
\frac{\partial}{\partial x}
&= \frac{\partial x'}{\partial x} \frac{\partial}{\partial x'} + \frac{\partial c t'}{\partial x} \frac{\partial}{\partial c t'} \\
&= \gamma \frac{\partial}{\partial x'} - \gamma \beta \frac{\partial}{\partial c t'} \\
\end{aligned}
\end{equation}
%
And:
%
\begin{equation}\label{eqn:lorentz:180}
\begin{aligned}
\frac{\partial}{\partial ct}
&= \frac{\partial x'}{\partial ct} \frac{\partial}{\partial x'} + \frac{\partial c t'}{\partial ct} \frac{\partial}{\partial c t'} \\
&= -\beta \gamma \frac{\partial}{\partial x'} + \gamma \frac{\partial}{\partial c t'} \\
\end{aligned}
\end{equation}
%
Thus the second partials in terms of the primed frame are:
%
\begin{equation}\label{eqn:lorentz:200}
\begin{aligned}
\frac{\partial^2}{\partial x^2}
&= \gamma^2
\left(\frac{\partial}{\partial x'} - \beta \frac{\partial}{\partial c t'} \right)
\left(\frac{\partial}{\partial x'} - \beta \frac{\partial}{\partial c t'} \right)
\\
&= \gamma^2
\left(
\frac{\partial^2}{\partial x'\partial x'} + \beta^2 \frac{\partial^2}{\partial c t'\partial c t'}
- \beta \left(
\frac{\partial^2}{\partial x' \partial c t'}
\frac{\partial^2}{\partial c t' \partial x'}
\right)
\right)
\\
\end{aligned}
\end{equation}
%
\begin{equation}\label{eqn:lorentz:220}
\begin{aligned}
\frac{\partial^2}{\partial ct \partial ct}
&= \gamma^2
\left(
\beta^2 \frac{\partial^2}{\partial x'\partial x'} + \frac{\partial^2}{\partial c t'\partial c t'}
- \beta \left(
\frac{\partial^2}{\partial x' \partial c t'}
\frac{\partial^2}{\partial c t' \partial x'}
\right)
\right)
\\
\end{aligned}
\end{equation}
%
Thus the wave equation transforms as:
%
\begin{equation}\label{eqn:lorentz:240}
\begin{aligned}
\frac{\partial^2}{\partial x^2} - \frac{\partial^2}{\partial ct \partial ct}
&=
\gamma^2
\left(
(1 - \beta^2) \frac{\partial^2}{\partial x'\partial x'} + (\beta^2 -1)\frac{\partial^2}{\partial c t'\partial c t'}
\right) \\
&=
\frac{\partial^2}{\partial x'\partial x'} - \frac{\partial^2}{\partial c t'\partial c t'}
\end{aligned}
\end{equation}
%
which is what we expect but nice to see written out in full without having to introduce Minkowski space, and its invariant norm,
or use Einstein's subtle arguments from his "Relativity, the special and general theory" (the latter requires actual understanding
whereas the former and this just require math).

\section{Derive Lorentz Transformation requiring invariance of the wave equation}

Now, lets look at a general change of variables for the wave equation for the electromagnetic field.  This will include
the Galilean transformation, as well as the Lorentz transformation above, as special cases.

Consider a two variable, scaled Laplacian:
%
\begin{equation}
\laplacian = m \frac{\partial^2}{\partial u^2} + n \frac{\partial^2}{\partial v^2}
\end{equation}
%
and a linear change of variables defined by:
%
\begin{equation}
\begin{bmatrix}
u \\
v \\
\end{bmatrix}
=
\begin{bmatrix}
e & f \\
g & h \\
\end{bmatrix}
\begin{bmatrix}
x \\
y \\
\end{bmatrix}
=
A
\begin{bmatrix}
x \\
y \\
\end{bmatrix}
\end{equation}
%
To perform the change of variables we need to evaluate the following:
%
\begin{equation}\label{eqn:lorentz:260}
\begin{aligned}
\frac{\partial}{\partial u}
&= \frac{\partial x}{\partial u} \frac{\partial}{\partial x} + \frac{\partial y}{\partial u} \frac{\partial}{\partial y} \\
\frac{\partial}{\partial v}
&= \frac{\partial x}{\partial v} \frac{\partial}{\partial x} + \frac{\partial y}{\partial v} \frac{\partial}{\partial y}
\end{aligned}
\end{equation}
%
To compute the partials we must invert \(A\).  Writing
%
\begin{equation}\label{eqn:lorentz:40}
J =
\begin{vmatrix}
e & f \\
g & h \\
\end{vmatrix}
=
\frac{\partial(u,v)}{\partial(x,y)},
\end{equation}
%
that inverse is
%
\begin{equation}\label{eqn:lorentz:60}
A^{-1} =
\inv
{
\begin{vmatrix}
e & f \\
g & h \\
\end{vmatrix}
}
\begin{bmatrix}
h & -f \\
-g & e \\
\end{bmatrix}.
\end{equation}
%
The first partials are therefore:
%
\begin{equation}\label{eqn:lorentz:280}
\begin{aligned}
\frac{\partial}{\partial u}
&= \inv{J} \left(h \frac{\partial}{\partial x} - g \frac{\partial}{\partial y}\right) \\
\frac{\partial}{\partial v}
&= \inv{J} \left(-f \frac{\partial}{\partial x} + e \frac{\partial}{\partial y}\right).
\end{aligned}
\end{equation}
%
Repeating for the second partials yields:
%
\begin{equation}\label{eqn:lorentz:300}
\begin{aligned}
\frac{\partial^2}{\partial u^2}
&= \inv{J^2} \left(
h^2 \frac{\partial^2}{\partial x^2} + g^2 \frac{\partial^2}{\partial y^2}
-g h \left( \frac{\partial^2}{\partial x \partial y} + \frac{\partial^2}{\partial y \partial x} \right)
\right) \\
\frac{\partial^2}{\partial v^2}
&= \inv{J^2} \left(
f^2 \frac{\partial^2}{\partial x^2} + e^2 \frac{\partial^2}{\partial y^2}
-e f \left( \frac{\partial^2}{\partial x \partial y} + \frac{\partial^2}{\partial y \partial x} \right)
\right)
\end{aligned}
\end{equation}
%
That is the last calculation required to compute the transformed Laplacian:
%
\begin{equation}\label{eqn:Lor:laplaciantx}
\laplacian = \inv{J^2}
\left(
(m h^2 + n f^2)\partial_{xx}
+(m g^2 + n e^2)\partial_{yy}
-(m g h + n e f)( \partial_{xy} + \partial_{yx}  )
\right)
\end{equation}
%
\subsection{Galilean transformation}
\index{Galilean transformation}

Lets apply this to the electrodynamics wave equation, first using a Galilean transformation \(x = x' + v t\), \(t = t'\), \(\beta = v/c\).
%
\begin{equation}
\begin{bmatrix}
x \\
ct \\
\end{bmatrix}
=
\begin{bmatrix}
1 & \beta \\
0 & 1 \\
\end{bmatrix}
\begin{bmatrix}
x' \\
c t' \\
\end{bmatrix}
\end{equation}
%
\begin{equation}
\partial_{xx} -\inv{c^2}\partial_{tt} =
(1 - \beta^2)\partial_{x'x'}
-\inv{c^2}\partial_{t't'}
+\inv{c}\beta( \partial_{x't'} + \partial_{t'x'} )
\end{equation}
%
Thus we see that the equations of light when subjected to a Galilean transformation have a different form after such a
transformation.  If this was correct we should see the effects of the mixed product terms and the reduced effect of the
spatial component when there is any motion.  However, light comes in a wave form regardless of motion, so there
is something wrong with the application of this transformation to the equations of light.  This was the big problem of physics
over a hundred years ago before Einstein introduced relativity to explain all this.

\subsection{Determine the transformation of coordinates that retains the form of the equations of light}

Before Einstein, Lorentz worked out the transformation that left Maxwell's equation ``invariant''.  I have not seen
any text that actually showed this.  Lorentz may have showed that his transformations left Maxwell's equations
invariant in their full generality, however that complexity is not required to derive the transformation itself.  Instead
this can be done considering only the wave equation for light in source free space.

Let us define the matrix \(A\) for a general change of space and time variables in one spatial dimension:
%
\begin{equation}
\begin{bmatrix}
x \\
ct \\
\end{bmatrix}
=
\begin{bmatrix}
e & f \\
g & h \\
\end{bmatrix}
\begin{bmatrix}
x' \\
c t' \\
\end{bmatrix}
\end{equation}
%
Application of this to \eqnref{eqn:Lor:laplaciantx} gives:
%
\begin{equation}\label{eqn:Lor:transformedlightwave}
\partial_{xx} -\partial_{ct,ct} =
\inv{J^2}
\left(
(h^2 - f^2)\partial_{x'x'}
+(g^2 - e^2)\partial_{ct',ct'}
-(g h - e f)( \partial_{x',ct'} + \partial_{ct',x'}  )
\right)
\end{equation}
%
Now, we observe that light has wavelike behavior regardless of our velocity (we do observe frequency variation with
velocity but the fundamental waviness does not change).  Once that is accepted as a requirement for a transformation
of coordinates of the wave equation for light we get the Lorentz transformation.

Expressed mathematically, this means that we want \eqnref{eqn:Lor:transformedlightwave} to have the form:
%
\begin{equation}
\partial_{xx} -\partial_{ct,ct} = \partial_{x'x'} -\partial_{ct',ct'}
\end{equation}
%
This requirement is equivalent to the following system of equations:
%
\begin{equation}\label{eqn:lorentz:320}
\begin{aligned}
J &= eh - fg \\
h^2 - f^2 &= J^2 \\
g^2 - e^2 &= -J^2 \\
g h &= e f.
\end{aligned}
\end{equation}
%
Attempting to solve this in full generality for any J gets messy (ie: non-linear).
To simplify things, it is not unreasonable to
require \(J = 1\), which is consistent with Galilean transformation, in particular for the limiting case as \(v \rightarrow 0\).

Additionally, we want to give physical significance to these values \(e,f,g,h\).  Following Einstein's simple derivation
of the Lorentz transformation, we do this by defining \(x'=0\) as the origin of the moving frame:
%
\begin{equation}\label{eqn:lorentz:80}
x'
=
\inv{J}
\begin{bmatrix}
h & -f \\
\end{bmatrix}
\begin{bmatrix}
x \\
c t \\
\end{bmatrix}
= 0
\end{equation}
%
This allows us to relate \(f,h\) to the velocity:
%
\begin{equation}\label{eqn:lorentz:100}
x h = f c t
\end{equation}
\begin{equation}\label{eqn:lorentz:120}
\implies
\frac{dx}{dt} = \frac{f c}{h} = v,
\end{equation}
%
and provides physical meaning to the first of the elements of the linear transformation:
%
\begin{equation}\label{eqn:Lor:betaterm}
f = h \frac{v}{c} = h \beta.
\end{equation}
%
The significance and values of \(e,g,h\) remain to be determined.  Substituting \eqnref{eqn:Lor:betaterm} into our
system of equations we have:
%
\begin{equation}\label{eqn:lorentz:340}
\begin{aligned}
h^2 - h^2 \beta^2 &= 1 \\
g^2 - e^2 &= -1 \\
g h &= e h \beta.
\end{aligned}
\end{equation}
%
From the first equation we have \(h^2 = \inv{1 -\beta^2}\), which is what is usually designated \(\gamma^2\).  Considering
the limiting case again of \(v \rightarrow 0\), we want to take the positive root.  Summarizing what has been found
so far we have:
%
\begin{equation}\label{eqn:lorentz:360}
\begin{aligned}
h &= \inv{\sqrt{1 - \beta^2}} = \gamma \\
f &= \gamma \beta \\
g^2 - e^2 &= -1 \\
g &= e \beta.
\end{aligned}
\end{equation}
%
Substitution of the last yields
%
\begin{equation}\label{eqn:lorentz:140}
e^2 (\beta^2 - 1) = -1
\end{equation}
%
which means that \(e^2 = \gamma^2\), or \(e = \gamma\), and \(g = \gamma \beta\) (again taking the positive root to avoid a reflective transformation in the limiting case).  This completely specifies the linear
transformation required to maintain the wave equation in wave equation form after a change of variables that includes
a velocity transformation in one direction:
%
\begin{equation}
\begin{bmatrix}
x \\
ct \\
\end{bmatrix}
=
\gamma
\begin{bmatrix}
1 & \beta \\
\beta & 1 \\
\end{bmatrix}
\begin{bmatrix}
x' \\
c t' \\
\end{bmatrix}
\end{equation}
%
Inversion of this yields the typical one dimensional Lorentz transformation where the position and time of a moving
frame is specified in terms of the inertial frame:
%
\begin{equation}
\begin{bmatrix}
x' \\
ct' \\
\end{bmatrix}
=
\gamma
\begin{bmatrix}
1 & -\beta \\
-\beta & 1 \\
\end{bmatrix}
\begin{bmatrix}
x \\
c t \\
\end{bmatrix}.
\end{equation}
%
That is perhaps more evident when this is written out explicitly in terms of velocity:
%
\begin{equation}\label{eqn:lorentz:380}
\begin{aligned}
x' &= \frac{x - v t}{\sqrt{1 - v^2/c^2}} \\
t' &= \frac{t - (v/c^2) x}{\sqrt{1 - v^2/c^2}}.
\end{aligned}
\end{equation}
%
\section{Light sphere, and relativistic metric}

TBD.

My old E\&M book did this derivation using a requirement for invariance of a spherical light shell.  ie:

\(x^2 - c^2 t^2 = {x'}^2 - c^2 {t'}^2\).

That approach requires less math (ie: to partial derivatives or change of variables), but I never understood why that invariance condition could be assumed (to understand that intuitively, you have to understand the constancy of light phenomena, which has a few subtleties that are not obvious in my opinion).

I like my approach, which has a bit more math, but I think is easier (vs. light constancy) to conceptualize the idea that light is wavelike regardless of the motion of the observer since it (ie: an electrodynamic field) must satisfy the wave equation (ie: Maxwell's equations).  I am curious if somebody else also new to the subject of relativity would agree?

\section{Derive relativistic Doppler shift}

TBD.

This is something I think would make sense to do considering solutions
to the wave
equation instead of utilizing more abstract wave number, and frequency
four vector concepts.  Have not yet done the calculations for this part.

   %
% Copyright � 2012 Peeter Joot.  All Rights Reserved.
% Licenced as described in the file LICENSE under the root directory of this GIT repository.
%

%
%
\mychapter{Equations of motion given mass variation with spacetime position}
\label{chap:massVaryLagrangian}
%\date{August 28, 2008}        % Deleting this command produces today's date.

\section{}

Let
\begin{equation}\label{eqn:massVaryLagrangian:20}
\begin{aligned}
x &= \sum \gamma_{\mu} {x}^{\mu} \\
v &= \frac{dx}{d\tau} = \sum \gamma_{\mu} \xdot^{\mu}
\end{aligned}
\end{equation}

Where whatever spacetime basis you pick has a corresponding reciprocal frame defined implicitly by:

\begin{equation*}
\gamma^{\mu} \cdot \gamma_{\nu} = {\delta^{\mu}}_{\nu}
\end{equation*}

You could for example pick these so that these are orthonormal with:

\begin{equation}\label{eqn:massVaryLagrangian:40}
\begin{aligned}
\gamma_{i}^2 &= \gamma_i \cdot \gamma_i = -1 \\
\gamma^{i} &= -\gamma_{i} \\
\gamma^{0} &= \gamma_{0} \\
\gamma_{0}^2 &= 1 \\
\gamma_{i} \cdot \gamma_0 &= 0
\end{aligned}
\end{equation}

ie: the frame vectors define the metric tensor implicitly:

\begin{equation}\label{eqn:massvary:minkowski}
g_{\mu\nu} = \gamma_{\mu} \cdot \gamma_{\nu} =
\begin{bmatrix}
1 & 0 & 0 & 0 \\
0 & -1 & 0 & 0 \\
0 & 0 & -1 & 0 \\
0 & 0 & 0 & -1 \\
\end{bmatrix}
\end{equation}

Now, my assumption is that given a Lagrangian of the form:

\begin{equation}
\LL = \inv{2} m v^2 + \phi
\end{equation}

That the equations of motion follow by computation of:

\begin{equation}
\PD{x^{\mu}}{\LL} = \frac{d}{d\tau} \PD{\xdot^{\mu}}{\LL}
\end{equation}

I do not have any proof of this (I do not yet know any calculus of variations, and this is a guess based on intuition).  It does however work out to get the covariant form of the Lorentz force law, so I think it is right.

To get the EOM we need the squared proper velocity.  This is just \(c^2\).  Example: for an orthonormal spacetime frame one has:

\begin{equation}\label{eqn:massVaryLagrangian:60}
\begin{aligned}
v^2 &=
\left(\gamma^0 c dt/d\tau + \sum \gamma_i dx/d\tau\right)^2  \\
&= \gamma \left(\gamma_0 c + \sum \gamma_i dx/dt\right)^2 \\
&= \gamma^2 \left(c^2 - \Bv^2\right) = c^2
\end{aligned}
\end{equation}

but if we leave this expressed in terms of coordinates (also do not have to assume the diagonal metric tensor, since we can use non-orthonormal basis vectors if desired) we have:

\begin{equation}\label{eqn:massVaryLagrangian:80}
\begin{aligned}
v^2
&= \left(\sum \gamma_{\mu} \xdot^{\mu}\right) \cdot \left(\sum \gamma_{\nu} \xdot^{\nu}\right) \\
&= \sum \gamma_{\mu} \cdot \gamma_{\nu} \xdot^{\mu} \xdot^{\nu} \\
&= \sum g_{\mu\nu} \xdot^{\mu} \xdot^{\nu}
\end{aligned}
\end{equation}

Therefore the Lagrangian to minimize is:

\begin{equation}
\LL = \inv{2} m \sum g_{\mu\nu} \xdot^{\mu} \xdot^{\nu} + \phi.
\end{equation}

Performing the calculations for the EOM, and in this case, also allowing mass to be a function of space or time position (\(m = m(x^{\mu})\))

\begin{equation}\label{eqn:massVaryLagrangian:100}
\begin{aligned}
\PD{x^{\mu}}{\LL} &= \frac{d}{d\tau} \PD{\xdot^{\mu}}{\LL} \\
\PD{x^{\mu}}{\phi} + \inv{2} \PD{x^{\mu}}{m} \sum g_{\alpha\beta} \xdot^{\alpha} \xdot^{\beta} &= \\
\PD{x^{\mu}}{\phi} + \inv{2} \PD{x^{\mu}}{m} v^2 &= \\
%\PD{x^{\mu}}{\phi} + \inv{2} \PD{x^{\mu}}{m} c^2 &= \\
&= \inv{2} \frac{d}{d\tau} m \sum g_{\alpha\beta} \PD{x^{\mu}}{}\left(\xdot^{\alpha} \xdot^{\beta}\right) \\
&= \inv{2} \frac{d}{d\tau} m \sum g_{\alpha\beta} \left(\delta^{\mu\alpha} \xdot^{\beta} + \xdot^{\alpha} \delta^{\mu\beta}\right) \\
&= \frac{d}{d\tau} m \sum g_{\alpha\mu} \xdot^{\alpha} \\
&= \sum \PD{x^{\beta}}{m} \xdot^{\beta} g_{\alpha\mu} \xdot^{\alpha} + m g_{\alpha\mu} \xddot^{\alpha} \\
\end{aligned}
\end{equation}

Now, the metric tensor values can be removed by summing since they can be used to switch upper and lower indices of the frame vectors:

\begin{equation}\label{eqn:massVaryLagrangian:120}
\begin{aligned}
\gamma_{\mu} &= \sum a^{\nu} \gamma^{\nu} \\
\gamma_{\mu} \cdot \gamma_{\beta}
&= \sum a^{\nu} \gamma^{\nu} \cdot \gamma_{\beta} \\
&= \sum a^{\nu} {\delta^{\nu}}_{\beta} \\
&= a^{\beta} \\
\implies \\
\gamma_{\mu}
&= \sum \gamma_{\mu} \cdot \gamma_{\nu} \gamma^{\nu} \\
&= \sum g_{\mu\nu} \gamma^{\nu} \\
\end{aligned}
\end{equation}

If you are already familiar with tensors then this may be obvious to you (but was not to me with only vector background).

Multiplying throughout by \(\gamma^{\mu}\), and summing over \(\mu\) one has:

\begin{equation}\label{eqn:massVaryLagrangian:140}
\begin{aligned}
\sum \gamma^{\mu} \left( \PD{x^{\mu}}{\phi} + \inv{2} \PD{x^{\mu}}{m} v^2 \right)
&= \sum \gamma^{\mu} \left(\PD{x^{\beta}}{m} \xdot^{\beta} g_{\alpha\mu} \xdot^{\alpha} + m g_{\alpha\mu} \xddot^{\alpha} \right) \\
+ \left(\sum \gamma^{\mu} \PD{x^{\mu}}{}\right) \phi + \inv{2} v^2 \left(\sum \gamma^{\mu} \PD{x^{\mu}}{}\right) m &= \\
&= \sum \PD{x^{\beta}}{m} \xdot^{\beta} \gamma^{\mu} \gamma_{\alpha} \cdot \gamma_{\mu} \xdot^{\alpha} + m \gamma^{\mu} \gamma_{\alpha} \cdot \gamma_{\mu} \xddot^{\alpha}  \\
&= \sum \PD{x^{\beta}}{m} \xdot^{\beta} \gamma_{\alpha} \xdot^{\alpha} + m \gamma_{\alpha} \xddot^{\alpha}  \\
\end{aligned}
\end{equation}

Writing:
\begin{equation*}
\nabla = \sum \gamma^{\mu} \frac{\partial}{\partial x^{\mu}}
\end{equation*}

This is:
\begin{equation*}
\grad \phi + \inv{2} v^2 \grad m = v \sum \PD{x^{\beta}}{m} \xdot^{\beta} + m \vdot
\end{equation*}

However,
\begin{equation}\label{eqn:massVaryLagrangian:160}
\begin{aligned}
(\grad m) \cdot v
&=
\left(\sum \gamma^{\mu} \PD{x^{\mu}}{m}\right) \cdot \left( \sum \gamma_{\nu} \xdot^{\nu} \right) \\
&= \sum \gamma^{\mu} \cdot \gamma_{\nu} \PD{x^{\mu}}{m} \xdot^{\nu} \\
&= \sum {\delta^{\mu}}_{\nu} \PD{x^{\mu}}{m} \xdot^{\nu} \\
&= \sum \PD{x^{\mu}}{m} \xdot^{\mu} = \frac{dm}{d\tau}
\end{aligned}
\end{equation}

That allows for expressing the EOM in strict vector form:
\begin{equation}
\grad \phi + \inv{2} v^2 \grad m = v \grad m \cdot v + m \vdot.
\end{equation}

However, there is still an asymmetry here, as one would expect a \(\mdot v\) term.  Regrouping slightly, and using some algebraic vector
manipulation we have:

\begin{equation}\label{eqn:massVaryLagrangian:180}
\begin{aligned}
m \vdot + v \grad m \cdot v - \inv{2} v^2 \grad m &= \grad \phi \\
m \vdot + \inv{2} v (
\mathLabelBox
[
   labelstyle={xshift=2cm},
   linestyle={out=270,in=90, latex-}
]
{2 \grad m \cdot v - v \grad m}{\(2 a \cdot b - b a = a b\)}) &= \\
m \vdot + \inv{2} v (\grad m) v &= \\
m \vdot + \inv{2} (v \grad m) v &= \\
m \vdot + \inv{2} (2 v \cdot \grad m - \grad m v) v &= \\
m \vdot + (v \cdot \grad m) v - \inv{2} (\grad m v) v &= \\
m \vdot + \mdot v - \inv{2} \grad m (v v) &= \\
\implies \\
\frac{d (m v)}{d\tau} = m \vdot + \mdot v
&= \inv{2} \grad m c^2 +\grad \phi \\
&= \grad \left(\phi - \inv{2} m c^2 \right) \\
&= \grad \left(\phi - \inv{2} m v^2 \right) \\
\end{aligned}
\end{equation}

So, after a whole wack of algebra, the end result is to show the proper time variant of the Lagrangian equations imply that our
proper force can be expressed as a (spacetime) gradient.

The caveat is that if the mass is allowed to vary, it also needs to be
included in the generalized potential associated with the equation of motion.

\subsection{Summarizing}

We took this Lagrangian with kinetic energy and non-velocity dependent potential terms, where the
mass in the kinetic energy term is allowed to vary with position or time.  That plus the
presumed proper-time Lagrange equations:

\begin{equation}\label{eqn:massVaryLagrangian:200}
\begin{aligned}
\LL &= \inv{2} m v^2 + \phi \\
\PD{x^{\mu}}{\LL} &= \frac{d}{d\tau} \PD{\xdot^{\mu}}{\LL},
\end{aligned}
\end{equation}

when followed to their algebraic conclusion together imply that the equation of motion is:

\begin{equation}\label{eqn:massvary:eom}
\frac{d (m v)}{d\tau} = \grad \LL,
\end{equation}

\section{Examine spatial components for comparison with Newtonian limit}

Now, in the original version of this document, the signs for all the \(\phi\) terms were inverted.  This was changed since we want agreement with the Newtonian limit, and there is an implied sign change hiding in the above equations.

Consider, the constant mass case, where the Lagrangian is specified in terms of spatial quantities:

\begin{equation*}
\LL = \inv{2} m v^2 + \phi = \inv{2} m \gamma^2 ( c^2 - \Bv^2 ) = \inv{2} m \gamma^2 c^2 - \gamma^2\left( \inv{2} m \Bv^2 - \phi \right)
\end{equation*}

For \(\abs{\Bv} << c\), \(\gamma \approx 1\), so we have a constant term in the Lagrangian of \(\inv{2} m c^2\) which will not change the
EOM and can be removed.  The remainder is our normal kinetic minus potential Lagrangian (the sign inversion on the entire remaining Lagrangian also will not change the EOM result).

Suppose one picks an orthonormal
spacetime frame as given in the example metric tensor of \eqnref{eqn:massvary:minkowski}.  To select our spatial quantities
we wedge with \(\gamma_0\).

For the left hand side of our equation of motion \eqnref{eqn:massvary:eom} we have:

\begin{equation}\label{eqn:massVaryLagrangian:220}
\begin{aligned}
\frac{d (m v)}{d\tau} \wedge \gamma_0
&= \frac{ d (m v) \wedge \gamma_0 }{dt} \frac{dt}{d\tau} \\
&= \frac{ d p \wedge \gamma_0 }{dt} \frac{dt}{d\tau} \\
&= \frac{dt}{d\tau} \frac{ d }{dt} m (c \gamma_0 + \sum \gamma_i \xdot^i ) \wedge \gamma_0 \\
&= \frac{dt}{d\tau} \frac{ d }{dt} m \sum (\gamma_i \wedge \gamma_0) \xdot^i  \\
&= \frac{dt}{d\tau} \frac{ d }{dt} m \sum \sigma_i \xdot^i  \\
&= \frac{dt}{d\tau} \frac{ d }{dt} (m \Bv \gamma) \\
&= \gamma \frac{ d (\gamma \Bp) }{dt}
\end{aligned}
\end{equation}

Now, looking at the right hand side of the EOM we have (again for the constant mass case where we expect agreement with our familiar Newtonian EOM):

\begin{equation}\label{eqn:massVaryLagrangian:240}
\begin{aligned}
\grad \left(\phi - \inv{2} m v^2 \right) \wedge \gamma_0
&= (\grad \phi) \wedge \gamma_0 \\
&= \sum \gamma^{\mu} \wedge \gamma_0 \PD{x^{\mu}}{\phi} \\
&= \sum \gamma^{i} \wedge \gamma_0 \PD{x^{i}}{\phi} \\
&= -\sum \gamma_{i} \wedge \gamma_0 \PD{x^{i}}{\phi} \\
&= -\sum \sigma_i \PD{x^{i}}{\phi} \\
&= - \spacegrad \phi
\end{aligned}
\end{equation}

Therefore in the limit \(\abs{\Bv} << c\) we have our agreement with the Newtonian EOM:

\begin{equation}
\gamma \frac{ d (\gamma \Bp) }{dt} = - \spacegrad \phi \approx \frac{d\Bp}{dt}
\end{equation}

   %
% Copyright � 2012 Peeter Joot.  All Rights Reserved.
% Licenced as described in the file LICENSE under the root directory of this GIT repository.
%

%
%
\mychapter{Understanding four velocity transform from rest frame}
\index{four velocity}
\index{boost}
\index{rest frame}
\label{chap:velocityTx}
%\date{August 13, 2008}

\section{}

\citep{doran2003gap} writes \(v = R \gamma_0 R^\dagger\), as a proper velocity expressed in terms of a rest frame velocity and a Lorentz boost.
This was not clear to me, and would probably be a lot more
obvious to me if I had fully read chapter 5, but in my defense it is a hard read without first getting
more familiarity with basic relativity.

Let us just expand this out
to see how this works.  First thing to note is that there is an omitted factor
of \(c\), and I will add that back in here, since I am not comfortable enough
without it explicitly for now.

With:
%
\begin{equation}\label{eqn:velocityTx:20}
\begin{aligned}
\Bv/c &= \tanh\left(\alpha\right)\vcap \\
R &= \exp\left(\alpha \vcap/2\right)
\end{aligned}
\end{equation}
%
We want to expansion this Lorentz boost exponential (see details section) and apply it to the rest frame basis vector.  Writing
\(C = \cosh\left(\alpha/2\right)\), and \(S = \sinh\left(\alpha/2\right)\), we have:
%
\begin{equation}\label{eqn:velocityTx:40}
\begin{aligned}
v
&= R \left(c \gamma_0\right) R^\dagger \\
&= c \left(C + \vcap S\right) \gamma_0 \left(C - \vcap S\right) \\
&= c \left(C \gamma_0 + S \vcap \gamma_0\right) \left(C - \vcap S\right) \\
&= c \left( C^2 \gamma_0 + SC \vcap \gamma_0 -CS \gamma_0\vcap - S^2 \vcap \gamma_0 \vcap \right) \\
\end{aligned}
\end{equation}
%
Now, here things can start to get confusing since \(\vcap\) is a spatial quantity with vector-like spacetime basis bivectors \(\sigma_i = \gamma_i \gamma_0\).  Factoring out the \(\gamma_0\) term, utilizing the fact that \(\gamma_0\) and \(\sigma_i\) anticommute (see below).
%
\begin{equation}\label{eqn:velocityTx:60}
\begin{aligned}
v
&= c \left( C^2 + S^2 + 2 SC \vcap \right) \gamma_0 \\
&= c \left( \cosh\left(\alpha\right) + \vcap \sinh\left(\alpha\right) \right) \gamma_0 \\
&= c \cosh\left(\alpha\right) \left( 1 + \vcap \tanh\left(\alpha\right) \right) \gamma_0 \\
&= c \cosh\left(\alpha\right) \left( 1 + \Bv/c \right) \gamma_0 \\
&= c \gamma \left( 1 + \Bv/c \right) \gamma_0 \\
&= \gamma \left( c \gamma_0 + \sum v^i \gamma_i\right) \\
&= \frac{dt}{d\tau}\left( c \gamma_0 + \sum v^i \gamma_i\right) \\
&= \frac{dt}{d\tau} \frac{d}{dt}\left( c t \gamma_0 + \sum x^i \gamma_i\right) \\
&= \frac{dt}{d\tau} \frac{d}{dt} \sum x^{\mu} \gamma_{\mu} \\
&= \frac{d}{d\tau} \sum x^{\mu} \gamma_{\mu} \\
&= \frac{dx}{d\tau}
\end{aligned}
\end{equation}
%
So, we get the end result that demonstrates that a Lorentz boost applied to the rest event vector \(x = x^0 \gamma_0 = c t \gamma_0\) directly produces the four velocity for the motion from the new viewpoint.  This makes some intuitive sense, but
I do not feel this is necessarily obvious without demonstration.

This also explains how the text is able to use the wedge and dot product ratios with the \(\gamma_0\) basis vector
to produce the relative spatial velocity.  If one introduces a rest frame proper velocity of
\(w = \frac{d}{dt}\left(ct \gamma_0\right) = c \gamma_0\), then one has:
%
\begin{equation}\label{eqn:velocityTx:80}
\begin{aligned}
v \cdot w
&= \left(\sum \frac{d x^{\mu}}{d\tau} \gamma_{\mu}\right) \cdot \left(c\gamma_0\right) \\
&= c^2 \gamma
\end{aligned}
\end{equation}
%
\begin{equation}\label{eqn:velocityTx:100}
\begin{aligned}
v \wedge w
&= \left(\sum \frac{d x^{\mu}}{d\tau} \gamma_{\mu}\right) \wedge \left(c\gamma_0\right) \\
&= \left(\sum \frac{d x^{i}}{d\tau} \gamma_{i}\right) \wedge \left(c\gamma_0\right) \\
&= c \sum \frac{d x^{i}}{d\tau} \sigma_{i} \\
&= c \frac{dt}{d\tau} \sum \frac{d x^{i}}{dt} \sigma_{i} \\
&= c \gamma \sum \frac{d x^{i}}{dt} \sigma_{i} \\
\end{aligned}
\end{equation}
%
Combining these one has the spatial observer dependent relative velocity:
%
\begin{equation}
\frac{v \wedge w}{v \cdot w} = \inv{c} \sum \frac{d x^{i}}{dt} \sigma_{i} = \frac{\Bv}{c}
\end{equation}
%
\subsection{Invariance of relative velocity?}

What is not clear to me is whether this can be used to determine the relative velocity between two particles in the general case, when one of them is not a rest frame velocity (time progression only at a fixed point in space.)
The text seems
to imply this is the case, so perhaps it is obvious to them only and not me;)

This can be verified relatively easily for the extreme case, where one boosts both the \(w\), and \(v\) velocities to measure \(v\) in its rest frame.

Expressed mathematically this is:
%
\begin{equation}\label{eqn:velocityTx:120}
\begin{aligned}
w &= c \gamma_0 \\
v &= R w R^\dagger \\
v' &= R^\dagger v R = R^\dagger R c \gamma_0 R^\dagger R = c \gamma_0 \\
w' &= R^\dagger w R \\
\end{aligned}
\end{equation}
%
Now, this last expression for \(w'\) can be expanded brute force as was done initially to calculate \(v\) (and I in
fact did that initially without thinking).  The end result matches what should have been the intuitive expectation, with the velocity components all negated in a conjugate like fashion:
%
\begin{equation*}
w' = \gamma\left( c\gamma_0 - \sum v^i \gamma_i \right)
\end{equation*}
%
With this result we have:
%
\begin{equation*}
v' \cdot w' = c \gamma_0 \cdot \gamma\left( c\gamma_0 - \sum v^i \gamma_i \right) = \gamma c^2
\end{equation*}
%
\begin{equation}\label{eqn:velocityTx:140}
\begin{aligned}
v' \wedge w'
&= c \gamma_0 \wedge \gamma\left( c\gamma_0 - \sum v^i \gamma_i \right) \\
&= -c \gamma \sum v^i \gamma_0 \gamma_i \\
&= c \gamma \sum v^i \sigma_i \\
\end{aligned}
\end{equation}
%
Dividing the two we have the following relative velocity between the two proper velocities:
%
\begin{equation*}
\frac{v' \wedge w'}{v' \cdot w'} = \inv{c} \sum v^i \sigma_i = \Bv/c.
\end{equation*}
%
Lo and behold, this is the same as when the first event worldline was in its rest frame, so we have the same
relative velocity regardless of which of the two are observed at rest.  The remaining obvious question is
how to show that this is a general condition, assuming that it is.

\subsection{General invariance?}

Intuitively, I would guess that this is fact the case because when only two particles are considered, the result should be the same independent of which of the
two is considered at rest.

Mathematically, I would express this statement by saying that if one has
a Lorentz boost that takes \(v' = T v T^\dagger\) to its rest frame, then application of this to both proper velocities leaves both the wedge and dot product
parts of this ratio unchanged:
%
\begin{equation}\label{eqn:velocityTx:160}
\begin{aligned}
v \cdot w
&= \left(T^\dagger v' T\right) \cdot \left(T^\dagger w' T\right) \\
&= \gpgradezero{\left(T^\dagger v' T\right) \left(T^\dagger w' T\right)} \\
&= \gpgradezero{T^\dagger v' w' T} \\
&= \gpgradezero{T^\dagger v' \cdot w' T} +
\mathLabelBox
[
   labelstyle={xshift=2cm},
   linestyle={out=270,in=90, latex-}
]
{\gpgradezero{T^\dagger v' \wedge w' T}}{\(=0\)} \\
&= \left(v' \cdot w'\right)\gpgradezero{T^\dagger T} \\
&= v' \cdot w'
\end{aligned}
\end{equation}
%
\begin{equation}\label{eqn:velocityTx:180}
\begin{aligned}
v \wedge w
&= \left(T^\dagger v' T\right) \wedge \left(T^\dagger w' T\right) \\
&= \gpgradetwo{\left(T^\dagger v' T\right) \left(T^\dagger w' T\right)} \\
&= \gpgradetwo{T^\dagger v' w' T} \\
&=
\mathLabelBox
[
   labelstyle={xshift=2cm},
   linestyle={out=270,in=90, latex-}
]
{\gpgradetwo{T^\dagger v' \cdot w' T}}{\(=0\)} + \gpgradetwo{T^\dagger v' \wedge w' T} \\
&= T^\dagger \left(v' \wedge w'\right) T
\end{aligned}
\end{equation}
%
FIXME: can not those last \(T\) factors be removed somehow?

\section{Appendix. Omitted details from above}

\subsection{exponential of a vector}

Understanding the vector exponential is a prerequisite above.  This is defined
and interpreted by series expansion as with matrix exponentials.
Expanding
in series the exponential of a vector \(\Bx = x\xcap\), we have:
%
\begin{equation}\label{eqn:velocityTx:200}
\begin{aligned}
\exp\left(\Bx\right)
&= \sum \frac{\Bx^{2k}}{\left(2k\right)!} + \sum \frac{\Bx^{2k+1}}{\left(2k+1\right)!} \\
&= \sum \frac{x^{2k}}{\left(2k\right)!} + \xcap \sum \frac{x^{2k+1}}{\left(2k+1\right)!} \\
&= \cosh\left(x\right) + \xcap \sinh\left(x\right)
\end{aligned}
\end{equation}
%
Notationally this can also be written:
%
\begin{equation*}
\exp\left(\Bx\right) = \cosh\left(\Bx\right) + \sinh\left(\Bx\right)
\end{equation*}
%
But doing so will not really help.

\subsection{\texorpdfstring{\(\Bv\)}{v} anticommutes with \texorpdfstring{\(\gamma_0\)}{gamma 0}}
%
\begin{equation}\label{eqn:velocityTx:220}
\begin{aligned}
\Bv \gamma_0
&= \sum v^i \sigma_i \gamma_0 \\
&= \sum v^i \gamma_i \gamma_0 \gamma_0 \\
&= -\sum v^i \gamma_0 \gamma_i \gamma_0 \\
&= - \gamma_0 \sum v^i \gamma_i \gamma_0 \\
&= - \gamma_0 \sum v^i \sigma_0 \\
&= - \gamma_0 \Bv
\end{aligned}
\end{equation}

   %
% Copyright � 2012 Peeter Joot.  All Rights Reserved.
% Licenced as described in the file LICENSE under the root directory of this GIT repository.
%

%
%
\chapter{Four vector dot product invariance and Lorentz rotors}
\index{Lorentz invariance}
\label{chap:fourvecDotinvariance}
%\date{August 1, 2008}        % Deleting this command produces today's date.
\section{}

Prof. Ramamurti Shankar's
In the relativity lectures of
\citep{ShankarPhy200} Prof. Shankar
indicates that the
four vector dot product
is a Lorentz invariant.  This makes some logical sense, but lets demonstrate it explicitly.

Start with a Lorentz transform matrix between coordinates for two four vectors (omitting the components perpendicular  to the motion) :

\begin{equation*}
{
\begin{bmatrix}
x^1 \\
x^0 \\
\end{bmatrix}
}'
=
\gamma
\begin{bmatrix}
1 & -\beta \\
-\beta & 1
\end{bmatrix}
\begin{bmatrix}
x^1 \\
x^0 \\
\end{bmatrix}
\end{equation*}

\begin{equation*}
{
\begin{bmatrix}
y^1 \\
y^0 \\
\end{bmatrix}
}'
=
\gamma
\begin{bmatrix}
1 & -\beta \\
-\beta & 1
\end{bmatrix}
\begin{bmatrix}
y^1 \\
y^0 \\
\end{bmatrix}
\end{equation*}

Now write out the dot product between the two vectors given the perceived length and time measurements for the same events in the moving frame:

\begin{equation}\label{eqn:fourvecDotinvariance:20}
\begin{aligned}
X' \cdot Y'
&= \gamma^2 \left( (-\beta x^1 + x^0)(-\beta y^1 + y^0) -(x^1 -\beta x^0) (y^1 -\beta y^0) \right) \\
&= \gamma^2 \left( (\beta^2 x^1 y^1 + x^0 y^0) + x^0 y^1( -\beta + \beta ) + x^1 y^0( -\beta + \beta ) -(x^1 y^1 + \beta^2 x^0 y^0) \right) \\
&= \gamma^2 \left( x^0 y^0 (1-\beta^2) - (1-\beta^2) x^1 y^1 \right) \\
&= x^0 y^0 - x^1 y^1 \\
&= X \cdot Y
\end{aligned}
\end{equation}

This completes the proof of dot product Lorentz invariance.  An automatic consequence of this is invariance
of the Minkowski length.

\subsection{Invariance shown with hyperbolic trig functions}

Dot product or length invariance can also be shown with the hyperbolic representation of the Lorentz transformation:

\begin{equation}\label{eqn:fVecDotInv:hyperbolicmatrix}
{
\begin{bmatrix}
x^1 \\
x^0 \\
\end{bmatrix}
}'
=
\begin{bmatrix}
\cosh(\alpha) & -\sinh(\alpha) \\
-\sinh(\alpha) & \cosh(\alpha)
\end{bmatrix}
\begin{bmatrix}
x^1 \\
x^0 \\
\end{bmatrix}
\end{equation}

Writing \(S=\sinh(\alpha)\), and \(C=\cosh(\alpha)\) for short, this gives:

\begin{equation}\label{eqn:fourvecDotinvariance:40}
\begin{aligned}
X' \cdot Y'
&= \left( (-S x^1 + C x^0)(-S y^1 + C y^0) -(C x^1 -S x^0) (C y^1 -S y^0) \right) \\
&= \left( (S^2  x^1 y^1 + C^2  x^0 y^0) + x^0 y^1( -SC + SC ) + x^1 y^0( -SC + SC ) -(C^2  x^1 y^1 + S^2  x^0 y^0) \right) \\
&= \left( x^0 y^0 (C^2  -S^2 ) - (C^2 -S^2 ) x^1 y^1 \right) \\
&= x^0 y^0 - x^1 y^1 \\
&= X \cdot Y
\end{aligned}
\end{equation}

This is not really any less work.

\section{Geometric product formulation of Lorentz transform}

We can show the above invariance almost trivially when we write the Lorentz boost in exponential form.  However we first have to show
how to do so.

Writing the spacetime bivector \(\gamma_{10} = \gamma_1 \wedge \gamma_0\) for short, lets calculate the exponential of this spacetime bivector, as scaled with a rapidity
angle \(\alpha\) :

\begin{equation}\label{eqn:fVecDotInv:bivecexponential}
\exp(\gamma_{10}\alpha) = \sum \frac{(\gamma_{10}\alpha)^k}{k!}
\end{equation}

Now, the spacetime bivector has a unit square:

\begin{equation*}
{\gamma_{10}}^2 = \gamma_{1010} = -\gamma_{1001} = -\gamma_{11} = 1
\end{equation*}

so, we can split the sum of \eqnref{eqn:fVecDotInv:bivecexponential} into even and odd parts, and pull out the common bivector factor:

\begin{equation}\label{eqn:fVecDotInv:bivechyper}
\exp(\gamma_{10}\alpha)
= \sum \frac{\alpha^{2k}}{(2k)!} + \gamma_{10}\sum \frac{\alpha^{2k+1}}{(2k+1)!}
= \cosh(\alpha) + \gamma_{10} \sinh(\alpha)
\end{equation}

\subsection{Spatial rotation}
\index{rotation invariance}

So, this quite a similar form as bivector exponential with a Euclidean metric.  For such a space the bivector had a negative square, just like the complex unit imaginary,
which allowed for the normal trigonometric split of the exponential:

\begin{equation}
\exp(\Be_{12}\theta)
= \sum (-1)^k\frac{\theta^{2k}}{(2k)!} + \Be_{12}\sum (-1)^k\frac{\theta^{2k+1}}{(2k+1)!}
= \cos(\theta) + \Be_{12} \sin(\theta)
\end{equation}

Now, with the Minkowski metric having a negative square for purely spatial components, how does a purely spacial bivector behave when squared?  Let us try it with

\begin{equation*}
{\gamma_{12}}^2
= \gamma_{1212}
= -\gamma_{1221}
= \gamma_{11}
= -1
\end{equation*}

This also has a square that behaves like the unit imaginary, so we can do spacial rotations with rotors like we can with Euclidean space.  However, we have to invert the sign of the angle when using a Minkowski metric.  Take a specific example of a 90 degree rotation in the x-y plane, expressed in complex form:

\begin{equation}\label{eqn:fourvecDotinvariance:60}
\begin{aligned}
R_{\pi/2}(\gamma_1)
&= \gamma_1 \exp({ \gamma_{12} \pi/2 }) \\
&= \gamma_1 (0 + \gamma_{12}) \\
&= -\gamma_2 \\
\end{aligned}
\end{equation}

In general our Rotor equation with a Minkowski \((+,-,-,-)\) metric will be thus be:

\begin{equation}\label{eqn:fVecDotInv:spacerot}
R_{\theta}(x) = \exp( i\theta/2) x \exp( -i\theta/2)
\end{equation}

Here \(i\) is a spatial bivector (a bivector with negative square), such as \(\gamma_{1}\wedge\gamma_{2}\), and the rotation sense is with increasing angle from \(\gamma_1\) towards \(\gamma_2\).

\subsection{Validity of the double sided spatial rotor formula}

To demonstrate the validity of \eqnref{eqn:fVecDotInv:spacerot} one has to observe how the unit vectors \(\gamma_{\mu}\) behave with respect to commutation, and how that behavior results in either commutation or conjugate commutation with the exponential rotor.  Without any loss of generality one can restrict attention to a specific example, such as bivector \(\gamma_{12}\).  By inspection, \(\gamma_0\), and \(\gamma_3\) both commute since an even number of exchanges in position is required for either:

\begin{equation}\label{eqn:fourvecDotinvariance:80}
\begin{aligned}
\gamma_{0} \gamma_{12}
&= \gamma_{0} \wedge \gamma_{1} \wedge \gamma_{2} \\
&= \gamma_{1} \wedge \gamma_{2} \wedge \gamma_{0} \\
&= \gamma_{12} \gamma_0
\end{aligned}
\end{equation}

For this reason, application of the double sided rotation does not change any such (perpendicular) vector that commutes with the rotor:

\begin{equation}\label{eqn:fourvecDotinvariance:100}
\begin{aligned}
R_{\theta}(x_{\perp})
&= \exp( i\theta/2) x_{\perp} \exp( -i\theta/2) \\
&= x_{\perp} \exp( i\theta/2) \exp( -i\theta/2) \\
&= x_{\perp}
\end{aligned}
\end{equation}

Now for the basis vectors that lie in the plane of the spatial rotation we have anticommutation:

\begin{equation}\label{eqn:fourvecDotinvariance:120}
\begin{aligned}
\gamma_{1} \gamma_{12}
&= -\gamma_{1} \gamma_{21}  \\
&= -\gamma_{121} \\
&= -\gamma_{12} \gamma_{1}
\end{aligned}
\end{equation}

\begin{equation}\label{eqn:fourvecDotinvariance:140}
\begin{aligned}
\gamma_{2} \gamma_{12}
&= \gamma_{21}\gamma_{2} \\
&= -\gamma_{12}\gamma_{2}
\end{aligned}
\end{equation}

Given an understanding of how the unit vectors either commute or anticommute with the bivector for the plane of rotation, one can now see how these behave when multiplied by a rotor expressed exponentially:

\begin{equation}\label{eqn:fVecDotInv:spatialcommutationrule}
\gamma_{\mu}\exp(i\theta)
= \gamma_{\mu}\left( \cos(\theta) + i\sin(\theta) \right)
=
\left\{
\begin{array}{l l}
\left( \cos(\theta) + i\sin(\theta) \right) \gamma_{\mu} & \quad \mbox{if \(\gamma_{\mu} \cdot i = 0\)} \\
\left( \cos(\theta) - i\sin(\theta) \right) \gamma_{\mu} & \quad \mbox{if \(\gamma_{\mu} \cdot i \ne 0\)} \\
\end{array} \right.
\end{equation}

The condition \(\gamma_{\mu} \cdot i = 0\) corresponds to a spacelike vector perpendicular to the plane of rotation, or a timelike vector, or any combination of the two, whereas
\(\gamma_{\mu} \cdot i \ne 0\) is true for any spacelike vector that lies completely in the plane of rotation.

Putting this information all together, we now complete the verification that the double sided rotor formula leaves the perpendicular spacelike or the timelike components untouched.  For for purely spacelike vectors in the plane of rotation we recover the single sided complex form rotation as illustrated by the following x-y plane rotation:

\begin{equation}\label{eqn:fourvecDotinvariance:160}
\begin{aligned}
R_{\theta}(x_{\parallel})
&= \exp( \gamma_{12}\theta/2) x_{\parallel} \exp( -\gamma_{12}\theta/2) \\
&= x_{\parallel} \exp( -\gamma_{12}\theta/2) \exp( -\gamma_{12}\theta/2) \\
&= x_{\parallel} \exp( -\gamma_{12}\theta) \\
\end{aligned}
\end{equation}

\subsection{Back to time space rotation}

Now, like we can express a spatial rotation in exponential form, we can do the same for the hyperbolic ``rotation'' matrix of \eqnref{eqn:fVecDotInv:hyperbolicmatrix}.  Direct expansion
\footnote{
The paper ``Generalized relativistic velocity addition with spacetime algebra'', http://arxiv.org/pdf/physics/0511247.pdf derives the bivector form of this Lorentz boost directly in an interesting fashion.  Simple relativistic arguments are used that are quite similar to those of Einstein in his ``Relativity, the special and general theory'' appendix.  This paper is written in a form that requires you to work out many of the details yourself (likely for brevity).  However, once that extra work is done, I found the first half of that paper quite readable.
}
of the product is the easiest way to see that this is the case:

\begin{equation}\label{eqn:fourvecDotinvariance:180}
\begin{aligned}
\left(\gamma_{1} x^1 + \gamma_{0} x^0 \right)\exp(\gamma_{10}\alpha)
&= \left(\gamma_{1} x^1 + \gamma_{0} x^0 \right) \left( \cosh(\alpha) +\gamma_{10}\sinh(\alpha) \right) \\
\end{aligned}
\end{equation}

\begin{equation}\label{eqn:fVecDotInv:lorentz}
\begin{aligned}
&\left(\gamma_{1} x^1 + \gamma_{0} x^0 \right)\exp(\gamma_{10}\alpha) \\
&\qquad = \gamma_1\left( x^1 \cosh(\alpha) - x^0 \sinh(\alpha)\right)
 + \gamma_0\left( x^0 \cosh(\alpha) - x^1 \sinh(\alpha)\right)
\end{aligned}
\end{equation}

As with the spatial rotation, full characterization of this exponential rotation operator, in both single and double sided form requires that one looks at how the various unit vectors commute with the unit bivector.  Without loss of generality one can restrict attention to a specific case, as done with the \(\gamma_{10}\) above.

As in the spatial case, \(\gamma_{2}\), and \(\gamma_{3}\) both commute with \(\gamma_{10} = \gamma_1 \wedge \gamma_0\).  Example:

\begin{equation*}
\gamma_{2} \gamma_{10}
= \gamma_2 \wedge \gamma_1 \wedge \gamma_0
= \gamma_1 \wedge \gamma_0 \wedge \gamma_2
= \gamma_{10} \gamma_{2}
\end{equation*}

Now, consider each of the basis vectors in the spacetime plane.

\begin{equation*}
\gamma_{0} \gamma_{10}
= \gamma_{010}
= \gamma_{01} \gamma_{0}
= -\gamma_{10} \gamma_{0}
\end{equation*}

\begin{equation*}
\gamma_{1} \gamma_{10}
= \gamma_{110}
= -\gamma_{101}
= -\gamma_{10} \gamma_{1}
\end{equation*}

Both of the basis vectors in the spacetime plane anticommute with the bivector that describes the plane, and as a result we have a conjugate change in the exponential comparing left and right multiplication as with a spatial rotor.  Summarizing for the general case by introducing a spacetime rapidity plane described by a bivector

\(\Balpha = \alphacap \alpha\), we have:

\begin{equation}\label{eqn:fVecDotInv:spacetimecommutationrule}
\begin{aligned}
\gamma_{\mu}\exp(\Balpha)
&= \gamma_{\mu}\left( \cosh(\alpha) + \alphacap\sinh(\alpha) \right) \\
&=
\left\{
\begin{array}{l l}
\left( \cosh(\alpha) + \alphacap\sinh(\alpha) \right) \gamma_{\mu} & \quad \mbox{if \(\gamma_{\mu} \cdot \alphacap = 0\)} \\
\left( \cosh(\alpha) - \alphacap\sinh(\alpha) \right) \gamma_{\mu} & \quad \mbox{if \(\gamma_{\mu} \cdot \alphacap \ne 0\)} \\
\end{array} \right.
\end{aligned}
\end{equation}

Observe the similarity between \eqnref{eqn:fVecDotInv:spatialcommutationrule}, and \eqnref{eqn:fVecDotInv:spacetimecommutationrule} for spatial
and spacetime rotors.  Regardless of whether the plane is spacelike, or a spacetime plane we have the same rule:

\begin{equation}\label{eqn:fVecDotInv:generalrule}
\gamma_{\mu}\exp(\BB)
=
\left\{
\begin{array}{l l}
\exp(\BB) \gamma_{\mu} & \quad \mbox{if \(\gamma_{\mu} \cdot \Bcap = 0\)} \\
\exp(-\BB) \gamma_{\mu} & \quad \mbox{if \(\gamma_{\mu} \cdot \Bcap \ne 0\)}
\end{array} \right.
\end{equation}

Here, if \(\BB\) is a spacelike bivector (\(\BB^2 < 0\)) we get trigonometric functions generated by the exponentials, and if it represents
the spacetime plane \(\BB^2 > 0\) we get the hyperbolic functions.  As with the spatial rotor formulation, we have the same result for the
general signature bivector, and can write the generalized spacetime or spatial rotation as:

\begin{equation}
R_{\BB}(x) = \exp(-\BB/2) x \exp(\BB/2)
\end{equation}

Some care is required assigning meaning to the bivector angle \(\BB\).  We have seen that this is an negatively oriented spatial rotation in the $
\Bcap$ plane when spacelike.  How about for the spacetime case?
Lets go back and rewrite \eqnref{eqn:fVecDotInv:lorentz} in terms of vector
relations, with \(\Bv = v \vcap\)

\begin{equation}
\begin{aligned}
&\left( x^1 \vcap + x^0 \gamma_0 \right)
\left(
\frac{1}{\sqrt{1 -{\abs{(\Bv/c)}}^2}} + \frac{(\Bv/c) \gamma_0}{\sqrt{1 -{\abs{(\Bv/c)}}^2}}
\right) \\
&\qquad =
\vcap \gamma
\left( x^1 - x^0 v/c \right)
+
\gamma_0 \gamma
\left( x^0 - x^1 v/c \right)
\end{aligned}
\end{equation}

This allows for the following identification:

\begin{equation*}
\cosh(\alpha) + \vcap \gamma_0 \sinh(\alpha) = \exp( \vcap \gamma_0 \alpha)
=
\frac{1 + (\Bv/c) \gamma_0}{\sqrt{1 -{\abs{\Bv/c}}^2}}
\end{equation*}

which gives us the rapidity bivector (\(\BB\) above) in terms of the values we are familiar with:

\begin{equation*}
\vcap \gamma_0 \alpha = \log\left(
\frac{1 + (\Bv/c) \gamma_0}{\sqrt{1 -{\abs{\Bv/c}}^2}} \right)
\end{equation*}

Or,

\begin{equation*}
\BB = \vcap \gamma_0 \alpha = \tanh^{-1}(v/c) \vcap \gamma_0
\end{equation*}

Now since \(\abs{v/c} < 1\), the hyperbolic inverse tangent here can be expanded in (the slowly convergent) power series:

\begin{equation*}
\tanh^{-1}(x) = \sum_{k=0} \frac{x^{2k+1}}{2k+1}
\end{equation*}

Observe that this has only odd powers, and \(((\Bv/c) \gamma_0)^{2k+1} = \vcap\gamma_0 (v/c)^{2k+1}\).  This allows for the notational nicety of working with the spacetime bivector directly instead of only its magnitude:

\begin{equation}
\BB = \tanh^{-1}((\Bv/c) \gamma_0)
\end{equation}

\subsection{FIXME}

Revisit the equivalence of the two identities above.  How can one get from the log
expression to the hyperbolic inverse tangent directly?

\subsection{Apply to dot product invariance}

With composition of rotation and boost rotors we can form a generalized Lorentz transformation.  For example application of a rotation with rotor \(R\), to a boost with spacetime rotor \(L_0\), we get a combined more general transformation:

\begin{equation*}
L(x) = R ( L_0 x {L_0}^\dagger ) R^\dagger
\end{equation*}

In both cases, the rotor and its reverse when multiplied are identity:

\begin{equation*}
1 = R R^\dagger = L L^\dagger
\end{equation*}

It is not hard to see one can also compose an arbitrary set of rotations and boosts in the same fashion.  The new rotor will also satisfy \(L L^\dagger = 1\).

Application of such a rotor to a four vector we have:

\begin{equation*}
X' = L X L^\dagger
\end{equation*}

\begin{equation*}
Y' = L Y L^\dagger
\end{equation*}

\begin{equation}\label{eqn:fourvecDotinvariance:200}
\begin{aligned}
X' \cdot Y' &= (L X L^\dagger) \cdot (L Y L^\dagger) \\
&= \gpgradezero{ L X L^\dagger L Y L^\dagger } \\
&= \gpgradezero{ L X Y L^\dagger } \\
&= \gpgradezero{ L (X \cdot Y) L^\dagger } + \gpgradezero{ L (X \wedge Y) L^\dagger } \\
&= (X \cdot Y) \gpgradezero{ L L^\dagger } \\
&= X \cdot Y
\end{aligned}
\end{equation}

It is also clear that the four bivector \(X \wedge Y\) will also be Lorentz invariant.  This also implies that the geometric product of two four vectors \(X Y\) will also be Lorentz invariant.

UPDATE (Aug 14): I
do not
recall my reasons for thinking that the bivector invariance was clear initially.  It does not seem so clear now after the fact so I should have written it down.

   %
% Copyright � 2012 Peeter Joot.  All Rights Reserved.
% Licenced as described in the file LICENSE under the root directory of this GIT repository.
%

%
%
\chapter{Lorentz transformation of spacetime gradient}
\index{spacetime gradient!Lorentz transformation}
\label{chap:spacetimegrad}
%\date{July 16, 2008}

\section{Motivation}

We have observed that the wave equation is Lorentz invariant, and conversely that invariance of the form of the wave equation under linear transformation for light can be used to calculate the Lorentz transformation.  Specifically, this means that we require the equations of light (wave equation) retain its form after a change of variables that includes
a (possibly scaled) translation.  The wave equation should have no mixed partial terms, and retain the form:

\begin{equation*}
\lr{
\spacegrad^2 - \partial_{ct}^2
} F =
\lr{ {\spacegrad'}^2 - \partial_{ct'}^2 }
F = 0
\end{equation*}

Having expressed the spacetime gradient with a (STA) Minkowski basis, and knowing that the Maxwell equation written using the spacetime gradient is Lorentz invariant:

\begin{equation*}
\grad F = J,
\end{equation*}

we therefore expect that the square root of the wave equation (Laplacian) operator is also Lorentz invariant.  Here this idea is explored, and we look at how the spacetime
gradient behaves under Lorentz transformation.

\subsection{Lets do it}

Our spacetime gradient is

\begin{equation*}
\grad = \sum \gamma^{\mu} \frac{\partial}{\partial x^{\mu}}
\end{equation*}

Under Lorentz transformation we can transform the \(x^1=x\), and \(x^0 = ct\) coordinates:

\begin{equation*}
\begin{bmatrix}
x' \\
ct' \\
\end{bmatrix}
=
\gamma
\begin{bmatrix}
1 & -\beta \\
-\beta & 1 \\
\end{bmatrix}
\begin{bmatrix}
x \\
ct \\
\end{bmatrix}
\end{equation*}

Set \(c=1\) for convenience, and use this to transform the partials:

\begin{equation}\label{eqn:spacetimegrad:20}
\begin{aligned}
\Partial{}{x}
&= \Partial{x'}{x} \Partial{}{x'} + \Partial{t'}{x} \Partial{}{t'} \\
&= \gamma
\left( \Partial{}{x'} -\beta \Partial{}{t'} \right) \\
\end{aligned}
\end{equation}

\begin{equation}\label{eqn:spacetimegrad:40}
\begin{aligned}
\Partial{}{t}
&= \Partial{x'}{t} \Partial{}{x'} + \Partial{t'}{t} \Partial{}{t'} \\
&= \gamma \left( -\beta \Partial{}{x'} + \Partial{}{t'} \right) \\
\end{aligned}
\end{equation}

Inserting this into our expression for the gradient we have

\begin{equation}\label{eqn:spacetimegrad:60}
\begin{aligned}
\grad
&= \gamma^0 \Partial{}{t}
 + \gamma^1 \Partial{}{x}
 + \gamma^2 \Partial{}{y}
 + \gamma^3 \Partial{}{z} \\
&= \gamma^0 \gamma \left( -\beta \Partial{}{x'} + \Partial{}{t'} \right)
 + \gamma^1 \gamma \left( \Partial{}{x'} -\beta \Partial{}{t'} \right)
 + \gamma^2 \Partial{}{y}
 + \gamma^3 \Partial{}{z}.
\end{aligned}
\end{equation}

Grouping by the primed partials this is:

\begin{equation}\label{eqn:spacetimegrad:80}
\begin{aligned}
\grad
&= \gamma \left(\gamma^0 - \beta\gamma^1\right) \Partial{}{t'}
 + \gamma \left(\gamma^1 - \beta\gamma^0\right) \Partial{}{x'}
 + \gamma^2 \Partial{}{y}
 + \gamma^3 \Partial{}{z}.
\end{aligned}
\end{equation}

Lo and behold, the basis vectors with respect to the new coordinates appear to themselves transform as a Lorentz pair.  Specifically:

\begin{equation*}
\begin{bmatrix}
{\gamma^1}' \\
{\gamma^0}' \\
\end{bmatrix}
=
\gamma
\begin{bmatrix}
1 & -\beta \\
-\beta & 1 \\
\end{bmatrix}
\begin{bmatrix}
{\gamma^1} \\
{\gamma^0} \\
\end{bmatrix}
\end{equation*}

Now this is a bit curious looking since these new basis vectors are a funny mix of the original time and space basis vectors.  Observe however that these linear combinations of the basis vectors \({\gamma^0}'\), and \({\gamma^1}'\) do behave just as adequately as timelike and spacelike basis vectors:

\begin{equation}\label{eqn:spacetimegrad:100}
\begin{aligned}
{\gamma^0}' {\gamma^0}'
&= \gamma^2
\lr{ -\beta \gamma^1 + \gamma^0 }
\lr{ -\beta \gamma^1 + \gamma^0 } \\
&= \gamma^2
\lr{ -\beta^2 + 1 -\beta \gamma^0\gamma^1 -\beta \gamma^1 \gamma^0 }
\\
&= \gamma^2 \Bigl(-\beta^2 + 1 +
\mathLabelBox
[
   labelstyle={xshift=2cm},
   linestyle={out=270,in=90, latex-}
]
{\beta \gamma^1\gamma^0 -\beta \gamma^1 \gamma^0}{\(=0\)} \Bigr) \\
&= 1
\end{aligned}
\end{equation}

and for the transformed ``spacelike'' vector, it squares like a spacelike vector:

\begin{equation}\label{eqn:spacetimegrad:120}
\begin{aligned}
{\gamma^1}' {\gamma^1}'
&= \gamma^2
\lr{ \gamma^1 -\beta \gamma^0 }
\lr{ \gamma^1 -\beta \gamma^0 }
\\
&= \gamma^2
\lr{ -1 + \beta^2 -\beta\gamma^0 \gamma^1 -\beta\gamma^1\gamma^0 }
\\
&= \gamma^2 \Bigl(-1 + \beta^2 +
\mathLabelBox
[
   labelstyle={xshift=2cm},
   linestyle={out=270,in=90, latex-}
]
{\beta \gamma^1\gamma^0 -\beta \gamma^1 \gamma^0}{\(=0\)} \Bigr) \\
&= -1
\end{aligned}
\end{equation}

The conclusion is that like the wave equation, its square root, the spacetime gradient is also Lorentz invariant, and to achieve this invariance we
transform both the coordinates and the basis vectors (there was no need to transform the basis vectors for the wave equation since it is a scalar
equation).

In fact, this gives a very interesting way to view the Lorentz transform.  It is not just notational that we can think of the spacetime gradient as one of the square roots of the wave equation.
Like the vector square root of a scalar
there are infinitely many such roots, all differing by an angle or rotation in the vector space:

\begin{equation*}
(R \Bn R^\dagger)^2 = 1
\end{equation*}

Requiring the mixed signature (Minkowski) metric for the space requires only that we need a slightly different meaning for any of the possible rotations
applied to the vector.

\subsection{transform the spacetime bivector}

I am not sure of the significance of the following yet, but it is interesting to note that the spacetime bivector for the transformed coordinate
pair is also invariant:

\begin{equation}\label{eqn:spacetimegrad:140}
\begin{aligned}
{\gamma^1}' {\gamma^0}'
&= \gamma^2
\lr{ \gamma^1 -\beta \gamma^0 }
\lr{ -\beta \gamma^1 + \gamma^0 } \\
&= \gamma^2
\lr{ \beta -\beta +\beta^2 \gamma^0 \gamma^1 + \gamma^1 \gamma^0 } \\
&= \gamma^2 \lr{ 1-\beta^2 } \gamma^1 \gamma^0 \\
&= \gamma^1 \gamma^0
\end{aligned}
\end{equation}

We can probably use this to figure out how to transform bivector quantities like the electromagnetic field \(F\).

   %
% Copyright � 2012 Peeter Joot.  All Rights Reserved.
% Licenced as described in the file LICENSE under the root directory of this GIT repository.
%

%
%
\mychapter{GravitoElectroMagnetism}
\index{Gravito-electromagnetism}
\label{chap:gem}
%\date{October 26, 2008.  gem.tex}

\section{Some rough notes on reading of GravitoElectroMagnetism review}

I found the GEM equations interesting, and explored the surface of them slightly.  Here are some notes, mostly as a reference for myself ... looking at the
GEM equations mostly generates questions, especially since I do not have the GR
background to understand where the potentials (ie: what is that stress energy
tensor \(T_{\mu\nu}\)) nor the specifics of where the metric tensor
(perturbation of the Minkowski metric) came from.

\section{Definitions}

The article \citep{mashhoon2003gbr} outlines the GEM equations, which in short
are

Scalar and potential fields
%
\begin{equation}\label{eqn:gem:20}
\begin{aligned}
\Phi \approx \frac{GM}{r}, \quad \BA \approx \frac{G}{c} \frac{\BJ \cross \Bx}{r^3}
\end{aligned}
\end{equation}
%
Gauge condition
%
\begin{equation}\label{eqn:gem:40}
\begin{aligned}
\inv{c}\PD{t}{\Phi} + \spacegrad \cdot \left( \inv{2} \BA \right) = 0.
\end{aligned}
\end{equation}
%
GEM fields
\begin{equation}\label{eqn:gem:60}
\begin{aligned}
\BE = - \spacegrad \Phi -\inv{c} \PD{t}{}\left( \inv{2} \BB \right), \quad \BB = \spacegrad \cross \BA
\end{aligned}
\end{equation}
%
and finally the Maxwell-like equations are
%
\begin{equation}\label{eqn:gem:80}
\begin{aligned}
\spacegrad \cross \BE &= -\inv{c} \PD{t}{}\left(\inv{2}\BB\right) \\
\spacegrad \cdot \left( \inv{2} \BB \right) &= 0 \\
\spacegrad \cdot \BE &= 4 \pi G \rho \\
\spacegrad \cross \left( \inv{2} \BB \right) &= \inv{c} \PD{t}{\BE} + \frac{4\pi G}{c}\BJ
\end{aligned}
\end{equation}
%
\section{STA form}

As with Maxwell's equations a Clifford algebra representation should be possible to put this into a more symmetric form.  Combining the spatial div and grads, following conventions from \citep{doran2003gap} we have
%
\begin{equation}\label{eqn:gem:100}
\begin{aligned}
\spacegrad \BE &= 4 \pi G \rho + \inv{c} \PD{t}{}\left(\inv{2}I \BB\right) \\
\spacegrad \left( \inv{2} I \BB \right) &= \inv{c} \PD{t}{\BE} + \frac{4\pi G}{c}\BJ
\end{aligned}
\end{equation}
%
Or
\begin{equation}\label{eqn:gem:120}
\begin{aligned}
\left( \spacegrad -\inv{c} \PD{t}{}\right) \left( \BE + \inv{2} I \BB \right) &= \frac{4\pi G}{c} \left( c \rho + \BJ \right)
\end{aligned}
\end{equation}
%
Left multiplication with \(\gamma_0\), using a time positive metric signature (\((\gamma_0)^2=1\)),
\begin{equation}\label{eqn:gem:140}
\begin{aligned}
\left( \spacegrad -\inv{c} \PD{t}{}\right) \gamma_0 \left( -\BE + \inv{2} I \BB \right) &= \frac{4\pi G}{c} \left( c \rho \gamma_0 + J^i \gamma_i \right)
\end{aligned}
\end{equation}
%
But \(\left( \spacegrad -\inv{c} \PD{t}{}\right) \gamma_0 = \gamma_i \partial_i - \gamma_0 \partial_0 = -\gamma^\mu \partial_\mu = -\grad\).  Introduction of a four vector mass density \(J = c\rho \gamma_0 + J^i \gamma_i = J^\mu \gamma_\mu\), and a bivector field \(F = \BE -\inv{2} I \BB\) this is
%
\begin{equation}\label{eqn:gem:160}
\begin{aligned}
\grad F = -\frac{4\pi G}{c} J
\end{aligned}
\end{equation}
%
The gauge condition suggests a four potential \(V = \Phi \gamma_0 + \BA \gamma_0 = V^\mu \gamma_\mu\), where \(V^0 = \Phi\), and \(V^i = A^i/2\).  This merges the
space and time parts of the gauge condition
%
\begin{equation}\label{eqn:gem:180}
\begin{aligned}
\grad \cdot V = \gamma^\mu \partial_\mu \cdot \gamma_\nu V^\nu = \partial_\mu V^\mu = \inv{c}\PD{t}{\Phi} + \inv{2}\partial_i A^i.
\end{aligned}
\end{equation}
%
It is reasonable to assume that \(F = \grad \wedge V\) as in electromagnetism.  Let us see if this is the case
%
\begin{equation}\label{eqn:gem:200}
\begin{aligned}
\BE - I\BB/2
&= - \spacegrad \Phi -\inv{c} \PD{t}{}\left( \inv{2} \BB \right) - I\spacegrad \cross \BA/2 \\
&= - \gamma_i \partial_i \gamma_0 V^0 - \inv{2} \partial_0 A^i \gamma_i \gamma_0 + \spacegrad \wedge \BA/2 \\
&= \gamma^i \partial_i \gamma_0 V^0 + \gamma^0 \partial_0 \gamma_i A^i/2 - \gamma_i \partial_i \wedge \gamma_j V^j \\
&= \gamma^i \partial_i \gamma_0 V^0 + \gamma^0 \partial_0 \gamma_i V^i + \gamma^i \partial_i \wedge \gamma_j V^j \\
&= \gamma^\mu \partial_\mu \wedge \gamma_\nu V^\nu \\
&= \grad \wedge V
\end{aligned}
\end{equation}
%
Okay, so in terms of potential we have the form as Maxwell's equation
%
\begin{equation}\label{eqn:gem:field}
\begin{aligned}
\grad (\grad \wedge V) &= -\frac{4\pi G}{c} J.
\end{aligned}
\end{equation}
%
With the gauge condition \(\grad \cdot V = 0\), this produces the wave equation
%
\begin{equation}\label{eqn:gem:220}
\begin{aligned}
\grad^2 V &= -\frac{4\pi G}{c} J.
\end{aligned}
\end{equation}
%
In terms of the author's original equation 1.2 it appears that roughly
\(V^\mu = \barh_{0\mu}\), and \(J^\mu \propto T_{0\mu}\).

This is logically how he is able to go from that equation to the Maxwell
form since both have the same four-vector wave equation form (when \(T_{ij} \approx 0\)).  To give the potentials specific values in terms of mass and current
distribution appears to be where the retarded integrals are used.

The author expresses \(T^{\mu\nu}\) in terms of \(\rho\), and mass current \(j\), but
the field equations are in terms of \(T_{\mu\nu}\).  What metric tensor is
used to translate from upper to lower indices in this case.  ie: is it \(g_{\mu\nu}\), or \(\eta_{\mu\nu}\) ?

\section{Lagrangians}

\subsection{Field Lagrangian}
%
Since the electrodynamic equation and corresponding field Lagrangian is
\begin{equation}\label{eqn:gem:240}
\begin{aligned}
\grad (\grad \wedge A) &= \frac{J}{\epsilon_0 c} \\
\LL &= -\frac{\epsilon_0 c}{2} (\grad \wedge A)^2 + A \cdot J
\end{aligned}
\end{equation}
%
Then, from \eqnref{eqn:gem:field}, the GEM field Lagrangian in covariant form is
%
\begin{equation}\label{eqn:gem:260}
\begin{aligned}
\LL &= \frac{c}{8 \pi G} (\grad \wedge V)^2 + V \cdot J \\
\end{aligned}
\end{equation}
%
Writing \(F^{\mu\nu} = \partial^\mu V^\nu - \partial^\nu V^\mu\), the scalar part of this Lagrangian is:
%
\begin{equation}\label{eqn:gem:280}
\begin{aligned}
\LL &= -\frac{c}{16 \pi G} F^{\mu\nu} F_{\mu\nu} + V^\sigma J_\sigma \\
\end{aligned}
\end{equation}
%
Is this expression hiding in the Einstein field equations?

What is the Lagrangian for Newtonian gravity, and how do they compare?

\subsection{Interaction Lagrangian}

The metric (equation 1.4) in the article is given to be
%
\begin{equation}\label{eqn:gem:300}
\begin{aligned}
ds^2 &=
-c^2\left(1 - 2 \frac{\Phi}{c^2}\right) dt^2
+\frac{4}{c}\left(\BA \cdot d\Bx \right) dt
+\left(1 + 2 \frac{\Phi}{c^2}\right) \delta_{ij}dx^i dx^j \\
\implies
\Abs{ds^2} = c^2 (d\tau)^2 &= (dx^0)^2 - \sum_i (dx^i)^2
-2 \frac{V_0}{c^2} (dx^0)^2
-\frac{8}{c^2} V_i dx^i dx^0
- 2 \frac{V_0}{c^2} \delta_{ij}dx^i dx^j
\end{aligned}
\end{equation}
%
With \(v = \gamma_\mu dx^\mu/d\tau\), the Lagrangian for interaction is
%
\begin{equation}\label{eqn:gem:320}
\begin{aligned}
\LL
&= \inv{2} m \Abs{\frac{ds}{d\tau}}^2  \\
&= \inv{2} m c^2 \\
&= \inv{2} m v^2 -2 \frac{m V_0}{c^2} \sum_\mu (\xdot^\mu)^2 -\frac{8 m}{c^2} V_i \xdot^0 \xdot^i  \\
\end{aligned}
\end{equation}
%
\begin{equation}\label{eqn:gem:interactionlagrangian}
\begin{aligned}
\LL &= \inv{2} m v^2 - 2m \left( V_0 \sum_\mu (\xdot^\mu / c)^2 + 4 V_i (\xdot^0/c) (\xdot^i/c) \right)
\end{aligned}
\end{equation}
%
Now, unlike the Lorentz force Lagrangian
\begin{equation}\label{eqn:gem:340}
\begin{aligned}
\LL &= \inv{2} m v^2 + q A \cdot v/c,
\end{aligned}
\end{equation}
%
the Lagrangian of \eqnref{eqn:gem:interactionlagrangian} is quadratic in powers of \(\xdot^\mu\).
There are remarks in the article saying that the non-covariant Lagrangian used to arrive at the Lorentz force equivalent was a first order approximation.
Evaluation of this interaction Lagrangian does not produce anything like the
\(\pdot_\mu = \kappa F_{\mu\nu}\xdot^\nu\) that we see in electrodynamics.

The calculation is not interesting but the end result for reference is
%
\begin{equation}\label{eqn:gem:360}
\begin{aligned}
\pdot
%&= \frac{4m}{c^2} \frac{d}{d\tau}\left( V_0 \gamma^\mu v^\mu + 2V_i (v^i \gamma^0 + v^0 \gamma^i) \right) \\
%&- \frac{2m}{c^2} \left( \sum_\mu (v^\mu)^2 \grad V_0 + 4 v^0 v^i \grad V_i \right) \\
&= \frac{4m}{c^2} \left( (v \cdot \grad V_0) \gamma^\mu v^\mu + 2 (v \cdot \grad V_i) (v^i \gamma^0 + v^0 \gamma^i) \right) \\
&+ \frac{4m}{c^2} \left( V_0 \gamma^\mu a^\mu + 2V_i (a^i \gamma^0 + a^0 \gamma^i) \right) \\
&- \frac{2m}{c^2} \left( \sum_\mu (v^\mu)^2 \grad V_0 + 4 v^0 v^i \grad V_i \right)
\end{aligned}
\end{equation}
%
This can be simplified somewhat, but no matter what it will be quadratic in the velocity coordinates.

The article also says that the line element is approximate.
Has some of what
is required for a more symmetric covariant interaction proper force been
discarded?

\section{Conclusion}

The ideas here are interesting.  At a high level, roughly, as I see it, the equation
%
\begin{equation}\label{eqn:gem:380}
\begin{aligned}
\grad^2 h_{0\mu} = T_{0\mu}
\end{aligned}
\end{equation}
%
has exactly the same form as Maxwell's equations in covariant form, so you can define an antisymmetric field tensor equation in the same way, treating these elements of h, and the corresponding elements of T as a four vector potential and mass current.

That said, I do not have the GR background to know understand the introduction.  For example, how to actually arrive at 1.2
or how to calculated your metric tensor in equation 1.4.  I would have expected 1.4 to have a more symmetric form like the covariant Lorentz force Lagrangian (\(v^2 + kA.v\)), since you can get a Lorentz force like equation out of it.  Because of the quadratic velocity terms, no matter how one varies that metric with respect to s as a parameter, one cannot get anything at all close to the electrodynamics Lorentz force equation \(m\ddot{x}^\mu = q F_\mu\nu \dot{x}_\nu\), so the correspondence between electromagnetism and GR breaks down once one considers the interaction.

   %
% Copyright � 2012 Peeter Joot.  All Rights Reserved.
% Licenced as described in the file LICENSE under the root directory of this GIT repository.
%

%
%
%\input{../peeter_prologue.tex}

\chapter{Relativistic Doppler formula}
\index{Doppler equation!relativistic}
\label{chap:frequencyTx}
%\blogpage{http://sites.google.com/site/peeterjoot/math2009/frequencyTx.pdf}
%%\date{June 27, 2009}
%%\revisionInfo{\(RCSfile: frequencyTx.tex,v \) Last \(Revision: 1.6 \) \(Date: 2009/10/22 02:07:20 \)}

%\date{June 27, 2009 \(RCSfile: frequencyTx.tex,v \) Last \(Revision: 1.6 \) \(Date: 2009/10/22 02:07:20 \)}

\beginArtWithToc

\section{Transform of angular velocity four vector}

It was possible to derive the Lorentz boost matrix by requiring that the wave equation operator

\begin{equation}\label{eqn:frequencyTx:20}
\begin{aligned}
\grad^2 = \inv{c^2}\frac{\partial^2}{\partial t^2} - \spacegrad^2
\end{aligned}
\end{equation}

retain its form under linear transformation (\chapcite{PJLorentzWave}).  Applying spatial Fourier transforms (\chapcite{PJwaveFourier}), one finds that solutions to the wave equation

\begin{equation}\label{eqn:frequencyTx:40}
\begin{aligned}
\grad^2 \psi(t,\Bx) = 0
\end{aligned}
\end{equation}

Have the form

\begin{equation}\label{eqn:frequencyTx:60}
\begin{aligned}
\psi(t, \Bx) = \int A(\Bk) e^{i(\Bk \cdot \Bx - \omega t)} d^3 k
\end{aligned}
\end{equation}

Provided that \(\omega = \pm c \Abs{\Bk}\).  Wave equation solutions can therefore be thought of as continuously weighted superpositions of constrained fundamental solutions

\begin{equation}\label{eqn:frequencyTx:80}
\begin{aligned}
\psi &= e^{i(\Bk \cdot \Bx - \omega t)} \\
c^2 \Bk^2 &= \omega^2
\end{aligned}
\end{equation}

The constraint on frequency and wave number has the look of a Lorentz square

\begin{equation}\label{eqn:frequencyTx:100}
\begin{aligned}
\omega^2 - c^2 \Bk^2 = 0
\end{aligned}
\end{equation}

Which suggests that in additional to the spacetime vector

\begin{equation}\label{eqn:frequencyTx:120}
\begin{aligned}
X = (ct, \Bx) = x^\mu \gamma_\mu
\end{aligned}
\end{equation}

evident in the wave equation fundamental solution, we also have a frequency-wavenumber four vector

\begin{equation}\label{eqn:frequencyTx:140}
\begin{aligned}
K = (\omega/c, \Bk) = k^\mu \gamma_\mu
\end{aligned}
\end{equation}

The pair of four vectors above allow the fundamental solutions to be put explicitly into covariant form

\begin{equation}\label{eqn:frequencyTx:160}
\begin{aligned}
K \cdot X = \omega t - \Bk \cdot \Bx = k_\mu x^\mu
\end{aligned}
\end{equation}

\begin{equation}\label{eqn:frequencyTx:180}
\begin{aligned}
\psi = e^{-i K \cdot X}
\end{aligned}
\end{equation}

Let us also examine the transformation properties of this fundamental solution, and see as a side effect that \(K\)
has transforms appropriately as a four vector.

\begin{equation}\label{eqn:frequencyTx:200}
\begin{aligned}
0 &= \grad^2 \psi(t,\Bx) \\
&= {\grad'}^2 \psi(t',\Bx') \\
&= {\grad'}^2 e^{i(\Bx' \cdot \Bk' - \omega' t')} \\
&= -\left(\frac{{\omega'}^2}{c^2} - {\Bk'}^2 \right) e^{i(\Bx' \cdot \Bk' - \omega' t')} \\
\end{aligned}
\end{equation}

We therefore have the same form of frequency wave number constraint in the transformed frame (if we require that
the wave function for light is unchanged under transformation)

\begin{equation}\label{eqn:frequencyTx:220}
\begin{aligned}
{\omega'}^2 = c^2 {\Bk'}^2
\end{aligned}
\end{equation}

Writing this as

\begin{equation}\label{eqn:frequencyTx:240}
\begin{aligned}
0 = {\omega}^2 - c^2 {\Bk}^2 = {\omega'}^2 - c^2 {\Bk'}^2
\end{aligned}
\end{equation}

singles out the Lorentz invariant nature of the \((\omega, \Bk)\) pairing, and we conclude that this pairing
does indeed transform as a four vector.

\section{Application of one dimensional boost}

Having attempted to justify the four vector nature of the wave number vector \(K\), now move on to application of a boost along the x-axis to this vector.

\begin{equation}\label{eqn:frequencyTx:260}
\begin{aligned}
\begin{bmatrix}
\omega' \\
c k' \\
\end{bmatrix}
&=
\gamma
\begin{bmatrix}
1 & -\beta \\
-\beta& 1 \\
\end{bmatrix}
\begin{bmatrix}
\omega \\
c k \\
\end{bmatrix}
\\
&=
\begin{bmatrix}
\omega - v k \\
c k - \beta \omega
\end{bmatrix}
\end{aligned}
\end{equation}

We can take ratios of the frequencies if we make use of the dependency between \(\omega\) and \(k\).  Namely, \(\omega = \pm c k\).  We then have

\begin{equation}\label{eqn:frequencyTx:280}
\begin{aligned}
\frac{\omega'}{\omega}
%&= \omega - (v/c) (\pm \omega)
&= \gamma(1 \mp \beta) \\
&= \frac{1 \mp \beta}{\sqrt{1 - \beta^2}} \\
&= \frac{1 \mp \beta}{\sqrt{1 - \beta}\sqrt{1 + \beta}} \\
\end{aligned}
\end{equation}

For the positive angular frequency this is

\begin{equation}\label{eqn:frequencyTx:300}
\begin{aligned}
\frac{\omega'}{\omega}
&= \frac{\sqrt{1 - \beta}}{\sqrt{1 + \beta}}
\\
\end{aligned}
\end{equation}

and for the negative frequency the reciprocal.

Deriving this with a Lorentz boost is much simpler than the time dilation argument in wikipedia doppler article \citep{wiki:relDoppler}.  EDIT: Later found exactly the above boost argument in the wiki k-vector article \citep{wiki:kvector}.

What is missing here is putting this in a physical context properly with source and reciever frequencies spelled out.  That would make this more than just math.

%\EndArticle

   %
% Copyright � 2012 Peeter Joot.  All Rights Reserved.
% Licenced as described in the file LICENSE under the root directory of this GIT repository.
%

%
%
%\input{../peeter_prologue.tex}

\chapter{Poincare transformations}
\index{Poincare transformation}
\label{chap:poincareTx}
%\blogpage{http://sites.google.com/site/peeterjoot/math2009/poincareTx.pdf}
%\author{Peeter Joot \quad peeterjoot@protonmail.com }
%\date{June 1, 2009.  poincareTx.tex}

\beginArtWithToc

\section{Motivation}

In \citep{montesinos2006sem} a Poincare transformation is used to develop the symmetric stress energy tensor directly, in contrast to the non-symmetric canonical stress energy tensor that results from spacetime translation.

Attempt to decode one part of this article, the use of a Poincare transformation.

\section{Incremental transformation in GA form}

Equation (11) in the article, is labeled an infinitesimal Poincare transformation

\begin{equation}\label{eqn:poincareTx:txComponents}
\begin{aligned}
{x'}^\mu
&=
{x'}^\mu
+ {{\epsilon}^\mu}_\nu x^\nu
+ {\epsilon}^\mu
\end{aligned}
\end{equation}

It is stated that an antisymmetrization condition \(\epsilon_{\mu\nu} = -\epsilon_{\nu\mu}\).  This is somewhat confusing since the infinitesimal transformation is given by a mixed upper and lower index tensor.   Due to the antisymmetry perhaps this all a coordinate statement of the following vector to vector linear transformation

\begin{equation}\label{eqn:poincareTx:linearTxGuess}
\begin{aligned}
x' = x + \epsilon + A \cdot x
\end{aligned}
\end{equation}

This transformation is less restricted than a plain old spacetime transformation, as it also contains a projective term, where \(x\) is projected onto the spacetime (or spatial) plane \(A\) (a bivector), plus a rotation in that plane.

Writing as usual

\begin{equation}\label{eqn:poincareTx:20}
\begin{aligned}
x = \gamma_\mu x^\mu
\end{aligned}
\end{equation}

So that components are recovered by taking dot products, as in
\begin{equation}\label{eqn:poincareTx:40}
\begin{aligned}
x^\mu = x \cdot \gamma^\mu
\end{aligned}
\end{equation}

For the bivector term, write

\begin{equation}\label{eqn:poincareTx:60}
\begin{aligned}
A = c \wedge d = c^\alpha d^\beta (\gamma_\alpha \wedge \gamma_\beta)
\end{aligned}
\end{equation}

For
\begin{equation}\label{eqn:poincareTx:80}
\begin{aligned}
(A \cdot x ) \cdot \gamma^\mu
&=
c^\alpha d^\beta x_\sigma ((\gamma_\alpha \wedge \gamma_\beta) \cdot \gamma^\sigma) \cdot \gamma^\mu \\
&=
c^\alpha d^\beta x_\sigma ( {\delta_\alpha}^\mu {\delta_\beta}^\sigma -{\delta_\beta}^\mu {\delta_\alpha}^\sigma ) \\
&=
(c^\mu d^\sigma -c^\sigma d^\mu ) x_\sigma
\end{aligned}
\end{equation}

This allows for an identification \(\epsilon^{\mu\sigma} = c^\mu d^\sigma -c^\sigma d^\mu\) which is antisymmetric as required.  With that identification we can write \eqnref{eqn:poincareTx:txComponents} via the equivalent vector relation \eqnref{eqn:poincareTx:linearTxGuess} if we write

\begin{equation}\label{eqn:poincareTx:100}
\begin{aligned}
{\epsilon^\mu}_\sigma x^\sigma = (c^\mu d_\sigma -c_\sigma d^\mu ) x^\sigma
\end{aligned}
\end{equation}

Where \({\epsilon^\mu}_\sigma\) is defined implicitly in terms of components of the bivector \(A = c \wedge d\).

Is this what a Poincare transformation is?  The \href{http://mathworld.wolfram.com/PoincareTransformation.html}{Poincare Transformation} article suggests not.  This article suggests that the Poincare transformation is a spacetime translation plus a Lorentz transformation (composition of boosts and rotations).  That Lorentz transformation will not be antisymmetric however, so how can these be reconciled?  The key is probably the fact that this was an infinitesimal Poincare transformation so lets consider a Taylor expansion of the Lorentz boost or rotation rotor, considering instead a transformation of the following form

\begin{equation}\label{eqn:poincareTx:spaceTimeTxAndLor}
\begin{aligned}
x' &= x + \epsilon + R x \tilde{R} \\
R \tilde{R} &= 1
\end{aligned}
\end{equation}

In particular, let us look at the Lorentz transformation in terms of the exponential form
\begin{equation}\label{eqn:poincareTx:120}
\begin{aligned}
R = e^{I \theta/2}
\end{aligned}
\end{equation}

Here \(\theta\) is either the angle of rotation (when the bivector is a unit spatial plane such as \(I = \gamma_k \wedge \gamma_m\)), or a rapidity angle (when the bivector is a unit spacetime plane such as \(I = \gamma_k \wedge \gamma_0\)).

Ignoring the translation in \eqnref{eqn:poincareTx:spaceTimeTxAndLor} for now, to calculate the first order term in Taylor series we need

\begin{equation}\label{eqn:poincareTx:140}
\begin{aligned}
\frac{dx'}{d\theta}
&=
\frac{dR}{d\theta}  x \tilde{R}
+
{R} x \frac{d\tilde{R}}{d\theta}
\\
&=
\frac{dR}{d\theta} \tilde{R} R x \tilde{R}
+
{R} x \tilde{R} R \frac{d\tilde{R}}{d\theta}
\\
&=
\inv{2} ( \Omega x' + x' \tilde{\Omega} ) \\
\end{aligned}
\end{equation}

where
\begin{equation}\label{eqn:poincareTx:160}
\begin{aligned}
\inv{2}\Omega = \frac{dR}{d\theta} \tilde{R}
\end{aligned}
\end{equation}

Now, what is the grade of the product \(\Omega\)?  We have both \(dR/d\theta\) and \(R\) in \(\{\bigwedge^0 \oplus \bigwedge^2\}\) so the product can only have even grades \(\Omega \in \{\bigwedge^0 \oplus \bigwedge^2 \oplus \bigwedge^4\}\), but the unitary constraint on \(R\) restricts this

Since \(R \tilde{R} = 1\) the derivative of this is zero

\begin{equation}\label{eqn:poincareTx:180}
\begin{aligned}
\frac{dR}{d\theta} \tilde{R} + {R} \frac{d\tilde{R}}{d\theta}  = 0
\end{aligned}
\end{equation}

Or
\begin{equation}\label{eqn:poincareTx:200}
\begin{aligned}
\frac{dR}{d\theta} \tilde{R} = - \left( \frac{dR}{d\theta} \tilde{R} \right)^{\tilde{}}
\end{aligned}
\end{equation}

Antisymmetry rules out grade zero and four terms, leaving only the possibility of grade 2.  That leaves

\begin{equation}\label{eqn:poincareTx:220}
\begin{aligned}
\frac{dx'}{d\theta} = \inv{2}(\Omega x' - x' \Omega) = \Omega \cdot x'
\end{aligned}
\end{equation}

And the first order Taylor expansion around \(\theta =0\) is

\begin{equation}\label{eqn:poincareTx:240}
\begin{aligned}
x'(d\theta)
&\approx x'(\theta = 0) + ( \Omega d\theta ) \cdot x' \\
&= x + ( \Omega d\theta ) \cdot x'
\end{aligned}
\end{equation}

This has close to the postulated form in \eqnref{eqn:poincareTx:linearTxGuess}, but differs in one notable way.  The dot product with the antisymmetric form \(A = \inv{2} \frac{dR}{d\theta} \tilde{R} d\theta\) is a dot product with \(x'\) and not \(x\)!  One can however invert the identity writing \(x\) in terms of \(x'\) (to first order)

\begin{equation}\label{eqn:poincareTx:260}
\begin{aligned}
x = x' - ( \Omega d\theta ) \cdot x'
\end{aligned}
\end{equation}

Replaying this argument in fast forward for the inverse transformation should give us a relation for \(x'\) in terms of \(x\) and the incremental Lorentz transform

\begin{equation}\label{eqn:poincareTx:280}
\begin{aligned}
x' &= R x \tilde{R} \\
\implies \\
x &= \tilde{R} x' {R} \\
\end{aligned}
\end{equation}

\begin{equation}\label{eqn:poincareTx:300}
\begin{aligned}
\frac{dx}{d\theta}
&= \frac{d \tilde{R}}{d\theta} R \tilde{R} x' R + \tilde{R} x' R \tilde{R} \frac{d {R}}{d\theta}  \\
&= \left(2 \frac{d \tilde{R}}{d\theta} R \right) \cdot x
\end{aligned}
\end{equation}

So we have our incremental transformation given by

% red box ... ugly.
%\colorbox{red}{\parbox{0.5\textwidth}{
\begin{equation}\label{eqn:poincareTx:gotcha}
\begin{aligned}
x'= x - \left(2 \frac{d \tilde{R}}{d\theta} R d\theta \right) \cdot x
\end{aligned}
\end{equation}
%}}

% plain box.  also ugly
%\boxedeqn{
%x'= x - \left(2 \frac{d \tilde{R}}{d\theta} R d\theta \right) \cdot x
%}\label{eqn:poincareTx:gotcha}

\section{Consider a specific infinitesimal spatial rotation}

The signs and primes involved in arriving at \eqnref{eqn:poincareTx:gotcha} were a bit confusing.  To firm things up a bit considering a specific example is called for.

For a rotation in the \(x,y\) plane, we have

\begin{equation}\label{eqn:poincareTx:320}
\begin{aligned}
R &= e^{\gamma_1 \gamma_2 \theta/2} \\
x' &= R x \tilde{R}
\end{aligned}
\end{equation}

Here also it is easy to get the signs wrong, and it is worth pointing out the sign convention picked here for the Dirac basis is \({\gamma_0}^2 = -{\gamma_k}^2 = 1\).  To verify that \(R\) does the desired job, we have

\begin{equation}\label{eqn:poincareTx:340}
\begin{aligned}
R \gamma_1 \tilde{R}
&=
\gamma_1 \tilde{R^2} \\
&=
\gamma_1 e^{\gamma_2 \gamma_1 \theta} \\
&=
\gamma_1 (\cos\theta + \gamma_2 \gamma_1 \sin\theta) \\
&=
\gamma_1 (\cos\theta - \gamma_1 \gamma_2 \sin\theta) \\
&=
\gamma_1 \cos\theta + \gamma_2 \sin\theta \\
\end{aligned}
\end{equation}

and

\begin{equation}\label{eqn:poincareTx:360}
\begin{aligned}
R \gamma_2 \tilde{R}
&=
\gamma_2 \tilde{R^2} \\
&=
\gamma_2 e^{\gamma_2 \gamma_1 \theta} \\
&=
\gamma_2 (\cos\theta + \gamma_2 \gamma_1 \sin\theta) \\
&=
\gamma_2 \cos\theta - \gamma_1 \sin\theta \\
\end{aligned}
\end{equation}

For \(\gamma_3\) or \(\gamma_0\), the quaternion \(R\) commutes, so we have

\begin{equation}\label{eqn:poincareTx:380}
\begin{aligned}
R \gamma_3 \tilde{R} &= R \tilde{R} \gamma_3 = \gamma_3 \\
R \gamma_0 \tilde{R} &= R \tilde{R} \gamma_0 = \gamma_0 \\
\end{aligned}
\end{equation}

(leaving the perpendicular basis directions unchanged).

Summarizing the action on the basis vectors in matrix form this is
%1: \gamma_1 \cos\theta + \gamma_2 \sin\theta \\
%2: \gamma_2 \cos\theta - \gamma_1 \sin\theta \\
\begin{equation}\label{eqn:poincareTx:400}
\begin{aligned}
\begin{bmatrix}
\gamma_0 \\
\gamma_1 \\
\gamma_2 \\
\gamma_3 \\
\end{bmatrix}
\rightarrow
\begin{bmatrix}
1 & 0 & 0 & 0 \\
0 & \cos\theta & \sin\theta & 0 \\
0 & -\sin\theta & \cos\theta & 0 \\
0 & 0 & 0 & 1 \\
\end{bmatrix}
\begin{bmatrix}
\gamma_0 \\
\gamma_1 \\
\gamma_2 \\
\gamma_3 \\
\end{bmatrix}
\end{aligned}
\end{equation}

Observe that the basis vectors transform with the transposed matrix to the coordinates, and we have

\begin{equation}\label{eqn:poincareTx:420}
\begin{aligned}
\gamma_0 x^0
+ \gamma_1 x^1
+ \gamma_2 x^2
+ \gamma_3 x^3
\rightarrow
\gamma_0 x^0
+x^1 (\gamma_1 \cos\theta + \gamma_2 \sin\theta)
+x^2 (\gamma_2 \cos\theta - \gamma_1 \sin\theta)
+\gamma_3 x^3
\end{aligned}
\end{equation}

Dotting \({x'}^\mu = x' \cdot \gamma^\mu\) we have

\begin{equation}\label{eqn:poincareTx:440}
\begin{aligned}
x^0 &\rightarrow x^0 \\
x^1 &\rightarrow x^1 \cos\theta - x^2 \sin\theta \\
x^2 &\rightarrow x^1 \sin\theta +x^2 \cos\theta  \\
x^3 &\rightarrow x^3
\end{aligned}
\end{equation}

In matrix form this is the expected and familiar rotation matrix in coordinate form
\begin{equation}\label{eqn:poincareTx:460}
\begin{aligned}
\begin{bmatrix}
x^0 \\
x^1 \\
x^2 \\
x^3 \\
\end{bmatrix}
\rightarrow
\begin{bmatrix}
1 & 0 & 0 & 0 \\
0 & \cos\theta & -\sin\theta & 0 \\
0 & \sin\theta & \cos\theta & 0 \\
0 & 0 & 0 & 1 \\
\end{bmatrix}
\begin{bmatrix}
x^0 \\
x^1 \\
x^2 \\
x^3 \\
\end{bmatrix}
\end{aligned}
\end{equation}

Moving on to the initial verification we have

\begin{equation}\label{eqn:poincareTx:480}
\begin{aligned}
2 \frac{d\tilde{R}}{d\theta}
&= 2\frac{d}{d\theta} e^{\gamma_2\gamma_1 \theta/2} \\
&= \gamma_1 \gamma_2 e^{\gamma_2\gamma_1 \theta/2} \\
\end{aligned}
\end{equation}

So we have

\begin{equation}\label{eqn:poincareTx:500}
\begin{aligned}
2 \frac{d\tilde{R}}{d\theta} R
&= \gamma_2 \gamma_1 e^{\gamma_2\gamma_1 \theta/2} e^{\gamma_1\gamma_2 \theta/2} \\
&= \gamma_2 \gamma_1
\end{aligned}
\end{equation}

The antisymmetric form \(\epsilon_{\mu\nu}\) in this case therefore appears to be nothing more than the unit bivector for the plane of rotation!  We should now be able to verify the incremental transformation result from \eqnref{eqn:poincareTx:gotcha}, which is in this specific case now calculated to be

\begin{equation}\label{eqn:poincareTx:xyRot}
\begin{aligned}
x'= x + d\theta (\gamma_1 \gamma_2) \cdot x
\end{aligned}
\end{equation}

As a final check let us look at the action of rotation part of the transformation \eqnref{eqn:poincareTx:xyRot} on the coordinates \(x^\mu\).  Only the \(x^1\) and \(x^2\) coordinates need be considered since there is no projection of \(\gamma_0\) or \(\gamma_3\) components onto the plane \(\gamma_1 \gamma_2\).

\begin{equation}\label{eqn:poincareTx:520}
\begin{aligned}
d\theta (\gamma_1 \gamma_2) \cdot (x^1 \gamma_1 + x^2 \gamma_2)
&=
d\theta \gpgradeone{ \gamma_1 \gamma_2 (x^1 \gamma_1 + x^2 \gamma_2) } \\
&=
d\theta (\gamma_2 x^1 - \gamma_1 x^2)  \\
\end{aligned}
\end{equation}

Now compare to the incremental transformation on the coordinates in matrix form.  That is

\begin{equation}\label{eqn:poincareTx:540}
\begin{aligned}
\delta R
&=
d\theta \frac{d}{d\theta}
{
\left.
\begin{bmatrix}
1 & 0 & 0 & 0 \\
0 & \cos\theta & -\sin\theta & 0 \\
0 & \sin\theta & \cos\theta & 0 \\
0 & 0 & 0 & 1 \\
\end{bmatrix}
\right\vert}_{\theta=0} \\
&=
d\theta
{
\left.
\begin{bmatrix}
0 & 0 & 0 & 0 \\
0 & -\sin\theta & -\cos\theta & 0 \\
0 & \cos\theta & -\sin\theta & 0 \\
0 & 0 & 0 & 0 \\
\end{bmatrix}
\right\vert}_{\theta=0} \\
&=
d\theta
\begin{bmatrix}
0 & 0 & 0 & 0 \\
0 & 0 & -1 & 0 \\
0 & 1 & 0 & 0 \\
0 & 0 & 0 & 0 \\
\end{bmatrix} \\
\end{aligned}
\end{equation}

So acting on the coordinate vector

\begin{equation}\label{eqn:poincareTx:560}
\begin{aligned}
\delta R &= d\theta
\begin{bmatrix}
0 & 0 & 0 & 0 \\
0 & 0 & -1 & 0 \\
0 & 1 & 0 & 0 \\
0 & 0 & 0 & 0 \\
\end{bmatrix}
\begin{bmatrix}
x^0 \\
x^1 \\
x^2 \\
x^3
\end{bmatrix} \\
&=
d\theta
\begin{bmatrix}
0 \\
-x^2 \\
x^1 \\
0
\end{bmatrix} \\
\end{aligned}
\end{equation}

This is exactly what we got above with the bivector dot product.  Good.

\section{Consider a specific infinitesimal boost}

For a boost along the \(x\) axis we have

\begin{equation}\label{eqn:poincareTx:580}
\begin{aligned}
R &= e^{\gamma_0\gamma_1 \alpha/2} \\
x' &= R x \tilde{R}
\end{aligned}
\end{equation}

Verifying, we have

\begin{equation}\label{eqn:poincareTx:600}
\begin{aligned}
x^0 \gamma_0
&\rightarrow x^0 ( \cosh\alpha + \gamma_0 \gamma_1 \sinh\alpha ) \gamma_0 \\
&= x^0 ( \gamma_0 \cosh\alpha - \gamma_1 \sinh\alpha ) \\
\end{aligned}
\end{equation}

\begin{equation}\label{eqn:poincareTx:620}
\begin{aligned}
x^1 \gamma_1
&\rightarrow x^1 ( \cosh\alpha + \gamma_0 \gamma_1 \sinh\alpha ) \gamma_1 \\
&= x^1 ( \gamma_1 \cosh\alpha - \gamma_0 \sinh\alpha ) \\
\end{aligned}
\end{equation}

Dot products recover the familiar boost matrix
\begin{equation}\label{eqn:poincareTx:640}
\begin{aligned}
\begin{bmatrix}
x^0 \\
x^1 \\
x^2 \\
x^3 \\
\end{bmatrix}'
&=
\begin{bmatrix}
\cosh\alpha & -\sinh\alpha & 0 & 0 \\
-\sinh\alpha & \cosh\alpha & 0 & 0 \\
0 & 0 & 1 & 0 \\
0 & 0 & 0 & 1 \\
\end{bmatrix}
\begin{bmatrix}
x^0 \\
x^1 \\
x^2 \\
x^3 \\
\end{bmatrix}
\end{aligned}
\end{equation}

Now, how about the incremental transformation given by \eqnref{eqn:poincareTx:gotcha}.  A quick calculation shows that we have

\begin{equation}\label{eqn:poincareTx:xLor}
\begin{aligned}
x' = x + d\alpha (\gamma_0 \gamma_1) \cdot x
\end{aligned}
\end{equation}

Just like the \eqnref{eqn:poincareTx:xyRot} case for a rotation in the \(x y\) plane, the antisymmetric form is again the unit bivector of the rotation plane (this time the unit bivector in the spacetime plane of the boost.)

This completes the examination of two specific incremental Lorentz transformations.  It is clear that the result will be the same for an arbitrarily oriented bivector, and the original guess \eqnref{eqn:poincareTx:linearTxGuess} of a geometric equivalent of tensor relation \eqnref{eqn:poincareTx:txComponents} was correct, provided that \(A\) is a unit bivector scaled by the magnitude of the incremental transformation.

The specific case not treated however are those transformations where the orientation of the bivector is allowed to change.  Parameterizing that by angle is not such an obvious procedure.

\section{In tensor form}

For an arbitrary bivector \(A = a \wedge b\), we can calculate \({\epsilon^\sigma}_\alpha\).  That is

\begin{equation}\label{eqn:poincareTx:660}
\begin{aligned}
\epsilon^{\sigma\alpha} x_\alpha
&=
d\theta
\frac
{
((a^\mu \gamma_\mu \wedge b^\nu \gamma_\nu) \cdot ( x_\alpha \gamma^\alpha)) \cdot \gamma^\sigma
}
{\Abs{((a^\mu \gamma_\mu) \wedge (b^\nu \gamma_\nu)) \cdot ((a_\alpha \gamma^\alpha) \wedge (b_\beta \gamma^\beta))}^{1/2}}
\\
&=
\frac{ a^\sigma b^\alpha - a^\alpha b^\sigma }{\Abs{a^\mu b^\nu( a_\nu b_\mu - a_\mu b_\nu)}^{1/2}} x_\alpha \\
\end{aligned}
\end{equation}

So we have

\begin{equation}\label{eqn:poincareTx:680}
\begin{aligned}
{\epsilon^\sigma}_\alpha
&=
d\theta
\frac{ a^\sigma b_\alpha - a_\alpha b^\sigma }{\Abs{a^\mu b^\nu( a_\nu b_\mu - a_\mu b_\nu)}^{1/2}}
\end{aligned}
\end{equation}

The denominator can be subsumed into \(d\theta\), so the important factor is just the numerator, which encodes an incremental boost or rotational in some arbitrary spacetime or spatial plane (respectively).  The associated antisymmetry can be viewed as a consequence of the bivector nature of the rotor derivative rotor product.

%\EndArticle


\part{Electrodynamics}
   %
% Copyright � 2012 Peeter Joot.  All Rights Reserved.
% Licenced as described in the file LICENSE under the root directory of this GIT repository.
%

%
%
\mychapter{Maxwell's equations expressed with Geometric Algebra}
\index{Maxwell's equations}
\label{chap:maxwellsGa}
%\date{January 29, 2008.  maxwellsGa.tex}

\section{On different ways of expressing Maxwell's equations}

One of the most striking applications of the geometric product is the ability to formulate the eight Maxwell's equations in a coherent fashion as a single equation.

This is not a new idea, and this has been done historically using formulations based on quaternions (~1910.  dig up citation).  A formulation in terms of antisymmetric second rank tensors \(F_{\mu \nu}\) and \(G_{\mu \nu}\) (See: wiki:Formulation of Maxwell's equations in special relativity) reduces the eight equations to two, but also introduces complexity and obfuscates the connection to the physically measurable quantities.

A formulation in terms of differential forms (See: wiki:Maxwell's equations) is also possible.  This does not have the complexity of the tensor formulation, but requires the electromagnetic field to be expressed as a differential form.  This is arguably strange given a traditional vector calculus education.  One also does not have to integrate a field in any fashion, so what meaning should be given to a electrodynamic field as a differential form?

\subsection{Introduction of complex vector electromagnetic field}

To explore the ideas, the starting point is the traditional set of Maxwell's equations
%
\begin{equation}\label{eqn:maxwellsGa:20}
\nabla \cdot \BE  = \frac {\rho} {\epsilon_0}
\end{equation}
\begin{equation}\label{eqn:maxwellsGa:40}
\nabla \cdot \BB  = 0
\end{equation}
\begin{equation}\label{eqn:maxwellsGa:60}
\nabla \times \BE  +\frac{\partial \BB } {\partial t} = 0
\end{equation}
\begin{equation}\label{eqn:maxwellsGa:80}
c^2 \nabla \times \BB  - \frac{\partial \BE } {\partial t}
= \frac{\BJ}{\epsilon_0}
\end{equation}
%
It is customary in relativistic treatments of electrodynamics to introduce a four vector \((x, y, z, ict)\).  Using this as a hint, one can write the time partials in terms of \(ict\) and regrouping slightly
%
\begin{equation}\label{eqn:maxwellsGa:100}
\nabla \cdot \BE  = \frac {\rho} {\epsilon_0}
\end{equation}
\begin{equation}\label{eqn:maxwellsGa:120}
\nabla \cdot (ic\BB ) = 0
\end{equation}
\begin{equation}\label{eqn:maxwellsGa:140}
\nabla \times \BE  +\frac{\partial (ic\BB )} {\partial (ict)} = 0
\end{equation}
\begin{equation}\label{eqn:maxwellsGa:160}
\nabla \times (ic\BB ) + \frac{\partial \BE } {\partial (ict)}
= i\frac{\BJ}{\epsilon_0 c}
\end{equation}
%
There is no use of geometric or wedge products here, but the opposing signs in the two sets of curl and time partial equations is removed.  The pairs of equations can be added together without loss of information since the original equations can be recovered by taking real and imaginary parts.
\begin{equation}\label{eqn:maxwellsGa:180}
\nabla \cdot (\BE + ic \BB) = \frac {\rho} {\epsilon_0}
\end{equation}
\begin{equation}\label{eqn:maxwellsGa:200}
\nabla \times (\BE + ic \BB) + \frac{\partial (\BE + ic \BB)} {\partial (ict)}
= i\frac{\BJ}{\epsilon_0 c}
\end{equation}
%
It is thus natural to define a combined electrodynamic field as a complex vector, expressing the natural orthogonality of the electric and magnetic fields
\begin{equation}\label{eqn:maxwellsGa:220}
\BF = \BE + ic \BB.
\end{equation}
%
The electric and magnetic fields can be recovered from this composite field by taking real and imaginary parts respectively, and we can now write Maxwell's equations in terms of this single electrodynamic field
\begin{equation}\label{eqn:maxwellsGa:240}
\nabla \cdot \BF = \frac {\rho} {\epsilon_0}
\end{equation}
\begin{equation}\label{eqn:maxwellsGa:260}
\nabla \times \BF + \frac{\partial \BF} {\partial (ict)}
= i\frac{\BJ}{\epsilon_0 c}
\end{equation}
%
\subsection{Converting the curls in the pair of Maxwell's equations for the electrodynamic field to wedge and geometric products}

The above manipulations didn't make any assumptions about the structure of the ``imaginary'' denoted \(i\) above.  What was implied was a requirement that \(i^2 = -1\), and that \(i\) commutes with vectors.  Both of these conditions are met by the use of the pseudoscalar for 3D Euclidean space \(\Be_1 \Be_2 \Be_3\).  This is usually denoted \(I\) and we'll now switch notations for clarity.

With multiplication of the second by a \(I\) factor to convert to a wedge product representation the remaining pair of equations can be written
%
\begin{equation}\label{eqn:maxwellsGa:280}
\nabla \cdot \BF = \frac {\rho} {\epsilon_0}
\end{equation}
\begin{equation}\label{eqn:maxwellsGa:300}
I\nabla \times \BF + \frac{1}{c} \frac{\partial \BF}{\partial t}
= -\frac{\BJ}{\epsilon_0 c}
\end{equation}
%
This last, in terms of the geometric product is,
\begin{equation}\label{eqn:maxwellsGa:320}
\nabla \wedge \BF + \frac{1}{c} \frac{\partial \BF}{\partial t}
= -\frac{\BJ}{\epsilon_0 c}
\end{equation}
%
These equations can be added without loss
%
\begin{equation}\label{eqn:maxwellsGa:340}
\nabla \cdot \BF + \nabla \wedge \BF + \frac{1}{c} \frac{\partial \BF}{\partial t} = \frac {\rho} {\epsilon_0} - \frac{\BJ}{\epsilon_0 c}
\end{equation}
%
Leading to the end result
%
\begin{equation}\label{eqn:maxwellsGa:5}
\left(\frac{1}{c} \frac{\partial}{\partial t} + \nabla\right)\BF = \frac {1} {\epsilon_0}\left(\rho - \frac{\BJ}{c}\right)
\end{equation}
%
Here we have all of Maxwell's equations as a single differential equation.
This gives a hint why it is hard to separately solve these equations for the electric or magnetic field components (the partials of which are scattered across the original eight different equations.)  Logically the electric and magnetic field components have to be kept together.

Solution of this equation will require some new tools.  Minimally, some relearning of existing vector calculus tools is required.

\subsection{Components of the geometric product Maxwell equation}

Explicit expansion of this equation, again using \(I={\Be}_1{\Be}_2{\Be}_3\), will yield a scalar, vector, bivector, and pseudoscalar components, and is an interesting exercise to verify the simpler field equation really describes the same thing.
%
\begin{equation}\label{eqn:maxwellsGa:360}
\left(\frac{1}{c} \frac{\partial}{\partial t} + \nabla\right)\BF
= \frac{1}{c} \frac{\partial \BE}{\partial t} + I\frac{1}{c} \frac{\partial \BB}{\partial t}
+ \nabla \cdot \BE + \nabla \wedge \BE + \nabla \cdot I \BB + \nabla \wedge I \BB
\end{equation}
%
The imaginary part of the field can be multiplied out as bivector components explicitly
%
\begin{equation}\label{eqn:maxwellsGa:620}
\begin{aligned}
I \BB &= \Be _1 \Be _2 \Be _3 ( \Be _1 B_1 + \Be _2 B_2 + \Be _3 B_3 ) \\
&= \Be _2 \Be _3 B_1 + \Be _3 \Be _1 B_2 + \Be _1 \Be _2 B_3
\end{aligned}
\end{equation}
%
which allows for direct calculation of the following
%
\begin{equation}\label{eqn:maxwellsGa:380}
\nabla \wedge I\BB = I\nabla \cdot \BB
\end{equation}
\begin{equation}\label{eqn:maxwellsGa:400}
\nabla \cdot I\BB = -\nabla \times \BB.
\end{equation}
%
These can be demonstrated by reducing \( \gpgradethree{ \nabla I \BB } \), and \( \gpgradeone{ \nabla I \BB } \) respectively.  Using these identities and writing the electric field curl term in terms of the cross product
%
\begin{equation}\label{eqn:maxwellsGa:420}
\nabla \wedge \BE = I \nabla \times \BE,
\end{equation}
%
allows for grouping of real and imaginary scalar and real and imaginary vector (bivector) components
%
\begin{equation}\label{eqn:maxwellsGa:440}
   \left(\nabla \cdot \BE\right) + I\left(\nabla \cdot \BB\right)
+
   \left(\frac{1}{c} \frac{\partial \BE}{\partial t} - \nabla \times \BB\right)
+ I\left(\frac{1}{c} \frac{\partial \BB}{\partial t} + \nabla \times \BE\right)
\end{equation}
\begin{equation}\label{eqn:maxwellsGa:460}
= \frac{\rho}{\epsilon_0} + I\left(0\right) + \left(-\frac{\BJ}{\epsilon_0 c}\right) + I \Bzero.
\end{equation}
%
Comparing each of the left and right side components recovers the original set of four (or eight depending on your point of view) Maxwell's equations.

%\section{Future: comparison to gravitation?}
%
%% Wed 03/05/2008
%
%The high school electrostatics equation, where \(\rho\) is either a continuous distribution or a spatial delta function for point masses:
%
%\begin{dmath}\label{eqn:maxwellsGa:480}
%\BE(\Br) = \inv{4 \pi \epsilon_0}\int{\rho(\Br') \frac{(\Br -\Br')}{(\Br -\Br')^2}}dV'
%\end{dmath}
%
%As a field equation this is written:
%\begin{dmath}\label{eqn:maxwellsGa:500}
%\nabla \cdot \BE(\Br) = \frac{\rho(\Br)}{\epsilon_0}
%\end{dmath}
%
%but this is both not relativistically correct nor does is
%include the propagation effects for ``electrostatics'' interactions
%which occur at the speed of light.
%
%We need the other three components of the Maxwell's
%equation \eqnref{eqn:maxwellsGa:5}, to get the propagation and relativistic
%corrections.
%
%Compare this to newton's gravitational field equation:
%
%\begin{dmath}\label{eqn:maxwellsGa:520}
%\BG(\Br) = -G\int{\rho(\Br') \frac{(\Br -\Br')}{(\Br -\Br')^2}}dV'
%\end{dmath}
%
%% G = 1/4 pi e
%% 4piG = 1/e
%
%which can be written as a field equation as:
%\begin{dmath}\label{eqn:maxwellsGa:540}
%\nabla \cdot \BG(\Br) = 4\pi G \rho(\Br).
%\end{dmath}
%
%If one assumes that electrodynamics and gravitation
%have the same form then is the corrected form of the gravitational field
%equation with respect to relativity and propagation at the speed of light
%as follows:
%
%\begin{equation}\label{eqn:maxwellsGa:600}
%\left(\frac{1}{c} \frac{\partial}{\partial t} + \nabla\right)\BG(\Br) =
%4 \pi G \rho(\Br)
%\end{equation}
%
%Is this correct in any sense?  Perhaps it matches the special relativity results
%but not the general relativity ones?

   %
% Copyright � 2012 Peeter Joot.  All Rights Reserved.
% Licenced as described in the file LICENSE under the root directory of this GIT repository.
%

%
%
\mychapter{Back to Maxwell's equations}
\label{chap:PJMaxwell2}
\index{Maxwell's equations}
%\date{ July 12, 2008. Revised \(Date: 2009/06/04 13:13:27 \)}

%\(Id: gaMaxwell.tex,v 1.12 2009/06/04 13:13:27 Peeter Exp \)

\section{}

Having observed and demonstrated that the Lorentz transformation is a natural consequence of requiring the electromagnetic wave equation retains the
form of the wave equation under change of space and time variables that includes a velocity change in one spacial direction.

Lets step back and look at Maxwell's equations in more detail.  In particular looking at how we get from integral to differential
to GA form.  Some of this is similar to the approach in GAFP, but that text is intended for more mathematically sophisticated readers.

We start with the equations in SI units:
%
\begin{equation}\label{eqn:gaMaxwell:20}
\begin{aligned}
\int_{S(\text{closed boundary of V})} \BE \cdot \ncap dA &= \inv{\epsilon_0} \int_V \rho dV \\
\int_{S(\text{any closed surface})} \BB \cdot \ncap dA &= 0 \\
\int_{C(\text{boundary of S})} \BE \cdot d\Bx &= - \int_{S} \frac{\partial \BB}{\partial t} \cdot \ncap dA \\
\int_{C(\text{boundary of S})} \BB \cdot d\Bx &= \mu_0\left(I + \epsilon_0\int_{S} \frac{\partial \BE}{\partial t} \cdot \ncap dA\right) \\
\end{aligned}
\end{equation}
%
As the surfaces and corresponding loops or volumes are made infinitely small, these equations (FIXME: demonstrate), can be written in differential form:
%
\begin{equation}\label{eqn:gaMaxwell:40}
\begin{aligned}
\spacegrad \cdot \BE &= \frac{\rho}{\epsilon_0} \\
\spacegrad \cdot \BB &= 0 \\
\spacegrad \cross \BE &= - \frac{\partial \BB}{\partial t} \\
\spacegrad \cross \BB &= \mu_0\left(\BJ + \epsilon_0 \frac{\partial \BE}{\partial t}\right) \\
\end{aligned}
\end{equation}
%
These are respectively, Gauss's Law for E, Gauss's Law for B, Faraday's Law, and the Ampere/Maxwell's Law.

This differential form can be manipulated to derive the wave equation for free space, or the wave equation with charge and current forcing terms in other space.

\subsection{Regrouping terms for dimensional consistency}

Derivation of the wave equation can be done nicely using geometric algebra, but first is it helpful to put these equations in a more dimensionally pleasant form.
Lets relate the dimensions of the electric and magnetic fields and the constants \(\mu_0, \epsilon_0\).

From Faraday's equation we can relate the dimensions of
\(\BB\), and \(\BE\):
%
\begin{equation}
\frac{[\BE]}{[d]} = \frac{[\BB]}{[t]}
\end{equation}
%
We therefore see that \(\BB\), and \(\BE\) are related dimensionally by a velocity factor.

Looking at the dimensions of the displacement current density in the Ampere/Maxwell equation we see:
%
\begin{equation}
\frac{[\BB]}{[d]} = [\mu_0\epsilon_0] \frac{[\BE]}{[t]}
\end{equation}
%
From the two of these the dimensions of the \(\mu_0\epsilon_0\) product can be seen to be:
%
\begin{equation}
[\mu_0\epsilon_0] = \frac{{[t]}^2}{{[d]}^2}
\end{equation}
%
So, we see that we have a velocity factor relating \(\BE\), and \(\BB\), and we also see that we have a squared velocity coefficient in Ampere/Maxwell's law.  Let us factor this out explicitly so that \(\BE\) and \(\BB\) take dimensionally consistent form:
%
\begin{equation}\label{eqn:gaMaxwell:60}
\begin{aligned}
\tau &= \frac{t}{\sqrt{\mu_0\epsilon_0}}  \\
\spacegrad \cdot \BE &= \frac{\rho}{\epsilon_0} \\
\spacegrad \cdot \frac{\BB}{\sqrt{\mu_0\epsilon_0}} &= 0 \\
\spacegrad \cross \BE &= - \frac{\partial}{\partial \tau} \frac{\BB}{\sqrt{\mu_0\epsilon_0}} \\
\spacegrad \cross \frac{\BB}{\sqrt{\mu_0\epsilon_0}} &= \sqrt{\frac{\mu_0}{\epsilon_0}} \BJ + \frac{\partial \BE}{\partial \tau}
\end{aligned}
\end{equation}
%
\subsection{Refactoring the equations with the geometric product}

Now that things are dimensionally consistent, we are ready to group these equations using the geometric product
%
\begin{equation}
\BA \BB = \BA \cdot \BB + \BA \wedge \BB = \BA \cdot \BB + i \BA \cross \BB
\end{equation}
%
where \(i = \Be_1\Be_2\Be_3\) is the spatial pseudoscalar.  By grouping the divergence and curl terms for each of \(\BB\), and \(\BE\) we can write vector gradient equations
for each of the Electric and Magnetic fields:

\begin{align}
\spacegrad \BE = \frac{\rho}{\epsilon_0} - i \frac{\partial}{\partial \tau} \frac{\BB}{\sqrt{\mu_0\epsilon_0}} \label{eqn:gaMax:grad_e} \\
\spacegrad \frac{\BB}{\sqrt{\mu_0\epsilon_0}} = i\sqrt{\frac{\mu_0}{\epsilon_0}} \BJ + i\frac{\partial \BE}{\partial \tau} \label{eqn:gaMax:grad_b}
\end{align}

Multiplication of \eqnref{eqn:gaMax:grad_b} with \(i\), and adding to \eqnref{eqn:gaMax:grad_e}, we have Maxwell's equations consolidated into:
%
\begin{equation}\label{eqn:gaMax:maxwelleb}
\spacegrad \left(\BE + i \frac{\BB}{\sqrt{\mu_0\epsilon_0}}\right) =
\left(\frac{\rho}{\epsilon_0} - \sqrt{\frac{\mu_0}{\epsilon_0}} \BJ\right)
- \frac{\partial}{\partial \tau} \left(\BE + \frac{i\BB}{\sqrt{\mu_0\epsilon_0}} \right)
\end{equation}
%
We see that we have a natural combined Electrodynamic field:
%
\begin{equation}
\BF = \epsilon_0\left(\BE + i \frac{\BB}{\sqrt{\mu_0\epsilon_0}}\right) = \epsilon_0\left(\BE + i c \BB\right)
\end{equation}
%
Note that here the \(\epsilon_0\) factor has been included as a convenience to remove it from the charge and current density terms later.  We have also looked ahead slightly and written:
%
\begin{equation}
c = \inv{\sqrt{\mu_0\epsilon_0}}
\end{equation}
%
The dimensional analysis above showed that this had dimensions of velocity.  This velocity is in fact the speed of light, and
we will see this more exactly when looking at the wave equation for electrodynamics.  Until that this can be viewed as a
nothing more than a convenient shorthand.

We use this to write (Maxwell's) \eqnref{eqn:gaMax:maxwelleb} as:
%
\begin{equation}\label{eqn:gaMax:maxwellvs}
\left(\spacegrad + \inv{c}\frac{\partial}{\partial t} \right) \BF = \rho - \frac{\BJ}{c}.
\end{equation}
%
These are still four equations, and the originals can be recovered by taking scalar, vector, bivector and trivector parts.  However, in this
consolidated form, we are able to see the structure more easily.

\subsection{Grouping by charge and current density}
\index{charge density}
\index{current density}

Before moving on to the wave equation, lets put equations \eqnref{eqn:gaMax:grad_e} and \eqnref{eqn:gaMax:grad_b} in a slightly more symmetric form,
grouping by charge and current density respectively:

\begin{align}
\spacegrad \BE + \frac{\partial ic\BB}{\partial ct} &= \frac{\rho}{\epsilon_0} \label{eqn:gaMax:gradE} \\
\spacegrad ic\BB + \frac{\partial \BE}{\partial ct} &= -\frac{\BJ}{\epsilon_0 c} \label{eqn:gaMax:gradB}
\end{align}

Here we see how spatial electric field variation and magnetic field time variation are related to charge density.  We also see the
opposite pairing, where
spatial magnetic field variation and electric field variation with time are related to current density.

TODO: examine Lorentz transformations of the coordinates here.

Perhaps the most interesting feature here is how the spacetime gradient ends up split across the \(\BE\) and \(\BB\) fields, but
it may not be worth revisiting this.  Let us move on.

\subsection{Wave equation for light}
\index{wave equation!light}

To arrive at the wave equation, we take apply the gradient twice to calculate the Laplacian.  First vector gradient is:
%
\begin{equation}\label{eqn:gaMax:gradF}
\spacegrad \BF = -\inv{c}\frac{\partial \BF}{\partial t} + \left(\rho - \frac{\BJ}{c}\right).
\end{equation}
%
Second application gives:
%
\begin{equation*}
\spacegrad^2 \BF = -\inv{c}\spacegrad\frac{\partial \BF}{\partial t} + \spacegrad \left(\rho - \frac{\BJ}{c}\right).
\end{equation*}
%
Assuming continuity sufficient for mixed partial equality, we can swap the order of spatial and time derivatives, and
substitute \eqnref{eqn:gaMax:gradF} back in.
%
\begin{equation}\label{eqn:gaMaxwell:120}
\begin{aligned}
\spacegrad^2 \BF
&= -\inv{c}\frac{\partial}{\partial t}\left( -\inv{c}\frac{\partial \BF}{\partial t} + \left(\rho - \frac{\BJ}{c}\right) \right) + \spacegrad \left(\rho - \frac{\BJ}{c}\right) \\
\end{aligned}
\end{equation}
%
Or,
%
\begin{equation} \label{eqn:gaMax:wave}
\left(\spacegrad^2 - \inv{c^2}\partial_{tt} \right)\BF = \left(\spacegrad - \inv{c}\partial_{t} \right) \left(\rho - \frac{\BJ}{c}\right)
\end{equation}
%
Now there are a number of things that can be read out of this equation.  The first is that in a charge and current free region the electromagnetic field is described by an unforced wave equation:
%
\begin{equation}
\left(\spacegrad^2 - \inv{c^2}\partial_{tt} \right)\BF = 0
\end{equation}
%
This confirms the meaning that was assigned to \(c\).
It is the speed that an electrodynamic wave propagates in a charge and current free region of space.

\subsection{Charge and current density conservation}
\index{charge conservation}
\index{current density conservation}

Now, lets look at the right hand side of \eqnref{eqn:gaMax:wave} a bit closer:
%
\begin{equation}\label{eqn:gaMaxwell:140}
\begin{aligned}
\left(\spacegrad -\partial_{ct} \right) \left(\rho - \frac{\BJ}{c}\right)
&= - \inv{c}\left(\frac{\partial \rho}{\partial t} + \spacegrad \cdot \BJ \right) + \spacegrad \rho - \inv{c}\spacegrad \wedge \BJ + \inv{c^2}\frac{\partial \BJ}{\partial t}
\end{aligned}
\end{equation}
%
Compare this to the left hand side of \eqnref{eqn:gaMax:wave} which has only vector and bivector parts.  This implies that the scalar components of the right hand side are zero.  Specifically:
%
\begin{equation*}
\frac{\partial \rho}{\partial t} + \spacegrad \cdot \BJ = 0
\end{equation*}
%
This is a statement of charge conservation, and is more easily interpreted in integral form:
%\partial_t \rho = -\spacegrad \cdot \BJ
%
\begin{equation}
-\int_{S(\text{closed boundary of V})} \BJ \cdot \ncap dA = \frac{\partial}{\partial t} \int_V \rho dV = \frac{\partial Q_{enc}}{\partial t}
\end{equation}
%
FIXME: think about signs fully here.

The flux of the current density vector through a closed surface equals the time rate of change of the charge enclosed by that volume (ie: the current).  This could perhaps be viewed as the definition of the current density itself.  This fact would probably be more obvious if I did the math myself to demonstrate exactly how to take Maxwell's equations in integral form and convert those to their differential form.  In lieu of having done that proof myself I can at least determine this as a side effect of a bit of math.

\subsection{Electric and Magnetic field dependence on charge and current density}

Removing the explicit scalar terms from \eqnref{eqn:gaMax:wave} we have:
%
\begin{equation*}
\left(\spacegrad^2 - \partial_{ct, ct}\right) \BF =
\inv{c } \left(\spacegrad c \rho + \frac{\partial \BJ}{\partial c t} \right)
- \inv{c } \spacegrad \wedge \BJ
\end{equation*}
%
This shows explicitly how the charge and current forced wave equations for the
electric and magnetic fields is split:
%
\begin{equation*}
\left(\spacegrad^2 - \partial_{ct, ct}\right) \BE =
\inv{c } \left(\spacegrad c \rho + \frac{\partial \BJ}{\partial c t} \right)
\end{equation*}
%
\begin{equation*}
\left(\spacegrad^2 - \partial_{ct, ct}\right) \BB = - \inv{c^2 }  \spacegrad \cross \BJ
\end{equation*}
%
\subsection{Spacetime basis}

Now, if we look back to Maxwell's equation in the form of \eqnref{eqn:gaMax:maxwellvs}, we have a spacetime ``gradient'' with vector and scalar parts, an electrodynamic field with vector and trivector parts, and a charge and current density term with scalar and vector parts.

It is still rather confused, but it all works out, and one can recover the original four vector equations by taking scalar, vector, bivector, and trivector parts.

We want however to put this into a natural orderly fashion, and can do so if we use a normal bivector basis for all the spatial basis vectors, and factor out a basis vector from that for each of the scalar (timelike) factors.

Since bivectors over a Euclidean space have negative square, and this is not what we want for our Euclidean basis, and will have to
pick a bivector basis with a mixed metric.  We will see that this defines a Minkowski metric space.  Amazingly, by the simple
desire that we want to express
Maxwell's equations be written in the most orderly fashion, we arrive at the mixed signature spacetime metric that is the basis
of special relativity.
% (and we can use either signature as convienent so long as we do so consistently.)

Now, perhaps the reasons why to try to factor the spatial basis into a bivector basis are not obvious.  It is worth noting that
we have suggestions of conjugate operations above.  Examples of this are the charge and current terms with alternate signs, and
the alternation in sign in the wave equation itself.  Also worth pointing out is the natural appearance of a complex factor \(i\)
in Maxwell's equation coupled with the time term (that idea is explored more in ../maxwell/maxwell.pdf).  This coupling was
observed long ago and Minkowski's original paper refactors Maxwell's equation using it.  Now we have also seen that complex numbers
are isomorphic with a scalar plus vector representation.  Quaternions, which were originally ``designed'' to fit naturally
in Maxwell's equation and express the inherent structure are exactly this, a scalar and bivector sum.  There is a lot of history
that leads up to this idea, and the ideas here are not too surprising with some reading of the past attempts to put structure to these
equations.

On to the math...

Having chosen to find a bivector representation for our spatial basis vectors we write:
%
\begin{equation*}
\Be_i
= \gamma_i \wedge \gamma_0
= \gamma_i \gamma_0
= \gamma^0 \wedge \gamma^i
= \gamma^0 \gamma^i
\end{equation*}
%
For our Euclidean space we want
%
\begin{equation*}
(\Be_i)^2 = \gamma_i \gamma_0 \gamma_i \gamma_0 = -(\gamma_i)^2 (\gamma_0)^2 = 1
\end{equation*}
%
This implies the mixed signature:
%
\begin{equation*}
(\gamma_i)^2 = -(\gamma_0)^2 = \pm 1
\end{equation*}
%
We are free to pick either \(\gamma_0\) or \(\gamma_i\) to have a negative square, but following GAFP we use:
%
\begin{equation}\label{eqn:gaMaxwell:160}
\begin{aligned}
(\gamma_0)^2 &= 1 \\
(\gamma_i)^2 &= -1 \\
\gamma^0 &= \gamma_0 \\
\gamma^i &= -\gamma_i
\end{aligned}
\end{equation}
%
Now, lets translate the other scalar, vector, bivector, and trivector representations to use this alternate basis, and see what we get.  Start with the spacial pseudoscalar that is part of our magnetic field:
%
\begin{equation}\label{eqn:gaMaxwell:180}
\begin{aligned}
i
&= \Be_{123} \\
&= \gamma_{102030} \\
&= -\gamma_{012030} \\
&= \gamma_{012300} \\
&= \gamma_{0123}
\end{aligned}
\end{equation}
%
We see that the three dimensional pseudoscalar represented with this four dimensional basis is in fact also a pseudoscalar for that space.  Lets now use this to expand the trivector part of our electromagnetic field in this new basis:
%
\begin{equation}\label{eqn:gaMaxwell:200}
\begin{aligned}
i \BB
&= \sum i \Be_{i} B^i
&= \sum \gamma_{0123i0} B^i
&=
\gamma_{32} B^1
+\gamma_{13} B^2
+\gamma_{21} B^3
\end{aligned}
\end{equation}
%012310 = 011230 = 11 0230 = (11 00) 23
%012320 = -012230 = 22 0130 = (22 00) 13
%012330 = 33 0120 = (33 00) 21

So we see that our electromagnetic field has a bivector only representation with this mixed signature basis:
%
\begin{equation}
\BF = \BE + ic\BB = \gamma_{10} E^1 +\gamma_{20} E^2 +\gamma_{30} E^3 +\gamma_{32} c B^1 +\gamma_{13} c B^2 +\gamma_{21} c B^3
\end{equation}
%
Each of the possible bivector basis vectors is associated with a component of the combined electromagnetic field.
I had the signs wrong initially for the \(\BB\) components, but I think it is right now (and signature independent in fact).  ?  If I did get it wrong the idea is the same ... \(F\) is naturally
viewed as a pure bivector, which fits well with the fact that the tensor formulation is two completely antisymmetric rank two tensors.

Now, lets look at the spacetime gradient terms, first writing the spacial gradient in index form:
%
\begin{equation}\label{eqn:gaMaxwell:220}
\begin{aligned}
\spacegrad
&= \sum \Be^i \frac{\partial}{\partial x^i} \\
&= \sum \Be_i \frac{\partial}{\partial x^i} \\
&= \sum \gamma_i \gamma_0 \frac{\partial}{\partial x^i} \\
&= \gamma_0 \sum \gamma^i \frac{\partial}{\partial x^i}.
\end{aligned}
\end{equation}
%
This allows the spacetime gradient to be written in vector form replacing the vector plus scalar formulation:
%
\begin{equation}\label{eqn:gaMaxwell:240}
\begin{aligned}
\spacegrad + \partial_{ct}
&= \gamma_0 \sum \gamma^i \frac{\partial}{\partial x^i} + \partial_{ct} \\
&= \gamma_0 \left(\sum \gamma^i \frac{\partial}{\partial x^i} + \gamma^0 \partial_{ct} \right) \\
&= \gamma_0 \sum \gamma^{\mu} \frac{\partial}{\partial x^i}
\end{aligned}
\end{equation}
%
Observe that after writing \(x^0 = ct\) we can factor out the \(\gamma_0\), and write the spacetime gradient in pure vector form, using this mixed signature basis.

Now, let us do the same thing for the charge and current density terms, writing \(\BJ = e_i J^i\):
%
\begin{equation}\label{eqn:gaMaxwell:260}
\begin{aligned}
\rho - \frac{\BJ}{c}
&= \inv{c} \left( c\rho - \sum \Be_i J^i \right) \\
&= \inv{c} \left( c\rho - \sum \gamma_i \gamma_0 J^i \right) \\
&= \inv{c} \left( c\rho + \gamma_0 \sum \gamma_i J^i \right) \\
&= \gamma_0 \inv{c} \left( \gamma_0 c \rho + \sum \gamma_i J^i \right) \\
\end{aligned}
\end{equation}
%
Thus after writing \(J^0 = c \rho\), we have:
%
\begin{equation*}
\rho - \frac{\BJ}{c} = \gamma_0 \inv{c} \sum \gamma_{\mu} J^{\mu}
\end{equation*}
%
Putting these together and canceling out the leading \(\gamma_0\) terms we have the final result:
%
\begin{equation}
\sum \gamma^{\mu} \frac{\partial}{\partial x^i} \BF = \inv{c} \sum \gamma_{\mu} J^{\mu}.
\end{equation}
%
Or with a four-gradient \(\grad = \sum \gamma^{\mu} \frac{\partial}{\partial x^i}\), and four current \(J = \sum \gamma_{\mu} J^{\mu}\), we have Maxwell's equation in their most compact and powerful form:
%
\begin{equation}\label{eqn:gaMax:maxwell}
\grad \BF = \frac{J}{c}.
\end{equation}
%
\subsection{Examining the GA form Maxwell equation in more detail}

From \eqnref{eqn:gaMax:maxwell}, the wave equation becomes quite simple to derive.  Lets look at this again from this point of
view.  Applying the gradient we have:
%
\begin{equation}
\grad^2 \BF = \frac{\grad J}{c}.
\end{equation}
%
\begin{equation}
\grad^2 = \grad \cdot \grad = \sum (\gamma^{\mu})^2 \partial_{x^{\mu},x^{\mu}} = -\spacegrad^2 + \inv{c^2}\partial_{tt}.
\end{equation}
%
Thus for a charge and current free region, we still have the wave equation.

Now, lets look at the right hand side, and verify that it meets the expectations:
%
\begin{equation}\label{eqn:gaMax:waveright}
\inv{c}\grad J = \inv{c}\left(\grad \cdot J + \grad \wedge J\right)
\end{equation}
%
First thing to observe is that the left hand side is a pure spacetime bivector, which implies that the scalar part of
\eqnref{eqn:gaMax:waveright} is zero as we previously observed.  Lets verify that this is still the charge conservation
condition:
%
\begin{equation}\label{eqn:gaMaxwell:280}
\begin{aligned}
0
&= \grad \cdot J \\
&= (\sum \gamma^{\mu} \partial_{\mu}) \cdot \sum \gamma_{\nu} J^{\nu} \\
&= \sum \gamma^{\mu} \cdot \gamma_{\nu} \partial_{\mu} J^{\nu} \\
&= \sum \delta^{\mu}_{\nu} \partial_{\mu} J^{\nu} \\
&= \sum \partial_{\mu} J^{\mu} \\
&= \partial_{ct}(c \rho) + \sum \partial_{i} J^{i} \\
\end{aligned}
\end{equation}
%
This is our previous result:
%
\begin{equation}
\frac{\partial \rho}{\partial_{t}} + \spacegrad \cdot \BJ = 0
\end{equation}
%
This allows a slight simplification of the current forced wave equation for an electrodynamic field, by taking just the bivector
parts:
%
\begin{equation}\label{eqn:gaMax:waveF}
\left(\spacegrad^2 - \inv{c^2}\partial_{tt}\right) \BF = -\grad \wedge \frac{J}{c}
\end{equation}
%
Now we know how to solve the left hand side of this equation in its homogeneous form, but the four space curl term on the right is
new.
%, and we need some new tools to deal with it.  Next step in the learning process looks like it has to be geometric calculus.

%Perhaps do not need new tools.
This is really a set of six equations, subject to coupled boundary value conditions.  Written this out in components, one for each \(F \cdot (\gamma^{\nu} \wedge \gamma^{\mu})\) term and the corresponding terms of the right hand side one ends up with:
%
\begin{equation*}
-\grad^2 \BE = \spacegrad \rho/\epsilon_0 + \mu_0 \partial_t{\BJ}
\end{equation*}
\begin{equation*}
-\grad^2 \BB = -\mu_0 \spacegrad \cross \BJ
\end{equation*}
%
I have not bothered transcribing my notes for how to get this.  One way (messy) was starting with \eqnref{eqn:gaMax:waveF} and dotting with \(\gamma^{\nu \mu}\) to calculate the tensor \(F^{\mu\nu}\) (components of which are \(E\) and \(B\) components).  Doing the same for the spacetime curl term the end result is:
%
\begin{equation*}
(\grad \wedge J) \cdot (\gamma^{\nu\mu}) = \partial_{\mu}J^{\nu} (\gamma^{\mu})^2 - \partial_{\nu}J^{\mu} (\gamma^{\nu})^2
\end{equation*}
%
For a spacetime split of indices one gets the \(\spacegrad\rho\), and \(\partial_t \BJ\) term, and for a space-space pair of indices one gets the spacial curl in the \(\BB\) equation.

An easier starting point for this is actually using equations \eqnref{eqn:gaMax:gradE} and \eqnref{eqn:gaMax:gradB} since they are already split into \(\BE\), and \(\BB\) fields.

\subsection{Minkowski metric}

Having observed that a mixed signature bivector basis with a space time mix of underlying basis vectors is what we want to
express Maxwell's equation in its most simple form, now lets step back and look at that in a bit more detail.  In particular
lets examine the dot product of a four vector with such a basis.  Our current density four vector is one such vector:
%
\begin{equation}\label{eqn:gaMaxwell:300}
\begin{aligned}
J^2 = J \cdot J = \sum (J^{\mu})^2 (\gamma_{\mu})^2 = (c\rho)^2 - \BJ^2
\end{aligned}
\end{equation}
%
The coordinate vector that is forms the partials of our four gradient is another such vector:
%
\begin{equation*}
x = (ct, x^1, x^2, x^3) = \sum \gamma_{\mu} x^{\mu}
\end{equation*}
%
Again, the length applied to this vector is:
%
\begin{equation}
x^2 = x \cdot x = (ct)^2 - \Bx^2
\end{equation}
%
As a result of nothing more than a desire to put Maxwell's equations into structured form, we have the special relativity metric
of Minkowski and Einstein.

   %
% Copyright � 2012 Peeter Joot.  All Rights Reserved.
% Licenced as described in the file LICENSE under the root directory of this GIT repository.
%

%
%
\chapter{Macroscopic Maxwell's equation}
\index{Maxwell's equation}
\label{chap:macroscopicMaxwell}
%\date{May 28, 2009.  macroscopicMaxwell.tex}

\section{Motivation}

In \citep{jackson1975cew} the macroscopic Maxwell's equations are given as

\begin{equation}\label{eqn:macroMax:MaxwellMixedFields}
\begin{aligned}
\spacegrad \cdot \BD &= 4 \pi \rho \\
\spacegrad \cross \BH - \inv{c} \PD{t}{\BD} &= \frac{4 \pi}{c} \BJ \\
\spacegrad \cross \BE + \inv{c} \PD{t}{\BB} &= 0 \\
\spacegrad \cdot \BB &= 0
\end{aligned}
\end{equation}

The \(\BH\) and \(\BD\) fields are then defined in terms of dipole, and quadrupole
fields

\begin{equation}\label{eqn:macroMax:dipoleCoordinates}
\begin{aligned}
D_\alpha &= E_\alpha + 4\pi \left( P_\alpha - \sum_\beta \PD{x_\beta}{Q'_{\alpha\beta}} + \cdots\right) \\
H_\alpha &= B_\alpha - 4\pi \left( M_\alpha + \cdots\right)
\end{aligned}
\end{equation}

Can this be put into the Geometric Algebra formulation that works so
nicely for microscopic Maxwell's equations, and if so what will it look like?

\section{Consolidation attempt}

Let us try this, writing
\begin{equation}\label{eqn:macroscopicMaxwell:20}
\begin{aligned}
\BP &= \sigma^\alpha \left( P_\alpha - \sum_\beta \PD{x_\beta}{Q'_{\alpha\beta}} + \cdots\right) \\
\BM &= \sigma^\alpha \left( M_\alpha + \cdots \right)
\end{aligned}
\end{equation}

We can then express the \(\BE\), \(\BB\) in terms of the derived fields

\begin{equation}\label{eqn:macroscopicMaxwell:40}
\begin{aligned}
\BE &= \BD - 4\pi \BP \\
\BB &= \BH + 4\pi \BM
\end{aligned}
\end{equation}

and in turn can write the macroscopic Maxwell equations \eqnref{eqn:macroMax:MaxwellMixedFields}
in terms of just the derived fields, the material properties, and the charges and currents

\begin{equation}\label{eqn:macroscopicMaxwell:60}
\begin{aligned}
\spacegrad \cdot \BD &= 4 \pi \rho \\
\spacegrad \cross \BH - \inv{c} \PD{t}{\BD} &= \frac{4 \pi}{c} \BJ \\
\spacegrad \cross \BD + \inv{c} \PD{t}{ \BH } &= 4 \pi \spacegrad \cross \BP + \frac{4\pi}{c} \PD{t}{ \BM }  \\
\spacegrad \cdot \BH &= - 4 \pi \spacegrad \cdot \BM \\
\end{aligned}
\end{equation}

Now, using \(\Ba \cross \Bb = -i (\Ba \wedge \Bb)\), we have

\begin{equation}\label{eqn:macroscopicMaxwell:80}
\begin{aligned}
\spacegrad \cdot \BD &= 4 \pi \rho \\
i \spacegrad \wedge \BH + \inv{c} \PD{t}{\BD} &= -\frac{4 \pi}{c} \BJ \\
\spacegrad \wedge \BD + \inv{c} \PD{t}{ i\BH } &= 4 \pi i \spacegrad \cross \BP + \frac{4\pi}{c} \PD{t}{ i \BM }  \\
i \spacegrad \cdot \BH &= - 4 \pi i \spacegrad \cdot \BM \\
\end{aligned}
\end{equation}

Summing these in pairs with \(\spacegrad \Ba = \spacegrad \cdot \Ba + \spacegrad \wedge \Ba\), and writing \(\PDi{(ct)}{} = \partial_0\) we have

\begin{equation}\label{eqn:macroscopicMaxwell:100}
\begin{aligned}
\spacegrad \BD + \partial_0 {i\BH } &= 4 \pi \rho + 4 \pi \spacegrad \wedge \BP + {4\pi} \partial_0 { i \BM }  \\
i \spacegrad \BH + \partial_0 {\BD} &= -\frac{4 \pi}{c} \BJ - 4 \pi i \spacegrad \cdot \BM \\
\end{aligned}
\end{equation}

Note that while had \(i\spacegrad \cdot \Ba \ne \spacegrad \cdot (i\Ba)\), and
\(i\spacegrad \wedge \Ba \ne \spacegrad \wedge (i\Ba)\)
(instead \(i\spacegrad \cdot \Ba = \spacegrad \wedge (i\Ba)\), and
\(i\spacegrad \wedge \Ba = \spacegrad \cdot (i\Ba)\)), but now that these are summed we can take advantage of the fact that the pseudoscalar \(i\)
commutes with all vectors (such as \(\spacegrad\)).  So, summing once again we have

\begin{equation}\label{eqn:macroscopicMaxwell:120}
\begin{aligned}
(\partial_0 + \spacegrad)(\BD + i\BH ) &=
\frac{4 \pi}{c} \left( c \rho - \BJ \right)
+ {4 \pi} \left( \spacegrad \wedge \BP + \partial_0 { i \BM }  - \spacegrad \wedge (i\BM) \right)
\\
\end{aligned}
\end{equation}

Finally, premultiplication by \(\gamma_0\), where \(\BJ = \sigma_k J^k = \gamma_k \gamma_0 J^k\), and \(\spacegrad = \sum_k \gamma_k \gamma_0 \partial_k\) we have

\begin{equation}\label{eqn:macroscopicMaxwell:140}
\begin{aligned}
\gamma^\mu \partial_\mu (\BD + i \BH)
&=
\frac{4 \pi}{c} \left( c \rho \gamma_0 + J^k \gamma_k \right)
+ {4 \pi \gamma_0} \left( \spacegrad \wedge \BP + \partial_0 { i \BM }  - \spacegrad \wedge (i\BM) \right)
\end{aligned}
\end{equation}

With
\begin{equation}\label{eqn:macroscopicMaxwell:160}
\begin{aligned}
J^0 &= c \rho \\
J &= \gamma_\mu J^\mu \\
\grad &= \gamma^\mu \partial_\mu \\
F &= \BD + i\BH
\end{aligned}
\end{equation}

For the remaining terms we have \(\spacegrad \wedge \BP, i\BM \in \Span\{\gamma_a \gamma_b\}\), and \(\gamma_0 \spacegrad \wedge (iM) \in \Span{ \gamma_1 \gamma_2 \gamma_3}\), so between the three of these we have a (Dirac) trivector, so it would be reasonable to write

\begin{equation}\label{eqn:macroscopicMaxwell:180}
\begin{aligned}
T &= {\gamma_0} \left( \spacegrad \wedge \BP + \partial_0 { i \BM }  - \spacegrad \wedge (i\BM) \right) \in \Span\{ \gamma_\mu \wedge \gamma_\nu \wedge \gamma_\sigma\}
\end{aligned}
\end{equation}

Putting things back together we have

\begin{equation}\label{eqn:macroscopicMaxwell:200}
\begin{aligned}
\grad F &= \frac{4\pi}{c} J + 4\pi T
\end{aligned}
\end{equation}

This has a nice symmetry, almost nicer than the original microscopic version of Maxwell's equation since we now have matched grades (vector plus trivector in the Dirac vector space) on both sides of the equation.

\subsection{Continuity equation}
\index{continuity equation}

Also observe that interestingly we still have the same continuity equation as in the microscopic case.  Application of another spacetime gradient and then selecting scalar grades we have

\begin{equation}\label{eqn:macroscopicMaxwell:220}
\begin{aligned}
\gpgradezero{ \grad \grad F } &= 4 \pi \gpgradezero{ \grad \left( \frac{J}{c} + T \right) }  \\
\grad^2 \gpgradezero{ F } &= \\
&= \frac{ 4 \pi }{c} \gpgradezero{ J } \\
&= \frac{ 4 \pi }{c} \partial_\mu J^\mu \\
\end{aligned}
\end{equation}

Since \(F\) is a Dirac bivector it has no scalar part, so this whole thing is zero by the grade selection on the LHS.  So, from the RHS we have

\begin{equation}\label{eqn:macroscopicMaxwell:240}
\begin{aligned}
0 &= \partial_\mu J^\mu \\
&= \inv{c} \PD{t}{c\rho} + \partial_k J^k \\
&= \PD{t}{\rho} + \spacegrad \cdot \BJ
\end{aligned}
\end{equation}

Despite the new trivector term in the equation due to the matter properties!

   %
% Copyright � 2012 Peeter Joot.  All Rights Reserved.
% Licenced as described in the file LICENSE under the root directory of this GIT repository.
%

%
%
\chapter{Expressing wave equation exponential solutions using four vectors}\label{chap:PJwaveFourVector}
\index{wave equation}
\index{four vector}
%\date{Nov 30, 2008.  waveEqn.tex}

\section{Mechanical Wave equation Solutions}

For the unforced wave equation in 3D one wants solutions to

\begin{equation}\label{eqn:wave_eqn:waveEquation}
\begin{aligned}
\left( \inv{\Bv^2} \partial_{tt} - \sum_{j=1}^3 \partial_{jj}\right) \phi = 0
\end{aligned}
\end{equation}

For the single spatial variable case one can verify that
\(\phi = f( \Bx \pm \Abs{\Bv} t)\) is a solution for any function \(f\).  In particular \(\phi = \exp(i (\pm \Abs{\Bv} t + x))\) is a solution.  Similarly
\(\phi = \exp(i (\pm \Abs{\Bv} t + \kcap \cdot \Bx))\) is a solution in the 3D case.

Can the relativistic four vector notation be used to put this in a more symmetric form with respect to time and position?  For the four
vector

\begin{equation}\label{eqn:waveEqn:20}
\begin{aligned}
x = x^\mu \gamma_\mu
\end{aligned}
\end{equation}

Lets try the following as a possible solution to \eqnref{eqn:wave_eqn:waveEquation}

\begin{equation}\label{eqn:waveEqn:40}
\begin{aligned}
\phi = \exp(i k \cdot x)
\end{aligned}
\end{equation}

verifying that this can be a solution, and determining the constraints required on the four vector \(k\).

Observe that

\begin{equation}\label{eqn:waveEqn:60}
\begin{aligned}
x \cdot k = x^\mu k_\mu
\end{aligned}
\end{equation}

so

\begin{equation}\label{eqn:waveEqn:80}
\begin{aligned}
\phi_\mu &= i k_\mu \\
\phi_{\mu\mu} &= (i k_\mu)^2 \phi = -(k_\mu)^2 \phi
\end{aligned}
\end{equation}

Since \(\partial_t = c\partial_0\), we have \(\phi_tt = c^2 \phi_{00}\), and

\begin{equation}\label{eqn:waveEqn:100}
\begin{aligned}
\left( \inv{\Bv^2} \partial_{tt} - \sum_{j=1}^3 \partial_{jj}\right) \phi &=
\left( -\inv{\Bv^2} c^2 {k_0}^2 - \sum_{j=1}^3 -(k_j)^2\right) \phi \\
\end{aligned}
\end{equation}

For equality with zero, and \(\Bbeta = \Bv/c\), we require

\begin{equation}\label{eqn:waveEqn:120}
\begin{aligned}
\Bbeta^2 = \frac{(k_0)^2}{\sum_j (k_j)^2}
\end{aligned}
\end{equation}

Now want the components of \(k = k_\mu \gamma^\mu\) in terms of \(k\) directly.  First

\begin{equation}\label{eqn:waveEqn:140}
\begin{aligned}
k_0 = k \cdot \gamma_0
\end{aligned}
\end{equation}

The spacetime relative vector for \(k\) is

\begin{equation}\label{eqn:waveEqn:160}
\begin{aligned}
\Bk &= k \wedge \gamma_0 = \sum k_\mu \gamma^\mu \wedge \gamma_0 = (\gamma_1)^2 \sum_j k_j \sigma_j \\
\Bk^2 &= (\pm 1)^2 \sum_j (k_j)^2
\end{aligned}
\end{equation}

So the constraint on the four vector parameter \(k\) is
\begin{equation}\label{eqn:waveEqn:180}
\begin{aligned}
\Bbeta^2
&= \frac{(k_0)^2}{\sum_j (k_j)^2} \\
&= \frac{(k \cdot \gamma_0)^2}{(k \wedge \gamma_0)^2} \\
\end{aligned}
\end{equation}

It is interesting to compare this to the relative spacetime bivector for \(x\)

\begin{equation}\label{eqn:waveEqn:200}
\begin{aligned}
v &= \frac{dx}{d\tau} = c \frac{dt}{d\tau} \gamma_0 + \frac{dx^i}{d\tau} \gamma_i \\
v \cdot \gamma^0 &= \frac{dx}{d\tau} \cdot \gamma^0 = c \frac{dt}{d\tau} \\
v \wedge \gamma_0 &= \frac{dx}{d\tau} \wedge \gamma_0 \\
&= \frac{dx^i}{d\tau} \sigma_i \\
&= \frac{dx^i}{dt} \frac{dt}{d\tau} \sigma_i \\
\end{aligned}
\end{equation}

\begin{equation}\label{eqn:waveEqn:220}
\begin{aligned}
\Bv/c
&= \frac{d (x^i \sigma_i) }{dt} \\
&= \frac{v \wedge \gamma_0}{ v \cdot \gamma^0 }
\end{aligned}
\end{equation}

So, for \(\phi = \exp(i k \cdot x)\) to be a solution to the wave equation for a wave traveling with velocity \(\Abs{\Bv}\), the constraint on k
in terms of proper velocity \(v\) is

\begin{equation}\label{eqn:waveEqn:240}
\begin{aligned}
\Abs{\frac{k \wedge \gamma_0}{ k \cdot \gamma^0 }}^{-1} &=
\Abs{\frac{v \wedge \gamma_0}{ v \cdot \gamma^0 }}
\end{aligned}
\end{equation}

So we see the relative spacetime vector of \(k\) has an inverse relationship with the relative spacetime velocity vector \(v = dx/d\tau\).

   %
% Copyright � 2012 Peeter Joot.  All Rights Reserved.
% Licenced as described in the file LICENSE under the root directory of this GIT repository.
%

%
%
\chapter{Gaussian Surface invariance for radial field}
\label{chap:gaussianSurface}
%\date{November 22, 2008.  gaussianSurface.tex}

\section{Flux independence of surface}

\imageFigure{../figures/gabook/surface_flux_element}{Flux through tilted spherical surface element}{fig:surface_flux_element}{0.4}

In \citep{purcell1963eam}, section \(1.10\) is a demonstration that the flux
through any closed surface is the same as that through a sphere.

A similar demonstration of the same is possible using a spherical polar basis \(\{\rcap, \thetacap, \phicap\}\) with an element of surface area that is tilted slightly as illustrated in \cref{fig:surface_flux_element}.

The tangential surface on the sphere at radius \(r\) will have bivector

\begin{equation}\label{eqn:gaussianSurface:20}
\begin{aligned}
d\BA_r = r^2 d\theta d\phi \thetacap\phicap
\end{aligned}
\end{equation}

where \(d\theta\), and \(d\phi\) are the subtended angles (should have put them in the figure).

Now, as in the figure we want to compute the bivector for the tilted surface at radius \(R\).  The vector \(\Bu\) in the figure is required.
This is \(\rcap R + R d\theta \thetacap - \rcap(R + dr)\), so the bivector for that area element is

\begin{equation}\label{eqn:gaussianSurface:40}
\begin{aligned}
\left(R \rcap + R d\theta \thetacap - (R + dr) \rcap \right) \wedge {R d\theta \phicap}
&= \left(R d\theta \thetacap - dr \rcap \right) \wedge {R d\phi \phicap} \\
\end{aligned}
\end{equation}

For
\begin{equation}\label{eqn:gaussianSurface:60}
\begin{aligned}
d\BA_R = R^2 d\theta d\phi \thetacap \phicap - R dr d\phi \rcap \phicap
\end{aligned}
\end{equation}

Now normal area elements can be calculated by multiplication with a \R{3} pseudoscalar such as \(I = \rcap \thetacap \phicap\).

\begin{equation}\label{eqn:gaussianSurface:80}
\begin{aligned}
\ncap_r \Abs{d\BA_r}
&= r^2 d\theta d\phi \rcap \thetacap \phicap \thetacap\phicap \\
&= -r^2 d\theta d\phi \rcap \\
\end{aligned}
\end{equation}

And

\begin{equation}\label{eqn:gaussianSurface:100}
\begin{aligned}
\ncap_R \Abs{d\BA_R}
&= \rcap \thetacap \phicap \left( R^2 d\theta d\phi \thetacap \phicap - R dr d\phi \rcap \phicap \right) \\
&= - R^2 d\theta d\phi \rcap - R dr d\phi \thetacap
\end{aligned}
\end{equation}

Calculating \(\BE \cdot \ncap dA\) for the spherical surface element at radius \(r\) we have

\begin{equation}\label{eqn:gaussianSurface:120}
\begin{aligned}
\BE(r) \cdot \ncap_r \Abs{d\BA_r}
&= \inv{4 \pi \epsilon_0 r^2} q \rcap \cdot (-r^2 d\theta d\phi \rcap) \\
&= \frac{-d\theta d\phi q}{4 \pi \epsilon_0}
\end{aligned}
\end{equation}

and for the tilted surface at \(R\)

\begin{equation}\label{eqn:gaussianSurface:140}
\begin{aligned}
\BE(R) \cdot \ncap_R \Abs{d\BA_R}
&= \frac{q}{4 \pi \epsilon_0 R^2} \rcap \cdot \left(- R^2 d\theta d\phi \rcap - R dr d\phi \thetacap \right) \\
&= \frac{-d\theta d\phi q}{4 \pi \epsilon_0}
\end{aligned}
\end{equation}

The \(\thetacap\) component of the surface normal has no contribution to the flux since it is perpendicular to the outwards (\(\rcap\) facing) field.  Here the particular normal to the surface happened to be inwards facing due to choice of the pseudoscalar, but because the normals chosen in each case had the same orientation this does not make a difference to the equivalence result.

\subsection{Suggests dual form of Gauss's law can be natural}
\index{Gauss's law}

The fact that the bivector area elements work well to describe the surface
can also be used to write Gauss's law in an alternate form.  Let \(\ncap dA = -I d\BA\)

\begin{equation}\label{eqn:gaussianSurface:160}
\begin{aligned}
\BE \cdot \ncap dA
&= -\BE \cdot (I d\BA) \\
&= \frac{-1}{2} ( \BE I d\BA + I d\BA \BE ) \\
&= \frac{-I}{2} ( \BE d\BA + d\BA \BE ) \\
&= -I ( \BE \wedge d\BA )
\end{aligned}
\end{equation}

So for

\begin{equation}\label{eqn:gaussianSurface:180}
\begin{aligned}
\int \BE \cdot \ncap dA
&= \int \frac{\rho}{\epsilon_0} dV
\end{aligned}
\end{equation}

with \(d\BV = I dV\), we have Gauss's law in dual form:

\begin{equation}\label{eqn:gaussianSurface:200}
\begin{aligned}
\int \BE \wedge d\BA &= \int \frac{\rho}{\epsilon_0} d\BV
\end{aligned}
\end{equation}

Writing Gauss's law in this form it becomes almost obvious that we can
deform the surface without changing the flux, since all the non-tangential
surface elements will have an \(\rcap\) factor and thus produce a zero
once wedged with the radial field.


   %
% Copyright � 2012 Peeter Joot.  All Rights Reserved.
% Licenced as described in the file LICENSE under the root directory of this GIT repository.
%

%
%
\chapter{Electrodynamic wave equation solutions}\label{chap:PJemWave}
\index{wave equation}
%\date{Jan 25, 2009.  emWave.tex}

\section{Motivation}

In \chapcite{PJwaveFourVector} four vector solutions to the mechanical wave
equations were explored.  What was obviously missing from that
was consideration of the special case for \(\Bv^2 = c^2\).

Here solutions to the electrodynamic wave equation will be examined.
Consideration of such solutions in more detail will is expected
to be helpful
as background for the more complex study of quantum (matter) wave equations.

\section{Electromagnetic wave equation solutions}

For electrodynamics our equation to solve is

\begin{equation}\label{eqn:emWave:20}
\begin{aligned}
\grad F = J/\epsilon_0 c
\end{aligned}
\end{equation}

For the unforced (vacuum) solutions, with
\(F = \grad \wedge A\), and the Coulomb gauge \(\grad \cdot A = 0\) this
reduces to

\begin{equation}\label{eqn:emWave:40}
\begin{aligned}
0
&= \left((\gamma^\mu)^2 \partial_{\mu\mu}\right) A  \\
&= \left( \inv{c^2}\partial_{tt} -\partial_{jj} \right) A
\end{aligned}
\end{equation}

These equations have the same form as the mechanical wave equation
where the wave velocity \(\Bv^2 = c^2\) is the speed of light

\begin{equation}\label{eqn:em_wave:waveEquation}
\begin{aligned}
\left( \inv{\Bv^2} \partial_{tt} - \sum_{j=1}^3 \partial_{jj}\right) \psi = 0
\end{aligned}
\end{equation}

\subsection{Separation of variables solution of potential equations}
\index{separation of variables}

Let us solve this using separation of variables, and write \(A^\nu = X Y Z T = \Pi_{\mu} X^{\mu}\)

From this we have

\begin{equation}\label{eqn:emWave:60}
\begin{aligned}
\sum_\mu (\gamma^\mu)^2 \frac{(X^\mu)''}{X^\mu} = 0
\end{aligned}
\end{equation}

and can proceed with the normal procedure of assuming that a solution can be
found by separately equating each term to a constant.  Writing those
constants explicitly as \((m_\mu)^2\), which we allow to be potentially complex
we have (no sum)

\begin{equation}\label{eqn:emWave:80}
\begin{aligned}
X^\mu = \exp\left( \pm \sqrt{(\gamma^\mu)^2} m_\mu x^\mu \right)
\end{aligned}
\end{equation}

Now, let \(k_\mu = \pm \sqrt{(\gamma^\mu)^2} m_\mu\), folding any sign variation
and complex factors into these constants.  Our complete solution
is thus

\begin{equation}\label{eqn:emWave:100}
\begin{aligned}
\Pi_\mu X^\mu = \exp\left( \sum k_\mu x^\mu \right)
\end{aligned}
\end{equation}

However, for this to be a solution, the wave equation imposes the constraint

\begin{equation}\label{eqn:emWave:120}
\begin{aligned}
\sum_\mu (\gamma^\mu)^2 (k_\mu)^2 = 0
\end{aligned}
\end{equation}

Or
\begin{equation}\label{eqn:emWave:140}
\begin{aligned}
(k_0)^2 - \sum_j (k_j)^2 = 0
\end{aligned}
\end{equation}

Summarizing each potential term has a solution expressible in terms of
null "wave-number" vectors \(K_\nu\)

\begin{equation}\label{eqn:em_wave:potentialSolution}
\begin{aligned}
A_\nu &= \exp\left( K_\nu \cdot x \right)  \\
\Abs{K_\nu} &= 0
\end{aligned}
\end{equation}

\subsection{Faraday bivector and tensor from the potential solutions}
\index{Faraday bivector}

From the components of the potentials
\eqnref{eqn:em_wave:potentialSolution}
we can compute the curl for the complete
field.  That is

\begin{equation}\label{eqn:emWave:160}
\begin{aligned}
F &= \grad \wedge A \\
A &= \gamma^\nu \exp\left( K_\nu \cdot x \right)  \\
\end{aligned}
\end{equation}

This is

\begin{equation}\label{eqn:emWave:180}
\begin{aligned}
F
&= \left(\gamma^\mu \wedge \gamma^\nu\right) \partial_\mu \exp\left( K_\nu \cdot x \right) \\
&= \left(\gamma^\mu \wedge \gamma^\nu\right) \partial_\mu \exp\left( \gamma^\alpha K_{\nu\alpha} \cdot \gamma_\sigma x^\sigma \right) \\
&= \left(\gamma^\mu \wedge \gamma^\nu\right) \partial_\mu \exp\left( K_{\nu\sigma} x^\sigma \right) \\
%&= \left(\gamma^\mu \wedge \gamma^\nu\right) K_{\nu\sigma} \delta_\mu\sigma \exp\left( K_{\nu\sigma} x^\sigma \right) \\
&= \left(\gamma^\mu \wedge \gamma^\nu\right) K_{\nu\mu} \exp\left( K_{\nu\sigma} x^\sigma \right) \\
&= \left(\gamma^\mu \wedge \gamma^\nu\right) K_{\nu\mu} \exp\left( K_{\nu} \cdot x \right) \\
&= \left(\gamma^\mu \wedge \gamma^\nu\right)
\inv{2} \left( K_{\nu\mu} \exp\left( K_{\nu} \cdot x \right) - K_{\mu\nu} \exp\left( K_{\mu} \cdot x \right) \right) \\
\end{aligned}
\end{equation}

Writing our field in explicit tensor form

\begin{equation}\label{eqn:emWave:200}
\begin{aligned}
F = F_{\mu\nu} \gamma^\mu \wedge \gamma^\nu
\end{aligned}
\end{equation}

our vacuum solution is therefore

\begin{equation}\label{eqn:emWave:220}
\begin{aligned}
F_{\mu\nu} &= \inv{2} \left( K_{\nu\mu} \exp\left( K_{\nu} \cdot x \right) - K_{\mu\nu} \exp\left( K_{\mu} \cdot x \right) \right)
\end{aligned}
\end{equation}

but subject to the null wave number and Lorentz gauge constraints

\begin{equation}\label{eqn:emWave:240}
\begin{aligned}
\Abs{K_\mu} &= 0 \\
\grad \cdot \left(\gamma^\mu \exp\left( K_\mu \cdot x \right)\right) &= 0
\end{aligned}
\end{equation}

\subsection{Examine the Lorentz gauge constraint}
\index{Lorentz gauge}

That Lorentz gauge constraint on the potential is a curious looking beastie.  Let us expand that out in full to examine it closer

\begin{equation}\label{eqn:emWave:260}
\begin{aligned}
\grad \cdot \left(\gamma^\mu \exp\left( K_\mu \cdot x \right)\right)
&= \gamma^\alpha \partial_\alpha \cdot \left(\gamma^\mu \exp\left( K_\mu \cdot x \right)\right)  \\
&= \sum_\mu (\gamma^\mu)^2 \partial_\mu \exp\left( K_\mu \cdot x \right) \\
&= \sum_\mu (\gamma^\mu)^2 \partial_\mu \exp\left( \sum \gamma^\nu K_{\mu\nu} \cdot \gamma_\alpha x^\alpha \right) \\
&= \sum_\mu (\gamma^\mu)^2 \partial_\mu \exp\left( \sum K_{\mu\alpha} x^\alpha \right) \\
&= \sum_\mu (\gamma^\mu)^2 K_{\mu\mu} \exp\left( K_\mu \cdot x \right)
\end{aligned}
\end{equation}

If this must be zero for any \(x\) it must also be zero for \(x =0\), so the Lorentz gauge imposes an additional restriction on the
wave number four vectors \(K_\mu\)

\begin{equation}\label{eqn:emWave:280}
\begin{aligned}
\sum_\mu (\gamma^\mu)^2 K_{\mu\mu} = 0
\end{aligned}
\end{equation}

Expanding in time and spatial coordinates this is

\begin{equation}\label{eqn:emWave:300}
\begin{aligned}
K_{00} - \sum_j K_{jj} = 0
\end{aligned}
\end{equation}

One obvious way to satisfy this is to require that the tensor \(K_{\mu\nu}\) be diagonal, but since we also have the null vector requirement
on each of the \(K_\mu\) four vectors it is not clear that this is an acceptable choice.

\subsection{Summarizing so far}

We have found that our field solution has the form

\begin{equation}\label{eqn:em_wave:WorkingSolution}
\begin{aligned}
F_{\mu\nu} &= \inv{2} \left( K_{\nu\mu} \exp\left( K_{\nu} \cdot x \right) - K_{\mu\nu} \exp\left( K_{\mu} \cdot x \right) \right)
\end{aligned}
\end{equation}

Where the vectors \(K_\mu\) have coordinates
\begin{equation}\label{eqn:em_wave:WorkingSolutionDefinitions}
\begin{aligned}
K_\mu &= \gamma^\nu K_{\mu\nu}
%x &= \gamma_\mu x^\mu
\end{aligned}
\end{equation}

This last allows us to write the field tensor completely in tensor formalism

\begin{equation}\label{eqn:emWave:320}
\begin{aligned}
F_{\mu\nu} &= \inv{2} \left( K_{\nu\mu} \exp\left( K_{\nu\sigma} x^\sigma \right) - K_{\mu\nu} \exp\left( K_{\mu\sigma} x^\sigma \right) \right)
\end{aligned}
\end{equation}

Note that we also require the constraints

\begin{equation}\label{eqn:em_wave:WorkingSolutionConstraints}
\begin{aligned}
0 &= \sum_\mu (\gamma^\mu)^2 K_{\mu\mu} \\
0 &= \sum_\mu (\gamma^\mu)^2 (K_{\nu\mu})^2
\end{aligned}
\end{equation}

Alternately, calling out the explicit space time split of the constraint, we can
remove the explicit \(\gamma^\mu\) factors

\begin{equation}\label{eqn:emWave:340}
\begin{aligned}
0 = K_{00} - \sum_j K_{jj} = (K_{00})^2 - \sum_j (K_{jj} )^2
\end{aligned}
\end{equation}

%If each of \(K_{\mu0} \ne 0\) we could alternately remove the explicit \(\gamma^\mu\) factors and write
%
%\begin{align*}
%1 = \sum_j \frac{K_{jj}}{K_{00}} = \sum_j \left(\frac{K_{jj}}{K_{00}}\right)^2
%\end{align*}
%
%Is this any better?

\section{Looking for more general solutions}

\subsection{Using mechanical wave solutions as a guide}

In the mechanical wave equation, we had exponential solutions of the form

\begin{equation}\label{eqn:emWave:360}
\begin{aligned}
f(\Bx,t) = \exp\left( \Bk \cdot \Bx + \omega t \right)
\end{aligned}
\end{equation}

which were solutions to \eqnref{eqn:em_wave:waveEquation} provided that

\begin{equation}\label{eqn:emWave:380}
\begin{aligned}
\inv{\Bv^2} \omega^2 - \Bk^2 = 0.
\end{aligned}
\end{equation}

This meant that
\begin{equation}\label{eqn:emWave:400}
\begin{aligned}
\omega = \pm \Abs{\Bv} \Abs{\Bk}
\end{aligned}
\end{equation}

and our function takes the (hyperbolic) form, or (sinusoidal) form respectively

\begin{equation}\label{eqn:emWave:420}
\begin{aligned}
f(\Bx,t) &= \exp\left( \Abs{\Bk}\left( \kcap \cdot \Bx \pm \Abs{\Bv} t \right) \right) \\
f(\Bx,t) &= \exp\left( i \Abs{\Bk}\left( \kcap \cdot \Bx \pm \Abs{\Bv} t \right) \right)
\end{aligned}
\end{equation}

Fourier series superposition of the latter solutions can be used to express any spatially periodic function, while Fourier transforms
can be used to express the non-periodic cases.

These superpositions, subject to boundary value conditions, allow for writing solutions to the wave equation in the form

\begin{equation}\label{eqn:em_wave:generalWaveSolution}
\begin{aligned}
f(\Bx,t) &= g\left( \kcap \cdot \Bx \pm \Abs{\Bv} t \right)
\end{aligned}
\end{equation}

Showing this logically follows from the original separation of variables approach has not been done.   However, despite this,
it is
simple enough to confirm that,
this more general function does satisfy the unforced wave equation \eqnref{eqn:em_wave:waveEquation}.

TODO: as followup here would like to go through the exercise of showing
that the solution of \eqnref{eqn:em_wave:generalWaveSolution} follows from a Fourier transform superposition.  Intuition says this is
possible, and I have said so without backing up the statement.

\subsection{Back to the electrodynamic case}

Using the above generalization argument as a guide we should be able to do something similar for the electrodynamic wave solution.

We want to solve for \(F\) the following gradient equation for the field in free space

\begin{equation}\label{eqn:em_wave:maxwell}
\begin{aligned}
\grad F = 0
\end{aligned}
\end{equation}

Let us suppose that the following is a solution and find the required constraints

\begin{equation}\label{eqn:em_wave:testSol}
\begin{aligned}
F = \gamma^\mu \wedge \gamma^\nu \left( K_{\mu\nu} f( x \cdot K_\mu ) -K_{\nu\mu} f( x \cdot K_\nu ) \right)
\end{aligned}
\end{equation}

We have two different grade equations built into Maxwell's equation \eqnref{eqn:em_wave:maxwell}, one of which is the vector equation, and the other
trivector.  Those are respectively

\begin{equation}\label{eqn:emWave:440}
\begin{aligned}
\grad \cdot F &= 0 \\
\grad \wedge F &= 0
\end{aligned}
\end{equation}

\subsubsection{zero wedge}

For the grade three term we have we can substitute \eqnref{eqn:em_wave:testSol} and see what comes out

\begin{equation}\label{eqn:emWave:460}
\begin{aligned}
\grad \wedge F
&=
\left( \gamma^\alpha \wedge \gamma^\mu \wedge \gamma^\nu\right)
\partial_\alpha \left( K_{\mu\nu} f( x \cdot K_\mu ) -K_{\nu\mu} f( x \cdot K_\nu ) \right) \\
\end{aligned}
\end{equation}

For the partial we will want the following

\begin{equation}\label{eqn:emWave:480}
\begin{aligned}
\partial_\mu ( x \cdot K_\beta )
&= \partial_\mu ( x^\nu \gamma_\nu \cdot K_{\beta\sigma} \gamma^\sigma ) \\
&= \partial_\mu ( x^\sigma K_{\beta\sigma} \\
&= K_{\beta\mu}
\end{aligned}
\end{equation}

and application of this with the chain rule we have

\begin{equation}\label{eqn:emWave:500}
\begin{aligned}
\grad \wedge F
&=
%\partial_\alpha ( x \cdot K_\beta ) &= K_{\beta\alpha}
\left( \gamma^\alpha \wedge \gamma^\mu \wedge \gamma^\nu\right)
\left( K_{\mu\nu} K_{\mu\alpha} f'( x \cdot K_\mu ) -K_{\nu\mu} K_{\nu\alpha} f'( x \cdot K_\nu ) \right) \\
&=
2\left( \gamma^\alpha \wedge \gamma^\mu \wedge \gamma^\nu\right) K_{\mu\nu} K_{\mu\alpha} f'( x \cdot K_\mu )
\\
\end{aligned}
\end{equation}

So, finally for this to be zero uniformly for all \(f\), we require

\begin{equation}\label{eqn:emWave:520}
\begin{aligned}
K_{\mu\nu} K_{\mu\alpha} = 0
\end{aligned}
\end{equation}

\subsubsection{zero divergence}
\index{divergence}

Now for the divergence term, corresponding to the current four vector condition \(J = 0\), we have

\begin{equation}\label{eqn:emWave:540}
\begin{aligned}
&\grad \cdot F \\
&= \gamma^\alpha \cdot
\left(\gamma^\mu \wedge \gamma^\nu\right) \partial_\alpha \left( K_{\mu\nu} f( x \cdot K_\mu ) -K_{\nu\mu} f( x \cdot K_\nu ) \right) \\
&=
(\gamma_\alpha)^2
%\gamma_\alpha \cdot \left(\gamma^\mu \wedge \gamma^\nu\right)
\left( \gamma^\nu {\delta_\alpha}^\mu -\gamma^\mu {\delta_\alpha}^\nu \right)
\partial_\alpha \left( K_{\mu\nu} f( x \cdot K_\mu ) -K_{\nu\mu} f( x \cdot K_\nu ) \right) \\
&=
\left(
(\gamma_\mu)^2 \gamma^\nu \partial_\mu
-(\gamma_\nu)^2 \gamma^\mu \partial_\nu
\right)
\left( K_{\mu\nu} f( x \cdot K_\mu ) -K_{\nu\mu} f( x \cdot K_\nu ) \right) \\
&=
(\gamma_\mu)^2 \gamma^\nu \partial_\mu \left( K_{\mu\nu} f( x \cdot K_\mu ) -K_{\nu\mu} f( x \cdot K_\nu ) \right)
-(\gamma_\mu)^2 \gamma^\nu \partial_\mu \left( K_{\nu\mu} f( x \cdot K_\nu ) -K_{\mu\nu} f( x \cdot K_\mu ) \right) \\
&= 2 (\gamma_\mu)^2 \gamma^\nu \partial_\mu \left( K_{\mu\nu} f( x \cdot K_\mu ) -K_{\nu\mu} f( x \cdot K_\nu ) \right) \\
\end{aligned}
\end{equation}

Application of the chain rule, and \(\partial_\mu ( x \cdot K_\beta ) = K_{\beta\mu}\), gives us

\begin{equation}\label{eqn:emWave:560}
\begin{aligned}
\grad \cdot F
&= 2 (\gamma_\mu)^2 \gamma^\nu \left( K_{\mu\nu} K_{\mu\mu} f'( x \cdot K_\mu ) -K_{\nu\mu} K_{\nu\mu} f'( x \cdot K_\nu ) \right) \\
\end{aligned}
\end{equation}

For \(\mu = \nu\) this is zero, which is expected since that should follow from the wedge product itself, but for the \(\mu \ne \nu\)
case it is not clear cut.

Damn.  On paper I missed some terms and it all canceled out nicely giving only a condition on \(K_{\mu\nu}\) from the wedge term.  The only
conclusion possible is that we require \(x \cdot K_\nu = x \cdot K_\mu\) for this form of solution, and therefore need to restrict the
test solution to a fixed spacetime direction.

\section{Take II.  A bogus attempt at a less general plane wave like solution}

Let us try instead

\begin{equation}\label{eqn:em_wave:solutionTensor}
\begin{aligned}
F = \gamma^\mu \wedge \gamma^\nu A_{\mu\nu} f( x \cdot k )
\end{aligned}
\end{equation}

and see if we can find conditions on the vector \(k\), and the tensor \(A\) that make this a solution to the unforced Maxwell equation \eqnref{eqn:em_wave:maxwell}.

\subsection{curl term}
\index{curl}

Taking the curl is straightforward

\begin{equation}\label{eqn:emWave:580}
\begin{aligned}
\grad \wedge F
&= \gamma^\alpha \wedge \gamma^\mu \wedge \gamma^\nu \partial_\alpha A_{\mu\nu} f( x \cdot k ) \\
&= \gamma^\alpha \wedge \gamma^\mu \wedge \gamma^\nu A_{\mu\nu} \partial_\alpha f( x^\sigma k_\sigma ) \\
&= \gamma^\alpha \wedge \gamma^\mu \wedge \gamma^\nu A_{\mu\nu} k_\alpha f'( x \cdot k ) \\
&= \inv{2} \gamma^\alpha \wedge \gamma^\mu \wedge \gamma^\nu (A_{\mu\nu} - A_{\nu\mu} ) k_\alpha f'( x \cdot k ) \\
\end{aligned}
\end{equation}

Curiously, the only condition that this yields is that we have

\begin{equation}\label{eqn:emWave:600}
\begin{aligned}
A_{\mu\nu} - A_{\nu\mu} = 0
\end{aligned}
\end{equation}

which is a symmetry requirement for the tensor

\begin{equation}\label{eqn:emWave:620}
\begin{aligned}
A_{\mu\nu} = A_{\nu\mu}
\end{aligned}
\end{equation}

\subsection{divergence term}

Now for the divergence

\begin{equation}\label{eqn:emWave:640}
\begin{aligned}
\grad \cdot F
&= \gamma_\alpha \cdot (\gamma^\mu \wedge \gamma^\nu) \partial^\alpha A_{\mu\nu} f( x_\sigma k^\sigma ) \\
&= \left( {\delta_\alpha}^\mu \gamma^\nu -{\delta_\alpha}^\nu \gamma^\mu \right) k^\alpha A_{\mu\nu} f'( x \cdot k ) \\
&=
 \gamma^\nu k^\mu A_{\mu\nu} f'( x \cdot k )
-\gamma^\mu k^\nu A_{\mu\nu} f'( x \cdot k )
\\
&= \gamma^\nu k^\mu (A_{\mu\nu} -A_{\nu\mu}) f'( x \cdot k )
\end{aligned}
\end{equation}

So, again, as in the divergence part of Maxwell's equation for the vacuum (\(\grad F = 0\)), we require, and it is sufficient that

\begin{equation}\label{eqn:emWave:660}
\begin{aligned}
A_{\mu\nu} -A_{\nu\mu} = 0,
\end{aligned}
\end{equation}

for \eqnref{eqn:em_wave:solutionTensor} to be a solution.  This is somewhat surprising since I would not have expected a symmetric tensor to fall out of
the analysis.

Actually, this is more than surprising and amounts to a requirement that the field solution is zero.  Going back to the proposed solution we have

\begin{equation}\label{eqn:emWave:680}
\begin{aligned}
F
&= \gamma^\mu \wedge \gamma^\nu A_{\mu\nu} f( x \cdot k ) \\
&= \gamma^\mu \wedge \gamma^\nu \inv{2} (A_{\mu\nu} - A_{\nu\mu})f( x \cdot k ) \\
\end{aligned}
\end{equation}

So, any symmetric components of the tensor \(A\) automatically cancel out.

\section{Summary}

A few dead ends have been chased and I am left with the original attempt summarized by
\eqnref{eqn:em_wave:WorkingSolution},
\eqnref{eqn:em_wave:WorkingSolutionDefinitions}, and
\eqnref{eqn:em_wave:WorkingSolutionConstraints}.

It appears that the TODO noted above to attempt the Fourier transform treatment will likely be required to put these exponentials into a more general form.
I had also intended to try to cover phase and group velocities for myself here but took too much time chasing the dead ends.  Will have to leave that
to another day.

   %
% Copyright � 2012 Peeter Joot.  All Rights Reserved.
% Licenced as described in the file LICENSE under the root directory of this GIT repository.
%

%
%
\chapter{Magnetic field between two parallel wires}
\index{magnetic field!parallel wires}
\label{chap:sgMx41}
%\date{July 20, 2008}

\section{Student's guide to Maxwell's' equations.  problem 4.1}

The
\href{http://www4.wittenberg.edu/maxwell/chapter4/problem1/}{problem is}:

Two parallel wires carry currents I1 and 2I1 in opposite directions.  Use Ampere is law to find the magnetic field at a point midway between the wires.

Do this instead (visualizing the cross section through the wires) for N wires
located at points \(P_k\), with currents \(I_k\).

\imageFigure{../figures/gabook/p41}{Currents through parallel wires}{fig:2wires}{0.4}

This is illustrated for two wires in \cref{fig:2wires}.

\subsection{}

Consider first just the magnetic field for one wire, temporarily putting
the origin at the point of the current.

\begin{equation*}
\int \BB \cdot d\Bl = \mu_0 I
\end{equation*}

At a point \(\Br\) from the local origin the tangent vector is obtained by
rotation of the unit vector:

\begin{equation*}
\ycap \exp{\left(\xcap\ycap \log{\left(\frac{\Br}{\norm{\Br}}\right)}\right)}
= \ycap {\left(\frac{\Br}{\norm{\Br}}\right)}^{\xcap\ycap}
\end{equation*}

% FIXME: this rotation above doesn't make sense to me.  What the hell was I doing?
Thus the magnetic field at the point \(\Br\) due to this particular current is:

\begin{equation*}
\BB(\Br)
= \frac{\mu_0 I \ycap}{2\pi \norm{\Br}} {\left(\frac{\Br}{\norm{\Br}}\right)}^{\xcap\ycap}
\end{equation*}

Considering additional currents with the wire centers at points \(P_k\), and measurement of the field at point \(\BR\) we have for each of those:

\begin{equation*}
\Br = \BR - \BP
\end{equation*}

Thus the total field at point \(\BR\) is:

\begin{equation}
\BB(\BR) = \frac{\mu_0 \ycap}{2\pi} \sum_k \frac{I_k}{\norm{\BR - \BP_k}} {\left(\frac{\BR - \BP_k}{\norm{\BR - \BP_k}}\right)}^{\xcap\ycap}
\end{equation}

\subsection{Original problem}

For the problem as stated, put the origin between the two points with those two points on the x-axis.

\begin{equation}\label{eqn:sgMx41:20}
\begin{aligned}
\BP_1 &= - \xcap d/2 \\
\BP_2 &= \xcap d/2
\end{aligned}
\end{equation}

Here \(\BR\) = 0, so \(\Br_1 = \BR - \BP_1 = \xcap d/2 \) and \(\Br_2 = - \xcap d/2\).  With \(\xcap\ycap = i\), this is:

\begin{equation}\label{eqn:sgMx41:40}
\begin{aligned}
\BB(0)
&= \frac{\mu_0 \ycap}{\pi d} \left( I_1 {(-\xcap)}^i + I_2 {\xcap^i} \right) \\
&= \frac{\mu_0 \ycap}{\pi d} \left( -I -2 I\right) \\
&= \frac{-3 I \mu_0 \ycap}{\pi d}
\end{aligned}
\end{equation}

Here unit vectors exponentials were evaluated with the equivalent complex number manipulations:

\begin{equation}\label{eqn:sgMx41:60}
\begin{aligned}
(-1)^i &= x \\
i \log{(-1)} &= \log{x} \\
i \pi &= \log{x} \\
\exp{(i \pi)} &= \log{x} \\
x &= -1
\end{aligned}
\end{equation}

\begin{equation}\label{eqn:sgMx41:80}
\begin{aligned}
(1)^i &= x \\
i \log{(1)} &= \log{x} \\
0 &= \log{x} \\
x &= 1
\end{aligned}
\end{equation}

   %
% Copyright � 2012 Peeter Joot.  All Rights Reserved.
% Licenced as described in the file LICENSE under the root directory of this GIT repository.
%

%
%
\mychapter{Field due to line charge in arc}
\index{line charge}
\label{chap:chargeArcElement}
%\date{Nov 23, 2008.  chargeArcElement.tex}

\section{Motivation}

Problem \(1.5\) from \citep{purcell1963eam}, is to calculate the field
at the center of a half circular arc of line charge.  Do this calculation
and setup for the calculation at other points.

\section{Just the stated problem}

%\begin{figure}[htp]
%\centering
%\includegraphics[totalheight=0.4\textheight]{picturepath}
%\caption{My Caption}\label{fig:pictlabel}
%\end{figure}
%
%... see \cref{fig:picturepath} ...

To solve for the field at just the center point in the plane of the arc, given line charge density \(\lambda\), and arc
radius \(R\) one has, and pseudoscalar for the plane \(i = \Be_1\Be_2\) one has
%
\begin{equation}\label{eqn:chargeArcElement:20}
\begin{aligned}
dq &= \lambda R d\theta \\
d\BE &= \inv{4 \pi \epsilon_0 R^2} dq (-\Be_1 e^{i\theta} )
\end{aligned}
\end{equation}
%
Straight integration gives the result in short order
%
\begin{equation}\label{eqn:chargeArcElement:40}
\begin{aligned}
\BE
&= \frac{-\lambda \Be_1}{4 \pi \epsilon_0 R} \int_0^\pi e^{i\theta} d\theta \\
&= \frac{\lambda \Be_2}{4 \pi \epsilon_0 R} \left. e^{i\theta} \right\vert_0^\pi \\
&= \frac{-\lambda \Be_2}{2 \pi \epsilon_0 R}
\end{aligned}
\end{equation}
%
So, if the total charge is \(Q = \pi R \lambda\), the field is then
%
\begin{equation}\label{eqn:chargeArcElement:60}
\begin{aligned}
\BE
&= \frac{-Q \Be_2}{2 \pi^2 \epsilon_0 R^2} \\
\end{aligned}
\end{equation}
%
So, at the center point the semicircular arc of charge behaves as if it is a point charge of magnitude \(2Q/\pi\) at the point \(R \Be_2\)
%
\begin{equation}\label{eqn:chargeArcEl:semicircleAtCenter}
\begin{aligned}
\BE = \frac{-Q \Be_2}{ 4 \pi \epsilon_0 R^2 } \frac{2}{\pi}
\end{aligned}
\end{equation}
%
\section{Field at other points}

Now, how about at points outside of the plane of the charge?

Suppose our point of measurement is expressed in cylindrical polar coordinates
%
\begin{equation}\label{eqn:chargeArcElement:80}
\begin{aligned}
P = \rho \Be_1 e^{i\alpha} + z \Be_3
\end{aligned}
\end{equation}
%
So that the vector from the element of charge at \(\theta\) is
\begin{equation}\label{eqn:chargeArcElement:100}
\begin{aligned}
\Bu = P - R \Be_1 e^{i\theta} &= \Be_1 (\rho e^{i\alpha} -R e^{i\theta}) + z \Be_3
\end{aligned}
\end{equation}
%
Relative to \(\theta\), writing \(\theta = \alpha + \beta\) this is
\begin{equation}\label{eqn:chargeArcElement:120}
\begin{aligned}
\Bu &= \Be_1 e^{i\alpha} (\rho -R e^{i\beta}) + z \Be_3
\end{aligned}
\end{equation}
%
The squared magnitude of this vector is
\begin{equation}\label{eqn:chargeArcElement:140}
\begin{aligned}
\Bu^2
&= \Abs{\rho -R e^{i\beta}}^2 + z^2 \\
&= z^2 + \rho^2 + R^2 -2 \rho R \cos\beta \\
\end{aligned}
\end{equation}
%
The field is thus
%
\begin{equation}\label{eqn:chargeArcEl:fieldAtP}
\begin{aligned}
\BE = \inv{4 \pi \epsilon_0} \lambda R
\int_{\beta=\theta_1 -\alpha}^{\beta=\theta_2-\alpha}
{
\left(z^2 + \rho^2 + R^2 -2 \rho R \cos\beta\right)
}^{-3/2}
\left(\Be_1 e^{i\alpha} (\rho -R e^{i\beta}) + z \Be_3\right)
d\beta
\end{aligned}
\end{equation}
%
This integral has two variations
\begin{equation}\label{eqn:chargeArcElement:160}
\begin{aligned}
&\int { \left(a^2 - b^2\cos\beta\right) }^{-3/2} d\beta \\
&\int { \left(a^2 - b^2\cos\beta\right) }^{-3/2} e^{i\beta} d\beta \\
\end{aligned}
\end{equation}
%
or
\begin{equation}\label{eqn:chargeArcElement:180}
\begin{aligned}
I_1 &= \int { \left(a^2 - b^2\cos\beta\right) }^{-3/2} d\beta \\
I_2 &= \int { \left(a^2 - b^2\cos\beta\right) }^{-3/2} \cos\beta d\beta \\
I_3 &= \int { \left(a^2 - b^2\cos\beta\right) }^{-3/2} \sin\beta d\beta \\
\end{aligned}
\end{equation}
%
Of these when only the last is obviously integrable (at least for \(b \ne 0\))
\begin{equation}\label{eqn:chargeArcElement:200}
\begin{aligned}
I_3
&= \int { \left(a^2 - b^2\cos\beta\right) }^{-3/2} \sin\beta d\beta \\
&= -2 {\left(a^2 - b^2\cos\beta\right) }^{-1/2} \\
\end{aligned}
\end{equation}
%
Having solved for the imaginary component can the Cauchy Riemann equations be used to supply the real part?  How about \(I_1\) ?

\subsection{On the z-axis}

Not knowing how to solve the integral of \eqnref{eqn:chargeArcEl:fieldAtP} (elliptic?), the easy case
of \(\rho = 0\) (up the z-axis) can at least be obtained
%
\begin{equation}\label{eqn:chargeArcElement:220}
\begin{aligned}
\BE
&= \inv{4 \pi \epsilon_0} \lambda R { \left(z^2 + R^2\right) }^{-3/2}
\int_{\theta_1}^{\theta_2} \left(-\Be_1 R e^{i\theta} + z \Be_3\right) d\theta \\
&= \inv{4 \pi \epsilon_0} \lambda R { \left(z^2 + R^2\right) }^{-3/2}
\left(\Be_2 R (e^{i\theta_2} -e^{i\theta_1}) + z \Be_3 \Delta\theta \right) \\
&= \inv{4 \pi \epsilon_0} \lambda R { \left(z^2 + R^2\right) }^{-3/2}
\left(\Be_2 R e^{i(\theta_1 + \theta_2)/2} \left(
e^{i(\theta_2 - \theta_1)/2}
-e^{-i(\theta_2 - \theta_1)/2}
\right) + z \Be_3 \Delta\theta \right) \\
&= \inv{4 \pi \epsilon_0} \lambda R { \left(z^2 + R^2\right) }^{-3/2}
\left(-2 \Be_1 R e^{i(\theta_1 + \theta_2)/2} \sin(\Delta\theta/2)
+ z \Be_3 \Delta\theta \right) \\
&= \inv{4 \pi \epsilon_0 \Delta\theta} Q { \left(z^2 + R^2\right) }^{-3/2}
\left(-2 \Be_1 R e^{i(\theta_1 + \theta_2)/2} \sin(\Delta\theta/2)
+ z \Be_3 \Delta\theta \right) \\
\end{aligned}
\end{equation}
%
Eliminating the explicit imaginary, and writing \(\overbar{\theta} = (\theta_1 + \theta_2)/2\), we have in vector form
the field on any position up and down the z-axis
\begin{equation}\label{eqn:chargeArcElement:240}
\begin{aligned}
\BE
&= \inv{4 \pi \epsilon_0 \Delta\theta} Q { \left(z^2 + R^2\right) }^{-3/2}
\left(
-2 R \left( \Be_1 \cos \overbar{\theta} +\Be_2 \sin\overbar{\theta} \right) \sin(\Delta\theta/2)
+ z \Be_3 \Delta\theta \right)
\end{aligned}
\end{equation}
%
For \(z = 0\), \(\theta_1 = 0\), and \(\theta_2 = \pi\), this matches with \eqnref{eqn:chargeArcEl:semicircleAtCenter} as expected, but
expressing this as an equivalent to a point charge is no longer possible at any point off the plane of the
charge.

   %
% Copyright � 2012 Peeter Joot.  All Rights Reserved.
% Licenced as described in the file LICENSE under the root directory of this GIT repository.
%

%
%
\chapter{Charge line element}
\index{line element}
\label{chap:chargeLineElement}
%\date{Nov 23, 2008.  chargeLineElement.tex}

\section{Motivation}

In \citep{purcell1963eam} the electric field for an infinite length charged line element is derived in two ways.  First using summation directly, then with Gauss's law.  Associated with the first was the statement that the field must be radial by symmetry.  This was not obvious to me when initially taking my E\&M course, so I thought it was worth revisiting.

\section{Calculation of electric field for non-infinite length line element}

\imageFigure{../gabook-figures/charge_line_element_figure}{Charge on wire}{fig:chargeLineElement}{0.4}

This calculation will be done with a thickness neglected wire running up and down along the \(y\) axis as illustrated in \cref{fig:chargeLineElement}, where the field is being measured at \(P = r \Be_1\), and the field contributions due to all charge elements \(dq = \lambda dy\) are to be summed.

We want to sum each of the field contributions along the line, so with

\begin{equation}\label{eqn:chargeLineElement:20}
\begin{aligned}
d\BE &= \frac{dq \ucap(\theta)}{4 \pi \epsilon_0 R^2} \\
r/R &= \cos\theta \\
dy &= r d(\tan\theta) = r \sec^2 \theta \\
\ucap(\theta) &= \Be_1 e^{i \theta} \\
i &= \Be_1 \Be_2
\end{aligned}
\end{equation}

% check:
% 3 pi/2 : e1 e^i\theta = e1 \cos 3\pi/2 + e2\sin 3\pi/2 = -e2
%   pi/2 : e1 e^i\theta = e1 \cos \pi/2 + e2\sin \pi/2   = e2
%      0 : e1 e^i\theta = e1 \cos 0 + e2\sin 0           = e1
%
% 3 pi/2 : e^i\theta = \cos 3\pi/2 + e1e2\sin 3\pi/2     = -e1e2
% pi/2   : e^i\theta = \cos \pi/2 + e1e2\sin \pi/2       = e1e2
% 0      : e^i\theta = \cos 0 + e1e2\sin 0               = 1

Putting things together we have

\begin{equation}\label{eqn:chargeLineElement:40}
\begin{aligned}
d\BE
&= \frac{\lambda r \sec^2 \theta \Be_1 e^{i\theta} d\theta}{4 \pi \epsilon_0 r^2 \sec^2 \theta} \\
&= \frac{\lambda \Be_1 e^{i\theta} d\theta}{4 \pi \epsilon_0 r} \\
&= -\frac{\lambda \Be_1 i d(e^{i\theta})}{4 \pi \epsilon_0 r} \\
\end{aligned}
\end{equation}

Thus the total field is
\begin{equation}\label{eqn:chargeLineElement:60}
\begin{aligned}
\BE
&= \int d\BE \\
&= -\frac{\lambda \Be_2}{4 \pi \epsilon_0 r} \int d(e^{i\theta}) \\
\end{aligned}
\end{equation}

We see that the integration, which has the value

\begin{equation}\label{eqn:chargeLineElement:80}
\begin{aligned}
\BE = -\frac{\lambda} {4 \pi \epsilon_0 r} \Be_2 e^{i\delta\theta}
\end{aligned}
\end{equation}

The integration range for the infinite wire is \(\theta \in [3\pi/2, \pi/2]\)
so the field for the infinite wire is

\begin{equation}\label{eqn:chargeLineElement:100}
\begin{aligned}
\BE
&= -\frac{\lambda} {4 \pi \epsilon_0 r} \Be_2 \left. e^{i\theta} \right\vert^{\theta = \pi/2}_{\theta = 3\pi/2} \\
&= -\frac{\lambda} {4 \pi \epsilon_0 r} \Be_2 (e^{i\pi/2} - e^{3i\pi/2}) \\
&= -\frac{\lambda} {4 \pi \epsilon_0 r} \Be_2 (\Be_1 \Be_2 - (-\Be_1 \Be_2)) \\
&= \frac{\lambda} {2 \pi \epsilon_0 r} \Be_1 \\
\end{aligned}
\end{equation}
% 3 pi/2 : e^i\theta = \cos 3\pi/2 + e1e2\sin 3\pi/2     = -e1e2
% pi/2   : e^i\theta = \cos \pi/2 + e1e2\sin \pi/2       = e1e2

%and \(e^{-i\pi}\)
%\ucap(\theta) &= \Be_1 e^{i \theta} \quad \theta \in [3\pi/2, \pi/2] \\

Invoking symmetry was done in order to work with coordinates, but working with the vector quantities directly
avoids this requirement and gives the general result for any subset of angles.

For a finite length wire all that is required is an angle parametrization of that wire's length

\begin{equation}\label{eqn:chargeLineElement:120}
\begin{aligned}
%[y_1, y_2] = r[\tan\theta_1, \tan\theta_2].
[\theta_1, \theta_2] = [\tan^{-1}(y_1/r), \tan^{-1}(y_2/r)]
\end{aligned}
\end{equation}

For such a range the exponential difference for the integral is

\begin{equation}\label{eqn:chargeLineElement:140}
\begin{aligned}
\left. e^{i\theta} \right \vert_{\theta_1}^{\theta_2}
&= e^{i\theta_2} - e^{i\theta_1} \\
&= e^{i(\theta_1 + \theta_2)/2} \left( e^{i(\theta_2 - \theta_1)/2} -e^{i(\theta_2 - \theta_1)/2} \right) \\
&= 2 i e^{i(\theta_1 + \theta_2)/2} \sin((\theta_2 - \theta_1)/2) \\
\end{aligned}
\end{equation}

thus the associated field is

\begin{equation}\label{eqn:chargeLineElement:160}
\begin{aligned}
\BE
&= -\frac{\lambda} {2 \pi \epsilon_0 r} \Be_2 i e^{i(\theta_1 + \theta_2)/2} \sin((\theta_2 - \theta_1)/2) \\
&= \frac{\lambda} {2 \pi \epsilon_0 r} \Be_1 e^{i(\theta_1 + \theta_2)/2} \sin((\theta_2 - \theta_1)/2) \\
\end{aligned}
\end{equation}

   %
% Copyright � 2012 Peeter Joot.  All Rights Reserved.
% Licenced as described in the file LICENSE under the root directory of this GIT repository.
%

%
%
\chapter{Biot Savart Derivation}\label{chap:biotSavart}
\index{Biot Savart}
%\date{April 18, 2009.  biotSavart.tex}

\section{Motivation}

Looked at my Biot-Savart derivation in \chapcite{PJelectricFieldEnergy}.  There I was playing with doing this without first dropping down to the
familiar vector relations, and end up with an expression of the Biot Savart law in terms of the complete Faraday bivector.  This is
an excessive approach, albeit interesting (to me).  Let us try this again in terms of just the magnetic field.

\section{Do it}

\subsection{Setup. Ampere-Maxwell equation for steady state}
\index{Ampere-Maxwell equation}

The starting point can still be Maxwell's equation

\begin{equation}\label{eqn:biotSavart:20}
\begin{aligned}
\grad F = J/\epsilon_0 c
\end{aligned}
\end{equation}

and the approach taken will be the more usual consideration of a loop of steady-state (no-time variation) current.

In the steady state we have

\begin{equation}\label{eqn:biotSavart:40}
\begin{aligned}
\grad = \gamma^0 \inv{c} \partial_t + \gamma^k \partial_k = \gamma^k \partial_k
\end{aligned}
\end{equation}

and in particular

\begin{equation}\label{eqn:biotSavart:60}
\begin{aligned}
\gamma_0 \grad F
&= \gamma_0 \gamma^k \partial_k F \\
&= \gamma_k \gamma_0 \partial_k F \\
&= \sigma_k \partial_k F \\
&= \spacegrad (\BE + I c \BB) \\
\end{aligned}
\end{equation}

and for the RHS,

\begin{equation}\label{eqn:biotSavart:80}
\begin{aligned}
\gamma_0 J/\epsilon_0 c
&=
\gamma_0 (c \rho \gamma_0 + J^k \gamma_k)/\epsilon_0 c  \\
&=
(c \rho - J^k \sigma_k)/\epsilon_0 c  \\
&=
(c \rho - \Bj)/\epsilon_0 c  \\
\end{aligned}
\end{equation}

So we have

\begin{equation}\label{eqn:biotSavart:spacetimeSplitSteadyState}
\begin{aligned}
\spacegrad (\BE + I c \BB)
&=
\inv{\epsilon_0}\rho - \frac{\Bj}{\epsilon_0 c}
\end{aligned}
\end{equation}

Selection of the (spatial) vector grades gives

\begin{equation}\label{eqn:biotSavart:100}
\begin{aligned}
I c (\spacegrad \wedge \BB) &= - \frac{\Bj}{\epsilon_0 c}
\end{aligned}
\end{equation}

or with \(\Ba \wedge \Bb = I (\Ba \cross \Bb)\), and \(\epsilon_0 \mu_0 c^2 = 1\), this is the familiar Ampere-Maxwell equation when \(\PDi{t}{\BE} = 0\).

\begin{equation}\label{eqn:biotSavart:AmpereMaxwellSteady}
\begin{aligned}
\spacegrad \cross \BB &= \mu_0 \Bj
\end{aligned}
\end{equation}

\subsection{Three vector potential solution}

With \(\spacegrad \cdot \BB = 0\) (the trivector part of \eqnref{eqn:biotSavart:spacetimeSplitSteadyState}), we can write

\begin{equation}\label{eqn:biotSavart:120}
\begin{aligned}
\BB = \spacegrad \cross \BA
\end{aligned}
\end{equation}

For some vector potential \(\BA\).  In particular, we have in \eqnref{eqn:biotSavart:AmpereMaxwellSteady},

\begin{equation}\label{eqn:biotSavart:140}
\begin{aligned}
\spacegrad \cross \BB
&=
\spacegrad \cross (\spacegrad \cross \BA) \\
&=
-I (\spacegrad \wedge (\spacegrad \cross \BA) ) \\
&=
-\frac{I}{2} (
\spacegrad (\spacegrad \cross \BA)
- (\spacegrad \cross \BA) \spacegrad
) \\
&=
\frac{I^2}{2} (
\spacegrad (\spacegrad \wedge \BA)
- (\spacegrad \wedge \BA) \spacegrad
) \\
&=
- \spacegrad \cdot (\spacegrad \wedge \BA)
\\
\end{aligned}
\end{equation}

Therefore the three vector potential equation for the magnetic field is

\begin{equation}\label{eqn:biotSavart:vecAwithJ}
\begin{aligned}
\spacegrad (\spacegrad \cdot \BA) - \spacegrad^2 \BA &= \mu_0 \Bj
\end{aligned}
\end{equation}

\subsection{Gauge freedom}
\index{gauge freedom}

We have the freedom to set \(\spacegrad \cdot \BA = 0\), in \eqnref{eqn:biotSavart:vecAwithJ}.  To see this suppose that the vector potential is
expressed in terms of some other potential \(\BA'\) that does have zero divergence (\(\spacegrad \cdot \BA' = 0\)) plus a (spatial) gradient

\begin{equation}\label{eqn:biotSavart:160}
\begin{aligned}
\BA = \BA' + \spacegrad \phi
\end{aligned}
\end{equation}

Provided such a construction is possible, then we have

\begin{equation}\label{eqn:biotSavart:180}
\begin{aligned}
\spacegrad (\spacegrad \cdot \BA) - \spacegrad^2 \BA
&=
\spacegrad (\spacegrad \cdot (\BA' + \spacegrad \phi)) - \spacegrad^2 (\BA' + \spacegrad \phi) \\
&=
- \spacegrad^2 \BA'
\end{aligned}
\end{equation}

and can instead solve the simpler equivalent problem

\begin{equation}\label{eqn:biotSavart:blah}
\begin{aligned}
\spacegrad^2 \BA'  &= -\mu_0 \Bj
\end{aligned}
\end{equation}

Addition of the gradient \(\spacegrad \phi\) to \(A'\) will not change the magnetic field \(\BB\) since \(\spacegrad \cross (\spacegrad \phi) = 0\).

FIXME: what was not shown here is that it is possible to express any vector potential \(\BA\) in terms of a divergence free potential and a
gradient.  How would one show this?

\subsection{Solution to the vector Poisson equation}
\index{Poisson equation}

The solution (dropping primes) to the Poisson \eqnref{eqn:biotSavart:blah} is

\begin{equation}\label{eqn:biotSavart:200}
\begin{aligned}
\BA = \frac{\mu_0}{4 \pi} \int \frac{\Bj}{r} dV
\end{aligned}
\end{equation}

(See \citep{schwartz1987pe} for example.)

The magnetic field follows by taking the spatial curl

\begin{equation}\label{eqn:biotSavart:220}
\begin{aligned}
\BB
&= \spacegrad \cross \BA \\
&= \frac{\mu_0}{4 \pi} \spacegrad \cross \int \frac{\Bj'}{\Abs{\Br-\Br'}} dV'
\end{aligned}
\end{equation}

Pulling the curl into the integral and writing the gradient in terms of radial components

\begin{equation}\label{eqn:biotSavart:240}
\begin{aligned}
\spacegrad &= \frac{\Br - \Br'}{\Abs{\Br - \Br'}} \PD{\Abs{\Br - \Br'}}{}
\end{aligned}
\end{equation}

we have
\begin{equation}\label{eqn:biotSavart:260}
\begin{aligned}
\BB
&= \frac{\mu_0}{4 \pi} \int
\frac{\Br - \Br'}{\Abs{\Br - \Br'}} \cross \Bj' \PD{\Abs{\Br - \Br'}}{}{\inv{ \Abs{\Br-\Br'} }} dV' \\
&= -\frac{\mu_0}{4 \pi} \int
\frac{\Br - \Br'}{\Abs{\Br - \Br'}^3} \cross \Bj' dV' \\
\end{aligned}
\end{equation}

Finally with \(\Bj' dV' = I \jCap' dl'\), we have

\begin{equation}\label{eqn:biotSavart:280}
\begin{aligned}
\BB(\Br) &= \frac{\mu_0}{4 \pi} \int dl' \jCap' \cross \frac{\Br - \Br'}{\Abs{\Br - \Br'}^3}
\end{aligned}
\end{equation}

   %
% Copyright � 2012 Peeter Joot.  All Rights Reserved.
% Licenced as described in the file LICENSE under the root directory of this GIT repository.
%

%
%
\chapter{Vector forms of Maxwell's equations as projection and rejection operations}\label{chap:PJMaxwellProj}
\index{Maxwell's equations!projection}
\index{Maxwell's equations!rejection}
%\date{Sept 9, 2008.  vectorMaxwellsProjection.tex}

\section{Vector form of Maxwell's equations}

%FIXME: used \(i\) here as a pseudoscalar and an index depending on context.  Switched to \(I\) instead for the pseudoscalar for clarity ... hope I got them all.

We saw how to extract the tensor formulation of Maxwell's equations
from \(\grad F = J\).  A little bit of play shows how to pick off the divergence
equations we are used to as well.

The end result is that we can pick off two of the eight coordinate equations
with specific product operations.

It is helpful in the following to write \(\grad F\) in index notation
% See: em_bivector_metric_dependencies.pdf
%\eqnref{eqn:vecMaxProj:Fcomp}

\begin{equation}
\grad F = \PD{x^\mu}{E^i} {\gamma^{\mu}}_{i 0} - \epsilon_{i j k} c \PD{x^\mu}{B^i} {\gamma^{\mu}}_{j k}
\end{equation}

In particular, look at the span of the vector, or trivector multiplicands of
the partials of the electric and magnetic field coordinates

\begin{equation}\label{eqn:vecMaxProj:spanpartialE}
{\gamma^{\mu}}_{i 0} \in \Span \{ \gamma_{\mu}, \gamma_{0 i j} \}
\end{equation}

\begin{equation}\label{eqn:vecMaxProj:spanpartialB}
{\gamma^{\mu}}_{j k} \in \Span \{ \gamma_{i j \mu}, \gamma_i \}
\end{equation}

\subsection{Gauss's law for electrostatics}
\index{Gauss's law}
\index{electrostatics}

For extract Gauss's law for electric fields that operation is to take the scalar
parts of the product with \(\gamma^0\).

Dotting with \(\gamma^0\) will pick off the \(\rho\) term from
\(J\)

\begin{equation*}
\frac{J}{\epsilon_0 c} \cdot \gamma^0 = \rho/\epsilon_0,
\end{equation*}

We see that dotting
with \(\gamma_0\) will leave bivector parts contributed by the trivectors in
the span of \eqnref{eqn:vecMaxProj:spanpartialE}.  Similarly the magnetic partials
will contribute bivectors and scalars with this product.  Therefore to
get an equation with strictly scalar parts equal to \(\rho/\epsilon_0\) we need
to compute

\begin{equation}\label{eqn:vectorMaxwellsProjection:20}
\begin{aligned}
\gpgradezero{\left(\grad F - J/\epsilon_0 c\right) \gamma^0}
&= \gpgradezero{\grad \BE \gamma^0} - \rho/\epsilon_0 \\
&= \gpgradezero{\grad E^k {\gamma_{k0}}^0} - \rho/\epsilon_0 \\
&= \gpgradezero{ \gamma^j \partial_{j} E^k \gamma_{k} } - \rho/\epsilon_0 \\
&= {\delta^j}_k \partial_{j} E^k - \rho/\epsilon_0 \\
&= \partial_{k} E^k - \rho/\epsilon_0 \\
\end{aligned}
\end{equation}

This is Gauss's law for electrostatics:
\begin{equation}\label{eqn:vecMaxProj:gausselectro}
\gpgradezero{\left(\grad F - J/\epsilon_0 c\right) \gamma^0} = \spacegrad \cdot \BE - \rho/\epsilon_0 = 0
\end{equation}

\subsection{Gauss's law for magnetostatics}
\index{Gauss's law}
\index{magnetostatics}

Here we are interested in just the trivector terms that are equal to zero that we saw before in \(\grad \wedge \grad \wedge A = 0\).

The divergence like equation of these four can be obtained by dotting with \(\gamma_{123} = \gamma^0 I\).  From the span enumerated
in \eqnref{eqn:vecMaxProj:spanpartialB}, we see that only the \(\BB\) field contributes such a trivector.  An addition scalar part selection is used
to eliminate the bivector that \(J\) contributes.

\begin{equation}\label{eqn:vectorMaxwellsProjection:40}
\begin{aligned}
\gpgradezero{\left(\grad F - J/\epsilon_0 c\right) \cdot \left(\gamma^0 I\right)}
&= (\grad I c \BB) \cdot \left(\gamma^0 I\right) \\
&= \gpgradezero{ \grad I c \BB \gamma^0 I } \\
&= \gpgradezero{ I \grad I c \BB \gamma^0 } \\
&= - c \gpgradezero{ I^2 \grad \BB \gamma^0 } \\
&= c \gpgradezero{ \grad \BB \gamma^0 } \\
&= c \gpgradezero{ \gamma^{\mu} \partial_{\mu} B^k \gamma_k } \\
&= c {\delta^{\mu}}_k \partial_{\mu} B^k \\
&= c \partial_{k} B^k \\
&= 0
\end{aligned}
\end{equation}

This is just the divergence, and therefore yields Gauss's law for magnetostatics:

\begin{equation}\label{eqn:vecMaxProj:gaussmagnet}
\left(\grad F - J/\epsilon_0 c\right) \cdot \left(\gamma^0 I / c \right) = \spacegrad \cdot \BB = 0
\end{equation}

\subsection{Faraday's Law}
\index{Faraday's law}

We have three more trivector equal zero terms to extract from our field equation.

Taking dot products for those remaining three trivectors we have

\begin{equation}\label{eqn:vectorMaxwellsProjection:60}
\begin{aligned}
( \grad F - J/\epsilon_0 c ) \cdot (\gamma^j I)
\end{aligned}
\end{equation}

This will leave a contribution from \(J\), so to exclude that we want to calculate

\begin{equation}\label{eqn:vectorMaxwellsProjection:80}
\begin{aligned}
\gpgradezero{( \grad F - J/\epsilon_0 c ) \cdot (\gamma^j I)}
\end{aligned}
\end{equation}

The electric field contribution gives us
\begin{equation}\label{eqn:vectorMaxwellsProjection:100}
\begin{aligned}
\partial_{\mu} E^k \gpgradezero{ \gamma^{\mu} \gamma_{k0} {\gamma^j}_{0123} }
&=
-\partial_{\mu} E^k (\gamma_0)^2 \gpgradezero{ \gamma^{\mu} \gamma_{k} {\gamma^j}_{123} }
\end{aligned}
\end{equation}

the terms \(\mu = 0\) will not produce a scalar, so this leaves
\begin{equation}\label{eqn:vectorMaxwellsProjection:120}
\begin{aligned}
-\partial_{i} E^k (\gamma_0)^2 \gpgradezero{ \gamma^{i} \gamma_{k} {\gamma^j}_{123} }
&= -\partial_{i} E^k (\gamma_0)^2 (\gamma_k)^2 \epsilon_{jki} \\
&= \partial_{i} E^k \epsilon_{j k i} \\
&= -\partial_{i} E^k \epsilon_{jik} \\
\end{aligned}
\end{equation}

Now, for the magnetic field contribution we have
\begin{equation}\label{eqn:vectorMaxwellsProjection:140}
\begin{aligned}
c \partial_{\mu} B^k \gpgradezero{ \gamma^{\mu} I \gamma_{k0} {\gamma^j} I }
&= - c \partial_{\mu} B^k \gpgradezero{ I \gamma^{\mu} \gamma_{k0} {\gamma^j} I } \\
&= - c \partial_{\mu} B^k \gpgradezero{ I^2 \gamma^{\mu} \gamma_{k0} {\gamma^j} } \\
&= c \partial_{\mu} B^k \gpgradezero{ \gamma^{\mu} \gamma_{k0} {\gamma^j} } \\
\end{aligned}
\end{equation}

For a scalar part we need \(\mu = 0\) leaving
\begin{equation}\label{eqn:vectorMaxwellsProjection:160}
\begin{aligned}
c \partial_{0} B^k \gpgradezero{ \gamma^{0} \gamma_{k0} {\gamma^j} }
&= -\partial_{t} B^k \gpgradezero{ \gamma_{k} {\gamma^j} } \\
&= -\partial_{t} B^k {\delta_{k}}^j \\
&= -\partial_{t} B^j
\end{aligned}
\end{equation}

Combining the results and summing as a vector we have:
\begin{equation}\label{eqn:vectorMaxwellsProjection:180}
\begin{aligned}
\sum \sigma_j \gpgradezero{( \grad F - J/\epsilon_0 c ) \cdot (\gamma^j I)}
&= -\partial_{i} E^k \epsilon_{jik} \sigma_j -\partial_{t} B^j \sigma_j \\
&= -\partial_{j} E^k \epsilon_{i j k} \sigma_i -\partial_{t} B^i \sigma_i \\
&= -\spacegrad \cross \BE - \PD{t}{\BB} \\
&= 0
\end{aligned}
\end{equation}

Moving one term to the opposite side of the equation yields the familiar vector form for Faraday's law

\begin{equation}
\spacegrad \cross \BE = -\PD{t}{\BB}
\end{equation}

\subsection{Ampere Maxwell law}
\index{Ampere Maxwell law}

For the last law, we want the current density, so to extract the Ampere Maxwell law we must have to wedge with \(\gamma^0\).  Such a wedge will eliminate all the trivectors from the span of \eqnref{eqn:vecMaxProj:spanpartialE}, but can contribute pseudoscalar components from the trivectors in \eqnref{eqn:vecMaxProj:spanpartialB}.  Therefore the desired calculation is

\begin{equation}\label{eqn:vectorMaxwellsProjection:200}
\begin{aligned}
\gpgradetwo{\left(\grad F - J/\epsilon_0 c\right) \wedge \gamma^0}
&= \gpgradetwo{ (({\gamma^{\mu}}_{j0}) \wedge \gamma^0 \partial_{\mu} E^j + (\grad I c B) \wedge \gamma^0 } - (\gamma_0)^2 \BJ/\epsilon_0 c \\
&= \gpgradetwo{ -(({\gamma^{0}}_{0j}) \wedge \gamma^0 \partial_{0} E^j + (\grad I c B) \wedge \gamma^0 } - (\gamma_0)^2 \BJ/\epsilon_0 c \\
&= -{\gamma_{j}}^0 \inv{c} \partial_t E^j + \gpgradetwo{ (\grad I c B) \wedge \gamma^0 } - (\gamma_0)^2 \BJ/\epsilon_0 c \\
&= - \frac{(\gamma_0)^2}{c} \PD{t}{\BE} + c \gpgradeone{ \grad I B} \wedge \gamma^0 - (\gamma_0)^2 \BJ/\epsilon_0 c \\
\end{aligned}
\end{equation}

Let us take just that middle term

\begin{equation}\label{eqn:vectorMaxwellsProjection:220}
\begin{aligned}
\gpgradeone{ \grad I B } \wedge \gamma^0
&= -\gpgradeone{ I \gamma^{\mu} \partial_{\mu} B^k \gamma_{k0} } \wedge \gamma^0 \\
&= - \partial_{\mu} B^k \gpgradeone{ \gamma_{0123} \gamma^{\mu} \gamma_{k0} } \wedge \gamma^0 \\
&= \partial_{\mu} B^k \left(\gpgradetwo{ \gamma_{0123} \gamma^{\mu} \gamma_{0} } \cdot \gamma_k\right) \wedge \gamma^0
\end{aligned}
\end{equation}

Here \(\mu \ne 0\) since that leaves just a pseudoscalar in the grade two selection.
\begin{equation}\label{eqn:vectorMaxwellsProjection:240}
\begin{aligned}
\gpgradeone{ \grad I B } \wedge \gamma^0
&= \partial_{j} B^k \left(\gpgradetwo{ \gamma_{0123} \gamma^{j} \gamma_{0} } \cdot \gamma_k\right) \wedge \gamma^0 \\
&= (\gamma_0)^2 \partial_{j} B^k \left(\gpgradetwo{ \gamma_{123} \gamma^{j} } \cdot \gamma_k\right) \wedge \gamma^0 \\
&= (\gamma_0)^2 \partial_{j} B^k \left(\gpgradetwo{ \epsilon^{hkj}\gamma_{hkj} \gamma^{j} } \cdot \gamma_k\right) \wedge \gamma^0 \\
&= \partial_{j} B^k \epsilon^{hkj} (\gamma_0)^2 (\gamma_k)^2 {\gamma_{h}}^0 \\
&= - (\gamma_0)^2 \partial_{j} B^k \epsilon^{hkj} \sigma_h \\
&= (\gamma_0)^2 \spacegrad \cross \BB
\end{aligned}
\end{equation}

%\partial_{j} B^k \epsilon^{hkj} \sigma_h = - \spacegrad \cross \BB
Putting things back together and factoring out the common metric dependent \((\gamma_0)^2\) term we have

\begin{equation}\label{eqn:vectorMaxwellsProjection:260}
\begin{aligned}
- \inv{c} \PD{t}{\BE} + c \spacegrad \cross \BB - \BJ/\epsilon_0 c &= 0 \\
\implies \\
- \inv{c^2} \PD{t}{\BE} + \spacegrad \cross \BB - \BJ/\epsilon_0 c^2 &= 0
\end{aligned}
\end{equation}

With \(\inv{c^2} = \mu_0 \epsilon_0\) this is the Ampere Maxwell law

\begin{equation}
\spacegrad \cross \BB = \mu_0 \left(\BJ + \epsilon_0 \PD{t}{\BE} \right)
\end{equation}

which we can put in the projection form of \eqnref{eqn:vecMaxProj:gausselectro} and \eqnref{eqn:vecMaxProj:gaussmagnet} as:

\begin{equation}\label{eqn:vecMaxProj:amperemaxwell}
\gpgradetwo{\left(\grad F - J/\epsilon_0 c\right) \wedge (\gamma_0/c)} =
\spacegrad \cross \BB - \mu_0 \left(\BJ + \epsilon_0 \PD{t}{\BE} \right) = 0
\end{equation}

\section{Summary of traditional Maxwell's equations as projective operations on Maxwell Equation}

\begin{equation}\label{eqn:vectorMaxwellsProjection:280}
\begin{aligned}
\gpgradezero{\left(\grad F - J/\epsilon_0 c\right) \gamma^0} &= \spacegrad \cdot \BE - \rho/\epsilon_0 = 0 \\
\gpgradezero{\left(\grad F - J/\epsilon_0 c\right) \cdot \left(\gamma^0 I / c \right)} &= \spacegrad \cdot \BB = 0 \\
\sum \sigma_j \gpgradezero{( \grad F - J/\epsilon_0 c ) \cdot (\gamma^j I)} &= -\spacegrad \cross \BE - \PD{t}{\BB} = 0 \\
\gpgradetwo{\left(\grad F - J/\epsilon_0 c\right) \wedge (\gamma_0/c)} &= \spacegrad \cross \BB - \mu_0 \left(\BJ + \epsilon_0 \PD{t}{\BE} \right) = 0
\end{aligned}
\end{equation}

Faraday's law requiring a sum suggests that this can likely be written instead using a rejective operation.  Will leave that as a possible future followup.


   % this was an earlier version, before I truely figured out how Stokes works on a manifold:
   %%
% Copyright � 2012 Peeter Joot.  All Rights Reserved.
% Licenced as described in the file LICENSE under the root directory of this GIT repository.
%

%
%
% ointclockwise, ointctrclockwise
%\usepackage{txfonts}

\mychapter{Application of Stokes Integrals to Maxwell's Equation}
\index{Stokes theorem!Maxwell's equation}
\label{chap:stokesMaxwellApplication}
%\date{Sept 26, 2008.  stokesMaxwellApplication.tex}

\section{Putting Maxwell's equation in curl form}

These notes contain an application of the bivector Stokes equations
detailed in
\chapcite{PJStokes1}
.  Background of interest can also be found
in \citep{DenkerWire}, which contained the core statement of
the multivector form of Stokes equation and Biot-Savart like application
of it.  Also informative as background is the following excellent
\citep{DenkerMaxwell}.
introduction to the STA form of Maxwell's equation.

Stokes equation applied to a bivector takes the following form
%
\begin{equation}\label{eqn:stokesMax:summaryStokesVolume}
\iiint (\grad \wedge F) \cdot d^3\Bx = \oiintclockwise F \cdot d^2\Bx,
\end{equation}
%
where we will write \(F\) as the electromagnetic field bivector, and apply
it to Maxwell's equation
%
\begin{equation}\label{eqn:stokesMax:maxwell}
\grad F = J/\epsilon_0 c.
\end{equation}
%
Taking vector and trivector parts we have two equations
\begin{equation}\label{eqn:stokesMax:maxwellv}
\grad \cdot F = J/\epsilon_0 c,
\end{equation}
%
and
\begin{equation}\label{eqn:stokesMax:maxwellt}
\grad \wedge F = 0.
\end{equation}
%
\subsection{Trivector equation part}

The second of these, \eqnref{eqn:stokesMax:maxwellt}, we can apply Stokes to
directly:
%
\begin{equation}
\iiint (\grad \wedge F) \cdot d^3 \Bx = \oiintclockwise F \cdot d^2\Bx = 0.
\end{equation}
%
This area integral is a flux like quantity.  Suppose we call this the field flux, then this says
says
the flux of the combined electromagnetic field through any surface is zero
independent of the charge or current densities.
Note that here \(d^3\Bx\)
can be a regular spatial volume trivector element, but one can also pick
a spacetime (area times time) ``volume'' to integrate over, in which case
\(d^2\Bx\) are the oriented ``surfaces'' of such a spacetime volume.

This does not seem like a result that I am familiar with based on the traditional
vector forms of Maxwell's equation.  Perhaps it is recognizable in terms of
\(\BE\) and \(\BB\) explicitly:
%
\begin{equation}\label{eqn:stokesMax:gaussmagnetostatics}
\oiintclockwise \BE \cdot d^2\Bx = - c \oiintclockwise \BB \cdot (d^2\Bx I)
\end{equation}
%
On the surface this does not look like a familiar identity.  It is in fact Gauss's law for magneto-statics, which will be shown
later.

Note also the subtle difference from traditional vector treatments where
\(\BE\) and \(\BB\) were spatial vectors.  Here they are written as spacetime
bivectors,
\(\BE = E^i \sigma_i = E^i \gamma_i \wedge \gamma_0\),
\(\BB = B^i \sigma_i = B^i \gamma_i \wedge \gamma_0\).


\subsection{Vector part}

Moving on to the charge and current dependent vector terms of Maxwell's equation, we want express \eqnref{eqn:stokesMax:maxwellv} as a spacetime curl so that we can apply stokes to it.

We can do this by temporarily writing our field in terms of a potential as well its dual bivector.
%
\begin{equation}\label{eqn:stokesMaxwellApplication:20}
\begin{aligned}
F = \grad \wedge A = I D
\end{aligned}
\end{equation}
\begin{equation}\label{eqn:stokesMaxwellApplication:40}
\begin{aligned}
\grad F
&= \grad (\grad \wedge A) \\
&= \grad \cdot (\grad \wedge A) + \grad \wedge (\grad \wedge A) \\
&= \grad \cdot (I D) \\
&= \gpgradeone{ \grad I D } \\
&= -\gpgradeone{ I (\mathLabelBox{\grad \cdot D}{1 vector} +
\mathLabelBox
[
   labelstyle={below of=m\themathLableNode, below of=m\themathLableNode}
]
{\grad \wedge D}{3 vector}) } \\
&= - I (\grad \wedge D) \\
\end{aligned}
\end{equation}
%
or
\begin{equation}\label{eqn:stokesMaxwellApplication:60}
\begin{aligned}
I \grad F = \grad \wedge D.
\end{aligned}
\end{equation}
%
Applying stokes we have
\begin{equation}\label{eqn:stokesMaxwellApplication:80}
\begin{aligned}
\int (\grad \wedge D) \cdot d^3\Bx &= \oiintclockwise D \cdot d^2\Bx \\
\int (I \grad F) \cdot d^3\Bx
&= \oiintclockwise (-I F) \cdot d^2\Bx \\
&= \oiintclockwise \gpgradezero{ - F d^2\Bx I } \\
&= -\oiintclockwise F \cdot (d^2\Bx I) \\
\inv{\epsilon_0 c} \int (I J) \cdot d^3\Bx &= \\
\inv{\epsilon_0 c} \int \gpgradezero{ I J d^3\Bx } &= \\
\inv{\epsilon_0 c} \int \gpgradezero{ J d^3\Bx I } &= \\
\inv{\epsilon_0 c} \int J \cdot (d^3\Bx I) &= \\
\end{aligned}
\end{equation}
%
Or
\begin{equation}\label{eqn:stokesMax:maxwellint}
\oiintctrclockwise F \cdot (d^2\Bx I) = \int \frac{J}{\epsilon_0 c} \cdot (d^3\Bx I)
\end{equation}
%
This is the integral form of the vector part of Maxwell's equation \eqnref{eqn:stokesMax:maxwell}.
This does not look terribly familiar, but we are not used to
seeing Maxwell's equations in a non-disassembled form.
Hiding in there should be a subset of the
traditional eight Maxwell's equations in integral form.  It will be
possible to extract these by considering variations of current and charge density and different
volume and surface integration regions.

\section{Extracting the vector integral forms of Maxwell's equations}

One can extract the integral forms of Maxwell's equations
from \eqnref{eqn:stokesMax:maxwell}, by first extracting the differential vector
equations, and then using the spatial
divergence and stokes equations.
However, having formulated Stokes equation in its bivector form
we can go directly to those equations by appropriate selection of spatial
or spacetime volumes.
Of course we also now have new tools to work with the field in its entirety,
but lets use this as an exercise to verify that all the previous computation
that led to Stokes equation gives us the expected results.  In particular
this should be a good way to verify that
sign mistakes or other similar small errors (which would not be too hard)
have not been made.

\subsection{Zero current density. Gauss's law for Electrostatics}

With \(J = c \rho \gamma_0\), the integral form of Maxwell's equation becomes
%
\begin{equation}\label{eqn:stokesMaxwellApplication:100}
\begin{aligned}
\oiintctrclockwise F \cdot (d^2\Bx I)
&= \int \frac{\rho}{\epsilon_0} \gpgradezero{\gamma_0 d^3\Bx I} \\
&= \int \frac{\rho}{\epsilon_0} \gpgradezero{\gamma_{0123} \gamma_0 d^3\Bx} \\
&= -\inv{\epsilon_0} {\gamma_0}^2 \int \frac{\rho}{\epsilon_0} \gpgradezero{\gamma_{123} d^3\Bx} \\
\end{aligned}
\end{equation}
%
From this we see that, in the absence of currents the LHS integral must be zero unless the volume is purely spatial.  Denoting the boundary of a spacetime volume as \(\partial A c t\), this is
%
\begin{equation}\label{eqn:stokesMaxwellApplication:120}
\begin{aligned}
\oiintctrclockwise_{\partial {A c t}} F \cdot (d^2\Bx I) &= 0.
\end{aligned}
\end{equation}
%
For a purely spatial volume the dual surfaces \(d^2\Bx I\) always includes a spacetime bivector, therefore the magnetic field contributes nothing
%
\begin{equation*}
\oiintctrclockwise_{\partial V} I c B \cdot (d^2\Bx I) =
-c \oiintctrclockwise_{\partial V} B \cdot d^2\Bx = 0
\end{equation*}
%
Although this looks similar to the integral equivalent of \(\spacegrad \cdot B = 0\), we should look elsewhere for that since
that is true for the non-zero current density case too.

That leaves
%
\begin{equation}\label{eqn:stokesMaxwellApplication:140}
\begin{aligned}
\oiintctrclockwise E \cdot (d^2\Bx I) &= -\inv{\epsilon_0} {\gamma_0}^2 \int_{V} \rho \gpgradezero{\gamma_{123} d^3\Bx} \\
\end{aligned}
\end{equation}
%
Letting \(d^3 \Bx = dx^1 dx^2 dx^3 \gamma_{123}\).  Within the charge integral becomes
%
\begin{equation}\label{eqn:stokesMaxwellApplication:160}
\begin{aligned}
-\inv{\epsilon_0} {\gamma_0}^2 \int_{V} \rho \gpgradezero{\gamma_{123} d^3\Bx}
&=
\inv{\epsilon_0}
\mathLabelBox{{\gamma_0}^2 {\gamma_1}^2}{\(=-1\)}
\mathLabelBox
[
   labelstyle={below of=m\themathLableNode, below of=m\themathLableNode}
]
{{\gamma_2}^2 {\gamma_3}^2}{\(=(\pm 1)^2\)}
 \int_{V} \rho dx^1 dx^2 dx^3
&= -\inv{\epsilon_0} \int_{V} \rho dx^1 dx^2 dx^3
\end{aligned}
\end{equation}
%
To put this in correspondence with the forms we are used to consider the surfaces separately.  For the dual to the
front surface (see: \chapcite{PJStokes1})
we have
%
\begin{equation}\label{eqn:stokesMaxwellApplication:180}
\begin{aligned}
d^2 \Bx I
&= dx^1 dx^2 \gamma_{12} I \\
&= dx^1 dx^2 \gamma_{120123} \\
&= dx^1 dx^2 \gamma_{112023} \\
&= -dx^1 dx^2 \gamma_{112203} \\
&= -(\pm 1)^2 dx^1 dx^2 \gamma_{03} \\
&= dx^1 dx^2 \sigma_3
\end{aligned}
\end{equation}
%
For the left surface
\begin{equation}\label{eqn:stokesMaxwellApplication:200}
\begin{aligned}
d^2 \Bx I
&= dx^3 dx^2 \gamma_{32} I \\
&= dx^3 dx^2 \gamma_{320123} \\
&= dx^3 dx^2 \gamma_{332012} \\
&= dx^3 dx^2 \gamma_{332201} \\
&= dx^3 dx^2 (\pm 1)^2 \gamma_{01} \\
&= -dx^3 dx^2 \sigma_1 \\
\end{aligned}
\end{equation}
%
and for the top
\begin{equation}\label{eqn:stokesMaxwellApplication:220}
\begin{aligned}
d^2 \Bx I
&= dx^1 dx^3 \gamma_{13} I \\
&= dx^1 dx^3 \gamma_{130123} \\
&= dx^1 dx^3 \gamma_{113023} \\
&= dx^1 dx^3 \gamma_{113302} \\
&= -dx^1 dx^3 \sigma_2 \\
\end{aligned}
\end{equation}
%
Assembling results, writing \((x^1, x^2, x^3) = (x,y,z)\) we have
%\begin{align*}
%\iint
%(E_x(x, y, z_0) - E_x(x, y, z_1)) dx dy \\
%-\iint
%(E_y(x_1, y, z) - E_y(x_0, y, z)) dy dz \\
%-\iint
%(E_z(x, y_1, z) - E_z(x, y_0, z)) dx dz \\
%&= -\inv{\epsilon_0} \int_{V} \rho dx dy dz
%\end{align*}
\begin{equation}\label{eqn:stokesMaxwellApplication:240}
\begin{aligned}
\inv{\epsilon_0} \int_{V} \rho dx dy dz
&=
\iint
(E_x(x, y, z_1) - E_x(x, y, z_0)) dx dy \\
&+\iint
(E_y(x_1, y, z) - E_y(x_0, y, z)) dy dz \\
&+\iint
(E_z(x, y_1, z) - E_z(x, y_0, z)) dx dz \\
\end{aligned}
\end{equation}
%
This is Gauss's law for electrostatics in integral form
%
\begin{equation}
\iint \BE \cdot \ncap dA = \iiint \frac{\rho}{\epsilon_0} dV
\end{equation}
%
Although this extraction method is easy to understand, it is apparent that having only a pictorial way of enumerating the
oriented bivector
area elements is not efficient for high level computation.  Revisiting the stokes derivation with a more algebraic enumeration
of the surfaces should be done!

\subsection{Gauss's law for magneto-statics}
\index{Gauss's law}
\index{magnetostatics}

Return now to \eqnref{eqn:stokesMax:gaussmagnetostatics}, which resulted from considering the trivector part of Maxwell's equation
%
\begin{equation}
\oiintclockwise \BE \cdot d^2\Bx = - c \oiintclockwise \BB \cdot (d^2\Bx I).
\end{equation}
%
To start some observations can be made.

Only the spacetime surfaces of the volume
contribute to the LHS integral since \(\sigma_i \cdot (\gamma_j \wedge \gamma_k) = 0\).

For the RHS, only the purely spatial surfaces contribute to that \(\BB\) integral, since the dual surface \(d^2\Bx I\) must have a spacetime component for that dot product to be non-zero.  We have also just enumerated these dual surface area elements \(d^2 \Bx I\) for a purely
spatial surface, therefore with a \(E,B\) substitution we must have
%
\begin{equation}\label{eqn:stokesMaxwellApplication:260}
\begin{aligned}
0 &=
\iint
(B_x(x, y, z_1) - B_x(x, y, z_0)) dx dy  \\
&+\iint
(B_y(x_1, y, z) - B_y(x_0, y, z)) dy dz  \\
&+\iint
(B_z(x, y_1, z) - B_z(x, y_0, z)) dx dz
\end{aligned}
\end{equation}
%
or, more compactly
%
\begin{equation}
\iint \BB \cdot \ncap dA = 0
\end{equation}
%
For any current or charge distribution.  We have therefore obtained two of the eight Maxwell's equations.

\subsection{Zero charge.  Current density in single direction}
\index{current density}

Next to consider is \(J = j^i \gamma_i\).  For simplicity, consider current in only one direction,
taking \(J = j^1 \gamma_1\).  The exercise will be to compute the integrals of \eqnref{eqn:stokesMax:maxwellint}.
%
\begin{equation}\label{eqn:stokesMaxwellApplication:280}
\begin{aligned}
\oiintclockwise F \cdot (I d^2\Bx)
&= \int \frac{J}{\epsilon_0 c} \cdot (I d^3\Bx) \\
&= \int \frac{j^1}{\epsilon_0 c} \gamma_1 \cdot (I d^3\Bx) \\
\end{aligned}
\end{equation}
%
Unlike the calculations for the Gauss's law equations above, this one
will be done using the area orientation
methods from \chapcite{PJStokes2} since algebraically enumerating the surfaces
should make life easier.  The two Gauss's law results above were done without this, which was not too bad for a purely spatial volume, but with spacetime
volumes this is probably confusing in addition to being harder.

Starting with the volume element, one can observe that the current density
will not contribute to the boundary integral unless \(d^3\Bx\) has no \(\gamma_1\)
component, thus for a rectangular prism integration spacetime volume let
\(d^3 \Bx = \gamma_{023} dx^0 dx^2 dx^3\)
%
\begin{equation}\label{eqn:stokesMaxwellApplication:300}
\begin{aligned}
\gamma_1 \cdot (I d^3\Bx)
&= \gamma_1 \cdot \gamma_{0123023} dx^0 dx^2 dx^3 \\
&= \gamma_1 \cdot \gamma_{0012233} dx^0 dx^2 dx^3 \\
&= \gamma_1 \cdot \gamma_{111} dx^0 dx^2 dx^3 \\
&= -\gamma_1 \cdot \gamma^{1} dx^0 dx^2 dx^3 \\
&= - dx^0 dx^2 dx^3 \\
\end{aligned}
\end{equation}
%
Now for all the surfaces we want to calculate
\(I d^2\Bx\) for each of the surfaces.
For each of \(\mu \in \{0, 2, 3\}\), calculation of \(I (d^2\Bx)_\mu\) is required where
%
\begin{equation}\label{eqn:stokesMaxwellApplication:320}
\begin{aligned}
(d^2 \Bx)_\mu &= d^3 \Bx \cdot \Br^{\mu} \\
\Br &= x^i \gamma_i \\
\Br_\mu
&= \PD{x^{\mu}}{\Br} \\
&= \gamma_\mu \\
\Br^\mu &= \gamma^\mu
\end{aligned}
\end{equation}
%
Calculating the surfaces
\begin{equation}\label{eqn:stokesMaxwellApplication:340}
\begin{aligned}
I (d^2 \Bx)_\mu \frac{dx^{\mu}}{dx^0 dx^2 dx^3}
&= \gpgradetwo{ \gamma_{0123} (\gamma_{023} \cdot \gamma^{\mu}) } \\
&= \inv{2} \gpgradetwo{ \gamma_{0123} ( \gamma_{023} \gamma^{\mu} + \gamma^{\mu} \gamma_{023} ) } \\
&= \inv{2} \gpgradetwo{ \gamma_{0123} ( \gamma_{023} \gamma^{\mu} + \gamma_{023} \gamma^{\mu} ) } \\
&= \gpgradetwo{ \gamma_{0012233} \gamma^{\mu} } \\
&= -\gpgradetwo{ \gamma_{133} \gamma^{\mu} } \\
&= -\gpgradetwo{ \gamma^{1} \gamma^{\mu} } \\
&= \gamma^{\mu} \wedge \gamma^{1} \\
\end{aligned}
\end{equation}
%
Putting things back together we have
%
\begin{equation}\label{eqn:stokesMaxwellApplication:360}
\begin{aligned}
-\int j^1 dx^0 dx^2 dx^3 = \int \sum_{\mu = 0,2,3}
\left. F \cdot \left(\gamma^{\mu} \wedge \gamma^{1}\right) \right\vert_{\partial x^{\mu}}
\frac{dx^0 dx^2 dx^3}{dx^{\mu}}
\end{aligned}
\end{equation}
%
Now, for \(\mu=0\) we pick up the electric field component of the field
%
\begin{equation}\label{eqn:stokesMaxwellApplication:380}
\begin{aligned}
F \cdot \gamma^{01}
&= \left( E^i \gamma_{i0} -\epsilon_{ijk} c B^k \gamma_{ij} \right) \cdot \gamma^{01} \\
&= E^i,
\end{aligned}
\end{equation}
%
and for \(\mu=2,3\) we pick up magnetic field components
\begin{equation}\label{eqn:stokesMaxwellApplication:400}
\begin{aligned}
F \cdot \gamma^{\mu1}
&= \left( E^i \gamma_{i0} -\epsilon_{ijk} c B^k \gamma_{ij} \right) \cdot \gamma^{\mu1} \\
&= -\epsilon_{1 \mu k} c B^k \gamma_{1\mu} \cdot \gamma^{\mu1}.
\end{aligned}
\end{equation}
%
For \(\mu=2\) this is \(- c B^3\), and for \(\mu=3\), \(-\epsilon_{132} c B^2 = c B^2\), so we have
%
\begin{equation}\label{eqn:stokesMaxwellApplication:420}
\begin{aligned}
0
&= \int \frac{j^1}{c\epsilon_0} dx^0 dx^2 dx^3
+ \int \left. E^1 dx^2 dx^3 \right\vert_{\partial x^{0}}
+c\int \left. B^2 dx^0 dx^2 \right\vert_{\partial x^{3}}
-c\int \left. B^3 dx^0 dx^3 \right\vert_{\partial x^{2}} \\
&= \int \frac{j^1}{c\epsilon_0} dx^0 dx^2 dx^3
+ \int \PD{x^0}{E^1} dx^0 dx^2 dx^3
+c\int \PD{x^3}{B^2} dx^3 dx^2 dx^0
-c\int \PD{x^2}{B^3} dx^2 dx^0 dx^3  \\
&= \int dx^0 \int dx^2 dx^3 \left(\frac{j^1}{c\epsilon_0} + \inv{c}\PD{t}{E^1} +c\PD{x^3}{B^2} -c\PD{x^2}{B^3} \right) \\
\end{aligned}
\end{equation}
%
If this is zero for all time intervals, then the inner integral is also zero.  Utilizing \(c^2\mu_0\epsilon_0 = 1\) this is
%
\begin{equation}\label{eqn:stokesMaxwellApplication:440}
\begin{aligned}
0
&= \int dx^2 dx^3 \left( \mu_0 \left( j^1 + \epsilon_0 \PD{t}{E^1} \right) + \left( \PD{x^3}{B^2} -\PD{x^2}{B^3} \right) \right).
\end{aligned}
\end{equation}
%
Writing \(d\BA = \sigma_1 dx^2 dx^3\), \(\Bj = j^1 \sigma_1\), \(\BE = E^1 \sigma_1\), and \(\BB = B^i \sigma_i\) we can pick off
the differential form of the Maxwell-Ampere equation
%
\begin{equation}
\spacegrad \cross \BB = \mu_0 \left( \Bj + \epsilon_0 \PD{t}{\BE} \right),
\end{equation}
%
as well as the integral form
\begin{equation}
\int (\spacegrad \cross \BB) \cdot d\BA
= \mu_0 \left(\int \Bj \cdot d\BA + \epsilon_0 \int \PD{t}{\BE} \cdot d\BA \right)
\end{equation}
%
Both of these forms come straight from the application of the generalized Stokes equation integrating an appropriate
spacetime volume.

Now it is normal to have the spatial curl of \(\BB\) written as a closed loop integral.  Stokes can be employed
again to get exactly that form.  This really just undoes the fact that the partials to used as a convenience enumerate
exactly those loop boundaries (although they were originally oriented area boundaries).
%
\begin{equation}\label{eqn:stokesMaxwellApplication:460}
\begin{aligned}
\int \PD{x^3}{B^2} dx^3 &= B^2(t, x, y, z_1) - B^2(t, x, y, {z}_0) \\
\int \PD{x^2}{B^3} dx^2 &= B^3(t, x, {y}_1, z) - B^3(t, x, {y}_0, z)
\end{aligned}
\end{equation}
%
Also observe that this whole treatment was
done with \(J = j^1 \gamma_1\) only.  It is not hard to see that doing the same with \(j^i\) and summing over \(\sigma_i\)
will produce the same result.  Of course more care is required to handle the more abstract symbolic indices since a nice
hard-coded number is easier.
On the other hand the usual dodge, employing freedom to orient the coordinate system along the
\(\gamma_1\) direction makes the more general algebraic approach a less interesting exercise.

\subsection{Faraday's law}
\index{Faraday's law}

We have five of the eight Maxwell's equations.  Gauss's law for electrostatics
from the vector part of \eqnref{eqn:stokesMax:maxwellv}, integrating over a spatial
volume, and the Maxwell-Ampere equation from the same, integrating over
a spacetime volume.  Gauss's law for magneto-statics from the trivector part
of \eqnref{eqn:stokesMax:maxwellv}, integrating over a spatial volume.  This suggests
that our remaining three (one three-vector) equation will come from
integrating the trivector parts over a spacetime volume.

Stokes' gives us
%
\begin{equation*}
\int_V (\grad \wedge F) \cdot d^3 \Bx = \int_{\partial V} F \cdot (d^2\Bx)
\end{equation*}
%
Picking a spacetime volume element, and corresponding area elements
%
\begin{equation}\label{eqn:stokesMaxwellApplication:480}
\begin{aligned}
d^3 \Bx &= \gamma_{0ij} dx^0 dx^i dx^j \\
(d^2 \Bx)_\mu &= (\gamma_{0ij} \cdot \gamma^\mu) \frac{dx^0 dx^i dx^j}{dx^\mu}
\end{aligned}
\end{equation}
%
Our area integral (expanding boundaries as one more integral of partials) is
\begin{equation*}
\int \sum_{\mu = 0,i,j} dx^0 dx^i dx^j \left( \PD{x^\mu}{F} \cdot (\gamma_{0ij}\cdot\gamma^\mu) \right).
\end{equation*}
%
For the dot products of the area elements we have
%
\begin{equation*}
\left\{
\begin{array}{l l}
\gamma_{ij} & \quad \mbox{if \(\mu = 0\)} \\
\gamma_{0i} = -\sigma_i & \quad \mbox{if \(\mu = j\)} \\
-\gamma_{0j} = \sigma_j & \quad \mbox{if \(\mu = i\)} \\
\end{array} \right.
\end{equation*}
%
Our field derivatives in coordinates are
%
\begin{equation*}
\PD{x^\mu}{F}
= \PD{x^\mu}{E^m} \sigma_m - \epsilon_{klm} c \PD{x^\mu}{B^m} \gamma_{kl}
\end{equation*}
%
Observe that \(\mu\ne0\) selects only the electric field components, and \(\mu=0\) only the magnetic field components are selected.  Specifically
%
\begin{equation*}
\PD{x^\mu}{F} =
\left\{
\begin{array}{l l l}
-\epsilon_{jim} c \PD{x^0}{B^m} (\gamma_i)^2(\gamma_j)^2 &= \epsilon_{ijk}\PD{t}{B^k} & \quad \mbox{if \(\mu = 0\)} \\
\PD{x^j}{E^m}\sigma_m \cdot (-\sigma_i) &= -\PD{x^j}{E^i} & \quad \mbox{if \(\mu = j\)} \\
\PD{x^i}{E^m}\sigma_m \cdot (\sigma_j) &= \PD{x^i}{E^j} & \quad \mbox{if \(\mu = i\)} \\
\end{array} \right.
\end{equation*}
%
Reassembling the integral we have
%
\begin{equation}\label{eqn:stokesMaxwellApplication:500}
\begin{aligned}
0
&= \int dx^0 dx^i dx^j \left( \PD{x^i}{E^j} -\PD{x^j}{E^i} + \epsilon_{ijk} \PD{t}{B^k} \right) \\
&= \int dx^0 \epsilon_{ijk} \int dx^i dx^j \sigma_k \left( \sigma_k \epsilon_{ijk} \left(\PD{x^i}{E^j} -\PD{x^j}{E^i}\right) + \sigma_k \PD{t}{B^k} \right)
\end{aligned}
\end{equation}
%
Summing over \(k\), we can pick out the differential form of Faraday's law
%
\begin{equation}
0 = \PD{t}{\BB} + \spacegrad \cross \BE
\end{equation}
%
as well as the integral form
%
\begin{equation}\label{eqn:stokesMaxwellApplication:520}
\begin{aligned}
0
&=
\sum_k
\int dx^i dx^j \sigma_k \left( \sigma_k \epsilon_{ijk} \left(\PD{x^i}{E^j} -\PD{x^j}{E^i}\right) + \sigma_k \PD{t}{B^k} \right) \\
&=
\sum_k \epsilon_{ijk} \int dx^j \left.E^j\right\vert_{\partial x^i}
-\sum_k \epsilon_{ijk} \int dx^i \left.E^i\right\vert_{\partial x^j}
+ \int \PD{t}{\BB} \cdot \ncap d\BA
\end{aligned}
\end{equation}
%
which is
\begin{equation}
0 = \ointctrclockwise \BE \cdot d\Br + \int \PD{t}{\BB} \cdot d\BA.
\end{equation}
%
\section{Conclusion}

In the treatment of these notes, the
traditional integral form of Maxwell's equations are
obtained directly from the STA Maxwell's equation
using the bivector Stokes equation, and various spacetime integration volumes.

\subsection{Summary of results}

We started with the bivector form of Stokes law
%
\begin{equation}
\iiint (\grad \wedge F) \cdot d^3\Bx = \oiintclockwise F \cdot d^2\Bx,
\end{equation}
%
and the multivector Maxwell equation
%
\begin{equation}\label{eqn:stokesMax:summaryMaxwell}
\grad F = J/\epsilon_0 c.
\end{equation}
%
The trivector parts of this can be integrated directly.  This integral is always zero for all spacetime or spatial surfaces
%
\begin{equation}\label{eqn:stokesMaxwellApplication:540}
\begin{aligned}
\int \left(\grad \wedge F\right) \cdot d^3 \Bx &= 0
\end{aligned}
\end{equation}
%
Duality relations were used to put the vector parts of \eqnref{eqn:stokesMax:summaryMaxwell} into a form that Stokes can be applied to.  This gives us
%
\begin{equation}\label{eqn:stokesMaxwellApplication:560}
\begin{aligned}
\oiintctrclockwise F \cdot (d^2\Bx I) &= \int \frac{J}{\epsilon_0 c} \cdot (d^3\Bx I).
\end{aligned}
\end{equation}
%
Integration of the trivector parts
%
\begin{equation}
\iiint (\grad \wedge F) \cdot d^3 \Bx = \oiintclockwise F \cdot d^2\Bx = 0,
\end{equation}
%
produces a combined electric and magnetic field form of a Faraday's law and Gauss' magneto-statics law that does not look terribly familiar
%
\begin{equation}\label{eqn:stokesMax:summaryGaussMandFaraday}
\oiintclockwise \BE \cdot d^2\Bx = - c \oiintclockwise \BB \cdot (d^2\Bx I),
\end{equation}
%
but integration of this using a spatial volume produces the familiar Gauss's magneto-static law
%
\begin{equation}\label{eqn:stokesMaxwellApplication:580}
\begin{aligned}
\iint \BB \cdot d\BA &= 0 \\
\spacegrad \cdot \BB  &= 0.
\end{aligned}
\end{equation}
%
Integration and summation of the same trivector parts in \eqnref{eqn:stokesMax:summaryGaussMandFaraday} over each of the possible three spacetime volumes gives us Faraday's law
in its familiar forms
\begin{equation}\label{eqn:stokesMaxwellApplication:600}
\begin{aligned}
\PD{t}{\BB} + \spacegrad \cross \BE &= 0 \\
\ointctrclockwise \BE \cdot d\Br + \int \PD{t}{\BB} \cdot d\BA &= 0.
\end{aligned}
\end{equation}
%
Now, the vector parts of Maxwell's multivector equation integrated over a spatial volume produces Gauss's law for electrostatics
%
\begin{equation}\label{eqn:stokesMaxwellApplication:620}
\begin{aligned}
\iint \BE \cdot d\BA &= \int \frac{\rho}{\epsilon_0} dV \\
\spacegrad \cdot \BE &= \frac{\rho}{\epsilon_0}.
\end{aligned}
\end{equation}
%
Finally, integration of the same with summation over all spacetime volumes gives us the famous Maxwell-Ampere equation
\begin{equation}\label{eqn:stokesMaxwellApplication:640}
\begin{aligned}
\spacegrad \cross \BB &= \mu_0 \left( \Bj + \epsilon_0 \PD{t}{\BE} \right) \\
\ointctrclockwise \BB \cdot d\Br &= \mu_0 \left(\int \Bj \cdot d\BA + \epsilon_0 \int \PD{t}{\BE} \cdot d\BA \right).
\end{aligned}
\end{equation}
%
In the process of arriving at these results it appears that some of the use of Stokes equation was actually superfluous.  One of the first things
that was done once the area elements were established was to undo the boundary integral writing things once more in terms of the partials over those boundaries.
Doing all this with just the volume integrals would possibly have been simpler.  That said, as an exercise to validate the generalized Stokes equation
formulation it worked well!

Conceptually the idea that integration of Maxwell's equation over various volumes produces all the traditional vector differential and integral forms
that we are used to is quite nice.  It seems less arbitrary than trying to figure out the exactly what specific projection like operations, as done in
\chapcite{PJMaxwellProj}, will produce the various traditional vector differential equations.  Of course those can be used once found to develop the integral relations,
but here we get them all in one shot.
%The old 1966 Britannica article uses the integral forms

\subsection{Getting a glimpse of how the pieces fit together?}

I think I am starting to see a bit of the big picture for electrodynamics.
In \chapcite{PJMaxwell2}, an earlier treatment of Maxwell's equations in a GA context, I used
dimensional analysis to group electric and magnetic fields in a logical way, and employs the spatial pseudoscalar to combine divergence and curl terms.  This
I thought was a good motivation for the STA form of the equation, using
ideas familiar from school.  Similar treatments can be found elsewhere
such as in \citep{doran2003gap}
but understanding that takes a lot more work.

Once the STA form is taken as more fundamental, one can take that and show
the types of spacetime projection operations, as in
\chapcite{PJMaxwellProj}, and produce the various traditional vector
differential forms of Maxwell's equations.  Alternatively, as in
\chapcite{PJMaxwellTensor}, we can extract the traditional tensor
form of the equations.

From an even higher level point of view we can relate the STA Maxwell's
equations to the least action principles, as done in
\citep{classicalmechanics:PJSrLagrangian}, to
find the Lorentz force law in STA form using the Euler-Lagrange equations,
and finally in \citep{classicalmechanics:PJMaxwellLagrangian} where the STA
form of Maxwell's equation is obtained directly from a complex valued
field Lagrangian.

Goldstein
\citep{goldstein1951cm}
has an interesting treatment of a combined Lagrangian for both
the Lorentz force law and the field equations (using spatial delta functions).  Minimization of the action for that Lagrangian with respect to the potential
produces the field equations, and with respect to coordinates produces the
Lorentz force law.  Have to work through that in a covariant form to see
how this relates to my previous treatments.

\subsection{Followup}

It would be interesting to see if any of the problems in a Maxwell's equation
text like
\citep{fleisch2007ssg}
would be any easier with a combined field as
is possible in the STA formulation (ie: the ones based on just current
or charge distributions).

There is also some interesting looking treatments of complex number residue like integrals for the field equation
in references such as
\citep{HestenesFormsGA}.
I re-encountered that paper after writing up these notes.  I had seen it before but
those parts that cover (tersely) the same material as above did not make much sense until I had independently worked it all
out in detail myself.
Perhaps I am dense, but I find that many academic papers are ironically not very good at all for learning from!

I believe these residue/green's function ideas both relate to the
Biot-Savart law, as mentioned in
\citep{HestenesFormsGA}, \citep{doran2003gap}, and
\citep{DenkerWire}.  All of those are either too terse or have details missing
that indicate I need to study the ideas in more depth to understand.

%\bibliographystyle{plain}

   %
% Copyright � 2016 Peeter Joot.  All Rights Reserved.
% Licenced as described in the file LICENSE under the root directory of this GIT repository.
%
%{
%\input{../blogpost.tex}
%\renewcommand{\basename}{maxwellStokes}
%\renewcommand{\dirname}{notes/phy1520/}
%%\newcommand{\dateintitle}{}
%%\newcommand{\keywords}{}
%
%\input{../peeter_prologue_print2.tex}
%
%\usepackage{peeters_layout_exercise}
%\usepackage{peeters_braket}
%\usepackage{peeters_figures}
%\usepackage{siunitx}
%
%\beginArtNoToc
%
%\generatetitle{Application of Stokes Theorem to the Maxwell equation}
\mychapter{Application of Stokes Theorem to the Maxwell equation}
%\label{chap:maxwellStokes}
% \citep{sakurai2014modern} pr X.Y
% \citep{pozar2009microwave}
% \citep{qftLectureNotes}
% \citep{griffiths1999introduction}

\section{Spacetime domain}
%Recall that the relativistic form of Maxwell's equation in Geometric Algebra is
We've seen the relativistic form of Maxwell's equation in Geometric Algebra

\begin{equation}\label{eqn:maxwellStokes:20}
\grad F = \inv{c \epsilon_0} J.
\end{equation}

%where \( \grad = \gamma^\mu \partial_\mu \) is the spacetime gradient, and \( J = (c\rho, \BJ) = J^\mu \gamma_\mu \) is the four (vector) current density.
%The pseudoscalar for the space is denoted \( I = \gamma_0 \gamma_1 \gamma_2 \gamma_3 \), where the basis elements satisfy \( \gamma_0^2 = 1 = -\gamma_k^2 \), and a dual basis satisfies \( \gamma_\mu \cdot \gamma^\nu = \delta_\mu^\nu \).
%The electromagnetic field \( F \) is a composite multivector \( F = \BE + I c \BB \).  This is actually a bivector because spatial vectors have a bivector representation in the space time algebra of the form \( \BE = E^k \gamma_k \gamma_0 \).
%
, but a dual representation, with \( F = I G \) is also possible

\begin{equation}\label{eqn:maxwellStokes:60}
\grad G = \frac{I}{c \epsilon_0} J.
\end{equation}

Either form of Maxwell's equation can be split into grade one and three components.  The standard (non-dual) form is

\begin{equation}\label{eqn:maxwellStokes:40}
\begin{aligned}
\grad \cdot F &= \inv{c \epsilon_0} J \\
\grad \wedge F &= 0,
\end{aligned}
\end{equation}

and the dual form is

\begin{equation}\label{eqn:maxwellStokes:41}
\begin{aligned}
\grad \cdot G &= 0 \\
\grad \wedge G &= \frac{I}{c \epsilon_0} J.
\end{aligned}
\end{equation}

In both cases a potential representation \( F = \grad \wedge A \), where \( A \) is a four vector potential can be used to kill off the non-current equation.  Such a potential representation reduces Maxwell's equation to

\begin{equation}\label{eqn:maxwellStokes:80}
\grad \cdot F = \inv{c \epsilon_0} J,
\end{equation}

or
\begin{equation}\label{eqn:maxwellStokes:100}
\grad \wedge G = \frac{I}{c \epsilon_0} J.
\end{equation}

In both cases, these reduce to
\begin{equation}\label{eqn:maxwellStokes:120}
\grad^2 A - \grad \lr{ \grad \cdot A } = \inv{c \epsilon_0} J.
\end{equation}

This can clearly be further simplified by using the Lorentz gauge, where \( \grad \cdot A = 0 \).  However, the aim for now is to try applying Stokes theorem to Maxwell's equation.  The dual form \eqnref{eqn:maxwellStokes:100} has the curl structure required for the application of Stokes.  Suppose that we evaluate this curl over the three parameter volume element \( d^3 x = i\, dx^0 dx^1 dx^2 \), where \( i = \gamma_0 \gamma_1 \gamma_2 \) is the unit pseudoscalar for the spacetime volume element.

\begin{equation}\label{eqn:maxwellStokes:101}
\begin{aligned}
\int_V d^3 x \cdot \lr{ \grad \wedge G }
&=
\int_V d^3 x \cdot \lr{ \gamma^\mu \wedge \partial_\mu G } \\
&=
\int_V \lr{ d^3 x \cdot \gamma^\mu } \cdot \partial_\mu G \\
&=
\sum_{\mu \ne 3} \int_V \lr{ d^3 x \cdot \gamma^\mu } \cdot \partial_\mu G.
\end{aligned}
\end{equation}

This uses the distribution identity \( A_s \cdot (a \wedge A_r) = (A_s \cdot a) \cdot A_r \) which holds for blades \( A_s, A_r \) provided \( s > r > 0 \).  Observe that only the component of the gradient that lies in the tangent space of the three volume manifold contributes to the integral, allowing the gradient to be used in the Stokes integral instead of the vector derivative (see: \citep{aMacdonaldVAGC}).
Defining the the surface area element

\begin{equation}\label{eqn:maxwellStokes:140}
\begin{aligned}
d^2 x
&= \sum_{\mu \ne 3} i \cdot \gamma^\mu \inv{dx^\mu} d^3 x \\
&= \gamma_1 \gamma_2 dx dy
+ c \gamma_2 \gamma_0 dt dy
+ c \gamma_0 \gamma_1 dt dx,
\end{aligned}
\end{equation}

Stokes theorem for this volume element is now completely specified

\begin{equation}\label{eqn:maxwellStokes:200}
\int_V d^3 x \cdot \lr{ \grad \wedge G }
=
\int_{\partial V} d^2 \cdot G.
\end{equation}

Application to the dual Maxwell equation gives

\begin{equation}\label{eqn:maxwellStokes:160}
\int_{\partial V} d^2 x \cdot G
= \inv{c \epsilon_0} \int_V d^3 x \cdot (I J).
\end{equation}

After some manipulation, this can be restated in the non-dual form

%\begin{dmath}\label{eqn:maxwellStokes:180}
\boxedEquation{eqn:maxwellStokes:180}{
\int_{\partial V} \inv{I} d^2 x \wedge F
= \frac{1}{c \epsilon_0 I} \int_V d^3 x \wedge J.
}
%\end{dmath}

It can be demonstrated that using this with each of the standard basis spacetime 3-volume elements recovers Gauss's law and the Ampere-Maxwell equation.  So, what happened to Faraday's law and Gauss's law for magnetism?  With application of Stokes to the curl equation from \eqnref{eqn:maxwellStokes:40}, those equations take the form

%\begin{dmath}\label{eqn:maxwellStokes:240}
\boxedEquation{eqn:maxwellStokes:240}{
\int_{\partial V} d^2 x \cdot F = 0.
}
%\end{dmath}

\makeproblem{}{problem:maxwellStokes:4}{
Demonstrate that the Ampere-Maxwell equation and Gauss's law can be recovered from the trivector (curl) equation \cref{eqn:maxwellStokes:100}.
} % problem

\makeanswer{problem:maxwellStokes:4}{

The curl equation is a trivector on each side, so dotting it with each of the four possible trivectors \( \gamma_0 \gamma_1 \gamma_2, \gamma_0 \gamma_2 \gamma_3, \gamma_0 \gamma_1 \gamma_3, \gamma_1 \gamma_2 \gamma_3 \) will give four different scalar equations.  For example, dotting with \( \gamma_0 \gamma_1 \gamma_2 \), we have for the curl side

\begin{equation}\label{eqn:maxwellStokes:460}
\begin{aligned}
\lr{ \gamma_0 \gamma_1 \gamma_2 } \cdot \lr{ \gamma^\mu \wedge \partial_\mu G }
&=
\lr{ \lr{ \gamma_0 \gamma_1 \gamma_2 } \cdot \gamma^\mu } \cdot \partial_\mu G \\
&=
 (\gamma_0 \gamma_1) \cdot \partial_2 G
+(\gamma_2 \gamma_0) \cdot \partial_1 G
+(\gamma_1 \gamma_2) \cdot \partial_0 G,
\end{aligned}
\end{equation}

and for the current side, we have

%0 1 2 0 1 2 => 0 0 1 2 1 2 => - 0 0 1 1 2 2 => -1
\begin{equation}\label{eqn:maxwellStokes:480}
\begin{aligned}
\inv{\epsilon_0 c} \lr{ \gamma_0 \gamma_1 \gamma_2 } \cdot \lr{ I J }
&=
\inv{\epsilon_0 c} \gpgradezero{ \gamma_0 \gamma_1 \gamma_2 (\gamma_0 \gamma_1 \gamma_2 \gamma_3) J } \\
&=
\inv{\epsilon_0 c} \gpgradezero{ -\gamma_3 J } \\
&=
\inv{\epsilon_0 c} \gamma^3 \cdot J \\
&=
\inv{\epsilon_0 c} J^3,
\end{aligned}
\end{equation}

so we have
\begin{equation}\label{eqn:maxwellStokes:500}
 (\gamma_0 \gamma_1) \cdot \partial_2 G
+(\gamma_2 \gamma_0) \cdot \partial_1 G
+(\gamma_1 \gamma_2) \cdot \partial_0 G
=
\inv{\epsilon_0 c} J^3.
\end{equation}

Similarly, dotting with \( \gamma_{013}, \gamma_{023}, and \gamma_{123} \) respectively yields
\begin{equation}\label{eqn:maxwellStokes:620}
\begin{aligned}
\gamma_{01} \cdot \partial_3 G + \gamma_{30} \partial_1 G + \gamma_{13} \partial_0 G &= - \inv{\epsilon_0 c} J^2 \\
\gamma_{02} \cdot \partial_3 G + \gamma_{30} \partial_2 G + \gamma_{23} \partial_0 G &= \inv{\epsilon_0 c} J^1 \\
\gamma_{12} \cdot \partial_3 G + \gamma_{31} \partial_2 G + \gamma_{23} \partial_1 G &= -\inv{\epsilon_0} \rho.
\end{aligned}
\end{equation}

Expanding the dual electromagnetic field, first in terms of the spatial vectors, and then in the space time basis, we have
\begin{equation}\label{eqn:maxwellStokes:520}
\begin{aligned}
G
&= -I F \\
&= -I \lr{ \BE + I c \BB } \\
&= -I \BE + c \BB. \\
&= -I \BE + c B^k \gamma_k \gamma_0 \\
&= \inv{2} \epsilon^{r s t} \gamma_r \gamma_s E^t + c B^k \gamma_k \gamma_0.
\end{aligned}
\end{equation}

So, dotting with a spatial vector will pick up a component of \( \BB \), we have
\begin{equation}\label{eqn:maxwellStokes:540}
\begin{aligned}
\lr{ \gamma_m \wedge \gamma_0 } \cdot \partial_\mu G
&=
\lr{ \gamma_m \wedge \gamma_0 } \cdot \partial_\mu \lr{
\inv{2} \epsilon^{r s t} \gamma_r \gamma_s E^t + c B^k \gamma_k \gamma_0
} \\
&=
c \partial_\mu B^k
\gpgradezero{
\gamma_m \gamma_0 \gamma_k \gamma_0
} \\
&=
c \partial_\mu B^k
\gpgradezero{
\gamma_m \gamma_0 \gamma_0 \gamma^k
} \\
&=
c \partial_\mu B^k
\delta_m^k \\
&=
c \partial_\mu B^m.
\end{aligned}
\end{equation}

Written out explicitly the electric field contributions to \( G \) are

\begin{equation}\label{eqn:maxwellStokes:560}
\begin{aligned}
-I \BE
&=
-\gamma_{0123k0} E^k \\
&=
-\gamma_{123k} E^k \\
&=
\left\{
\begin{array}{l l}
\gamma_{12} E^3 & \quad \mbox{\( k = 3 \)} \\
\gamma_{31} E^2 & \quad \mbox{\( k = 2 \)} \\
\gamma_{23} E^1 & \quad \mbox{\( k = 1 \)} \\
\end{array}
\right.,
\end{aligned}
\end{equation}

so
\begin{equation}\label{eqn:maxwellStokes:580}
\begin{aligned}
\gamma_{23} \cdot G &= -E^1 \\
\gamma_{31} \cdot G &= -E^2 \\
\gamma_{12} \cdot G &= -E^3.
\end{aligned}
\end{equation}

We now have the pieces required to expand \cref{eqn:maxwellStokes:500} and \cref{eqn:maxwellStokes:620}, which are respectively

\begin{equation}\label{eqn:maxwellStokes:501}
\begin{aligned}
- c \partial_2 B^1 + c \partial_1 B^2 - \partial_0 E^3 &= \inv{\epsilon_0 c} J^3 \\
- c \partial_3 B^1 + c \partial_1 B^3 + \partial_0 E^2 &= -\inv{\epsilon_0 c} J^2 \\
- c \partial_3 B^2 + c \partial_2 B^3 - \partial_0 E^1 &= \inv{\epsilon_0 c} J^1 \\
- \partial_3 E^3 - \partial_2 E^2 - \partial_1 E^1 &= - \inv{\epsilon_0} \rho
\end{aligned}
\end{equation}

which are the components of the Ampere-Maxwell equation, and Gauss's law

\begin{equation}\label{eqn:maxwellStokes:600}
\begin{aligned}
\inv{\mu_0} \spacegrad \cross \BB - \epsilon_0 \PD{t}{\BE} &= \BJ \\
\spacegrad \cdot \BE &= \frac{\rho}{\epsilon_0}.
\end{aligned}
\end{equation}
} % answer

\makeproblem{}{problem:maxwellStokes:0}{
Prove \cref{eqn:maxwellStokes:180}.
} % problem

\makeanswer{problem:maxwellStokes:0}{

The proof just requires the expansion of the dot products using scalar selection

\begin{equation}\label{eqn:maxwellStokes:260}
\begin{aligned}
d^2 x \cdot G
&=
\gpgradezero{ d^2 x (-I) F } \\
&=
-\gpgradezero{ I d^2 x F } \\
&=
-I \lr{ d^2 x \wedge F },
\end{aligned}
\end{equation}

and
for the three volume dot product

\begin{equation}\label{eqn:maxwellStokes:280}
\begin{aligned}
d^3 x \cdot (I J)
&=
\gpgradezero{
d^3 x\, I J
} \\
&=
-\gpgradezero{
I d^3 x\, J
} \\
&=
-I \lr{ d^3 x \wedge J }.
\end{aligned}
\end{equation}
} % answer

\makeproblem{}{problem:maxwellStokes:1}{
Using each of the four possible spacetime volume elements, write out the components of the Stokes integral
\cref{eqn:maxwellStokes:180}.
} % problem

\makeanswer{problem:maxwellStokes:1}{
The four possible volume and associated area elements are
\begin{equation}\label{eqn:maxwellStokes:220}
\begin{aligned}
d^3 x = c \gamma_0 \gamma_1 \gamma_2 dt dx dy & \qquad d^2 x = \gamma_1 \gamma_2 dx dy + c \gamma_2 \gamma_0 dy dt + c \gamma_0 \gamma_1 dt dx \\
d^3 x = c \gamma_0 \gamma_1 \gamma_3 dt dx dz & \qquad d^2 x = \gamma_1 \gamma_3 dx dz + c \gamma_3 \gamma_0 dz dt + c \gamma_0 \gamma_1 dt dx \\
d^3 x = c \gamma_0 \gamma_2 \gamma_3 dt dy dz & \qquad d^2 x = \gamma_2 \gamma_3 dy dz + c \gamma_3 \gamma_0 dz dt + c \gamma_0 \gamma_2 dt dy \\
d^3 x = \gamma_1 \gamma_2 \gamma_3 dx dy dz & \qquad d^2 x = \gamma_1 \gamma_2 dx dy + \gamma_2 \gamma_3 dy dz + c \gamma_3 \gamma_1 dz dx \\
\end{aligned}
\end{equation}

Wedging the area element with \( F \) will produce pseudoscalar multiples of the various \( \BE \) and \( \BB \) components, but a recipe for these components is required.

First note that for \( k \ne 0 \), the wedge \( \gamma_k \wedge \gamma_0 \wedge F \) will just select components of \( \BB \).  This can be seen first by simplifying

\begin{equation}\label{eqn:maxwellStokes:300}
\begin{aligned}
I \BB
&=
\gamma_{0 1 2 3} B^m \gamma_{m 0} \\
&=
\left\{
\begin{array}{l l}
   \gamma_{3 2} B^1 & \quad \mbox{\( m = 1 \)}  \\
   \gamma_{1 3} B^2 & \quad \mbox{\( m = 2 \)}  \\
   \gamma_{2 1} B^3 & \quad \mbox{\( m = 3 \)}
\end{array}
\right.,
\end{aligned}
\end{equation}

or

\begin{equation}\label{eqn:maxwellStokes:320}
I \BB = - \epsilon_{a b c} \gamma_{a b} B^c.
\end{equation}

From this it follows that

\begin{equation}\label{eqn:maxwellStokes:340}
\gamma_k \wedge \gamma_0 \wedge F = I c B^k.
\end{equation}

The electric field components are easier to pick out.  Those are selected by

\begin{equation}\label{eqn:maxwellStokes:360}
\begin{aligned}
\gamma_m \wedge \gamma_n \wedge F
&= \gamma_m \wedge \gamma_n \wedge \gamma_k \wedge \gamma_0 E^k \\
&= -I E^k \epsilon_{m n k}.
\end{aligned}
\end{equation}

The respective volume element wedge products with \( J \) are

\begin{equation}\label{eqn:maxwellStokes:400}
\begin{aligned}
\inv{I} d^3 x \wedge J = \inv{c \epsilon_0} J^3
\inv{I} d^3 x \wedge J = \inv{c \epsilon_0} J^2
\inv{I} d^3 x \wedge J = \inv{c \epsilon_0} J^1,
\end{aligned}
\end{equation}

and the respective sum of surface area elements wedged with the electromagnetic field are
%d^3 x = c \gamma_0 \gamma_1 \gamma_2 dt dx dy & \qquad d^2 x = \gamma_1 \gamma_2 dx dy + c \gamma_2 \gamma_0 dy dt + c \gamma_0 \gamma_1 dt dx \\
%d^3 x = c \gamma_0 \gamma_1 \gamma_3 dt dx dz & \qquad d^2 x = \gamma_1 \gamma_3 dx dz + c \gamma_3 \gamma_0 dz dt + c \gamma_0 \gamma_1 dt dx \\
%d^3 x = c \gamma_0 \gamma_2 \gamma_3 dt dy dz & \qquad d^2 x = \gamma_2 \gamma_3 dy dz + c \gamma_3 \gamma_0 dz dt + c \gamma_0 \gamma_2 dt dy \\
\begin{equation}\label{eqn:maxwellStokes:380}
\begin{aligned}
\inv{I} d^2 x \wedge F &= - \evalbar{E^3}{c \Delta t} dx dy + c \lr{ \evalbar{B^2}{\Delta x} dy - \evalbar{B^1}{\Delta y} dx } dt \\
\inv{I} d^2 x \wedge F &=   \evalbar{E^2}{c \Delta t} dx dz + c \lr{ \evalbar{B^3}{\Delta x} dz - \evalbar{B^1}{\Delta z} dx } dt \\
\inv{I} d^2 x \wedge F &= - \evalbar{E^1}{c \Delta t} dy dz + c \lr{ \evalbar{B^3}{\Delta y} dz - \evalbar{B^2}{\Delta z} dy } dt \\
\inv{I} d^2 x \wedge F &= - \evalbar{E^3}{\Delta z} dy dx - \evalbar{E^2}{\Delta y} dx dz - \evalbar{E^1}{\Delta x} dz dy,
\end{aligned}
\end{equation}

so
\begin{equation}\label{eqn:maxwellStokes:381}
\begin{aligned}
\int_{\partial V} - \evalbar{E^3}{c \Delta t} dx dy + c \lr{ \evalbar{B^2}{\Delta x} dy - \evalbar{B^1}{\Delta y} dx } dt &=
 c \int_V dx dy dt \inv{c \epsilon_0} J^3 \\
\int_{\partial V} \evalbar{E^2}{c \Delta t} dx dz + c \lr{ \evalbar{B^3}{\Delta x} dz - \evalbar{B^1}{\Delta z} dx } dt &=
 -c \int_V dx dy dt \inv{c \epsilon_0} J^2 \\
\int_{\partial V} - \evalbar{E^1}{c \Delta t} dy dz + c \lr{ \evalbar{B^3}{\Delta y} dz - \evalbar{B^2}{\Delta z} dy } dt &=
 c \int_V dx dy dt \inv{c \epsilon_0} J^1 \\
\int_{\partial V} - \evalbar{E^3}{\Delta z} dy dx - \evalbar{E^2}{\Delta y} dx dz - \evalbar{E^1}{\Delta x} dz dy &=
 -\int_V dx dy dz \inv{\epsilon_0} \rho.
\end{aligned}
\end{equation}

Observe that if the volume elements are taken to their infinitesimal limits, we recover the traditional differential forms of the Ampere-Maxwell and Gauss's law equations.
} % answer

%}
%\EndArticle

   %
% Copyright � 2016 Peeter Joot.  All Rights Reserved.
% Licenced as described in the file LICENSE under the root directory of this GIT repository.
%
%{
%\input{../blogpost.tex}
%\renewcommand{\basename}{stokesMaxwellSpaceTimeSplit}
%\renewcommand{\dirname}{notes/phy1520/}
%%\newcommand{\dateintitle}{}
%%\newcommand{\keywords}{}
%
%\input{../peeter_prologue_print2.tex}
%
%\usepackage{peeters_layout_exercise}
%\usepackage{peeters_braket}
%\usepackage{peeters_figures}
%\usepackage{siunitx}
%\usepackage{txfonts} % \ointclockwise
%
%\beginArtNoToc
%
%\generatetitle{Stokes integrals for Maxwell's equations in Geometric Algebra}
%\chapter{Stokes integrals for Maxwell's equations in Geometric Algebra}
\section{Spatial domain}
%\label{chap:stokesMaxwellSpaceTimeSplit}
% \citep{sakurai2014modern} pr X.Y
% \citep{pozar2009microwave}
% \citep{qftLectureNotes}
% \citep{griffiths1999introduction}

Recall that the relativistic form of Maxwell's equation in Geometric Algebra is
%
\begin{equation}\label{eqn:stokesMaxwellSpaceTimeSplit:20}
\grad F = \inv{c \epsilon_0} J.
\end{equation}
%
where \( \grad = \gamma^\mu \partial_\mu \) is the spacetime gradient, and \( J = (c\rho, \BJ) = J^\mu \gamma_\mu \) is the four (vector) current density.  The pseudoscalar for the space is denoted \( I = \gamma_0 \gamma_1 \gamma_2 \gamma_3 \), where the basis elements satisfy \( \gamma_0^2 = 1 = -\gamma_k^2 \), and a dual basis satisfies \( \gamma_\mu \cdot \gamma^\nu = \delta_\mu^\nu \).  The electromagnetic field \( F \) is a composite multivector \( F = \BE + I c \BB \).  This is actually a bivector because spatial vectors have a bivector representation in the space time algebra of the form \( \BE = E^k \gamma_k \gamma_0 \).

Previously, I wrote out the Stokes integrals for Maxwell's equation in GA form using some three parameter spacetime manifold volumes.  This time I'm going to use two and three parameter spatial volumes, again with the Geometric Algebra form of Stokes theorem.

Multiplication by a timelike unit vector transforms Maxwell's equation from their relativistic form.  When that vector is the standard basis timelike unit vector \( \gamma_0 \), we obtain Maxwell's equations from the point of view of a stationary observer
%
\begin{equation}\label{eqn:stokesMaxwellSpaceTimeSplit:40}
\lr{\partial_0 + \spacegrad} \lr{ \BE + c I \BB } = \inv{\epsilon_0 c} \lr{ c \rho - \BJ },
\end{equation}
%
Extracting the scalar, vector, bivector, and trivector grades respectively, we have
\begin{equation}\label{eqn:stokesMaxwellSpaceTimeSplit:60}
\begin{aligned}
     \spacegrad \cdot \BE &= \frac{\rho}{\epsilon_0} \\
c I \spacegrad \wedge \BB &= -\partial_0 \BE - \inv{\epsilon_0 c} \BJ \\
    \spacegrad \wedge \BE &= - I c \partial_0 \BB \\
 c I \spacegrad \cdot \BB &= 0.
\end{aligned}
\end{equation}
%
Each of these can be written as a curl equation

%\begin{equation}\label{eqn:stokesMaxwellSpaceTimeSplit:80}
\boxedEquation{eqn:stokesMaxwellSpaceTimeSplit:80}{
\begin{aligned}
\spacegrad \wedge (I \BE) &= I \frac{\rho}{\epsilon_0} \\
\inv{\mu_0} \spacegrad \wedge \BB &= \epsilon_0 I \partial_t \BE + I \BJ \\
\spacegrad \wedge \BE &= -I \partial_t \BB \\
\spacegrad \wedge (I \BB) &= 0,
\end{aligned}
}
%\end{equation}

a form that allows for direct application of Stokes integrals.  The first and last of these require a three parameter volume element, whereas the two bivector grade equations can be integrated using either two or three parameter volume elements.  Suppose that we have can parameterize the space with parameters \( u, v, w \), for which the gradient has the representation
%
\begin{equation}\label{eqn:stokesMaxwellSpaceTimeSplit:100}
\spacegrad = \Bx^u \partial_u + \Bx^v \partial_v + \Bx^w \partial_w,
\end{equation}
%
but we integrate over a two parameter subset of this space spanned by \( \Bx(u,v) \), with area element
%
\begin{equation}\label{eqn:stokesMaxwellSpaceTimeSplit:120}
\begin{aligned}
d^2 \Bx
&= d\Bx_u \wedge d\Bx_v \\
&=
\PD{u}{\Bx}
\wedge
\PD{v}{\Bx}
\,du dv \\
&=
\Bx_u
\wedge
\Bx_v
\,du dv,
\end{aligned}
\end{equation}
%
as illustrated in \cref{fig:twoParameterAreaElement:twoParameterAreaElementFig1}.

\imageFigure{../figures/gabook/twoParameterAreaElementFig1}{Two parameter manifold.}{fig:twoParameterAreaElement:twoParameterAreaElementFig1}{0.3}

Our curvilinear coordinates \( \Bx_u, \Bx_v, \Bx_w \) are dual to the reciprocal basis \( \Bx^u, \Bx^v, \Bx^w \), but we won't actually have to calculate that reciprocal basis.  Instead we need only know that it can be calculated and is defined by the relations \( \Bx_a \cdot \Bx^b = \delta_a^b \).  Knowing that we can reduce (say),
%
\begin{equation}\label{eqn:stokesMaxwellSpaceTimeSplit:140}
\begin{aligned}
d^2 \Bx \cdot ( \spacegrad \wedge \BE )
&=
d^2 \Bx \cdot ( \Bx^a \partial_a \wedge \BE ) \\
&=
(\Bx_u \wedge \Bx_v) \cdot ( \Bx^a \wedge \partial_a \BE ) \,du dv \\
&=
(((\Bx_u \wedge \Bx_v) \cdot  \Bx^a) \cdot \partial_a \BE  \,du dv \\
&=
d\Bx_u \cdot \partial_v \BE \,dv
-d\Bx_v \cdot \partial_u \BE \,du,
\end{aligned}
\end{equation}
%
Because each of the differentials, for example \( d\Bx_u = (\PDi{u}{\Bx}) du \), is calculated with the other (i.e.\( v \)) held constant, this is directly integrable, leaving
%
\begin{equation}\label{eqn:stokesMaxwellSpaceTimeSplit:160}
\begin{aligned}
\int d^2 \Bx \cdot ( \spacegrad \wedge \BE )
&=
\int \evalrange{\lr{d\Bx_u \cdot \BE}}{v=0}{v=1}
-\int \evalrange{\lr{d\Bx_v \cdot \BE}}{u=0}{u=1} \\
&=
\ointclockwise d\Bx \cdot \BE.
\end{aligned}
\end{equation}
%
That direct integration of one of the parameters, while the others are held constant, is the basic idea behind Stokes theorem.

The pseudoscalar grade Maxwell's equations from \cref{eqn:stokesMaxwellSpaceTimeSplit:80} require a three parameter volume element to apply Stokes theorem to.  Again, allowing for curvilinear coordinates such a differential expands as
%
\begin{equation}\label{eqn:stokesMaxwellSpaceTimeSplit:180}
\begin{aligned}
d^3 \Bx \cdot (\spacegrad \wedge (I\BB))
&=
(( \Bx_u \wedge \Bx_v \wedge \Bx_w ) \cdot \Bx^a ) \cdot \partial_a (I\BB) \,du dv dw \\
&=
 (d\Bx_u \wedge d\Bx_v) \cdot \partial_w (I\BB) dw
+(d\Bx_v \wedge d\Bx_w) \cdot \partial_u (I\BB) du
+(d\Bx_w \wedge d\Bx_u) \cdot \partial_v (I\BB) dv.
\end{aligned}
\end{equation}
%
Like the two parameter volume, this is directly integrable
%
\begin{equation}\label{eqn:stokesMaxwellSpaceTimeSplit:200}
\int
d^3 \Bx \cdot (\spacegrad \wedge (I\BB))
=
 \int \evalbar{(d\Bx_u \wedge d\Bx_v) \cdot (I\BB) }{\Delta w}
+\int \evalbar{(d\Bx_v \wedge d\Bx_w) \cdot (I\BB)}{\Delta u}
+\int \evalbar{(d\Bx_w \wedge d\Bx_u) \cdot (I\BB)}{\Delta v}.
\end{equation}
%
After some thought (or a craft project such as that of \cref{fig:threeParameterSurface:threeParameterSurfaceFig2}) is can be observed that this is conceptually an oriented surface integral

\imageFigure{../figures/gabook/threeParameterSurfaceFig2}{Oriented three parameter surface.}{fig:threeParameterSurface:threeParameterSurfaceFig2}{0.3}

Noting that \( d^2 \Bx \cdot (I\Bf) = \gpgradezero{ d^2 \Bx I B } = I (d^2\Bx \wedge \Bf) \), we can now write down the results of application of Stokes theorem to each of Maxwell's equations in their curl forms

%\begin{equation}\label{eqn:stokesMaxwellSpaceTimeSplit:220}
\boxedEquation{eqn:stokesMaxwellSpaceTimeSplit:220}{
\begin{aligned}
\ointclockwise d\Bx \cdot \BE &= -I \partial_t \int d^2 \Bx \wedge \BB \\
\inv{\mu_0} \ointclockwise d\Bx \cdot \BB &= \epsilon_0 I \partial_t \int d^2 \Bx \wedge \BE + I \int d^2 \Bx \wedge \BJ \\
\oint d^2 \Bx \wedge \BE &= \inv{\epsilon_0} \int (d^3 \Bx \cdot I) \rho \\
\oint d^2 \Bx \wedge \BB &= 0.
\end{aligned}
}
%\end{equation}
%
In the three parameter surface integrals the specific meaning to apply to \( d^2 \Bx \wedge \Bf \) is
\begin{equation}\label{eqn:stokesMaxwellSpaceTimeSplit:240}
\begin{aligned}
\oint d^2 \Bx \wedge \Bf
&=
 \int \evalbar{\lr{d\Bx_u \wedge d\Bx_v \wedge \Bf}}{\Delta w} \\
&+\int \evalbar{\lr{d\Bx_v \wedge d\Bx_w \wedge \Bf}}{\Delta u} \\
&+\int \evalbar{\lr{d\Bx_w \wedge d\Bx_u \wedge \Bf}}{\Delta v}.
\end{aligned}
\end{equation}
%
Note that in each case only the component of the vector \( \Bf \) that is projected onto the normal to the area element contributes.

%}
%\EndNoBibArticle

   %
% Copyright � 2016 Peeter Joot.  All Rights Reserved.
% Licenced as described in the file LICENSE under the root directory of this GIT repository.
%
%{
%\input{../blogpost.tex}
%\renewcommand{\basename}{maxwellBoundaryConditions}
%\renewcommand{\dirname}{notes/phy1520/}
%%\newcommand{\dateintitle}{}
%%\newcommand{\keywords}{}
%
%\input{../peeter_prologue_print2.tex}
%
%\usepackage{peeters_layout_exercise}
%\usepackage{peeters_braket}
%\usepackage{peeters_figures}
%\usepackage{siunitx}
%
%\beginArtNoToc
%
%\generatetitle{Maxwell equation boundary conditions}
\mychapter{Maxwell equation boundary conditions}
\section{Free space}
%\label{chap:maxwellBoundaryConditions}
% \citep{sakurai2014modern} pr X.Y
% \citep{pozar2009microwave}
% \citep{qftLectureNotes}
% \citep{griffiths1999introduction}

\paragraph{Motivation}

Most electrodynamics textbooks either start with or contain a treatment of boundary value conditions.  These typically involve evaluating Maxwell's equations over areas or volumes of decreasing height, such as those illustrated in \cref{fig:boundaryConditionsPillBox:boundaryConditionsPillBoxFig2}, and \cref{fig:boundaryConditionsTwoSurfaces:boundaryConditionsTwoSurfacesFig1}.  These represent surfaces and volumes where the height is allowed to decrease to infinitesimal levels, and are traditionally used to find the boundary value constraints of the normal and tangential components of the electric and magnetic fields.

\imageFigure{../figures/gabook/boundaryConditionsPillBoxFig2}{Two surfaces normal to the interface.}{fig:boundaryConditionsPillBox:boundaryConditionsPillBoxFig2}{0.3}
\imageFigure{../figures/gabook/boundaryConditionsTwoSurfacesFig1}{A pillbox volume encompassing the interface.}{fig:boundaryConditionsTwoSurfaces:boundaryConditionsTwoSurfacesFig1}{0.3}

More advanced topics, such as evaluation of the Fresnel reflection and transmission equations, also rely on similar consideration of boundary value constraints.  I've wondered for a long time how the Fresnel equations could be attacked by looking at the boundary conditions for the combined field \( F = \BE + I c \BB \), instead of the considering them separately.

\paragraph{A unified approach.}

The Geometric Algebra (and relativistic tensor) formulations of Maxwell's equations put the electric and magnetic fields on equal footings.  It is in fact possible to specify the boundary value constraints on the fields without first separating Maxwell's equations into their traditional forms.  The starting point in Geometric Algebra is Maxwell's equation, premultiplied by a stationary observer's timelike basis vector

\begin{equation}\label{eqn:maxwellBoundaryConditions:20}
\gamma_0 \grad F = \inv{\epsilon_0 c} \gamma_0 J,
\end{equation}

or

\begin{equation}\label{eqn:maxwellBoundaryConditions:40}
\lr{ \partial_0 + \spacegrad} F = \frac{\rho}{\epsilon_0} - \frac{\BJ}{\epsilon_0}.
\end{equation}

The electrodynamic field \( F = \BE + I c \BB \) is a multivector in this spatial domain (whereas it is a bivector in the spacetime algebra domain), and has vector and bivector components.  The product of the spatial gradient and the field can still be split into dot and curl components \( \spacegrad M = \spacegrad \cdot M + \spacegrad \wedge M \).  If \( M = \sum M_i \), where \( M_i \) is an grade \( i \) blade, then we give this the Hestenes' \citep{hestenes1999nfc} definitions

\begin{equation}\label{eqn:maxwellBoundaryConditions:60}
\begin{aligned}
\spacegrad \cdot M &= \sum_i \gpgrade{\spacegrad M_i}{i-1} \\
\spacegrad \wedge M &= \sum_i \gpgrade{\spacegrad M_i}{i+1}.
\end{aligned}
\end{equation}

With that said, Maxwell's equation can be rearranged into a pair of multivector equations

\begin{equation}\label{eqn:maxwellBoundaryConditions:80}
\begin{aligned}
\spacegrad \cdot F &= \gpgrade{-\partial_0 F + \frac{\rho}{\epsilon_0} - \frac{\BJ}{\epsilon_0 c}}{0,1} \\
\spacegrad \wedge F &= \gpgrade{-\partial_0 F + \frac{\rho}{\epsilon_0} - \frac{\BJ}{\epsilon_0 c}}{2,3},
\end{aligned}
\end{equation}

The former equation can be integrated with Stokes theorem, but we need to apply a duality transformation to the latter in order to apply Stokes to it

\begin{equation}\label{eqn:maxwellBoundaryConditions:120}
\begin{aligned}
\spacegrad \cdot F
&=
-I^2 \spacegrad \cdot F \\
&=
-I^2 \gpgrade{\spacegrad F}{0,1} \\
&=
-I \gpgrade{I \spacegrad F}{2,3} \\
&=
-I \spacegrad \wedge (IF),
\end{aligned}
\end{equation}

so

\begin{equation}\label{eqn:maxwellBoundaryConditions:100}
\begin{aligned}
\spacegrad \wedge (I F) &= I \lr{ -\inv{c} \partial_t \BE + \frac{\rho}{\epsilon_0} - \frac{\BJ}{\epsilon_0 c} } \\
\spacegrad \wedge F &= -I \partial_t \BB.
\end{aligned}
\end{equation}

Integrating each of these over the pillbox volume gives

\begin{equation}\label{eqn:maxwellBoundaryConditions:140}
\begin{aligned}
\oint_{\partial V} d^2 \Bx \cdot (I F)
&=
\int_{V} d^3 \Bx \cdot \lr{ I \lr{ -\inv{c} \partial_t \BE + \frac{\rho}{\epsilon_0} - \frac{\BJ}{\epsilon_0 c} } } \\
\oint_{\partial V} d^2 \Bx \cdot F
&=
- \partial_t \int_{V} d^3 \Bx \cdot \lr{ I \BB }.
\end{aligned}
\end{equation}

In the absence of charges and currents on the surface, and if the height of the volume is reduced to zero, the volume integrals vanish, and only the upper surfaces of the pillbox contribute to the surface integrals.

\begin{equation}\label{eqn:maxwellBoundaryConditions:200}
\begin{aligned}
\oint_{\partial V} d^2 \Bx \cdot (I F) &= 0 \\
\oint_{\partial V} d^2 \Bx \cdot F &= 0.
\end{aligned}
\end{equation}

With a multivector \( F \) in the mix, the geometric meaning of these integrals is not terribly clear.  They do describe the boundary conditions, but to see exactly what those are, we can now resort to the split of \( F \) into its electric and magnetic fields.  Let's look at the non-dual integral to start with

\begin{equation}\label{eqn:maxwellBoundaryConditions:160}
\begin{aligned}
\oint_{\partial V} d^2 \Bx \cdot F
&=
\oint_{\partial V} d^2 \Bx \cdot \lr{ \BE + I c \BB } \\
&=
\oint_{\partial V} d^2 \Bx \cdot \BE + I c d^2 \Bx \wedge \BB \\
&=
0.
\end{aligned}
\end{equation}

No component of \( \BE \) that is normal to the surface contributes to \( d^2 \Bx \cdot \BE \), whereas only components of \(\BB \) that are normal contribute to \( d^2 \Bx \wedge \BB \).  That means that we must have tangential components of \( \BE \) and the normal components of \( \BB \) matching on the surfaces

\begin{equation}\label{eqn:maxwellBoundaryConditions:180}
\begin{aligned}
\lr{\BE_2 \wedge \ncap} \ncap - \lr{\BE_1 \wedge (-\ncap)} (-\ncap) &= 0 \\
\lr{\BB_2 \cdot \ncap} \ncap - \lr{\BB_1 \cdot (-\ncap)} (-\ncap) &= 0 .
\end{aligned}
\end{equation}

Similarly, for the dot product of the dual field, this is

\begin{equation}\label{eqn:maxwellBoundaryConditions:220}
\begin{aligned}
\oint_{\partial V} d^2 \Bx \cdot (I F)
&=
\oint_{\partial V} d^2 \Bx \cdot (I \BE - c \BB) \\
&=
\oint_{\partial V} I d^2 \Bx \wedge \BE - c d^2 \Bx \cdot \BB.
\end{aligned}
\end{equation}

For this integral, only the normal components of \( \BE \) contribute, and only the tangential components of \( \BB \) contribute.  This means that

\begin{equation}\label{eqn:maxwellBoundaryConditions:240}
\begin{aligned}
\lr{\BE_2 \cdot \ncap} \ncap - \lr{\BE_1 \cdot (-\ncap)} (-\ncap) &= 0 \\
\lr{\BB_2 \wedge \ncap} \ncap - \lr{\BB_1 \wedge (-\ncap)} (-\ncap) &= 0.
\end{aligned}
\end{equation}

This is why we end up with a seemingly strange mix of tangential and normal components of the electric and magnetic fields.  These constraints can be summarized as

\begin{equation}\label{eqn:maxwellBoundaryConditions:260}
\begin{aligned}
( \BE_2 - \BE_1 ) \cdot \ncap &= 0 \\
( \BE_2 - \BE_1 ) \wedge \ncap &= 0 \\
( \BB_2 - \BB_1 ) \cdot \ncap &= 0 \\
( \BB_2 - \BB_1 ) \wedge \ncap &= 0
\end{aligned}
\end{equation}

These relationships are usually expressed in terms of all of \( \BE, \BD, \BB \) and \( \BH \).  Because I'd started with Maxwell's equations for free space, I don't have the \( \epsilon \) and \( \mu \) factors that produce those more general relationships.  Those more general boundary value relationships are usually the starting point for the Fresnel interface analysis.  It is also possible to further generalize these relationships to include charges and currents on the surface.

%}
%\EndArticle

   %
% Copyright � 2016 Peeter Joot.  All Rights Reserved.
% Licenced as described in the file LICENSE under the root directory of this GIT repository.
%
%{
%\input{../blogpost.tex}
%\renewcommand{\basename}{boundaryConditionsInMedia}
%\renewcommand{\dirname}{notes/phy1520/}
%%\newcommand{\dateintitle}{}
%%\newcommand{\keywords}{}
%
%\input{../peeter_prologue_print2.tex}
%
%\usepackage{peeters_layout_exercise}
%\usepackage{peeters_braket}
%\usepackage{peeters_figures}
%\usepackage{siunitx}
%
%\beginArtNoToc
%
%\generatetitle{Maxwell equation boundary conditions in media}
\section{Maxwell equation boundary conditions in media}
%\label{chap:boundaryConditionsInMedia}

Following \citep{balanis1989advanced}, Maxwell's equations in media, including both electric and magnetic sources and currents are

\begin{subequations}
\label{eqn:boundaryConditionsInMedia:20}
\begin{equation}\label{eqn:boundaryConditionsInMedia:40}
\spacegrad \cross \BE = -\BM - \partial_t \BB
\end{equation}
\begin{equation}\label{eqn:boundaryConditionsInMedia:60}
\spacegrad \cross \BH = \BJ + \partial_t \BD
\end{equation}
\begin{equation}\label{eqn:boundaryConditionsInMedia:80}
\spacegrad \cdot \BD = \rho
\end{equation}
\begin{equation}\label{eqn:boundaryConditionsInMedia:100}
\spacegrad \cdot \BB = \rho_\txtm
\end{equation}
\end{subequations}
%
In general, it is not possible to assemble these into a single Geometric Algebra equation unless specific assumptions about the permeabilities are made, but we can still use Geometric Algebra to examine the boundary condition question.  First, these equations can be expressed in a more natural multivector form

\begin{subequations}
\label{eqn:boundaryConditionsInMedia:120}
\begin{equation}\label{eqn:boundaryConditionsInMedia:140}
\spacegrad \wedge \BE = -I \lr{ \BM + \partial_t \BB }
\end{equation}
\begin{equation}\label{eqn:boundaryConditionsInMedia:160}
\spacegrad \wedge \BH = I \lr{ \BJ + \partial_t \BD }
\end{equation}
\begin{equation}\label{eqn:boundaryConditionsInMedia:180}
\spacegrad \cdot \BD = \rho
\end{equation}
\begin{equation}\label{eqn:boundaryConditionsInMedia:200}
\spacegrad \cdot \BB = \rho_\txtm
\end{equation}
\end{subequations}
%
Then duality relations can be used on the divergences to write all four equations in their curl form

\begin{subequations}
\label{eqn:boundaryConditionsInMedia:220}
\begin{equation}\label{eqn:boundaryConditionsInMedia:240}
\spacegrad \wedge \BE = -I \lr{ \BM + \partial_t \BB }
\end{equation}
\begin{equation}\label{eqn:boundaryConditionsInMedia:260}
\spacegrad \wedge \BH = I \lr{ \BJ + \partial_t \BD }
\end{equation}
\begin{equation}\label{eqn:boundaryConditionsInMedia:280}
\spacegrad \wedge (I\BD) = \rho I
\end{equation}
\begin{equation}\label{eqn:boundaryConditionsInMedia:300}
\spacegrad \wedge (I\BB) = \rho_\txtm I.
\end{equation}
\end{subequations}
%
Now it is possible to employ Stokes theorem to each of these.  The usual procedure is to both use the loops of
\cref{fig:boundaryConditionsPillBox:boundaryConditionsPillBoxFig2}
and the pillbox of
\cref{fig:boundaryConditionsTwoSurfaces:boundaryConditionsTwoSurfacesFig1}
, where in both cases the height is made infinitesimal.

% FIXME:remove all but the references when merging this into the gabookII.
%\imageFigure{../figures/gabook/boundaryConditionsPillBoxFig2}{Two surfaces normal to the interface.}{fig:boundaryConditionsPillBox:boundaryConditionsPillBoxFig2}{0.2}
%\imageFigure{../figures/gabook/boundaryConditionsTwoSurfacesFig1}{A pillbox volume encompassing the interface.}{fig:boundaryConditionsTwoSurfaces:boundaryConditionsTwoSurfacesFig1}{0.2}

With all these relations expressed in curl form as above, we can use just the pillbox configuration to evaluate the Stokes integrals.
Let the height \( h \) be measured along the normal axis, and assume that all the charges and currents are localized to the surface
%
\begin{equation}\label{eqn:boundaryConditionsInMedia:320}
\begin{aligned}
\BM &= \BM_\txts \delta( h ) \\
\BJ &= \BJ_\txts \delta( h ) \\
\rho &= \rho_\txts \delta( h ) \\
\rho_\txtm &= \rho_{\txtm\txts} \delta( h ),
\end{aligned}
\end{equation}
%
where \( \ncap \wedge \BM_\txts = \ncap \wedge \BJ_\txts = 0 \).
We can enumerate the Stokes integrals \( \int d^3 \Bx \cdot \lr{ \spacegrad \wedge \BX } = \oint_{\partial V} d^2 \Bx \cdot \BX \).  The three-volume area element will be written as \( d^3 \Bx = d^2 \Bx \wedge \ncap dh \), giving

\begin{subequations}
\label{eqn:boundaryConditionsInMedia:340}
\begin{equation}\label{eqn:boundaryConditionsInMedia:360}
\oint_{\partial V} d^2 \Bx \cdot \BE = -\int (d^2 \Bx \wedge \ncap) \cdot \lr{ I \BM_\txts + \partial_t I \BB \Delta h}
\end{equation}
\begin{equation}\label{eqn:boundaryConditionsInMedia:380}
\oint_{\partial V} d^2 \Bx \cdot \BH = \int (d^2 \Bx \wedge \ncap) \cdot \lr{ I \BJ_\txts + \partial_t I \BD \Delta h}
\end{equation}
\begin{equation}\label{eqn:boundaryConditionsInMedia:400}
\oint_{\partial V} d^2 \Bx \cdot (I\BD) = \int (d^2 \Bx \wedge \ncap) \cdot \lr{ \rho_\txts I }
\end{equation}
\begin{equation}\label{eqn:boundaryConditionsInMedia:420}
\oint_{\partial V} d^2 \Bx \cdot (I\BB) = \int (d^2 \Bx \wedge \ncap) \cdot \lr{ \rho_{\txtm\txts} I }
\end{equation}
\end{subequations}
%
In the limit with \( \Delta h \rightarrow 0 \), the LHS integrals are reduced to just the top and bottom surfaces, and the \( \Delta h \) contributions on the RHS are eliminated.  With \( i = I \ncap \), and \( d^2 \Bx = dA\, i \) on the top surface, we are left with

\begin{subequations}
\label{eqn:boundaryConditionsInMedia:440}
\begin{equation}\label{eqn:boundaryConditionsInMedia:460}
   0 = \int dA \lr{ i \cdot \Delta \BE + I \cdot \lr{ I \BM_\txts } }
\end{equation}
\begin{equation}\label{eqn:boundaryConditionsInMedia:480}
0 = \int dA \lr{ i \cdot \Delta \BH - I \cdot \lr{ I \BJ_\txts } }
\end{equation}
\begin{equation}\label{eqn:boundaryConditionsInMedia:500}
0 = \int dA \lr{ i \cdot \Delta (I\BD) + \rho_\txts }
\end{equation}
\begin{equation}\label{eqn:boundaryConditionsInMedia:520}
0 = \int dA \lr{ i \cdot \Delta (I\BB) + \rho_{\txtm\txts} }
\end{equation}
\end{subequations}
%
Consider the first integral.  Any component of \( \BE \) that is normal to the plane of the pillbox top (or bottom) has no contribution to the integral, so this constraint is one that effects only the tangential components \( \ncap (\ncap \wedge (\Delta \BE)) \).  Writing out the vector portion of the integrand, we have
%
\begin{equation}\label{eqn:boundaryConditionsInMedia:540}
\begin{aligned}
i \cdot \Delta \BE + I \cdot \lr{ I \BM_\txts }
&=
\gpgradeone{ i \Delta \BE + I^2 \BM_\txts } \\
&=
\gpgradeone{ I \ncap \Delta \BE - \BM_\txts } \\
&=
\gpgradeone{ I \ncap \ncap (\ncap \wedge \Delta \BE) - \BM_\txts } \\
&=
\gpgradeone{ I (\ncap \wedge (\Delta \BE)) - \BM_\txts } \\
&=
\gpgradeone{ -\ncap \cross (\Delta \BE) - \BM_\txts }.
\end{aligned}
\end{equation}
%
The dot product (a scalar) in the two surface charge integrals can also be reduced
%
\begin{equation}\label{eqn:boundaryConditionsInMedia:560}
\begin{aligned}
i \cdot \Delta (I\BD)
&=
\gpgradezero{ i \Delta (I\BD) } \\
&=
\gpgradezero{ I \ncap \Delta (I\BD) } \\
&=
\gpgradezero{ -\ncap \Delta \BD } \\
&=
-\ncap \cdot \Delta \BD,
\end{aligned}
\end{equation}
%
so the integral equations are satisfied provided

%\begin{equation}\label{eqn:boundaryConditionsInMedia:580}
\boxedEquation{eqn:boundaryConditionsInMedia:580}{
\begin{aligned}
\ncap \cross (\BE_2 - \BE_1) &= - \BM_\txts \\
\ncap \cross (\BH_2 - \BH_1) &= \BJ_\txts \\
\ncap \cdot (\BD_2 - \BD_1) &= \rho_\txts \\
\ncap \cdot (\BB_2 - \BB_1) &= \rho_{\txtm\txts}.
\end{aligned}
}
%\end{equation}

The cross products may also be written in a (perhaps more natural) dual form

%\begin{equation}\label{eqn:boundaryConditionsInMedia:581}
\boxedEquation{eqn:boundaryConditionsInMedia:581}{
\begin{aligned}
\ncap \wedge (\BE_2 - \BE_1) &= - I \BM_\txts \\
\ncap \wedge (\BH_2 - \BH_1) &= I \BJ_\txts \\
\ncap \cdot (\BD_2 - \BD_1) &= \rho_\txts \\
\ncap \cdot (\BB_2 - \BB_1) &= \rho_{\txtm\txts}.
\end{aligned}
}
%\end{equation}

It is tempting to try to assemble these into a results expressed in terms of a four-vector surface current and composite STA bivector fields like the \( F = \BE + I c \BB \) that we can use for the free space Maxwell's equation.  Dimensionally, we need something with velocity in that mix, but what velocity should be used when the speed of the field propagation in each media is potentially different?

%}
%\EndArticle

      \section{Problems}
         %
% Copyright � 2016 Peeter Joot.  All Rights Reserved.
% Licenced as described in the file LICENSE under the root directory of this GIT repository.
%
%{
%\input{../blogpost.tex}
%\renewcommand{\basename}{fieldsAtInterface}
%\renewcommand{\dirname}{notes/ece1228-electromagnetic-theory/}
%%\newcommand{\dateintitle}{}
%%\newcommand{\keywords}{}
%
%\input{../peeter_prologue_print2.tex}
%
%\usepackage{peeters_layout_exercise}
%\usepackage{peeters_braket}
%\usepackage{peeters_figures}
%\usepackage{siunitx}
%
%\beginArtNoToc
%
%\generatetitle{Electric and magnetic fields at an interface}
%\chapter{Electric and magnetic fields at an interface}
%\label{chap:fieldsAtInterface}

\makeproblem{Fields accross dielectric boundary.}{problem:fieldsAtInterface:1}{
Given a plane \( \ncap \cdot \Bx = a \) describing the interface between two dielectric mediums, and the fields on one side of the plane \( \BE_1, \BB_1 \), what are the fields on the other side of the interface?
} % makeproblem

\makeanswer{problem:fieldsAtInterface:1}{
Assuming that neither interface is perfectly conductoring,
\cref{eqn:boundaryConditionsInMedia:581} can be rearranged in terms of the products of the fields in interface 2
%As pointed out in \citep{balanis1989advanced} the fields at an interface that is not a perfect conductor on either side are related by

\begin{equation}\label{eqn:fieldsAtInterface:20}
\begin{aligned}
\ncap \cdot \lr{ \BD_2 - \BD_1 } &= \rho_{es} \\
\ncap \cross \lr{ \BE_2 - \BE_1 } &= -\BM_s \\
\ncap \cdot \lr{ \BB_2 - \BB_1 } &= \rho_{ms} \\
\ncap \cross \lr{ \BH_2 - \BH_1 } &= \BJ_s.
\end{aligned}
\end{equation}

Given the fields in medium 1, assuming that boths sets of media are linear, we can use these relationships to determine the fields in the other medium.

\begin{equation}\label{eqn:fieldsAtInterface:40}
\begin{aligned}
\ncap \cdot \BE_2 &= \inv{\epsilon_2} \lr{ \epsilon_1 \ncap \cdot \BE_1 + \rho_{es} } \\
\ncap \wedge \BE_2 &= \ncap \wedge \BE_1 -I \BM_s \\
\ncap \cdot \BB_2 &= \ncap \cdot \BB_1 + \rho_{ms} \\
\ncap \wedge \BB_2 &= \mu_2 \lr{ \inv{\mu_1} \ncap \wedge \BB_1 + I \BJ_s}.
\end{aligned}
\end{equation}

Now the fields in interface 2 can be obtained by adding the normal and tangential projections.  For the electric field

\begin{equation}\label{eqn:fieldsAtInterface:60}
\begin{aligned}
\BE_2
&=
\ncap (\ncap \cdot \BE_2 )
+ \ncap \cdot (\ncap \wedge \BE_2) \\
&=
\inv{\epsilon_2} \ncap \lr{ \epsilon_1 \ncap \cdot \BE_1 + \rho_{es} }
+
\ncap \cdot (\ncap \wedge \BE_1 -I \BM_s).
\end{aligned}
\end{equation}

%Note that this manipulation can also be done without Geometric Algebra by writing \( \BE_2 = \ncap (\ncap \cdot \BE_2 ) - \ncap \cross (\ncap \cross \BE_2) \).
Expanding \( \ncap \cdot (\ncap \wedge \BE_1) = \BE_1 - \ncap (\ncap \cdot \BE_1) \), and \( \ncap \cdot (I \BM_s) = -\ncap \cross \BM_s \), that is

%\begin{dmath}\label{eqn:fieldsAtInterface:80}
\boxedEquation{eqn:fieldsAtInterface:80}{
\BE_2
=
\BE_1
+ \ncap (\ncap \cdot \BE_1) \lr{ \frac{\epsilon_1}{\epsilon_2} - 1 }
+ \frac{\rho_{es}}{\epsilon_2}
+ \ncap \cross \BM_s.
}
%\end{dmath}

For the magnetic field

\begin{equation}\label{eqn:fieldsAtInterface:100}
\begin{aligned}
\BB_2
&=
\ncap (\ncap \cdot \BB_2 )
+
\ncap \cdot (\ncap \wedge \BB_2) \\
&=
\ncap \lr{ \ncap \cdot \BB_1 + \rho_{ms} }
+
\mu_2 \ncap \cdot \lr{ \lr{ \inv{\mu_1} \ncap \wedge \BB_1 + I \BJ_s} },
\end{aligned}
\end{equation}

which is

%\begin{dmath}\label{eqn:fieldsAtInterface:120}
\boxedEquation{eqn:fieldsAtInterface:140}{
\BB_2
=
\frac{\mu_2}{\mu_1} \BB_1
+
\ncap (\ncap \cdot \BB_1) \lr{ 1 - \frac{\mu_2}{\mu_1} }
+ \ncap \rho_{ms}
- \ncap \cross \BJ_s.
}
%\end{dmath}

These are kind of pretty results, having none of the explicit angle dependence that we see in the Fresnel relationships.  In this analysis, it is assumed there is only a transmitted component of the ray in question, and no reflected component.  Can we do a purely vectoral treatment of the Fresnel equations along these same lines?

} % answer
%}

%\EndArticle


   %
% Copyright � 2012 Peeter Joot.  All Rights Reserved.
% Licenced as described in the file LICENSE under the root directory of this GIT repository.
%

%
%
\chapter{Tensor relations from bivector field equation}\label{chap:PJMaxwellTensor}
\index{Maxwell's equations!tensor}
%\date{Sept 7, 2008.  maxwellToTensor.tex}

\section{Motivation}

This contains a somewhat unstructured collection of notes translating between tensor and bivector forms of Maxwell's equation(s).

\section{Electrodynamic tensor}

John Denker's paper \citep{DenkerMaxwell} writes:

\begin{equation}
F = (\BE + ic\BB) \gamma_0,
\end{equation}

with
\begin{equation}\label{eqn:maxwellToTensor:20}
\begin{aligned}
\BE &= E^i \gamma_i \\
\BB &= B^i \gamma_i.
\end{aligned}
\end{equation}

Since he uses the positive end of the metric for spatial indices this works fine.  Contrast to \citep{doran2003gap} who write:

\begin{equation}
F = \BE + ic\BB,
\end{equation}

with the following implied spatial bivector representation:
\begin{equation}\label{eqn:maxwellToTensor:40}
\begin{aligned}
\BE &= E^i \sigma_i = E^i \gamma_{i0} \\
\BB &= B^i \sigma_i = B^i \gamma_{i0}.
\end{aligned}
\end{equation}

That implied representation was not obvious to me, but I eventually figured out what they meant.  They also use \(c=1\), so I have added it back in here for clarity.

The end result in both cases is a pure bivector representation for the complete field:

\begin{equation*}
F = E^j \gamma_{j0} + ic B^j \gamma_{j0}.
\end{equation*}

%ASIDE: Is bivector the term used for a completely grade two multivector, but not neccessarily a wedge product?
%This field multivector is not definitely not a blade a term reserved for something that can be created by wedging (simple element in grassman algebra terms).
%Here the field multivector cannot be expressed as the wedge product of two vectors unless one of the electric or magnetic fields is entirely zero (essentially
%reducing the dimension of the spanning basis to a 3D bivector).

Let us look at the \(B^j\) basis bivectors a bit more closely:

\begin{equation*}
i\gamma_{j0}
= \gamma_{0123j0}
= -\gamma_{01230j}
= +\gamma_{00123j}
= (\gamma_0)^2 \gamma_{123j}.
\end{equation*}

Where,
\begin{equation*}
\gamma_{123j} =
\left\{
\begin{array}{l l}
(\gamma_{j})^2 \gamma_{23} & \quad \mbox{if \(j = 1\)} \\
(\gamma_{j})^2 \gamma_{31} & \quad \mbox{if \(j = 2\)} \\
(\gamma_{j})^2 \gamma_{12} & \quad \mbox{if \(j = 3\)}.
\end{array} \right.
\end{equation*}

Combining these results we have a \((\gamma_0)^2 (\gamma_{j})^2 = -1\) coefficient that is metric invariant, and can write:

\begin{equation*}
i \sigma_{j} =
i \gamma_{j0} =
\left\{
\begin{array}{l l}
\gamma_{32} & \quad \mbox{if \(j = 1\)} \\
\gamma_{13} & \quad \mbox{if \(j = 2\)} \\
\gamma_{21} & \quad \mbox{if \(j = 3\)}.
\end{array} \right.
\end{equation*}

% -1:23
% -2:31
% -3:12

Or, more compactly:

\begin{equation*}
i \sigma_{a} =
i \gamma_{a0} =
-\epsilon_{abc} \gamma_{bc}.
\end{equation*}

Putting things back together, our bivector field in index notation is:

\begin{equation}\label{eqn:maxToTensor:Fcomp}
F = E^i \gamma_{i 0} - \epsilon_{i j k} c B^i \gamma_{j k}.
\end{equation}

\subsection{Tensor components}

Now, given a grade two multivector such as our field, how can we in general compute the components of that field given any arbitrary basis.  This can be done using the reciprocal bivector frame:

\begin{equation*}
F = \sum a_{{\mu} {\nu}} (e_{\mu} \wedge e_{\nu}).
\end{equation*}

To calculate the coordinates \(a_{{\mu} {\nu}}\) we can dot with
\(e^{\nu} \wedge e^{\mu}\):

\begin{equation}\label{eqn:maxwellToTensor:60}
\begin{aligned}
F \cdot (e^{\nu} \wedge e^{\mu})
&= \sum a_{{\alpha} {\beta}} (e_{\alpha} \wedge e_{\beta}) \cdot (e^{\nu} \wedge e^{\mu}) \\
&= ( a_{{\mu} {\nu}} (e_{\mu} \wedge e_{\nu}) + a_{{\nu} {\mu}} (e_{\nu} \wedge e_{\mu}) ) \cdot (e^{\nu} \wedge e^{\mu}) \\
&= a_{{\mu} {\nu}} - a_{{\nu} {\mu}} \\
&= 2 a_{{\mu} {\nu}}.
\end{aligned}
\end{equation}

Therefore
\begin{equation*}
F = \inv{2} \sum (F \cdot (e^{\nu} \wedge e^{\mu})) (e_{\mu} \wedge e_{\nu}) = \sum_{{\mu}<{\nu}} (F \cdot (e^{\nu} \wedge e^{\mu})) (e_{\mu} \wedge e_{\nu}).
\end{equation*}

With \(F^{{\mu} {\nu}} = F \cdot (e^{\nu} \wedge e^{\mu})\) and summation convention:

\begin{equation}
F = \inv{2} F^{{\mu} {\nu}} (e_{\mu} \wedge e_{\nu}).
\end{equation}

It is not hard to see that the representation with respect to the reciprocal frame, with
\(F_{{\mu} {\nu}} = F \cdot (e_{\nu} \wedge e_{\mu})\) must be:

\begin{equation}
F = \inv{2} F_{{\mu} {\nu}} (e^{\mu} \wedge e^{\nu}).
\end{equation}

Writing \(F^{\mu\nu}\) or \(F_{\mu\nu}\) leaves a lot unspecified.  You will get a different tensor for each choice of basis.  Using this form amounts to the equivalent of using the matrix of a linear transformation with respect to a specified basis.

\subsection{Electromagnetic tensor components}

Next, let us calculate these
\(F_{{\mu} {\nu}}\), and \(F^{{\mu} {\nu}}\) values and relate them to our electric and magnetic fields so we can work in or translate to and from all of the traditional vector, the tensor, and the Clifford/geometric languages.

\begin{equation*}
F^{{\mu} {\nu}} = \left( E^i \gamma_{i 0} - \epsilon_{i j k} c B^i \gamma_{j k} \right) \cdot \gamma^{\nu\mu}.
\end{equation*}

By inspection our electric field components we have:

\begin{equation*}
F^{i0} = E^i,
\end{equation*}

and for the magnetic field:

\begin{equation*}
F^{{i} {j}} = - \epsilon_{k i j} c B^k = - \epsilon_{i j k} c B^k.
\end{equation*}

Putting in sample numbers this is:

\begin{equation}\label{eqn:maxwellToTensor:80}
\begin{aligned}
F^{{3} {2}} &= - \epsilon_{3 2 1} c B^1 = c B^1 \\
F^{{1} {3}} &= - \epsilon_{1 3 2} c B^2 = c B^2 \\
F^{{2} {1}} &= - \epsilon_{2 1 3} c B^3 = c B^3.
\end{aligned}
\end{equation}

This can be summarized in matrix form:

\begin{equation}\label{eqn:maxToTensor:matrixtensor}
F^{\mu\nu} =
\begin{bmatrix}
0   & -E^1 & -E^2 & -E^3 \\
E^1 &   0  & -c B^3 &  c B^2 \\
E^2 &  c B^3 &   0  & -c B^1 \\
E^3 & -c B^2 &  c B^1 &   0
\end{bmatrix}
.
\end{equation}

Observe that no specific reference to a metric was required to evaluate these components.

\subsection{reciprocal tensor (name?)}

The reciprocal frame representation of \eqnref{eqn:maxToTensor:Fcomp} is %metric dependent when expressed

\begin{equation}\label{eqn:maxwellToTensor:100}
\begin{aligned}
F
&= E^i \gamma_{i 0} - \epsilon_{i j k} c B^i \gamma_{j k} \\
&= -E^i \gamma^{i 0} - \epsilon_{i j k} c B^i \gamma^{j k}.
\end{aligned}
\end{equation}

%FIXME: WHAT AM I TALKING ABOUT HERE "is metric dependent"?

Calculation of the reciprocal representation of the field tensor \(F_{{\mu} {\nu}} = F \cdot \gamma_{\nu\mu}\) is now possible, and by inspection
%, regardless of the metric:

\begin{equation}\label{eqn:maxwellToTensor:120}
\begin{aligned}
F_{i0} &= -E^i = -F^{i0} \\
F_{ij} &= - \epsilon_{i j k} c B^k = F^{ij}.
\end{aligned}
\end{equation}

So, all the electric field components in the tensor have inverted sign:
\begin{equation*}
F_{\mu\nu} =
\begin{bmatrix}
0   & E^1 & E^2 & E^3 \\
-E^1 &   0  & -c B^3 &  c B^2 \\
-E^2 &  c B^3 &   0  & -c B^1 \\
-E^3 & -c B^2 &  c B^1 &   0  \\
\end{bmatrix}
.
\end{equation*}

This is metric independent with this bivector based definition of \(F_{\mu\nu}\), and \(F^{\mu\nu}\).  Surprising, since I thought I had read otherwise.

\subsection{Lagrangian density}

\citep{doran2003gap}
%Doran/Lasenby
write the Lagrangian density in terms of \(\gpgradezero{F^2}\), whereas Denker writes it in terms of \(\gpgradezero{F \tilde{F}}\).  Is their
alternate choice in metric responsible for this difference.

Reversing the field since it is a bivector, just inverts the sign:

\begin{equation}\label{eqn:maxwellToTensor:140}
\begin{aligned}
F &= E^i \gamma_{i 0} - \epsilon_{i j k} c B^i \gamma_{j k} \\
\tilde{F} &= E^i \gamma_{0 i} - \epsilon_{i j k} c B^i \gamma_{k j} = -F.
\end{aligned}
\end{equation}

So the choice of \(\gpgradezero{F^2}\) vs. \(\gpgradezero{F \tilde{F}}\) is just a sign choice, and does not have anything to do with the metric.

Let us evaluate one of these:

\begin{equation}\label{eqn:maxwellToTensor:160}
\begin{aligned}
F^2
&=
(E^i \gamma_{i 0} - \epsilon_{i j k} c B^i \gamma_{j k}) (E^u \gamma_{u 0} - \epsilon_{u v w} c B^u \gamma_{v w})  \\
&=
E^i E^u \gamma_{i 0} \gamma_{u 0}
- \epsilon_{u v w} E^i c B^u \gamma_{v w} \gamma_{i 0}
- \epsilon_{i j k} E^u c B^i \gamma_{j k} \gamma_{u 0}
+ \epsilon_{i j k} \epsilon_{u v w} c^2 B^i B^u \gamma_{v w} \gamma_{j k}.
\end{aligned}
\end{equation}

That first term is:

\begin{equation}\label{eqn:maxwellToTensor:180}
\begin{aligned}
E^i E^u \gamma_{i 0} \gamma_{u 0}
&= \BE^2 + \sum_{i \ne j} E^i E^j ( \sigma_i \sigma_j + \sigma_j \sigma_i ) \\
&= \BE^2 + \sum_{i \ne j} 2 E^i E^j \sigma_i \cdot \sigma_j \\
&= \BE^2.
\end{aligned}
\end{equation}

Hmm.  This is messy.  Let us try with \(F = \BE + i c \BB\) directly (with the Doran/Lasenby convention: \(\BE = E^k \sigma_k\)) :

\begin{equation}\label{eqn:maxwellToTensor:200}
\begin{aligned}
F^2
&= (\BE + i c \BB) (\BE + i c \BB) \\
&= \BE^2 + c^2 (i \BB) (i \BB) + c (i \BB \BE + \BE i \BB) \\
&= \BE^2 + c^2 (\BB i) (i \BB) + i c (\BB \BE + \BE \BB) \\
&= \BE^2 - c^2 \BB^2 + 2 i c (\BB \cdot \BE).
\end{aligned}
\end{equation}

\subsubsection{Compared to tensor form}

Now lets compare to the tensor form, where the Lagrangian density is written in terms of the product of upper and lower index tensors:

\begin{equation}\label{eqn:maxwellToTensor:220}
\begin{aligned}
F_{\mu\nu}F^{\mu\nu}
&= F_{i 0}F^{i 0} +F_{0 i}F^{0 i} +\sum_{i<j} F_{i j}F^{i j} +\sum_{j<i} F_{i j}F^{i j} \\
&= 2 F_{i 0}F^{i 0} + 2 \sum_{i<j} F_{i j}F^{i j} \\
&= 2 (-E^i)(E^i) + 2 \sum_{i<j} (F^{i j})^2 \\
&= -2 \BE^2 + 2 \sum_{i<j} ( -\epsilon_{ijk} c B^k )^2 \\
&= -2 ( \BE^2 - c^2 \BB^2 ).
\end{aligned}
\end{equation}

Summarizing with a comparison of the bivector and tensor forms we have:

\begin{equation}
\inv{2} F_{\mu\nu}F^{\mu\nu} = c^2 \BB^2 - \BE^2 = - \gpgradezero{F^2} = \gpgradezero{ F \tilde{F} }.
\end{equation}

But to put this in context we need to figure out how to apply this in the Lagrangian.  That appears to require a potential formulation of the field equations, so that is the next step.

\subsubsection{Potential and relation to electromagnetic tensor}

Since the field is a bivector is it reasonable to assume that it may be possible to express as the curl of a vector

\begin{equation*}
F = \grad \wedge A.
\end{equation*}

Inserting this into the field equation we have:
\begin{equation}\label{eqn:maxwellToTensor:240}
\begin{aligned}
\grad (\grad \wedge A)
&= \grad \cdot (\grad \wedge A) + \mathLabelBox{\grad \wedge \grad}{\(=0\)} \wedge A \\
&= \grad^2 A - \grad ( \grad \cdot A ) \\
&= \inv{\epsilon_0 c} J.
\end{aligned}
\end{equation}

With application of the gauge condition \(\grad \cdot A = 0\), one is left with the four scalar equations:

\begin{equation}\label{eqn:maxToTensor:potential}
\grad^2 A = \inv{\epsilon_0 c} J.
\end{equation}

This can also be seen more directly since the gauge condition implies:

\begin{equation*}
\grad \wedge A = \grad \wedge A + \grad \cdot A = \grad A,
\end{equation*}

from which \eqnref{eqn:maxToTensor:potential} follows directly.  Observe that although the field equation was not metric
dependent, the equivalent potential equation is since it has a squared Laplacian.

\subsubsection{Index raising or lowering}
Any raising or lowering of indices, whether it be in the partials or the basis vectors corresponds to a multiplication by a \((\gamma_{\alpha})^2 = \pm 1\) value, so doing this twice cancels out \((\pm 1)^2 = 1\).

Vector coordinates in the reciprocal basis is translated by such a squared
factor when we are using an orthonormal basis:

\begin{equation}\label{eqn:maxwellToTensor:260}
\begin{aligned}
x
&= \sum \gamma^{\mu} ( \gamma_{\mu} \cdot x ) \\
&= \sum \gamma^{\mu} x_{\mu} \\
&= \sum \gamma^{\mu} (\gamma^{\mu} \gamma_{\mu}) x_{\mu} \\
&= \sum (\gamma^{\mu})^2 \gamma_{\mu} x_{\mu},
\end{aligned}
\end{equation}

therefore

\begin{equation*}
x^{\mu} = x \cdot \gamma^{\mu} = (\gamma^{\mu})^2 x_{\mu}.
\end{equation*}

Similarly our partial derivatives can be raised or lowered since they are just derivatives in terms of one of the choices of coordinates

\begin{equation*}
\partial_{\mu} = \PD{x^{\mu}}{} = \PD{(\gamma_{\mu})^2 x_{\mu}}{} = (\gamma_{\mu})^2 \PD{x_{\mu}}{} = (\gamma_{\mu})^2 \partial^{\mu},
\end{equation*}

when written as a gradient, we have two pairs of \((\gamma_{\mu})^2\) factors that cancel if we switch both indices:

\begin{equation*}
\grad = \gamma^{\mu} \PD{x^{\mu}}{} = (\gamma_{\mu})^2 (\gamma_{\mu})^2 \gamma_{\mu} \PD{x_{\mu}}{} = (\pm 1)^2 \gamma_{\mu} \PD{x_{\mu}}{}.
\end{equation*}

Or in short with the notation above

\begin{equation*}
\grad = \gamma^{\mu} \partial_{\mu} = \gamma_{\mu} \partial^{\mu}.
\end{equation*}

\subsubsection{Back to tensor in terms of potential}
Utilizing matched raising and lowering of indices, our field can be written in any of the following ways
%This metric dependency also shows up if one calculates the em tensor in terms of potential.

\begin{equation}\label{eqn:maxwellToTensor:280}
\begin{aligned}
\grad \wedge A
&= {\gamma_{\mu}} \wedge \gamma_{\nu} \partial^{\mu} A^{\nu} = \sum_{\mu<\nu} {\gamma_{\mu}} \wedge \gamma_{\nu} \left( \partial^{\mu} A^{\nu} - \partial^{\nu} A^{\mu} \right) \\
&= {\gamma^{\mu}} \wedge \gamma^{\nu} \partial_{\mu} A_{\nu} = \sum_{\mu<\nu} {\gamma^{\mu}} \wedge \gamma^{\nu} \left( \partial_{\mu} A_{\nu} - \partial_{\nu} A_{\mu} \right) \\
&= {\gamma_{\mu}} \wedge \gamma^{\nu} \partial^{\mu} A_{\nu} = \sum_{\mu<\nu} {\gamma_{\mu}} \wedge \gamma^{\nu} \left( \partial^{\mu} A_{\nu} - \partial_{\nu} A^{\mu} \right) \\
&= {\gamma^{\mu}} \wedge \gamma_{\nu} \partial_{\mu} A^{\nu} = \sum_{\mu<\nu} {\gamma^{\mu}} \wedge \gamma_{\nu} \left( \partial_{\mu} A^{\nu} - \partial^{\nu} A_{\mu} \right).
\end{aligned}
\end{equation}

These implicitly define the tensor in terms of potential, so we can write:
Calculating the tensor in terms of the bivector we have:

\begin{equation}\label{eqn:maxToTensor:tensorpot}
\begin{aligned}
F^{\mu\nu} &= F \cdot (\gamma^{\nu} \wedge \gamma^{\mu}) = \partial^{\mu} A^{\nu} - \partial^{\nu} A^{\mu} \\
F_{\mu\nu} &= F \cdot (\gamma_{\nu} \wedge \gamma_{\mu}) = \partial_{\mu} A_{\nu} - \partial_{\nu} A_{\mu} \\
{F^{\mu}}_{\nu} &= F \cdot (\gamma^{\nu} \wedge \gamma_{\mu}) = \partial^{\mu} A_{\nu} - \partial_{\nu} A^{\mu} \\
{F_{\mu}}^{\nu} &= F \cdot (\gamma_{\nu} \wedge \gamma^{\mu}) = \partial_{\mu} A^{\nu} - \partial^{\nu} A_{\mu}.
\end{aligned}
\end{equation}

These potential based equations of \eqnref{eqn:maxToTensor:tensorpot}, are consistent with the definition of the field tensor in terms of potential in the
\href{https://en.wikipedia.org/wiki/Covariant\_formulation\_of\_classical\_electromagnetism}{ wikipedia Covariant electromagnetism } article.
That article's definition of the field tensor is also consistent with the field tensor in matrix form of \eqnref{eqn:maxToTensor:matrixtensor}.

However, the \href{https://en.wikipedia.org/wiki/Electromagnetic_tensor}{wikipedia Electromagnetic Tensor}
uses different conventions (at the time of this writing), but both claim a \(-+++\) metric, so I think one is wrong.  I had naturally favor the
covariant article since it agrees with my results.

\subsection{Field equations in tensor form}

\begin{equation}\label{eqn:maxwellToTensor:300}
\begin{aligned}
J/c \epsilon_0
&= \grad (\grad \wedge A) \\
&= \grad \cdot ( \grad \wedge A ) + \grad \wedge \grad \wedge A.
\end{aligned}
\end{equation}

This produces two equations
\begin{equation*}
\grad \cdot ( \grad \wedge A ) = J/c \epsilon_0
\end{equation*}
\begin{equation*}
\grad \wedge \grad \wedge A = 0.
\end{equation*}

\subsubsection{Vector equation part}

Expanding the first in coordinates we have
\begin{equation}\label{eqn:maxwellToTensor:320}
\begin{aligned}
J/c \epsilon_0
&= \gamma^{\alpha} \partial_{\alpha} \cdot ( \gamma^{\mu} \wedge \gamma_{\nu} \partial_{\mu} A^{\nu} ) \\
&= (\gamma^{\alpha} \cdot \gamma_{\mu\nu}) \partial_{\alpha} \partial^{\mu} A^{\nu} \\
&= (
%%\gamma^{\alpha} \cdot \gamma_{\mu}_{\nu}
\delta^{\alpha}_{\mu} \gamma_{\nu}
-\delta^{\alpha}_{\nu} \gamma_{\mu}
) \partial_{\alpha} \partial^{\mu} A^{\nu} \\
&= ( \gamma_{\nu} \partial_{\mu} - \gamma_{\mu} \partial_{\nu} ) \partial^{\mu} A^{\nu} \\
&= \gamma_{\nu} \partial_{\mu} (\partial^{\mu} A^{\nu} -\partial^{\nu} A^{\mu} ) \\
&= \gamma_{\nu} \partial_{\mu} F^{\mu\nu}.
\end{aligned}
\end{equation}

Dotting the LHS with \(\gamma^{\alpha}\) we have
\begin{equation}\label{eqn:maxwellToTensor:340}
\begin{aligned}
\gamma^{\alpha} \cdot J/c \epsilon_0
&= \gamma^{\alpha} \cdot \gamma_{\beta }J^{\beta}/c \epsilon_0 \\
&= \delta^{\alpha}_{\beta }J^{\beta}/c \epsilon_0 \\
&= J^{\alpha}/c \epsilon_0.
\end{aligned}
\end{equation}

and for the RHS
\begin{equation}\label{eqn:maxwellToTensor:360}
\begin{aligned}
\gamma^{\alpha} \cdot \gamma_{\nu} \partial_{\mu} F^{\mu\nu}
&= \partial_{\mu} F^{\mu\alpha}.
\end{aligned}
\end{equation}

Or,
\begin{equation}
\partial_{\mu} F^{\mu\alpha} = J^{\alpha}/c \epsilon_0.
\end{equation}

This is exactly (with index switch) the tensor equation in
\href{https://en.wikipedia.org/wiki/Covariant\_formulation\_of\_classical\_electromagnetism}{ wikipedia Covariant electromagnetism } article.
It however, differs from the
\href{https://en.wikipedia.org/wiki/Electromagnetic_tensor}{wikipedia Electromagnetic Tensor} article.

\subsubsection{Trivector part}

Now, the trivector part of this equation does not seem like it is worth much consideration

\begin{equation}
\grad \wedge \grad \wedge A = 0.
\end{equation}

But this is four of the eight traditional Maxwell's equations when written out in terms of coordinates.  Let us write this out in tensor form
and see how this follows.

\begin{equation}\label{eqn:maxwellToTensor:380}
\begin{aligned}
\grad \wedge \grad \wedge A
&= (\gamma^{\alpha} \partial_{\alpha}) \wedge (\gamma^{\beta} \partial_{\beta}) \wedge (\gamma^{\sigma} A_{\sigma}) \\
&= (\gamma^{\alpha} \wedge \gamma^{\beta} \wedge \gamma^{\sigma}) \partial_{\alpha} \partial_{\beta} A_{\sigma} \\
&=
\inv{2}
(\gamma^{\alpha} \wedge \gamma^{\beta} \wedge \gamma^{\sigma}) \partial_{\alpha} \partial_{\beta} A_{\sigma}
+
\inv{2}
(\gamma^{\alpha} \wedge \gamma^{\sigma} \wedge \gamma^{\beta}) \partial_{\alpha} \partial_{\sigma} A_{\beta}
\\
&=
\inv{2}(\gamma^{\alpha} \wedge \gamma^{\beta} \wedge \gamma^{\sigma}) \partial_{\alpha} (\partial_{\beta} A_{\sigma} - \partial_{\sigma} A_{\beta}) \\
&= \inv{2}(\gamma^{\alpha} \wedge \gamma^{\beta} \wedge \gamma^{\sigma}) \partial_{\alpha} F_{\beta\sigma}.
\end{aligned}
\end{equation}

For each of the four trivectors that span the trivector space the coefficients of those trivectors must all therefore equal zero.  The duality set

\begin{equation*}
\{ i \gamma^{\mu} \}
\end{equation*}

can be used to enumerate these four equations, so to separate these from the wedge products we have to perform the dot products.  Here \(i\) can be any pseudoscalar
associated with the four vector space, and it will be convenient to use an "index-up" pseudoscalar \(i=\gamma^{0123}\).   This will still anticommute with any of the \(\gamma^{\mu}\) vectors.

\begin{equation}\label{eqn:maxwellToTensor:400}
\begin{aligned}
(\gamma^{\alpha} \wedge \gamma^{\beta} \wedge \gamma^{\sigma}) \cdot ( i \gamma^{\mu} )
&= \gpgradezero{ (\gamma^{\alpha} \wedge \gamma^{\beta} \wedge \gamma^{\sigma}) ( i \gamma^{\mu} ) } \\
&= -\gpgradezero{ \gamma^{\alpha} \gamma^{\beta} \gamma^{\sigma} \gamma^{\mu 0123} } \\
&= -\gpgradezero{ \gamma^{\alpha \beta \sigma \mu 0123} } \\
&= \epsilon^{ \alpha \beta \sigma \mu }.
\end{aligned}
\end{equation}

The last line follows with the observation that the scalar part will be zero unless \(\alpha\), \(\beta\), \(\sigma\), and \(\mu\) are all unique.  When they are \(0,1,2,3\) for example then we have \(i^2 = -1\), and any odd permutation will change the sign.

% i^2 : metric independent
%i^2
%=01230123
%=-00123123
%=-00112323
%=00112233
%=00 (\pm 1)^2 33
%=00 33
%=-1

Application of this to our curl of curl expression we have

\begin{equation*}
(\grad \wedge \grad \wedge A) \cdot (i \gamma^{\mu} ) = \inv{2}\epsilon^{ \alpha \beta \sigma \mu } \partial_{\alpha} F_{\beta\sigma}.
\end{equation*}

Because there are no magnetic sources, the one-half scale factor can be dropped, which leaves the remaining four equations of Maxwell's equation in standard tensor form

\begin{equation}
\epsilon^{ \alpha \beta \sigma \mu } \partial_{\alpha} F_{\beta\sigma} = 0.
\end{equation}

One of these will be Gauss's law \(\spacegrad \cdot \BB = 0\), and the other three can be summed in vector form for Faraday's law \(\spacegrad \cross \BE + \PD{t}{\BB} = 0\).

\subsection{Lagrangian density in terms of potential}

We have seen that we can write the core of the Lagrangian density in two forms:

\begin{equation*}
\inv{2} F_{\mu\nu}F^{\mu\nu} = -\gpgradezero{F^2} = c^2 \BB^2 -\BE^2,
\end{equation*}

where summarizing the associated relations we have:

\begin{equation*}
F = \BE + i c \BB = \inv{2} F^{\mu\nu} \gamma_{\mu\nu} = \grad \wedge A = E^i \gamma_{i 0} - \epsilon_{i j k} c B^i \gamma_{j k}
\end{equation*}
\begin{equation}\label{eqn:maxwellToTensor:420}
\begin{aligned}
F^{\mu\nu} &= \partial^{\mu} A^{\nu} - \partial^{\nu} A^{\mu} \\
F_{\mu\nu} &= \partial_{\mu} A_{\nu} - \partial_{\nu} A_{\mu} \\
F^{i0} &= E^i = -F_{i0} \\
F^{ij} &= -\epsilon_{i j k} c B^k = F_{ij}.
\end{aligned}
\end{equation}

Now, if we want the density in terms of potential, by inspection we can form this from the tensor as:

\begin{equation*}
\inv{2} F_{\mu\nu}F^{\mu\nu} = \inv{2} (\partial_{\mu} A_{\nu} - \partial_{\nu} A_{\mu} ) (\partial^{\mu} A^{\nu} - \partial^{\nu} A^{\mu} ).
\end{equation*}

We should also be able to calculate this directly from the bivector square.  Lets verify this:

\begin{equation}\label{eqn:maxwellToTensor:440}
\begin{aligned}
\gpgradezero{ F^2 }
&= \gpgradezero{ (\grad \wedge A)(\grad \wedge A) } \\
&= \gpgradezero{ (\gamma^{\mu} \wedge \gamma_{\nu} \partial_{\mu} A^{\nu}) (\gamma^{\alpha} \wedge \gamma_{\beta} \partial_{\alpha} A^{\beta}) } \\
&= (\gamma^{\mu} \wedge \gamma^{\nu} \partial_{\mu} A_{\nu}) \cdot (\gamma_{\alpha} \wedge \gamma_{\beta} \partial^{\alpha} A^{\beta}) \\
&= ( ( (\gamma^{\mu} \wedge \gamma^{\nu}) \cdot \gamma_{\alpha} ) \cdot \gamma_{\beta} ) \partial_{\mu} A_{\nu} \partial^{\alpha} A^{\beta} \\
&= \left( \delta^{\mu}_{\beta} \delta^{\nu}_{\alpha} - \delta^{\mu}_{\alpha} \delta^{\nu}_{\beta} \right) \partial_{\mu} A_{\nu} \partial^{\alpha} A^{\beta} \\
&= \partial_{\mu} A_{\nu} \partial^{\nu} A^{\mu} - \partial_{\mu} A_{\nu} \partial^{\mu} A^{\nu} \\
&= \partial_{\mu} A_{\nu} \left( \partial^{\nu} A^{\mu} - \partial^{\mu} A^{\nu} \right) \\
&=
\inv{2} \left( \partial_{\mu} A_{\nu} \left( \partial^{\nu} A^{\mu} - \partial^{\mu} A^{\nu} \right)
+\partial_{\nu} A_{\mu} \left( \partial^{\mu} A^{\nu} - \partial^{\nu} A^{\mu} \right) \right)
\\
&= \inv{2} \left( \partial_{\nu} A_{\mu} - \partial_{\mu} A_{\nu} \right) \left( \partial^{\nu} A^{\mu} - \partial^{\mu} A^{\nu} \right) \\
&= -\inv{2} F_{\mu\nu}F^{\mu\nu},
\end{aligned}
\end{equation}

as expected.

The factor of \(1/2\) appearance is a \(x = (1/2)(x + x)\) operation, plus a switch of dummy indices in one half of the sum.

With the density expanded completely in terms of potentials things are in a form for an attempt to
evaluate the Lagrangian equations or do the
variational exercise (as in Feynman
\citep{feynman1963flp}
with the electrostatic case) and see that this recovers the field equations (covered in a subsequent set of notes in both fashions).

   %
% Copyright � 2012 Peeter Joot.  All Rights Reserved.
% Licenced as described in the file LICENSE under the root directory of this GIT repository.
%

%
%
\chapter{Four vector potential}
\index{vector potential}
\index{four potential}
\label{chap:emPotential}
%\date{August 15, 2008}

\section{}

Goldstein's classical mechanics, and many other texts, will introduce the four potential starting with
Maxwell's equation in scalar, vector, bivector, trivector expanded form:

\begin{equation}\label{eqn:emPotential:20}
\begin{aligned}
\spacegrad \cdot \BE &= \frac{\rho}{\epsilon_0} \\
\spacegrad \cdot \BB &= 0 \\
\spacegrad \cross \BE &= - \frac{\partial \BB}{\partial t} \\
\spacegrad \cross \BB &= \mu_0\left(\BJ + \epsilon_0 \frac{\partial \BE}{\partial t}\right) \\
\end{aligned}
\end{equation}

ie: E can not be a gradient, since it has a curl, but B can be the curl of something since it has zero
divergence, so we have \(\BB = \spacegrad \cross \BA\).  Faraday's law above gives:

\begin{equation}\label{eqn:emPotential:40}
\begin{aligned}
0 &= \spacegrad \cross \BE + \frac{\partial \spacegrad \cross \BA}{\partial t} \\
&= \spacegrad \cross \left(\BE + \frac{\partial \BA}{\partial t}\right) \\
\end{aligned}
\end{equation}

Because this curl is zero, one can write it as a gradient, say \(-\spacegrad \phi\).

The end result are the equations:

\begin{align}
\BE &= - \left( \spacegrad \phi + \partial_t \BA \right) \label{eqn:fourPot:BE} \\
\BB &= \spacegrad \cross \BA \label{eqn:fourPot:BB}
\end{align}

Looking at what Goldstein does with this (which I re-derived above to put in the SI form I am used to), my
immediate question is how would the combined bivector field look when expressed using an STA basis, and
then once that is resolved, how would his Lagrangian for a charged point particle look in explicit four
vector form?

Intuition says that this is all going to work out to be a spacetime gradient of a four vector, but
I am not sure how the Lorentz gauge freedom will turn out.  Here is an exploration of this.

\subsection{}

Forming as usual

\begin{equation}\label{eqn:fourPot:BF}
\BF = \BE + i c \BB
\end{equation}

We can combine the equations \eqnref{eqn:fourPot:BE} and \eqnref{eqn:fourPot:BB} into bivector form

\begin{equation}\label{eqn:fourPot:BFA}
\BF = - \left( \spacegrad \phi + \partial_t \BA \right) + c \spacegrad \wedge \BA
\end{equation}

\subsection{Dimensions}

Let us do a dimensional check before continuing:

\Eqnref{eqn:fourPot:BF} gives:

\begin{equation*}
[\BE] = \frac{[m][d]}{[q][t]^2}
\end{equation*}

That and \eqnref{eqn:fourPot:BFA} gives
\begin{equation*}
[\phi] = \frac{[m][d]^2}{[q][t]^2}
\end{equation*}

And the two \(\BA\) terms of \eqnref{eqn:fourPot:BFA} both give:
\begin{equation*}
[\BA] = \frac{[m][d]}{[q][t]}.
\end{equation*}

Therefore if we create a four vector out of \(\phi\), and \(\BA\) in SI units we will need that factor \(c\) with \(\BA\) with velocity dimensions to fix things up.

\subsection{Expansion in terms of components.  STA split}

\begin{equation}\label{eqn:emPotential:80}
\begin{aligned}
\BF
&= - \left( \spacegrad \phi + \partial_t \BA \right) + c \spacegrad \wedge \BA \\
&= - \sum \gamma_i \gamma_0 \partial_{x^i}\phi - \sum \gamma_i \gamma_0 \partial_t A^i
+ c \left(\sum \sigma_i \partial_{x^i}\right) \wedge \left(\sum \sigma_j A^j\right) \\
&= \sum \gamma^i \partial_{x^i} (\gamma_0 \phi) + \sum \gamma_0 \partial_{ct} c \gamma_i A^i
- \left(\sum \gamma_i \partial_{x^i}\right) \wedge \left(\sum \gamma_j c A^j\right) \\
&= \sum \gamma^i \wedge \gamma_0 \partial_{x^i} \phi + \sum \gamma^0 \wedge \gamma_i \partial_{x^0} c A^i
+ \sum \gamma^i \wedge \gamma_j \partial_{x^i} c A^j \\
&= \left(\sum \gamma^i \partial_{x^i}\right) \wedge \left(\gamma_0 \phi + \gamma_i c A^i \right) \\
&= \grad \wedge \left(\gamma_0 \phi + \sum \gamma_i c A^i \right)
\end{aligned}
\end{equation}

Once the electric and magnetic fields are treated as one entity, the separate equations of
\eqnref{eqn:fourPot:BE} and \eqnref{eqn:fourPot:BB} become nothing more than a statement that the bivector field \(\BF\) is the spacetime curl
of a four vector potential \(A = \gamma_0 \phi + \sum \gamma_i c A^i\).

This original choice of components \(A^i\), defined such that \(\BB = \spacegrad \cross \BA\) is a bit unfortunate in SI
units.  Setting \(\calA^i = cA^i\), and \(\calA^0 = \phi\), one then has the more symmetric form.

\begin{equation*}
A = \sum \gamma_{\mu} \calA^{\mu}.
\end{equation*}

Of course the same thing could be achieved with \(c=1\) ;)

Anyways, substitution of this back into Maxwell's equation gives:

\begin{equation*}
\grad (\grad \wedge A) = \grad \cdot (\grad \wedge A) + \mathLabelBox{\grad \wedge \grad \wedge A}{\(=0\)} = J
\end{equation*}

One can see an immediate simplification possible if one requires:

\begin{equation*}
\grad \cdot A = 0.
\end{equation*}

Then we are left with a forced wave equation to solve for the four potential:

\begin{equation*}
\grad^2 A = -\left(\sum \partial_{x^i x^i} - \inv{c^2}\partial_{tt}\right) A = J.
\end{equation*}

Now, without all this mess of algebra, I could have gone straight to this end result (and had done so previously).  I just
wanted to see where I would get applying the STA basis to the classical vector+scalar four vector ideas.

\subsection{Lorentz gauge}
\index{Lorentz gauge}

Looking at \(\grad \cdot A = 0\), I was guessing that this was what I recalled being called the Lorentz gauge, but in a
slightly different form.

If one expands this you get:

\begin{equation}\label{eqn:emPotential:100}
\begin{aligned}
0
&= \grad \cdot A \\
&= \sum \gamma^{\mu} \partial_{\mu} \cdot \left(\gamma_0 \phi + c \sum \gamma_j A^j \right) \\
&= \partial_{ct} \phi + c \sum \partial_{x^i}A^i \\
&= \partial_{ct} \phi + c \spacegrad \cdot \BA
\end{aligned}
\end{equation}

Or,

\begin{equation}
\spacegrad \cdot \BA = -\inv{c^2}\partial_{t} \phi
\end{equation}

Checked my Feynman book.  Yes, this is the Lorentz Gauge.

Another note.  Again the SI units make things ugly.  With the above modification of components that hide this, where one sets \(A = \sum \gamma_i \calA^i\), this gauge equation also takes a simpler form:

\begin{equation*}
0 = \grad \cdot A = \left(\sum \gamma^{\mu} \partial_{x^{\mu}}\right) \cdot \left(\sum \gamma_{\nu} \calA^{\nu}\right) = \sum \partial_{x^{\mu}} \calA^{\mu}.
\end{equation*}

\section{Appendix}

\subsection{wedge of spacetime bivector basis elements}

For \(i \ne j\):

\begin{equation}\label{eqn:emPotential:120}
\begin{aligned}
\sigma_i \wedge \sigma_j
&= \inv{2}\left( \sigma_i \sigma_j - \sigma_j \sigma_i \right) \\
&= \inv{2}\left( \gamma_{i0j0} - \gamma_{j0i0} \right) \\
&= \inv{2}\left( -\gamma_{ij} +\gamma_{ji} \right) \\
&= \gamma_{ji}
\end{aligned}
\end{equation}

   %
% Copyright � 2012 Peeter Joot.  All Rights Reserved.
% Licenced as described in the file LICENSE under the root directory of this GIT repository.
%

%
%
\mychapter{Metric signature dependencies}
\index{metric}
\label{chap:emBivectorMetricDependencies}
%\date{Sept 5, 2008.  emBivectorMetricDependencies.tex}

\section{Motivation}

Doran/Lasenby use a \(+,-,-,-\) signature, and I had gotten used to that.  On first seeing the alternate signature used by
\href{http://www.av8n.com/physics/maxwell-ga.pdf}{ John Denker's excellent GA explanatory paper },
I found myself disoriented.  How many of the identities that I was used to were metric dependent?   Here are some notes that explore some of the
metric dependencies of STA, in particular observing which identities are metric dependent and which are not.

In the end this exploration turned into a big meandering examination and comparison of the bivector and tensor forms of Maxwell's equation.  That part has been split into a different writeup.

\section{The guts}

\subsection{Spatial basis}

Our spatial (bivector) basis:
%
\begin{equation*}
\sigma_i = \gamma_i \wedge \gamma_0 = \gamma_{i0},
\end{equation*}
%
that behaves like Euclidean vectors (positive square) still behave as desired, regardless of the signature:
%
\begin{equation}\label{eqn:emBivectorMetricDependencies:20}
\begin{aligned}
\sigma_i \cdot \sigma_j
&= \gpgradezero{\gamma_{i0j0}}  \\
&= - \gpgradezero{\gamma_{ij}} (\gamma_{0})^2  \\
&= -\delta_{ij} (\gamma_i)^2 (\gamma_{0})^2
\end{aligned}
\end{equation}
%
Regardless of the signature the pair of products \((\gamma_i)^2 (\gamma_{0})^2 = -1\), so our spatial bivectors are metric invariant.

\subsection{How about commutation?}
%
Commutation with
\begin{equation*}
i \gamma_{\mu} = \gamma_{0123\mu} = \gamma_{\mu0123}
\end{equation*}
%
\(\mu\) has to "pass" three indices regardless of metric, so anticommutes for any \(\mu\).
%
\begin{equation*}
\sigma_k \gamma_{\mu} = \gamma_{k0\mu}
\end{equation*}
%
If \(k = \mu\), or \(0 = \mu\), then we get a sign inversion, and otherwise commute (pass two indices).  This is also metric invariant.

\subsection{Spatial and time component selection}

With a positive time metric (Doran/Lasenby) selection of the \(x^0\) component of a vector \(x\) requires a dot product:
%
\begin{equation*}
x = x^0 \gamma_0 + x^i \gamma_i
\end{equation*}
%
\begin{equation*}
x \cdot \gamma_0 = x^0 (\gamma_0)^2
\end{equation*}
%
Obviously this is a metric dependent operation.  To generalize it appropriately, we need to dot with \(\gamma^0\) instead:
%
\begin{equation*}
x \cdot \gamma^0 = x^0
\end{equation*}
%
Now, what do we get when wedging with this upper index quantity instead.
%
\begin{equation}\label{eqn:emBivectorMetricDependencies:40}
\begin{aligned}
x \wedge \gamma^0
&= \left(x^0 \gamma_0 + x^i \gamma_i\right) \wedge \gamma^0 \\
&= x^i \gamma_i \wedge \gamma^0 \\
&= x^i \gamma_{i0} (\gamma^0)^2 \\
&= x^i \sigma_i (\gamma^0)^2 \\
&= \Bx \left(\gamma^0\right)^2
\end{aligned}
\end{equation}
%
Not quite the usual expression we are used to, but it still behaves as a Euclidean vector (positive square), regardless of the metric:
%
\begin{equation*}
(x \wedge \gamma^0)^2 = \left(\pm \Bx\right)^2 = \Bx^2
\end{equation*}
%
This suggests that we should define our spatial projection vector as \(x \wedge \gamma^0\) instead of \(x \wedge \gamma_0\) as done in
Doran/Lasenby (where a positive time metric is used).

\subsubsection{Velocity}

Variation of a event path with some parameter we have:
%
\begin{equation}\label{eqn:emBivectorMetricDependencies:60}
\begin{aligned}
\frac{ d x }{ d \lambda }
&= \frac{ d x^{\mu} }{ d \lambda } \gamma_{\mu} = c \frac{dt}{d\lambda} \gamma_0 + \frac{d x^i }{d\lambda} \gamma_i \\
&= \frac{d t}{d \lambda} \left( c \gamma_0 + \frac{d x^i }{dt} \gamma_i \right)
\end{aligned}
\end{equation}
%
The square of this is:
%becomes metric dependent:
\begin{equation}\label{eqn:emBivectorMetricDependencies:80}
\begin{aligned}
\inv{c^2} \left(\frac{ d x }{ d \lambda } \right)^2
&= \left(\frac{dt }{d\lambda}\right)^2 (\gamma_0)^2 \left( 1 + \inv{c^2}\left(\frac{d x^i }{dt}\right)^2 (\gamma_i)^2 (\gamma_0)^2 \right) \\
&= \left(\frac{d t}{d \lambda}\right)^2 (\gamma_0)^2 \left( 1 - (\Bv/c)^2 \right) \\
\frac{ (\gamma_0)^2 }{c^2} \left(\frac{ d x }{ d \lambda } \right)^2 &= \left(\frac{d t}{d \lambda}\right)^2 \left( 1 - (\Bv/c)^2 \right) \\
\end{aligned}
\end{equation}
%
We define the proper time \(\tau\) as that particular parametrization \(c \tau = \lambda\) such that the LHS equals 1.  This is implicitly defined
via the integral
%
\begin{equation*}
\tau = \int \sqrt{ 1 - (\Bv/c)^2 } dt = \int \sqrt{ 1 - \left(\inv{c} \frac{dx^i }{d \alpha} \right)^2 } d\alpha
\end{equation*}
%
Regardless of this parametrization \(\alpha = \alpha(t)\), this velocity scaled 4D arc length is the same.  This is a bit of a digression from the
ideas of metric dependence investigation.  There is however a metric dependence in the first steps arriving at this result.

with proper velocity defined in terms of proper time \(v = dx/d\tau\), we also have:
%
\begin{equation}\label{eqn:emBiMetDep:gamma}
\gamma = \frac{dt}{d\tau} = \inv{ \sqrt{ 1 - (\Bv/c)^2 } }
\end{equation}
\begin{equation}
v = \gamma \left(c \gamma_0 + \frac{d x^i }{d t} \gamma_i \right)
\end{equation}
%
Therefore we can select this quantity \(\gamma\), and our spatial velocity components, from our proper velocity:
%
\begin{equation*}
c \gamma = v \cdot \gamma^0
\end{equation*}
%
In \eqnref{eqn:emBiMetDep:gamma} we did not define \(\Bv\), only implicitly requiring that its square was \(\sum (dx^i/dt)^2\), as we require for correspondence with Euclidean meaning.  This can be made more exact by
taking wedge products to weed out the time component:
%
\begin{equation*}
v \wedge \gamma^0 = \gamma \frac{d x^i }{d t} \gamma_i \wedge \gamma^0
\end{equation*}
%
With a definition of \(\Bv = \frac{d x^i }{d t} \gamma_i \wedge \gamma^0\) (which has the desired positive square), we therefore have:
%
\begin{equation}\label{eqn:emBivectorMetricDependencies:100}
\begin{aligned}
\Bv
&= \frac{v \wedge \gamma^0 }{\gamma} \\
&= \frac{v \wedge \gamma^0 }{ v/c \cdot \gamma^0 } \\
\end{aligned}
\end{equation}
%
Or,
\begin{equation}
\Bv/c = \frac{v/c \wedge \gamma^0 }{ v/c \cdot \gamma^0 }
\end{equation}
%
All the lead up to this allows for expression of the spatial component of the proper velocity in a metric independent fashion.

\subsection{Reciprocal Vectors}
\index{reciprocal frame}

By reciprocal frame we mean the set of vectors \(\{u^{\alpha}\}\) associated with a basis
for some linear subspace \(\{u_{\alpha}\}\) such that:
%
\begin{equation*}
u_{\alpha} \cdot u^{\beta} = \delta_{\alpha}^\beta
\end{equation*}
%
In the special case of orthonormal vectors \(u_{\alpha} \cdot u_{\beta} = \pm \delta_{\alpha\beta}\) the reciprocal frame vectors
are just the inverses (literally reciprocals), which can be verified by taking dot products:
%
\begin{equation}\label{eqn:emBivectorMetricDependencies:120}
\begin{aligned}
\inv{u_{\alpha}} \cdot {u_{\alpha}}
&= \gpgradezero{ \inv{u_{\alpha}} {u_{\alpha}} } \\
&= \gpgradezero{ \inv{u_{\alpha}} \frac{u_{\alpha}}{u_{\alpha}} {u_{\alpha}} } \\
&= \gpgradezero{ \frac{(u_{\alpha})^2}{(u_{\alpha})^2} } \\
&= 1
\end{aligned}
\end{equation}
%
Written out explicitly for our positive "orthonormal" time metric:
%
\begin{equation}\label{eqn:emBivectorMetricDependencies:140}
\begin{aligned}
(\gamma_0)^2 &= 1 \\
(\gamma_i)^2 &= -1,
\end{aligned}
\end{equation}
%
we have the reciprocal vectors:
\begin{equation}\label{eqn:emBivectorMetricDependencies:160}
\begin{aligned}
\gamma_0 &= \gamma^0 \\
\gamma_i &= -\gamma^i \\
\end{aligned}
\end{equation}
%
Note that this last statement is consistent with \((\gamma_i)^2 = -1\), since \((\gamma_i)^2 = \gamma_i (-\gamma^i) = -\delta_i^i = -1\)

Contrast this with a positive spatial metric:
%
\begin{equation}\label{eqn:emBivectorMetricDependencies:180}
\begin{aligned}
(\gamma_0)^2 &= -1 \\
(\gamma_i)^2 &= 1,
\end{aligned}
\end{equation}
%
with reciprocal vectors:
\begin{equation}\label{eqn:emBivectorMetricDependencies:200}
\begin{aligned}
\gamma_0 &= -\gamma^0 \\
\gamma_i &= \gamma^i \\
\end{aligned}
\end{equation}
%
where we have the opposite.

\subsection{Reciprocal Bivectors}

Now, let us examine the bivector reciprocals.  Given our orthonormal vector basis, let us invert the bivector and verify that is what we want:
%
\begin{equation}\label{eqn:emBivectorMetricDependencies:220}
\begin{aligned}
\inv{\gamma_{\mu\nu}}
&= \inv{\gamma_{\mu\nu}} \frac{ \gamma_{\nu\mu} }{ \gamma_{\nu\mu} } \\
&= \inv{\gamma_{\mu\nu}} \inv{ \gamma_{\nu\mu} }{ \gamma_{\nu\mu} } \\
&= \inv{\gamma_{\mu\nu\nu\mu} }{ \gamma_{\nu\mu}} \\
&= \inv{ (\gamma_{\mu})^2 (\gamma_{\nu})^2 } { \gamma_{\nu\mu}} \\
\end{aligned}
\end{equation}
%
Multiplication with our vector we will get 1 if this has the required reciprocal relationship:
\begin{equation}\label{eqn:emBivectorMetricDependencies:240}
\begin{aligned}
\inv{\gamma_{\mu\nu}} \gamma_{\mu\nu}
&= \inv{ (\gamma_{\mu})^2 (\gamma_{\nu})^2 } { \gamma_{\nu\mu}} \gamma_{\mu\nu} \\
&= \frac{ (\gamma_{\mu})^2 (\gamma_{\nu})^2 }{ (\gamma_{\mu})^2 (\gamma_{\nu})^2 } \\
&= 1
\end{aligned}
\end{equation}
%
Observe that unlike our basis vectors the bivector reciprocals are metric independent.  Let us verify this explicitly:
%
\begin{equation}\label{eqn:emBivectorMetricDependencies:260}
\begin{aligned}
\inv{\gamma_{i0}} &= \inv{ (\gamma_{i})^2 (\gamma_{0})^2 } { \gamma_{0i}} \\
\inv{\gamma_{ij}} &= \inv{ (\gamma_{i})^2 (\gamma_{j})^2 } { \gamma_{ji}} \\
\inv{\gamma_{0i}} &= \inv{ (\gamma_{0})^2 (\gamma_{i})^2 } { \gamma_{i0}} \\
\end{aligned}
\end{equation}
%
With a spacetime mix of indices we have a \(-1\) denominator for either metric.  With a spatial only mix (\(B\) components) we have \(1\) in the denominator \(1^2 = (-1)^2\) for either metric.

Now, perhaps counter to intuition the reciprocal \(\inv{\gamma_{\mu\nu}}\) of \(\gamma_{\mu\nu}\) is not \(\gamma^{\mu\nu}\), but instead \(\gamma^{\nu\mu}\).  Here the shorthand can be deceptive and it is worth verifying this statement explicitly:
%
\begin{equation}\label{eqn:emBivectorMetricDependencies:280}
\begin{aligned}
\gamma_{\mu\nu} \cdot \gamma^{\alpha\beta}
&= (\gamma_{\mu} \wedge \gamma_{\nu}) \cdot (\gamma^{\alpha} \wedge \gamma^{\beta}) \\
&= ((\gamma_{\mu} \wedge \gamma_{\nu}) \cdot \gamma^{\alpha}) \cdot \gamma^{\beta}) \\
&= ( \gamma_{\mu} (\gamma_{\nu} \cdot \gamma^{\alpha}) - \gamma_{\nu} (\gamma_{\mu} \cdot \gamma^{\alpha}) ) \cdot \gamma^{\beta}) \\
&= ( \gamma_{\mu} {\delta_{\nu}}^{\alpha} - \gamma_{\nu} {\delta_{\mu}}^{\alpha} ) \cdot \gamma^{\beta} \\
\end{aligned}
\end{equation}
%
Or,
\begin{equation}
\gamma_{\mu\nu} \cdot \gamma^{\alpha\beta} = {\delta_{\mu}}^{\beta} {\delta_{\nu}}^{\alpha} - {\delta_{\nu}}^{\beta} {\delta_{\mu}}^{\alpha}
\end{equation}
%
In particular for matched pairs of indices we have:
\begin{equation*}
\gamma_{\mu\nu} \cdot \gamma^{\nu\mu} = {\delta_{\mu}}^{\mu} {\delta_{\nu}}^{\nu} - {\delta_{\nu}}^{\mu} {\delta_{\mu}}^{\nu} = 1
\end{equation*}
%
\subsection{Pseudoscalar expressed with reciprocal frame vectors}
\index{pseudoscalar}

With a positive time metric
%
\begin{equation*}
\gamma_{0123} = -\gamma^{0123}
\end{equation*}
%
(three inversions for each of the spatial quantities).  This is metric invariant too since it will match the single negation for the same operation
using a positive spatial metric.

\subsection{Spatial bivector basis commutation with pseudoscalar}
%
I have been used to writing:
\begin{equation*}
\sigma_j = \gamma_{j0}
\end{equation*}
%
as a spatial basis, and having this equivalent to the four-pseudoscalar, but this only works with a time positive metric:
\begin{equation*}
i_3 = \sigma_{123} = \gamma_{102030} = \gamma_{0123} (\gamma_0)^2
\end{equation*}
%
With the spatial positive spacetime metric we therefore have:
%
\begin{equation*}
i_3 = \sigma_{123} = \gamma_{102030} = -i_4
\end{equation*}
%
instead of \(i_3 = i_4\) as is the case with a time positive spacetime metric.  We see that the metric choice can also be interpreted as a choice of handedness.

That choice allowed Doran/Lasenby to initially write the field as a vector plus trivector where \(i\) is the spatial pseudoscalar:
%
\begin{equation}\label{eqn:emBiMetDep:field}
F = \BE + i c \BB,
\end{equation}
%
and then later switch the interpretation of \(i\) to the four space pseudoscalar.  The freedom to do so is metric dependent freedom, but
\eqnref{eqn:emBiMetDep:field} works regardless of metric when \(i\) is uniformly interpreted as the spacetime pseudoscalar.

Regardless of the metric the spacetime pseudoscalar commutes with \(\sigma_j = \gamma_{j0}\), since it anticommutes twice to cross:
%
\begin{equation*}
\sigma_j i = \gamma_{j00123} = \gamma_{00123j} = \gamma_{0123j0} = i \sigma_j
\end{equation*}
%
\subsection{Gradient and Laplacian}
\index{gradient}
\index{Laplacian}

As seen by the Lagrangian based derivation of the (spacetime or spatial) gradient, the form is metric independent and valid even for non-orthonormal frames:
%
\begin{equation*}
\grad = \gamma^{\mu} \PD{x^{\mu}}{}
\end{equation*}
%
\subsubsection{Vector derivative}
\index{vector derivative}

A cute aside, as pointed out in John Denker's paper, for orthonormal frames, this can also be written as:
%
\begin{equation}\label{eqn:emBiMetDep:gradient}
\grad = \inv{\gamma_{\mu}} \PD{x^{\mu}}{}
\end{equation}
%
as a mnemonic for remembering where the signs go, since in that form the upper and lower indices are nicely matched in summation convention fashion.

Now, \(\gamma_{\mu}\) is a constant when we are not working in curvilinear coordinates, and for constants we are used to the freedom to pull them into our
derivatives as in:
%
\begin{equation*}
\inv{c} \PD{t}{} = \PD{(ct)}{}
\end{equation*}
%
Supposing that one had an orthogonal vector decomposition:
%
\begin{equation*}
\Bx = \sum \gamma_i x^i = \sum \Bx_i
\end{equation*}
%
then, we can abuse notation and do the same thing with our unit vectors, rewriting the gradient \eqnref{eqn:emBiMetDep:gradient} as:
%
\begin{equation}\label{eqn:emBiMetDep:gradvec}
\grad = \PD{(\gamma_{\mu} x^{\mu})}{} = \sum \PD{\Bx_i}{}
\end{equation}
%
Is there anything to this that is not just abuse of notation?  I think so, and I am guessing the notational freedom to do this is closely related to
what Hestenes calls geometric calculus.

Expanding out the gradient in the form of \eqnref{eqn:emBiMetDep:gradvec} as a limit statement this becomes, rather loosely:
%
\begin{equation*}
\grad = \sum_i \lim_{d\Bx_i \to 0} \inv{ d \Bx_i } \left(f( \Bx + d\Bx_i ) - f( \Bx )\right)
\end{equation*}
%
If nothing else this justifies the notation for the polar form gradient of a function that is only radially dependent, where the quantity:
%
\begin{equation*}
\spacegrad = \rcap\PD{r}{} = \inv{\rcap}\PD{r}{}
\end{equation*}
%
is sometimes written:
%
\begin{equation*}
\spacegrad = \PD{\Br}{}
\end{equation*}
%
Tong does this for example in his online dynamics paper, although there it appears to be not much more than a fancy shorthand for gradient.

\subsection{Four-Laplacian}
\index{four-Laplacian}

Now, although our gradient is metric invariant, its square the four-Laplacian is not.  There we have:
%
\begin{equation}\label{eqn:emBivectorMetricDependencies:300}
\begin{aligned}
\grad^2
&= \sum (\gamma^{\mu})^2 \PDsQ{x^{\mu}}{} \\
&= (\gamma^0)^2 \left( \PDsQ{x^0}{} + (\gamma^0)^2 (\gamma^i)^2 \PDsQ{x^i}{} \right) \\
&= (\gamma^0)^2 \left( \PDsQ{x^0}{} - \PDsQ{x^i}{} \right)
\end{aligned}
\end{equation}
%
This makes the metric dependency explicit so that we have:
%
\begin{equation*}
\grad^2 = \inv{c^2} \PDsQ{t}{} - \PDsQ{x^i}{} \quad \mbox{if \((\gamma^0)^2 = 1\)}
\end{equation*}
\begin{equation*}
\grad^2 = \PDsQ{x^i}{} - \inv{c^2} \PDsQ{t}{} \quad \mbox{if \((\gamma^0)^2 = -1\)}
\end{equation*}

   %
% Copyright � 2012 Peeter Joot.  All Rights Reserved.
% Licenced as described in the file LICENSE under the root directory of this GIT repository.
%

%
%
%\input{../peeter_prologue.tex}

\mychapter{Wave equation form of Maxwell's equations}
\label{chap:emVacWave}
\index{wave equation}
\index{Maxwell's equation}
%\blogpage{http://sites.google.com/site/peeterjoot/math2009/emVacWave.pdf}
%\date{ June 21, 2009.  \(RCSfile: emVacWave.tex,v \) Last \(Revision: 1.5 \) \(Date: 2009/07/11 05:49:02 \) }

\beginArtWithToc

\section{Motivation}

In \citep{jackson1975cew}, on plane waves, he writes "we find easily..." to show that the wave equation for each of the components of \(\BE\), and \(\BB\) in the absence of current and charge satisfy the wave equation.  Do this calculation.

\section{Vacuum case}

Avoiding the non-vacuum medium temporarily, Maxwell's vacuum equations (in SI units) are
%
\begin{equation}\label{eqn:emVacWave:divE}
\begin{aligned}
\spacegrad \cdot \BE = 0
\end{aligned}
\end{equation}
\begin{equation}\label{eqn:emVacWave:divB}
\begin{aligned}
\spacegrad \cdot \BB = 0
\end{aligned}
\end{equation}
\begin{equation}\label{eqn:emVacWave:curlB}
\begin{aligned}
\spacegrad \cross \BB = \inv{c^2} \frac{\partial \BE}{\partial t}
\end{aligned}
\end{equation}
\begin{equation}\label{eqn:emVacWave:curlE}
\begin{aligned}
\spacegrad \cross \BE = -\frac{\partial \BB}{\partial t}
\end{aligned}
\end{equation}
%
The last two curl equations can be decoupled by once more calculating the curl.  Illustrating by example
%
\begin{equation}\label{eqn:emVacWave:curlCurlE}
\begin{aligned}
\spacegrad \cross (\spacegrad \cross \BE) = -\frac{\partial }{\partial t} \spacegrad \cross \BB = -\inv{c^2} \frac{\partial^2 \BE}{\partial t^2}
\end{aligned}
\end{equation}
%
Digging out vector identities and utilizing the zero divergence we have
%
\begin{equation}\label{eqn:emVacWave:identForcurlCurlE}
\begin{aligned}
\spacegrad \cross (\spacegrad \cross \BE) = \spacegrad (\spacegrad \cdot \BE) - \spacegrad^2 \BE = -\spacegrad^2 \BE
\end{aligned}
\end{equation}
%
Putting \eqnref{eqn:emVacWave:curlCurlE}, and \eqnref{eqn:emVacWave:identForcurlCurlE} together provides a wave equation for the electric field vector
%
\begin{equation}\label{eqn:emVacWave:waveE}
\begin{aligned}
\inv{c^2} \frac{\partial^2 \BE}{\partial t^2} - \spacegrad^2 \BE = 0
\end{aligned}
\end{equation}
%
Operating with curl on the remaining Maxwell equation similarly produces a wave equation for the magnetic field vector
%
\begin{equation}\label{eqn:emVacWave:waveB}
\begin{aligned}
\inv{c^2} \frac{\partial^2 \BB}{\partial t^2} - \spacegrad^2 \BB = 0
\end{aligned}
\end{equation}
%
This is really six wave equations, one for each of the field coordinates.

\section{With Geometric Algebra}

Arriving at \eqnref{eqn:emVacWave:waveE}, and \eqnref{eqn:emVacWave:waveB} is much easier using the GA formalism of \citep{doran2003gap}.

Pre or post multiplication of the gradient with the observer frame time basis unit vector \(\gamma_0\) has a conjugate like
action
%
\begin{equation}\label{eqn:emVacWave:20}
\begin{aligned}
\grad \gamma_0
&=
\gamma^0 \gamma_0 \partial_0 + \gamma^k \gamma_0 \partial_k \\
&=
\partial_0 - \spacegrad \\
\end{aligned}
\end{equation}
%
(where as usual our spatial basis is \(\sigma_k = \gamma_k \gamma_0\)).
%
Similarly
\begin{equation}\label{eqn:emVacWave:40}
\begin{aligned}
\gamma_0 \grad
&=
\partial_0 + \spacegrad \\
\end{aligned}
\end{equation}
%
For the vacuum Maxwell's equation is just
\begin{equation}\label{eqn:emVacWave:60}
\begin{aligned}
\grad F = \grad (\BE + I c \BB) = 0
\end{aligned}
\end{equation}
%
With nothing more than an algebraic operation we have
%
\begin{equation}\label{eqn:emVacWave:80}
\begin{aligned}
0
&= \grad \gamma_0 \gamma_0 \grad F \\
&=
( \partial_0 - \spacegrad ) ( \partial_0 + \spacegrad ) (\BE + I c \BB) \\
&=
\left( \inv{c^2} \frac{\partial^2}{\partial t^2} - \spacegrad^2 \right) (\BE + I c \BB) \\
\end{aligned}
\end{equation}
%
This equality is true independently for each of the components of \(\BE\) and \(\BB\), so we have as before

These wave equations are still subject to the constraints of the original Maxwell equations.
%
\begin{equation}\label{eqn:emVacWave:100}
\begin{aligned}
0 &= \gamma_0 \grad F \\
&= (\partial_0 + \spacegrad) (\BE + I c \BB) \\
&=
  \spacegrad \cdot \BE
+ (\partial_0 \BE - c \spacegrad \cross \BB)
+ I ( c \partial_0 \BB + \spacegrad \cross \BE )
+ I c \spacegrad \cdot \BB
\\
\end{aligned}
\end{equation}
%
\section{Tensor approach?}

In both the traditional vector and the GA form one can derive the wave equation relations of \eqnref{eqn:emVacWave:waveE}, \eqnref{eqn:emVacWave:waveB}.  One can obviously summarize these in tensor form as
%
\begin{equation}\label{eqn:emVacWave:waveFaraday}
\begin{aligned}
\partial_\mu\partial^\mu F^{\alpha\beta} = 0
\end{aligned}
\end{equation}
%
working backwards from the vector or GA result.  In this notation, the coupling constraint would be that the field variables \(F^{\alpha\beta}\) are subject to the Maxwell divergence equation (name?)
%
\begin{equation}\label{eqn:emVacWave:divergenceFaraday}
\begin{aligned}
\partial_\mu F^{\mu\nu} = 0
\end{aligned}
\end{equation}
%
and also the dual tensor relation
%
\begin{equation}\label{eqn:emVacWave:dualFaraday}
\begin{aligned}
\epsilon^{\sigma\mu\alpha\beta} \partial_\mu F_{\alpha\beta} = 0
\end{aligned}
\end{equation}
%
I cannot seem to figure out how to derive \eqnref{eqn:emVacWave:waveFaraday} starting from these tensor relations?

This probably has something to do with the fact that we require both the divergence and the dual relations \eqnref{eqn:emVacWave:divergenceFaraday}, \eqnref{eqn:emVacWave:dualFaraday} expressed together to do this.

\section{Electromagnetic waves in media}

Jackson lists the Macroscopic Maxwell equations in (6.70) as
%
\begin{equation}\label{eqn:emVacWave:120}
\begin{aligned}
\spacegrad \cdot \BB &= 0 \\
\spacegrad \cdot \BD &= 4 \pi \rho \\
\spacegrad \cross \BE + \inv{c}\PD{t}{\BB} &= 0 \\
\spacegrad \cross \BH - \inv{c}\PD{t}{\BD} &= \frac{4 \pi}{c} \BJ  \\
\end{aligned}
\end{equation}
%
(for this note this means unfortunately a switch from SI to CGS midstream)

For linear material (\(\BB = \mu \BH\), and \(\BD = \epsilon \BE\)) that is devoid of unbound charge and current (\(\rho =0\), and \(\BJ =0\)), we can assemble these into his (7.1) equations
%
\begin{equation}\label{eqn:emVacWave:140}
\begin{aligned}
\spacegrad \cdot \BB &= 0 \\
\spacegrad \cdot \BE &= 0 \\
\spacegrad \cross \BE + \inv{c}\PD{t}{\BB} &= 0 \\
\spacegrad \cross \BB - \frac{\epsilon \mu}{c}\PD{t}{\BE} &= 0  \\
\end{aligned}
\end{equation}
%
In this macroscopic form, it is not obvious how to assemble the equations into a nice tidy GA form.  A compromise is
%
\begin{equation}\label{eqn:emVacWave:compromise}
\begin{aligned}
\spacegrad \BE + \partial_0 (I\BB) &= 0 \\
\spacegrad (I \BB) + \epsilon \mu \partial_0 \BE &= 0
\end{aligned}
\end{equation}
%
Although not as pretty, we can at least derive the wave equations from these.  For example for \(\BE\), we apply one additional spatial gradient
%
\begin{equation}\label{eqn:emVacWave:160}
\begin{aligned}
0
&= \spacegrad^2 \BE + \partial_0 (\spacegrad I \BB) \\
&= \spacegrad^2 \BE + \partial_0 ( -\epsilon \mu \partial_0 \BE ) \\
\end{aligned}
\end{equation}
%
For \(\BB\) we get the same, and have two wave equations
%
\begin{equation}\label{eqn:emVacWave:waveEquationsMedia}
\begin{aligned}
\frac{\mu \epsilon}{c^2} \frac{\partial^2 \BE}{\partial t^2} - \spacegrad^2 \BE &= 0 \\
\frac{\mu \epsilon}{c^2} \frac{\partial^2 \BB}{\partial t^2} - \spacegrad^2 \BB &= 0
\end{aligned}
\end{equation}
%
The wave velocity is thus not \(c\), but instead the reduced speed of \(c/{\sqrt{\mu\epsilon}}\).

The fact that it is possible to assemble wave equations of this form means that there must also be a simpler form than \eqnref{eqn:emVacWave:compromise}.  The reduced velocity is the clue, and that can be used to refactor the constants
%
\begin{equation}\label{eqn:emVacWave:180}
\begin{aligned}
\spacegrad \BE + \sqrt{\mu\epsilon}\partial_0 \left(\frac{I\BB}{\sqrt{\mu\epsilon}}\right) &= 0 \\
\spacegrad \left(\frac{I \BB}{\sqrt{\mu\epsilon}}\right) + \sqrt{\mu\epsilon} \partial_0 \BE &= 0
\end{aligned}
\end{equation}
%
These can now be added
%
\begin{equation}\label{eqn:emVacWave:200}
\begin{aligned}
\left(\spacegrad + \sqrt{\mu\epsilon} \partial_0\right)\left(\BE + \frac{I\BB}{\sqrt{\mu\epsilon}}\right) = 0
\end{aligned}
\end{equation}
%
This allows for the one liner derivation of \eqnref{eqn:emVacWave:waveEquationsMedia} by premultiplying by the conjugate
operator \(-\spacegrad + \sqrt{\mu\epsilon} \partial_0\)
%
\begin{equation}\label{eqn:emVacWave:220}
\begin{aligned}
0
&=
\left(-\spacegrad + \sqrt{\mu\epsilon} \partial_0\right)
\left(\spacegrad + \sqrt{\mu\epsilon} \partial_0\right)
\left(\BE + \frac{I\BB}{\sqrt{\mu\epsilon}}\right) \\
&=
\left(-\spacegrad^2 + \frac{\mu\epsilon}{c^2} \partial_{tt}\right)
\left(\BE + \frac{I\BB}{\sqrt{\mu\epsilon}}\right)
\end{aligned}
\end{equation}
%
Using the same hint, and doing some rearrangement, we can write Jackson's equations (6.70) as
%
\begin{equation}\label{eqn:emVacWave:maxwellEquationMedia}
\begin{aligned}
\left(\spacegrad + \sqrt{\mu\epsilon} \partial_0\right)\left(\BE + \frac{I\BB}{\sqrt{\mu\epsilon}}\right) =
\frac{4\pi}{\epsilon}\left( \rho - \frac{\sqrt{\mu\epsilon}}{c}\BJ \right)
\end{aligned}
\end{equation}
%
%\EndArticle

   %
% Copyright � 2012 Peeter Joot.  All Rights Reserved.
% Licenced as described in the file LICENSE under the root directory of this GIT repository.
%

%
%
%\input{../peeter_prologue.tex}

\mychapter{Space time algebra solutions of the Maxwell equation for discrete frequencies}
\index{Maxwell equation!space-time algebra}
\label{chap:maxwellVacuum}
%\date{July 2, 2009 \(RCSfile: maxwellVacuum.tex,v \) Last \(Revision: 1.8 \) \(Date: 2009/08/06 09:35:17 \)}
%%\date{July 2, 2009}
%%\revisionInfo{\(RCSfile: maxwellVacuum.tex,v \) Last \(Revision: 1.8 \) \(Date: 2009/08/06 09:35:17 \)}
%\blogpage{http://sites.google.com/site/peeterjoot/math2009/maxwellVacuum.pdf}

\beginArtWithToc

\section{Motivation}

How to obtain solutions to Maxwell's equations in vacuum is well known.  The aim here is to explore the same problem starting with the Geometric Algebra (GA) formalism \citep{doran2003gap} of the Maxwell equation.

\begin{equation}\label{eqn:maxwellVacuum:maxwell}
\begin{aligned}
\grad F &= J/\epsilon_0 c \\
F &= \grad \wedge A = \BE + i c \BB
\end{aligned}
\end{equation}

A Fourier transformation attack on the equation should be possible, so let us see what falls out doing so.

\subsection{Fourier problem}

Picking an observer bias for the gradient by premultiplying with \(\gamma_0\) the vacuum equation for light can therefore also be written as

\begin{equation}\label{eqn:maxwellVacuum:20}
\begin{aligned}
0
&= \gamma_0 \grad F \\
&= \gamma_0 (\gamma^0 \partial_0 + \gamma^k \partial_k) F \\
&= (\partial_0 - \gamma^k \gamma_0 \partial_k) F \\
&= (\partial_0 + \sigma^k \partial_k) F \\
&= \left(\inv{c}\partial_t + \spacegrad \right) F \\
\end{aligned}
\end{equation}

A Fourier transformation of this equation produces

\begin{equation}\label{eqn:maxwellVacuum:40}
\begin{aligned}
0 &= \inv{c} \frac{\partial F}{\partial t}(\Bk,t) + \inv{(\sqrt{2\pi})^3} \int \sigma^m \partial_m F(\Bx,t) e^{-i \Bk \cdot \Bx} d^3 x
\end{aligned}
\end{equation}

and with a single integration by parts one has

\begin{equation}\label{eqn:maxwellVacuum:60}
\begin{aligned}
0
&= \inv{c} \frac{\partial F}{\partial t}(\Bk,t) - \inv{(\sqrt{2\pi})^3} \int \sigma^m F(\Bx,t) (-i k_m) e^{-i \Bk \cdot \Bx} d^3 x \\
&= \inv{c} \frac{\partial F}{\partial t}(\Bk,t) + \inv{(\sqrt{2\pi})^3} \int \Bk F(\Bx,t) i e^{-i \Bk \cdot \Bx} d^3 x \\
&= \inv{c} \frac{\partial F}{\partial t}(\Bk,t) + i \Bk \hat{F}(\Bk,t)
\end{aligned}
\end{equation}

The flexibility to employ the pseudoscalar as the imaginary \(i = \gamma_0 \gamma_1 \gamma_2 \gamma_3\) has been employed above, so it should be noted that pseudoscalar commutation with Dirac bivectors was implied above, but also that we do not have the flexibility to commute \(\Bk\) with \(F\).

Having done this, the problem to solve is now Maxwell's vacuum equation in the frequency domain

\begin{equation}\label{eqn:maxwellVacuum:80}
\begin{aligned}
\frac{\partial F}{\partial t}(\Bk,t) = -i c \Bk \hat{F}(\Bk,t)
\end{aligned}
\end{equation}

Introducing an angular frequency (spatial) bivector, and its vector dual

\begin{equation}\label{eqn:maxwellVacuum:100}
\begin{aligned}
\Omega &= -i c \Bk \\
\Bomega &= c \Bk
\end{aligned}
\end{equation}

This becomes

\begin{equation}\label{eqn:maxwellVacuum:MaxwellFreq}
\begin{aligned}
\hat{F}' = \Omega F
\end{aligned}
\end{equation}

With solution

\begin{equation}\label{eqn:maxwellVacuum:120}
\begin{aligned}
\hat{F} = e^{\Omega t} \hat{F}(\Bk,0)
\end{aligned}
\end{equation}

Differentiation with respect to time verifies that the ordering of the terms is correct and this does in fact solve \eqnref{eqn:maxwellVacuum:MaxwellFreq}.  This is something we have to be careful of due to the possibility of non-commuting variables.

Back substitution into the inverse transform now supplies the time evolution of the field given the initial time specification

\begin{equation}\label{eqn:maxwellVacuum:140}
\begin{aligned}
F(\Bx,t)
&= \inv{(\sqrt{2\pi})^3} \int e^{\Omega t} \hat{F}(\Bk,0) e^{i \Bk \cdot \Bx} d^3 k \\
&= \inv{(2\pi)^3} \int e^{\Omega t} \left( \int {F}(\Bx',0) e^{-i \Bk \cdot \Bx'} d^3 x' \right) e^{i \Bk \cdot \Bx} d^3 k
\end{aligned}
\end{equation}

Observe that Pseudoscalar exponentials commute with the field because \(i\) commutes with spatial vectors and itself

\begin{equation}\label{eqn:maxwellVacuum:160}
\begin{aligned}
F e^{i\theta}
&= (\BE + i c \BB) (C + iS) \\
&=
C (\BE + i c \BB)
+ S (\BE + i c \BB) i  \\
&=
C (\BE + i c \BB)
+ S i (\BE + i c \BB) \\
&=
e^{i\theta} F
\end{aligned}
\end{equation}

This allows the specifics of the initial time conditions to be suppressed

\begin{equation}\label{eqn:maxwellVacuum:180}
\begin{aligned}
F(\Bx,t) &= \int d^3 k e^{\Omega t} e^{i \Bk \cdot \Bx} \int \inv{(2\pi)^3} {F}(\Bx',0) e^{-i \Bk \cdot \Bx'}  d^3 x'
\end{aligned}
\end{equation}

The interior integral has the job of a weighting function over plane wave solutions, and this can be made explicit writing

\begin{equation}\label{eqn:maxwellVacuum:200}
\begin{aligned}
D(\Bk) &= \inv{(2\pi)^3} \int {F}(\Bx',0) e^{-i \Bk \cdot \Bx'}  d^3 x' \\
F(\Bx,t) &= \int e^{\Omega t} e^{i \Bk \cdot \Bx} D(\Bk) d^3 k
\end{aligned}
\end{equation}

Many assumptions have been made here, not the least of which was a requirement for the Fourier transform of a bivector valued function to be meaningful, and have an inverse.  It is therefore reasonable to verify that this weighted plane wave result is in fact a solution to the original Maxwell vacuum equation.  Differentiation verifies that things are okay so far

\begin{equation}\label{eqn:maxwellVacuum:220}
\begin{aligned}
\gamma_0 \grad F(\Bx,t)
&=
\left(\inv{c}\partial_t + \spacegrad \right)\int e^{\Omega t} e^{i \Bk \cdot \Bx} D(\Bk) d^3 k \\
&=
\int \left(\inv{c}\Omega e^{\Omega t} + \sigma^m e^{\Omega t} i k_m \right) e^{i \Bk \cdot \Bx} D(\Bk) d^3 k \\
&=
\int \left(\inv{c}(-i \Bk c) + i \Bk \right) e^{\Omega t} e^{i \Bk \cdot \Bx} D(\Bk) d^3 k \\
&= 0
\qedmarker
\end{aligned}
\end{equation}

\subsection{Discretizing and grade restrictions}

The fact that it the integral has zero gradient does not mean that it is a bivector, so there must also be at least also be restrictions on the grades of \(D(\Bk)\).

To simplify discussion, let us discretize the integral writing

\begin{equation}\label{eqn:maxwellVacuum:240}
\begin{aligned}
D(\Bk') = D_\Bk \delta^3 (\Bk - \Bk')
\end{aligned}
\end{equation}

So we have

\begin{equation}\label{eqn:maxwellVacuum:260}
\begin{aligned}
F(\Bx,t)
&= \int e^{\Omega t} e^{i \Bk' \cdot \Bx} D(\Bk') d^3 k' \\
&= \int e^{\Omega t} e^{i \Bk' \cdot \Bx} D_\Bk \delta^3(\Bk - \Bk') d^3 k' \\
\end{aligned}
\end{equation}

This produces something planewave-ish

\begin{equation}\label{eqn:maxwellVacuum:planewaveish}
\begin{aligned}
F(\Bx,t) &= e^{\Omega t} e^{i \Bk \cdot \Bx} D_\Bk
\end{aligned}
\end{equation}

Observe that at \(t=0\) we have

\begin{equation}\label{eqn:maxwellVacuum:280}
\begin{aligned}
F(\Bx,0)
&= e^{i \Bk \cdot \Bx} D_\Bk  \\
&= (\cos (\Bk \cdot \Bx) + i \sin(\Bk \cdot \Bx)) D_\Bk  \\
\end{aligned}
\end{equation}

There is therefore a requirement for \(D_\Bk\) to be either a spatial vector or its dual, a spatial bivector.  For example taking \(D_k\) to be a spatial vector we can then identify the electric and magnetic components of the field

\begin{equation}\label{eqn:maxwellVacuum:300}
\begin{aligned}
\BE(\Bx,0) &= \cos (\Bk \cdot \Bx) D_\Bk \\
c \BB(\Bx,0) &= \sin (\Bk \cdot \Bx) D_\Bk
\end{aligned}
\end{equation}

and if \(D_k\) is taken to be a spatial bivector, this pair of identifications would be inverted.

Considering \eqnref{eqn:maxwellVacuum:planewaveish} at \(\Bx=0\), we have

\begin{equation}\label{eqn:maxwellVacuum:320}
\begin{aligned}
F(0, t)
&= e^{\Omega t} D_\Bk \\
&= (\cos(\Abs{\Omega} t) + \hat{\Omega} \sin(\Abs{\Omega} t)) D_\Bk \\
&= (\cos(\Abs{\Omega} t) - i \hat{\Bk} \sin(\Abs{\Omega} t)) D_\Bk \\
\end{aligned}
\end{equation}

If \(D_\Bk\) is first assumed to be a spatial vector, then \(F\) would have a pseudoscalar component if \(D_\Bk\) has any component parallel to \(\hat{\Bk}\).

\begin{equation}\label{eqn:maxwellVacuum:commutationRequirementVector}
\begin{aligned}
D_\Bk \in \span\{\sigma^m\} \implies D_\Bk \cdot \hat{\Bk} = 0
\end{aligned}
\end{equation}
\begin{equation}\label{eqn:maxwellVacuum:commutationRequirementBiVector}
\begin{aligned}
D_\Bk \in \span\{\sigma^a \wedge \sigma^b\} \implies D_\Bk \cdot (i\hat{\Bk}) = 0
\end{aligned}
\end{equation}

Since we can convert between the spatial vector and bivector cases using a duality transformation, there may not appear to be any loss of generality imposing a spatial vector restriction on \(D_\Bk\), at least in this current free case.  However, an attempt to do so leads to trouble.  In particular, this leads to collinear electric and magnetic fields, and thus the odd seeming condition where the field energy density is non-zero but the field momentum density (Poynting vector \(\BP \propto \BE \cross \BB\)) is zero.  In retrospect being forced down the path of including both grades is not unreasonable, especially since this gives \(D_\Bk\) precisely the form of the field itself \(F = \BE + i c \BB\).

\section{Electric and Magnetic field split}

With the basic form of the Maxwell vacuum solution determined, we are now ready to start extracting information from the solution and making comparisons with the more familiar vector form.  To start doing the phasor form of the fundamental solution can be expanded explicitly in terms of two arbitrary spatial parametrization vectors \(\BE_\Bk\) and \(\BB_\Bk\).

\begin{equation}\label{eqn:maxwellVacuum:phasor}
\begin{aligned}
F &= e^{-i\Bomega t} e^{i \Bk \cdot \Bx} (\BE_\Bk + i c \BB_\Bk)
\end{aligned}
\end{equation}

Whether these parametrization vectors have any relation to electric and magnetic fields respectively will have to be determined, but making that assumption for now to label these uniquely does not seem unreasonable.

From \eqnref{eqn:maxwellVacuum:phasor} we can compute the electric and magnetic fields by the conjugate relations \eqnref{eqn:maxwellVacuum:conjuagateSplit}.  Our conjugate is

\begin{equation}\label{eqn:maxwellVacuum:340}
\begin{aligned}
F^\dagger
&= (\BE_\Bk - i c \BB_\Bk) e^{-i \Bk \cdot \Bx} e^{i\Bomega t} \\
&=
e^{-i\Bomega t}
e^{-i \Bk \cdot \Bx}
(\BE_\Bk - i c \BB_\Bk)
\end{aligned}
\end{equation}

Thus for the electric field

\begin{equation}\label{eqn:maxwellVacuum:360}
\begin{aligned}
F + F^\dagger
&=
e^{-i\Bomega t} \left(
 e^{i \Bk \cdot \Bx} (\BE_\Bk + i c \BB_\Bk)
+e^{-i \Bk \cdot \Bx} (\BE_\Bk - i c \BB_\Bk)
\right) \\
&=
e^{-i\Bomega t} \left(
 2 \cos(\Bk \cdot \Bx) \BE_\Bk
+ i c (2 i) \sin(\Bk \cdot \Bx) \BB_\Bk
\right) \\
&=
2 \cos(\omega t) \left(
 \cos(\Bk \cdot \Bx) \BE_\Bk
- c \sin(\Bk \cdot \Bx) \BB_\Bk
\right) \\
&+ 2
\sin(\omega t)
\kcap \cross
\left(
 \cos(\Bk \cdot \Bx) \BE_\Bk
- c \sin(\Bk \cdot \Bx) \BB_\Bk
\right) \\
\end{aligned}
\end{equation}

So for the electric field \(\BE = \inv{2}(F + F^\dagger)\) we have

\begin{equation}\label{eqn:maxwellVacuum:electricSplit}
\begin{aligned}
\BE &=
\left( \cos(\omega t) + \sin(\omega t) \kcap \cross \right)
\left(
 \cos(\Bk \cdot \Bx) \BE_\Bk
- c \sin(\Bk \cdot \Bx) \BB_\Bk
\right)
\end{aligned}
\end{equation}

Similarly for the magnetic field we have
\begin{equation}\label{eqn:maxwellVacuum:380}
\begin{aligned}
F - F^\dagger
&=
e^{-i\Bomega t} \left(
 e^{i \Bk \cdot \Bx} (\BE_\Bk + i c \BB_\Bk)
-e^{-i \Bk \cdot \Bx} (\BE_\Bk - i c \BB_\Bk)
\right) \\
&=
e^{-i\Bomega t} \left(
 2 i \sin(\Bk \cdot \Bx) \BE_\Bk
+ 2 i c \cos(\Bk \cdot \Bx) \BB_\Bk
\right) \\
\end{aligned}
\end{equation}

This gives \(c \BB = \inv{2i}(F - F^\dagger)\) we have

\begin{equation}\label{eqn:maxwellVacuum:magneticSplit}
\begin{aligned}
c \BB &=
\left( \cos(\omega t) + \sin(\omega t) \kcap \cross \right)
\left(
 \sin(\Bk \cdot \Bx) \BE_\Bk
+ c \cos(\Bk \cdot \Bx) \BB_\Bk
\right)
\end{aligned}
\end{equation}

Observe that the action of the time dependent phasor has been expressed, somewhat abusively and sneakily, in a scalar plus cross product operator form.  The end result, when applied to a vector perpendicular to \(\kcap\), is still a vector

\begin{equation}\label{eqn:maxwellVacuum:400}
\begin{aligned}
e^{-i\Bomega t} \Ba
&=
\left( \cos(\omega t) + \sin(\omega t) \kcap \cross \right) \Ba
\end{aligned}
\end{equation}

Also observe that the Hermitian conjugate split of the total field bivector \(F\) produces vectors \(\BE\) and \(\BB\), not phasors.  There is no further need to take real or imaginary parts nor treat the phasor \eqnref{eqn:maxwellVacuum:phasor} as an artificial mathematical construct used for convenience only.

With \(\BE \cdot \kcap = \BB \cdot \kcap = 0\), we have here what Jackson (\citep{jackson1975cew}, ch7), calls a transverse wave.

\subsection{Polar Form}

Suppose an explicit polar form is introduced for the plane vectors \(\BE_\Bk\), and \(\BB_\Bk\).  Let

\begin{equation}\label{eqn:maxwellVacuum:420}
\begin{aligned}
\BE_\Bk &= E {\hat{\BE}_k} \\
\BB_\Bk &= B {\hat{\BE}_k} e^{i\kcap \theta}
\end{aligned}
\end{equation}

Then for the field we have

\begin{equation}\label{eqn:maxwellVacuum:phasorPolar}
\begin{aligned}
F &= e^{-i\Bomega t} e^{i \Bk \cdot \Bx} (E + i c B e^{-i\kcap \theta}) \hat{\BE}_k
\end{aligned}
\end{equation}

For the conjugate
\begin{equation}\label{eqn:maxwellVacuum:440}
\begin{aligned}
F^\dagger
&=
\hat{\BE}_k
(E - i c B e^{i\kcap \theta})
e^{-i \Bk \cdot \Bx}
e^{i\Bomega t} \\
&=
e^{-i\Bomega t} e^{-i \Bk \cdot \Bx} (E - i c B e^{-i\kcap \theta}) \hat{\BE}_k
\end{aligned}
\end{equation}

So, in the polar form we have for the electric, and magnetic fields

\begin{equation}\label{eqn:maxwellVacuum:fieldsPolar}
\begin{aligned}
\BE &= e^{-i\Bomega t} (E \cos(\Bk \cdot \Bx) - c B \sin(\Bk \cdot \Bx) e^{-i \kcap\theta}) \hat{\BE}_k \\
c \BB &= e^{-i\Bomega t} (E \sin(\Bk \cdot \Bx) + c B \cos(\Bk \cdot \Bx) e^{-i \kcap\theta}) \hat{\BE}_k
\end{aligned}
\end{equation}

Observe when \(\theta\) is an integer multiple of \(\pi\), \(\BE\) and \(\BB\) are colinear, having the zero Poynting vector mentioned previously.
Now, for arbitrary \(\theta\) it does not appear that there is any inherent perpendicularity between the electric and magnetic fields.  It is common
to read of light being the propagation of perpendicular fields, both perpendicular to the propagation direction.  We have perpendicularity to the
propagation direction by virtue of requiring that the field be a (Dirac) bivector, but it does not look like the solution requires any inherent perpendicularity for the field components.  It appears that a normal triplet of field vectors and propagation directions must actually be a special case.
Intuition says that this freedom to pick different magnitude or angle between \(\BE_\Bk\) and \(\BB_\Bk\) in the plane perpendicular to the transmission direction may correspond to different mixes of linear, circular, and elliptic polarization, but this has to be confirmed.

Working towards confirming (or disproving) this intuition, lets find the constraints on the fields that lead to normal electric and magnetic fields.  This should follow by taking dot products

\begin{equation}\label{eqn:maxwellVacuum:460}
\begin{aligned}
\BE \cdot \BB c
&=
%\gpgradezero{
\left\langle{
e^{-i\Bomega t} (E \cos(\Bk \cdot \Bx) - c B \sin(\Bk \cdot \Bx) e^{-i \kcap\theta}) \hat{\BE}_k
\hat{\BE}_k
e^{i\Bomega t} (E \sin(\Bk \cdot \Bx) + c B \cos(\Bk \cdot \Bx) e^{i \kcap\theta})
%} \\
}\right\rangle \\
&=
%\gpgradezero{
\left\langle{
(E \cos(\Bk \cdot \Bx) - c B \sin(\Bk \cdot \Bx) e^{-i \kcap\theta})
(E \sin(\Bk \cdot \Bx) + c B \cos(\Bk \cdot \Bx) e^{i \kcap\theta})
%} \\
}\right\rangle \\
&=
(E^2 - c^2 B^2) \cos(\Bk \cdot \Bx) \sin(\Bk \cdot \Bx)
+ c E B
%\gpgradezero{
\left\langle{
\cos^2(\Bk \cdot \Bx) e^{i \kcap \theta}
-\sin^2(\Bk \cdot \Bx) e^{-i \kcap \theta}
%} \\
}\right\rangle \\
&=
(E^2 - c^2 B^2) \cos(\Bk \cdot \Bx) \sin(\Bk \cdot \Bx)
+ c E B \cos(\theta) ( \cos^2(\Bk \cdot \Bx) -\sin^2(\Bk \cdot \Bx) ) \\
&=
(E^2 - c^2 B^2) \cos(\Bk \cdot \Bx) \sin(\Bk \cdot \Bx)
+ c E B \cos(\theta) ( \cos^2(\Bk \cdot \Bx) -\sin^2(\Bk \cdot \Bx) ) \\
&=
\inv{2} (E^2 - c^2 B^2) \sin(2 \Bk \cdot \Bx)
+ c E B \cos(\theta) \cos(2 \Bk \cdot \Bx) \\
\end{aligned}
\end{equation}

The only way this can be zero for any \(\Bx\) is if the left and right terms are separately zero, which means

\begin{equation}\label{eqn:maxwellVacuum:480}
\begin{aligned}
\Abs{\BE_k} &= c \Abs{\BB_k} \\
\theta &= \frac{\pi}{2} + n \pi
\end{aligned}
\end{equation}

This simplifies the phasor considerably, leaving

\begin{equation}\label{eqn:maxwellVacuum:500}
\begin{aligned}
E + i c B e^{-i\kcap \theta}
&=
E(1 + i (\mp i\kcap )) \\
&=
E(1 \pm \kcap)
\end{aligned}
\end{equation}

So the field is just

\begin{equation}\label{eqn:maxwellVacuum:520}
\begin{aligned}
F = e^{-i \Bomega t} e^{i \Bk \cdot \Bx} (1 \pm \kcap) \BE_\Bk
\end{aligned}
\end{equation}

Using this, and some regrouping, a calculation of the field components yields

\begin{equation}\label{eqn:maxwellVacuum:540}
\begin{aligned}
\BE &= e^{i \kcap( \pm \Bk \cdot \Bx -\omega t )} \BE_\Bk \\
c \BB &= \pm e^{i \kcap( \pm \Bk \cdot \Bx -\omega t )} i \Bk \BE_\Bk
\end{aligned}
\end{equation}

Observe that \(i\Bk\) rotates any vector in the plane perpendicular to \(\kcap\) by 90 degrees, so we have here \(c \BB = \pm \kcap \cross \BE\).  This is consistent with the transverse wave restriction (7.11) of Jackson \citep{jackson1975cew}, where he says, the ``curl equations provide a further restriction, namely'', and

\begin{equation}\label{eqn:fooX}
\begin{aligned}
\calB = \sqrt{\mu\epsilon} \Bn \cross \calE
\end{aligned}
\end{equation}

He works in explicit complex phasor form and CGS units.  He also allows \(\Bn\) to be complex.  With real \(\Bk\), and no \(\BE \cdot \BB = 0\) constraint, it appears that we cannot have such a simple coupling between the field components?  Is it possible that allowing \(\Bk\) to be complex allows this cross product coupling constraint on the fields without the explicit 90 degree phase difference between the electric and magnetic fields?

\section{Energy and momentum for the phasor}

To calculate the field energy density we can work with the two fields of equations \eqnref{eqn:maxwellVacuum:fieldsPolar}, or work with the phasor \eqnref{eqn:maxwellVacuum:phasor} directly.  From the phasor and the energy-momentum four vector \eqnref{eqn:maxwellVacuum:emFourVect} we have for the energy density

\begin{equation}\label{eqn:maxwellVacuum:560}
\begin{aligned}
U &= T(\gamma_0) \cdot \gamma_0 \\
&= \frac{-\epsilon_0}{2}\gpgradezero{ F \gamma_0 F \gamma_0 } \\
&= \frac{-\epsilon_0}{2}
%\gpgradezero{
\left\langle{
e^{-i\Bomega t} e^{i \Bk \cdot \Bx} (\BE_\Bk + i c \BB_\Bk) \gamma_0 e^{-i\Bomega t} e^{i \Bk \cdot \Bx} (\BE_\Bk + i c \BB_\Bk) \gamma_0
%} \\
}\right\rangle \\
&= \frac{-\epsilon_0}{2}
%\gpgradezero{
\left\langle{
e^{-i\Bomega t} e^{i \Bk \cdot \Bx} (\BE_\Bk + i c \BB_\Bk) (\gamma_0)^2 e^{-i\Bomega t} e^{-i \Bk \cdot \Bx} (-\BE_\Bk + i c \BB_\Bk)
%} \\
}\right\rangle \\
&= \frac{-\epsilon_0}{2}
%\gpgradezero{
\left\langle{
e^{-i\Bomega t} (\BE_\Bk + i c \BB_\Bk) e^{-i\Bomega t} (-\BE_\Bk + i c \BB_\Bk)
%} \\
}\right\rangle \\
&= \frac{\epsilon_0}{2}\gpgradezero{ (\BE_\Bk + i c \BB_\Bk) (\BE_\Bk - i c \BB_\Bk) } \\
&=
\frac{\epsilon_0}{2} \left( (\BE_k)^2 + c^2 (\BB_\Bk)^2\right) + {c \epsilon_0} \gpgradezero{ i \BE_\Bk \wedge \BB_\Bk } \\
&=
\frac{\epsilon_0}{2} \left( (\BE_k)^2 + c^2 (\BB_\Bk)^2\right) + {c \epsilon_0} \gpgradezero{ \BB_\Bk \cross \BE_\Bk } \\
\end{aligned}
\end{equation}

Quite anticlimactically we have for the energy the sum of the energies associated with the parametrization constants, lending some justification for the initial choice to label these as electric and magnetic fields

\begin{equation}\label{eqn:maxwellVacuum:580}
\begin{aligned}
U = \frac{\epsilon_0}{2} \left( (\BE_k)^2 + c^2 (\BB_\Bk)^2\right)
\end{aligned}
\end{equation}

For the momentum, we want the difference of \(F F^\dagger\), and \(F^\dagger F\)

\begin{equation}\label{eqn:maxwellVacuum:600}
\begin{aligned}
F F^\dagger
&= e^{-i\Bomega t} e^{i \Bk \cdot \Bx} (\BE_\Bk + i c \BB_\Bk) (\BE_\Bk - i c \BB_\Bk) e^{-i \Bk \cdot \Bx} e^{i\Bomega t}  \\
&= (\BE_\Bk + i c \BB_\Bk) (\BE_\Bk - i c \BB_\Bk) \\
&= (\BE_\Bk)^2 + c^2 (\BB_\Bk)^2 - 2 c \BB_\Bk \cross \BE_\Bk
\end{aligned}
\end{equation}

\begin{equation}\label{eqn:maxwellVacuum:620}
\begin{aligned}
F F^\dagger
&= (\BE_\Bk - i c \BB_\Bk) e^{-i \Bk \cdot \Bx} e^{i\Bomega t}  e^{-i\Bomega t} e^{i \Bk \cdot \Bx} (\BE_\Bk + i c \BB_\Bk)  \\
&= (\BE_\Bk - i c \BB_\Bk) (\BE_\Bk + i c \BB_\Bk) \\
&= (\BE_\Bk)^2 + c^2 (\BB_\Bk)^2 + 2 c \BB_\Bk \cross \BE_\Bk
\end{aligned}
\end{equation}

So we have for the momentum, also anticlimactically

\begin{equation}\label{eqn:maxwellVacuum:640}
\begin{aligned}
\BP = \inv{c} T(\gamma_0) \wedge \gamma_0 = \epsilon_0 \BE_\Bk \cross \BB_\Bk
\end{aligned}
\end{equation}

\section{Followup}

Well, that is enough for one day.  Understanding how to express circular and elliptic polarization is one of the logical next steps.  I seem to recall from Susskind's QM lectures that these can be considered superpositions of linearly polarized waves, so examining a sum of two co-directionally propagating fields would seem to be in order.  Also there ought to be a more natural way to express the perpendicularity requirement for the field and the propagation direction.  The fact that the field components and propagation direction when all multiplied is proportional to the spatial pseudoscalar can probably be utilized to tidy this up and also produce a form that allows for simpler summation of fields in different propagation directions.  It also seems reasonable to consider a planar Fourier decomposition of the field components, perhaps framing the superposition of multiple fields in that context.

Reconsilation of the Jackson's (7.11) restriction for perpendicularity of the fields noted above has not been done.  If such a restriction is required with an explicit dot and cross product split of Maxwell's equation, it would make sense to also have this required of a GA based solution.  Is this just a conquense of the differences between his explicit phasor representation, and this geometric approach where the phasor has an explicit representation in terms of the transverse plane?

\section{Appendix.  Background details}

\subsection{Conjugate split}

The Hermitian conjugate is defined as

\begin{equation}\label{eqn:maxwellVacuum:660}
\begin{aligned}
A^\dagger = \gamma_0 \tilde{A} \gamma_0
\end{aligned}
\end{equation}

The conjugate action on a multivector product is straightforward to calculate

\begin{equation}\label{eqn:maxwellVacuum:680}
\begin{aligned}
(A B)^\dagger
&= \gamma_0 (A B)^{\tilde{}} \gamma_0 \\
&= \gamma_0 \tilde{B} \tilde{A} \gamma_0 \\
&= \gamma_0 \tilde{B} {\gamma_0}^2 \tilde{A} \gamma_0 \\
&= B^\dagger A^\dagger
\end{aligned}
\end{equation}

For a spatial vector Hermitian conjugation leaves the vector unaltered

\begin{equation}\label{eqn:maxwellVacuum:700}
\begin{aligned}
\Ba
&= \gamma_0 (\gamma_k \gamma_0)^{\tilde{}} a^k \gamma_0 \\
&= \gamma_0 (\gamma_0 \gamma_k) a^k \gamma_0 \\
&= \gamma_k a^k \gamma_0 \\
&= \Ba
\end{aligned}
\end{equation}

But the pseudoscalar is negated

\begin{equation}\label{eqn:maxwellVacuum:720}
\begin{aligned}
i^\dagger
&=
\gamma_0 \tilde{i} \gamma_0 \\
&=
\gamma_0 i \gamma_0 \\
&=
-\gamma_0 \gamma_0 i \\
&=
- i \\
\end{aligned}
\end{equation}

This allows for a split by conjugation of the field into its electric and magnetic field components.

\begin{equation}\label{eqn:maxwellVacuum:740}
\begin{aligned}
F^\dagger
&= -\gamma_0 ( \BE + i c \BB) \gamma_0 \\
&= -\gamma_0^2 ( -\BE + i c \BB) \\
&= \BE - i c\BB \\
\end{aligned}
\end{equation}

So we have

\begin{equation}\label{eqn:maxwellVacuum:conjuagateSplit}
\begin{aligned}
\BE &= \inv{2}(F + F^\dagger) \\
c \BB &= \inv{2i}(F - F^\dagger)
\end{aligned}
\end{equation}

\subsection{Field Energy Momentum density four vector}

In the GA formalism the energy momentum tensor is

\begin{equation}\label{eqn:maxwellVacuum:760}
\begin{aligned}
T(a) = \frac{\epsilon_0}{2} F a \tilde{F}
\end{aligned}
\end{equation}

It is not necessarily obvious this bivector-vector-bivector product construction is even a vector quantity.  Expansion of \(T(\gamma_0)\) in terms of the electric and magnetic fields demonstrates this vectorial nature.

\begin{equation}\label{eqn:maxwellVacuum:780}
\begin{aligned}
F \gamma_0 \tilde{F}
&=
-(\BE + i c \BB) \gamma_0 (\BE + i c \BB) \\
&=
-\gamma_0 (-\BE + i c \BB) (\BE + i c \BB) \\
&=
-\gamma_0 (-\BE^2 - c^2 \BB^2 + i c (\BB \BE - \BE \BB) ) \\
&=
\gamma_0 (\BE^2 + c^2 \BB^2) - 2 \gamma_0 i c (\BB \wedge \BE) ) \\
&=
\gamma_0 (\BE^2 + c^2 \BB^2) + 2 \gamma_0 c (\BB \cross \BE) \\
&=
\gamma_0 (\BE^2 + c^2 \BB^2) + 2 \gamma_0 c \gamma_k \gamma_0 (\BB \cross \BE)^k \\
&=
\gamma_0 (\BE^2 + c^2 \BB^2) + 2 \gamma_k (\BE \cross (c \BB))^k \\
\end{aligned}
\end{equation}

Therefore, \(T(\gamma_0)\), the energy momentum tensor biased towards a particular observer frame \(\gamma_0\)
is

\begin{equation}\label{eqn:maxwellVacuum:emFourVect}
\begin{aligned}
T(\gamma_0)
&=
\gamma_0 \frac{\epsilon_0}{2} (\BE^2 + c^2 \BB^2) + \gamma_k \epsilon_0 (\BE \cross (c \BB))^k
\end{aligned}
\end{equation}

Recognizable here in the components \(T(\gamma_0)\) are the field energy density and momentum density.  In particular the energy density can be obtained by dotting with \(\gamma_0\), whereas the (spatial vector) momentum by wedging with \(\gamma_0\).

These are

\begin{equation}\label{eqn:maxwellVacuum:800}
\begin{aligned}
U \equiv T(\gamma_0) \cdot \gamma_0 &= \frac{1}{2} \left( \epsilon_0 \BE^2 + \inv{\mu_0} \BB^2 \right) \\
c \BP \equiv T(\gamma_0) \wedge \gamma_0 &= \inv{\mu_0} \BE \cross \BB
\end{aligned}
\end{equation}

In terms of the combined field these are

\begin{equation}\label{eqn:maxwellVacuum:820}
\begin{aligned}
U &= \frac{-\epsilon_0}{2}( F \gamma_0 F \gamma_0 + \gamma_0 F \gamma_0 F) \\
c \BP &= \frac{-\epsilon_0}{2}( F \gamma_0 F \gamma_0 - \gamma_0 F \gamma_0 F)
\end{aligned}
\end{equation}

Summarizing with the Hermitian conjugate

\begin{equation}\label{eqn:maxwellVacuum:840}
\begin{aligned}
U &= \frac{\epsilon_0}{2}( F F^\dagger + F^\dagger F) \\
c \BP &= \frac{\epsilon_0}{2}( F F^\dagger - F^\dagger F)
\end{aligned}
\end{equation}

\subsubsection{Divergence}

Calculation of the divergence produces the components of the Lorentz force densities

\begin{equation}\label{eqn:maxwellVacuum:860}
\begin{aligned}
\grad \cdot T(a)
&= \frac{\epsilon_0}{2} \gpgradezero{ \grad (F a F) } \\
&= \frac{\epsilon_0}{2} \gpgradezero{ (\grad F) a F + (F \grad) F a } \\
\end{aligned}
\end{equation}

Here the gradient is used implicitly in bidirectional form, where the direction is implied by context.  From Maxwell's equation we have

\begin{equation}\label{eqn:maxwellVacuum:880}
\begin{aligned}
J/\epsilon_0 c
&= (\grad F)^{\tilde{}} \\
&= (\tilde{F} \tilde{\grad}) \\
&= -(F \grad)
\end{aligned}
\end{equation}

and continuing the expansion

\begin{equation}\label{eqn:maxwellVacuum:900}
\begin{aligned}
\grad \cdot T(a)
&= \frac{1}{2c} \gpgradezero{ J a F - J F a } \\
&= \frac{1}{2c} \gpgradezero{ F J a - J F a } \\
&= \frac{1}{2c} \gpgradezero{ (F J - J F) a } \\
\end{aligned}
\end{equation}

Wrapping up, the divergence and the adjoint of the energy momentum tensor are

\begin{equation}\label{eqn:maxwellVacuum:920}
\begin{aligned}
\grad \cdot T(a) &= \frac{1}{c} (F \cdot J) \cdot a \\
\overbar{T}(\grad) &= F \cdot J/c
\end{aligned}
\end{equation}

When integrated over a volume, the quantities \(F \cdot J/c\) are the components of the RHS of the Lorentz force equation \(\dot{p} = q F \cdot v/c\).

%\EndArticle

   %
% Copyright � 2012 Peeter Joot.  All Rights Reserved.
% Licenced as described in the file LICENSE under the root directory of this GIT repository.
%

%
%
%\input{../peeter_prologue.tex}

\chapter{Transverse electric and magnetic fields}
\index{electric field!transverse}
\index{magnetic field!transverse}
\label{chap:transverseField}

%\blogpage{http://sites.google.com/site/peeterjoot/math2009/transverseField.pdf}

%%\date{July 30, 2009}
%%\revisionInfo{\(RCSfile: transverseField.tex,v \) Last \(Revision: 1.12 \) \(Date: 2009/08/06 09:35:17 \)}

%\date{July 30, 2009.  \(RCSfile: transverseField.tex,v \) Last \(Revision: 1.12 \) \(Date: 2009/08/06 09:35:17 \)}

\beginArtWithToc

\section{Motivation}

In Eli's \href{http://behindtheguesses.blogspot.com/2009/07/transverse-electric-and-magnetic-fields.html}{Transverse Electric and Magnetic Fields in a Conducting Waveguide} blog entry he works through the algebra calculating the transverse components, the perpendicular to the propagation direction components.

This should be possible using Geometric Algebra too, and trying this made for a good exercise.

\section{Setup}

The starting point can be the same, the source free Maxwell's equations.  Writing \(\partial_0 = (1/c) \partial/{\partial t}\), we have

\begin{equation}\label{eqn:transverseField:blah1}
\begin{aligned}
\spacegrad \cdot \BE &= 0 \\
\spacegrad \cdot \BB &= 0 \\
\spacegrad \cross \BE &= - \partial_0 \BB \\
\spacegrad \cross \BB &= \mu \epsilon \partial_0 \BE
\end{aligned}
\end{equation}

Multiplication of the last two equations by the spatial pseudoscalar \(I\), and using \(I \Ba \cross \Bb = \Ba \wedge \Bb\), the curl equations can be written in their dual bivector form

\begin{equation}\label{eqn:transverseField:blah2}
\begin{aligned}
\spacegrad \wedge \BE &= - \partial_0 I \BB \\
\spacegrad \wedge \BB &= \mu \epsilon \partial_0 I \BE
\end{aligned}
\end{equation}

Now adding the dot and curl equations using \(\Ba \Bb = \Ba \cdot \Bb + \Ba \wedge \Bb\) eliminates the cross products

\begin{equation}\label{eqn:transverseField:twoEquations}
\begin{aligned}
\spacegrad \BE &= - \partial_0 I \BB \\
\spacegrad \BB &= \mu \epsilon \partial_0 I \BE
\end{aligned}
\end{equation}

These can be further merged without any loss, into the GA first order equation for Maxwell's equation in \textAndIndex{cgs} units

\begin{equation}\label{eqn:transverseField:maxwell}
\begin{aligned}
\left(\spacegrad + \frac{\sqrt{\mu\epsilon}}{c}\partial_t\right) \left(\BE + \frac{I\BB}{\sqrt{\mu\epsilon}} \right) = 0.
\end{aligned}
\end{equation}

We are really after solutions to the total multivector field \(F = \BE + I \BB/\sqrt{\mu\epsilon}\).  For this problem where separate electric and magnetic field components are desired, working from \eqnref{eqn:transverseField:twoEquations} is perhaps what we want?

Following Eli and Jackson, write \(\spacegrad = \spacegrad_t + \zcap \partial_z\), and

\begin{equation}\label{eqn:transverseField:blah3}
\begin{aligned}
\BE(x,y,z,t) &= \BE(x,y) e^{\pm i k z - i \omega t} \\
\BB(x,y,z,t) &= \BB(x,y) e^{\pm i k z - i \omega t}
\end{aligned}
\end{equation}

Evaluating the \(z\) and \(t\) partials we have
\begin{equation}\label{eqn:transverseField:blah4}
\begin{aligned}
(\spacegrad_t \pm i k \zcap) \BE(x,y) &= \frac{i\omega}{c} I \BB(x,y) \\
(\spacegrad_t \pm i k \zcap) \BB(x,y) &= -\mu \epsilon \frac{i\omega}{c} I \BE(x,y)
\end{aligned}
\end{equation}

For the remainder of these notes, the explicit \((x,y)\) dependence will be assumed for \(\BE\) and \(\BB\).

An obvious thing to try with these equations is just substitute one into the other.  If that is done we get the pair of second order harmonic equations

\begin{equation}\label{eqn:transverseField:blah5}
\begin{aligned}
{\spacegrad_t}^2
\begin{pmatrix}\BE \\ \BB \end{pmatrix}
= \left( k^2 - \mu \epsilon \frac{\omega^2}{c^2} \right)
\begin{pmatrix}\BE \\ \BB \end{pmatrix}
\end{aligned}
\end{equation}

One could consider the problem solved here.  Separately equating both sides of this equation to zero, we have the \(k^2 = \mu\epsilon \omega^2/c^2\) constraint on the wave number and angular velocity, and the second order Laplacian on the left hand side is solved by the real or imaginary parts of any analytic function.  Especially when one considers that we are after a multivector field that of intrinsic complex nature.

However, that is not really what we want as a solution.  Doing the same on the unified Maxwell equation \eqnref{eqn:transverseField:maxwell}, we have

\begin{equation}\label{eqn:transverseField:max2}
\begin{aligned}
\left(\spacegrad_t \pm i k \zcap - \sqrt{\mu\epsilon}\frac{i\omega}{c}\right) \left(\BE + \frac{I\BB}{\sqrt{\mu\epsilon}} \right) = 0
\end{aligned}
\end{equation}

Selecting scalar, vector, bivector and trivector grades of this equation produces the following respective relations between the various components

\begin{equation}\label{eqn:transverseField:blah6}
\begin{aligned}
0 = \gpgradezero{\cdots} &= \spacegrad_t \cdot \BE \pm i k \zcap \cdot \BE \\
0 = \gpgradeone{\cdots} &= I \spacegrad_t \wedge \BB/\sqrt{\mu\epsilon} \pm i I k \zcap \wedge \BB/\sqrt{\mu\epsilon} - i \sqrt{\mu\epsilon}\frac{\omega}{c} \BE \\
0 = \gpgradetwo{\cdots} &= \spacegrad_t \wedge \BE \pm i k \zcap \wedge \BE - i \frac{\omega}{c} I \BB \\
0 = \gpgradethree{\cdots} &= I \spacegrad_t \cdot \BB/\sqrt{\mu\epsilon} \pm i I k \zcap \cdot \BB/\sqrt{\mu\epsilon}
\end{aligned}
\end{equation}

From the scalar and pseudoscalar grades we have the propagation components in terms of the transverse ones

\begin{equation}\label{eqn:transverseField:blah7}
\begin{aligned}
E_z &= \frac{\pm i}{k} \spacegrad_t \cdot \BE_t \\
B_z &= \frac{\pm i}{k} \spacegrad_t \cdot \BB_t
\end{aligned}
\end{equation}

But this is the opposite of the relations that we are after.  On the other hand from the vector and bivector grades we have

\begin{equation}\label{eqn:transverseField:messy}
\begin{aligned}
i \frac{\omega}{c} \BE &= -\inv{\mu\epsilon}\left(\spacegrad_t \cross \BB_z \pm i k \zcap \cross \BB_t\right) \\
i \frac{\omega}{c} \BB &= \spacegrad_t \cross \BE_z \pm i k \zcap \cross \BE_t
\end{aligned}
\end{equation}

\section{A clue from the final result}

From \eqnref{eqn:transverseField:messy} and a lot of messy algebra we should be able to get the transverse equations.  Is there a slicker way?  The end result that Eli obtained suggests a path.  That result was

\begin{equation}\label{eqn:transverseField:blah8}
\begin{aligned}
\BE_t = \frac{i}{\mu\epsilon \frac{\omega^2}{c^2} - k^2} \left( \pm k \spacegrad_t E_z - \frac{\omega}{c} \zcap \cross \spacegrad_t B_z \right)
\end{aligned}
\end{equation}

The numerator looks like it can be factored, and after a bit of playing around a suitable factorization can be obtained:

\begin{equation}\label{eqn:transverseField:30}
\begin{aligned}
\gpgradeone{ \left( \pm k + \frac{\omega}{c} \zcap \right) \spacegrad_t \zcap \left( \BE_z + I \BB_z \right) }
&=
\gpgradeone{ \left( \pm k + \frac{\omega}{c} \zcap \right) \spacegrad_t \left( E_z + I B_z \right) } \\
&=
\pm k \spacegrad E_z + \frac{\omega}{c} \gpgradeone{ I \zcap \spacegrad_t B_z } \\
&=
\pm k \spacegrad E_z + \frac{\omega}{c} I \zcap \wedge \spacegrad_t B_z \\
&=
\pm k \spacegrad E_z - \frac{\omega}{c} \zcap \cross \spacegrad_t B_z \\
\end{aligned}
\end{equation}

Observe that the propagation components of the field \(\BE_z + I\BE_z\) can be written in terms of the symmetric product

\begin{equation}\label{eqn:transverseField:50}
\begin{aligned}
\inv{2} \left( \zcap (\BE + I\BB) + (\BE + I\BB) \zcap \right)
&=
\inv{2} \left( \zcap \BE + \BE \zcap \right) + \frac{I}{2} \left( \zcap \BB + \BB \zcap + I \right) \\
&=
\zcap \cdot \BE + I \zcap \cdot \BB
\end{aligned}
\end{equation}

Now the total field in CGS units was actually \(F = \BE + I \BB/\sqrt{\mu\epsilon}\), not \(F = \BE + I \BB\), so the factorization above is not exactly what we want.   It does however, provide the required clue.  We probably get the result we want by forming the symmetric product (a hybrid dot product selecting both the vector and bivector terms).

\section{Symmetric product of the field with the direction vector}

Rearranging Maxwell's equation \eqnref{eqn:transverseField:max2} in terms of the transverse gradient and the total field \(F\) we have

\begin{equation}\label{eqn:transverseField:blah9}
\begin{aligned}
\spacegrad_t F = \left( \mp i k \zcap + \sqrt{\mu\epsilon}\frac{i\omega}{c}\right) F
\end{aligned}
\end{equation}

With this our symmetric product is

\begin{equation}\label{eqn:transverseField:70}
\begin{aligned}
\spacegrad_t ( F \zcap + \zcap F)
&= (\spacegrad_t F) \zcap - \zcap (\spacegrad_t F) \\
&=
\left( \mp i k \zcap + \sqrt{\mu\epsilon}\frac{i\omega}{c}\right) F \zcap
- \zcap \left( \mp i k \zcap + \sqrt{\mu\epsilon}\frac{i\omega}{c}\right) F \\
&=
i \left( \mp k \zcap + \sqrt{\mu\epsilon}\frac{\omega}{c}\right) (F \zcap - \zcap F) \\
\end{aligned}
\end{equation}

The antisymmetric product on the right hand side should contain the desired transverse field components.  To verify multiply it out

\begin{equation}\label{eqn:transverseField:90}
\begin{aligned}
\inv{2}(F \zcap - \zcap F)
&=
\inv{2}\left( \left(\BE + I \BB/\sqrt{\mu\epsilon}\right) \zcap - \zcap \left(\BE + I \BB/\sqrt{\mu\epsilon}\right) \right)  \\
&=
\BE \wedge \zcap + I \BB/\sqrt{\mu\epsilon} \wedge \zcap \\
&=
(\BE_t + I \BB_t/\sqrt{\mu\epsilon}) \zcap \\
\end{aligned}
\end{equation}

Now, with multiplication by the conjugate quantity \(-i(\pm k \zcap + \sqrt{\mu\epsilon}\omega/c)\), we can extract these transverse components.

\begin{equation}\label{eqn:transverseField:110}
\begin{aligned}
\left( \pm k \zcap + \sqrt{\mu\epsilon}\frac{\omega}{c}\right) \left( \mp k \zcap + \sqrt{\mu\epsilon}\frac{\omega}{c}\right) (F \zcap - \zcap F) &=
\left( -k^2 + {\mu\epsilon}\frac{\omega^2}{c^2}\right) (F \zcap - \zcap F)
\end{aligned}
\end{equation}

Rearranging, we have the transverse components of the field

\begin{equation}\label{eqn:transverseField:blah10}
\begin{aligned}
(\BE_t + I \BB_t/\sqrt{\mu\epsilon}) \zcap &=
\frac{i}{k^2 - \mu\epsilon\frac{\omega^2}{c^2}} \left( \pm k \zcap + \sqrt{\mu\epsilon}\frac{\omega}{c}\right) \spacegrad_t \inv{2}( F \zcap + \zcap F)
\end{aligned}
\end{equation}

With left multiplication by \(\zcap\), and writing \(F = F_t + F_z\) we have

\begin{equation}\label{eqn:transverseField:transverseBoth}
\begin{aligned}
F_t &= \frac{i}{k^2 - \mu\epsilon\frac{\omega^2}{c^2}} \left( \pm k \zcap + \sqrt{\mu\epsilon}\frac{\omega}{c}\right) \spacegrad_t F_z
\end{aligned}
\end{equation}

While this is a complete solution, we can additionally extract the electric and magnetic fields to compare results with Eli's calculation.  We take
vector grades to do so with \(\BE_t = \gpgradeone{F_t}\), and \(\BB_t/\sqrt{\mu\epsilon} = \gpgradeone{-I F_t}\).   For the transverse electric field

\begin{equation}\label{eqn:transverseField:130}
\begin{aligned}
\gpgradeone{ \left( \pm k \zcap + \sqrt{\mu\epsilon}\frac{\omega}{c}\right) \spacegrad_t (\BE_z + I \BB_z/\sqrt{/\mu\epsilon}) }
&=
\pm k \zcap (-\zcap) \spacegrad_t E_z + \frac{\omega}{c} \mathLabelBox{\gpgradeone{I \spacegrad_t \zcap}}{\(-I^2 \zcap \cross \spacegrad_t\)} B_z \\
&=
\mp k \spacegrad_t E_z + \frac{\omega}{c} \zcap \cross \spacegrad_t B_z \\
\end{aligned}
\end{equation}

and for the transverse magnetic field

\begin{equation}\label{eqn:transverseField:150}
\begin{aligned}
&\gpgradeone{ -I \left( \pm k \zcap + \sqrt{\mu\epsilon}\frac{\omega}{c}\right) \spacegrad_t (\BE_z + I \BB_z/\sqrt{\mu\epsilon}) }  \\
&=
-I \sqrt{\mu\epsilon}\frac{\omega}{c} \spacegrad_t \BE_z
+\gpgradeone{ \left( \pm k \zcap + \sqrt{\mu\epsilon}\frac{\omega}{c}\right) \spacegrad_t \BB_z/\sqrt{\mu\epsilon} }  \\
&=
- \sqrt{\mu\epsilon}\frac{\omega}{c} \zcap \cross \spacegrad_t E_z
\mp k \spacegrad_t B_z/\sqrt{\mu\epsilon} \\
\end{aligned}
\end{equation}

Thus the split of transverse field into the electric and magnetic components yields

\begin{equation}\label{eqn:transverseField:transversePair}
\begin{aligned}
\BE_t &= \frac{i}{k^2 - \mu\epsilon\frac{\omega^2}{c^2}} \left( \mp k \spacegrad_t E_z + \frac{\omega}{c} \zcap \cross \spacegrad_t B_z \right) \\
\BB_t &= \frac{i}{k^2 - \mu\epsilon\frac{\omega^2}{c^2}} \left( - {\mu\epsilon}\frac{\omega}{c} \zcap \cross \spacegrad_t E_z \mp k \spacegrad_t B_z \right)
\end{aligned}
\end{equation}

Compared to Eli's method using messy traditional vector algebra, this method also has a fair amount of messy tricky algebra, but of a different sort.

\section{Summary}

There is potentially a lot of new ideas above (some for me even with previous exposure to the Geometric Algebra formalism).  There was no real attempt to teach GA here, but for completeness the GA form of Maxwell's equation was developed from the traditional divergence and curl formulation of Maxwell's equations.  That was mainly due to use of CGS units which differ since this makes Maxwell's equation take a different form from the usual (see \citep{doran2003gap}).

Here a less exploratory summary of the previous results above is assembled.

In these CGS units our field \(F\), and Maxwell's equation (in absence of charge and current), take the form

\begin{equation}\label{eqn:transverseField:foo2}
\begin{aligned}
F &= \BE + \frac{I\BB}{\sqrt{\mu\epsilon}} \\
0 &= \left(\spacegrad + \frac{\sqrt{\mu\epsilon}}{c}\partial_t\right) F
\end{aligned}
\end{equation}

The electric and magnetic fields can be picked off by selecting the grade one (vector) components

\begin{equation}\label{eqn:transverseField:foo8}
\begin{aligned}
\BE &= \gpgradeone{F} \\
\BB &= \sqrt{\mu\epsilon} \gpgradeone{-I F}
\end{aligned}
\end{equation}

With an explicit sinusoidal and \(z\)-axis time dependence for the field

\begin{equation}\label{eqn:transverseField:foo3}
\begin{aligned}
F(x,y,z,t) &= F(x,y) e^{\pm i k z - i \omega t}
\end{aligned}
\end{equation}

and a split of the gradient into transverse and \(z\)-axis components \(\spacegrad = \spacegrad_t + \zcap \partial_z\), Maxwell's equation takes the form
\begin{equation}\label{eqn:transverseField:summaryMax2}
\begin{aligned}
\left(\spacegrad_t \pm i k \zcap - \sqrt{\mu\epsilon}\frac{i\omega}{c}\right) F(x,y) = 0
\end{aligned}
\end{equation}

Writing for short \(F = F(x,y)\), we can split the field into transverse and \(z\)-axis components with the commutator and anticommutator products respectively.  For the \(z\)-axis components we have
\begin{equation}\label{eqn:transverseField:foo4}
F_z \zcap \equiv E_z + I B_z = \inv{2} (F \zcap + \zcap F).
\end{equation}
The projections onto the \(z\)-axis and transverse directions are respectively
\begin{equation}\label{eqn:transverseField:foo5}
\begin{aligned}
F_z &= \BE_z + I \BB_z = \inv{2} (F + \zcap F \zcap) \\
F_t &= \BE_t + I \BB_t = \inv{2} (F - \zcap F \zcap)
\end{aligned}
\end{equation}

With an application of the transverse gradient to the \(z\)-axis field we easily found the relation between the two field components
\begin{equation}\label{eqn:transverseField:foo6}
\spacegrad_t F_z = i \left( \pm k \zcap - \sqrt{\mu\epsilon}\frac{\omega}{c}\right) F_t
\end{equation}

A left division by the multivector factor gives the total transverse field
\begin{equation}\label{eqn:transverseField:foo7}
F_t = \inv{i \left( \pm k \zcap - \sqrt{\mu\epsilon}\frac{\omega}{c}\right) } \spacegrad_t F_z.
\end{equation}

Multiplication of both the numerator and denominator by the conjugate normalizes this
\begin{equation}\label{eqn:transverseField:summaryTransverseBoth}
\begin{aligned}
F_t &= \frac{i}{k^2 - \mu\epsilon\frac{\omega^2}{c^2}} \left( \pm k \zcap + \sqrt{\mu\epsilon}\frac{\omega}{c}\right) \spacegrad_t F_z
\end{aligned}
\end{equation}

From this the transverse electric and magnetic fields may be picked off using the projective grade selection operations of \eqnref{eqn:transverseField:foo8}, and are

\begin{equation}\label{eqn:transverseField:SummaryTransversePair}
\begin{aligned}
\BE_t &= \frac{i}{\mu\epsilon\frac{\omega^2}{c^2} -k^2} \left( \pm k \spacegrad_t E_z - \frac{\omega}{c} \zcap \cross \spacegrad_t B_z \right) \\
\BB_t &= \frac{i}{\mu\epsilon\frac{\omega^2}{c^2} -k^2} \left( {\mu\epsilon}\frac{\omega}{c} \zcap \cross \spacegrad_t E_z \pm k \spacegrad_t B_z \right)
\end{aligned}
\end{equation}

%\EndArticle

   %
% Copyright � 2012 Peeter Joot.  All Rights Reserved.
% Licenced as described in the file LICENSE under the root directory of this GIT repository.
%

%
%
%\input{../peeter_prologue.tex}

\chapter{Comparing phasor and geometric transverse solutions to the Maxwell equation}
\index{Maxwell equation!transverse solution}
\label{chap:transverseWave}

%\blogpage{http://sites.google.com/site/peeterjoot/math2009/transverseWave.pdf}
%\date{August 6, 2009}
%\revisionInfo{\(RCSfile: transverseWave.tex,v \) Last \(Revision: 1.10 \) \(Date: 2009/10/22 02:07:20 \)}

\beginArtWithToc

\section{Motivation}

In (\chapcite{maxwellVacuum}) a phasor like form of the transverse wave equation was found by considering Fourier solutions of the Maxwell equation.  This will be called the ``geometric phasor'' since it is hard to refer and compare it without giving it a name.  Curiously no perpendicularity condition for \(\BE\) and \(\BB\) seemed to be required for this geometric phasor.  Why would that be the case?  In Jackson's treatment, which employed the traditional dot and cross product form of Maxwell's equations, this followed by back substituting the assumed phasor solution back into the equations.  This back substitution was not done in (\chapcite{maxwellVacuum}).  If we attempt this we should find the same sort of additional mutual perpendicularity constraints on the fields.

Here we start with the equations from Jackson (\citep{jackson1975cew}, ch7), expressed in GA form.  Using the same assumed phasor form we should get the same results using GA.  Anything else indicates a misunderstanding or mistake, so as an intermediate step we should at least recover the Jackson result.

After using a more traditional phasor form (where one would have to take real parts) we revisit the geometric phasor found in (\chapcite{maxwellVacuum}).  It will be found that the perpendicular constraints of the Jackson phasor solution lead to a representation where the geometric phasor is reduced to the Jackson form with a straight substitution of the imaginary \(i\) with the pseudoscalar \(I = \sigma_1\sigma_2\sigma_3\).  This representation however, like the more general geometric phasor requires no selection of real or imaginary parts to construct a ``physical'' solution.

\section{With assumed phasor field}

Maxwell's equations in absence of charge and current ((7.1) of Jackson) can be summarized by

\begin{equation}\label{eqn:transverseWave:foo1}
\begin{aligned}
0 &= (\spacegrad + \sqrt{\mu\epsilon}\partial_0) F
\end{aligned}
\end{equation}

The \(F\) above is a composite electric and magnetic field merged into a single multivector.  In the spatial basic the electric field component \(\BE\) is a vector, and the magnetic component \(I\BB\) is a bivector (in the Dirac basis both are bivectors).

\begin{equation}\label{eqn:transverseWave:foo0}
\begin{aligned}
F &= \BE + I \BB/\sqrt{\mu\epsilon}
\end{aligned}
\end{equation}

With an assumed phasor form

\begin{equation}\label{eqn:transverseWave:foo2}
\begin{aligned}
F = \calF e^{ i(\Bk \cdot \Bx - \omega t) } = (\bcE + I\bcB/\sqrt{\mu\epsilon}) e^{ i(\Bk \cdot \Bx - \omega t) }
\end{aligned}
\end{equation}

Although there are many geometric multivectors that square to -1, we do not assume here that the imaginary \(i\) has any specific geometric meaning, and in fact commutes with all multivectors.  Because of this we have to take the real parts later when done.

Operating on \(F\) with Maxwell's equation we have

\begin{equation}\label{eqn:transverseWave:foo3}
\begin{aligned}
0 = (\spacegrad + \sqrt{\mu\epsilon}\partial_0) F = i \left( \Bk - \sqrt{\mu\epsilon}\frac{\omega}{c} \right) F
\end{aligned}
\end{equation}

Similarly, left multiplication of Maxwell's equation by the conjugate operator \(\spacegrad - \sqrt{\mu\epsilon}\partial_0\), we have the wave equation

\begin{equation}\label{eqn:transverseWave:foo4}
\begin{aligned}
0 &= \left(\spacegrad^2 - \frac{\mu\epsilon}{c^2}\frac{\partial^2}{\partial t^2}\right) F
\end{aligned}
\end{equation}

and substitution of the assumed phasor solution gives us

\begin{equation}\label{eqn:transverseWave:foo7}
\begin{aligned}
0 = \lr{\spacegrad^2 - {\mu\epsilon}\partial_{00}} F = -\left( \Bk^2 - {\mu\epsilon}\frac{\omega^2}{c^2} \right) F
\end{aligned}
\end{equation}

This provides the relation between the magnitude of \(\Bk\) and \(\omega\), namely

\begin{equation}\label{eqn:transverseWave:foo8}
\begin{aligned}
\Abs{\Bk} = \pm \sqrt{\mu\epsilon}\frac{\omega}{c}
\end{aligned}
\end{equation}

Without any real loss of generality we can pick the positive root, so the result of the Maxwell equation operator on the phasor is

\begin{equation}\label{eqn:transverseWave:foo5}
\begin{aligned}
0 = (\spacegrad + \sqrt{\mu\epsilon}\partial_0) F = i \sqrt{\mu\epsilon}\frac{\omega}{c} \left( \kcap - 1\right) F
\end{aligned}
\end{equation}

Rearranging we have the curious property that the field \(F\) can ``swallow'' a left multiplication by the propagation direction unit vector

\begin{equation}\label{eqn:transverseWave:fooA}
\begin{aligned}
\kcap F = F
\end{aligned}
\end{equation}

Selection of the scalar and pseudoscalar grades of this equation shows that the electric and magnetic fields \(\BE\) and \(\BB\) are both completely transverse to the propagation direction \(\kcap\).  For the scalar grades we have

\begin{equation}\label{eqn:transverseWave:28}
\begin{aligned}
0 &= \gpgradezero{\kcap F - F} \\
  &= \kcap \cdot \BE
\end{aligned}
\end{equation}

and for the pseudoscalar
\begin{equation}\label{eqn:transverseWave:48}
\begin{aligned}
0 &= \gpgradethree{\kcap F - F} \\
  &= I \kcap \cdot \BB
\end{aligned}
\end{equation}

From this we have \(\kcap \cdot \BB = \kcap \cdot \BB = 0\).  Because of this transverse property we see that the \(\kcap\) multiplication of \(F\) in \eqnref{eqn:transverseWave:fooA} serves to map electric field (vector) components into bivectors, and the magnetic bivector components into vectors.  For the result to be the same means we must have an additional coupling between the field components.  Writing out \eqnref{eqn:transverseWave:fooA} in terms of the field components we have

\begin{equation}\label{eqn:transverseWave:68}
\begin{aligned}
\BE + I\BB/\sqrt{\mu\epsilon}
&= \kcap (\BE + I\BB/\sqrt{\mu\epsilon} ) \\
&= \kcap \wedge \BE + I (\kcap \wedge \BB)/\sqrt{\mu\epsilon}  \\
&= I \kcap \cross \BE + I^2 (\kcap \cross \BB)/\sqrt{\mu\epsilon}
\end{aligned}
\end{equation}

Equating left and right hand grades we have

\begin{equation}\label{eqn:transverseWave:fooD}
\begin{aligned}
\BE &= -(\kcap \cross \BB)/\sqrt{\mu\epsilon} \\
\BB &= \sqrt{\mu\epsilon} (\kcap \cross \BE)
\end{aligned}
\end{equation}

Since \(\BE\) and \(\BB\) both have the same phase relationships we also have

\begin{equation}\label{eqn:transverseWave:fooDD}
\begin{aligned}
\bcE &= -(\kcap \cross \bcB)/\sqrt{\mu\epsilon} \\
\bcB &= \sqrt{\mu\epsilon} (\kcap \cross \bcE)
\end{aligned}
\end{equation}

With phasors as used in electrical engineering it is usual to allow the fields to have complex values.  Assuming this is allowed here too, taking real parts of \(F\), and separating by grade, we have for the electric and magnetic fields

\begin{equation}\label{eqn:transverseWave:fooX}
\begin{aligned}
\begin{pmatrix}
\BE \\
\BB
\end{pmatrix}
=
\Real
\begin{pmatrix}
\bcE \\
\bcB
\end{pmatrix}
\cos(\Bk \cdot \Bx - \omega t)
+\Imag
\begin{pmatrix}
\bcE \\
\bcB
\end{pmatrix}
\sin(\Bk \cdot \Bx - \omega t)
\end{aligned}
\end{equation}

We will find a slightly different separation into electric and magnetic fields with the geometric phasor.

\section{Geometrized phasor}

Translating from SI units to the CGS units of Jackson the geometric phasor representation of the field was found previously to be

\begin{equation}\label{eqn:transverseWave:fooE}
\begin{aligned}
F = e^{ -I \kcap \omega t } e^{ I \Bk \cdot \Bx } (\bcE + I\bcB/\sqrt{\mu\epsilon})
\end{aligned}
\end{equation}

As above the transverse requirement \(\bcE \cdot \Bk = \bcB \cdot \Bk = 0\) was required.  Application of Maxwell's equation operator should show if we require any additional constraints.  That is

\begin{equation}\label{eqn:transverseWave:88}
\begin{aligned}
0
&= (\spacegrad + \sqrt{\mu\epsilon}\partial_0) F \\
&=
(\spacegrad + \sqrt{\mu\epsilon}\partial_0) e^{ -I \kcap \omega t } e^{ I \Bk \cdot \Bx } (\bcE + I\bcB/\sqrt{\mu\epsilon}) \\
&=
\sum \sigma_m e^{ -I \kcap \omega t } (I k^m) e^{ I \Bk \cdot \Bx } (\bcE + I\bcB/\sqrt{\mu\epsilon})
-I \kcap \sqrt{\mu\epsilon} \frac{\omega}{c} e^{ -I \kcap \omega t } e^{ I \Bk \cdot \Bx } (\bcE + I\bcB/\sqrt{\mu\epsilon}) \\
&=
I \left(\Bk - \kcap \sqrt{\mu\epsilon} \frac{\omega}{c} \right) e^{ -I \kcap \omega t } e^{ I \Bk \cdot \Bx } (\bcE + I\bcB/\sqrt{\mu\epsilon})
\end{aligned}
\end{equation}

This is zero for any combinations of \(\bcE\) or \(\bcB\) since \(\Bk = \kcap \sqrt{\mu\epsilon} \omega/c\).  It therefore appears that this geometric phasor has a fundamentally different nature than the non-geometric version.  We have two exponentials that commute, but due to the difference in grades of the arguments, it does not appear that there is any easy way to express this as an single argument exponential.  Multiplying these out, and using the trig product to sum identities helps shed some light on the differences between the geometric phasor and the one using a generic imaginary.  Starting off we have

\begin{equation}\label{eqn:transverseWave:108}
\begin{aligned}
&e^{ -I \kcap \omega t } e^{ I \Bk \cdot \Bx } \\
&=
(\cos(\omega t) -I\kcap \sin(\omega t)) (\cos(\Bk \cdot \Bx) +I\sin(\Bk \cdot \Bx)) \\
&=
\cos(\omega t)\cos(\Bk \cdot \Bx)
+ \kcap \sin(\omega t)\sin(\Bk \cdot \Bx)
-I\kcap \sin(\omega t)\cos(\Bk \cdot \Bx)
+I \cos(\omega t) \sin(\Bk \cdot \Bx) \\
\end{aligned}
\end{equation}

In this first expansion we see that this product of exponentials has scalar, vector, bivector, and pseudoscalar grades, despite the fact that we have only
vector and bivector terms in the end result.  That will be seen to be due to the transverse nature of \(\calF\) that we multiply with.  Before performing that final multiplication, writing \(C_{-} = \cos(\omega t - \Bk \cdot \Bx)\), \(C_{+} = \cos(\omega t + \Bk \cdot \Bx)\), \(S_{-} = \sin(\omega t - \Bk \cdot \Bx)\), and \(S_{+} = \sin(\omega t + \Bk \cdot \Bx)\), we have

\begin{equation}\label{eqn:transverseWave:fooH}
\begin{aligned}
e^{ -I \kcap \omega t } e^{ I \Bk \cdot \Bx }
&=
\inv{2}
\left(
(C_{-} + C_{+})
+\kcap (C_{-} - C_{+})
-I \kcap (S_{-} + S_{+})
-I (S_{-} - S_{+})
\right)
\end{aligned}
\end{equation}

As an operator the left multiplication of \(\kcap\) on a transverse vector has the action

\begin{equation}\label{eqn:transverseWave:128}
\begin{aligned}
\kcap ( \cdot )
&= \kcap \wedge (\cdot) \\
&= I (\kcap \cross (\cdot)) \\
\end{aligned}
\end{equation}

This gives
\begin{equation}\label{eqn:transverseWave:fooI}
\begin{aligned}
e^{ -I \kcap \omega t } e^{ I \Bk \cdot \Bx }
&=
\inv{2}
\left(
(C_{-} + C_{+})
+(C_{-} - C_{+}) I \kcap \cross
+(S_{-} + S_{+}) \kcap \cross
-I (S_{-} - S_{+})
\right)
\end{aligned}
\end{equation}

Now, lets apply this to the field with \(\calF = \bcE + I\bcB/\sqrt{\mu\epsilon}\).  To avoid dragging around the \(\sqrt{\mu\epsilon}\) factors, let us also temporarily
work with units where \(\mu\epsilon = 1\).  We then have

\begin{equation}\label{eqn:transverseWave:148}
\begin{aligned}
2 e^{ -I \kcap \omega t } e^{ I \Bk \cdot \Bx } \calF
&= (C_{-} + C_{+}) (\bcE + I\bcB) \\
&+ (C_{-} - C_{+}) (I (\kcap \cross \bcE) - \kcap \cross \bcB) \\
&+ (S_{-} + S_{+}) (\kcap \cross \bcE +I (\kcap \cross \bcB))  \\
&+ (S_{-} - S_{+}) (-I \bcE + \bcB)
\end{aligned}
\end{equation}

Rearranging explicitly in terms of the electric and magnetic field components this is
\begin{equation}\label{eqn:transverseWave:168}
\begin{aligned}
2 e^{ -I \kcap \omega t } e^{ I \Bk \cdot \Bx } \calF
&=
 (C_{-} + C_{+}) \bcE
-(C_{-} - C_{+}) (\kcap \cross \bcB)
+(S_{-} + S_{+}) (\kcap \cross \bcE)
+(S_{-} - S_{+}) \bcB \\
&+{I}
\left(
 (C_{-} + C_{+}) \bcB
+(C_{-} - C_{+}) (\kcap \cross \bcE)
+(S_{-} + S_{+}) (\kcap \cross \bcB)
-(S_{-} - S_{+}) \bcE
\right) \\
\end{aligned}
\end{equation}

Quite a mess!  A first observation is that the application of the perpendicularity conditions \eqnref{eqn:transverseWave:fooDD} we have a remarkable reduction in complexity.  That is
%\bcE &= -(\kcap \cross \bcB)/\sqrt{\mu\epsilon} \\
%\bcB &= \sqrt{\mu\epsilon} (\kcap \cross \bcE)
\begin{equation}\label{eqn:transverseWave:188}
\begin{aligned}
2 e^{ -I \kcap \omega t } e^{ I \Bk \cdot \Bx } \calF
&=
 (C_{-} + C_{+}) \bcE
+(C_{-} - C_{+}) \bcE
+(S_{-} + S_{+}) \bcB
+(S_{-} - S_{+}) \bcB
\\
&+{I}
\left(
 (C_{-} + C_{+}) \bcB
+(C_{-} - C_{+}) \bcB
-(S_{-} + S_{+}) \bcE
-(S_{-} - S_{+}) \bcE
\right) \\
\end{aligned}
\end{equation}

This wipes out the receding wave terms leaving only the advanced wave terms, leaving

\begin{equation}\label{eqn:transverseWave:208}
\begin{aligned}
e^{ -I \kcap \omega t } e^{ I \Bk \cdot \Bx } \calF
&=
 C_{-} \bcE
+S_{-} (\kcap \cross \bcE)
+{I}
\left(
 C_{-} \bcB +S_{-} \kcap \cross \bcB
\right) \\
&=
 C_{-} (\bcE + I\bcB)
+S_{-} (\bcB -I\bcE) \\
&=
( C_{-} -I S_{-} ) (\bcE + I\bcB) \\
\end{aligned}
\end{equation}
%\bcE &= -(\kcap \cross \bcB)/\sqrt{\mu\epsilon} \\
%\bcB &= \sqrt{\mu\epsilon} (\kcap \cross \bcE)

We see therefore for this special case of mutually perpendicular (equ-magnitude) field components, our geometric phasor has only the advanced wave term

\begin{equation}\label{eqn:transverseWave:fooG}
\begin{aligned}
e^{ -I \kcap \omega t } e^{ I \Bk \cdot \Bx } \calF &= e^{-I(\omega t - \Bk \cdot \Bx)} \calF
\end{aligned}
\end{equation}

If we pick this as the starting point for the assumed solution, it is clear that the same perpendicularity constraints will follow as in Jackson's treatment, or the GA version of it above.  We have something that is slightly different though, for we have no requirement to take real parts of this simplified geometric phasor, since the result already contains just the vector and bivector terms of the electric and magnetic fields respectively.

A small aside, before continuing.  Having made this observation that we can write the assumed phasor for this transverse field in the form of \eqnref{eqn:transverseWave:fooG} an easier way to demonstrate that the product of exponentials reduces only to the advanced wave term is now clear.  Instead of using \eqnref{eqn:transverseWave:fooDD} we could start back at \eqnref{eqn:transverseWave:fooH} and employ the absorption property \(\kcap \calF = \calF\).  That gives

\begin{equation}\label{eqn:transverseWave:228}
\begin{aligned}
e^{ -I \kcap \omega t } e^{ I \Bk \cdot \Bx } \calF
&=
\inv{2}
\left(
(C_{-} + C_{+})
+(C_{-} - C_{+})
-I (S_{-} + S_{+})
-I (S_{-} - S_{+})
\right) \calF \\
&=
\left( C_{-} -I S_{-} \right) \calF
\end{aligned}
\end{equation}

That is the same result, obtained in a slicker manner.

\section{Explicit split of geometric phasor into advanced and receding parts}

For a more general split of the geometric phasor into advanced and receding wave terms, will there be interdependence between the electric and magnetic field components?   Going back to \eqnref{eqn:transverseWave:fooH}, and rearranging, we have

\begin{equation}\label{eqn:transverseWave:248}
\begin{aligned}
2 e^{ -I \kcap \omega t } e^{ I \Bk \cdot \Bx }
&=
(C_{-} -I S_{-})
+\kcap (C_{-} -I S_{-} )
+(C_{+} +I S_{+})
-\kcap (C_{+} +I S_{+}) \\
\end{aligned}
\end{equation}

So we have

\begin{equation}\label{eqn:transverseWave:fooJ}
\begin{aligned}
e^{ -I \kcap \omega t } e^{ I \Bk \cdot \Bx }
&=
\inv{2}(1 + \kcap)e^{-I(\omega t - \Bk \cdot \Bx)}
+\inv{2}(1 - \kcap)e^{I(\omega t + \Bk \cdot \Bx)}
\end{aligned}
\end{equation}

As observed if we have \(\kcap \calF = \calF\), the result is only the advanced wave term

\begin{equation}\label{eqn:transverseWave:268}
\begin{aligned}
e^{ -I \kcap \omega t } e^{ I \Bk \cdot \Bx } \calF = e^{-I(\omega t - \Bk \cdot \Bx)} \calF
\end{aligned}
\end{equation}

Similarly, with absorption of \(\kcap\) with the opposing sign \(\kcap \calF = -\calF\), we have only the receding wave

\begin{equation}\label{eqn:transverseWave:288}
\begin{aligned}
e^{ -I \kcap \omega t } e^{ I \Bk \cdot \Bx } \calF = e^{I(\omega t + \Bk \cdot \Bx)} \calF
\end{aligned}
\end{equation}

Either of the receding or advancing wave solutions should independently satisfy the Maxwell equation operator.  Let us verify both of these, and verify that for either the \(\pm\) cases the following is a solution and examine the constraints for that to be the case.

\begin{equation}\label{eqn:transverseWave:fooK}
\begin{aligned}
F = \inv{2}(1 \pm \kcap) e^{\pm I(\omega t \pm \Bk \cdot \Bx)} \calF
\end{aligned}
\end{equation}

Now we wish to apply the Maxwell equation operator \(\spacegrad + \sqrt{\mu\epsilon}\partial_0\) to this assumed solution.  That is

\begin{equation}\label{eqn:transverseWave:308}
\begin{aligned}
0
&= (\spacegrad + \sqrt{\mu\epsilon}\partial_0) F \\
&=
\sigma_m \inv{2}(1 \pm \kcap) (\pm I \pm k^m) e^{\pm I(\omega t \pm \Bk \cdot \Bx)} \calF
+ \inv{2}(1 \pm \kcap) (\pm I \sqrt{\mu\epsilon}\omega/c) e^{\pm I(\omega t \pm \Bk \cdot \Bx)} \calF \\
&=
\frac{\pm I}{2}\left(\pm \kcap + \sqrt{\mu\epsilon}\frac{\omega}{c}\right)(1 \pm \kcap) e^{\pm I(\omega t \pm \Bk \cdot \Bx)} \calF
\end{aligned}
\end{equation}

By left multiplication with the conjugate of the Maxwell operator \(\grad - \sqrt{\mu\epsilon}\partial_0\) we have the wave equation operator, and applying that, we have as before, a magnitude constraint on the wave number \(\Bk\)

\begin{equation}\label{eqn:transverseWave:328}
\begin{aligned}
0
&= (\spacegrad - \sqrt{\mu\epsilon}\partial_0) (\spacegrad + \sqrt{\mu\epsilon}\partial_0) F \\
&= (\spacegrad^2 - {\mu\epsilon}\partial_{00}) F \\
&= \frac{-1}{2}(1 \pm \kcap) \left( \Bk^2 - \mu\epsilon\frac{\omega^2}{c^2}\right) e^{\pm I(\omega t \pm \Bk \cdot \Bx)} \calF
\end{aligned}
\end{equation}

So we have as before \(\Abs{\Bk} = \sqrt{\mu\epsilon}\omega/c\).  Substitution into the first order operator result we have
\begin{equation}\label{eqn:transverseWave:348}
\begin{aligned}
0
&= (\spacegrad + \sqrt{\mu\epsilon}\partial_0) F \\
&=
\frac{\pm I}{2}\sqrt{\mu\epsilon}\frac{\omega}{c}\left(\pm \kcap + 1\right)(1 \pm \kcap) e^{\pm I(\omega t \pm \Bk \cdot \Bx)} \calF
\end{aligned}
\end{equation}

Observe that the multivector \(1 \pm \kcap\), when squared is just a multiple of itself

\begin{equation}\label{eqn:transverseWave:368}
\begin{aligned}
(1 \pm \kcap)^2 = 1 + \kcap^2 \pm 2 \kcap = 2 (1 \pm \kcap)
\end{aligned}
\end{equation}

So we have

\begin{equation}\label{eqn:transverseWave:388}
\begin{aligned}
0
&= (\spacegrad + \sqrt{\mu\epsilon}\partial_0) F \\
&=
{\pm I}\sqrt{\mu\epsilon}\frac{\omega}{c}(1 \pm \kcap) e^{\pm I(\omega t \pm \Bk \cdot \Bx)} \calF
\end{aligned}
\end{equation}

So we see that the constraint again on the individual assumed solutions is again that of absorption.  Separately the advanced or receding parts of the geometric phasor as expressed in \eqnref{eqn:transverseWave:fooK} are solutions provided

\begin{equation}\label{eqn:transverseWave:fooN}
\begin{aligned}
\kcap F = \mp F
\end{aligned}
\end{equation}

The geometric phasor is seen to be curious superposition of both advancing and receding states.  Independently we have something pretty much like the standard transverse phasor wave states.  Is this superposition state physically meaningful.  It is a solution to the Maxwell equation (without any constraints on \(\bcE\) and \(\bcB\)).

%\EndArticle

   %
% Copyright � 2012 Peeter Joot.  All Rights Reserved.
% Licenced as described in the file LICENSE under the root directory of this GIT repository.
%

%
%
%\input{../peeter_prologue.tex}

\mychapter{Covariant Maxwell equation in media}
\index{Maxwell equation!in matter}
\index{Maxwell equation!covariant}
\label{chap:covariantMedia}

%\blogpage{http://sites.google.com/site/peeterjoot/math2009/covariantMedia.pdf}
%\date{Aug 10, 2009}
%\revisionInfo{\(RCSfile: covariantMedia.tex,v \) Last \(Revision: 1.4 \) \(Date: 2009/10/22 02:07:20 \)}

\beginArtWithToc

\section{Motivation, some notation, and review}

Adjusting to Jackson's of CGS \citep{jackson1975cew} and Maxwell's equations in matter takes some work.  A first pass at a GA form was assembled in (\chapcite{macroscopicMaxwell}), based on what was in the introduction chapter for media that includes \(\BP\), and \(\BM\) properties.  He later changes conventions, and also assumes linear media in most cases, so we want something different than what was previously derived.

The non-covariant form of Maxwell's equation in absence of current and charge has been convenient to use in some initial attempts to look at wave propagation.  That was
%
\begin{equation}\label{eqn:covariantMedia:foo1}
\begin{aligned}
F &= \BE + I\BB/\sqrt{\mu\epsilon} \\
0 &= (\spacegrad + \sqrt{\mu\epsilon} \partial_0) F
\end{aligned}
\end{equation}
%
To examine the energy momentum tensor, it is desirable to express this in a fashion that has no such explicit spacetime dependence.  This suggests a spacetime gradient definition that varies throughout the media.
\begin{equation}\label{eqn:covariantMedia:foo2}
\begin{aligned}
\grad \equiv \gamma^m \partial_m + \sqrt{\mu\epsilon} \gamma^0 \partial_0
\end{aligned}
\end{equation}
%
Observe that this spacetime gradient is adjusted by the speed of light in the media, and is not one that is naturally relativistic.  Even though the differential form of Maxwell's equation is implicitly defined only in a neighborhood of the point it is evaluated at, we now have a reason to say this explicitly, because this non-isotropic condition is now hiding in the (perhaps poor) notation for the operator.  Ignoring the obscuring nature of this operator, and working with it, we can that Maxwell's equation in the neighborhood (where \(\mu\epsilon\) is ``fixed'') is
%
\begin{equation}\label{eqn:covariantMedia:foo3}
\begin{aligned}
\grad F = 0
\end{aligned}
\end{equation}
%
We also want a variant of this that includes the charge and current terms.

\section{Linear media}

Lets pick Jackson's equation (6.70) as the starting point.  A partial translation to GA form, with \(\BD = \epsilon \BE\), and \(\BB = \mu \BH\), and \(\partial_0 = \partial/\partial ct\) is
%
\begin{equation}\label{eqn:covariantMedia:foo4}
\begin{aligned}
\spacegrad \cdot \BB &= 0 \\
\spacegrad \cdot \epsilon \BE &= 4 \pi \rho \\
-I \spacegrad \wedge \BE + \partial_0 \BB &= 0 \\
-I \spacegrad \wedge \BB/\mu - \partial_0 \epsilon \BE &= \frac{4 \pi}{c} \BJ
\end{aligned}
\end{equation}
%
Scaling and adding we have
%
\begin{equation}\label{eqn:covariantMedia:foo5}
\begin{aligned}
\spacegrad \BE + \partial_0 I \BB &= \frac{4 \pi \rho}{\epsilon} \\
\spacegrad \BB - I \partial_0 \mu \epsilon \BE &= \frac{4 \pi \mu I}{c} \BJ
\end{aligned}
\end{equation}
%
Once last scaling prepares for addition of these last two equations
%
\begin{equation}\label{eqn:covariantMedia:foo6}
\begin{aligned}
\spacegrad \BE + \sqrt{\mu\epsilon}\partial_0 I \BB/\sqrt{\mu\epsilon} &= \frac{4 \pi \rho}{\epsilon} \\
\spacegrad I \BB/\sqrt{\mu\epsilon} + \partial_0 \sqrt{\mu \epsilon} \BE &= -\frac{4 \pi \mu }{c\sqrt{\mu\epsilon}}\BJ
\end{aligned}
\end{equation}
%
This gives us a non-covariant assembly of Maxwell's equations in linear media
%
\begin{equation}\label{eqn:covariantMedia:foo7}
\begin{aligned}
(\spacegrad + \sqrt{\mu\epsilon}\partial_0) F &= \frac{4 \pi}{c} \left( \frac{c \rho}{\epsilon} - \sqrt{\frac{\mu}{\epsilon}} \BJ \right)
\end{aligned}
\end{equation}
%
Premultiplication by \(\gamma_0\), and utilizing the definition of \eqnref{eqn:covariantMedia:foo2} we have
%
\begin{equation}\label{eqn:covariantMedia:foo8}
\begin{aligned}
\grad F &= \frac{4 \pi}{c} \left( c \frac{\rho}{\epsilon} \gamma_0 + \sqrt{\frac{\mu}{\epsilon}} J^m \gamma_m \right)
\end{aligned}
\end{equation}
%
We can then define
%
\begin{equation}\label{eqn:covariantMedia:foo9}
\begin{aligned}
J \equiv \frac{c \rho}{\epsilon} \gamma_0 + \sqrt{\frac{\mu}{\epsilon}} J^m \gamma_m
\end{aligned}
\end{equation}
%
and are left with an expression of Maxwell's equation that puts space and time on a similar footing.  It is probably not really right to call this a covariant expression since it is not naturally relativistic.
%
\begin{equation}\label{eqn:covariantMedia:foo10}
\begin{aligned}
\grad F &= \frac{4 \pi}{c} J
\end{aligned}
\end{equation}
%
\section{Energy momentum tensor}

My main goal was to find the GA form of the stress energy tensor in media.  With the requirement for both an alternate spacetime gradient and the inclusion of the scaling factors for the media it is not obviously clear to me how to do translate from the vacuum expression in SI units to the CGS in media form.  It makes sense to step back to see how the divergence conservation equation translates with both of these changes.  In SI units our tensor (a four vector parametrized by another direction vector \(a\)) was
%
\begin{equation}\label{eqn:covariantMedia:foo11}
\begin{aligned}
T(a) \equiv \frac{-1}{2\epsilon_0} F a F
\end{aligned}
\end{equation}
%
Ignoring units temporarily, let us calculate the media-spacetime divergence of \(-FaF/2\).  That is
%
\begin{equation}\label{eqn:covariantMedia:33}
\begin{aligned}
-\inv{2} \grad \cdot (FaF)
&=
-\inv{2} \gpgradezero{\grad (FaF)} \\
&=
-\inv{2} \gpgradezero{(F(\rgrad F) + (F\lgrad)F) a} \\
&=
-\frac{4\pi}{c} \gpgradezero{\inv{2}(F J - J F) a} \\
&=
-\frac{4\pi}{c} (F \cdot J) \cdot a \\
\end{aligned}
\end{equation}
%
We want the \(T^{\mu 0}\) components of the tensor \(T(\gamma_0)\).  Noting the anticommutation relation for the pseudoscalar \(I \gamma_0 = -\gamma_0 I\), and the anticommutation behavior for spatial vectors such as \(\BE \gamma_0 = -\gamma_0\) we have
%
\begin{equation}\label{eqn:covariantMedia:53}
\begin{aligned}
-\inv{2} (\BE + I\BB/\sqrt{\mu\epsilon}) \gamma_0 (\BE + I\BB/\sqrt{\mu\epsilon})
&=
\frac{\gamma_0}{2} (\BE - I\BB/\sqrt{\mu\epsilon}) (\BE + I\BB/\sqrt{\mu\epsilon}) \\
&=
\frac{\gamma_0}{2} \left( (\BE^2 + \BB^2/{\mu\epsilon}) + I \inv{\sqrt{\mu\epsilon}} (\BE\BB - \BB\BE) \right) \\
&=
\frac{1}{2} (\BE^2 + \BB^2/{\mu\epsilon}) + \gamma_0 I \inv{\sqrt{\mu\epsilon}} (\BE \wedge \BB) \\
&=
\frac{\gamma_0}{2} (\BE^2 + \BB^2/{\mu\epsilon}) - \gamma_0 \inv{\sqrt{\mu\epsilon}} (\BE \cross \BB) \\
&=
\frac{\gamma_0}{2} (\BE^2 + \BB^2/{\mu\epsilon}) - \gamma_0 \inv{\sqrt{\mu\epsilon}} \gamma_m \gamma_0 (\BE \cross \BB)^m \\
&=
\frac{\gamma_0}{2} (\BE^2 + \BB^2/{\mu\epsilon}) + \inv{\sqrt{\mu\epsilon}} \gamma_m (\BE \cross \BB)^m \\
\end{aligned}
\end{equation}
%
Calculating the divergence of this using the media spacetime gradient we have
%
\begin{equation}\label{eqn:covariantMedia:73}
\begin{aligned}
\grad \cdot \left( -\inv{2} F \gamma_0 F \right)
&=
\frac{\sqrt{\mu\epsilon}}{c} \frac{\partial}{\partial t} \frac{1}{2} \left(\BE^2 + \inv{\mu\epsilon}\BB^2\right)
+ \sum_m
\frac{\partial}{\partial x^m} \left( \inv{\sqrt{\mu\epsilon}} (\BE \cross \BB)^m \right) \\
&=
\frac{\sqrt{\mu\epsilon}}{c} \frac{\partial}{\partial t} \frac{1}{2} \left(\BE^2 + \inv{\mu\epsilon}\BB^2 \right)
+ \spacegrad \cdot \left( \inv{\sqrt{\mu\epsilon}} (\BE \cross \BB)^m \right)
\end{aligned}
\end{equation}
%
Multiplying this by \((c/4\pi) \sqrt{\epsilon/\mu}\), we have
%
\begin{equation}\label{eqn:covariantMedia:93}
\begin{aligned}
\grad \cdot \left( -\frac{c}{8 \pi} \sqrt{\frac{\epsilon}{\mu}} F \gamma_0 F \right)
&=
\frac{\partial}{\partial t} \frac{1}{2} \left(\BE \cdot \BD + \BB \cdot \BH \right) + \spacegrad \cdot \frac{c}{4\pi}(\BE \cross \BH) \\
&=
- \sqrt{\frac{\epsilon}{\mu}}(F \cdot J) \cdot \gamma_0 \\
\end{aligned}
\end{equation}
%
Now expand the RHS.  We have
%
\begin{equation}\label{eqn:covariantMedia:113}
\begin{aligned}
\sqrt{\frac{\epsilon}{\mu}}
(F \cdot J) \cdot \gamma_0
&=
\left((\BE + I \BB/\sqrt{\mu\epsilon}) \cdot \left( \frac{\rho}{\sqrt{\mu\epsilon}} \gamma_0 + J^m \gamma_m \right) \right) \cdot \gamma_0  \\
&=
\gpgradezero{E^q \gamma_q \gamma_0 J^m \gamma_m \gamma_0} \\
&=
\BE \cdot \BJ
\end{aligned}
\end{equation}
%
Assembling results the energy conservation relation, first in covariant form is
%
\begin{equation}\label{eqn:covariantMedia:foo12}
\begin{aligned}
\grad \cdot \left( -\frac{c}{8 \pi} \sqrt{\frac{\epsilon}{\mu}} F a F \right) &= - \sqrt{\frac{\epsilon}{\mu}}(F \cdot J) \cdot a
\end{aligned}
\end{equation}
%
and the same with an explicit spacetime split in vector quantities is
%
\begin{equation}\label{eqn:covariantMedia:foo13}
\begin{aligned}
\frac{\partial}{\partial t} \frac{1}{2} \left(\BE \cdot \BD + \BB \cdot \BH \right) + \spacegrad \cdot \frac{c}{4\pi}(\BE \cross \BH)
&=
-\BE \cdot \BJ
\end{aligned}
\end{equation}
%
The first of these two \eqnref{eqn:covariantMedia:foo12} is what I was after for application to optics where the radiation field in media can be expressed directly in terms of \(F\) instead of \(\BE\) and \(\BB\).  The second sets the dimensions appropriately and provides some confidence in the result since we can compare to the well known Poynting results in these units.

%It appears (6.106) the electrostatic energy density in CGS units is given by
%
%\begin{align}\label{eqn:covariantMedia:foo1}
%\frac{dW}{dV} = \inv{8\pi} \BE \cdot \BD
%\end{align}

%\EndArticle

   %
% Copyright � 2012 Peeter Joot.  All Rights Reserved.
% Licenced as described in the file LICENSE under the root directory of this GIT repository.
%

%
%
%\input{../peeter_prologue.tex}

\mychapter{Electromagnetic Gauge invariance}
\index{gauge invariance}
\label{chap:jackson12Dash1Gauge}

%\blogpage{http://sites.google.com/site/peeterjoot/math2009/jackson12Dash1Gauge.pdf}
%\date{Sept 24, 2009}
%\revisionInfo{\(RCSfile: jackson12Dash1Gauge.tex,v \) Last \(Revision: 1.5 \) \(Date: 2009/10/22 02:07:20 \)}

%\beginArtWithToc
\beginArtNoToc

At the end of section 12.1 in Jackson \citep{jackson1975cew} he states that it is obvious that the Lorentz force equations are gauge invariant.
%
\begin{equation}\label{eqn:jacksonGaugeInv:foo1}
\begin{aligned}
\frac{d \Bp}{dt} &= e \left( \BE + \frac{\Bu}{c} \cross \BB \right) \\
\frac{d E}{dt} &= e \Bu \cdot \BE
\end{aligned}
\end{equation}
%
Since I did not remember what Gauge invariance was, it was not so obvious.  But if I looking ahead to one of the problem 12.2 on this invariance we have a Gauge transformation defined in four vector form as
%
\begin{equation}\label{eqn:jacksonGaugeInv:foo2}
\begin{aligned}
A^\alpha \rightarrow A^\alpha + \partial^\alpha \psi
\end{aligned}
\end{equation}
%
In vector form with \(A = \gamma_\alpha A^\alpha\), this gauge transformation can be written
%
\begin{equation}\label{eqn:jacksonGaugeInv:foo3}
\begin{aligned}
A \rightarrow A + \grad \psi
\end{aligned}
\end{equation}
%
so this is really a statement that we add a spacetime gradient of something to the four vector potential.  Given this, how does the field transform?
%
\begin{equation}\label{eqn:jackson12Dash1Gauge:27}
\begin{aligned}
F
&= \grad \wedge A \\
&\rightarrow \grad \wedge (A + \grad \psi) \\
&= F + \grad \wedge \grad \psi
\end{aligned}
\end{equation}
%
But \(\grad \wedge \grad \psi = 0\) (assuming partials are interchangeable) so the field is invariant regardless of whether we are talking about the Lorentz force
%
\begin{equation}\label{eqn:jacksonGaugeInv:foo4}
\begin{aligned}
\grad F = J/\epsilon_0 c
\end{aligned}
\end{equation}
%
or the field equations themselves
%
\begin{equation}\label{eqn:jacksonGaugeInv:foo5}
\begin{aligned}
\frac{dp}{d\tau} = e F \cdot v/c
\end{aligned}
\end{equation}
%
So, once you know the definition of the gauge transformation in four vector form, yes this justifiably obvious, however, to anybody who is not familiar with Geometric Algebra, perhaps this is still not so obvious.  How does this translate to the more common place tensor or space time vector notations?  The tensor four vector translation is the easier of the two, and there we have
%
\begin{equation}\label{eqn:jackson12Dash1Gauge:47}
\begin{aligned}
F^{\alpha\beta}
&= \partial^\alpha A^\beta -\partial^\beta A^\alpha \\
&\rightarrow \partial^\alpha (A^\beta + \partial^\beta \psi) -\partial^\beta (A^\alpha + \partial^\alpha \psi) \\
&= F^{\alpha\beta} + \partial^\alpha \partial^\beta \psi -\partial^\beta \partial^\alpha \psi \\
\end{aligned}
\end{equation}
%
As required for \(\grad \wedge \grad \psi = 0\) interchange of partials means the field components \(F^{\alpha\beta}\) are unchanged by adding this gradient.  Finally, in plain old spatial vector form, how is this gauge invariance expressed?

In components we have
%
\begin{equation}\label{eqn:jacksonGaugeInv:foo6}
\begin{aligned}
A^0 &\rightarrow A^0 + \partial^0 \psi = \phi + \inv{c}\frac{\partial \psi}{\partial t} \\
A^k &\rightarrow A^k + \partial^k \psi = A^k - \frac{\partial \psi}{\partial x^k}
\end{aligned}
\end{equation}
%
This last in vector form is \(\BA \rightarrow \BA - \spacegrad \psi\), where the sign inversion comes from \(\partial^k = -\partial_k = -\partial/\partial x^k\), assuming a \(+---\) metric.

We want to apply this to the electric and magnetic field components
%
\begin{equation}\label{eqn:jacksonGaugeInv:foo7}
\begin{aligned}
\BE &= -\spacegrad \phi - \inv{c}\frac{\partial \BA}{\partial t} \\
\BB &= \spacegrad \cross \BA
\end{aligned}
\end{equation}
%
The electric field transforms as
%
\begin{equation}\label{eqn:jackson12Dash1Gauge:67}
\begin{aligned}
\BE &\rightarrow -\spacegrad \left( \phi + \inv{c}\frac{\partial \psi}{\partial t}\right) - \inv{c}\frac{\partial }{\partial t} \left( \BA - \spacegrad \psi \right) \\
&= \BE -\inv{c} \spacegrad \frac{\partial \psi}{\partial t} + \inv{c}\frac{\partial }{\partial t} \spacegrad \psi
\end{aligned}
\end{equation}
%
With partial interchange this is just \(\BE\).  For the magnetic field we have
%
\begin{equation}\label{eqn:jackson12Dash1Gauge:87}
\begin{aligned}
\BB
&\rightarrow \spacegrad \cross \left( \BA - \spacegrad \psi \right) \\
&= \BB  - \spacegrad \cross \spacegrad \psi
\end{aligned}
\end{equation}
%
Again since the partials interchange we have \(\spacegrad \cross \spacegrad \psi = 0\), so this is just the magnetic field.

Alright.  Worked this in three different ways, so now I can say its obvious.

%\EndArticle
%%\EndNoBibArticle

   %
% Copyright � 2012 Peeter Joot.  All Rights Reserved.
% Licenced as described in the file LICENSE under the root directory of this GIT repository.
%

%
%
%\input{../peeter_prologue_print.tex}
%\input{../peeter_prologue_widescreen.tex}

\mychapter{Multivector commutators and Lorentz boosts}
\index{commutator}
\index{Lorentz boost}
\label{chap:boostCommutation}

%\blogpage{http://sites.google.com/site/peeterjoot/math2010/boostCommutation.pdf}
%\date{Oct 30, 2010}
%\revisionInfo{boostCommutation.tex}

\beginArtWithToc
%\beginArtNoToc

\section{Motivation}

In some reading there I found that the electrodynamic field components transform in a reversed sense to that of vectors, where instead of the perpendicular to the boost direction remaining unaffected, those are the parts that are altered.

To explore this, look at the Lorentz boost action on a multivector, utilizing symmetric and antisymmetric products to split that vector into portions effected and unaffected by the boost.  For the bivector (electrodynamic case) and the four vector case, examine how these map to dot and wedge (or cross) products.

The underlying motivator for this boost consideration is an attempt to see where equation (6.70) of \citep{desai2009quantum} comes from.  We get to this by the very end.

\section{Guts}
\subsection{Structure of the bivector boost}

Recall that we can write our Lorentz boost in exponential form with
%
\begin{equation}\label{eqn:boostCommutator:1}
\begin{aligned}
L &= e^{\alpha \Bsigma/2} \\
X' &= L^\dagger X L,
\end{aligned}
\end{equation}
%
where \(\Bsigma\) is a spatial vector.  This works for our bivector field too, assuming the composite transformation is an outermorphism of the transformed four vectors.  Applying the boost to both the gradient and the potential our transformed field is then
%
\begin{equation}\label{eqn:boostCommutation:72}
\begin{aligned}
F'
&= \grad' \wedge A' \\
&= (L^\dagger \grad L) \wedge (L^\dagger A L) \\
&= \inv{2} \left(
(L^\dagger \rgrad L) (L^\dagger A L)
-
(L^\dagger A L) (L^\dagger \lgrad L)
\right) \\
&= \inv{2} L^\dagger \left( \rgrad A - A \lgrad \right) L  \\
&= L^\dagger (\grad \wedge A) L.
\end{aligned}
\end{equation}
%
Note that arrows were used briefly to indicate that the partials of the gradient are still acting on \(A\) despite their vector components being to one side.  We are left with the very simple transformation rule
%
\begin{equation}\label{eqn:boostCommutator:2}
\begin{aligned}
F' = L^\dagger F L,
\end{aligned}
\end{equation}
%
which has exactly the same structure as the four vector boost.

\subsection{Employing the commutator and anticommutator to find the parallel and perpendicular components}

If we apply the boost to a four vector, those components of the four vector that commute with the spatial direction \(\Bsigma\) are unaffected.  As an example, which also serves to ensure we have the sign of the rapidity angle \(\alpha\) correct, consider \(\Bsigma = \Bsigma_1\).  We have
%
\begin{equation}\label{eqn:boostCommutator:3}
\begin{aligned}
X' =
e^{-\alpha \Bsigma/2} (
x^0 \gamma_0 +
x^1 \gamma_1 +
x^2 \gamma_2 +
x^3 \gamma_3 ) (\cosh \alpha/2 + \gamma_1 \gamma_0 \sinh \alpha/2 )
\end{aligned}
\end{equation}
%
We observe that the scalar and \(\Bsigma_1 = \gamma_1 \gamma_0\) components of the exponential commute with \(\gamma_2\) and \(\gamma_3\) since there is no vector in common, but that \(\Bsigma_1\) anticommutes with \(\gamma_0\) and \(\gamma_1\).  We can therefore write
%
\begin{equation}\label{eqn:boostCommutation:92}
\begin{aligned}
X'
&=
x^2 \gamma_2 +
x^3 \gamma_3
+(
x^0 \gamma_0 +
x^1 \gamma_1 +
) (\cosh \alpha + \gamma_1 \gamma_0 \sinh \alpha ) \\
&=
x^2 \gamma_2 +
x^3 \gamma_3
+
\gamma_0 ( x^0 \cosh\alpha - x^1 \sinh \alpha )
+
\gamma_1 ( x^1 \cosh\alpha - x^0 \sinh \alpha )
\end{aligned}
\end{equation}
%
reproducing the familiar matrix result should we choose to write it out.  How can we express the commutation property without resorting to components.  We could write the four vector as a spatial and timelike component, as in
%
\begin{equation}\label{eqn:boostCommutator:4}
\begin{aligned}
X = x^0 \gamma_0 + \Bx \gamma_0,
\end{aligned}
\end{equation}
%
and further separate that into components parallel and perpendicular to the spatial unit vector \(\Bsigma\) as
%
\begin{equation}\label{eqn:boostCommutator:5}
\begin{aligned}
X = x^0 \gamma_0 + (\Bx \cdot \Bsigma) \Bsigma \gamma_0 + (\Bx \wedge \Bsigma) \Bsigma \gamma_0.
\end{aligned}
\end{equation}
%
However, it would be nicer to group the first two terms together, since they are ones that are affected by the transformation.  It would also be nice to not have to resort to spatial dot and wedge products, since we get into trouble too easily if we try to mix dot and wedge products of four vector and spatial vector components.

What we can do is employ symmetric and antisymmetric products (the anticommutator and commutator respectively).  Recall that we can write any multivector product this way, and in particular
%
\begin{equation}\label{eqn:boostCommutator:6}
\begin{aligned}
M \Bsigma = \inv{2} (M \Bsigma  + \Bsigma M) + \inv{2} (M \Bsigma - \Bsigma M).
\end{aligned}
\end{equation}
%
Left multiplying by the unit spatial vector \(\Bsigma\) we have
\begin{equation}\label{eqn:boostCommutator:7}
\begin{aligned}
M = \inv{2} (M + \Bsigma M \Bsigma) + \inv{2} (M - \Bsigma M \Bsigma) =
\inv{2} \symmetric{M}{\Bsigma} \Bsigma + \inv{2} \antisymmetric{M}{\Bsigma} \Bsigma.
\end{aligned}
\end{equation}
%
When \(M = \Ba\) is a spatial vector this is our familiar split into parallel and perpendicular components with the respective projection and rejection operators
%
\begin{equation}\label{eqn:boostCommutator:8}
\begin{aligned}
\Ba = \inv{2} \symmetric{\Ba}{\Bsigma} \Bsigma + \inv{2} \antisymmetric{\Ba}{\Bsigma} \Bsigma = (\Ba \cdot \Bsigma) \Bsigma + (\Ba \wedge \Bsigma) \Bsigma.
\end{aligned}
\end{equation}
%
However, the more general split employing symmetric and antisymmetric products in \eqnref{eqn:boostCommutator:7}, is something we can use for our four vector and bivector objects too.

Observe that we have the commutation and anti-commutation relationships
%
\begin{equation}\label{eqn:boostCommutator:9}
\begin{aligned}
\left( \inv{2} \symmetric{M}{\Bsigma} \Bsigma \right) \Bsigma &= \Bsigma \left( \inv{2} \symmetric{M}{\Bsigma} \Bsigma \right) \\
\left( \inv{2} \antisymmetric{M}{\Bsigma} \Bsigma \right) \Bsigma &= -\Bsigma \left( \inv{2} \antisymmetric{M}{\Bsigma} \Bsigma \right).
\end{aligned}
\end{equation}
%
This split therefore serves to separate the multivector object in question nicely into the portions that are acted on by the Lorentz boost, or left unaffected.

\subsection{Application of the symmetric and antisymmetric split to the bivector field}

Let us apply \eqnref{eqn:boostCommutator:7} to the spacetime event \(X\) again with an x-axis boost \(\sigma = \sigma_1\).  The anticommutator portion of X in this boost direction is
%
\begin{equation}\label{eqn:boostCommutation:112}
\begin{aligned}
\inv{2} \symmetric{X}{\Bsigma_1} \Bsigma_1
&=
\inv{2} \left(
\left(
x^0 \gamma_0 +
x^1 \gamma_1 +
x^2 \gamma_2 +
x^3 \gamma_3 \right)
+
\gamma_1 \gamma_0
\left(
x^0 \gamma_0 +
x^1 \gamma_1 +
x^2 \gamma_2 +
x^3 \gamma_3 \right)
\gamma_1 \gamma_0
\right) \\
&=
x^2 \gamma_2 + x^3 \gamma_3,
\end{aligned}
\end{equation}
%
whereas the commutator portion gives us
\begin{equation}\label{eqn:boostCommutation:132}
\begin{aligned}
\inv{2} \antisymmetric{X}{\Bsigma_1} \Bsigma_1
&=
\inv{2} \left(
\left(
x^0 \gamma_0 +
x^1 \gamma_1 +
x^2 \gamma_2 +
x^3 \gamma_3 \right)
-
\gamma_1 \gamma_0
\left(
x^0 \gamma_0 +
x^1 \gamma_1 +
x^2 \gamma_2 +
x^3 \gamma_3 \right)
\gamma_1 \gamma_0
\right) \\
&=
x^0 \gamma_0 + x^1 \gamma_1.
\end{aligned}
\end{equation}
%
We have seen that only these commutator portions are acted on by the boost.  We have therefore found the desired logical grouping of the four vector \(X\) into portions that are left unchanged by the boost and those that are affected.  That is
%
\begin{equation}\label{eqn:boostCommutator:15}
\begin{aligned}
\inv{2} \antisymmetric{X}{\Bsigma} \Bsigma &= x^0 \gamma_0 + (\Bx \cdot \Bsigma) \Bsigma \gamma_0  \\
\inv{2} \symmetric{X}{\Bsigma} \Bsigma &= (\Bx \wedge \Bsigma) \Bsigma \gamma_0
\end{aligned}
\end{equation}
%
Let us now return to the bivector field \(F = \grad \wedge A = \BE + I c \BB\), and split that multivector into boostable and unboostable portions with the commutator and anticommutator respectively.

Observing that our pseudoscalar \(I\) commutes with all spatial vectors we have for the anticommutator parts that will not be affected by the boost
%
\begin{equation}\label{eqn:boostCommutator:20}
\begin{aligned}
\inv{2} \symmetric{\BE + I c \BB}{\Bsigma} \Bsigma &= (\BE \cdot \Bsigma) \Bsigma + I c (\BB \cdot \Bsigma) \Bsigma,
\end{aligned}
\end{equation}
%
and for the components that will be boosted we have
\begin{equation}\label{eqn:boostCommutator:21}
\begin{aligned}
\inv{2} \antisymmetric{\BE + I c \BB}{\Bsigma} \Bsigma &= (\BE \wedge \Bsigma) \Bsigma + I c (\BB \wedge \Bsigma) \Bsigma.
\end{aligned}
\end{equation}
%
For the four vector case we saw that the components that lay ``perpendicular'' to the boost direction, were unaffected by the boost.  For the field we see the opposite, and the components of the individual electric and magnetic fields that are parallel to the boost direction are unaffected.
%
Our boosted field is therefore
\begin{equation}\label{eqn:boostCommutator:152}
\begin{aligned}
F' =
(\BE \cdot \Bsigma) \Bsigma + I c (\BB \cdot \Bsigma) \Bsigma
+
\left(
(\BE \wedge \Bsigma) \Bsigma + I c (\BB \wedge \Bsigma) \Bsigma
\right) \left( \cosh \alpha + \Bsigma \sinh \alpha \right)
\end{aligned}
\end{equation}
%
Focusing on just the non-parallel terms we have
\begin{equation}\label{eqn:boostCommutation:172}
\begin{aligned}
&\left(
(\BE \wedge \Bsigma) \Bsigma + I c (\BB \wedge \Bsigma) \Bsigma
\right) \left( \cosh \alpha + \Bsigma \sinh \alpha \right) \\
&=
(\BE_\perp + I c \BB_\perp ) \cosh\alpha
+(I \BE \cross \Bsigma - c \BB \cross \Bsigma ) \sinh\alpha \\
&=
\BE_\perp \cosh\alpha - c (\BB \cross \Bsigma ) \sinh\alpha
+ I ( c \BB_\perp \cosh\alpha + (\BE \cross \Bsigma) \sinh\alpha ) \\
&=
\gamma \left(
\BE_\perp - c (\BB \cross \Bsigma ) \Abs{\Bv}/c
+ I ( c \BB_\perp + (\BE \cross \Bsigma) \Abs{\Bv}/c)
\right)
\end{aligned}
\end{equation}
%
A final regrouping gives us
%
\begin{equation}\label{eqn:boostCommutator:50}
\begin{aligned}
F'
&=
\BE_\parallel + \gamma \left( \BE_\perp - \BB \cross \Bv \right)
+I c \left( \BB_\parallel + \gamma \left( \BB_\perp + \BE \cross \Bv/c^2 \right) \right)
\end{aligned}
\end{equation}
%
In particular when we consider the proton, electron system as in equation (6.70) of \citep{desai2009quantum} where it is stated that the electron will feel a magnetic field given by
%
\begin{equation}\label{eqn:boostCommutator:51}
\begin{aligned}
\BB = - \frac{\Bv}{c} \cross \BE
\end{aligned}
\end{equation}
%
we can see where this comes from.  If \(F = \BE + I c (0)\) is the field acting on the electron, then application of a \(\Bv\) boost to the electron perpendicular to the field (ie: radial motion), we get
%
\begin{equation}\label{eqn:boostCommutator:52}
\begin{aligned}
F' = \gamma \BE + I c \gamma \BE \cross \Bv/c^2 =
\gamma \BE + -I c \gamma \frac{\Bv}{c^2} \cross \BE
\end{aligned}
\end{equation}
%
We also have an additional \(1/c\) factor in our result, but that is a consequence of the choice of units where the dimensions of \(\BE\) match \(c \BB\), whereas in the text we have \(\BE\) and \(\BB\) in the same units.  We also have an additional \(\gamma\) factor, so we must presume that \(\Abs{\Bv} << c\) in this portion of the text.  That is actually a requirement here, for if the electron was already in motion, we would have to boost a field that also included a magnetic component.  A consequence of this is that the final interaction Hamiltonian of (6.75) is necessarily non-relativistic.

%\EndArticle

   %
% Copyright � 2012 Peeter Joot.  All Rights Reserved.
% Licenced as described in the file LICENSE under the root directory of this GIT repository.
%

%
%
%\input{../peeter_prologue_print.tex}
%\input{../peeter_prologue_widescreen.tex}

\mychapter{A cylindrical Lienard-Wiechert potential calculation using multivector matrix products}
\index{Lienard-Wiechert potential}
\label{chap:matrixVectorPotentials}

%\blogpage{http://sites.google.com/site/peeterjoot/math2011/matrixVectorPotentials.pdf}
%\date{April 30, 2011}
%\revisionInfo{matrixVectorPotentials.tex}

\beginArtWithToc
%\beginArtNoToc

\section{Motivation}

A while ago I worked the problem of determining the equations of motion for a chain like object \citep{classicalmechanics:multiPendulumSpherical2}.  This was idealized as a set of \(N\) interconnected spherical pendulums.  One of the aspects of that problem that I found fun was that it allowed me to use a new construct, factoring vectors into multivector matrix products, multiplied using the Geometric (Clifford) product.  It seemed at the time that this made the problem tractable, whereas a traditional formulation was much less so.  Later I realized that a very similar factorization was possible with matrices directly \citep{multiPendulumSphericalMatrix}.  This was a bit disappointing since I was enamored by my new calculation tool, and realized that the problem could be tackled with much less learning cost if the same factorization technique was applied using plain old matrices.

I have now encountered a new use for this idea of factoring a vector into a product of multivector matrices.  Namely, a calculation of the four vector Lienard-Wiechert potentials, given a general motion described in cylindrical coordinates.  This I thought I had try since we had a similar problem on our exam (with the motion of the charged particle additionally constrained to a circle).

\section{The goal of the calculation}

Our problem is to calculate
%
\begin{equation}\label{eqn:matrixVectorPotentials:10}
\begin{aligned}
A^0 &= \frac{q}{R^\conj} \\
\BA &= \frac{q \Bv_c}{c R^\conj}
\end{aligned}
\end{equation}
%
where \(\Bx_c(t)\) is the location of the charged particle, \(\Br\) is the point that the field is measured, and
%
\begin{equation}\label{eqn:matrixVectorPotentials:30}
\begin{aligned}
R^\conj &= R - \frac{\Bv_c}{c} \cdot \BR \\
R^2 &= \BR^2 = c^2( t - t_r)^2 \\
\BR &= \Br - \Bx_c(t_r) \\
\Bv_c &= \PD{t_r}{\Bx_c}.
\end{aligned}
\end{equation}
%
\section{Calculating the potentials for an arbitrary cylindrical motion}

Suppose that our charged particle has the trajectory
%
\begin{equation}\label{eqn:matrixVectorPotentials:50}
\Bx_c(t) = h(t) \Be_3 + a(t) \Be_1 e^{i \theta(t)}
\end{equation}
%
where \(i = \Be_1 \Be_2\), and we measure the field at the point
%
\begin{equation}\label{eqn:matrixVectorPotentials:70}
\Br = z \Be_3 + \rho \Be_1 e^{i \phi}
\end{equation}
%
The vector separation between the two is
%
\begin{equation}\label{eqn:matrixVectorPotentials:310}
\begin{aligned}
\BR
&= \Br - \Bx_c \\
&= (z - h) \Be_3 + \Be_1 ( \rho e^{i\phi} - a e^{i\theta} ) \\
&=
\begin{bmatrix}
\Be_1 e^{i\phi} & - \Be_1 e^{i\theta} & \Be_3
\end{bmatrix}
\begin{bmatrix}
\rho \\
a \\
z - h
\end{bmatrix}
\end{aligned}
\end{equation}
%
Transposition does not change this at all, so the (squared) length of this vector difference is
%
\begin{equation}\label{eqn:matrixVectorPotentials:330}
\begin{aligned}
\BR^2 &=
\begin{bmatrix}
\rho &
a &
(z - h)
\end{bmatrix}
\begin{bmatrix}
\Be_1 e^{i\phi} \\
- \Be_1 e^{i\theta} \\
 \Be_3
\end{bmatrix}
\begin{bmatrix}
\Be_1 e^{i\phi} & - \Be_1 e^{i\theta} & \Be_3
\end{bmatrix}
\begin{bmatrix}
\rho \\
a \\
z - h
\end{bmatrix} \\
&=
\begin{bmatrix}
\rho &
a &
(z - h)
\end{bmatrix}
\begin{bmatrix}
\Be_1 e^{i\phi} \Be_1 e^{i\phi} & - \Be_1 e^{i\phi} \Be_1 e^{i\theta} & \Be_1 e^{i\phi} \Be_3 \\
- \Be_1 e^{i\theta} \Be_1 e^{i\phi} & \Be_1 e^{i\theta} \Be_1 e^{i\theta} & - \Be_1 e^{i\theta} \Be_3 \\
 \Be_3 \Be_1 e^{i\phi} & -\Be_3 \Be_1 e^{i\theta} & \Be_3 \Be_3 \\
\end{bmatrix}
\begin{bmatrix}
\rho \\
a \\
z - h
\end{bmatrix} \\
&=
\begin{bmatrix}
\rho &
a &
(z - h)
\end{bmatrix}
\begin{bmatrix}
1 & - e^{i(\theta-\phi)} & \Be_1 e^{i\phi} \Be_3 \\
- e^{i(\phi -\theta)} & 1 & - \Be_1 e^{i\theta} \Be_3 \\
 \Be_3 \Be_1 e^{i\phi} & -\Be_3 \Be_1 e^{i\theta} & 1 \\
\end{bmatrix}
\begin{bmatrix}
\rho \\
a \\
z - h
\end{bmatrix} \\
\end{aligned}
\end{equation}
%
\subsection{A motivation for a Hermitian like transposition operation}
\index{Hermitian transpose}

There are a few things of note about this matrix.  One of which is that it is \textunderline{not} symmetric.  This is a consequence of the non-commutative nature of the vector products.  What we do have is a Hermitian transpose like symmetry.  Observe that terms like the \((1,2)\) and the \((2,1)\) elements of the matrix are equal after all the vector products are reversed.

Using tilde to denote this reversion, we have
%
\begin{equation}\label{eqn:matrixVectorPotentials:350}
\begin{aligned}
(e^{i (\theta - \phi)})^{\tilde{}}
&=
\cos(\theta - \phi)
+ (\Be_1 \Be_2)^{\tilde{}}
\sin(\theta - \phi) \\
&=
\cos(\theta - \phi)
+ \Be_2 \Be_1
\sin(\theta - \phi) \\
&=
\cos(\theta - \phi)
- \Be_1 \Be_2
\sin(\theta - \phi) \\
&=
e^{-i (\theta -\phi)}.
\end{aligned}
\end{equation}
%
The fact that all the elements of this matrix, if non-scalar, have their reversed value in the transposed position, is sufficient to show that the end result is a scalar as expected.  Consider a general quadratic form where the matrix has scalar and bivector grades as above, where there is reversion in all the transposed positions.  That is
%
\begin{equation}\label{eqn:matrixVectorPotentials:90}
b^\T A b
\end{equation}
%
where \(A = \Norm{A_{ij}}\), a \(m \times m\) matrix where \(A_{ij} = \tilde{A_{ji}}\) and contains scalar and bivector grades, and \(b = \Norm{b_i}\), a \(m\times 1\) column matrix of scalars.  Then the product is
%
\begin{equation}\label{eqn:matrixVectorPotentials:370}
\begin{aligned}
\sum_{ij} b_i A_{ij} b_j
&=
\sum_{i<j} b_i A_{ij} b_j
+\sum_{j<i} b_i A_{ij} b_j
+\sum_{k} b_k A_{kk} b_k \\
&=
\sum_{i<j} b_i A_{ij} b_j
+\sum_{i<j} b_{j} A_{ji} b_i
+\sum_{k} b_k A_{kk} b_k \\
&=
\sum_{k} b_k A_{kk} b_k + 2 \sum_{i<j} b_i (A_{ij} + A_{ji}) b_j \\
&=
\sum_{k} b_k A_{kk} b_k + 2 \sum_{i<j} b_i (A_{ij} + \tilde{A_{ij}}) b_j
\end{aligned}
\end{equation}
%
The quantity in braces \(A_{ij} + \tilde{A_{ij}}\) is a scalar since any of the bivector grades in \(A_{ij}\) cancel out.  Consider a similar general product of a vector after the vector has been factored into a product of matrices of multivector elements
%
\begin{equation}\label{eqn:matrixVectorPotentials:110}
\Bx =
\begin{bmatrix}
a_1 & a_2 & \hdots & a_m
\end{bmatrix}
\begin{bmatrix}
b_1 \\ b_2 \\ \vdots \\ b_m
\end{bmatrix}
%=
%\left(
%\begin{bmatrix}
%\tilde{b_1} & \tilde{b_2} & \hdots & \tilde{b_m}
%\end{bmatrix}
%\begin{bmatrix}
%\tilde{a_1} \\ \tilde{a_2} \\ \vdots \\ \tilde{a_m}
%\end{bmatrix}
%\right)^{\tilde{}}.
\end{equation}
%
The (squared) length of the vector is
%
\begin{equation}\label{eqn:matrixVectorPotentials:390}
\begin{aligned}
\Bx^2
&= (a_i b_i) (a_j b_j)  \\
&= (a_i b_i)^{\tilde{}} a_j b_j  \\
&= \tilde{b_i} \tilde{a_i} a_j b_j  \\
&= \tilde{b_i} (\tilde{a_i} a_j) b_j.
\end{aligned}
\end{equation}
%
It is clear that we want a transposition operation that includes reversal of its elements, so with a general factorization of a vector into matrices of multivectors \(\Bx = A b\), its square will be \(\Bx = {\tilde{b}}^\T {\tilde{A}}^\T A b\).

As with purely complex valued matrices, it is convenient to use the dagger notation, and define
%
\begin{equation}\label{eqn:matrixVectorPotentials:150}
A^\dagger = \tilde{A}^\T
\end{equation}
%
where \(\tilde{A}\) contains the reversed elements of \(A\).  By extension, we can define dot and wedge products of vectors expressed as products of multivector matrices.  Given \(\Bx = A b\), a row vector and column vector product, and \(\By = C d\), where each of the rows or columns has \(m\) elements, the dot and wedge products are
%
\begin{equation}\label{eqn:matrixVectorPotentials:170}
\begin{aligned}
\Bx \cdot \By &= \gpgradezero{ d^\dagger C^\dagger A b } \\
\Bx \wedge \By &= \gpgradetwo{ d^\dagger C^\dagger A b }.
\end{aligned}
\end{equation}
%
In particular, if \(b\) and \(d\) are matrices of scalars we have
%
\begin{equation}\label{eqn:matrixVectorPotentials:190}
\begin{aligned}
\Bx \cdot \By &= d^\T \gpgradezero{C^\dagger A} b = d^\T \frac{C^\dagger A + A^\dagger C}{2} b \\
\Bx \wedge \By &= d^\T \gpgradetwo{C^\dagger A} b = d^\T \frac{C^\dagger A - A^\dagger C}{2} b.
\end{aligned}
\end{equation}
%
The dot product is seen as a generator of symmetric matrices, and the wedge product a generator of purely antisymmetric matrices.

\subsection{Back to the problem}

Now, returning to the example above, where we want \(\BR^2\).  We have seen that we can drop any bivector terms from the matrix, so that the squared length can be reduced as
%
\begin{equation}\label{eqn:matrixVectorPotentials:410}
\begin{aligned}
\BR^2
&=
\begin{bmatrix}
\rho &
a &
(z - h)
\end{bmatrix}
\begin{bmatrix}
1 & - e^{i(\theta-\phi)} & 0 \\
- e^{i(\phi -\theta)} & 1 & 0 \\
0 & 0 & 1 \\
\end{bmatrix}
\begin{bmatrix}
\rho \\
a \\
z - h
\end{bmatrix} \\
&=
\begin{bmatrix}
\rho &
a &
(z - h)
\end{bmatrix}
\begin{bmatrix}
1 & - \cos(\theta-\phi) & 0 \\
- \cos(\theta -\phi) & 1 & 0 \\
0 & 0 & 1 \\
\end{bmatrix}
\begin{bmatrix}
\rho \\
a \\
z - h
\end{bmatrix} \\
&=
\begin{bmatrix}
\rho &
a &
(z - h)
\end{bmatrix}
\begin{bmatrix}
\rho - a \cos(\theta - \phi) \\
- \rho \cos(\theta - \phi) + a \\
z - h
\end{bmatrix}
\end{aligned}
\end{equation}
%
So we have
\begin{equation}\label{eqn:matrixVectorPotentials:210}
\begin{aligned}
\BR^2 &= \rho^2 + a^2 + (z -h)^2 - 2 a \rho \cos(\theta - \phi) \\
R &= \sqrt{\rho^2 + a^2 + (z -h)^2 - 2 a \rho \cos(\theta - \phi)}
\end{aligned}
\end{equation}
%
Now consider the velocity of the charged particle.  We can write this as
%
\begin{equation}\label{eqn:matrixVectorPotentials:230}
\frac{d \Bx_c}{dt} =
\begin{bmatrix}
\Be_3 & \Be_1 e^{i \theta} & \Be_2 e^{i\theta}
\end{bmatrix}
\begin{bmatrix}
\hdot \\
\adot \\
a \thetadot
\end{bmatrix}
\end{equation}
%
To compute \(\Bv_c \cdot \BR\) we have to extract scalar grades of the matrix product
%
\begin{equation}\label{eqn:matrixVectorPotentials:430}
\begin{aligned}
\gpgradezero{
\begin{bmatrix}
\Be_1 e^{i\phi} \\
- \Be_1 e^{i\theta} \\
 \Be_3
\end{bmatrix}
\begin{bmatrix}
\Be_3 & \Be_1 e^{i \theta} & \Be_2 e^{i\theta}
\end{bmatrix}
}
&=
\gpgradezero{
\begin{bmatrix}
\Be_1 e^{i\phi} \\
- \Be_1 e^{i\theta} \\
 \Be_3
\end{bmatrix}
\begin{bmatrix}
\Be_3 & \Be_1 e^{i \theta} & \Be_2 e^{i\theta}
\end{bmatrix}
} \\
&=
\gpgradezero{
\begin{bmatrix}
\Be_1 e^{i\phi} \Be_3 & \Be_1 e^{i\phi} \Be_1 e^{i \theta}  & \Be_1 e^{i\phi} \Be_2 e^{i\theta} \\
- \Be_1 e^{i\theta} \Be_3 & - \Be_1 e^{i\theta} \Be_1 e^{i \theta}  & - \Be_1 e^{i\theta} \Be_2 e^{i\theta} \\
 \Be_3 \Be_3 & \Be_3 \Be_1 e^{i \theta}  & \Be_3 \Be_2 e^{i\theta} \\
\end{bmatrix}
} \\
&=
\begin{bmatrix}
0 & \cos(\theta-\phi)  & - \sin(\theta - \phi) \\
0 & - 1  & 0 \\
1 & 0 & 0 \\
\end{bmatrix}.
\end{aligned}
\end{equation}
%
So the dot product is
\begin{equation}\label{eqn:matrixVectorPotentials:450}
\begin{aligned}
\BR \cdot \Bv
&=
\begin{bmatrix}
\rho &
a &
(z - h)
\end{bmatrix}
\begin{bmatrix}
0 & \cos(\theta-\phi)  & - \sin(\theta - \phi) \\
0 & - 1  & 0 \\
1 & 0 & 0 \\
\end{bmatrix}
\begin{bmatrix}
\hdot \\
\adot \\
a \thetadot
\end{bmatrix} \\
&=
\begin{bmatrix}
\rho &
a &
(z - h)
\end{bmatrix}
\begin{bmatrix}
\adot \cos(\theta - \phi) - a \thetadot \sin(\theta - \phi) \\
- \adot \\
\hdot
\end{bmatrix} \\
&=
(z - h) \hdot - \adot a + \rho \adot \cos(\theta - \phi) - \rho a \thetadot \sin(\theta - \phi)
\end{aligned}
\end{equation}
%
This is the last of what we needed for the potentials, so we have
%
\begin{equation}\label{eqn:matrixVectorPotentials:250}
\begin{aligned}
A^0 &= \frac{q}{
%\sqrt{\rho^2 + a^2 + (z -h)^2 - 2 a \rho \cos(\theta - \phi)}
R
-(z - h) \hdot/c + a \adot/c + \rho \cos(\theta - \phi) \adot/c - \rho a \sin(\theta - \phi) \thetadot/c
} \\
\BA &= \frac{ \hdot \Be_3 + (\adot \Be_1 + a \thetadot \Be_2) e^{i\theta} }{c} A^0,
\end{aligned}
\end{equation}
%
where all the time dependent terms in the potentials are evaluated at the retarded time \(t_r\), defined implicitly by the messy relationship
%
\begin{equation}\label{eqn:matrixVectorPotentials:270}
c(t - t_r) = \sqrt{(\rho(t_r))^2 + (a(t_r))^2 + (z -h(t_r))^2 - 2 a(t_r) \rho \cos(\theta(t_r) - \phi)} .
\end{equation}
%
\section{Doing this calculation with plain old cylindrical coordinates}

It is worth trying this same calculation without any geometric algebra to contrast it.  I had expect that the same sort of factorization could also be performed.  Let us try it
%
\begin{equation}\label{eqn:matrixVectorPotentials:290}
\begin{aligned}
\Bx_c &=
\begin{bmatrix}
a \cos\theta \\
a \sin\theta \\
h
\end{bmatrix}
\\
\Br &=
\begin{bmatrix}
\rho \cos\phi \\
\rho \sin\phi \\
z
\end{bmatrix}
\end{aligned}
\end{equation}
%
\begin{equation}\label{eqn:matrixVectorPotentials:470}
\begin{aligned}
\BR
&= \Br - \Bx_c \\
&=
\begin{bmatrix}
\rho \cos\phi - a \cos\theta \\
\rho \sin\phi - a \sin\theta \\
z - h
\end{bmatrix} \\
&=
\begin{bmatrix}
\cos\phi & - \cos\theta & 0 \\
\sin\phi & - \sin\theta & 0 \\
0 & 0 & 1
\end{bmatrix}
\begin{bmatrix}
\rho \\
a \\
z - h
\end{bmatrix}
\end{aligned}
\end{equation}
%
So for \(\BR^2\) we really just need to multiply out two matrices
%
\begin{equation}\label{eqn:matrixVectorPotentials:490}
\begin{aligned}
&\begin{bmatrix}
\cos\phi & \sin\phi & 0 \\
-\cos\theta & - \sin\theta & 0 \\
0 & 0 & 1
\end{bmatrix}
\begin{bmatrix}
\cos\phi & - \cos\theta & 0 \\
\sin\phi & - \sin\theta & 0 \\
0 & 0 & 1
\end{bmatrix} \\
&=
\begin{bmatrix}
\cos^2\phi + \sin^2\phi & -(\cos\phi \cos\phi + \sin\phi \sin\theta) & 0 \\
-(\cos\phi \cos\theta + \sin\theta \sin\phi) & \cos^2\theta + \sin^2\theta & 0 \\
0 & 0 & 1
\end{bmatrix} \\
&=
\begin{bmatrix}
1 & - \cos(\phi - \theta) & 0 \\
- \cos(\phi - \theta) & 1  & 0 \\
0 & 0 & 1
\end{bmatrix} \\
\end{aligned}
\end{equation}
%
So for \(\BR^2\) we have
%
\begin{equation}\label{eqn:matrixVectorPotentials:510}
\begin{aligned}
\BR^2
&=
\begin{bmatrix}
\rho & a & (z -h)
\end{bmatrix}
\begin{bmatrix}
1 & - \cos(\phi - \theta) & 0 \\
- \cos(\phi - \theta) & 1  & 0 \\
0 & 0 & 1
\end{bmatrix}
\begin{bmatrix}
\rho \\
a \\
z - h
\end{bmatrix} \\
&=
\begin{bmatrix}
\rho & a & (z -h)
\end{bmatrix}
\begin{bmatrix}
\rho - a \cos(\phi - \theta) \\
-\rho \cos(\phi - \theta) + a \\
z - h
\end{bmatrix} \\
&=
(z - h)^2 + \rho^2 + a^2 - 2 a \rho \cos(\phi - \theta)
\end{aligned}
\end{equation}
%
We get the same result this way, as expected.  The matrices of multivector products provide a small computational savings, since we do not have to look up the \(\cos\phi \cos\phi + \sin\phi \sin\theta = \cos(\phi - \theta)\) identity, but other than that minor detail, we get the same result.

For the particle velocity we have
%
\begin{equation}\label{eqn:matrixVectorPotentials:530}
\begin{aligned}
\Bv_c
&=
\begin{bmatrix}
\adot \cos\theta - a \thetadot \sin\theta \\
\adot \sin\theta + a \thetadot \cos\theta \\
\hdot
\end{bmatrix} \\
&=
\begin{bmatrix}
\cos\theta & - \sin\theta & 0 \\
\sin\theta & \cos\theta & 0 \\
0 & 0 & 1
\end{bmatrix}
\begin{bmatrix}
\adot \\
a \thetadot  \\
\hdot
\end{bmatrix}
\end{aligned}
\end{equation}
%
So the dot product is
%
\begin{equation}\label{eqn:matrixVectorPotentials:550}
\begin{aligned}
\Bv_c \cdot \BR
&=
\begin{bmatrix}
\adot & a \thetadot  & \hdot
\end{bmatrix}
\begin{bmatrix}
\cos\theta & \sin\theta & 0 \\
-\sin\theta & \cos\theta & 0 \\
0 & 0 & 1
\end{bmatrix}
\begin{bmatrix}
\cos\phi & - \cos\theta & 0 \\
\sin\phi & - \sin\theta & 0 \\
0 & 0 & 1
\end{bmatrix}
\begin{bmatrix}
\rho \\
a \\
z - h
\end{bmatrix} \\
&=
\begin{bmatrix}
\adot & a \thetadot  & \hdot
\end{bmatrix}
\begin{bmatrix}
\cos\theta \cos\phi + \sin\theta \sin\phi & -\cos^2 \theta - \sin^2 \theta & 0 \\
-\cos\phi \sin\theta + \cos\theta \sin\phi & 0 \\
0 & 0 & 1
\end{bmatrix}
\begin{bmatrix}
\rho \\
a \\
z - h
\end{bmatrix} \\
&=
\begin{bmatrix}
\adot & a \thetadot  & \hdot
\end{bmatrix}
\begin{bmatrix}
\cos(\phi - \theta) & -1 & 0 \\
\sin(\phi - \theta) & 0 & 0 \\
0 & 0 & 1
\end{bmatrix}
\begin{bmatrix}
\rho \\
a \\
z - h
\end{bmatrix} \\
&=
\hdot(z - h) - \adot a + \rho \adot \cos(\phi - \theta) + \rho a \thetadot \sin(\phi - \theta)
\end{aligned}
\end{equation}
%
\section{Reflecting on two the calculation methods}

With a learning curve to both Geometric Algebra, and overhead required for this new multivector matrix formalism, it is definitely not a clear winner as a calculation method.  Having worked a couple examples now this way, the first being the N spherical pendulum problem, and now this potentials problem, I will keep my eye out for new opportunities.  If nothing else this can be a useful private calculation tool, and the translation into more pedestrian matrix methods has been seen in both cases to not be too difficult.

%\EndArticle


   % eventually it would be good to merge things into some sort of
   % coherent structure, in which case this would probably be part of
   % a chapter on light, reflection, absorbtion, TE/TM modes, waveguides, ...
   %
   \mychapter{Plane wave solutions of Maxwell's equation}
      %
% Copyright � 2012 Peeter Joot.  All Rights Reserved.
% Licenced as described in the file LICENSE under the root directory of this GIT repository.
%
%\input{../blogpost.tex}
%\renewcommand{\basename}{gaPlaneWaveSolutions}
%\renewcommand{\dirname}{notes/gabook/electrodynamics/}
%\newcommand{\keywords}{Plane wave, Maxwell's equation, geometric algebra, geometric product, wave equation, phasor, wave number, angular frequency, scalar, vector, bivector, pseudoscalar, imaginary, real, multivector, cross product, wedge product}

%\input{../peeter_prologue_print2.tex}
%\beginArtNoToc

%\generatetitle{Plane wave solutions of Maxwell's equation using Geometric Algebra}
%\mychapter{A griffiths problem related to electromagnetic reflection and transmission}
\mychapter{Plane wave solutions of Maxwell's equation using Geometric Algebra}
\index{plane wave}
\index{Maxwell's equation}
\label{chap:gaPlaneWaveSolutions}
\section{Motivation}

Study of reflection and transmission of radiation in isotropic, charge and current free, linear matter utilizes the plane wave solutions to Maxwell's equations.  These have the structure of phasor equations, with some specific constraints on the components and the exponents.

These constraints are usually derived starting with the plain old vector form of Maxwell's equations, and it is natural to wonder how this is done directly using Geometric Algebra.  \citep{doran2003gap} provides one such derivation, using the covariant form of Maxwell's equations.  Here's a slightly more pedestrian way of doing the same.

\section{Maxwell's equations in media}

We start with Maxwell's equations for linear matter as found in \citep{griffiths1999introduction}

\begin{subequations}
\begin{equation}\label{eqn:gaPlaneWaveSolutions:10}
\spacegrad \cdot \BE = 0
\end{equation}
\begin{equation}\label{eqn:gaPlaneWaveSolutions:30}
\spacegrad \cross \BE = -\PD{t}{\BB}
\end{equation}
\begin{equation}\label{eqn:gaPlaneWaveSolutions:50}
\spacegrad \cdot \BB = 0
\end{equation}
\begin{equation}\label{eqn:gaPlaneWaveSolutions:70}
\spacegrad \cross \BB = \mu\epsilon \PD{t}{\BE}.
\end{equation}
\end{subequations}

We merge these using the geometric identity

\begin{equation}\label{eqn:gaPlaneWaveSolutions:90}
\spacegrad \cdot \Ba + I \spacegrad \cross \Ba = \spacegrad \Ba,
\end{equation}

where \(I\) is the 3D pseudoscalar \(I = \Be_1 \Be_2 \Be_3\), to find

\begin{subequations}
\begin{equation}\label{eqn:gaPlaneWaveSolutions:110}
\spacegrad \BE = -I \PD{t}{\BB}
\end{equation}
\begin{equation}\label{eqn:gaPlaneWaveSolutions:130}
\spacegrad \BB = I \mu\epsilon \PD{t}{\BE}.
\end{equation}
\end{subequations}

We want dimensions of \(1/L\) for the derivative operator on the RHS of \eqnref{eqn:gaPlaneWaveSolutions:130}, so we divide through by \(\sqrt{\mu\epsilon} I\) for

\begin{equation}\label{eqn:gaPlaneWaveSolutions:130b}
-I \inv{\sqrt{\mu\epsilon}} \spacegrad \BB = \sqrt{\mu\epsilon} \PD{t}{\BE}.
\end{equation}

This can now be added to \eqnref{eqn:gaPlaneWaveSolutions:110} for

\begin{equation}\label{eqn:gaPlaneWaveSolutions:150}
\left(
\spacegrad + \sqrt{\mu\epsilon} \PD{t}{} \right)
\left( \BE +
\frac{I}{\sqrt{\mu\epsilon}} \BB
\right)
= 0.
\end{equation}

This is Maxwell's equation in linear isotropic charge and current free matter in Geometric Algebra form.

\section{Phasor solutions}

We write the electromagnetic field as

\begin{equation}\label{eqn:gaPlaneWaveSolutions:170}
F =
\left( \BE +
\frac{I}{\sqrt{\mu\epsilon}} \BB
\right),
\end{equation}

so that for vacuum where \(1/\sqrt{\mu \epsilon} = c\) we have the usual \(F = \BE + I c \BB\).  Assuming a phasor solution of

\begin{equation}\label{eqn:gaPlaneWaveSolutions:190}
\tilde{F} = F_0 e^{i (\Bk \cdot \Bx - \omega t)}
\end{equation}

where \(F_0\) is allowed to be complex, and the actual field is obtained by taking the real part

\begin{equation}\label{eqn:gaPlaneWaveSolutions:210}
F = \Real \tilde{F} =
\Real(F_0) \cos(\Bk \cdot \Bx - \omega t)
-\Imag(F_0) \sin(\Bk \cdot \Bx - \omega t).
\end{equation}

Note carefully that we are using a scalar imaginary \(i\), as well as the multivector (pseudoscalar) \(I\), despite the fact that both have the square to scalar minus one property.

We now seek the constraints on \(\Bk\), \(\omega\), and \(F_0\) that allow \(\tilde{F}\) to be a solution to \eqnref{eqn:gaPlaneWaveSolutions:150}

\begin{equation}\label{eqn:gaPlaneWaveSolutions:230}
0 =
\left(
\spacegrad + \sqrt{\mu\epsilon} \PD{t}{}
\right)
\tilde{F}.
\end{equation}

As usual in the non-geometric algebra treatment, we observe that any such solution \(\tilde{F}\) to Maxwell's equation is also a wave equation solution.  In GA we can do so by right multiplying an operator that has a conjugate form,

\begin{equation}\label{eqn:gaPlaneWaveSolutions:250}
\begin{aligned}
0
&=
\left(
\spacegrad + \sqrt{\mu\epsilon} \PD{t}{}
\right)
\tilde{F} \\
&=
\left(
\spacegrad - \sqrt{\mu\epsilon} \PD{t}{}
\right)
\left(
\spacegrad + \sqrt{\mu\epsilon} \PD{t}{}
\right)
\tilde{F} \\
&=
\left( \spacegrad^2 - \mu\epsilon \frac{\partial^2}{\partial t^2} \right) \tilde{F} \\
&=
\left( \spacegrad^2 - \inv{v^2} \frac{\partial^2}{\partial t^2} \right) \tilde{F},
\end{aligned}
\end{equation}

where \(v = 1/\sqrt{\mu\epsilon}\) is the speed of the wave described by this solution.

Inserting the exponential form of our assumed solution \eqnref{eqn:gaPlaneWaveSolutions:190} we find

\begin{equation}\label{eqn:gaPlaneWaveSolutions:270}
0 = -(\Bk^2 - \omega^2/v^2) F_0 e^{i (\Bk \cdot \Bx - \omega t)},
\end{equation}

which implies that the wave number vector \(\Bk\) and the angular frequency \(\omega\) are related by

\begin{equation}\label{eqn:gaPlaneWaveSolutions:290}
v^2 \Bk^2 = \omega^2.
\end{equation}

Our assumed solution must also satisfy the first order system \eqnref{eqn:gaPlaneWaveSolutions:230}

\begin{equation}\label{eqn:gaPlaneWaveSolutions:310}
\begin{aligned}
0
&=
\left(
\spacegrad + \sqrt{\mu\epsilon} \PD{t}{}
\right)
F_0
e^{i (\Bk \cdot \Bx - \omega t)} \\
&=
i
\left(
\Be_m k_m - \frac{\omega}{v}
\right)
F_0
e^{i (\Bk \cdot \Bx - \omega t)} \\
&=
i k ( \kcap - 1 ) F_0 e^{i (\Bk \cdot \Bx - \omega t)}.
\end{aligned}
\end{equation}

The constraints on \(F_0\) must then be given by

\begin{equation}\label{eqn:gaPlaneWaveSolutions:330}
0 = \left( \kcap - 1 \right) F_0.
\end{equation}

With

\begin{equation}\label{eqn:gaPlaneWaveSolutions:350}
F_0 = \BE_0 + I v \BB_0,
\end{equation}

we must then have all grades of the multivector equation equal to zero

\begin{equation}\label{eqn:gaPlaneWaveSolutions:370}
0 =
( \kcap - 1 )
\left(
\BE_0 + I v \BB_0
\right).
\end{equation}

Writing out all the geometric products, grouping into columns by grade, we have

\begin{equation}\label{eqn:gaPlaneWaveSolutions:390}
\begin{array}{l l l l l}
0 &= \kcap \cdot \BE_0 & - \BE_0                   & + \kcap \wedge \BE_0 & I v \kcap \cdot \BB_0 \\
  &                    & + I v \kcap \wedge \BB_0  & + I v \BB_0          &
\end{array}
\end{equation}

We've made use of the fact that \(I\) commutes with all of \(\kcap\), \(\BE_0\), and \(\BB_0\) and employed the identity \(\Ba \Bb = \Ba \cdot \Bb + \Ba \wedge \Bb\).

Collecting the scalar, vector, bivector, and pseudoscalar grades and using \(\Ba \wedge \Bb = I \Ba \cross \Bb\) again, we have a set of constraints resulting from the first order system

\begin{subequations}
\label{eqn:gaPlaneWaveSolutions:405}
\begin{equation}\label{eqn:gaPlaneWaveSolutions:410}
0 = \kcap \cdot \BE_0
\end{equation}
\begin{equation}\label{eqn:gaPlaneWaveSolutions:430}
\BE_0 =- \kcap \cross v \BB_0
\end{equation}
\begin{equation}\label{eqn:gaPlaneWaveSolutions:450}
v \BB_0 = \kcap \cross \BE_0
\end{equation}
\begin{equation}\label{eqn:gaPlaneWaveSolutions:470}
0 = \kcap \cdot \BB_0.
\end{equation}
\end{subequations}

This and \eqnref{eqn:gaPlaneWaveSolutions:290} describe all the constraints on our phasor that are required for it to be a solution.  Note that only one of the two cross product equations in \eqnref{eqn:gaPlaneWaveSolutions:405} are required because the two are not independent (problem \ref{ch:gaPlaneWaveSolutions:pr1}).

Writing out the complete expression for \(F_0\) we have

\begin{equation}\label{eqn:gaPlaneWaveSolutions:510}
\begin{aligned}
F_0
&=
\BE_0 + I v \BB_0 \\
&=
\BE_0 + I \kcap \cross \BE_0 \\
&=
\BE_0 + \kcap \wedge \BE_0.
\end{aligned}
\end{equation}

Since \(\kcap \cdot \BE_0 = 0\), this is

\begin{equation}\label{eqn:gaPlaneWaveSolutions:530}
F_0 = (1 + \kcap) \BE_0.
\end{equation}

Had we been clever enough this could have been deduced directly from the \eqnref{eqn:gaPlaneWaveSolutions:330} directly, since we require a product that is killed by left multiplication with \(\kcap - 1\).  Our complete plane wave solution to Maxwell's equation is therefore given by

\begin{equation}\label{eqn:gaPlaneWaveSolutions:550}
\begin{aligned}
F &= \Real(\tilde{F}) = \BE + \frac{I}{\sqrt{\mu\epsilon}} \BB \\
\tilde{F} &= (1 \pm \kcap) \BE_0 e^{i (\Bk \cdot \Bx \mp \omega t)} \\
0 &= \kcap \cdot \BE_0 \\
\Bk^2 &= \omega^2 \mu \epsilon.
\end{aligned}
\end{equation}

%%
% Copyright � 2013 Peeter Joot.  All Rights Reserved.
% Licenced as described in the file LICENSE under the root directory of this GIT repository.
%
\makeproblem{Electrodynamic plane wave constraints}{ch:gaPlaneWaveSolutions:pr1}{

It was claimed that
\begin{subequations}
\begin{equation}\label{eqn:gaPlaneWaveSolutionsProblems:430a}
\BE_0 =- \kcap \cross v \BB_0
\end{equation}
\begin{equation}\label{eqn:gaPlaneWaveSolutionsProblems:450b}
v \BB_0 = \kcap \cross \BE_0
\end{equation}
\end{subequations}

%equations \eqnref{eqn:gaPlaneWaveSolutionsProblems:430} and \eqnref{eqn:gaPlaneWaveSolutionsProblems:450}
relating the electric and magnetic field of electrodynamic plane waves were dependent.  Show this.
}
\makeanswer{ch:gaPlaneWaveSolutions:pr1}{
This can be shown by crossing \(\kcap\) with \eqnref{eqn:gaPlaneWaveSolutionsProblems:430a} and using the identity

\begin{equation}\label{eqn:gaPlaneWaveSolutionsProblems:490}
\Ba \cross (\Ba \cross \Bb) = - \Ba^2 \Bb + \Ba (\Ba \cdot \Bb).
\end{equation}

This gives
\begin{equation}\label{eqn:gaPlaneWaveSolutionsProblems:490c}
\begin{aligned}
\kcap \cross \BE_0
&= - \kcap \cross (\kcap \cross v \BB_0 ) \\
&= \kcap^2 v \BB_0 - \kcap ( \cancel{\kcap \cdot \BE_0} ) \\
&= v \BB_0.
\end{aligned}
\end{equation}
}
\makeproblem{Proving that the wavevectors are all coplanar}{ch:gaPlaneWaveSolutions:2}{

\citep{griffiths1999introduction} poses the following simple but excellent problem, related to the relationship between the incident, transmission and reflection phasors, which he states has the following form

\begin{equation}\label{eqn:gaPlaneWaveSolutionsProblems:510}
() e^{i (\Bk_i \cdot \Bx - \omega t)}
+ () e^{i (\Bk_r \cdot \Bx - \omega t)}
= () e^{i (\Bk_t \cdot \Bx - \omega t)},
\end{equation}

He poses the problem (9.15)

Suppose \(A e^{i a x} + B e^{i b x} = C e^{i c x}\) for some nonzero constants \(A\), \(B\), \(C\), \(a\), \(b\), \(c\), and for all \(x\).  Prove that \(a = b = c\) and \(A + B = C\).
}

\makeanswer{ch:gaPlaneWaveSolutions:2}{
If this relation holds for all \(x\), then for \(x = 0\), we have \(A + B = C\).  We are left to show that

\begin{equation}\label{eqn:gaPlaneWaveSolutionsProblems:530}
A \left( e^{i a x} - e^{i c x} \right)
+ B \left( e^{i b x} - e^{i c x} \right) = 0.
\end{equation}

Let \(a = c + \delta\) and \(b = c + \epsilon\), so that

\begin{equation}\label{eqn:gaPlaneWaveSolutionsProblems:550}
A \left( e^{i \delta x} - 1 \right)
+ B \left( e^{i \epsilon x} - 1 \right) = 0.
\end{equation}

Now consider some special values of \(x\).  For \(x = 2 \pi/\epsilon\) we have

\begin{equation}\label{eqn:gaPlaneWaveSolutionsProblems:570}
A \left( e^{2 \pi i \delta/\epsilon} - 1 \right) = 0,
\end{equation}

and because \(A \ne 0\), we must conclude that \(\delta/\epsilon\) is an integer.

Similarily, for \(x = 2 \pi/\delta\), we have

\begin{equation}\label{eqn:gaPlaneWaveSolutionsProblems:590}
B \left( e^{2 \pi i \epsilon/\delta} - 1 \right) = 0,
\end{equation}

and this time must conclude that \(\epsilon/\delta\) is an integer.  These ratios must therefore take one of the values \(0, 1, -1\).  Consider the points \(x = 2 n \pi/\epsilon\) or \(x = 2 m \pi/\delta\) we find that \(n \delta/\epsilon\) and \(m \epsilon/\delta\) must be integers for any integers \(m, n\).  This only leaves \(\epsilon = \delta = 0\), or \(a = b = c\) as possibilities.
}


%\vcsinfo
%\EndArticle

      \section{Problems}
         %
% Copyright � 2013 Peeter Joot.  All Rights Reserved.
% Licenced as described in the file LICENSE under the root directory of this GIT repository.
%
\makeproblem{Electrodynamic plane wave constraints}{ch:gaPlaneWaveSolutions:pr1}{

It was claimed that
\begin{subequations}
\begin{equation}\label{eqn:gaPlaneWaveSolutionsProblems:430a}
\BE_0 =- \kcap \cross v \BB_0
\end{equation}
\begin{equation}\label{eqn:gaPlaneWaveSolutionsProblems:450b}
v \BB_0 = \kcap \cross \BE_0
\end{equation}
\end{subequations}

%equations \eqnref{eqn:gaPlaneWaveSolutionsProblems:430} and \eqnref{eqn:gaPlaneWaveSolutionsProblems:450}
relating the electric and magnetic field of electrodynamic plane waves were dependent.  Show this.
}
\makeanswer{ch:gaPlaneWaveSolutions:pr1}{
This can be shown by crossing \(\kcap\) with \eqnref{eqn:gaPlaneWaveSolutionsProblems:430a} and using the identity

\begin{equation}\label{eqn:gaPlaneWaveSolutionsProblems:490}
\Ba \cross (\Ba \cross \Bb) = - \Ba^2 \Bb + \Ba (\Ba \cdot \Bb).
\end{equation}

This gives
\begin{equation}\label{eqn:gaPlaneWaveSolutionsProblems:490c}
\begin{aligned}
\kcap \cross \BE_0
&= - \kcap \cross (\kcap \cross v \BB_0 ) \\
&= \kcap^2 v \BB_0 - \kcap ( \cancel{\kcap \cdot \BE_0} ) \\
&= v \BB_0.
\end{aligned}
\end{equation}
}
\makeproblem{Proving that the wavevectors are all coplanar}{ch:gaPlaneWaveSolutions:2}{

\citep{griffiths1999introduction} poses the following simple but excellent problem, related to the relationship between the incident, transmission and reflection phasors, which he states has the following form

\begin{equation}\label{eqn:gaPlaneWaveSolutionsProblems:510}
() e^{i (\Bk_i \cdot \Bx - \omega t)}
+ () e^{i (\Bk_r \cdot \Bx - \omega t)}
= () e^{i (\Bk_t \cdot \Bx - \omega t)},
\end{equation}

He poses the problem (9.15)

Suppose \(A e^{i a x} + B e^{i b x} = C e^{i c x}\) for some nonzero constants \(A\), \(B\), \(C\), \(a\), \(b\), \(c\), and for all \(x\).  Prove that \(a = b = c\) and \(A + B = C\).
}

\makeanswer{ch:gaPlaneWaveSolutions:2}{
If this relation holds for all \(x\), then for \(x = 0\), we have \(A + B = C\).  We are left to show that

\begin{equation}\label{eqn:gaPlaneWaveSolutionsProblems:530}
A \left( e^{i a x} - e^{i c x} \right)
+ B \left( e^{i b x} - e^{i c x} \right) = 0.
\end{equation}

Let \(a = c + \delta\) and \(b = c + \epsilon\), so that

\begin{equation}\label{eqn:gaPlaneWaveSolutionsProblems:550}
A \left( e^{i \delta x} - 1 \right)
+ B \left( e^{i \epsilon x} - 1 \right) = 0.
\end{equation}

Now consider some special values of \(x\).  For \(x = 2 \pi/\epsilon\) we have

\begin{equation}\label{eqn:gaPlaneWaveSolutionsProblems:570}
A \left( e^{2 \pi i \delta/\epsilon} - 1 \right) = 0,
\end{equation}

and because \(A \ne 0\), we must conclude that \(\delta/\epsilon\) is an integer.

Similarily, for \(x = 2 \pi/\delta\), we have

\begin{equation}\label{eqn:gaPlaneWaveSolutionsProblems:590}
B \left( e^{2 \pi i \epsilon/\delta} - 1 \right) = 0,
\end{equation}

and this time must conclude that \(\epsilon/\delta\) is an integer.  These ratios must therefore take one of the values \(0, 1, -1\).  Consider the points \(x = 2 n \pi/\epsilon\) or \(x = 2 m \pi/\delta\) we find that \(n \delta/\epsilon\) and \(m \epsilon/\delta\) must be integers for any integers \(m, n\).  This only leaves \(\epsilon = \delta = 0\), or \(a = b = c\) as possibilities.
}

      %\section{Solutions}
         %\shipoutAnswer

\part{Lorentz Force}
   %
% Copyright � 2012 Peeter Joot.  All Rights Reserved.
% Licenced as described in the file LICENSE under the root directory of this GIT repository.
%

%
%
\chapter{Lorentz boost of Lorentz force equations}
\index{Lorentz force!boost}
\label{chap:lorentzForceTx}
%\date{May 23, 2009.  lorentzForceTx.tex}

\section{Motivation}

Reading of \citep{bohm1996str} is a treatment of the Lorentz transform
properties of the Lorentz force equation.  This is not clear to me
without working through it myself, so do this.

I also have the urge to
try this with the GA formulation of the Lorentz transformation.  That may not end up being simpler
if one works with the non-covariant form of the Lorentz force equation, but only trying it will tell.

\section{Compare forms of the Lorentz Boost}

Working from the Geometric Algebra form of the Lorentz boost, show equivalence to the standard
coordinate matrix form and the vector form from Bohm.

\subsection{Exponential form}

Write the Lorentz boost of a four vector \(x = x^\mu \gamma_\mu = ct \gamma_0 + x^k \gamma_k\) as

\begin{equation}\label{eqn:lorentzFboost:LorentzBoost}
\begin{aligned}
L(x) &=
e^{-\alpha \vcap/2}
x
e^{\alpha \vcap/2}
\end{aligned}
\end{equation}

\subsection{Invariance property}

A Lorentz transformation (boost or rotation) can be defined as those transformation that leave the four vector square unchanged.

Following \citep{doran2003gap}, work with a \(+---\) metric signature (\(1 = \gamma_0^2 = -\gamma_k^2\)), and \(\sigma_k = \gamma_k \gamma_0\).  Our four vector square in this representation has the familiar invariant form

\begin{equation}\label{eqn:lorentzForceTx:20}
\begin{aligned}
x^2
&=
\lr{ ct \gamma_0 + x^m \gamma_m }
\lr{ ct \gamma_0 + x^k \gamma_k } \\
&=
\lr{ ct \gamma_0 + x^m \gamma_m }
\gamma_0^2
\lr{ ct \gamma_0 + x^k \gamma_k } \\
&=
\lr{ ct + x^m \sigma_m }
\lr{ ct - x^k \sigma_k } \\
&= (ct + \Bx) (ct - \Bx) \\
&= (ct)^2 - \Bx^2
\end{aligned}
\end{equation}

and we expect this of the Lorentz boost of \eqnref{eqn:lorentzFboost:LorentzBoost}.  To verify we have

\begin{equation}\label{eqn:lorentzForceTx:40}
\begin{aligned}
L(x)^2
&=
e^{-\alpha \vcap/2}
x
e^{\alpha \vcap/2}
e^{-\alpha \vcap/2}
x
e^{\alpha \vcap/2} \\
&=
e^{-\alpha \vcap/2}
x
x
e^{\alpha \vcap/2} \\
&=
x^2
e^{-\alpha \vcap/2}
e^{\alpha \vcap/2} \\
&=
x^2
\end{aligned}
\end{equation}

\subsection{Sign of the rapidity angle}
\index{rapidity}

The factor \(\alpha\) will be the rapidity angle, but what sign do we want for a boost along the positive \(\vcap\) direction?

Dropping to coordinates is an easy way to determine the sign convention in effect.  Write \(\vcap = \sigma_1\)

\begin{equation}\label{eqn:lorentzForceTx:60}
\begin{aligned}
L(x) &=
e^{-\alpha \vcap/2}
x
e^{\alpha \vcap/2} \\
&=
(\cosh(\alpha/2) - \sigma_1 \sinh(\alpha/2))
(
x^0 \gamma_0
+x^1 \gamma_1
+x^2 \gamma_2
+x^3 \gamma_3
)
(\cosh(\alpha/2) + \sigma_1 \sinh(\alpha/2))
\end{aligned}
\end{equation}

\(\sigma_1\) commutes with \(\gamma_2\) and \(\gamma_3\) and anticommutes otherwise, so we have

\begin{equation}\label{eqn:lorentzForceTx:80}
\begin{aligned}
L(x) &=
\lr{
x^2 \gamma_2
+x^3 \gamma_3
}
e^{-\alpha \vcap/2}
e^{\alpha \vcap/2}
+
\lr{
x^0 \gamma_0
+x^1 \gamma_1
}
e^{\alpha \vcap} \\
&=
x^2 \gamma_2
+x^3 \gamma_3
+\lr{
x^0 \gamma_0
+x^1 \gamma_1
}
e^{\alpha \vcap} \\
&=
x^2 \gamma_2
+x^3 \gamma_3
+\lr{
x^0 \gamma_0
+x^1 \gamma_1
}
(\cosh(\alpha) + \sigma_1 \sinh(\alpha))
\end{aligned}
\end{equation}

Expanding out just the \(0,1\) terms changed by the transformation we have

\begin{dmath*}
\lr{
x^0 \gamma_0
+x^1 \gamma_1
}
(\cosh(\alpha) + \sigma_1 \sinh(\alpha))
=
x^0 \gamma_0 \cosh(\alpha)
+x^1 \gamma_1 \cosh(\alpha)
+x^0 \gamma_0 \sigma_1 \sinh(\alpha)
+x^1 \gamma_1 \sigma_1 \sinh(\alpha)
=
x^0 \gamma_0 \cosh(\alpha)
+x^1 \gamma_1 \cosh(\alpha)
+x^0 \gamma_0 \gamma_1 \gamma_0 \sinh(\alpha)
+x^1 \gamma_1 \gamma_1 \gamma_0 \sinh(\alpha)
=
x^0 \gamma_0 \cosh(\alpha)
+x^1 \gamma_1 \cosh(\alpha)
-x^0 \gamma_1 \sinh(\alpha)
-x^1 \gamma_0 \sinh(\alpha)
=
\gamma_0 (x^0 \cosh(\alpha) -x^1 \sinh(\alpha) )
+\gamma_1 (x^1 \cosh(\alpha) -x^0 \sinh(\alpha) )
\end{dmath*}

Writing \({x^\mu}' = L(x) \cdot \gamma^\mu\), and \(x^\mu = x \cdot \gamma^\mu\),
and a substitution of \(\cosh(\alpha) = 1/\sqrt{1 - \Bv^2/c^2}\), and \(\alpha \vcap = \tanh^{-1}(\Bv/c)\),
we have the traditional coordinate
expression for the one directional Lorentz boost

\begin{equation}\label{eqn:lorentzForceTx:100}
\begin{aligned}
\begin{bmatrix}
{x^0}' \\
{x^1}' \\
{x^2}' \\
{x^3}'
\end{bmatrix}
&=
\begin{bmatrix}
\cosh\alpha & -\sinh\alpha & 0 & 0 \\
-\sinh\alpha & \cosh\alpha & 0 & 0 \\
0 & 0 & 1 & 0 \\
0 & 0 & 0 & 1 \\
\end{bmatrix}
\begin{bmatrix}
x^0 \\
x^1 \\
x^2 \\
x^3
\end{bmatrix}
\end{aligned}
\end{equation}

Performing this expansion showed initially showed that I had the wrong sign for \(\alpha\) in the exponentials and I went back and
adjusted it all accordingly.

\subsection{Expanding out the Lorentz boost for projective and rejective directions}

Two forms of Lorentz boost representations have been compared above.  An additional one is used in the Bohm text (a
vector form of the Lorentz transformation not using coordinates).  Let us see
if we can derive that from the exponential form.

Start with computation of components of a four vector relative to an observer timelike unit vector \(\gamma_0\).

\begin{equation}\label{eqn:lorentzForceTx:120}
\begin{aligned}
x
&= x \gamma_0 \gamma_0 \\
&= (x \gamma_0) \gamma_0 \\
&=
\lr{ x \cdot \gamma_0 + x \wedge \gamma_0 }
\gamma_0 \\
\end{aligned}
\end{equation}

For the spatial vector factor above write \(\Bx = x \wedge \gamma_0\), for

\begin{equation}\label{eqn:lorentzForceTx:140}
\begin{aligned}
x
&= \lr{ x \cdot \gamma_0 } \gamma_0 + \Bx \gamma_0 \\
&=
\lr{ x \cdot \gamma_0 }
\gamma_0 + \Bx \vcap \vcap \gamma_0 \\
&=
\lr{ x \cdot \gamma_0 }
\gamma_0 + (\Bx \cdot \vcap) \vcap \gamma_0 + (\Bx \wedge \vcap) \vcap \gamma_0 \\
\end{aligned}
\end{equation}

We have the following commutation relations for the various components
\begin{equation}\label{eqn:lorentzForceTx:160}
\begin{aligned}
\vcap (\gamma_0) &= - \gamma_0 \vcap \\
\vcap (\vcap \gamma_0) &= - (\vcap \gamma_0) \vcap \\
\vcap ((\Bx \wedge \vcap) \vcap \gamma_0 )
%&= -(\Bx \wedge \vcap) \vcap \vcap \gamma_0
&= ((\Bx \wedge \vcap) \vcap \gamma_0) \vcap
\end{aligned}
\end{equation}

For a four vector \(u\) that commutes with \(\vcap\) we have \(e^{-\alpha \vcap/2} u = u e^{-\alpha \vcap/2}\), and if it anticommutes
we have the conjugate relation
\(e^{-\alpha \vcap/2} u = u e^{\alpha \vcap/2}\).  This gives us

\begin{equation}\label{eqn:lorentzForceTx:180}
\begin{aligned}
L(x)
&=
(\Bx \wedge \vcap) \vcap \gamma_0 +
\left( (x \cdot \gamma_0) \gamma_0 + (\Bx \cdot \vcap) \vcap \gamma_0 \right) e^{\alpha \vcap} \\
\end{aligned}
\end{equation}

Now write the exponential as a scalar and spatial vector sum
\begin{equation}\label{eqn:lorentzForceTx:200}
\begin{aligned}
e^{\alpha \vcap}
&=
\cosh\alpha
+\vcap \sinh\alpha
\\
&=
\gamma (1 +\vcap \tanh\alpha )
\\
&=
\gamma (1 +\vcap \beta)
\\
&=
\gamma (1 + \Bv/c )
\\
\end{aligned}
\end{equation}

Expanding out the exponential product above, also writing \(x^0 = ct = x \cdot \gamma_0\), we have

\begin{equation}\label{eqn:lorentzForceTx:220}
\begin{aligned}
( &x^0 \gamma_0 + (\Bx \cdot \vcap) \vcap \gamma_0 ) e^{\alpha \vcap} \\
&=
\gamma ( x^0 \gamma_0 + (\Bx \cdot \vcap) \vcap \gamma_0 ) ( 1 + \Bv/c ) \\
&=
\gamma (
x^0 \gamma_0
+ (\Bx \cdot \vcap) \vcap \gamma_0
+x^0 \gamma_0 \Bv/c
+ (\Bx \cdot \vcap) \vcap \gamma_0 \Bv/c
) \\
\end{aligned}
\end{equation}

So for the total Lorentz boost in vector form we have

\begin{equation}\label{eqn:lorentzFboost:fourVectorExpanded}
\begin{aligned}
L(x)
&=
(\Bx \wedge \vcap) \vcap \gamma_0 +
\gamma \left(x^0 - \Bx \cdot \frac{\Bv}{c} \right) \gamma_0
+ \gamma \left( \Bx \cdot \inv{\Bv/c} - x^0 \right) \frac{\Bv}{c} \gamma_0
\end{aligned}
\end{equation}

Now a visual inspection shows that this does match
equation (15-12) from the text:

\begin{equation}\label{eqn:lorentzFboost:explicitSpaceAndTimeTransformed}
\begin{aligned}
\Bx'
&= \Bx - (\vcap \cdot \Bx) \vcap + \frac{ (\vcap \cdot \Bx )\vcap - \Bv t }{ \sqrt{1 - (v^2/c^2)} } \\
t'
&= \frac{ t - (\Bv \cdot \Bx )/c^2 }{ \sqrt{1 - (v^2/c^2)} }
\end{aligned}
\end{equation}

but the equivalence of these is perhaps not so obvious without familiarity
with the GA constructs.

\subsection{differential form}

Bohm utilizes a vector differential form of the Lorentz transformation
for both the spacetime and energy-momentum vectors.  From equation
\eqnref{eqn:lorentzFboost:explicitSpaceAndTimeTransformed}
we can derive the expressions used.  In particular for the transformed
spatial component we have

\begin{equation}\label{eqn:lorentzForceTx:240}
\begin{aligned}
\Bx'
&= \Bx + \gamma \left( -(\vcap \cdot \Bx) \vcap \inv{\gamma} + (\vcap \cdot \Bx )\vcap - \Bv t \right) \\
&= \Bx + \gamma \left( (\vcap \cdot \Bx) \vcap \left(1-\inv{\gamma} \right) - \Bv t \right) \\
&= \Bx + (\gamma-1)(\vcap \cdot \Bx) \vcap - \gamma \Bv t \\
\end{aligned}
\end{equation}

So in differential vector form we have
\begin{equation}\label{eqn:lorentzFboost:differentialSpaceTime}
\begin{aligned}
d\Bx'
&= d\Bx + (\gamma-1)(\vcap \cdot d\Bx) \vcap - \gamma \Bv dt \\
dt'
&= \gamma ( dt -
\lr{ \Bv \cdot d\Bx }/c^2 )
\end{aligned}
\end{equation}

and by analogy with \(dx^0 = cdt \rightarrow dE/c\), and \(d\Bx \rightarrow d\Bp\), we also have the energy momentum transformation

\begin{equation}\label{eqn:lorentzFboost:differentialEnergyMomentum}
\begin{aligned}
d\Bp'
&= d\Bp + (\gamma-1)(\vcap \cdot d\Bp) \vcap - \gamma \Bv dE/c^2 \\
dE'
&= \gamma ( dE - \Bv \cdot d\Bp )
\end{aligned}
\end{equation}

Reflecting on these forms of the Lorentz transformation, they are quite
natural ways to express the vector results.  The terms with \(\gamma\) factors
are exactly what we are used to in the coordinate representation (transformation
of only the time component and the projection of the spatial vector in the
velocity direction), while the \(-1\) part of the \((\gamma-1)\) term just
subtracts off the projection unaltered, leaving
\(d\Bx - (d\Bx \cdot \vcap) \vcap = (d\Bx \wedge \vcap) \vcap\), the rejection
from the \(\vcap\) direction.

\section{Lorentz force transformation}

Preliminaries out of the way, now we want to examine the transform of the electric and magnetic field as used in the Lorentz force equation.  In
CGS units as in the text we have

\begin{equation}\label{eqn:lorentzForceTx:260}
\begin{aligned}
\frac{d\Bp}{dt} &= q \left( \bcE + \frac{\Bv}{c} \cross \bcH \right) \\
\frac{dE}{dt} &= q \bcE \cdot \Bv
\end{aligned}
\end{equation}

After writing this
in differential form
\begin{equation}\label{eqn:lorentzFboost:untransformedLorentzForce}
\begin{aligned}
d\Bp &= q \left( \bcE dt + \frac{d\Bx}{c} \cross \bcH \right) \\
dE &= q \bcE \cdot d\Bx
\end{aligned}
\end{equation}

and the transformed variation of this equation, also in differential form
\begin{equation}\label{eqn:lorentzFboost:transformedLorentzForce}
\begin{aligned}
d\Bp' &= q \left( \bcE' dt' + \frac{d\Bx'}{c} \cross \bcH' \right) \\
dE' &= q \bcE' \cdot d\Bx'
\end{aligned}
\end{equation}

A brute force insertion of the transform results of equations
\eqnref{eqn:lorentzFboost:differentialSpaceTime}, and \eqnref{eqn:lorentzFboost:differentialEnergyMomentum} into these is performed.  This is mostly
a mess of algebra.
%If it was not for mechanical algebra and calculus physics would be much easier!

While the Bohm book covers some of this, other parts are left for the reader.  Do the whole thing here as an exercise.

\subsection{Transforming the Lorentz power equation}

Let us start with the energy rate equation in its entirety without interleaving the momentum calculation.

\begin{equation}\label{eqn:lorentzForceTx:280}
\begin{aligned}
\inv{q} dE'
&= \bcE' \cdot d\Bx' \\
&= \bcE' \cdot \left( d\Bx + (\gamma-1)(\Vcap \cdot d\Bx) \Vcap - \gamma \BV dt \right) \\
&= \bcE' \cdot d\Bx + (\gamma-1)(\Vcap \cdot d\Bx) \bcE' \cdot \Vcap - \gamma \bcE' \cdot \BV dt \\
\inv{q}\gamma ( dE - \BV \cdot d\Bp ) &= \\
\gamma \bcE \cdot d\Bx - \gamma \BV \cdot \left(\bcE dt + \frac{d\Bx}{c} \cross \bcH \right) &= \\
\gamma \bcE \cdot d\Bx
- \gamma \BV \cdot \bcE dt
- \gamma \inv{c} d\Bx \cdot (\bcH \cross \BV) &= \\
\end{aligned}
\end{equation}

Grouping \(dt\) and \(d\Bx\) terms we have
\begin{equation}\label{eqn:lorentzForceTx:300}
\begin{aligned}
0 &=
d\Bx \cdot \left(
\bcE' + (\gamma-1) \Vcap (\bcE' \cdot \Vcap)
-\gamma \bcE
+ \gamma (\bcH \cross \BV/c)
\right)
+ dt \gamma \BV \cdot (\bcE - \bcE')
\end{aligned}
\end{equation}

Now the argument is that both the \(dt\) and \(d\Bx\) factors must separately equal zero.
Assuming
that for now (but come back to this and think it through), and writing \(\bcE = \bcE_\parallel + \bcE_\perp\) for the projective
and rejective components of the field relative to the boost direction \(\BV\) (same for \(\bcH\) and the transformed fields) we have from the \(dt\) term

\begin{equation}\label{eqn:lorentzForceTx:320}
\begin{aligned}
0
&= \BV \cdot ( \bcE_\parallel + \bcE_\perp -\bcE_\parallel' -\bcE_\perp' ) \\
&= \BV \cdot ( \bcE_\parallel -\bcE_\parallel' ) \\
\end{aligned}
\end{equation}

So we can conclude
\begin{equation}\label{eqn:lorentzForceTx:340}
\begin{aligned}
\bcE_\parallel' =\bcE_\parallel
\end{aligned}
\end{equation}

Now from the \(d\Bx\) coefficient, we have

\begin{equation}\label{eqn:lorentzForceTx:360}
\begin{aligned}
0
&=
\bcE_\parallel'
+\bcE_\perp'
 + (\gamma-1) \Vcap (\bcE_\parallel' \cdot \Vcap)
-\gamma \bcE_\parallel
-\gamma \bcE_\perp
+ \gamma (\bcH_\perp \cross \BV/c) \\
&=
\mathLabelBox
[
   labelstyle={below of=m\themathLableNode, below of=m\themathLableNode}
]
{
\left(
\bcE_\parallel' -\Vcap (\bcE_\parallel' \cdot \Vcap)
\right)
}{\(\bcE_\parallel' - \bcE_\parallel'\)}
+\bcE_\perp'
-\gamma
\mathLabelBox
[
   labelstyle={below of=m\themathLableNode, below of=m\themathLableNode}
]
{
\left(
\bcE_\parallel - \Vcap (\bcE_\parallel' \cdot \Vcap)
\right)
}{\(\bcE_\parallel - \bcE_\parallel\)}
-\gamma \bcE_\perp
+ \gamma (\bcH_\perp \cross \BV/c) \\
\end{aligned}
\end{equation}

This now completely specifies the transformation properties of the electric field under a \(\BV\) boost

\begin{equation}\label{eqn:lorentzFboost:finalResultsForPower}
\begin{aligned}
\bcE_\perp' &= \gamma \left( \bcE_\perp + \frac{\BV}{c} \cross \bcH_\perp \right) \\
\bcE_\parallel' &= \bcE_\parallel
\end{aligned}
\end{equation}

(it also confirms the typos in the text).

\subsection{Transforming the Lorentz momentum equation}

Now we do the exercise for the reader part, and express the transformed momentum differential of equation
\eqnref{eqn:lorentzFboost:transformedLorentzForce} in terms of \eqnref{eqn:lorentzFboost:differentialSpaceTime}

\begin{equation}\label{eqn:lorentzForceTx:380}
\begin{aligned}
\inv{q} d\Bp'
&= \bcE' dt' + \frac{d\Bx'}{c} \cross \bcH' \\
&= \gamma \bcE' dt - \gamma \bcE' (\BV \cdot d\Bx )/c^2
+ d\Bx \cross \bcH'/c + (\gamma-1)(\Vcap \cdot d\Bx) \Vcap \cross \bcH'/c - \gamma \BV \cross \bcH'/c dt \\
\end{aligned}
\end{equation}

Now for the LHS using \eqnref{eqn:lorentzFboost:differentialEnergyMomentum} and \eqnref{eqn:lorentzFboost:untransformedLorentzForce} we have
\begin{equation}\label{eqn:lorentzForceTx:400}
\begin{aligned}
\inv{q} d\Bp'
&= d\Bp/q + (\gamma-1)(\Vcap \cdot d\Bp/q) \Vcap - \gamma \BV dE/qc^2 \\
&= \bcE dt + \frac{d\Bx}{c} \cross \bcH
+ (\gamma-1)( \Vcap \cdot \bcE dt + \Vcap \cdot (d\Bx \cross \bcH/c) ) \Vcap - \gamma \BV (
\bcE \cdot d\Bx
)/c^2 \\
&= \bcE dt + \frac{d\Bx}{c} \cross \bcH
+ (\gamma-1) (\Vcap \cdot \bcE) \Vcap dt
+ (\gamma-1)( d\Bx \cdot (\bcH \cross \Vcap/c) ) \Vcap
- \gamma \BV ( \bcE \cdot d\Bx )/c^2 \\
\end{aligned}
\end{equation}

Combining these and grouping by \(dt\) and \(d\Bx\) we have
\begin{equation}\label{eqn:lorentzForceTx:420}
\begin{aligned}
dt &\left(
-( \bcE - (\Vcap \cdot \bcE) \Vcap )
+ \gamma (\bcE' - (\Vcap \cdot \bcE) \Vcap )
- \gamma \BV \cross \bcH'/c
\right) \\
&=
 \frac{\gamma}{c^2} \left( \bcE' (\BV \cdot d\Bx ) - \BV ( \bcE \cdot d\Bx ) \right)
+ \frac{d\Bx}{c} \cross (\bcH -\bcH')  \\
&\qquad + \frac{\gamma-1}{c}
\left(
( d\Bx \cdot (\bcH \cross \Vcap) ) \Vcap - (\Vcap \cdot d\Bx) (\Vcap \cross \bcH')
\right) \\
\end{aligned}
\end{equation}

What a mess, and this is after some initial grouping!  From the power result we have \(\Vcap \cdot \bcE = \Vcap \cdot \bcE'\) so we can write the LHS of this mess as

\begin{equation}\label{eqn:lorentzForceTx:440}
\begin{aligned}
dt &\left(
-( \bcE - (\Vcap \cdot \bcE) \Vcap )
+ \gamma (\bcE' - (\Vcap \cdot \bcE) \Vcap )
- \gamma \BV \cross \bcH'/c
\right) \\
&=
dt \left(
-( \bcE - (\Vcap \cdot \bcE) \Vcap )
+ \gamma (\bcE' - (\Vcap \cdot \bcE') \Vcap )
- \gamma \BV \cross \bcH'/c
\right)
\\
&=
dt \left(
-\bcE_\perp
+ \gamma \bcE_\perp'
- \gamma \BV \cross \bcH'/c
\right)
\\
&=
dt \left(
-\bcE_\perp
+ \gamma \bcE_\perp'
- \gamma \BV \cross \bcH_\perp'/c
\right)
\\
\end{aligned}
\end{equation}

If this can separately equal zero independent of the \(d\Bx\) terms we have

\begin{equation}\label{eqn:lorentzForceTx:460}
\begin{aligned}
\bcE_\perp = \gamma \left( \bcE_\perp' - \frac{\BV}{c} \cross \bcH_\perp' \right)
\end{aligned}
\end{equation}

Contrast this to the result for \(\bcE_\perp'\) in the first of \eqnref{eqn:lorentzFboost:finalResultsForPower}.  It differs only by
a sign which has an intuitive relativistic (anti)symmetry that is not entirely unsurprising.  If a boost along \(\BV\)
takes \(\bcE\) to \(\bcE'\), then an boost with opposing direction makes sense for the reverse.

Despite being reasonable seeming, a relation like \(\bcH_\parallel = \bcH_\parallel'\) was expected ... does that follow from this somehow?
Perhaps things will become more clear after examining the mess on the RHS involving all the \(d\Bx\) terms?

The first part of this looks amenable to some algebraic manipulation.  Using
\((\bcE' \wedge \BV) \cdot d\Bx = \bcE' (\BV \cdot d\Bx) - \BV (\bcE' \cdot d\Bx)\), we have

\begin{equation}\label{eqn:lorentzForceTx:480}
\begin{aligned}
\bcE' (\BV \cdot d\Bx ) - \BV ( \bcE \cdot d\Bx )
&=
(\bcE' \wedge \BV) \cdot d\Bx + \BV (\bcE' \cdot d\Bx) - \BV ( \bcE \cdot d\Bx ) \\
&=
(\bcE' \wedge \BV) \cdot d\Bx + \BV ((\bcE' - \bcE) \cdot d\Bx) \\
\end{aligned}
\end{equation}

and
\begin{equation}\label{eqn:lorentzForceTx:500}
\begin{aligned}
(\bcE' \wedge \BV) \cdot d\Bx
&=
\gpgradeone{ (\bcE' \wedge \BV) d\Bx } \\
&=
\gpgradeone{ i(\bcE' \cross \BV) d\Bx } \\
&=
\gpgradeone{ i ((\bcE' \cross \BV) \wedge d\Bx) } \\
&=
\gpgradeone{ i^2 ((\bcE' \cross \BV) \cross d\Bx) } \\
&=
d\Bx \cross (\bcE' \cross \BV)
\end{aligned}
\end{equation}

Putting things back together, does it improve things?

\begin{equation}\label{eqn:lorentzForceTx:520}
\begin{aligned}
0 &=
% \frac{\gamma}{c^2} \left( d\Bx \cross (\bcE' \cross \BV) + \BV ((\bcE' - \bcE) \cdot d\Bx) \right)
%+ \frac{d\Bx}{c} \cross (\bcH -\bcH')
{d\Bx} \cross \left(
{\gamma} \left(\bcE' \cross \frac{\BV}{c} \right)
+ (\bcH -\bcH')
\right) \\
&+\frac{\gamma}{c} \BV ((\bcE' - \bcE) \cdot d\Bx)  \\
%
&+ (\gamma-1)
\left(
( d\Bx \cdot (\bcH \cross \Vcap) ) \Vcap - (\Vcap \cdot d\Bx) (\Vcap \cross \bcH')
\right) \\
\end{aligned}
\end{equation}

Perhaps the last bit can be factored into \(d\Bx\) crossed with some function of \(\bcH - \bcH'\)?

   %
% Copyright � 2012 Peeter Joot.  All Rights Reserved.
% Licenced as described in the file LICENSE under the root directory of this GIT repository.
%

%
%
\mychapter{Lorentz force Law}
\label{chap:PJSrGAFPLorentzForce}
\index{Lorentz force}
%\date{August 16, 2008}

\section{Some notes on GAFP 5.5.3 The Lorentz force Law}

Expand on treatment of \citep{doran2003gap}.

The idea behind this derivation, is to express the vector part of the proper force in covariant form, and then
do the same for the energy change part of the proper momentum.  That first part is:

\begin{equation}\label{eqn:gafpLorentz:20}
\begin{aligned}
\frac{dp}{d\tau} \wedge \gamma_0
&= \frac{d (\gamma \Bp)}{d\tau} \\
&= \frac{d (\gamma \Bp)}{dt} \frac{dt}{d\tau} \\
&= \frac{dt}{d\tau} q \left( \BE + \Bv \cross \BB \right)
\end{aligned}
\end{equation}

Now, the spacetime split of velocity is done in the normal fashion:

\begin{equation}\label{eqn:gafpLorentz:40}
\begin{aligned}
x &= c t \gamma_0 + \sum x^i \gamma_i \\
v &= \frac{dx}{d\tau} = c \frac{dt}{d\tau} \gamma_0 + \sum \frac{dx^i}{d\tau} \gamma_i \\
v \cdot \gamma_0 &= c \frac{dt}{d\tau} = c \gamma \\
v \wedge \gamma_0
&= \sum \frac{dx^i}{dt} \frac{dt}{d\tau} \gamma_i \gamma_0 \\
&= (v \cdot \gamma_0)/c \sum v^i \sigma_i \\
&= (v \cdot \gamma_0) \Bv/c.
\end{aligned}
\end{equation}

Writing \(\dot{p} = dp/d\tau\), substitute the gamma factor into the force equation:

\begin{equation*}
\dot{p} \wedge \gamma_0 = ( v/c \cdot \gamma_0 ) q \left( \BE + \Bv \cross \BB \right)
\end{equation*}

Now, GAFP goes on to show that the \(\gamma \BE\) term can be reduced to the form \((\BE \cdot v) \wedge \gamma_0\).  Their
method is not exactly obvious, for example writing \(\BE = (1/2)(\BE + \BE)\) to start.  Let us just do this backwards
instead, expanding \(\BE \cdot v\) to see the form of that term:

\begin{equation}\label{eqn:gafpLorentz:60}
\begin{aligned}
\BE \cdot v
&= \left(\sum E^i \gamma_{i0}\right) \cdot \left( \sum v^{\mu} \gamma_{\mu}\right) \\
&= \sum E^i v^{\mu} \gpgradeone{ \gamma_{i0\mu}} \\
&= v^0 \sum E^i \gamma_{i} + \sum E^i v^{j}
\mathLabelBox
[
   labelstyle={xshift=2cm},
   linestyle={out=270,in=90, latex-}
]
{\gpgradeone{ \gamma_{i0j}}}{\(-\delta_{ij} \gamma_0\)} \\
&= v^0 \sum E^i \gamma_i - \sum E^i v^i \gamma_0.
\end{aligned}
\end{equation}

Wedging with \(\gamma_0\) we have the desired result:

\begin{equation*}
(\BE \cdot v) \wedge \gamma_0 = v^0 \sum E^i \gamma_{i0} = (v \cdot \gamma_0) \BE = c \gamma \BE
\end{equation*}

Now, for equation 5.164 there are not any surprising steps, but lets try this backwards too:

\begin{equation}\label{eqn:gafpLorentz:80}
\begin{aligned}
(I \BB) \cdot v
&= \left(\sum B^i \mathLabelBox{\gamma_{102030i0}}{\(\gamma_{123i}\)} \right) \cdot \left( \sum v^{\mu} \gamma_{\mu} \right) \\
&= \sum B^i v^{\mu} \gpgradeone{\gamma_{123i\mu}}
\end{aligned}
\end{equation}

That vector selection does yield the cross product as expected:

\begin{equation*}
\gpgradeone{\gamma_{123i\mu}} =
\left\{
\begin{array}{l l}
0 & \quad \mu = 0 \\
0 & \quad i = \mu \\
\gamma_1 & \quad i\mu = 32 \\
-\gamma_2 & \quad i\mu = 31 \\
\gamma_3 & \quad i\mu = 21 \\
\end{array} \right.
\end{equation*}

(with alternation for the missing set of index pairs).

This gives:
\begin{equation}\label{eqn:gafpLorentz:100}
\begin{aligned}
(I \BB) \cdot v
= (B^3 v^2 - B^2 v^3) \gamma_1
+ (B^1 v^3 - B^3 v^1) \gamma_2
+ (B^2 v^1 - B^1 v^2) \gamma_3,
\end{aligned}
\end{equation}

thus, since \(v^i = \gamma d{x^i}/dt\), this yields the desired result

\begin{equation*}
((I\BB) \cdot v) \wedge \gamma_0 = \gamma \Bv \cross \BB
\end{equation*}

In retrospect, for this magnetic field term, the GAFP approach is cleaner and easier than to try to do it the dumb way.

Combining the results we have:

\begin{equation}\label{eqn:gafpLorentz:120}
\begin{aligned}
\dot{p} \wedge \gamma_0
&= q \gamma ( \BE + \Bv \cross \BB ) \\
&= q (( \BE + c I \BB ) \cdot (v/c)) \wedge \gamma_0 \\
\end{aligned}
\end{equation}

Or with \(F = \BE + c I \BB\), we have:

\begin{equation}\label{eqn:gafpLor:pvec}
\dot{p} \wedge \gamma_0 = q ( F \cdot v/c ) \wedge \gamma_0
\end{equation}

It is tempting here to attempt to cancel the \(\wedge \gamma_0\) parts of this equation, but that cannot be done
until one also shows:

\begin{equation*}
\dot{p} \cdot \gamma_0 = q ( F \cdot v/c ) \cdot \gamma_0
\end{equation*}

I follow most of the details of GAFP on this fine.  I found they omitted a couple steps that would have been helpful.

For the four momentum we have:

\begin{equation*}
p_0 = p \cdot \gamma_0 = E/c
\end{equation*}

The rate of change work done on the particle by the force is:

\begin{equation}\label{eqn:gafpLorentz:140}
\begin{aligned}
dW &= q\BE \cdot d\Bx \\
\frac{dW}{dt} &= q\BE \cdot \frac{d\Bx}{dt} = c \frac{dp_0}{dt} \\
\frac{dp_0}{dt} &= q\BE \cdot \Bv/c \\
\frac{dp_0}{d\tau} &=
\mathLabelBox
[
   labelstyle={xshift=2cm, yshift=0.5cm},
   linestyle={out=270,in=90, latex-}
]
{\frac{dt}{d\tau}}{\(v/c \cdot \gamma_0\)} q\BE \cdot \left( \frac{v \wedge \gamma_0}{v \cdot \gamma_0} \right) \\
                   &= q\BE \cdot \left(v/c \wedge \gamma_0\right) \\
                   &= q \left(\BE + c I \BB\right) \cdot \left(v/c \wedge \gamma_0\right) \\
\end{aligned}
\end{equation}

\(I\BB\) has only purely spatial bivectors, \(\gamma_{12}\), \(\gamma_{13}\), and \(\gamma_{23}\).  On the other hand \(v \wedge \gamma_0 = \sum v^i \gamma_{i0}\) has only spacetime bivectors, so \(I\BB \cdot (v/c \wedge \gamma_0) = 0\), which is why it can be added above to complete the field.

That leaves:

\begin{equation}\label{eqn:gafpLor:p0wedgedotwedge}
\frac{dp_0}{d\tau} = q F \cdot \left(v/c \wedge \gamma_0\right),
\end{equation}

but we want to put this in the same form as \eqnref{eqn:gafpLor:pvec}.  To do so, note how we can reduce the dot product of two bivectors:

\begin{equation}\label{eqn:gafpLorentz:160}
\begin{aligned}
( a \wedge b ) \cdot ( c \wedge d )
&= \gpgradezero{ ( a \wedge b ) ( c \wedge d ) } \\
&= \gpgradezero{ ( a \wedge b ) ( c d - c \cdot d ) } \\
&= \gpgradezero{ (( a \wedge b ) \cdot c) d + (( a \wedge b ) \wedge c) d } \\
&= (( a \wedge b ) \cdot c) \cdot d .
\end{aligned}
\end{equation}

Using this, and adding the result to \eqnref{eqn:gafpLor:pvec} we have:

\begin{equation*}
\dot{p} \cdot \gamma_0 + \dot{p} \wedge \gamma_0 = q (F \cdot v/c) \cdot \gamma_0 + q ( F \cdot v/c ) \wedge \gamma_0
\end{equation*}

Or
\begin{equation*}
\dot{p} \gamma_0 = q (F \cdot v/c) \gamma_0
\end{equation*}

Right multiplying by \(\gamma_0\) on both sides to cancel those terms we have our end result, the covariant form of the Lorentz proper force equation:

\begin{equation}\label{eqn:gafpLor:LorentzCovariant}
\dot{p} = q ( F \cdot v/c )
\end{equation}

\section{Lorentz force in terms of four potential}

If one expresses the Faraday bivector in terms of a spacetime curl of a potential vector:

\begin{equation}
F = \grad \wedge A,
\end{equation}

then inserting into \eqnref{eqn:gafpLor:LorentzCovariant} we have:

\begin{equation}\label{eqn:gafpLorentz:180}
\begin{aligned}
\dot{p}
&= q ( F \cdot v/c )  \\
&= q (\grad \wedge A) \cdot v/c \\
&= q \left( \grad (A \cdot v/c) - A (\grad \cdot v/c) \right)
\end{aligned}
\end{equation}

Let us look at that proper velocity divergence term:

\begin{equation}\label{eqn:gafpLorentz:200}
\begin{aligned}
\grad \cdot v/c
&= \inv{c} \left(\grad \cdot \frac{dx}{d\tau}\right) \\
&= \inv{c} \frac{d}{d\tau} \grad \cdot x \\
&= \inv{c} \frac{d}{d\tau} \sum \frac{\partial x^{\mu}}{\partial x^{\mu}} \\
&= \inv{c} \frac{d 4}{d\tau} \\
&= 0
\end{aligned}
\end{equation}

This leaves the proper Lorentz force expressible as the (spacetime) gradient of a scalar quantity:

\begin{equation}
\dot{p} = q \grad (A \cdot v/c)
\end{equation}

I believe this dot product is likely an invariant of electromagnetism.  Looking from the rest frame one has:

\begin{equation}
\dot{p} = q \grad A^0 = q \sum \gamma^{\mu} \partial_{\mu} A^0 = \sum E^i \gamma_i
\end{equation}

Wedging with \(\gamma_0\) to calculate \(\BE = \sum E^i \gamma_i\), we have:

\begin{equation*}
q \sum -\gamma_{i0} \partial_{i} A^0 = - q \spacegrad A^0
\end{equation*}

So we want to identify this component of the four vector potential with electrostatic potential:

\begin{equation}
A^0 = \phi
\end{equation}

\section{Explicit expansion of potential spacetime curl in components}

Having used the gauge condition \(\grad \cdot A = 0\), to express the Faraday bivector as a gradient, we should be able to
verify that this produces the familiar equations for \(\BE\), and \(\BB\) in terms of \(\phi\), and \(\BA\).

First lets do the electric field components, which are easier.

With \(F = E + icB = \grad \wedge A\), we calculate \(\BE = \sum \sigma_i E^i = \sum \gamma_{i0} E^i\).

\begin{equation}\label{eqn:gafpLorentz:220}
\begin{aligned}
E^i
&= F \cdot \left(\gamma^{0} \wedge \gamma^{i}\right) = F \cdot \gamma^{0i} \\
&= \left(\sum \gamma^{\mu} \partial_{\mu} \wedge \gamma_{\nu} A^{\nu} \right) \cdot \gamma^{0i} \\
&= \sum \partial_{\mu} A^{\nu} {\gamma^{\mu}}_{\nu} \cdot \gamma^{0i} \\
&= \partial_{0} A^{i} {\gamma^{0}}_{i} \cdot \gamma^{0i} + \partial_{i} A^{0} {\gamma^{i}}_{0} \cdot \gamma^{0i} \\
&= - \left(\partial_{0} A^{i} + \partial_{i} A^{0} \right) \\
\sum E^i \sigma_i
&= - \left(\partial_{ct} \sum \sigma_i A^{i} + \sum \sigma_i \partial_{i} A^{0} \right) \\
&= - \left( \inv{c} \frac{\partial \BA}{\partial t} + \spacegrad A^{0} \right) \\
\end{aligned}
\end{equation}

Again we see that we should identify \(A^0 = \phi\), and write:

\begin{equation}
\BE + \inv{c} \frac{\partial \BA}{\partial t} = -\spacegrad \phi
\end{equation}

Now, let us calculate the magnetic field components (setting \(c=1\) temporarily):

\begin{equation}\label{eqn:gafpLorentz:240}
\begin{aligned}
i \BB
&= \sigma_{123} \sum \sigma_i B^i \\
&= \sum \sigma_{123i} B^i \\
&= \sigma_{1231} B^1 + \sigma_{1232} B^2 + \sigma_{1233} B^3 \\
&= \sigma_{23} B^1 + \sigma_{31} B^2 + \sigma_{12} B^3 \\
&= \gamma_{2030} B^1 + \gamma_{3010} B^2 + \gamma_{1020} B^3 \\
&= \gamma_{32} B^1 + \gamma_{13} B^2 + \gamma_{21} B^3 \\
\end{aligned}
\end{equation}

Thus, we can calculate the magnetic field components with:
\begin{equation}\label{eqn:gafpLorentz:260}
\begin{aligned}
B^1 &= F \cdot \gamma^{23} \\
B^2 &= F \cdot \gamma^{31} \\
B^3 &= F \cdot \gamma^{12}
\end{aligned}
\end{equation}

Here the components of \(F\) of interest are: \(\gamma^i \wedge \gamma_j \partial_i A^j = -\gamma_{ij} \partial_i A^j\).

\begin{equation}\label{eqn:gafpLorentz:280}
\begin{aligned}
B^1 &= - \partial_2 A^3 \gamma_{23} \cdot \gamma^{23} - \partial_3 A^2 \gamma_{32} \cdot \gamma^{23} \\
B^2 &= - \partial_3 A^1 \gamma_{31} \cdot \gamma^{31} - \partial_1 A^3 \gamma_{13} \cdot \gamma^{31} \\
B^3 &= - \partial_1 A^2 \gamma_{12} \cdot \gamma^{12} - \partial_2 A^1 \gamma_{21} \cdot \gamma^{12} \\
\implies \\
B^1 &= \partial_2 A^3 - \partial_3 A^2 \\
B^2 &= \partial_3 A^1 - \partial_1 A^3 \\
B^3 &= \partial_1 A^2 - \partial_2 A^1 \\
\end{aligned}
\end{equation}

Or, with \(\BA = \sum \sigma_i A^i\) and \(\spacegrad = \sum \sigma_i \partial_i\), this is our familiar:

\begin{equation}
\BB = \spacegrad \cross \BA
\end{equation}

   %
% Copyright � 2012 Peeter Joot.  All Rights Reserved.
% Licenced as described in the file LICENSE under the root directory of this GIT repository.
%

%
%
\chapter{Lorentz force rotor formulation}
\index{Lorentz force!rotor}
\label{chap:electronRotor}
%\date{March 18, 2009.  electronRotor.tex}

\section{Motivation}

Both \citep{baylis-2007} and \citep{doran2003gap} cover rotor formulations
of the Lorentz force equation.  Work through some of this on my own to
better understand it.

\section{In terms of GA}

An active Lorentz transformation can be used to translate from the rest frame of a particle with worldline \(x\) to
an observer frame, as in

\begin{equation}\label{eqn:eRotor:LorentzTx}
\begin{aligned}
y &= \Lambda x \reverse{\Lambda}
\end{aligned}
\end{equation}

Here Lorentz transformation is used in the general sense, and can include both spatial rotation and boost effects, but satisfies \(\Lambda\reverse{\Lambda} = 1\).  Taking proper time derivatives we have

\begin{equation}\label{eqn:electronRotor:20}
\begin{aligned}
\ydot
&= \Lambdadot x \reverse{\Lambda} + \Lambda x \reverse{\Lambdadot} \\
&= \Lambda \left(\reverse{\Lambda} \Lambdadot\right) x \reverse{\Lambda} + \Lambda x \left(\reverse{\Lambdadot} \Lambda \right) \reverse{\Lambda} \\
\end{aligned}
\end{equation}

Since \(\reverse{\Lambda}\Lambda = \Lambda\reverse{\Lambda} = 1\) we also have

\begin{equation}\label{eqn:electronRotor:40}
\begin{aligned}
0 &= \Lambdadot\reverse{\Lambda} + \Lambda\reverse{\Lambdadot}  \\
0 &= \reverse{\Lambda}\Lambdadot + \reverse{\Lambdadot}\Lambda
\end{aligned}
\end{equation}

Here is where a bivector variable

\begin{equation}\label{eqn:electronRotor:60}
\begin{aligned}
\Omega/2 = \reverse{\Lambda} \Lambdadot
\end{aligned}
\end{equation}

is introduced, from which we have \(\reverse{\Lambdadot} \Lambda = -\Omega/2\), and

\begin{equation}\label{eqn:electronRotor:80}
\begin{aligned}
\ydot &= \inv{2} \left( \Lambda \Omega x \reverse{\Lambda} - \Lambda x \Omega \reverse{\Lambda} \right) \\
\end{aligned}
\end{equation}

Or
\begin{equation}\label{eqn:electronRotor:100}
\begin{aligned}
\reverse{\Lambda} \ydot \Lambda &= \inv{2} \left( \Omega x - x \Omega \right) \\
\end{aligned}
\end{equation}

The inclusion of the factor of two in the definition of \(\Omega\) was cheating, so that we get the bivector vector dot product above.  Presuming \(\Omega\) is really a bivector (return to this in a bit), we then have

\begin{equation}\label{eqn:electronRotor:120}
\begin{aligned}
\reverse{\Lambda} \ydot \Lambda &= \Omega \cdot x
\end{aligned}
\end{equation}

We can express the time evolution of \(y\) using this as a stepping stone, since we have

\begin{equation}\label{eqn:electronRotor:140}
\begin{aligned}
\reverse{\Lambda} y \Lambda &= x
\end{aligned}
\end{equation}

Using this we have
\begin{equation}\label{eqn:electronRotor:160}
\begin{aligned}
0
&= \gpgradeone{ \reverse{\Lambda} \ydot \Lambda - \Omega \cdot x } \\
&= \gpgradeone{ \reverse{\Lambda} \ydot \Lambda - \Omega x } \\
&= \gpgradeone{ \reverse{\Lambda} \ydot \Lambda - \Omega \reverse{\Lambda} y \Lambda } \\
&= \gpgradeone{ \left( \reverse{\Lambda} \ydot - \reverse{\Lambda} \Lambda \Omega \reverse{\Lambda} y \right) \Lambda } \\
&= \gpgradeone{ \reverse{\Lambda} \left( \ydot - \Lambda \Omega \reverse{\Lambda} y \right) \Lambda } \\
\end{aligned}
\end{equation}

So we have the complete time evolution of our observer frame worldline for the particle, as a sort of an eigenvalue
equation for the proper time differential operator

\begin{equation}\label{eqn:electronRotor:180}
\begin{aligned}
\ydot
&= \left( \Lambda \Omega \reverse{\Lambda} \right) \cdot y = \left( 2 \Lambdadot \reverse{\Lambda} \right) \cdot y
\end{aligned}
\end{equation}

Now, what
\href{http://www.ime.unicamp.br/%7Eicca8/videos/baylis.avi}{Baylis did in his lecture}, and what Doran/Lasenby did as
well in the text (but I did not understand it then when I read it the first time) was to identify this time evolution
in terms of Lorentz transform change with the Lorentz force.

Recall that the Lorentz force equation is

\begin{equation}\label{eqn:eRotor:LorentzForce}
\begin{aligned}
\vdot = \frac{e}{m c} F \cdot v
\end{aligned}
\end{equation}

where \(F = \BE + i c \BB\), like \(\Lambdadot\reverse{\Lambda}\) is also a bivector.  If we write the velocity worldline
of the particle in the lab frame in terms of the rest frame particle worldline as

\begin{equation}\label{eqn:electronRotor:200}
\begin{aligned}
v = \Lambda c t \gamma_0 \reverse{\Lambda}
\end{aligned}
\end{equation}

Then for the field \(F\) observed in the lab frame we are left with a differential equation
\(2 \Lambdadot \reverse{\Lambda} = e F / mc\)
for the Lorentz transformation
that produces the observed motion of the particle given the field that acts on it

\begin{equation}\label{eqn:eRotor:LorentzTxEvolution}
\begin{aligned}
\Lambdadot = \frac{e}{2 m c} F \Lambda
\end{aligned}
\end{equation}

Okay, good.  I understand now well enough what they have done to reproduce the end result (with the exception of my
result including a factor of \(c\) since they have worked with \(c=1\)).

\subsection{Omega bivector}

It has been assumed above that \(\Omega = 2 \reverse{\Lambda} \Lambdadot\) is a bivector.  One way to confirm this is by examining the grades of this product.  Two bivectors, not necessarily related can only have grades 0, 2, and 4.  Because \(\Omega = -\reverse{\Omega}\), as seen above, it can have no grade 0 or grade 4 parts.

While this is a powerful way to verify the bivector nature of this object it is fairly abstract.  To get a better feel for this, let us
consider this object in detail for a purely spatial rotation, such as
\begin{equation}\label{eqn:electronRotor:220}
\begin{aligned}
R_\theta(x) &= \Lambda x \reverse{\Lambda} \\
\Lambda &= \exp( -i n \theta/ 2 ) = \cos( \theta/ 2 ) - i n \sin( \theta/ 2 ),
\end{aligned}
\end{equation}
where \(n\) is a spatial unit bivector, \(n^2 = 1\), in the span of \(\{\sigma_k = \gamma_k \gamma_0\}\).

\subsubsection{Verify rotation form}
To verify that this has the appropriate action, by linearity two cases must be considered.
First is the action on \(n\) or the components of any vector in this direction.
\begin{equation}\label{eqn:electronRotor:240}
\begin{aligned}
R_\theta(n)
&= \Lambda n \reverse{\Lambda} \\
&= \left( \cos( \theta/ 2 ) - i n \sin( \theta/ 2 ) \right) n \reverse{\Lambda} \\
&= n \left( \cos( \theta/ 2 ) - i n \sin( \theta/ 2 ) \right) \reverse{\Lambda} \\
&= n \Lambda \reverse{\Lambda} \\
&= n.
\end{aligned}
\end{equation}

The rotation operator does not change any vector colinear with the axis of rotation (the normal).  For a
vector \(m\) that is perpendicular to axis of rotation \(n\) (ie: \(2 ( m \cdot n ) = mn + nm = 0 \)), we have
\begin{equation}\label{eqn:electronRotor:260}
\begin{aligned}
R_\theta(m)
&= \Lambda m \reverse{\Lambda} \\
&= \left( \cos( \theta/ 2 ) - i n \sin( \theta/ 2 ) \right) m \reverse{\Lambda} \\
&= \left( m \cos( \theta/ 2 ) - i (n m) \sin( \theta/ 2 ) \right) \reverse{\Lambda} \\
&= \left( m \cos( \theta/ 2 ) + i (m n) \sin( \theta/ 2 ) \right) \reverse{\Lambda} \\
&= m (\reverse{\Lambda})^2 \\
&= m \exp( i n \theta )
\end{aligned}
\end{equation}

This is a rotation of the vector \(m\) that lies in the \(i n\) plane by \(\theta\) as desired.

\subsubsection{The rotation bivector}

We want derivatives of the \(\Lambda\) object.

\begin{equation}\label{eqn:electronRotor:280}
\begin{aligned}
\Lambdadot
&= \frac{\thetadot}{2} \left( -\sin( \theta/ 2 ) - i n \cos( \theta/ 2 ) \right) - i \ndot \cos(\theta/2) \\
&= \frac{i n \thetadot}{2} \left( i n \sin( \theta/ 2 ) - \cos( \theta/ 2 ) \right) - i \ndot \cos(\theta/2) \\
&= -\inv{2} \exp( -i n \theta/2 ) {i n \thetadot} - i \ndot \cos(\theta/2) \\
\end{aligned}
\end{equation}

So we have

\begin{equation}\label{eqn:electronRotor:300}
\begin{aligned}
\Omega
&= 2 \reverse{\Lambda} \Lambdadot \\
&= -{i n \thetadot} - 2 \exp(i n \theta/2) i \ndot \cos(\theta/2) \\
&= -{i n \thetadot} - 2 \cos(\theta/2) \left( \cos(\theta/2) - i n \sin(\theta/2)  \right) i \ndot \\
&= -{i n \thetadot} - 2 \cos(\theta/2) \left( \cos(\theta/2) i \ndot + n \ndot \sin(\theta/2)  \right) \\
\end{aligned}
\end{equation}

Since \(n \cdot \ndot = 0\), we have \(n \ndot = n \wedge \ndot\), and sure enough all the terms are bivectors.  Specifically
we have

\begin{equation}\label{eqn:electronRotor:320}
\begin{aligned}
\Omega
&= -\thetadot(i n) - (1 + \cos\theta ) (i \ndot) - \sin\theta (n \wedge \ndot)
\end{aligned}
\end{equation}

\subsection{Omega bivector for boost}

TODO.

\section{Tensor variation of the Rotor Lorentz force result}

There is not anything in the initial Lorentz force rotor result that intrinsically requires geometric algebra.  At least until
one actually
wants to express the Lorentz transformation concisely in terms of half angle or boost rapidity exponentials.

In fact
the logic above is not much different than the approach used in \citep{TongDynamics} for rigid body motion.  Let us try this in matrix or tensor
form and see how it looks.

\subsection{Tensor setup}
\index{Lorentz force!tensor}

Before anything else some notation for the tensor work must be established.  Similar to \eqnref{eqn:eRotor:LorentzTx} write a Lorentz transformed vector as a
linear transformation.  Since we want only the matrix of this linear transformation with respect to a specific observer frame, the details
of the transformation can be omitted for now.  Write

\begin{equation}\label{eqn:electronRotor:340}
\begin{aligned}
y = \LL(x)
\end{aligned}
\end{equation}

and introduce an orthonormal frame \(\{\gamma_\mu\}\), and the corresponding reciprocal frame
\(\{\gamma^\mu\}\), where \(\gamma_\mu \cdot \gamma^\nu = {\delta_\mu}^\nu\).
In this basis, the relationship between the vectors becomes

\begin{equation}\label{eqn:electronRotor:360}
\begin{aligned}
y^\mu \gamma_\mu
&= \LL(x^\nu \gamma_\nu) \\
&= x^\nu \LL(\gamma_\nu) \\
\end{aligned}
\end{equation}

Or
\begin{equation}\label{eqn:electronRotor:380}
\begin{aligned}
y^\mu &= x^\nu \LL(\gamma_\nu) \cdot \gamma^\mu \\
\end{aligned}
\end{equation}

The matrix of the linear transformation can now be written as

\begin{equation}\label{eqn:electronRotor:400}
\begin{aligned}
{\Lambda_\nu}^\mu &= \LL(\gamma_\nu) \cdot \gamma^\mu
\end{aligned}
\end{equation}

and this can now be used to express the coordinate transformation in abstract index notation

\begin{equation}\label{eqn:electronRotor:420}
\begin{aligned}
y^\mu &= x^\nu {\Lambda_\nu}^\mu
\end{aligned}
\end{equation}

Similarly, for the inverse transformation, we can write

\begin{equation}\label{eqn:electronRotor:440}
\begin{aligned}
x &= \LL^{-1}(y) \\
{\ILambda_\nu}^\mu &= \LL^{-1}(\gamma_\nu) \cdot \gamma^\mu \\
x^\mu &= y^\nu {\ILambda_\nu}^\mu
\end{aligned}
\end{equation}

I have seen this expressed using primed indices and the same symbol \(\Lambda\) used for both the forward and inverse
transformation ... lacking skill in tricky index manipulation I have avoided such a notation because I will probably get it
wrong.  Instead different symbols for the two different matrices will be used here and \(\Pi\) was picked for the inverse
rather arbitrarily.

With substitution

\begin{equation}\label{eqn:electronRotor:460}
\begin{aligned}
y^\mu &= x^\nu {\Lambda_\nu}^\mu = (y^\alpha {\ILambda_\alpha}^\nu) {\Lambda_\nu}^\mu  \\
x^\mu &= y^\nu {\ILambda_\nu}^\mu = (x^\alpha {\Lambda_\alpha}^\nu) {\ILambda_\nu}^\mu
\end{aligned}
\end{equation}

the pair of explicit inverse relationships between the two matrices can be read off as

\begin{equation}\label{eqn:electronRotor:480}
\begin{aligned}
{\delta_\alpha}^\mu &= {\ILambda_\alpha}^\nu {\Lambda_\nu}^\mu = {\Lambda_\alpha}^\nu {\ILambda_\nu}^\mu
\end{aligned}
\end{equation}

\subsection{Lab frame velocity of particle in tensor form}

In tensor form we want to express the worldline of the particle in the lab frame coordinates.  That is

\begin{equation}\label{eqn:electronRotor:500}
\begin{aligned}
v
&= \LL(c t \gamma_0) \\
&= \LL(x^0 \gamma_0) \\
&= x^0 \LL(\gamma_0) \\
\end{aligned}
\end{equation}

Or
\begin{equation}\label{eqn:electronRotor:520}
\begin{aligned}
v^\mu
&= x^0 \LL(\gamma_0) \cdot \gamma^\mu \\
&= x^0 {\Lambda_0}^\mu
\end{aligned}
\end{equation}

\subsection{Lorentz force in tensor form}

The Lorentz force equation \eqnref{eqn:eRotor:LorentzForce} in tensor form will also be needed.  The bivector \(F\) is

\begin{equation}\label{eqn:electronRotor:540}
\begin{aligned}
F = \inv{2} F_{\mu\nu} \gamma^\mu \wedge \gamma^\nu
\end{aligned}
\end{equation}

So we can write

\begin{equation}\label{eqn:electronRotor:560}
\begin{aligned}
F \cdot v
&= \inv{2} F_{\mu\nu} (\gamma^\mu \wedge \gamma^\nu) \cdot \gamma_\alpha v^\alpha \\
&= \inv{2} F_{\mu\nu} (\gamma^\mu {\delta^\nu}_\alpha - \gamma^\nu {\delta^\mu}_\alpha) v^\alpha \\
&= \inv{2} (v^\alpha F_{\mu\alpha} \gamma^\mu -v^\alpha F_{\alpha\nu} \gamma^\nu )
\end{aligned}
\end{equation}

And
\begin{equation}\label{eqn:electronRotor:580}
\begin{aligned}
\vdot_\sigma
&= \frac{e}{m c} ( F \cdot v ) \cdot \gamma_\sigma \\
&= \frac{e}{2 m c} (v^\alpha F_{\mu\alpha} \gamma^\mu -v^\alpha F_{\alpha\nu} \gamma^\nu ) \cdot \gamma_\sigma \\
&= \frac{e}{2 m c} v^\alpha ( F_{\sigma\alpha} - F_{\alpha\sigma} ) \\
&= \frac{e}{m c} v^\alpha F_{\sigma\alpha} \\
\end{aligned}
\end{equation}

Or

\begin{equation}\label{eqn:electronRotor:600}
\begin{aligned}
\vdot^\sigma &= \frac{e}{m c} v^\alpha {F^\sigma}_\alpha
\end{aligned}
\end{equation}

\subsection{Evolution of Lab frame vector}

Given a lab frame vector with all the (proper) time evolution expressed via the Lorentz transformation

\begin{equation}\label{eqn:electronRotor:620}
\begin{aligned}
y^\mu
&= x^\nu {\Lambda_\nu}^\mu \\
\end{aligned}
\end{equation}

we want to calculate the derivatives as in the GA procedure

\begin{equation}\label{eqn:electronRotor:640}
\begin{aligned}
\ydot^\mu
&= x^\nu {\Lambdadot_\nu}^\mu \\
&= x^\alpha {\delta_\alpha}^\nu {\Lambdadot_\nu}^\mu \\
&= x^\alpha {\Lambda_\alpha}^\beta {\ILambda_\beta}^\nu {\Lambdadot_\nu}^\mu \\
\end{aligned}
\end{equation}

With \(y = v\), this is

\begin{equation}\label{eqn:electronRotor:660}
\begin{aligned}
\vdot^\sigma
&= v^\alpha {\ILambda_\alpha}^\nu {\Lambdadot_\nu}^\sigma \\
&= v^\alpha \frac{e}{m c} {F^\sigma}_\alpha
\end{aligned}
\end{equation}

So we can make the identification of the bivector field with the Lorentz transformation matrix

\begin{equation}\label{eqn:electronRotor:680}
\begin{aligned}
{\ILambda_\alpha}^\nu {\Lambdadot_\nu}^\sigma &= \frac{e}{m c} {F^\sigma}_\alpha
\end{aligned}
\end{equation}

With an additional summation to invert we have
\begin{equation}\label{eqn:electronRotor:700}
\begin{aligned}
{\Lambda_\beta}^\alpha {\ILambda_\alpha}^\nu {\Lambdadot_\nu}^\sigma &= {\Lambda_\beta}^\alpha \frac{e}{m c} {F^\sigma}_\alpha
\end{aligned}
\end{equation}
%{\delta_\beta}^\nu &= {\ILambda_\beta}^\alpha {\Lambda_\alpha}^\nu = {\Lambda_\beta}^\alpha {\ILambda_\alpha}^\nu

This leaves a tensor differential equation that will provide the complete time evolution of the lab frame worldline for the particle in the field

\begin{equation}\label{eqn:electronRotor:720}
\begin{aligned}
{{\Lambdadot}_\mu}^\nu &= \frac{e}{m c} {\Lambda_\mu}^\alpha {F^\nu}_\alpha
\end{aligned}
\end{equation}

This is the equivalent of the GA equation \eqnref{eqn:eRotor:LorentzTxEvolution}.  However, while the GA equation is directly integrable for constant \(F\), how to do this in the equivalent tensor formulation is not so clear.

Want to revisit this, and try to perform this integral in both forms, ideally
for both the simpler constant field case, as well as for a more general field.
Even better would be to be able to express \(F\) in terms of the current
density vector, and then treat the proper interaction of two charged particles.

\section{Gauge transformation for spin}

In the Baylis article \eqnref{eqn:eRotor:LorentzTxEvolution} is transformed as
\(\Lambda \rightarrow \Lambda_{\omega_0} \exp( -i \Be_3 \omega_0 \tau)\).

Using this we have

\begin{equation}\label{eqn:electronRotor:740}
\begin{aligned}
\Lambdadot
&\rightarrow \frac{d}{d\tau}\left(\Lambda_{\omega_0} \exp( -i \Be_3 \omega_0 \tau) \right) \\
&= \Lambdadot_{\omega_0} \exp( -i \Be_3 \omega_0 \tau)
- \Lambda_{\omega_0} ( i \Be_3 \omega_0 ) \exp( -i \Be_3 \omega_0 \tau)
\end{aligned}
\end{equation}

For the transformed \eqnref{eqn:eRotor:LorentzTxEvolution} this gives

\begin{equation}\label{eqn:electronRotor:760}
\begin{aligned}
\Lambdadot_{\omega_0} \exp( -i \Be_3 \omega_0 \tau)
- \Lambda_{\omega_0} ( i \Be_3 \omega_0 ) \exp( -i \Be_3 \omega_0 \tau)
&= \frac{e}{2 m c} F \Lambda_{\omega_0} \exp( -i \Be_3 \omega_0 \tau)
\end{aligned}
\end{equation}

Canceling the exponentials, and shuffling

\begin{equation}\label{eqn:eRotor:firstTry}
\begin{aligned}
\Lambdadot_{\omega_0} &= \frac{e}{2 m c} F \Lambda_{\omega_0} + \Lambda_{\omega_0} ( i \Be_3 \omega_0 )
\end{aligned}
\end{equation}

How does he commute the \(i\Be_3\) term with the Lorentz transform?  How about instead
transforming as
\(\Lambda \rightarrow \exp( -i \Be_3 \omega_0 \tau) \Lambda_{\omega_0}\).

Using this we have

\begin{equation}\label{eqn:electronRotor:780}
\begin{aligned}
\Lambdadot
&\rightarrow \frac{d}{d\tau}\left(
\exp( -i \Be_3 \omega_0 \tau)
\Lambda_{\omega_0}
\right) \\
&=
\exp( -i \Be_3 \omega_0 \tau)
\Lambdadot_{\omega_0}
-
( i \Be_3 \omega_0 ) \exp( -i \Be_3 \omega_0 \tau)
\Lambda_{\omega_0}
\end{aligned}
\end{equation}

then, the transformed \eqnref{eqn:eRotor:LorentzTxEvolution} gives

\begin{equation}\label{eqn:electronRotor:800}
\begin{aligned}
\exp( -i \Be_3 \omega_0 \tau)
\Lambdadot_{\omega_0}
-
( i \Be_3 \omega_0 ) \exp( -i \Be_3 \omega_0 \tau)
\Lambda_{\omega_0}
&= \frac{e}{2 m c} F
\exp( -i \Be_3 \omega_0 \tau)
\Lambda_{\omega_0}
\end{aligned}
\end{equation}

Multiplying by the inverse exponential, and shuffling, noting that \(\exp(i\Be_3\alpha)\) commutes with \(i\Be_3\), we have

\begin{equation}\label{eqn:electronRotor:820}
\begin{aligned}
\Lambdadot_{\omega_0}
&=
( i \Be_3 \omega_0 ) \Lambda_{\omega_0}
+ \frac{e}{2 m c}
\exp( i \Be_3 \omega_0 \tau)
F
\exp( -i \Be_3 \omega_0 \tau)
\Lambda_{\omega_0}  \\
&=
\frac{e}{2 m c} \left(
\frac{2 m c}{e} ( i \Be_3 \omega_0 )
+
\exp( i \Be_3 \omega_0 \tau)
F
\exp( -i \Be_3 \omega_0 \tau)
\right)
\Lambda_{\omega_0}
\end{aligned}
\end{equation}

So, if one writes \(F_{\omega_0} = \exp( i \Be_3 \omega_0 \tau) F \exp( -i \Be_3 \omega_0 \tau)\), then
the transformed differential equation for the Lorentz transformation takes the form

\begin{equation}\label{eqn:electronRotor:840}
\begin{aligned}
\Lambdadot_{\omega_0}
&=
\frac{e}{2 m c} \left(
\frac{2 m c}{e} ( i \Be_3 \omega_0 )
+
F_{\omega_0}
\right)
\Lambda_{\omega_0}
\end{aligned}
\end{equation}

This is closer to Baylis's equation 31.
Dropping \(\omega_0\) subscripts this is

\begin{equation}\label{eqn:electronRotor:860}
\begin{aligned}
\Lambdadot
&=
\frac{e}{2 m c} \left(
\frac{2 m c}{e} ( i \Be_3 \omega_0 )
+
F
\right)
\Lambda
\end{aligned}
\end{equation}

A phase change in the Lorentz transformation rotor has introduced an additional term, one that
Baylis appears to identify with the spin vector \(\BS\).  My way of getting there seems fishy, so I think that
I am missing something.

Ah, I see.  If we go back to \eqnref{eqn:eRotor:firstTry}, then with
\(\BS = \Lambda_{\omega_0} ( i \Be_3 ) \reverse{\Lambda}_{\omega_0}\) (an application of a Lorentz transform to the unit bivector for the \(\Be_2 \Be_3\) plane), one has

\begin{equation}\label{eqn:electronRotor:880}
\begin{aligned}
\Lambdadot_{\omega_0}
&= \inv{2} \left( \frac{e}{m c} F + 2 \omega_0 \BS \right) \Lambda_{\omega_0}
\end{aligned}
\end{equation}

   %
% Copyright � 2012 Peeter Joot.  All Rights Reserved.
% Licenced as described in the file LICENSE under the root directory of this GIT repository.
%

%
%
%\input{../peeter_prologue.tex}

\mychapter{(INCOMPLETE) Geometry of Maxwell radiation solutions}
\label{chap:radiationGeometry}

%\blogpage{http://sites.google.com/site/peeterjoot/math2009/radiationGeometry.pdf}
%\date{Aug 14, 2009}
%\revisionInfo{\(RCSfile: radiationGeometry.tex,v \) Last \(Revision: 1.10 \) \(Date: 2009/10/22 02:07:20 \)}

\beginArtWithToc
%\beginArtNoToc

\section{Motivation}

We have in GA multiple possible ways to parametrize an oscillatory time dependence for a radiation field.

This was going to be an attempt to systematically solve the resulting eigen-multivector problem, starting with the a \(I\zcap \omega t\) exponential time parametrization, but I got stuck part way.  Perhaps using a plain old \(I \omega t\) would work out better, but I have spent more time on this than I want for now.

\section{Setup.  The eigenvalue problem}

Again following Jackson \citep{jackson1975cew}, we use CGS units.  Maxwell's equation in these units, with \(F = \BE + I\BB/\sqrt{\mu\epsilon}\) is

\begin{equation}\label{eqn:radiationGeometry:foo1}
\begin{aligned}
0 &= (\spacegrad + \sqrt{\mu\epsilon} \partial_0) F
\end{aligned}
\end{equation}

With an assumed oscillatory time dependence

\begin{equation}\label{eqn:radiationGeometry:foo2}
\begin{aligned}
F =
\calF
e^{i\omega t}
\end{aligned}
\end{equation}

Maxwell's equation reduces to a multivariable eigenvalue problem

\begin{equation}\label{eqn:radiationGeometry:foo3}
\begin{aligned}
\spacegrad \calF &= - \calF i \lambda \\
\lambda &= \sqrt{\mu\epsilon} \frac{\omega}{c}
\end{aligned}
\end{equation}

We have some flexibility in picking the imaginary.  As well as a non-geometric imaginary \(i\) typically used for a phasor representation where we take real parts of the field, we have additional possibilities, two of which are

\begin{equation}\label{eqn:radiationGeometry:foo4}
\begin{aligned}
i &= \xcap\ycap\zcap = I \\
i &= \xcap \ycap = I \zcap
\end{aligned}
\end{equation}

The first is the spatial pseudoscalar, which commutes with all vectors and bivectors.  The second is the unit bivector for the transverse plane, here parametrized by duality using the perpendicular to the plane direction \(\zcap\).

Let us examine the geometry required of the object \(\calF\) for each of these two geometric modeling choices.

\section{Using the transverse plane bivector for the imaginary}

Assuming no prior assumptions about \(\calF\) let us allow for the possibility of scalar, vector, bivector and pseudoscalar components

\begin{equation}\label{eqn:radiationGeometry:foo5}
\begin{aligned}
F = e^{-I\zcap \omega t} ( F_0 + F_1 + F_2 + F_3 )
\end{aligned}
\end{equation}

Writing \(e^{-I\zcap \omega t} = \cos(\omega t) -I \zcap \sin(\omega t) = C_\omega -I \zcap S_\omega\), an expansion of this product separated into grades is

\begin{equation}\label{eqn:radiationGeometry:29}
\begin{aligned}
F &=
  C_\omega F_0 - I S_\omega (\zcap \wedge F_2) \\
&+ C_\omega F_1 - \zcap S_\omega (I F_3) + S_\omega (\zcap \cross F_1)  \\
&+ C_\omega F_2 - I \zcap S_\omega F_0 - I S_\omega (\zcap \cdot F_2) \\
&+ C_\omega F_3 - I S_\omega (\zcap \cdot F_1)
\end{aligned}
\end{equation}

By construction \(F\) has only vector and bivector grades, so a requirement for zero scalar and pseudoscalar for all \(t\) means that we have four immediate constraints (with \(\Bn \perp \zcap\).)

\begin{equation}\label{eqn:radiationGeometry:49}
\begin{aligned}
F_0 &= 0 & \\
F_3 &= 0 & \\
F_2 &= \zcap \wedge \Bm \\
F_1 &= \Bn
\end{aligned}
\end{equation}

Since we have the flexibility to add or subtract any scalar multiple of \(\zcap\) to \(\Bm\) we can write \(F_2 = \zcap \Bm\) where \(\Bm \perp \zcap\).  Our field can now be written as just

\begin{equation}\label{eqn:radiationGeometry:69}
\begin{aligned}
F &=
 C_\omega \Bn - I S_\omega (\zcap \wedge \Bn)  \\
&+ C_\omega \zcap \Bm - I S_\omega (\zcap \cdot (\zcap \Bm)) \\
\end{aligned}
\end{equation}

We can similarly require \(\Bn \perp \zcap\), leaving

\begin{equation}\label{eqn:radiationGeometry:foo7}
\begin{aligned}
F &= (C_\omega - I \zcap S_\omega ) \Bn  + (C_\omega - I \zcap S_\omega) \Bm \zcap
\end{aligned}
\end{equation}

So, just the geometrical constraints give us

\begin{equation}\label{eqn:radiationGeometry:foo6}
\begin{aligned}
F &= e^{-I\zcap \omega t}(\Bn + \Bm \zcap)
\end{aligned}
\end{equation}

The first thing to be noted is that this phasor representation utilizing for the imaginary the transverse plane bivector \(I\zcap\) cannot be the most general.  This representation allows for only transverse fields!  This can be seen two ways.  Computing the transverse and propagation field components we have

\begin{equation}\label{eqn:radiationGeometry:89}
\begin{aligned}
F_z
&= \inv{2}(F + \zcap F \zcap) \\
&=
\inv{2} e^{-I\zcap \omega t}( \Bn + \Bm \zcap + \zcap \Bn \zcap + \zcap \Bm \zcap \zcap) \\
&=
\inv{2} e^{-I\zcap \omega t}( \Bn + \Bm \zcap - \Bn - \Bm \zcap ) \\
&= 0
\end{aligned}
\end{equation}

The computation for the transverse field \(F_t = (F - \zcap F \zcap)/2\) shows that \(F = F_t\) as expected since the propagation component is zero.

Another way to observe this is from the split of \(F\) into electric and magnetic field components.  From \eqnref{eqn:radiationGeometry:foo7} we have

\begin{equation}\label{eqn:radiationGeometry:foo8}
\begin{aligned}
\BE &= \cos(\omega t) \Bm + \sin(\omega t) (\zcap \cross \Bm) \\
\BB &= \cos(\omega t) (\zcap \cross \Bn) - \sin(\omega t) \Bn
\end{aligned}
\end{equation}

The space containing each of the \(\BE\) and \(\BB\) vectors lies in the span of the transverse plane.  We also see that there is some potential redundancy in the representation visible here since we have four vectors describing this span \(\Bm\), \(\Bn\), \(\zcap \cross \Bm\), and \(\zcap \cross \Bn\), instead of just two.

\subsection{General wave packet}

If \eqnref{eqn:radiationGeometry:foo1} were a scalar equation for \(F(\Bx,t)\) it can be readily shown using Fourier transforms the field propagation in time given initial time description of the field is

\begin{equation}\label{eqn:radiationGeometry:foob1}
\begin{aligned}
F(\Bx, t) = \int \left( \inv{(2\pi)^3} \int F(\Bx', 0) e^{i\Bk \cdot (\Bx' -\Bx)} d^3 x \right) e^{i c \Bk t/ \sqrt{\mu\epsilon}} d^3 k
\end{aligned}
\end{equation}

In traditional complex algebra the vector exponentials would not be well formed.  We do not have the problem in the GA formalism, but this does lead to a contraction since the resulting \(F(\Bx,t)\) cannot be scalar valued.  However, by using this as a motivational tool, and
also using assumed structure for the discrete frequency infinite wavetrain phasor, we can guess that a transverse only (to \(z\)-axis) wave packet may be described by a single direction variant of the Fourier result above.  That is

\begin{equation}\label{eqn:radiationGeometry:foob2}
\begin{aligned}
F(\Bx, t) =
\inv{\sqrt{2\pi}} \int
e^{-I \zcap \omega t}
\calF(\Bx, \omega)
d\omega
\end{aligned}
\end{equation}

Since \eqnref{eqn:radiationGeometry:foob2} has the same form as the earlier single frequency phasor test solution, we now know that \(\calF\) is required to anticommute with \(\zcap\).  Application of Maxwell's equation to this test solution gives us

\begin{equation}\label{eqn:radiationGeometry:109}
\begin{aligned}
%\begin{align}\label{eqn:radiationGeometry:foob3}
(\spacegrad +\sqrt{\mu\epsilon} \partial_0) F(\Bx,t) &=
(\spacegrad +\sqrt{\mu\epsilon} \partial_0)
\inv{\sqrt{2\pi}} \int
\calF(\Bx, \omega)
e^{I \zcap \omega t}
d\omega \\
&=
\inv{\sqrt{2\pi}}\int
\left(\spacegrad \calF + \calF I \zcap \sqrt{\mu\epsilon} \frac{\omega}{c}\right)
e^{I \zcap \omega t}
d\omega
\end{aligned}
\end{equation}
%\end{align}

This means that \(\calF\) must satisfy the gradient eigenvalue equation for all \(\omega\)

\begin{equation}\label{eqn:radiationGeometry:foob4}
\begin{aligned}
\spacegrad \calF = -\calF I \zcap \sqrt{\mu\epsilon} \frac{\omega}{c}
\end{aligned}
\end{equation}

Observe that this is the single frequency problem of equation \eqnref{eqn:radiationGeometry:foo3}, so for mono-directional light we can consider the infinite wave train instead of a wave packet with no loss of generality.

\subsection{Applying separation of variables}

While this may not lead to the most general solution to the radiation problem, the transverse only propagation problem is still one of interest.  Let us see where this leads.  In order to reduce the scope of the problem by one degree of freedom, let us split out the \(\zcap\) component of the gradient, writing

\begin{equation}\label{eqn:radiationGeometry:fooc1}
\begin{aligned}
\spacegrad = \spacegrad_t + \zcap \partial_z
\end{aligned}
\end{equation}

Also introduce a product split for separation of variables for the \(z\) dependence.  That is

\begin{equation}\label{eqn:radiationGeometry:fooc2}
\begin{aligned}
\calF = G(x,y) Z(z)
\end{aligned}
\end{equation}

Again we are faced with the problem of too many choices for the grades of each of these factors.  We can pick one of these, say \(Z\), to have only scalar and pseudoscalar grades so that the two factors commute.  Then we have

\begin{equation}\label{eqn:radiationGeometry:129}
\begin{aligned}
(\spacegrad_t + \spacegrad_z) \calF = (\spacegrad_t G) Z + \zcap G \partial_z Z = -G Z I \zcap \lambda
\end{aligned}
\end{equation}

With \(Z\) in an algebra isomorphic to the complex numbers, it is necessarily invertible (and commutes with it is derivative).  Similar arguments to the grade fixing for \(\calF\) show that \(G\) has only vector and bivector grades, but does \(G\) have the inverse required to do the separation of variables?  Let us blindly suppose that we can do this (and if we can not we can probably fudge it since we multiply again soon after).  With some rearranging we have

\begin{equation}\label{eqn:radiationGeometry:fooc3}
\begin{aligned}
-\inv{G} \zcap (\spacegrad_t G + G I \zcap \lambda) = (\partial_z Z)\inv{Z} = \text{constant}
\end{aligned}
\end{equation}

We want to separately equate these to a constant.  In order to commute these factors we have only required that \(Z\) have only scalar and pseudoscalar grades, so for the constant let us pick an arbitrary element in this subspace.  That is

\begin{equation}\label{eqn:radiationGeometry:fooc4}
\begin{aligned}
(\partial_z Z)\inv{Z} = \alpha + k I
\end{aligned}
\end{equation}

The solution for the \(Z\) factor in the separation of variables is thus

\begin{equation}\label{eqn:radiationGeometry:fooc5}
\begin{aligned}
Z \propto e^{(\alpha + k I)z}
\end{aligned}
\end{equation}

For \(G\) the separation of variables gives us

\begin{equation}\label{eqn:radiationGeometry:fooc6}
\begin{aligned}
\spacegrad_t G + (G \zcap \lambda + \zcap G k) I + \zcap G \alpha = 0
\end{aligned}
\end{equation}

We have now reduced the problem to something like a two variable eigenvalue problem, where the differential operator to find eigenvectors for is the transverse gradient \(\spacegrad_t\).  We unfortunately have an untidy split of the eigenvalue into left and right hand factors.
%Can we relate \eqnref{eqn:radiationGeometry:fooc6} to anything we know?  The very simplest case is for constant \(G\) and no hyperbolic angle (\(\alpha=0\)).  A solution then requires \(\lambda = \sqrt{\mu\epsilon} \omega/c = k\).  This wave number and angular frequency dependency is familiar since we also had that when we started with the assumption of oscillatory \(z\) and \(t\) dependence.
%
%How do we solve this when \(G\) is non constant?

While the product \(GZ\) was transverse only, we have now potentially lost that nice property for \(G\) itself, and do not know if \(G\) is strictly commuting or anticommuting with \(\zcap\).  Assuming either possibility for now, we can split this multivector into transverse and propagation direction fields \(G = G_t + G_z\)

\begin{equation}\label{eqn:radiationGeometry:fooc7}
\begin{aligned}
G_t &= \inv{2}(G - \zcap G \zcap) \\
G_z &= \inv{2}(G + \zcap G \zcap)
\end{aligned}
\end{equation}

With this split, noting that \(\zcap G_t = -G_t \zcap\), and \(\zcap G_z = G_z \zcap\) a rearrangement of \eqnref{eqn:radiationGeometry:fooc6} produces

\begin{equation}\label{eqn:radiationGeometry:fooc8}
\begin{aligned}
(\grad_t + \zcap ((k-\lambda) I + \alpha)) G_t = -(\grad_t + \zcap ((k+\lambda) I + \alpha)) G_z
\end{aligned}
\end{equation}

How do we find the eigen multivectors \(G_t\) and \(G_z\)?  A couple possibilities come to mind (perhaps not encompassing all solutions).  One is for one of \(G_t\) or \(G_z\) to be zero, and the other to separately require both halves of \eqnref{eqn:radiationGeometry:fooc8} equal a constant, very much like separation of variables despite the fact that both of these functions \(G_t\) and \(G_z\) are functions of \(x\) and \(y\).  The easiest non-trivial path is probably letting both sides of \eqnref{eqn:radiationGeometry:fooc8} separately equal zero, so that we are left with two independent eigen-multivector problems to solve

\begin{equation}\label{eqn:radiationGeometry:fooc9}
\begin{aligned}
\grad_t G_t &= -\zcap ((k-\lambda) I + \alpha)) G_t \\
\grad_t G_z &= -\zcap ((k+\lambda) I + \alpha)) G_z
\end{aligned}
\end{equation}

Damn.  have to mull this over.  Do not know where to go with it.

%\EndArticle
%%\EndNoBibArticle

   %
% Copyright � 2012 Peeter Joot.  All Rights Reserved.
% Licenced as described in the file LICENSE under the root directory of this GIT repository.
%

%
%
%\input{../peeter_prologue.tex}

\chapter{Relativistic classical proton electron interaction}
\label{chap:nuclearInteraction}

%\blogpage{http://sites.google.com/site/peeterjoot/math2009/nuclearInteraction.pdf}
%\date{Sept 13, 2009}
%\revisionInfo{\(RCSfile: nuclearInteraction.tex,v \) Last \(Revision: 1.9 \) \(Date: 2009/10/22 02:07:20 \)}

\beginArtWithToc
%\beginArtNoToc

\section{Motivation}

The problem of a solving for the relativistically correct trajectories of classically interacting proton and electron is one that I have wanted to try for a while.  Conceptually this is just about the simplest interaction problem in electrodynamics (other than motion of a particle in a field), but it is not obvious to me how to even set up the right equations to solve.  I should have the tools now to at least write down the equations to solve, and perhaps solve them too.

Familiarity with Geometric Algebra, and the STA form of the Maxwell and Lorentz force equation will be assumed.  Writing \(F = \BE + c I \BB\) for the Faraday bivector, these equations are respectively

\begin{equation}\label{eqn:nuclearInteraction:boo1}
\begin{aligned}
\grad F &= J/\epsilon_0 c \\
m\frac{d^2 X}{d\tau} &= \frac{q}{c} F \cdot \frac{dX}{d\tau}
\end{aligned}
\end{equation}

%  To avoid confusion no use of \(F\) or \(\BP\) for force will be used here, instead using \(d\BP/d\tau\).
The possibility of self interaction will also be ignored here.  From what I have read this self interaction is more complex than regular two particle interaction.

\section{With only Coulomb interaction}

With just Coulomb (non-relativistic) interaction setup of the equations of motion for the relative vector difference between the particles is straightforward.  Let us write this out as a reference.  Whatever we come up with for the relativistic case should reduce to this at small velocities.

Fixing notation, lets write the proton and electron positions respectively by \(\Br_p\) and \(\Br_e\), the proton charge as \(Z e\), and the electron charge \(-e\).  For the forces we have

FIXME: picture

\begin{equation}\label{eqn:nuclearInteraction:hoo1}
\begin{aligned}
\text{Force on electron} &= m_e \frac{d^2 \Br_e}{dt^2} = - \inv{4 \pi \epsilon_0} Z e^2 \frac{\Br_e - \Br_p}{\Abs{\Br_e - \Br_p}^3} \\
\text{Force on proton} &= m_p \frac{d^2 \Br_p}{dt^2} = \inv{4 \pi \epsilon_0} Z e^2 \frac{\Br_e - \Br_p}{\Abs{\Br_e - \Br_p}^3}
\end{aligned}
\end{equation}

Subtracting the two after mass division yields the reduced mass equation for the relative motion

\begin{equation}\label{eqn:nuclearInteraction:hoo2}
\begin{aligned}
\frac{d^2 (\Br_e -\Br_p)}{dt^2} = - \inv{4 \pi \epsilon_0} Z e^2 \left( \inv{m_e} + \inv{m_p}\right) \frac{\Br_e - \Br_p}{\Abs{\Br_e - \Br_p}^3}
\end{aligned}
\end{equation}

This is now of the same form as the classical problem of two particle gravitational interaction, with the well known conic solutions.

\section{Using the divergence equation instead}

While use of the Coulomb force above provides the equation of motion for the relative motion of the charges, how to generalize this to the relativistic case is not entirely clear.  For the relativistic case we need to consider all of Maxwell's equations, and not just the divergence equation.  Let us back up a step and setup the problem using the divergence equation instead of Coulomb's law.  This is a bit closer to the use of all of Maxwell's equations.

To start off we need a discrete charge expression for the charge density, and can use the delta distribution to express this.

\begin{equation}\label{eqn:nuclearInteraction:boo2}
\begin{aligned}
0 = \int d^3 x \left( \spacegrad \cdot \BE - \inv{\epsilon_0} \left( Z e \delta^3(\Bx - \Br_p) - e \delta^3(\Bx - \Br_e) \right) \right)
\end{aligned}
\end{equation}

Picking a volume element that only encloses one of the respective charges gives us the Coulomb law for the field produced by those charges as above

\begin{equation}\label{eqn:nuclearInteraction:boo3}
\begin{aligned}
0 &= \int_{\text{Volume around proton only}} d^3 x \left( \spacegrad \cdot \BE_p - \inv{\epsilon_0} Z e \delta^3(\Bx - \Br_p) \right) \\
0 &= \int_{\text{Volume around electron only}} d^3 x \left( \spacegrad \cdot \BE_e + \inv{\epsilon_0} e \delta^3(\Bx - \Br_e) \right)
\end{aligned}
\end{equation}

Here \(\BE_p\) and \(\BE_e\) denote the electric fields due to the proton and electron respectively.  Ignoring the possibility of self interaction the Lorentz forces on the particles are

\begin{equation}\label{eqn:nuclearInteraction:31}
\begin{aligned}
\text{Force on proton/electron} = \text{charge of proton/electron times field due to electron/proton}
\end{aligned}
\end{equation}

In symbols, this is

\begin{equation}\label{eqn:nuclearInteraction:boo4}
\begin{aligned}
m_p \frac{d^2 \Br_p}{dt^2} &= Z e \BE_e \\
m_e \frac{d^2 \Br_e}{dt^2} &= - e \BE_p
\end{aligned}
\end{equation}

If we were to substitute back into the volume integrals we would have

\begin{equation}\label{eqn:nuclearInteraction:boo5}
\begin{aligned}
0 &= \int_{\text{Volume around proton only}} d^3 x \left( -\frac{m_e}{e}\spacegrad \cdot \frac{d^2 \Br_e}{dt^2} - \inv{\epsilon_0} Z e \delta^3(\Bx - \Br_p) \right) \\
0 &= \int_{\text{Volume around electron only}} d^3 x \left( \frac{m_p}{Z e}\spacegrad \cdot \frac{d^2 \Br_p}{dt^2} + \inv{\epsilon_0} e \delta^3(\Bx - \Br_e) \right)
\end{aligned}
\end{equation}

It is tempting to take the differences of these two equations so that we can write this in terms of the relative acceleration \(d^2 (\Br_e - \Br_p)/dt^2\).  I did just this initially, and was surprised by a mass term of the form \(1/m_e - 1/m_p\) instead of reduced mass, which cannot be right.  The key to avoiding this mistake is the proper considerations of the integration volumes.  Since the volumes are different and can in fact be entirely disjoint, subtracting these is not possible.  For this reason we have to be especially careful if a differential form of the divergence integrals \eqnref{eqn:nuclearInteraction:boo4} were to be used, as in

\begin{equation}\label{eqn:nuclearInteraction:boo6}
\begin{aligned}
\spacegrad \cdot \BE_p &= \inv{\epsilon_0} Z e \delta^3(\Bx - \Br_p) \\
\spacegrad \cdot \BE_e &= -\inv{\epsilon_0} e \delta^3(\Bx - \Br_e)
\end{aligned}
\end{equation}

The domain of applicability of these equations is no longer explicit, since each has to omit a neighborhood around the other charge.  When using a delta distribution to express the point charge density it is probably best to stick with an explicit integral form.

Comparing how far we can get starting with the Gauss's law instead of the Coulomb force, and looking forward to the relativistic case, it seems likely that solving the field equations due to the respective current densities will be the first required step.  Only then can we substitute that field solution back into the Lorentz force equation to complete the search for the particle trajectories.

\section{Relativistic interaction}

First order of business is an expression for a point charge current density four vector.  Following Jackson \citep{jackson1975cew}, but switching to vector notation from coordinates, we can apparently employ an arbitrary parametrization for the four-vector particle trajectory \(R = R^\mu \gamma_\mu\), as measured in the observer frame, and write

\begin{equation}\label{eqn:nuclearInteraction:boo7}
\begin{aligned}
J(X) = q c \int d\lambda \frac{dX}{d\lambda} \delta^4 (X - R(\lambda))
\end{aligned}
\end{equation}

Here \(X = X^\mu \gamma_\mu\) is the four vector event specifying the spacetime position of the current, also as measured in the observer frame.  Reparameterizating in terms of time should get us back something more familiar looking

\begin{equation}\label{eqn:nuclearInteraction:51}
\begin{aligned}
J(X)
&= q c \int dt \frac{dX}{dt} \delta^4 (X - R(t)) \\
&= q c \int dt \frac{d}{dt} (c t \gamma_0 + \gamma_k X^k)\delta^4 (X - R(t)) \\
&= q c \int dt \frac{d}{dt} (c t + \Bx)\delta^4 (X - R(t)) \gamma_0 \\
&= q c \int dt (c + \Bv)\delta^4 (X - R(t)) \gamma_0 \\
&= q c \int dt' (c + \Bv(t'))\delta^3 (\Bx - \Br(t')) \delta(c t' - c t) \gamma_0 \\
\end{aligned}
\end{equation}

Note that the scaling property of the delta function implies \(\delta(c t) = \delta(t)/c\).  With the split of the four-volume delta function \(\delta^4(X - R(t)) = \delta^3(\Bx - \Br(t)) \delta( {x^0}' - x^0 )\), where \(x^0 = c t\), we have an explanation for why Jackson had a factor of \(c\) in his representation.  I initially thought this factor of \(c\) was due to CGS vs SI units!  One more Jackson equation decoded.  We are left with the following spacetime split for a point charge current density four vector

\begin{equation}\label{eqn:nuclearInteraction:boo11}
\begin{aligned}
J(X)
= q (c + \Bv(t))\delta^3 (\Bx - \Br(t)) \gamma_0
\end{aligned}
\end{equation}

Comparing to the continuous case where we have \(J = \rho ( c + \Bv ) \gamma_0\), it appears that this works out right.  One thing worth noting is that in this time reparameterization I accidentally mixed up \(X\), the observation event coordinates of \(J(X)\), and \(R\), the spacetime trajectory of the particle itself.  Despite this, I am saved by the delta function since no contributions to the current can occur on trajectories other than \(R\), the worldline of the particle itself.  So in the final result it should be correct to interpret \(\Bv\) as the spatial particle velocity as I did accidentally.

With the time reparameterization of the current density, we have for the field due to our proton and electron

\begin{equation}\label{eqn:nuclearInteraction:boo8}
\begin{aligned}
0 = \int d^3 x \left( \epsilon_0 c \grad F - Z e (c + \Bv_p(t))\delta^3 (\Bx - \Br_p(t)) + e (c + \Bv_e(t))\delta^3 (\Bx - \Br_e(t)) \gamma_0 \right)
\end{aligned}
\end{equation}

How to write this in a more tidy covariant form?  If we reparametrize with any of the other spatial coordinates, say \(x\) we end up having to integrate the field gradient with a spacetime three form (\(dt dy dz\) if parametrizing the current density with \(x\)).  Since the entire equation must be zero I suppose we can just integrate that once more, and simply write

\begin{equation}\label{eqn:nuclearInteraction:boo9}
\begin{aligned}
\text{constant} = \int d^4 x \left( \grad F - \frac{e}{\epsilon_0 c}\int d\tau \frac{dX}{d\tau} \left( Z \delta^4 (X - R_p(\tau)) - \delta^4 (X - R_e(\tau)) \right) \right)
\end{aligned}
\end{equation}

Like \eqnref{eqn:nuclearInteraction:boo3} we can pick spacetime volumes that surround just the individual particle worldlines, in which case we have a Coulomb's law like split where the field depends on just the enclosed current.  That is

\begin{equation}\label{eqn:nuclearInteraction:boo10}
\begin{aligned}
\text{constant} &= \int_{\text{spacetime volume around only the proton}} d^4 x \left( \grad F_p - \frac{Z e}{\epsilon_0 c} \int d\tau \frac{dX}{d\tau} \delta^4 (X - R_e(\tau)) \right) \\
\text{constant} &= \int_{\text{spacetime volume around only the electron}} d^4 x \left( \grad F_e + \frac{e}{\epsilon_0 c} \int d\tau \frac{dX}{d\tau} \delta^4 (X - R_e(\tau)) \right)
\end{aligned}
\end{equation}

Here \(F_e\) is the field due to only the electron charge, whereas \(F_p\) would be that part of the total field due to the proton charge.

FIXME: attempt to draw a picture (one or two spatial dimensions) to develop some comfort with tossing out a phrase like ``spacetime volume surrounding a particle worldline''.

Having expressed the equation for the total field \eqnref{eqn:nuclearInteraction:boo9}, we are tracking a nice parallel to the setup for the non-relativistic treatment.  Next is the pair of Lorentz force equations.  As in the non-relativistic setup, if we only consider the field due to the other charge we have in covariant Geometric Algebra form, the following pair of proper force equations in terms of the particle worldline trajectories
\begin{equation}\label{eqn:nuclearInteraction:hoo6}
\begin{aligned}
\text{proper Force on electron} &= m_e \frac{d^2 R_e}{d\tau^2} = - e F_p \cdot \frac{d R_e}{c d\tau} \\
\text{proper Force on proton} &= m_p \frac{d^2 R_p}{d\tau^2} = Z e F_e \cdot \frac{d R_p}{c d\tau}
\end{aligned}
\end{equation}

We have the four sets of coupled multivector equations to be solved, so the question remains how to do so.  Each of the two Lorentz force equations supplies four equations with four unknowns, and the field equations are really two sets of eight equations with six unknown field variables each.  Then they are all tied up together is a big coupled mess.  Wow.  How do we solve this?

With \eqnref{eqn:nuclearInteraction:boo10}, and \eqnref{eqn:nuclearInteraction:hoo6} committed to pdf at least the first goal of writing down the equations is done.

As for the actual solution.  Well, that is a problem for another night.  TO BE CONTINUED (if I can figure out an attack).

%\section{BLAH: Relativistic interaction}
%
%Any self interaction effects will not be o
%Since we are not considering the effect of
%The superposition of the fields from the two point particles wou
%We seek the individual fields
%For the relativistic case, lets write the proton and electron worldlines in the observer frame (the origin) respectively by \(X_p = (ct, \Br_p)\) and \(X_e = (ct,\Br_e)\), the observer frame proper time as \(\tau\).  We should be able to calculate the total field at any point in space with superposition of the individual fields
%
%\begin{align}\label{eqn:nuclearInteraction:hoo3}
%\text{Field due to electron acting on proton} &= F_e \equiv \BE_e + c I\BB_e \\
%\text{Field due to proton acting on electron} &= F_p \equiv \BE_p + c I\BB_p
%\end{align}
%
%
%
%It should be possible to obtain the fields \(F_e\) and \(F_p\) by solving for the pair of Maxwell's equations
%
%\begin{align}\label{eqn:nuclearInteraction:hoo4}
%\grad F_e &= J_e/\epsilon_0 c \\
%\grad F_p &= J_p/\epsilon_0 c
%\end{align}
%
%Here the \(J\)'s are the four vector current densities, each dependent on the particle trajectories, also to be determined.
%
%Unfortunately we do not want the gradients of the fields, but the fields themselves, so life is made more complex.  That issue was avoided in the Coulomb case since we started not with \(\spacegrad \cdot \BE = \rho/\epsilon_0\), but the solution to this divergence equation (the only part of Maxwell's equations left in the static limit).
%
%
%
%
%
%With the question of how to solve or even express the respective fields sidestepped for now, we can at least express the proper force Lorentz interactions.  That is
%
%\begin{align}\label{eqn:nuclearInteraction:hoo5}
%J = Q \int d\tau' \frac{dX'}{d\tau'} \delta^4 (X' - X(\tau))
%\end{align}
%
%
%How to express \(J_e\) and \(J_p\) will have to be considered more carefully, and
%This is a set of second order four-vector equations, really eight equations, further coupled by the addition pair of eight equations for each of the fields.  What a mess!  So, where do we start?
%
%One possible starting point is to encode all of the particle tragectory in an active Lorentz transformation and solve for that transformation.  This technique was used successfully in \citep{doran2003gap} for the single particle in a field problem, so it seems worthwhile to at least give it a try.
%
%Suppose we relate our electron and proton event vector at two different proper times by a Lorentz transforms
%
%\begin{align*}
%X_e' &= X_e(\tau_0 + d\tau) = \tilde{R_e} X_e(\tau_0) R_e = X_e(\tau_0) + d\tau {\left. \frac{dX_e}{d\tau} \right\vert}_{\tau = \tau_0} + \cdots \\
%X_p' &= X_p(\tau_0 + d\tau) = \tilde{R_p} X_p(\tau_0) R_p = X_p(\tau_0) + d\tau {\left. \frac{dX_p}{d\tau} \right\vert}_{\tau = \tau_0} + \cdots
%\end{align*}
%
%%\begin{align}\label{eqn:nuclearInteraction:hoo6}
%%X' = X(\tau_0 + d\tau) = \tilde{R}(\tau) X(\tau_0) R(\tau)
%%\end{align}
%%
%Inverting and taking derivatives, also utilizing \(\tilde{R}{R} = 1\), we find the proper velocity expressed in terms of the commutator of the bivector \(\Omega = (dR/d\tau) \tilde{R}\).   That is
%
%\begin{align*}
%\frac{dX}{d\tau}
%&= \frac{d}{d\tau} \left( R X' \tilde{R} \right) \\
%&= \frac{dR }{d\tau} \tilde{R} X + X \mathLabelBox{R \frac{d\tilde{R}}{d\tau}}{\(= - (dR/d\tau) \tilde{R}\)} \\
%&= \antisymmetric{ \Omega }{X}
%\end{align*}
%

%\EndArticle
%%\EndNoBibArticle


\part{Electrodynamics Stress Energy}
   %
% Copyright � 2012 Peeter Joot.  All Rights Reserved.
% Licenced as described in the file LICENSE under the root directory of this GIT repository.
%

%
%
\mychapter{Poynting vector and Electromagnetic Energy conservation}
\label{chap:PJpoynting}
\index{Poynting vector}
\index{energy conservation!electromagnetic}
%\date{Dec 29, 2008.  poynting.tex}

\section{Motivation}

Clarify Poynting discussion from \citep{doran2003gap}.

Equation 7.59 and 7.60 derives a \(\BE \cross \BB\) quantity, the Poynting vector, as a sort of energy flux through the surface of the containing volume.

There are a couple of magic steps here that were not at all obvious to me.  Go through this in enough detail that it makes sense to me.

\section{Charge free case}

In SI units the Energy density is given as

\begin{equation}\label{eqn:poynting:energy}
\begin{aligned}
U = \frac{\epsilon_0}{2}\left( \BE^2 + c^2 \BB^2 \right)
\end{aligned}
\end{equation}

In \chapcite{PJelectricFieldEnergy} the electrostatic energy portion of this
energy was observed.
FIXME: A magnetostatics derivation (ie: unchanging currents)
is possible for the \(\BB^2\) term, but I have not done this myself yet.

It is somewhat curious that the total field energy is just this
sum without any cross terms (all those cross terms show up in the
field momentum).  A logical confirmation of this in a general
non-electrostatics and non-magnetostatics context will not be done here.
Instead it will be assumed that \eqnref{eqn:poynting:energy} has been correctly identified
as the field energy (density), and a mechanical calculation of the time
rate of change of this
quantity (the power density) will be performed.  In doing so we can find the
analogue of the momentum.  How to truly identify this quantity with momentum
will hopefully become clear as we work with it.

Given this energy density the rate of change of energy in a volume is then

\begin{equation}\label{eqn:poynting:20}
\begin{aligned}
\frac{dU}{dt}
&=
\frac{d}{dt}
\frac{\epsilon_0}{2} \int dV \left( \BE^2 + c^2 \BB^2 \right) \\
&=
\epsilon_0 \int dV \left( \BE \cdot \PD{t}{\BE} + c^2 \BB \cdot \PD{t}{\BB} \right) \\
\end{aligned}
\end{equation}

The next (omitted in the text) step is to utilize Maxwell's equation to eliminate the time derivatives.  Since this is the
charge and current free case, we can write Maxwell's as

\begin{equation}\label{eqn:poynting:40}
\begin{aligned}
0
&= \gamma_0 \grad F \\
&= \gamma_0 (\gamma^0 \partial_0 + \gamma^k \partial_k) F \\
&= (\partial_0 + \gamma_k\gamma_0 \partial_k) F \\
&= (\partial_0 + \sigma_k \partial_k) F \\
&= (\partial_0 + \spacegrad)F \\
&= (\partial_0 + \spacegrad)(\BE + ic \BB) \\
&= \partial_0 \BE + ic \partial_0 \BB + \spacegrad \BE + ic \spacegrad \BB \\
\end{aligned}
\end{equation}

In the spatial (\(\sigma\)) basis we can separate this into even and odd grades, which are separately equal to zero

\begin{equation}\label{eqn:poynting:60}
\begin{aligned}
0 &= \partial_0 \BE + ic \spacegrad \BB \\
%   1                  3,1
0 &= ic \partial_0 \BB + \spacegrad \BE
%  2                    0,2
\end{aligned}
\end{equation}

A selection of just the vector parts is

\begin{equation}\label{eqn:poynting:80}
\begin{aligned}
\partial_t \BE &= - ic^2 \spacegrad \wedge \BB \\
\partial_t \BB &= i\spacegrad \wedge \BE
\end{aligned}
\end{equation}

Which can be back substituted into the energy flux
\begin{equation}\label{eqn:poynting:100}
\begin{aligned}
\frac{dU}{dt}
&= \epsilon_0 \int dV \left( \BE \cdot (-i c^2 \spacegrad \wedge \BB) + c^2 \BB \cdot (i \spacegrad \wedge \BE) \right) \\
&= \epsilon_0 c^2 \int dV \gpgradezero{ \BB i \spacegrad \wedge \BE -\BE i \spacegrad \wedge \BB } \\
\end{aligned}
\end{equation}

Since the two divergence terms are zero we can drop the wedges here for

\begin{equation}\label{eqn:poynting:120}
\begin{aligned}
\frac{dU}{dt}
&= \epsilon_0 c^2 \int dV \gpgradezero{ \BB i \spacegrad \BE -\BE i \spacegrad \BB } \\
&= \epsilon_0 c^2 \int dV \gpgradezero{ (i \BB) \spacegrad \BE -\BE \spacegrad (i\BB) } \\
&= \epsilon_0 c^2 \int dV \spacegrad \cdot ( (i \BB) \cdot \BE ) \\
\end{aligned}
\end{equation}

Justification for this last step can be found below in the derivation of \eqnref{eqn:poynting:poyntingDivergence}.

We can now use Stokes theorem to change this into a surface integral for a final energy flux

\begin{equation}\label{eqn:poynting:140}
\begin{aligned}
\frac{dU}{dt}
&= \epsilon_0 c^2 \int d\BA \cdot ( (i \BB) \cdot \BE ) \\
\end{aligned}
\end{equation}

This last bivector/vector dot product is the Poynting vector

\begin{equation}\label{eqn:poynting:160}
\begin{aligned}
(i \BB) \cdot \BE
&= \gpgradeone{ (i \BB) \cdot \BE } \\
&= \gpgradeone{ i \BB \BE } \\
&= \gpgradeone{ i (\BB \wedge \BE) } \\
&= i (\BB \wedge \BE) \\
&= i^2(\BB \cross \BE) \\
&= \BE \cross \BB \\
\end{aligned}
\end{equation}

So, we can identity the quantity

\begin{equation}\label{eqn:poynting:poynting}
\begin{aligned}
\BP = \epsilon_0 c^2 \BE \cross \BB = \epsilon_0 c (i c \BB) \cdot \BE
\end{aligned}
\end{equation}

as a directed energy density flux through the surface of a containing volume.

\section{With charges and currents}

To calculate time derivatives we want to take Maxwell's equation and put into a form with explicit time derivatives, as was done before, but this time be more careful with the handling of the four vector current term.  Starting with left factoring out of a \(\gamma_0\) from the spacetime gradient.

\begin{equation}\label{eqn:poynting:180}
\begin{aligned}
\grad &= \gamma^0 \partial_0 + \gamma^k \partial_k \\
&= \gamma^0 (\partial_0 - \gamma^k \gamma_0 \partial_k) \\
&= \gamma^0 (\partial_0 + \sigma_k \partial_k) \\
\end{aligned}
\end{equation}

Similarly, the \(\gamma_0\) can be factored from the current density

\begin{equation}\label{eqn:poynting:200}
\begin{aligned}
J
&= \gamma_0 c \rho + \gamma_k J^k \\
&= \gamma_0 (c \rho - \gamma_k \gamma_0 J^k) \\
&= \gamma_0 (c \rho - \sigma_k J^k) \\
&= \gamma_0 (c \rho - \Bj )
\end{aligned}
\end{equation}

With this Maxwell's equation becomes

\begin{equation}\label{eqn:poynting:220}
\begin{aligned}
\gamma_0 \grad F &= \gamma_0 J / \epsilon_0 c \\
(\partial_0 + \spacegrad) ( \BE + i c \BB ) &= \rho/\epsilon_0 - \Bj/\epsilon_0 c \\
\end{aligned}
\end{equation}

A split into even and odd grades including current and charge density is thus

\begin{equation}\label{eqn:poynting:240}
\begin{aligned}
\spacegrad \BE + \partial_t (i \BB) &= \rho/\epsilon_0 \\
\spacegrad (i \BB) c^2 + \partial_t \BE &= -\Bj/\epsilon_0
\end{aligned}
\end{equation}

Now, taking time derivatives of the energy density gives

\begin{equation}\label{eqn:poynting:260}
\begin{aligned}
\PD{t}{U}
&= \PD{t}{}\inv{2} \epsilon_0 \left( \BE^2 - (ic \BB)^2 \right) \\
&= \epsilon_0 \left( \BE \cdot \partial_t \BE - c^2 (i\BB) \cdot \partial_t (i\BB) \right) \\
&= \epsilon_0 \gpgradezero{ \BE ( -\Bj/\epsilon_0 -\spacegrad (i \BB) c^2 ) - c^2 (i\BB) ( -\spacegrad \BE + \rho/\epsilon_0 ) } \\
&= -\BE \cdot \Bj + c^2 \epsilon_0 \gpgradezero{ i\BB \spacegrad \BE -\BE \spacegrad (i \BB) } \\
&= -\BE \cdot \Bj + c^2 \epsilon_0 \left( (i\BB) \cdot (\spacegrad \wedge \BE) - \BE \cdot (\spacegrad \cdot (i \BB)) \right) \\
\end{aligned}
\end{equation}

Using \eqnref{eqn:poynting:poyntingDivergence}, we now have the rate of change of
field energy for the general case including currents.  That is

\begin{equation}\label{eqn:poynting:280}
\begin{aligned}
\PD{t}{U} &= -\BE \cdot \Bj + c^2 \epsilon_0 \spacegrad \cdot (\BE \cdot (i\BB))
\end{aligned}
\end{equation}

Written out in full, and in terms of the Poynting vector this is

\begin{equation}\label{eqn:poynting:300}
\begin{aligned}
\PD{t}{}\frac{\epsilon_0}{2} \left(\BE^2 + c^2 \BB^2\right) + c^2 \epsilon_0 \spacegrad \cdot (\BE \cross \BB) &= -\BE \cdot \Bj
\end{aligned}
\end{equation}

\section{Poynting vector in terms of complete field}

In \eqnref{eqn:poynting:poynting} the individual parts of the complete Faraday
bivector \(F = \BE + i c \BB\) stand out.  How would the Poynting vector be
expressed in terms of \(F\) or in tensor form?

One possibility is to write \(\BE \cross \BB\) in terms of F using
a conjugate split of the Maxwell bivector

\begin{equation}\label{eqn:poynting:320}
\begin{aligned}
F \gamma_0 = - \gamma_0(\BE - i c \BB)
\end{aligned}
\end{equation}

we have
\begin{equation}\label{eqn:poynting:340}
\begin{aligned}
\gamma^0 F \gamma_0 = - (\BE - i c \BB)
\end{aligned}
\end{equation}

and
\begin{equation}\label{eqn:poynting:360}
\begin{aligned}
i c \BB &= \inv{2}(F + \gamma^0 F \gamma_0) \\
\BE &= \inv{2}(F - \gamma^0 F \gamma_0) \\
\end{aligned}
\end{equation}

However \citep{doran2003gap} has the answer more directly in terms of the
electrodynamic stress tensor.

\begin{equation}\label{eqn:poynting:380}
\begin{aligned}
T(a) = -\frac{\epsilon_0}{2} F a F
\end{aligned}
\end{equation}

In particular for \(a = \gamma_0\), this is

\begin{equation}\label{eqn:poynting:400}
\begin{aligned}
T(\gamma_0)
&= -\frac{\epsilon_0}{2} F \gamma_0 F \\
&= \frac{\epsilon_0}{2} (\BE + i c\BB) (\BE - i c \BB) \gamma_0 \\
&= \frac{\epsilon_0}{2} (\BE^2 + c^2 \BB^2 + i c (\BB \BE -\BB \BE)) \gamma_0 \\
&= \frac{\epsilon_0}{2} (\BE^2 + c^2 \BB^2) \gamma_0 + i c \epsilon_0 (\BB \wedge \BE) \gamma_0 \\
&= \frac{\epsilon_0}{2} (\BE^2 + c^2 \BB^2) \gamma_0 + c \epsilon_0 (\BE \cross \BB) \gamma_0 \\
\end{aligned}
\end{equation}

So one sees that the energy and the Poynting vector are components of an energy density momentum four vector

\begin{equation}\label{eqn:poynting:420}
\begin{aligned}
T(\gamma_0)
&= U \gamma_0 + \inv{c} \BP \gamma_0 \\
\end{aligned}
\end{equation}

Writing \(U^0 = U\) and \(U^k = P^k/c\), this is \(T(\gamma_0) = U^\mu \gamma_\mu\).

(inventing such a four vector is how Doran/Lasenby started, so this is not be too surprising).  This
relativistic context helps justify the Poynting vector as a momentum like quantity, but is not quite
satisfactory.  It would make sense to do some classical comparisons, perhaps of interacting wave functions
or something like that, to see how exactly this quantity is momentum like.  Also how exactly is this energy
momentum tensor used, how does it transform, ...

\section{Energy Density from Lagrangian?}

I did not get too far trying to calculate the electrodynamic Hamiltonian density for the general case, so I tried it for a very
simple special case, with just an electric field component in one direction:

\begin{equation}\label{eqn:poynting:440}
\begin{aligned}
\calL
&= \frac{1}{2}(E_x)^2 \\
&= \frac{1}{2}(F_{01})^2 \\
&= \frac{1}{2}(\partial_0 A_1 - \partial_1 A_0)^2 \\
\end{aligned}
\end{equation}

\citep{goldstein1951cm} gives the Hamiltonian density as

\begin{equation}\label{eqn:poynting:460}
\begin{aligned}
\pi &= \frac{\partial \calL}{\partial \dot{n}} \\
\calH &= \dot{n} \pi - \calL
\end{aligned}
\end{equation}

If I try calculating this I get

\begin{equation}\label{eqn:poynting:480}
\begin{aligned}
\pi
&= \frac{\partial}{\partial (\partial_0 A_1)} \left( \frac{1}{2}(\partial_0 A_1 - \partial_1 A_0)^2 \right) \\
&= \partial_0 A_1 - \partial_1 A_0 \\
&= F_{01} \\
\end{aligned}
\end{equation}

So this gives a Hamiltonian of
\begin{equation}\label{eqn:poynting:500}
\begin{aligned}
\calH
&= \partial_0 A_1 F_{01} - \frac{1}{2}(\partial_0 A_1 - \partial_1 A_0)F_{01} \\
&= \frac{1}{2} (\partial_0 A_1 + \partial_1 A_0 )F_{01} \\
&= \frac{1}{2} ((\partial_0 A_1)^2 - (\partial_1 A_0)^2 )
\end{aligned}
\end{equation}

For a Lagrangian density of \(E^2 - B^2\) we have an energy density of \(E^2 + B^2\), so I had have expected the Hamiltonian density here to stay equal to \(E_x^2/2\), but it
does not look like that is what I get (what I calculated is not at all familiar seeming).

If I have not made a mistake here, perhaps I am incorrect in assuming that the Hamiltonian density of the electrodynamic Lagrangian should be the energy density?

\section{Appendix.  Messy details}

For both the charge and the charge free case, we need a proof of

\begin{equation}\label{eqn:poynting:520}
\begin{aligned}
(i\BB) \cdot (\spacegrad \wedge \BE) - \BE \cdot (\spacegrad \cdot (i \BB))
&= \spacegrad \cdot (\BE \cdot (i\BB))
\end{aligned}
\end{equation}

This is relativity straightforward, albeit tedious, to do backwards.

\begin{equation}\label{eqn:poynting:540}
\begin{aligned}
\spacegrad \cdot ((i \BB) \cdot \BE)
&= \gpgradezero{ \spacegrad ((i \BB) \cdot \BE)} \\
&= \inv{2} \gpgradezero{ \spacegrad ( i \BB \BE - \BE i \BB ) } \\
&= \inv{2} \gpgradezero{
  \dot{\spacegrad} i \dot{\BB} \BE
+ \dot{\spacegrad} i \BB \dot{\BE}
- \dot{\spacegrad} \dot{\BE} i \BB
- \dot{\spacegrad} \BE i \dot{\BB}
} \\
&= \inv{2} \gpgradezero{
  \BE \spacegrad (i \BB) - (i\dot{\BB}) \dot{\spacegrad} \BE
+ \dot{\BE} \dot{\spacegrad} i \BB - i \BB \spacegrad \BE
} \\
&= \inv{2} \left(
  \BE \cdot (\spacegrad \cdot (i \BB)) - ((i\dot{\BB}) \cdot \dot{\spacegrad}) \cdot \BE
+ (\dot{\BE} \wedge \dot{\spacegrad}) \cdot (i \BB) - (i \BB) \cdot (\spacegrad \wedge \BE)
\right)
\\
\end{aligned}
\end{equation}

Grouping the two sets of repeated terms after reordering and the associated sign adjustments we have

\begin{equation}\label{eqn:poynting:poyntingDivergence}
\begin{aligned}
\spacegrad \cdot ((i \BB) \cdot \BE) &= \BE \cdot (\spacegrad \cdot (i \BB)) - (i \BB) \cdot (\spacegrad \wedge \BE)
\end{aligned}
\end{equation}

which is the desired identity (in negated form) that was to be proved.

There is likely some theorem that could be used to avoid some of this algebra.

\section{References for followup study}

Some of the content available in the
article \href{http://farside.ph.utexas.edu/teaching/em/lectures/node89.html}{Energy Conservation} looks like it will also be useful to study (in particular
it goes through some examples that convert this from a math treatment to
a physics story).

   %
% Copyright � 2012 Peeter Joot.  All Rights Reserved.
% Licenced as described in the file LICENSE under the root directory of this GIT repository.
%

%
%
\chapter{Time rate of change of the Poynting vector, and its conservation law}\label{chap:PJpoyntingRate}
\index{Poynting vector!conservation law}
%\date{Jan 18, 2009.  poyntingRate.tex}

\section{Motivation}

Derive the conservation laws for the time rate of change of the Poynting vector, which appears to be a momentum density like quantity.

The Poynting conservation relationship has been derived previously.  Additionally a starting
exploration
\chapcite{PJemstresstensor}
of the related four vector quantity has been related to a subset of the energy momentum stress tensor.
This was incomplete since the meaning of the \(T_{kj}\) terms of the tensor were unknown and the expected
Lorentz transform relationships had not been determined.  The aim here is to try to figure out this remainder.

\section{Calculation}

Repeating again from \chapcite{PJpoynting}, the electrodynamic energy density \(U\) and momentum flux density vectors are related as follows

\begin{equation}\label{eqn:poynting_rate:fromPoyntingNotes}
\begin{aligned}
U &= \frac{\epsilon_0}{2}\left( \BE^2 + c^2 \BB^2 \right) \\
\BP &= \inv{\mu_0}\BE \cross \BB = \inv{\mu_0} (i \BB) \cdot \BE \\
0 &= \PD{t}{U} + \spacegrad \cdot \BP + \BE \cdot \Bj
\end{aligned}
\end{equation}

We want to now calculate the time rate of change of this Poynting (field momentum density) vector.

\begin{equation}\label{eqn:poyntingRate:20}
\begin{aligned}
\PD{t}{\BP}
&= \PD{t}{} \left( \inv{\mu_0} \BE \cross \BB \right) \\
&= \PD{t}{} \left( \inv{\mu_0}(i\BB) \cdot \BE \right) \\
&= \partial_0 \left( \inv{\mu_0}(i c\BB) \cdot \BE \right) \\
&= \inv{\mu_0} \left( \partial_0 (i c\BB) \cdot \BE  + (i c\BB) \cdot \partial_0 \BE  \right)
\end{aligned}
\end{equation}

%Let us ignore the \(\mu_0\) factor for now, and focus just on the field dot products.
We will want to express these time derivatives in
terms of the current and spatial derivatives to determine the conservation identity.  To do this let us go back to Maxwell's equation
once more, with a premultiplication by \(\gamma_0\) to provide us with an observer dependent spacetime split

\begin{equation}\label{eqn:poyntingRate:40}
\begin{aligned}
\gamma_0 \grad F &= \gamma_0 J / \epsilon_0 c \\
(\partial_0 + \spacegrad) ( \BE + i c \BB ) &= \rho/\epsilon_0 - \Bj/\epsilon_0 c \\
\end{aligned}
\end{equation}

We want the grade one and grade two components for the time derivative terms.  For grade one we have

\begin{equation}\label{eqn:poyntingRate:60}
\begin{aligned}
- \Bj/\epsilon_0 c
&= \gpgradeone{(\partial_0 + \spacegrad) ( \BE + i c \BB )} \\
&= \partial_0 \BE + \spacegrad \cdot (ic \BB)
\end{aligned}
\end{equation}

and for grade two
\begin{equation}\label{eqn:poyntingRate:80}
\begin{aligned}
0
&= \gpgradetwo{(\partial_0 + \spacegrad) ( \BE + i c \BB )} \\
&= \partial_0 (i c \BB) + \spacegrad \wedge \BE
\end{aligned}
\end{equation}

Using these we can express the time derivatives for back substitution
\begin{equation}\label{eqn:poyntingRate:100}
\begin{aligned}
\partial_0 \BE &= - \Bj/\epsilon_0 c - \spacegrad \cdot (ic \BB) \\
\partial_0 (i c \BB) &= -\spacegrad \wedge \BE
\end{aligned}
\end{equation}

yielding
\begin{equation}\label{eqn:poyntingRate:120}
\begin{aligned}
\mu_0 \PD{t}{\BP}
&= \partial_0 (i c\BB) \cdot \BE  + (i c\BB) \cdot \partial_0 \BE \\
&= -(\spacegrad \wedge \BE) \cdot \BE  - (i c\BB) \cdot \left( \Bj/\epsilon_0 c + \spacegrad \cdot (ic \BB) \right) \\
\end{aligned}
\end{equation}

Or
\begin{equation}\label{eqn:poyntingRate:140}
\begin{aligned}
%\PD{t}{((i\BB) \cdot \BE)} +(\spacegrad \wedge \BE) \cdot \BE + (i c\BB) \cdot (\spacegrad \cdot (ic \BB)) &= - (i c\BB) \cdot \Bj/\epsilon_0 c \\
0
&= \partial_0 {((ic \BB) \cdot \BE)} +(\spacegrad \wedge \BE) \cdot \BE + (i c\BB) \cdot (\spacegrad \cdot (ic \BB)) + (i c\BB) \cdot \Bj/\epsilon_0 c \\
&= \gpgradeone{ \partial_0 (ic \BB \BE) +(\spacegrad \wedge \BE) \BE + i c\BB (\spacegrad \cdot (ic \BB)) + i c\BB \Bj/\epsilon_0 c } \\
&= \gpgradeone{ \partial_0 (ic \BB \BE) +(\spacegrad \wedge \BE) \BE + (\spacegrad \wedge (c \BB)) c \BB + i c\BB \Bj/\epsilon_0 c } \\
0 &= i\partial_0 (c \BB \wedge \BE) + (\spacegrad \wedge \BE) \cdot \BE + (\spacegrad \wedge (c \BB)) \cdot (c \BB) + i( c\BB \wedge \Bj)/\epsilon_0 c
\end{aligned}
\end{equation}

This appears to be the conservation law that is expected for the change in vector field momentum density.

\begin{equation}\label{eqn:poynting_rate:momCons}
\begin{aligned}
\partial_t (\BE \cross \BB) + (\spacegrad \wedge \BE) \cdot \BE + c^2 (\spacegrad \wedge \BB) \cdot \BB = (\BB \cross \Bj)/\epsilon_0
\end{aligned}
\end{equation}

In terms of the original Poynting vector this is
\begin{equation}\label{eqn:poynting_rate:withPoynting}
\begin{aligned}
\PD{t}{\BP} + \inv{\mu_0}(\spacegrad \wedge \BE) \cdot \BE + c^2 \inv{\mu_0}(\spacegrad \wedge \BB) \cdot \BB = c^2 (\BB \cross \Bj)
\end{aligned}
\end{equation}

%For clarity let us temporarily switch to natural units \(c = \epsilon_0 = \mu_0\)
%
%\begin{align}
%0 = \partial_0 (\BB \cross \BE) + (\spacegrad \wedge \BE) \cdot \BE + (\spacegrad \wedge \BB) \cdot \BB + \BB \wedge \Bj
%\end{align}
%
%The SI units can be restored here easily enough since we just have to put a \(c\) with each \(\BB\), and divide our current density by \(\epsilon_0 c\).

Now, there are a few things to pursue here.

\begin{itemize}
\item How to or can we put this in four vector divergence form.
\item Relate this to the wikipedia result which is very different looking.
\item Find the relation to the stress energy tensor.
\item Lorentz transformation relation to Poynting energy momentum conservation law.
\end{itemize}

\subsection{Four vector form?}

If \(\BP = P^m \sigma_m\), then each of the \(P^m\) coordinates could be thought of as the zero coordinate of a four vector.  Can we get a four vector
divergence out of \eqnref{eqn:poynting_rate:momCons}?

Let us expand the wedge-dot term in coordinates.

\begin{equation}\label{eqn:poyntingRate:160}
\begin{aligned}
( (\spacegrad \wedge \BE ) \cdot \BE ) \cdot \sigma_m
&= ((\sigma^a \wedge \sigma_b) \cdot \sigma_k ) \cdot \sigma_m (\partial_a E^b) E^k \\
&= (\delta^a_m \delta_{bk} - \delta_{bm} \delta^a_k) (\partial_a E^b) E^k \\
&= \sum_k (\partial_m E^k - \partial_k E^m) E^k \\
&= \partial_m \frac{\BE^2}{2} - (\BE \cdot \spacegrad) E^m \\
\end{aligned}
\end{equation}

So we have three equations, one for each \(m = \{1,2,3\}\)

\begin{equation}\label{eqn:poynting_rate:coordinates}
\begin{aligned}
%0 &= \partial_0 (c \BB \wedge \BE) + (\spacegrad \wedge \BE) \cdot \BE + (\spacegrad \wedge (c \BB)) \cdot (c \BB) + c\BB \wedge \Bj/\epsilon_0 c
\PD{t}{P^m} + c^2 \PD{x^m}{U} - \inv{\mu_0}( (\BE \cdot \spacegrad) E^m + c^2 (\BB \cdot \spacegrad) B^m ) &= c^2 (\BB \cross \Bj)_m
\end{aligned}
\end{equation}

Damn.  This does not look anything like the four vector divergence that we had with the Poynting conservation equation.  In the second last line
of the wedge dot expansion we do see that we only have to sum over the \(k \ne m\) terms.  Can that help simplify this?

\subsection{Compare to wikipedia form}

To compare \eqnref{eqn:poynting_rate:withPoynting} with the
\href{https://en.wikipedia.org/wiki/Electromagnetic_stress-energy_tensor#Conservation_laws}{wikipedia article}
, the first thing we have to do is eliminate the wedge
products.

This can be done in a couple different ways.  One, is conversion to cross products

\begin{equation}\label{eqn:poyntingRate:180}
\begin{aligned}
(\spacegrad \wedge \Ba) \cdot \Ba
&= \gpgradeone{(\spacegrad \wedge \Ba) \Ba } \\
&= \gpgradeone{i (\spacegrad \cross \Ba) \Ba } \\
&= \gpgradeone{i ((\spacegrad \cross \Ba) \cdot \Ba) + i ((\spacegrad \cross \Ba) \wedge \Ba) } \\
&= \gpgradeone{i ((\spacegrad \cross \Ba) \wedge \Ba) } \\
&= i^2 ((\spacegrad \cross \Ba) \cross \Ba) \\
\end{aligned}
\end{equation}

So we have
\begin{equation}\label{eqn:poyntingRate:200}
\begin{aligned}
(\spacegrad \wedge \Ba) \cdot \Ba &= \Ba \cross (\spacegrad \cross \Ba)
\end{aligned}
\end{equation}

so we can rewrite the Poynting time change \eqnref{eqn:poynting_rate:withPoynting} as
\begin{equation}\label{eqn:poyntingRate:220}
\begin{aligned}
\PD{t}{\BP} + \inv{\mu_0}
\left( \BE \cross (\spacegrad \cross \BE) + c^2 \BB \cross (\spacegrad \cross \BB) \right)
 = c^2 (\BB \cross \Bj)
\end{aligned}
\end{equation}

However, the wikipedia article has \(\rho \BE\) terms, which suggests that a \(\spacegrad \cdot \BE\) based expansion has been used.  Take II.

Let us try expanding this wedge dot differently, and to track what is being operated on write \(\Bx\) as a variable vector, and
\(\Ba\) as a constant vector.  Now expand

\begin{equation}\label{eqn:poyntingRate:240}
\begin{aligned}
(\spacegrad \wedge \Bx) \cdot \Ba
&= - \Ba \cdot (\spacegrad \wedge \Bx) \\
&= \spacegrad (\Ba \cdot \Bx) - (\Ba \cdot \spacegrad) \wedge \Bx \\
%&= - \gpgradeone{\Ba (\spacegrad \wedge \Bx)} \\
%&= - \gpgradeone{\Ba (\spacegrad \Bx - \spacegrad \cdot \Bx)} \\
%&= \Ba (\spacegrad \cdot \Bx) - \gpgradeone{\Ba \spacegrad \Bx} \\
%&= \Ba (\spacegrad \cdot \Bx) - (\Ba \cdot \spacegrad) \Bx - \gpgradeone{ (\Ba \wedge \spacegrad) \Bx } \\
%&= \Ba (\spacegrad \cdot \Bx) - (\Ba \cdot \spacegrad) \Bx - (\Ba \wedge \spacegrad) \cdot \Bx \\
%&= \Ba (\spacegrad \cdot \Bx) - (\Ba \cdot \spacegrad) \Bx - \Ba (\spacegrad \cdot \Bx) + \spacegrad (\Ba \cdot \Bx) \\
\end{aligned}
\end{equation}

%Now it looks like we are going in circles a bit here, and there is probably a more direct way.  We do however have as a final
%result
%
%\begin{align*}
%(\spacegrad \wedge \Bx) \cdot \Ba &= \spacegrad (\Ba \cdot \Bx) - (\Ba \cdot \spacegrad) \Bx \\
%\end{align*}

What we really want is an expansion of \((\spacegrad \wedge \Bx) \cdot \Bx\).  To get there consider

\begin{equation}\label{eqn:poyntingRate:260}
\begin{aligned}
\spacegrad \Bx^2
&= \dot{\spacegrad} \dot{\Bx} \cdot \Bx + \dot{\spacegrad} {\Bx} \cdot \dot{\Bx} \\
&= 2 \dot{\spacegrad} {\Bx} \cdot \dot{\Bx} \\
\end{aligned}
\end{equation}

This has the same form as the first term above.  We take the gradient and apply it to a dot product where one of the vectors is kept constant, so we can write

\begin{equation}\label{eqn:poyntingRate:280}
\begin{aligned}
{\spacegrad} {\Bx} \cdot \dot{\Bx} &= \inv{2} \spacegrad \Bx^2
\end{aligned}
\end{equation}

and finally
\begin{equation}\label{eqn:poynting_rate:wedgedot}
\begin{aligned}
(\spacegrad \wedge \Bx) \cdot \Bx
%&= \dot{\spacegrad} (\Bx \cdot \dot{\Bx}) - (\Bx \cdot \spacegrad) \Bx \\
&= \inv{2} \spacegrad \Bx^2 - (\Bx \cdot \spacegrad) \Bx
\end{aligned}
\end{equation}

We can now reassemble the equations and write

\begin{equation}\label{eqn:poyntingRate:300}
\begin{aligned}
(\spacegrad \wedge \BE) \cdot \BE + c^2 (\spacegrad \wedge \BB) \cdot \BB
&=
\inv{2} \spacegrad \BE^2 - (\BE \cdot \spacegrad) \BE
+ c^2 \left(\inv{2} \spacegrad \BB^2 - (\BB \cdot \spacegrad) \BB \right) \\
&= \inv{\epsilon_0} \spacegrad U - (\BE \cdot \spacegrad) \BE - c^2 (\BB \cdot \spacegrad) \BB \\
\end{aligned}
\end{equation}

Now, we have the time derivative of momentum and the spatial derivative of the energy grouped together in a nice
relativistic seeming pairing.  For comparison let us also put the energy density rate change equation with this
to observe them together

\begin{equation}\label{eqn:poynting_rate:withPoyntingAndEnergy}
\begin{aligned}
\PD{t}{U} + \spacegrad \cdot \BP &= -\Bj \cdot \BE \\
\PD{t}{\BP} + c^2 \spacegrad U &= -c^2 (\Bj \cross \BB) + \inv{\mu_0} \left( (\BE \cdot \spacegrad) \BE + c^2 (\BB \cdot \spacegrad) \BB \right)
\end{aligned}
\end{equation}

The second equation here is exactly what we worked out above by coordinate expansion when looking for a four vector formulation
of this equation.  This however, appears much closer to the desired result, which was not actually clear looking at the
coordinate expansion.

These equations are not tidy enough seeming, so one can intuit that there is some more natural way to express those
misfit seeming \((\Bx \cdot \spacegrad) \Bx\) terms.
It would be logically tidier if we could express those both in terms of charge and current densities.

Now, it is too bad that it is not true that
\begin{equation}\label{eqn:poyntingRate:320}
\begin{aligned}
(\BE \cdot \spacegrad) \BE &= \BE (\spacegrad \cdot \BE)
\end{aligned}
\end{equation}

If that were the case then we would have on the right hand side
\begin{equation}\label{eqn:poyntingRate:340}
\begin{aligned}
-c^2(\Bj \cross \BB)
+
\inv{\mu} \left(
\BE (\spacegrad \cdot \BE) + c^2 \BB (\spacegrad \cdot \BB)
\right)
&=
-c^2(\Bj \cross \BB) +
\inv{\mu_0} (\BE \rho + c^2 \BB (0)) \\
&= -c^2(\Bj \cross \BB) + \inv{\mu_0} \rho \BE \\
\end{aligned}
\end{equation}

This has a striking similarity to the Lorentz force law, and is also fairly close to the wikipedia equation, with the exception that the
\(\Bj \cross \BB\) and \(\rho \BE\) terms have opposing signs.

Lets instead adding and subtracting this term so that the conservation equation remains correct

\begin{equation}\label{eqn:poyntingRate:360}
\begin{aligned}
\inv{c^2} \PD{t}{\BP} + \spacegrad U &-\epsilon_0 \left(
\BE (\spacegrad \cdot \BE) +(\BE \cdot \spacegrad) \BE
+ c^2 \BB (\spacegrad \cdot \BB) + c^2 (\BB \cdot \spacegrad) \BB
\right) \\
&= -(\Bj \cross \BB) - \epsilon_0 \rho \BE
\end{aligned}
\end{equation}
% c^2 \mu_0 = 1/\e_0

Now we are left with quantities of the following form.

\begin{equation}\label{eqn:poyntingRate:380}
\begin{aligned}
\Bx (\spacegrad \cdot \Bx) +(\Bx \cdot \spacegrad) \Bx
\end{aligned}
\end{equation}

The sum of these for the electric and magnetic fields appears to be what the
wiki article calls \(\spacegrad \cdot \sigma\), although it appears
there that \(\sigma\) is a scalar so this does not quite make sense.

It appears that
we should therefore be looking to
express these in terms of a gradient of the squared fields?  We have such \(\BE^2\) and \(\BB^2\) terms in the energy so it would make some logical sense if this
could be done.

The essence of the desired reduction is to see if we can find a scalar function \(\sigma(\Bx)\) such that

\begin{equation}\label{eqn:poyntingRate:400}
\begin{aligned}
\spacegrad \sigma(\Bx) &= \inv{2} \spacegrad \Bx^2 - \left(\Bx (\spacegrad \cdot \Bx) + (\Bx \cdot \spacegrad) \Bx )\right)
\end{aligned}
\end{equation}

\subsection{stress tensor}
\index{stress tensor}

From \citep{doran2003gap} we expect that there is a relationship between
the equations \eqnref{eqn:poynting_rate:coordinates}, and \(F \gamma_k F\).  Let us see
if we can find exactly how these relate.

%First.  In the spatial basis, as a three dimensional cross product, we can express the Poynting vector as an antisymmetric tensor.

%\BP = \BP ...

TODO: ...

\section{Take II}

After going in circles and having a better idea now where I am going, time to restart and make sure that errors are not compounding.

The starting point will be

\begin{equation}\label{eqn:poyntingRate:420}
\begin{aligned}
\PD{t}{\BP} &= \inv{\mu_0} \left( \partial_0 (i c\BB) \cdot \BE  + (i c\BB) \cdot \partial_0 \BE  \right) \\
\partial_0 \BE &= - \Bj/\epsilon_0 c - \spacegrad \cdot (ic \BB) \\
\partial_0 (i c \BB) &= -\spacegrad \wedge \BE
\end{aligned}
\end{equation}

Assembling we have

\begin{equation}\label{eqn:poyntingRate:440}
\begin{aligned}
\PD{t}{\BP} + \inv{\mu_0} \left( (\spacegrad \wedge \BE) \cdot \BE + (i c\BB) \cdot ( \Bj/\epsilon_0 c + \spacegrad \cdot (ic \BB) ) \right) &= 0 \\
\end{aligned}
\end{equation}

This is

\begin{equation}\label{eqn:poyntingRate:460}
\begin{aligned}
\PD{t}{\BP} + \inv{\mu_0} \left( (\spacegrad \wedge \BE) \cdot \BE + (i c\BB) \cdot ( \spacegrad \cdot (ic \BB) ) \right) &= -c^2 (i \BB) \cdot \Bj.
\end{aligned}
\end{equation}

Now get rid of the pseudoscalars
\begin{equation}\label{eqn:poyntingRate:480}
\begin{aligned}
(i \BB) \cdot \Bj
&= \gpgradeone{ i \BB \Bj } \\
&= i (\BB \wedge \Bj) \\
&= i^2 (\BB \cross \Bj) \\
&= -(\BB \cross \Bj) \\
\end{aligned}
\end{equation}

and
\begin{equation}\label{eqn:poyntingRate:500}
\begin{aligned}
(i c\BB) \cdot ( \spacegrad \cdot (ic \BB) )
&= c^2 \gpgradeone{ i \BB ( \spacegrad \cdot (i\BB) ) } \\
&= c^2 \gpgradeone{ i \BB \gpgradeone{\spacegrad i\BB} } \\
&= c^2 \gpgradeone{ i \BB i (\spacegrad \wedge \BB) } \\
&= -c^2 \gpgradeone{ \BB (\spacegrad \wedge \BB) } \\
&= -c^2 \BB \cdot (\spacegrad \wedge \BB) \\
\end{aligned}
\end{equation}

So we have
\begin{equation}\label{eqn:poyntingRate:520}
\begin{aligned}
\PD{t}{\BP} - \inv{\mu_0} \left( \BE \cdot (\spacegrad \wedge \BE) +c^2 \BB \cdot (\spacegrad \wedge \BB) \right) &= c^2 (\BB \cross \Bj) \\
\end{aligned}
\end{equation}

Now we subtract \((\BE (\spacegrad \cdot \BE) + c^2 \BB (\spacegrad \cdot \BB))/\mu_0 = \BE \rho/\epsilon_0\mu_0\) from both sides yielding

\begin{equation}\label{eqn:poyntingRate:540}
\begin{aligned}
\PD{t}{\BP} - \inv{\mu_0} \left(
\BE \cdot (\spacegrad \wedge \BE) + \BE (\spacegrad \cdot \BE)
+ c^2 \BB \cdot (\spacegrad \wedge \BB)
+ c^2 \BB (\spacegrad \cdot \BB)
\right) &= -c^2 (\Bj \cross \BB + \rho \BE) \\
\end{aligned}
\end{equation}

Regrouping slightly

\begin{equation}\label{eqn:poyntingRate:560}
\begin{aligned}
0 = \inv{c^2}\PD{t}{\BP} &+ (\Bj \cross \BB + \rho \BE)
\\
&-{\epsilon_0} \left(
\BE \cdot (\spacegrad \wedge \BE) + \BE (\spacegrad \cdot \BE)
+ c^2 \BB \cdot (\spacegrad \wedge \BB)
+ c^2 \BB (\spacegrad \cdot \BB)
\right)
\end{aligned}
\end{equation}

Now, let us write the \(\BE\) gradient terms here explicitly in coordinates.

\begin{equation}\label{eqn:poyntingRate:580}
\begin{aligned}
-\BE \cdot (\spacegrad \wedge \BE) - \BE (\spacegrad \cdot \BE)
&= -\sigma_k \cdot (\sigma^m \wedge \sigma_n) E^k \partial_m E^n
- E^k \sigma_k \partial_m E^m \\
&=
-\delta_{k}^m \sigma_n E^k \partial_m E^n
+\delta_{kn} \sigma^m E^k \partial_m E^n
- E^k \sigma_k \partial_m E^m \\
&=
- \sigma_n E^k \partial_k E^n
+ \sigma^m E^k \partial_m E^k
- E^k \sigma_k \partial_m E^m \\
&= \sum_{k,m} \sigma_k \left( -E^m \partial_m E^k +E^m \partial_k E^m -E^k \partial_m E^m \right) \\
\end{aligned}
\end{equation}

We could do the \(\BB\) terms too, but they will have the same form.  Now \citep{schwartz1987pe} contains a relativistic treatment of
the stress tensor that would take some notation study to digest, but the end result appears to have the divergence
result that is desired.  It is a second rank tensor which probably explains the \(\grad \cdot \sigma\) notation in wikipedia.

For the \(x\) coordinate of the \(\PDi{t}{\BP}\) vector the book says we have a vector of the form

\begin{equation}\label{eqn:poyntingRate:600}
\begin{aligned}
\BT_x = \inv{2}(- E_x^2 + E_y^2 + E_z^2)\sigma_1 - E_x E_y \sigma_2 - E_x E_z \sigma_3
\end{aligned}
\end{equation}

and it looks like the divergence of this should give us our desired mess.  Let us try this, writing \(k,m,n\) as distinct indices.

\begin{equation}\label{eqn:poyntingRate:620}
\begin{aligned}
\BT_k &= \inv{2}(- (E^k)^2 + (E^m)^2 + (E^n)^2)\sigma_k - E^k E^m \sigma_m - E^k E^n \sigma_n
\end{aligned}
\end{equation}

\begin{equation}\label{eqn:poyntingRate:640}
\begin{aligned}
\spacegrad \cdot \BT_k
&= \inv{2} \partial_k (-(E^k)^2 + (E^m)^2 + (E^n)^2)
- \partial_m (E^k E^m)
- \partial_n(E^k E^n) \\
&=
-E^k \partial_k E^k
+ E^m \partial_k E^m
+ E^n \partial_k E^n
- E^k \partial_m E^m
- E^m \partial_m E^k
- E^k \partial_n E^n
- E^n \partial_n E^k \\
&=
- E^k \partial_k E^k
- E^k \partial_m E^m
- E^k \partial_n E^n \\
&- E^m \partial_m E^k
+ E^m \partial_k E^m \\
&- E^n \partial_n E^k
+ E^n \partial_k E^n
\\
\end{aligned}
\end{equation}

Does this match?  Let us expand our \(k\) term above to see if it looks the same.  That is

\begin{equation}\label{eqn:poyntingRate:660}
\begin{aligned}
\sum_m (-E^m \partial_m E^k +E^m \partial_k E^m -E^k \partial_m E^m)
&=
-E^k \partial_k E^k
+E^k \partial_k E^k
-E^k \partial_k E^k \\
&-E^m \partial_m E^k
+E^m \partial_k E^m
-E^k \partial_m E^m \\
&-E^n \partial_n E^k
+E^n \partial_k E^n
-E^k \partial_n E^n \\
&=
-E^k \partial_k E^k
-E^k \partial_m E^m
-E^k \partial_n E^n \\
&-E^m \partial_m E^k
+E^m \partial_k E^m \\
&-E^n \partial_n E^k
+E^n \partial_k E^n \\
\end{aligned}
\end{equation}

Yeah!  Finally have a form of the momentum conservation equation that is strictly in terms of gradients and time partials.  Summarizing the results, this is

\begin{equation}\label{eqn:poyntingRate:680}
\begin{aligned}
\inv{c^2}\PD{t}{\BP} + \Bj \cross \BB + \rho \BE + \sum_k \sigma_k \spacegrad \cdot \BT_k = 0
\end{aligned}
\end{equation}

Where
\begin{equation}\label{eqn:poyntingRate:700}
\begin{aligned}
\sum_k \sigma_k \spacegrad \cdot \BT_k
&=
-\epsilon_0 \left(
\BE \cdot (\spacegrad \wedge \BE) + \BE (\spacegrad \cdot \BE)
+ c^2 \BB \cdot (\spacegrad \wedge \BB)
+ c^2 \BB (\spacegrad \cdot \BB) \right)
\end{aligned}
\end{equation}

For \(\BT_k\) itself, with \(k \ne m \ne n\) we have

\begin{equation}\label{eqn:poyntingRate:720}
\begin{aligned}
\BT_k &=
\epsilon_0 \left( \inv{2}(-(E^k)^2 + (E^m)^2 + (E^n)^2)\sigma_k - E^k E^m \sigma_m - E^k E^n \sigma_n \right) \\
&+ \inv{\mu_0}\left(\inv{2}(-(B^k)^2 + (B^m)^2 + (B^n)^2)\sigma_k - B^k B^m \sigma_m - B^k B^n \sigma_n \right)
\end{aligned}
\end{equation}

   %
% Copyright � 2012 Peeter Joot.  All Rights Reserved.
% Licenced as described in the file LICENSE under the root directory of this GIT repository.
%

%
%
\chapter{Field and wave energy and momentum}\label{chap:PJelectricFieldEnergy}
%\date{Jan 03, 2009.  electricFieldEnergy.tex}

\section{Motivation}

The concept of energy in the electric and magnetic fields I am getting closer to understanding, but there is a few ways that I would like to approach it.

I have now explored the Poynting vector energy conservation relationships in
\chapcite{PJpoynting}, and
\chapcite{PJemstresstensor}
, but had not understood fully where the energy expressions in the electro and magneto statics cases came from
separately.  I also do not yet know where the \(F \gamma_k F\) terms of the stress tensor fit in the big picture?  I suspect that they can be obtained by Lorentz transforming the rest frame expression \(F \gamma_0 F\) (the energy density, Poynting momentum density four vector).

It also ought to be possible to relate the field energies to a Lagrangian and Hamiltonian, but I have not had success doing so.

The last thing that I had like to understand is how the energy and momentum of a wave can be expressed, both in terms of the abstract conjugate field momentum concept and with a concrete example such as the one dimensional oscillating rod that can be treated in a limiting coupled oscillator approach as in
\citep{goldstein1951cm}.

Once I have got all these down I think I will be ready to revisit Bohm's Rayleigh-Jeans law treatment in \citep{bohm1989qt}.  Unfortunately, each time I try perusing some
interesting aspect of QM I find that I end up back studying electrodynamics, and suspect that needs to be my focus for the foreseeable future (perhaps working thoroughly through Feynman's
volume II).

\section{Electrostatic energy in a field}

Feynman's treatment in
\citep{feynman1963flp}
of the energy \(\frac{\epsilon_0}{2}\BE^2\) associated with the electrostatic \(\BE\) field is very easy to understand.  Here is a write up of this myself without looking at the book to see if I really understood the ideas.

The first step is consideration of the force times distance for two charges gives you the energy required (or gained) by moving one of those charges from infinity to some given separation

\begin{equation}\label{eqn:electricFieldEnergy:20}
\begin{aligned}
W &= \frac{1}{4\pi\epsilon_0} \int_{\infty}^{r} \frac{q_1 q_2}{x^2} \Be_1 \cdot (-\Be_1 dx) \\
&= \frac{q_1 q_2}{4 \pi \epsilon_0 r}
\end{aligned}
\end{equation}

This provides a quantization for an energy in a field concept.  A distribution of charge requires energy to assemble and it is possible to enumerate that energy separately by considering all the charges, or alternatively, by not looking at the final charge distribution, but only considering the net field associated with this charge distribution.  This is a pretty powerful, but somewhat abstract seeming idea.

The generalization to continuous charge distribution from there was pretty straightforward, requiring a double integration over all space

\begin{equation}\label{eqn:electricFieldEnergy:40}
\begin{aligned}
W &= \frac{1}{2} \int \frac{1}{4\pi\epsilon_0} \frac{\rho_1 dV_1 \rho_2 dV_2}{r_{12}}  \\
&= \frac{1}{2} \int \rho_1 \phi_2 dV_1
\end{aligned}
\end{equation}

The \(1/2\) factor was due to double counting all "pairs" of charge elements.  The next step was to rewrite the charge density by using Maxwell's equations.  In terms of the four vector potential Maxwell's equation (with the \(\grad \cdot A = 0\) gauge) is

\begin{equation}\label{eqn:electricFieldEnergy:60}
\begin{aligned}
\grad^2 A = \inv{\epsilon_0 c}( c \rho \gamma_0 + J^k \gamma_k)
\end{aligned}
\end{equation}

So, to write \(\rho\) in terms of potential \(A^0 = \phi\), we have

\begin{equation}\label{eqn:electricFieldEnergy:80}
\begin{aligned}
\left(\inv{c^2}\PDsq{t}{} - \spacegrad^2\right) \phi = \inv{\epsilon_0} \rho
\end{aligned}
\end{equation}

In the statics case, where \(\PD{t}{\phi} = 0\), we can thus write the charge density in terms of the potential

\begin{equation}\label{eqn:electricFieldEnergy:100}
\begin{aligned}
\rho = -\epsilon_0 \spacegrad^2 \phi
\end{aligned}
\end{equation}

and substitute back into the energy summation

\begin{equation}\label{eqn:electricFieldEnergy:120}
\begin{aligned}
W
&= \frac{1}{2} \int \rho \phi dV \\
&= \frac{-\epsilon_0}{2} \int \phi \spacegrad^2 \phi dV \\
\end{aligned}
\end{equation}

Now, Feynman's last step was a bit sneaky, which was to convert the \(\phi \spacegrad^2 \phi\) term into a divergence integral.  Working backwards to derive the identity that he used

\begin{equation}\label{eqn:electricFieldEnergy:140}
\begin{aligned}
\spacegrad \cdot (\phi \spacegrad \phi)
&= \gpgradezero{ \spacegrad (\phi \spacegrad \phi) } \\
&= \gpgradezero{ (\spacegrad \phi) \spacegrad \phi + \phi \spacegrad (\spacegrad \phi) } \\
&= (\spacegrad \phi)^2 + \phi \spacegrad^2 \phi \\
\end{aligned}
\end{equation}

This can then be used with Stokes theorem in its dual form to convert our \(\phi \spacegrad^2 \phi\) the into plain volume and surface integral
\begin{equation}\label{eqn:electricFieldEnergy:160}
\begin{aligned}
W
&= \frac{\epsilon_0}{2} \int \left( (\spacegrad \phi)^2 -\spacegrad \cdot (\phi \spacegrad \phi) \right) dV \\
&= \frac{\epsilon_0}{2} \int (\spacegrad \phi)^2 dV - \frac{\epsilon_0}{2} \int_{\partial V} (\phi \spacegrad \phi) \cdot \ncap dA \\
\end{aligned}
\end{equation}

Letting the surface go to infinity and employing a limiting argument on the magnitudes of the \(\phi\) and \(\spacegrad \phi\) terms was enough to produce the final electrostatics
energy result

\begin{equation}\label{eqn:electricFieldEnergy:180}
\begin{aligned}
W
&= \frac{\epsilon_0}{2} \int (\spacegrad \phi)^2 dV \\
&= \frac{\epsilon_0}{2} \int \BE^2 dV
\end{aligned}
\end{equation}

\section{Magnetostatic field energy}

Feynman's energy discussion of the magnetic field for a constant current loop (magnetostatics), is not so easy to follow.  He considers the dipole moment of a small loop, obtained by comparison to previous electrostatic results (that I had have to go back and read or re-derive) and some
subtle seeming arguments about the mechanical vs. total energy of the system.

\subsection{Biot Savart}

As an attempt to understand all this, let us break it up into pieces.  First, is calculation of the field for a current loop.  Let us also use
this as an opportunity to see how one would work directly and express the Biot-Savart law in the STA formulation.

Going back to Maxwell's equation (with the \(\grad \cdot A\) gauge again), we have

\begin{equation}\label{eqn:electricFieldEnergy:200}
\begin{aligned}
\grad F
&= \grad (\grad \wedge A) \\
&= \grad^2 A^\mu \\
&= J^\mu/\epsilon_0 c
\end{aligned}
\end{equation}

For a static current configuration with \(J^0 = c \rho = 0\), we have \(\PDi{t}{A^\mu} = 0\), and our vector potential equations are

\begin{equation}\label{eqn:electricFieldEnergy:220}
\begin{aligned}
\spacegrad^2 A^k = -J^k/\epsilon_0 c
\end{aligned}
\end{equation}

Recall that the solution of \(A^k\) can be expressed as the impulse response of a function of the following form

\begin{equation}\label{eqn:electricFieldEnergy:240}
\begin{aligned}
A^k = C\inv{r}
\end{aligned}
\end{equation}

and that \(\spacegrad \cdot (\spacegrad (1/r))\) is zero for all \(r \ne 0\).  Performing a volume integral of the expected Laplacian we
can integrate over an infinitesimal spherical volume of radius \(R\)

\begin{equation}\label{eqn:electricFieldEnergy:260}
\begin{aligned}
\int \spacegrad^2 A^k dV
&= C \int \spacegrad \cdot \spacegrad \inv{r} dV \\
&= C \int \spacegrad \cdot \left( -\rcap \inv{r^2} \right) dV \\
&= -C \int_{\partial_V} \rcap \inv{r^2} \cdot \rcap dA \\
&= -C \inv{R^2} 4 \pi R^2 \\
&= - 4 \pi C \\
\end{aligned}
\end{equation}

Equating we can solve for \(C\)

\begin{equation}\label{eqn:electricFieldEnergy:280}
\begin{aligned}
- 4 \pi C &= -J^k/\epsilon_0 c \\
C &= \inv{4 \pi \epsilon_0 c} J^k
\end{aligned}
\end{equation}

Note that this is cheating slightly since C was kind of treated as a constant, whereas this equality makes it a function.  It works because
the infinitesimal volume can be made small enough that \(J^k\) can be treated as a constant.  This therefore provides our
potential function in terms of this impulse response

\begin{equation}\label{eqn:electricFieldEnergy:300}
\begin{aligned}
A^k &= \inv{4 \pi \epsilon_0 c} \int \frac{J^k}{r} dV
\end{aligned}
\end{equation}

Now, this could have all been done with a comparison to the electrostatic result.  Regardless, it now leaves us in the position to
calculate the field bivector

\begin{equation}\label{eqn:electricFieldEnergy:320}
\begin{aligned}
F
&= \grad \wedge A \\
&= (\gamma^\mu \wedge \gamma_k) \partial_\mu A^k \\
&= -(\gamma_m \wedge \gamma_k) \partial_m A^k \\
\end{aligned}
\end{equation}

So our field in terms of components is
\begin{equation}\label{eqn:electric_field_energy:magneticComponents}
\begin{aligned}
F &= (\sigma_m \wedge \sigma_k) \partial_m A^k
\end{aligned}
\end{equation}

Which in terms of spatial vector potential \(\BA = A^k \sigma_k\) is also
\begin{equation}\label{eqn:electricFieldEnergy:340}
\begin{aligned}
F &= \spacegrad \wedge \BA
\end{aligned}
\end{equation}

From \eqnref{eqn:electric_field_energy:magneticComponents} we can calculate the field in terms of our potential directly

\begin{equation}\label{eqn:electricFieldEnergy:360}
\begin{aligned}
\partial_m A^k
&= \inv{4 \pi \epsilon_0 c} \int dV \partial_m \frac{J^k}{r} \\
&= \inv{4 \pi \epsilon_0 c} \int dV \left( J^k \partial_m \inv{r} + \inv{r} \partial_m {J^k} \right) \\
&= \inv{4 \pi \epsilon_0 c} \int dV \left( J^k \partial_m \left(\sum_j ((x^j)^2)^{-1/2}\right) + \inv{r} \partial_m {J^k} \right) \\
&= \inv{4 \pi \epsilon_0 c} \int dV \left( J^k \left(-\inv{2}\right) 2 x^m \inv{r^3} + \inv{r} \partial_m {J^k} \right) \\
&= \inv{4 \pi \epsilon_0 c} \int \inv{r^3} dV \left( -x^m J^k + r^2 \partial_m {J^k} \right) \\
\end{aligned}
\end{equation}

So with \(\Bj = J^k \sigma_k\) we have

\begin{equation}\label{eqn:electricFieldEnergy:380}
\begin{aligned}
F
&= \inv{4 \pi \epsilon_0 c} \int \inv{r^3} dV \left( -\Br \wedge \Bj + r^2 (\spacegrad \wedge \Bj) \right) \\
&= \inv{4 \pi \epsilon_0 c} \int dV \left( \frac{\Bj \wedge \rcap}{r^2} + \inv{r}(\spacegrad \wedge \Bj) \right) \\
\end{aligned}
\end{equation}

The first term here is essentially the Biot Savart law once the current density is converted to current \(\int \Bj dV = I \int \jCap dl\), so we expect the second term to be zero.

To calculate the current density divergence we first need the current density in vector form

\begin{equation}\label{eqn:electricFieldEnergy:400}
\begin{aligned}
\Bj
&= -\epsilon_0 c \spacegrad^2 \BA  \\
&= -\epsilon_0 c \gpgradeone{\spacegrad (\spacegrad \BA) } \\
&= -\epsilon_0 c \spacegrad (\spacegrad \cdot \BA) + \spacegrad \cdot (\spacegrad \wedge \BA) \\
\end{aligned}
\end{equation}

Now, recall the gauge choice was

\begin{equation}\label{eqn:electricFieldEnergy:420}
\begin{aligned}
0 &= \grad \cdot A \\
&= \partial_0 A^0 + \partial_k A^k \\
&= \inv{c}\PD{t}{A^0} + \spacegrad \cdot \BA
\end{aligned}
\end{equation}

So, provided we also have \(\PDi{t}{A^0} = 0\), we also have \(\spacegrad \cdot \BA = 0\), which is true due to the assumed static conditions, we are left with

\begin{equation}\label{eqn:electricFieldEnergy:440}
\begin{aligned}
\Bj
&= -\epsilon_0 c \spacegrad \cdot (\spacegrad \wedge \BA) \\
\end{aligned}
\end{equation}

Now we can take the curl of \(\Bj\), also writing this magnetic field \(F\) in its dual form \(F = i c \BB\), we see that the curl of our static current density vector is zero:

\begin{equation}\label{eqn:electricFieldEnergy:460}
\begin{aligned}
\spacegrad \wedge \Bj
&= \spacegrad \wedge (\spacegrad \cdot F) \\
&= c \spacegrad \wedge (\spacegrad \cdot (i \BB)) \\
&= \frac{c}{2} \spacegrad \wedge (\spacegrad (i \BB) - i \dot{\BB} \dot{\spacegrad} ) \\
&= c \spacegrad \wedge (i \spacegrad \wedge \BB) \\
&= c \spacegrad \wedge (i^2 \spacegrad \cross \BB) \\
&= -ci \spacegrad \cross (\spacegrad \cross \BB) \\
&= 0
\end{aligned}
\end{equation}

This leaves us with
\begin{equation}\label{eqn:electricFieldEnergy:480}
\begin{aligned}
F
&= \inv{4 \pi \epsilon_0 c} \int \frac{\Bj \wedge \rcap}{r^2} dV \\
\end{aligned}
\end{equation}

Which with the current density expressed in terms of current is the desired Biot-Savart law

\begin{equation}\label{eqn:electricFieldEnergy:500}
\begin{aligned}
F &= \inv{4 \pi \epsilon_0 c} \int \frac{I d\Bs \wedge \rcap}{r^2}
\end{aligned}
\end{equation}

Much shorter derivations are possible than this one which was essentially done from first principles.  The one in \citep{doran2003gap}, which also uses the STA formulation, is the shortest I have ever seen, utilizing a vector Green's function for the Laplacian.  However, that requires understanding the geometric calculus chapter of that book, which is a battle for a different day.

\subsection{Magnetic field torque and energy}

TODO: work out on paper and write up.

I created a
\href{http://www.physicsforums.com/showthread.php?p=2072561}{PF thread, electric and magnetic field energy }, to followup on these ideas, and now have
an idea how to proceed.

\section{Complete field energy}

Can a integral of the Lorentz force coupled with Maxwell's equations in their entirety produce the energy expression \(\frac{\epsilon_0}{2}\left(\BE^2 + c^2\BB^2\right)\)?  It seems like cheating to add these arbitrarily and then follow the Poynting derivation by taking derivatives.  That shows this quantity is a conserved quantity, but does it really show that it is the
energy?  One could imagine that there could be other terms in a total energy expression such as \(\BE \cdot \BB\).

Looking in more detail at the right hand side of the energy/Poynting
relationship is the key.   That is

\begin{equation}\label{eqn:electric_field_energy:poyntingConservation}
\begin{aligned}
\PD{t}{}\frac{\epsilon_0}{2} \left(\BE^2 + c^2 \BB^2\right) + c^2 \epsilon_0 \spacegrad \cdot (\BE \cross \BB) &= -\BE \cdot \Bj
\end{aligned}
\end{equation}

Two questions to ask.  The first is that if the left hand side is to be
a conserved quantity then we need the right hand side to be one too?
Is that really the case?
Second, how can this be related to work done (a line integral of the Lorentz
force).

The second question is easiest, and the result actually follows directly.

\begin{equation}\label{eqn:electricFieldEnergy:520}
\begin{aligned}
\text{Work done moving a charge against the Lorentz force}
&= \int \BF \cdot (-d\Bx) \\
&= \int q ( \BE + \Bv \cross \BB ) \cdot (-d\Bx) \\
&= -\int q ( \BE + \Bv \cross \BB ) \cdot \Bv dt \\
&= -\int q \BE \cdot \Bv dt \\
&= -\int \BE \cdot \Bj dt dV \\
\end{aligned}
\end{equation}

From this we see that \(-\BE \cdot \Bj\) is the rate of change of power density
in an infinitesimal volume!

Let us write
\begin{equation}\label{eqn:electricFieldEnergy:540}
\begin{aligned}
U &= \frac{\epsilon_0}{2} \left(\BE^2 + c^2 \BB^2\right) \\
\BP &= \inv{\mu_0} (\BE \cross \BB)
\end{aligned}
\end{equation}

and now take \eqnref{eqn:electric_field_energy:poyntingConservation} and integrate over a (small)
volume

\begin{equation}\label{eqn:electricFieldEnergy:560}
\begin{aligned}
\int_V \PD{t}{U} dV + \int_{\partial V} \BP \cdot \ncap dA &=
-\int_V (\BE \cdot \Bj ) dV
\end{aligned}
\end{equation}

So, for a small time increment \(\Delta t = t_1 - t_0\),
corresponding to the start and end times of the particle
at the boundaries of the work line integral, we have

\begin{equation}\label{eqn:electricFieldEnergy:580}
\begin{aligned}
\text{ Work done on particle against field } &=
\int_{t_0}^{t_1}
\int_V \PD{t}{U} dV dt +
\int_{t_0}^{t_1}
\int_{\partial V} \BP \cdot \ncap dA dt  \\
&=
\int_V (U(t_1) - U(t_0)) dV +
\int_{t_0}^{t_1}
\int_{\partial V} \BP \cdot \ncap dA dt  \\
&=
\int_V \Delta U dV +
\int_{t_0}^{t_1}
\int_{\partial V} \BP \cdot \ncap dA dt
\end{aligned}
\end{equation}

Roughly speaking, it appears that the energy provided to move a charge against the field is absorbed into the field in one of two parts, one of which is
what gets identified as the energy of the field \(\int U dV\).  The other part is the time integral of the flux through the surface of the volume of this Poynting
vector \(\BP\).

\subsection{Dimensional analysis}

That is a funny looking term though?  Where would we see momentum integrated over time in classical mechanics?

\begin{equation}\label{eqn:electricFieldEnergy:600}
\begin{aligned}
\int m v dt = m x
\end{aligned}
\end{equation}

Let us look at the dimensions of all the terms in the conservation equation.  We have identified the \(\Bj \cdot \BE\) term with energy density, and
should see this

\begin{equation}\label{eqn:electricFieldEnergy:620}
\begin{aligned}
[ \Bj \BE ]
&= [ (q v /x^3) (F/ q) ] \\
&= [ (x /(x^3 t)) (m x/ t^2) ] \\
&= [ m(x^2/t^2) /(x^3 t) ] \\
&= \frac{\text{Energy}}{ \text{Volume} \times \text{Time}}
\end{aligned}
\end{equation}

Good.  That is what should have been the case.

Now, for the \(U\) term we must then have

\begin{equation}\label{eqn:electricFieldEnergy:640}
\begin{aligned}
[ {U} ]
&= \frac{\text{Energy}}{ {Volume} }
\end{aligned}
\end{equation}

Okay, that is good too, since we were calling \(U\) energy density.  Now for the Poynting term we have

\begin{equation}\label{eqn:electricFieldEnergy:660}
\begin{aligned}
[ \spacegrad \cdot \BP ] &= [1/x] [ \BP ]
\end{aligned}
\end{equation}

So we have
\begin{equation}\label{eqn:electricFieldEnergy:680}
\begin{aligned}
[ \BP ] &= [1/x] [ \BP ] \\
&= \frac{\text{Energy} \times \text{velocity} }{ \text{Volume} }
\end{aligned}
\end{equation}

For uniform dimensions of all the terms this suggests that it is perhaps more natural to work with velocity scaled quantity, with

\begin{equation}\label{eqn:electricFieldEnergy:700}
\begin{aligned}
\frac{[ \BP ]}{\text{Velocity}} &= \frac{\text{Energy} }{ \text{Volume} }
\end{aligned}
\end{equation}

Rewriting the conservation equation scaling by a velocity, for which the obvious generic velocity choice is naturally \(c\), we have

\begin{equation}\label{eqn:electricFieldEnergy:720}
\begin{aligned}
\inv{c} \PD{t}{} U + \spacegrad \cdot \frac{\BP}{c} &= -\frac{\Bj}{c} \cdot \BE
\end{aligned}
\end{equation}

Written this way we have \(1/ct\) with dimensions of inverse distance matching the divergence, and the dimensions of \(U\), and \(\BP/c\) are both energy density.  Now it makes a bit more sense to say that the work done moving the charge against the field supplies energy to the field in some fashion between these two terms.

\subsection{A note on the scalar part of the covariant Lorentz force}

The covariant formulation of the Lorentz force equation, when wedged with \(\gamma_0\) has been seen to recover the traditional Lorentz force equation
(with a required modification to use relativistic momentum), but there was a scalar term that was unaccounted for.

Recall that the covariant Lorentz force, with derivatives all in terms of proper time, was

\begin{equation}\label{eqn:electricFieldEnergy:740}
\begin{aligned}
m \dot{p}
&= q F \cdot (v/c) \\
&= \frac{q}{2c} (F v - v F) \\
&= \frac{q}{2c} ((\BE + ic\BB) \gamma_0(\dot{x^0} - \dot{x^k}\sigma_k) - \gamma_0 ( \dot{x^0} - \dot{x^k}\sigma_k) (\BE + i c \BB)) \\
\end{aligned}
\end{equation}

In terms of time derivatives, where factors of \(\gamma\) can be canceled
on each side, we have
\begin{equation}\label{eqn:electricFieldEnergy:760}
\begin{aligned}
m \frac{dp}{dt}
&= \frac{q}{2} \gamma_0 ( (-\BE + ic\BB) (1 - \Bv/c) - ( 1 - \Bv/c) (\BE + i c \BB)) \\
\end{aligned}
\end{equation}

After some reduction this is
\begin{equation}\label{eqn:electricFieldEnergy:780}
\begin{aligned}
m \frac{dp}{dt} &= q ( -\BE \cdot \Bv/c + (\BE + \Bv \cross \BB) ) \gamma_0
\end{aligned}
\end{equation}

Or, with an explicit spacetime split for all components
\begin{equation}\label{eqn:electricFieldEnergy:800}
\begin{aligned}
mc \frac{d\gamma}{dt} &= -q \BE \cdot \Bv/c \\
m \frac{d \gamma \Bv}{dt} &= q (\BE + \Bv \cross \BB) )
\end{aligned}
\end{equation}

We have got the spatial vector Lorentz force in the second term, and now have an idea what this \(-\Bj \cdot \BE\) term is in the energy momentum
vector.  It is not a random thing, but an intrinsic part (previously ignored) of the covariant Lorentz force.

Now recall that when the time variation of the Poynting was studied in \chapcite{PJpoyntingRate} we had what looked like the Lorentz force
components in all the right hand side terms.  Let us reiterate that here, putting all the bits together

\begin{equation}\label{eqn:electricFieldEnergy:820}
\begin{aligned}
\inv{c} \PD{t}{} U + \spacegrad \cdot \frac{\BP}{c} &= -\frac{\Bj}{c} \cdot \BE \\
\inv{c^2}\PD{t}{\BP} + \sum_k \sigma_k \spacegrad \cdot \BT_k &= -(\Bj \cross \BB + \rho \BE)
\end{aligned}
\end{equation}

We have four scalar equations, where each one contains exactly one of the four vector components of the Lorentz force.  This makes the stress
energy tensor seem a lot less random.  Now the interesting thing about this is that each of these equations required nothing more
than a bunch of algebra applied to the Maxwell equation.  Doing so required no use of the Lorentz force, but it shows up magically
as an intrinsic quantity
associated with the Maxwell equation.  Before this I thought that
one really needed both Maxwell's equation and the Lorentz force equation
(or their corresponding Lagrangians), but looking at this result
the Lorentz force seems to more of a property of the field than
a fundamental quantity in its own right (although some means to relate
this stress energy tensor to force is required).

   %
% Copyright � 2012 Peeter Joot.  All Rights Reserved.
% Licenced as described in the file LICENSE under the root directory of this GIT repository.
%

%
%
\mychapter{Energy momentum tensor}
\label{chap:PJemstresstensor}
%\date{Jan 01, 2009.  energyMomentumTensor.tex}

\section{Expanding out the stress energy vector in tensor form}

\citep{doran2003gap} defines (with \(\epsilon_0\) omitted),
the energy momentum stress tensor as a vector to
vector mapping of the following form:

\begin{equation}\label{eqn:energyMomentumTensor:20}
\begin{aligned}
T(a)
&= \frac{\epsilon_0}{2} F a \tilde{F}
= - \frac{\epsilon_0}{2} F a F
\end{aligned}
\end{equation}

This quantity can only have vector, trivector, and five vector grades.  The
grade five term must be zero

\begin{equation}\label{eqn:energyMomentumTensor:40}
\begin{aligned}
\gpgrade{T(a)}{5}
&= \frac{\epsilon_0}{2} F \wedge a \wedge \tilde{F} \\
&= \frac{\epsilon_0}{2} a \wedge (F \wedge \tilde{F}) \\
&= 0
\end{aligned}
\end{equation}

Since \((T(a))^{\tilde{}} = T(a)\), the grade three term is also zero (trivectors invert on reversion), so this must therefore be a vector.

As a vector this can be expanded in coordinates

\begin{equation}\label{eqn:energyMomentumTensor:60}
\begin{aligned}
T(a)
&= \left(T(a) \cdot \gamma^\nu \right) \gamma_\nu \\
&= \left(T(a^\mu \gamma_\mu) \cdot \gamma^\nu \right) \gamma_\nu \\
&= a^\mu \gamma_\nu \left(T(\gamma_\mu) \cdot \gamma^\nu \right) \\
\end{aligned}
\end{equation}

It is this last bit that has the form of a traditional tensor, so we can write

\begin{equation}\label{eqn:energyMomentumTensor:80}
\begin{aligned}
T(a) &= a^\mu \gamma_\nu {T_\mu}^{\nu} \\
{T_\mu}^{\nu} &= T(\gamma_\mu) \cdot \gamma^\nu
\end{aligned}
\end{equation}

Let us expand this tensor \({T_\mu}^{\nu}\) explicitly to verify its form.

We want to expand, and dot with \(\gamma^\nu\), the following

\begin{equation}\label{eqn:energyMomentumTensor:100}
\begin{aligned}
-2 \inv{\epsilon_0} \left(T(\gamma_\mu) \cdot \gamma^\nu \right) \gamma_\nu
&= \gpgradeone{(\grad \wedge A) \gamma_\mu (\grad \wedge A)} \\
&= \gpgradeone{
(\grad \wedge A) \cdot \gamma_\mu (\grad \wedge A)
+ (\grad \wedge A) \wedge \gamma^\mu (\grad \wedge A)
} \\
&=
((\grad \wedge A) \cdot \gamma_\mu) \cdot (\grad \wedge A)
+ ((\grad \wedge A) \wedge \gamma_\mu) \cdot (\grad \wedge A)
\\
\end{aligned}
\end{equation}

Both of these will get temporarily messy, so let us do them in parts.  Starting
with

\begin{equation}\label{eqn:energyMomentumTensor:120}
\begin{aligned}
(\grad \wedge A) \cdot \gamma_\mu
&= (\gamma^{\alpha} \wedge \gamma^{\beta}) \cdot \gamma_{\mu} \partial_{\alpha} A_{\beta} \\
&= (\gamma^{\alpha} {\delta^{\beta}}_{\mu} -\gamma^{\beta} {\delta^{\alpha}}_{\mu} ) \partial_{\alpha} A_{\beta} \\
&=
\gamma^{\alpha} \partial_{\alpha} A_{\mu}
-\gamma^{\beta} \partial_{\mu} A_{\beta} \\
&= \gamma^{\alpha} (\partial_{\alpha} A_{\mu} -\partial_{\mu} A_{\alpha} ) \\
&= \gamma^{\alpha} F_{\alpha \mu} \\
\end{aligned}
\end{equation}


\begin{equation}\label{eqn:energyMomentumTensor:140}
\begin{aligned}
((\grad \wedge A) \cdot \gamma_\mu) \cdot (\grad \wedge A)
&= (\gamma^{\alpha} F_{\alpha \mu}) \cdot (\gamma_{\beta} \wedge \gamma_{\lambda}) \partial^{\beta} A^{\lambda} \\
%&=
%\partial^{\beta} A^{\lambda} F_{\alpha \mu}
%\gamma^{\alpha} \cdot (\gamma_{\beta} \wedge \gamma_{\lambda})
%\\
&=
\partial^{\beta} A^{\lambda} F_{\alpha \mu}
(
{\delta^{\alpha}}_{\beta} \gamma_{\lambda}
-{\delta^{\alpha}}_{\lambda} \gamma_{\beta}
)
\\
&=
(\partial^{\alpha} A^{\beta} F_{\alpha \mu} -\partial^{\beta} A^{\alpha} F_{\alpha \mu} )\gamma_{\beta}
\\
&= F^{\alpha \beta} F_{\alpha \mu} \gamma_{\beta} \\
\end{aligned}
\end{equation}

So, by dotting with \(\gamma^\nu\) we have

\begin{equation}\label{eqn:energy_momentum_tensor:firstPartDone}
\begin{aligned}
((\grad \wedge A) \cdot \gamma_\mu) \cdot (\grad \wedge A) \cdot \gamma^{\nu}
&= F^{\alpha \nu} F_{\alpha \mu}
\end{aligned}
\end{equation}

Moving on to the next bit,
\((((\grad \wedge A) \wedge \gamma^\mu) \cdot (\grad \wedge A)) \cdot \gamma^\nu\).� By inspection the first part of this is

\begin{equation}\label{eqn:energyMomentumTensor:160}
\begin{aligned}
(\grad \wedge A) \wedge \gamma_\mu
&= (\gamma_\mu)^2 (\gamma^{\alpha} \wedge \gamma^{\beta}) \wedge \gamma^{\mu} \partial_{\alpha} A_{\beta} \\
\end{aligned}
\end{equation}

so dotting with \(\grad \wedge A\), we have

\begin{equation}\label{eqn:energyMomentumTensor:180}
\begin{aligned}
((\grad \wedge A) \wedge \gamma_\mu) \cdot (\grad \wedge A)
&=
(\gamma_\mu)^2
\partial_{\alpha} A_{\beta}
\partial^{\lambda} A^{\delta}
(\gamma^{\alpha} \wedge \gamma^{\beta} \wedge \gamma^{\mu}) \cdot
(\gamma_{\lambda} \wedge \gamma_{\delta})
\\
&=
(\gamma_\mu)^2
\partial_{\alpha} A_{\beta}
\partial^{\lambda} A^{\delta}
((\gamma^{\alpha} \wedge \gamma^{\beta} \wedge \gamma^{\mu}) \cdot \gamma_{\lambda} ) \cdot \gamma_{\delta}
\\
\end{aligned}
\end{equation}

Expanding just the dot product parts of this we have
\begin{equation}\label{eqn:energyMomentumTensor:200}
\begin{aligned}
&(((\gamma^{\alpha} \wedge \gamma^{\beta}) \wedge \gamma^{\mu}) \cdot \gamma_{\lambda} ) \cdot \gamma_{\delta} \\
&=
(\gamma^{\alpha} \wedge \gamma^{\beta}) {\delta^{\mu}}_{\lambda}
-(\gamma^{\alpha} \wedge \gamma^{\mu}) {\delta^{\beta}}_{\lambda}
+(\gamma^{\beta} \wedge \gamma^{\mu}) {\delta^{\alpha}}_{\lambda}
) \cdot \gamma_{\delta}
\\
%&=
%(
%  \gamma^{\alpha} {\delta^{\beta}}_{\delta} {\delta^{\mu}}_{\lambda}
%- \gamma^{\alpha} {\delta^{\mu}}_{\delta} {\delta^{\beta}}_{\lambda}
%+ \gamma^{\beta} {\delta^{\mu}}_{\delta} {\delta^{\alpha}}_{\lambda}
%- \gamma^{\beta} {\delta^{\alpha}}_{\delta} {\delta^{\mu}}_{\lambda}
%+ \gamma^{\mu} {\delta^{\alpha}}_{\delta} {\delta^{\beta}}_{\lambda}
%- \gamma^{\mu} {\delta^{\beta}}_{\delta} {\delta^{\alpha}}_{\lambda}
%)
%\\
&=
  \gamma^{\alpha} ({\delta^{\beta}}_{\delta} {\delta^{\mu}}_{\lambda}
-            {\delta^{\mu}}_{\delta} {\delta^{\beta}}_{\lambda})
+ \gamma^{\beta} ({\delta^{\mu}}_{\delta} {\delta^{\alpha}}_{\lambda}
-            {\delta^{\alpha}}_{\delta} {\delta^{\mu}}_{\lambda})
+ \gamma^{\mu} ({\delta^{\alpha}}_{\delta} {\delta^{\beta}}_{\lambda}
-                 {\delta^{\beta}}_{\delta} {\delta^{\alpha}}_{\lambda})
\\
\end{aligned}
\end{equation}

This can now be applied to \(\partial^{\lambda} A^{\delta}\)

\begin{equation}\label{eqn:energyMomentumTensor:220}
\begin{aligned}
\partial^{\lambda} A^{\delta} &(((\gamma^{\alpha} \wedge \gamma^{\beta}) \wedge \gamma^{\mu}) \cdot \gamma_{\lambda} ) \cdot \gamma_{\delta} \\
&=
  \partial^{\mu} A^{\beta} \gamma^{\alpha}
- \partial^{\beta} A^{\mu} \gamma^{\alpha}
+ \partial^{\alpha} A^{\mu} \gamma^{\beta}
- \partial^{\mu} A^{\alpha} \gamma^{\beta}
+ \partial^{\beta} A^{\alpha} \gamma^{\mu}
- \partial^{\alpha} A^{\beta} \gamma^{\mu}
\\
&=
(  \partial^{\mu} A^{\beta}
- \partial^{\beta} A^{\mu} ) \gamma^{\alpha}
+( \partial^{\alpha} A^{\mu}
- \partial^{\mu} A^{\alpha} ) \gamma^{\beta}
+( \partial^{\beta} A^{\alpha}
- \partial^{\alpha} A^{\beta} ) \gamma^{\mu}
\\
&=
%(  \partial^{\mu} A^{\beta} - \partial^{\beta} A^{\mu} )
F^{\mu \beta}
\gamma^{\alpha}
+
%( \partial^{\alpha} A^{\mu} - \partial^{\mu} A^{\alpha} )
F^{\alpha \mu}
\gamma^{\beta}
+
%( \partial^{\beta} A^{\alpha} - \partial^{\alpha} A^{\beta} )
F^{\beta \alpha}
\gamma^{\mu}
\\
\end{aligned}
\end{equation}

This is getting closer, and we can now write
\begin{equation}\label{eqn:energyMomentumTensor:240}
\begin{aligned}
((\grad \wedge A) \wedge \gamma_\mu) \cdot (\grad \wedge A) &=
(\gamma_\mu)^2 \partial_{\alpha} A_{\beta}
(
  F^{\mu \beta} \gamma^{\alpha}
+ F^{\alpha \mu} \gamma^{\beta}
+ F^{\beta \alpha} \gamma^{\mu}
) \\
&=
  (\gamma_\mu)^2 \partial_{\beta} A_{\alpha} F^{\mu \alpha} \gamma^{\beta}
+ (\gamma_\mu)^2 \partial_{\alpha} A_{\beta} F^{\alpha \mu} \gamma^{\beta}
+ (\gamma_\mu)^2 \partial_{\alpha} A_{\beta} F^{\beta \alpha} \gamma^{\mu}
\\
&=
  F^{\beta \alpha} F_{\mu \alpha} \gamma_{\beta}
+ \partial_{\alpha} A_{\beta} F^{\beta \alpha} \gamma_{\mu}
\\
\end{aligned}
\end{equation}

This can now be dotted with \(\gamma^\nu\),

\begin{equation}\label{eqn:energyMomentumTensor:260}
\begin{aligned}
((\grad \wedge A) \wedge \gamma_\mu) \cdot (\grad \wedge A) \cdot \gamma^\nu
&=
 F^{\beta \alpha} F_{\mu \alpha} {\delta_{\beta}}^\nu
+ \partial_{\alpha} A_{\beta} F^{\beta \alpha} {\delta_{\mu}}^\nu
\\
%&= F^{\nu \alpha} F_{\mu \alpha} + \inv{2} F_{\alpha \beta} F^{\beta \alpha} {\delta_{\mu}}^\nu \\
\end{aligned}
\end{equation}

which is
\begin{equation}\label{eqn:energy_momentum_tensor:secondPart}
\begin{aligned}
((\grad \wedge A) \wedge \gamma_\mu) \cdot (\grad \wedge A) \cdot \gamma^\nu
&= F^{\nu \alpha} F_{\mu \alpha} + \inv{2} F_{\alpha \beta} F^{\beta \alpha} {\delta_{\mu}}^\nu
%F^{\nu \beta} F_{\mu \beta} +\inv{2} F_{\alpha \beta} F^{\beta \alpha} {\delta_{\mu}}^\nu
\end{aligned}
\end{equation}

The final combination of results \eqnref{eqn:energy_momentum_tensor:firstPartDone}, and
\eqnref{eqn:energy_momentum_tensor:secondPart} gives

\begin{equation}\label{eqn:energyMomentumTensor:280}
\begin{aligned}
(F \gamma_\mu F ) \cdot \gamma^\nu
&=
2 F^{\alpha \nu} F_{\alpha \mu}
+\inv{2} F_{\alpha \beta} F^{\beta \alpha} {\delta_{\mu}}^\nu
\end{aligned}
\end{equation}

Yielding the tensor

\begin{equation}\label{eqn:energy_momentum_tensor:messyTensor}
\begin{aligned}
{T_\mu}^{\nu}
&=
\epsilon_0 \left(
\inv{4} F_{\alpha \beta} F^{\alpha \beta} {\delta_{\mu}}^\nu
-
F_{\alpha \mu}
F^{\alpha \nu}
\right)
\end{aligned}
\end{equation}

\section{Validate against previously calculated Poynting result}

In \chapcite{PJpoynting}, the electrodynamic energy density \(U\) and momentum flux density vectors were related as follows

\begin{equation}\label{eqn:energy_momentum_tensor:fromPoyntingNotes}
\begin{aligned}
U &= \frac{\epsilon_0}{2}\left( \BE^2 + c^2 \BB^2 \right) \\
\BP &= \epsilon_0 c^2 \BE \cross \BB = \epsilon_0 c (i c \BB) \cdot \BE \\
0 &= \PD{t}{}\frac{\epsilon_0}{2} \left(\BE^2 + c^2 \BB^2\right) + c^2 \epsilon_0 \spacegrad \cdot (\BE \cross \BB) + \BE \cdot \Bj
\end{aligned}
\end{equation}

Additionally the energy and momentum flux densities are components of this stress tensor four vector

\begin{equation}\label{eqn:energyMomentumTensor:300}
\begin{aligned}
T(\gamma_0) &= U \gamma_0 + \inv{c} \BP \gamma_0 \\
\end{aligned}
\end{equation}

From this we can read the first row of the tensor elements

\begin{equation}\label{eqn:energyMomentumTensor:320}
\begin{aligned}
{T_0}^0 &= U
= \frac{\epsilon_0}{2}\left( \BE^2 + c^2 \BB^2 \right) \\
{T_0}^k &= \inv{c} (\BP \gamma_0) \cdot \gamma^k = \epsilon_0 c E^a B^b \epsilon_{k a b}
\end{aligned}
\end{equation}

Let us compare these to \eqnref{eqn:energy_momentum_tensor:messyTensor}, which gives
\begin{equation}\label{eqn:energyMomentumTensor:340}
\begin{aligned}
{T_0}^{0}
&= \epsilon_0 \left( \inv{4} F_{\alpha \beta} F^{\alpha \beta} - F_{\alpha 0} F^{\alpha 0} \right) \\
&= \frac{\epsilon_0}{4} \left( F_{\alpha j} F^{\alpha j} - {3} F_{j 0} F^{j 0} \right) \\
&= \frac{\epsilon_0}{4} \left( F_{m j} F^{m j} +F_{0 j} F^{0 j} - {3} F_{j 0} F^{j 0} \right) \\
&= \frac{\epsilon_0}{4} \left( F_{m j} F^{m j} - {2} F_{j 0} F^{j 0} \right) \\
{T_0}^{k}
&= -\epsilon_0 F_{\alpha 0} F^{\alpha k} \\
&= -\epsilon_0 F_{j 0} F^{j k} \\
\end{aligned}
\end{equation}

Now, our field in terms of electric and magnetic coordinates is

\begin{equation}\label{eqn:energyMomentumTensor:360}
\begin{aligned}
F &= \BE + i c \BB \\
  &= E^k \gamma_k \gamma_0 + i c B^k \gamma_k \gamma_0 \\
  &= E^k \gamma_k \gamma_0 - c \epsilon_{a b k} B^k \gamma_a \gamma_b
\end{aligned}
\end{equation}

so the electric field tensor components are

\begin{equation}\label{eqn:energyMomentumTensor:380}
\begin{aligned}
F^{j 0}
&= (F \cdot \gamma^0) \cdot \gamma^j \\
&= E^k {\delta_k}^j \\
&= E^j
\end{aligned}
\end{equation}

and
\begin{equation}\label{eqn:energyMomentumTensor:400}
\begin{aligned}
F_{j 0} &= (\gamma_j)^2 (\gamma_0)^2 F^{j 0} \\
&= -E^j
\end{aligned}
\end{equation}

and the magnetic tensor components are

\begin{equation}\label{eqn:energyMomentumTensor:420}
\begin{aligned}
F^{m j} &= F_{m j} \\
&= - c \epsilon_{a b k} B^k ((\gamma_a \gamma_b) \cdot \gamma_{j}) \cdot \gamma_m \\
&= - c \epsilon_{m j k} B^k
\end{aligned}
\end{equation}

This gives
\begin{equation}\label{eqn:energyMomentumTensor:440}
\begin{aligned}
{T_0}^{0}
&= \frac{\epsilon_0}{4} \left( 2 c^2 B^k B^k + {2} E^j E^{j} \right) \\
&= \frac{\epsilon_0}{2} \left( c^2 \BB^2 + \BE^2 \right) \\
{T_0}^{k}
&= \epsilon_0 E^{j} F^{j k} \\
&= \epsilon_0 c \epsilon_{k e f} E^e B^f \\
&= \epsilon_0 (c \BE \cross \BB)_k \\
&= \inv{c} (\BP \cdot \sigma_k)
\end{aligned}
\end{equation}

Okay, good.  This checks 4 of the elements of \eqnref{eqn:energy_momentum_tensor:messyTensor} against the explicit \(\BE\) and \(\BB\) based representation of \(T(\gamma_0)\) in \eqnref{eqn:energy_momentum_tensor:fromPoyntingNotes}, leaving only 6 unique elements in the remaining parts of the (symmetric) tensor to verify.

\section{Four vector form of energy momentum conservation relationship}

One can observe that there is a spacetime divergence hiding there directly
in the energy conservation equation of
\eqnref{eqn:energy_momentum_tensor:fromPoyntingNotes}.  In particular, writing the last of those as

\begin{equation}\label{eqn:energyMomentumTensor:460}
\begin{aligned}
0 &= \partial_{0}{}\frac{\epsilon_0}{2} \left(\BE^2 + c^2 \BB^2\right) + \spacegrad \cdot \BP/c + \BE \cdot \Bj/c
\end{aligned}
\end{equation}

We can then write the energy-momentum parts as a four vector divergence
\begin{equation}\label{eqn:energyMomentumTensor:480}
\begin{aligned}
\grad \cdot \left(
\frac{\epsilon_0 \gamma_0}{2} \left(\BE^2 + c^2 \BB^2\right)
+ \inv{c} P^k \gamma_k
\right) &= - \BE \cdot \Bj/c
\end{aligned}
\end{equation}

Since we have a divergence relationship, it should also be possible to convert a spacetime hypervolume
integration of this quantity into a time-surface integral or a pure volume integral.  Pursing this
will probably clarify how the tensor is related to the
hypersurface flux as mentioned in the text here, but
making this concrete will take a bit more thought.

Having seen that we have a divergence relationship for the energy momentum tensor in the rest frame, it is clear that the
Poynting energy momentum flux relationship should follow much more directly if we play it backwards in a relativistic setting.

This is a very sneaky way to do it since we have to have seen the answer to get there, but it should avoid the complexity of
trying to factor out the spacial gradients and recover the divergence relationship that provides the Poynting vector.  Our sneaky
starting point is to compute

\begin{equation}\label{eqn:energyMomentumTensor:500}
\begin{aligned}
\grad \cdot ( F \gamma_0 \tilde{F} )
&= \gpgradezero{ \grad (F \gamma_0 \tilde{F}) } \\
&= \gpgradezero{
(\grad F) \gamma_0 \tilde{F}
+ \dot{\grad} F \gamma_0 \dot{\tilde{F}}
} \\
&= \gpgradezero{
(\grad F) \gamma_0 \tilde{F}
+ \dot{\tilde{F}} \dot{\grad} F \gamma_0
} \\
\end{aligned}
\end{equation}

Since this is a scalar quantity, it is equal to its own reverse and we can reverse all factors in this second term to convert the left acting
gradient to a more regular right acting form.  This is

\begin{equation}\label{eqn:energyMomentumTensor:520}
\begin{aligned}
\grad \cdot ( F \gamma_0 \tilde{F} )
&= \gpgradezero{
(\grad F) \gamma_0 \tilde{F}
+ \gamma_0 \tilde{F} (\grad F)
} \\
\end{aligned}
\end{equation}

Now using Maxwell's equation \(\grad F = J/\epsilon_0 c\), we have

\begin{equation}\label{eqn:energyMomentumTensor:540}
\begin{aligned}
\grad \cdot ( F \gamma_0 \tilde{F} )
&=
\inv{\epsilon_0 c}
\gpgradezero{
J \gamma_0 \tilde{F}
+ \gamma_0 \tilde{F} J
} \\
&=
\frac{2}{\epsilon_0 c}
\gpgradezero{
J \gamma_0 \tilde{F}
} \\
&= \frac{2}{\epsilon_0 c} (J \wedge \gamma_0) \cdot \tilde{F} \\
\end{aligned}
\end{equation}

Now, \(J = \gamma_0 c \rho + \gamma_k J^k\), so \(J \wedge \gamma_0 = J^k \gamma_k \gamma_0 = J^k \sigma_k = \Bj\), and dotting this with \(\tilde{F} = -\BE - i c \BB\) will pick up only the (negated) electric field components, so we have

\begin{equation}\label{eqn:energyMomentumTensor:560}
\begin{aligned}
(J \wedge \gamma_0) \cdot \tilde{F} &= \Bj \cdot (-\BE)
\end{aligned}
\end{equation}

Although done in \chapcite{PJpoynting}, for
completeness let us re-expand \(F \gamma_0 \tilde{F}\) in terms of the electric and magnetic field vectors.

\begin{equation}\label{eqn:energyMomentumTensor:580}
\begin{aligned}
F \gamma_0 \tilde{F}
&= -(\BE + i c \BB) \gamma_0 (\BE + i c \BB) \\
%&= - \gamma_0 (-\BE + i c \BB) (\BE + i c \BB) \\
&= \gamma_0 (\BE - i c \BB) (\BE + i c \BB) \\
&= \gamma_0 (\BE^2 + c^2 \BB^2 + i c (\BE \BB -\BB \BE) ) \\
&= \gamma_0 (\BE^2 + c^2 \BB^2 + 2 i c (\BE \wedge \BB) ) \\
&= \gamma_0 (\BE^2 + c^2 \BB^2 - 2 c (\BE \cross \BB) ) \\
\end{aligned}
\end{equation}

Next, we want an explicit spacetime split of the gradient

\begin{equation}\label{eqn:energyMomentumTensor:600}
\begin{aligned}
\grad \gamma_0
&= (\gamma^0 \partial_0 + \gamma^k \partial_k) \gamma_0 \\
&= \partial_0 - \gamma_k \gamma_0 \partial_k \\
&= \partial_0 - \sigma_k \partial_k \\
&= \partial_0 - \spacegrad \\
\end{aligned}
\end{equation}

We are now in shape to assemble all the intermediate results for the left hand side

\begin{equation}\label{eqn:energyMomentumTensor:620}
\begin{aligned}
\grad \cdot (F \gamma_0 \tilde{F})
&= \gpgradezero{ \grad (F \gamma_0 \tilde{F}) } \\
&= \gpgradezero{ (\partial_0 - \spacegrad) (\BE^2 + c^2 \BB^2 - 2 c (\BE \cross \BB) ) } \\
&= \partial_0 (\BE^2 + c^2 \BB^2) + 2 c \spacegrad \cdot (\BE \cross \BB)
\end{aligned}
\end{equation}

With a final reassembly of the left and right hand sides of
\(\grad \cdot T(\gamma_0)\),
the spacetime divergence of the rest frame stress vector we have
\begin{equation}\label{eqn:energyMomentumTensor:640}
\begin{aligned}
\inv{c} \partial_t (\BE^2 + c^2 \BB^2) + 2 c \spacegrad \cdot (\BE \cross \BB) &= -\frac{2}{c \epsilon_0}\Bj \cdot \BE
\end{aligned}
\end{equation}

Multiplying through by \(\epsilon_0 c/2\) we have the classical Poynting vector energy conservation relationship.

\begin{equation}\label{eqn:energy_momentum_tensor:conservation}
\begin{aligned}
\PD{t}{} \frac{\epsilon_0}{2}(\BE^2 + c^2 \BB^2) + \spacegrad \cdot \inv{\mu_0}(\BE \cross \BB) &= -\Bj \cdot \BE
\end{aligned}
\end{equation}

Observe that the momentum flux density, the Poynting vector \(\BP = (\BE \cross \BB)/\mu_0\),
is zero in the rest frame, which makes sense since there is no magnetic field
for a static charge distribution.  So with no currents and therefore no magnetic fields the field energy is a constant.

\subsection{Transformation properties}

\Eqnref{eqn:energy_momentum_tensor:conservation} is the explicit spacetime expansion of the equivalent relativistic equation

\begin{equation}\label{eqn:energyMomentumTensor:660}
\begin{aligned}
\grad \cdot \left( c T(\gamma_0) \right) &=
\grad \cdot \left(\frac{c \epsilon_0}{2} F \gamma_0 \tilde{F}\right) = \gpgradezero{ J \gamma_0 \tilde{F} }
\end{aligned}
\end{equation}

This has all the same content, but in relativistic form seems almost trivial.  While the stress vector \(T(\gamma_0)\) is not itself
a relativistic invariant, this divergence equation is.

Suppose we form a Lorentz transformation \(\LL(x) = R x \tilde{R}\), applied to this equation we have

\begin{equation}\label{eqn:energyMomentumTensor:680}
\begin{aligned}
F'
&= (R\grad \tilde{R}) \wedge (R A \tilde{R}) \\
&= \gpgradetwo{ R \grad \tilde{R} R A \tilde{R} } \\
&= \gpgradetwo{ R \grad A \tilde{R} } \\
&= R (\grad \wedge A) \tilde{R} \\
&= R F \tilde{R} \\
\end{aligned}
\end{equation}

Transforming all the objects in the equation we have
\begin{equation}\label{eqn:energyMomentumTensor:700}
\begin{aligned}
\grad' \cdot \left(\frac{c \epsilon_0}{2} F' \gamma_0' \tilde{F'} \right) &= \gpgradezero{ J' \gamma_0' \tilde{F'} } \\
(R \grad \tilde{R}) \cdot \left(\frac{c \epsilon_0}{2} R F \tilde{R} R \gamma_0 R \tilde{R} (R F\tilde{R})^{\tilde{}} \right)
&= \gpgradezero{ R J \tilde{R} R \gamma_0 \tilde{R} (R F \tilde{R})^{\tilde{}} } \\
\end{aligned}
\end{equation}

This is nothing more than the original untransformed quantity

\begin{equation}\label{eqn:energyMomentumTensor:720}
\begin{aligned}
\grad \cdot \left(\frac{c \epsilon_0}{2} F \gamma_0 \tilde{F} \right) &= \gpgradezero{ J \gamma_0 \tilde{F} } \\
\end{aligned}
\end{equation}

\section{Validate with relativistic transformation}

As a relativistic quantity we should be able to verify the messy tensor relationship
by Lorentz transforming the energy density from a rest frame to a
moving frame.

Now let us try the Lorentz transformation of the energy density.

FIXME: TODO.

   %
% Copyright � 2012 Peeter Joot.  All Rights Reserved.
% Licenced as described in the file LICENSE under the root directory of this GIT repository.
%

%
%
\mychapter{Lorentz force relation to the energy momentum tensor}
\label{chap:PJstressEnergyLorentz}
\index{Lorentz force!energy momentum tensor}
%\date{Feb 13, 2009.  stressEnergyLorentz.tex}

\section{Motivation}

Have now made a few excursions related to the concepts of electrodynamic
field energy and momentum.

In \chapcite{PJpoynting} the energy density rate and Poynting divergence
relationship was demonstrated using Maxwell's equation.  That was:

\begin{equation}\label{eqn:stressEnergyLorentz:20}
\begin{aligned}
\PD{t}{}\frac{\epsilon_0}{2} \left(\BE^2 + c^2 \BB^2\right) + \spacegrad \cdot \inv{\mu_0}(\BE \cross \BB) &= -\BE \cdot \Bj
\end{aligned}
\end{equation}

In terms of the field energy density \(U\), and Poynting vector \(\BP\), this is
\begin{equation}\label{eqn:seLorentz:energyDensityPoyntingDefined}
\begin{aligned}
U &= \frac{\epsilon_0}{2} \left(\BE^2 + c^2 \BB^2\right) \\
\BP &= \inv{\mu_0}(\BE \cross \BB) \\
\PD{t}{U} + \spacegrad \cdot \BP &= -\BE \cdot \Bj
\end{aligned}
\end{equation}

In \chapcite{PJemstresstensor} this was related to the
energy momentum four vectors

\begin{equation}\label{eqn:seLorentz:lorentzForceT}
\begin{aligned}
T(a) &= \frac{\epsilon_0}{2} F a \tilde{F}
\end{aligned}
\end{equation}

as defined
in \citep{doran2003gap}, but the big picture view
of things was missing.

Later in \chapcite{PJpoyntingRate} the rate of change of Poynting vector
was calculated, with an additional attempt to relate this to \(T(\gamma_\mu)\).

These relationships, and the operations required to factoring out the divergence were considerably messier.

Finally, in \chapcite{PJelectricFieldEnergy} the four vector \(T(\gamma_\mu)\)
was related to the Lorentz force and the work done moving a charge against
a field.  This provides the natural context for the energy momentum tensor,
since it appears that the spacetime divergence of each of the
\(T(\gamma_\mu)\) four vectors appears to be a component of the
four vector Lorentz force (density).

In these notes the divergences will be calculated to confirm the
connection between the Lorentz force and energy momentum tensor directly.
This is actually expected to be simpler than the previous calculations.

It is also
potentially of interest, as shown in \chapcite{PJFourierVacuum}, and
\chapcite{PJplaneWave}
that the energy density and Poynting vectors, and energy momentum four vector,
were seen to be naturally expressible as Hermitian conjugate operations

\begin{equation}\label{eqn:stressEnergyLorentz:40}
\begin{aligned}
F^\dagger &= \gamma_0 \tilde{F} \gamma_0
\end{aligned}
\end{equation}
\begin{equation}\label{eqn:stressEnergyLorentz:60}
\begin{aligned}
T(\gamma_0) &= \frac{\epsilon_0}{2} F F^\dagger \gamma_0
\end{aligned}
\end{equation}
\begin{equation}\label{eqn:stressEnergyLorentz:80}
\begin{aligned}
U &= T(\gamma_0) \cdot \gamma_0 = \frac{\epsilon_0}{4} \left(F F^\dagger + F^\dagger F \right) \\
\BP/c &= T(\gamma_0) \wedge \gamma_0 = \frac{\epsilon_0}{4} \left(F F^\dagger - F^\dagger F \right)
\end{aligned}
\end{equation}

It is conceivable that a generalization of Hermitian conjugation, where the spatial basis vectors are used instead of \(\gamma_0\), will
provide a mapping and driving structure from the Four vector quantities and the somewhat scrambled seeming set
of relationships observed in the split spatial and time domain.  That will also be explored here.

\section{Spacetime divergence of the energy momentum four vectors}

The spacetime divergence of the field energy momentum four vector \(T(\gamma_0)\) has been calculated previously.  Let us redo this
calculation for the other components.

\begin{equation}\label{eqn:stressEnergyLorentz:100}
\begin{aligned}
\grad \cdot T(\gamma_\mu)
&= \frac{\epsilon_0}{2} \gpgradezero{ \grad (F \gamma_\mu \tilde{F}) } \\
&= \frac{\epsilon_0}{2} \gpgradezero{ (\grad F) \gamma_\mu \tilde{F} + (\tilde{F} \grad) F \gamma_\mu } \\
&= \frac{\epsilon_0}{2} \gpgradezero{ (\grad F) \gamma_\mu \tilde{F} + \gamma_\mu \tilde{F} (\grad F) } \\
&= {\epsilon_0} \gpgradezero{ (\grad F) \gamma_\mu \tilde{F} } \\
&= \inv{c} \gpgradezero{ J \gamma_\mu \tilde{F} } \\
\end{aligned}
\end{equation}

The ability to perform cyclic reordering of terms in a scalar product has been used above.  Application of one more
reverse operation (which does not change a scalar), gives us

\begin{equation}\label{eqn:seLorentz:lorentzForceTdivergence}
\begin{aligned}
\grad \cdot T(\gamma_\mu) &= \inv{c} \gpgradezero{ F \gamma_\mu J }
\end{aligned}
\end{equation}

Let us expand the right hand size first.

\begin{equation}\label{eqn:stressEnergyLorentz:120}
\begin{aligned}
\inv{c} \gpgradezero{ F \gamma_\mu J } &= \inv{c} \gpgradezero{ (\BE + i c \BB) \gamma_\mu (c \rho \gamma_0 + \Bj \gamma_0) }
\end{aligned}
\end{equation}

The \(\mu = 0\) term looks the easiest, and for that one we have

\begin{equation}\label{eqn:stressEnergyLorentz:140}
\begin{aligned}
\inv{c} \gpgradezero{ (\BE + i c \BB) (c \rho - \Bj) }  = -\Bj \cdot \BE
\end{aligned}
\end{equation}

Now, for the other terms, say \(\mu = k\), we have

\begin{equation}\label{eqn:stressEnergyLorentz:160}
\begin{aligned}
\inv{c} \gpgradezero{ (\BE + i c \BB) (c \rho \sigma_k - \sigma_k \Bj ) }
&= E^k \rho - \gpgradezero{ i \BB \sigma_k \Bj }  \\
&= E^k \rho - J^a B^b \gpgradezero{ \sigma_1 \sigma_2 \sigma_3 \sigma_b \sigma_k \sigma_a }  \\
&= E^k \rho - J^a B^b \epsilon_{a k b} \\
&= E^k \rho + J^a B^b \epsilon_{k a b} \\
&= (\rho \BE + \Bj \cross \BB) \cdot \sigma_k
\end{aligned}
\end{equation}

Summarizing the two results we have

\begin{equation}\label{eqn:seLorentz:lorentzForcePair}
\begin{aligned}
\inv{c} \gpgradezero{ F \gamma_0 J } &= -\Bj \cdot \BE \\
\inv{c} \gpgradezero{ F \gamma_k J } &= (\rho \BE + \Bj \cross \BB) \cdot \sigma_k
\end{aligned}
\end{equation}

The second of these is easily recognizable as components of the Lorentz force for an element of charge (density).  The first
of these is actually the energy component of the four vector Lorentz force, so expanding that in terms of spacetime quantities
is the next order of business.

\section{Four vector Lorentz Force}

The Lorentz force in covariant form is

%\begin{align}
%m \ddot{x}_\mu &= q F_{} \cdot \frac{dx}{d\tau}
%\end{align}
%
%Or in vector/bivector form

\begin{equation}\label{eqn:seLorentz:lorentzForceGA}
\begin{aligned}
m \ddot{x} &= q F \cdot \dot{x}/c
\end{aligned}
\end{equation}

Two verifications of this are in order.  One is that we get the traditional vector form of the Lorentz force equation from
this and the other is that we can get the traditional tensor form from this equation.

\subsection{Lorentz force in tensor form}
\index{Lorentz force!tensor}

Recovering the tensor form is probably the easier of the two operations.  We have

\begin{equation}\label{eqn:stressEnergyLorentz:180}
\begin{aligned}
m \ddot{x}_\mu \gamma^\mu
&= \frac{q}{2} F_{\alpha\beta} \dot{x}_\sigma (\gamma^{\alpha} \wedge \gamma^\beta) \cdot \gamma^\sigma \\
&= \frac{q}{2} F_{\alpha\beta} \dot{x}^\sigma (\gamma^{\alpha} {\delta^\beta}_\sigma -\gamma^{\beta} {\delta^\alpha}_\sigma ) \\
&= \frac{q}{2} F_{\alpha\beta} \dot{x}^\beta \gamma^{\alpha} - \frac{q}{2} F_{\alpha\beta} \dot{x}^\alpha \gamma^{\beta} \\
\end{aligned}
\end{equation}

Dotting with \(\gamma_\mu\) the right hand side is

\begin{equation}\label{eqn:stressEnergyLorentz:200}
\begin{aligned}
\frac{q}{2} F_{\mu\beta} \dot{x}^\beta - \frac{q}{2} F_{\alpha\mu} \dot{x}^\alpha
&= {q} F_{\mu\alpha} \dot{x}^\alpha
\end{aligned}
\end{equation}

Which recovers the tensor form of the equation
\begin{equation}\label{eqn:seLorentz:lorentzForceTensor}
\begin{aligned}
m \ddot{x}_\mu &= {q} F_{\mu\alpha} \dot{x}^\alpha
\end{aligned}
\end{equation}

\subsection{Lorentz force components in vector form}
\index{Lorentz force!vector form}

\begin{equation}\label{eqn:stressEnergyLorentz:220}
\begin{aligned}
m \gamma \frac{d}{dt} \gamma \left(c + \sigma_k \frac{dx^k}{dt}\right) \gamma_0
&= \frac{q}{2c}(F v - v F) \\
&=
\frac{q \gamma}{2c}
(\BE + i c\BB)
\left(c + \sigma_k \frac{dx^k}{dt}\right) \gamma_0
\\
&\quad -
\frac{q \gamma}{2c}
\left(c + \sigma_k \frac{dx^k}{dt}\right) \gamma_0
(\BE + i c\BB) \\
\end{aligned}
\end{equation}

Right multiplication by \(\gamma_0/\gamma\) we have
\begin{equation}\label{eqn:stressEnergyLorentz:240}
\begin{aligned}
m \frac{d}{dt} \gamma \left(c + \Bv \right)
&= \frac{q }{2c} \left( (\BE + i c\BB) \left(c + \Bv \right) -\left(c + \Bv \right) (-\BE + i c\BB) \right) \\
&= \frac{q }{2c} \left(
%(\BE + i c\BB) (c + \Bv )
%-(c + \Bv ) (-\BE + i c\BB)
+ 2 \BE c
+ \BE \Bv  + \Bv \BE
+ i c (\BB \Bv  - \Bv \BB)
\right) \\
\end{aligned}
\end{equation}

After a last bit of reduction this is
\begin{equation}\label{eqn:stressEnergyLorentz:260}
\begin{aligned}
m \frac{d}{dt} \gamma \left(c + \Bv \right) &= q (\BE + \Bv \cross \BB) + q \BE \cdot \Bv/c
\end{aligned}
\end{equation}

In terms of four vector momentum this is
\begin{equation}\label{eqn:seLorentz:lorentzForceVec}
\begin{aligned}
\dot{p} = q ( \BE \cdot \Bv/c + \BE + \Bv \cross \BB ) \gamma_0
\end{aligned}
\end{equation}

\subsection{Relation to the energy momentum tensor}
\index{Lorentz force!energy momentum tensor}

It appears that to relate the energy momentum tensor to the Lorentz force we have
to work with the upper index quantities rather than the lower index stress tensor vectors.  Doing so
our four vector force per unit volume is

\begin{equation}\label{eqn:stressEnergyLorentz:280}
\begin{aligned}
\PD{V}{\dot{p}}
&= (\Bj \cdot \BE + \rho \BE + \Bj \cross \BB) \gamma_0 \\
%&= - \inv{c} \left(\gpgradezero{ F \gamma^0 J } \gamma_0 + \gpgradezero{ F \gamma^k J } \gamma_k \right) \\
&= - \inv{c} \gpgradezero{ F \gamma^\mu J } \gamma_\mu \\
&= - (\grad \cdot T(\gamma^\mu)) \gamma_\mu
\end{aligned}
\end{equation}

The term \(\gpgradezero{ F \gamma^\mu J } \gamma_\mu \) appears to be expressed simply has \(F \cdot J\) in
\citep{doran2003gap}.  Understanding that simple statement is now possible now that an exploration
of some of the underlying ideas has been made.  In retrospect having seen the bivector product form of the Lorentz force equation, it should have been
clear, but some of the associated trickiness in their treatment obscured this
fact ( Although their treatment is only two pages, I still only
understand half of what they are doing!)


\section{Expansion of the energy momentum tensor}

While all the components of the divergence of the energy momentum tensor have been expanded explicitly, this has not been
done here for the tensor itself.  A mechanical expansion of the tensor in terms of field tensor components \(F^{\mu\nu}\) has been
done previously and is not particularly enlightening.  Let us work it out here in terms of electric and magnetic field components.  In particular for the \(T^{0\mu}\) and \(T^{\mu0}\) components of the tensor in terms of energy density and the Poynting vector.

\subsection{In terms of electric and magnetic field components}

Here we want to expand

\begin{equation}\label{eqn:stressEnergyLorentz:300}
\begin{aligned}
T(\gamma^\mu) = \frac{-\epsilon_0}{2} (\BE + i c \BB) \gamma^\mu (\BE + i c \BB)
\end{aligned}
\end{equation}

It will be convenient here to temporarily work with \(\epsilon_0 = c = 1\), and put them back in afterward.
%T(\gamma^\mu) = \frac{-1}{2} (\BE + i \BB) \gamma^\mu (\BE + i \BB)

\subsubsection{First row}

First expanding \(T(\gamma^0)\) we have
\begin{equation}\label{eqn:stressEnergyLorentz:320}
\begin{aligned}
T(\gamma^0)
&= \frac{1}{2} (\BE + i \BB) (\BE - i \BB) \gamma^0 \\
&= \frac{1}{2} (\BE^2 + \BB^2 + i (\BB \BE - \BE \BB)) \gamma^0 \\
&= \frac{1}{2} (\BE^2 + \BB^2) \gamma^0 + i ( \BB \wedge \BE ) \gamma^0 \\
\end{aligned}
\end{equation}

Using the wedge product dual \(\Ba \wedge \Bb = i (\Ba \cross \Bb)\), and putting back in the units, we have our
first stress energy four vector,

\begin{equation}\label{eqn:stressEnergyLorentz:340}
\begin{aligned}
T(\gamma^0) &= \lr{ \frac{\epsilon_0}{2} \lr{\BE^2 + c^2 \BB^2} + \inv{\mu_0 c} (\BE \cross \BB) } \gamma^0
\end{aligned}
\end{equation}

In particular the energy density and the components of the Poynting vector can be picked off by dotting with each of the \(\gamma^\mu\) vectors.  That is

\begin{equation}\label{eqn:stressEnergyLorentz:360}
\begin{aligned}
U                    &= T(\gamma^0) \cdot \gamma^0 \\
\BP/c \cdot \sigma_k &= T(\gamma^0) \cdot \gamma^k
\end{aligned}
\end{equation}

\subsubsection{First column}

We have Poynting vector terms in the \(T^{0k}\) elements of the matrix.  Let us quickly verify that we have them
in the \(T^{k0}\) positions too.

To do so, again with \(c = \epsilon_0 = 1\) temporarily this is a computation of

\begin{equation}\label{eqn:stressEnergyLorentz:380}
\begin{aligned}
T(\gamma^k) \cdot \gamma^0
&= \inv{2} (T(\gamma^k) \gamma^0 + \gamma^0 T(\gamma^k)) \\
&= \frac{-1}{4} (F \gamma^k F \gamma^0 + \gamma^0 F \gamma^k F) \\
&= \frac{1}{4} (F \sigma_k \gamma_0 F \gamma^0 - \gamma^0 F \gamma_0 \sigma_k F) \\
&= \frac{1}{4} (F \sigma_k (-\BE + i \BB) - (-\BE + i\BB) \sigma_k F) \\
&= \frac{1}{4} \gpgradezero{\sigma_k (-\BE + i \BB)(\BE + i \BB) - \sigma_k (\BE + i \BB)(-\BE + i\BB) } \\
&= \frac{1}{4} \gpgradezero{\sigma_k (-\BE^2 -\BB^2 +2 (\BE \cross \BB)) - \sigma_k (-\BE^2 -\BB^2 - 2(\BE \cross \BB)) } \\
\end{aligned}
\end{equation}

Adding back in the units we have

\begin{equation}\label{eqn:stressEnergyLorentz:400}
\begin{aligned}
T(\gamma^k) \cdot \gamma^0 &= \epsilon_0 c (\BE \cross \BB) \cdot \sigma_k = \inv{c}\BP \cdot \sigma_k
\end{aligned}
\end{equation}

As expected, these are the components of the Poynting vector (scaled by \(1/c\) for units of energy density).

\subsubsection{Diagonal and remaining terms}

\begin{equation}\label{eqn:stressEnergyLorentz:420}
\begin{aligned}
T(\gamma^a) \cdot \gamma^b
&= \inv{2} (T(\gamma^a) \gamma^b + \gamma^b T(\gamma^a)) \\
&= \frac{-1}{4} (F \gamma^a F \gamma^b + \gamma^a F \gamma^b F) \\
&= \frac{1}{4} (F \sigma_a \gamma_0 F \gamma^b - \gamma^a F \gamma_0 \sigma_b F) \\
&= \frac{1}{4} (F \sigma_a (-\BE + i \BB) \sigma_b + \sigma_a (-\BE + i \BB) \sigma_b F) \\
&= \frac{1}{2} \gpgradezero{\sigma_a (-\BE + i \BB) \sigma_b (\BE + i\BB) } \\
\end{aligned}
\end{equation}

From this point is there any particularly good or clever way to do the remaining reduction?  Doing it with
coordinates looks like it would be easy, but also messy.  A decomposition of \(\BE\) and \(\BB\) that are parallel
and perpendicular to the spatial basis vectors also looks feasible.

Let us try the dumb way first

\begin{equation}\label{eqn:stressEnergyLorentz:440}
\begin{aligned}
T(\gamma^a) \cdot \gamma^b
&= \frac{1}{2} \gpgradezero{\sigma_a (-E^k \sigma_k + i B^k \sigma_k) \sigma_b (E^m \sigma_m + i B^m \sigma_m) } \\
&=
\inv{2} (B^k E^m - E^k B^m) \gpgradezero{ i \sigma_a \sigma_k \sigma_b \sigma_m }
- \inv{2} (E^k E^m + B^k B^m) \gpgradezero{ \sigma_a \sigma_k \sigma_b \sigma_m } \\
\end{aligned}
\end{equation}

Reducing the scalar operations is going to be much different for the \(a = b\), and \(a \ne b\) cases.  For the diagonal case
we have

\begin{equation}\label{eqn:stressEnergyLorentz:460}
\begin{aligned}
T(\gamma^a) \cdot \gamma^a
&=
\inv{2} (B^k E^m - E^k B^m) \gpgradezero{ i \sigma_a \sigma_k \sigma_a \sigma_m }
- \inv{2} (E^k E^m + B^k B^m) \gpgradezero{ \sigma_a \sigma_k \sigma_a \sigma_m } \\
&=
- \inv{2} \sum_{m, k \ne a} \inv{2} (B^k E^m - E^k B^m) \gpgradezero{ i \sigma_k \sigma_m }
+ \inv{2} \sum_{m, k \ne a} (E^k E^m + B^k B^m) \gpgradezero{ \sigma_k \sigma_m } \\
&+ \inv{2} \sum_{m} (B^a E^m - E^a B^m) \gpgradezero{ i \sigma_a \sigma_m }
- \inv{2} \sum_m (E^a E^m + B^a B^m) \gpgradezero{ \sigma_a \sigma_m } \\
\end{aligned}
\end{equation}

Inserting the units again we have
\begin{equation}\label{eqn:stressEnergyLorentz:480}
\begin{aligned}
T(\gamma^a) \cdot \gamma^a
&=
\frac{\epsilon_0}{2} \left( \sum_{k \ne a} \left( (E^k)^2 + c^2 (B^k)^2 \right) - \left( (E^a)^2  + c^2 (B^a)^2  \right) \right)
\end{aligned}
\end{equation}

Or, adding and subtracting, we have the diagonal in terms of energy density (minus a fudge)

\begin{equation}\label{eqn:stressEnergyLorentz:500}
\begin{aligned}
T(\gamma^a) \cdot \gamma^a &= U - \epsilon_0 \left( (E^a)^2  + c^2 (B^a)^2  \right)
\end{aligned}
\end{equation}

Now, for the off diagonal terms.  For \(a \ne b\) this is
\begin{equation}\label{eqn:stressEnergyLorentz:520}
\begin{aligned}
T(\gamma^a) \cdot \gamma^b
&=
\inv{2} \sum_m (B^a E^m - E^a B^m) \gpgradezero{ i \sigma_b \sigma_m }
+\inv{2} \sum_{m}(B^b E^m - E^b B^m) \gpgradezero{ i \sigma_a \sigma_m } \\
&- \inv{2} \sum_m (E^a E^m + B^a B^m) \gpgradezero{ \sigma_b \sigma_m }
- \inv{2} \sum_{m}(E^b E^m + B^b B^m) \gpgradezero{ \sigma_a \sigma_m } \\
&+\inv{2} \sum_{m, k \ne a,b}(B^k E^m - E^k B^m) \gpgradezero{ i \sigma_a \sigma_k \sigma_b \sigma_m }
- \inv{2} \sum_{m, k \ne a,b}(E^k E^m + B^k B^m) \gpgradezero{ \sigma_a \sigma_k \sigma_b \sigma_m } \\
\end{aligned}
\end{equation}

The first two scalar filters that include \(i\) will be zero, and we have deltas
\(\gpgradezero{ \sigma_b \sigma_m } = \delta_{bm}\) in the next two.
The remaining two terms have only vector and bivector terms, so we have zero scalar parts.
That leaves (restoring units)

\begin{equation}\label{eqn:stressEnergyLorentz:540}
\begin{aligned}
T(\gamma^a) \cdot \gamma^b
&= - \frac{\epsilon_0}{2} \left( E^a E^b + E^b E^a + c^2 (B^a B^b + B^b B^a) \right)
\end{aligned}
\end{equation}

\subsection{Summarizing}

Collecting all the results, with \(T^{\mu\nu} = T(\gamma^\mu) \cdot \gamma^\nu\), we have

\begin{equation}\label{eqn:stressEnergyLorentz:560}
\begin{aligned}
T^{00} &= \frac{\epsilon_0}{2} \left(\BE^2 + c^2 \BB^2\right) \\
T^{aa} &= \frac{\epsilon_0}{2} \left(\BE^2 + c^2 \BB^2\right) - \epsilon_0 \left( (E^a)^2  + c^2 (B^a)^2  \right) \\
T^{k0} = T^{0k} &= \inv{c} \left( \inv{\mu_0}(\BE \cross \BB) \right) \cdot \sigma_k \\
T^{ab} = T^{ba} &= - \frac{\epsilon_0}{2} \left( E^a E^b + E^b E^a + c^2 (B^a B^b + B^b B^a) \right)
\end{aligned}
\end{equation}

\subsection{Assembling a four vector}

Let us see what one of the \(T^{a\mu} \gamma_\mu\) rows of the tensor looks like in four vector form.  Let \(f \ne g \ne h\)
represent an even permutation of the integers \(1,2,3\).  Then we have

\begin{equation}\label{eqn:stressEnergyLorentz:580}
\begin{aligned}
T^f
&= T^{f\mu} \gamma_\mu \\
&=
\frac{\epsilon_0}{2} c (E^g B^h - E^h B^g) \gamma_0 \\
&+\frac{\epsilon_0}{2} \left( -(E^f)^2 +(E^g)^2 +(E^h)^2 + c^2 ( -(B^f)^2 +(B^g)^2 +(B^h)^2 ) \right) \gamma_f \\
&-\frac{\epsilon_0}{2} \left( E^f E^g + E^g E^f + c^2 (B^f B^g + B^g B^f) \right) \gamma_g \\
&-\frac{\epsilon_0}{2} \left( E^f E^h + E^h E^f + c^2 (B^f B^h + B^h B^f) \right) \gamma_h \\
\end{aligned}
\end{equation}

It is pretty amazing that the divergence of this produces the \(f\) component of the Lorentz force (density)

\begin{equation}\label{eqn:stressEnergyLorentz:600}
\begin{aligned}
\partial_\mu T^{f\mu} = (\rho \BE + \Bj \cross \BB) \cdot \sigma_f
\end{aligned}
\end{equation}

Demonstrating this directly without having STA as an available tool would be quite tedious, and looking at this
expression inspires no particular attempt to try!

%where \(U\), and \(\BP\) are as defined in \eqnref{eqn:seLorentz:energyDensityPoyntingDefined}.

% other way:
%
%To reduce this, let us write the fields in terms of projections and rejections onto the \(\sigma_k\) direction, as in \(\BE_\parallel = (\BE \cdot \sigma_k) \sigma_k\), and \(\BE_\perp = (\BE \wedge \sigma_k) \sigma_k\).  Then we have
%

\section{Conjugation?}
% from planewave.ltx
\subsection{Followup: energy momentum tensor}

This also suggests a relativistic generalization of conjugation, since the time basis vector should perhaps not have
a distinguishing role.  Something like this:

\begin{equation}\label{eqn:stressEnergyLorentz:620}
\begin{aligned}
F^{\dagger_\mu} &= \gamma_\mu \tilde{F} \gamma_\mu
\end{aligned}
\end{equation}

Or perhaps:
\begin{equation}\label{eqn:stressEnergyLorentz:640}
\begin{aligned}
F^{\dagger_\mu} &= \gamma_\mu \tilde{F} \gamma^\mu
\end{aligned}
\end{equation}

may make sense for consideration of the other components of the general energy momentum tensor, which had roughly the form:

\begin{equation}\label{eqn:stressEnergyLorentz:660}
\begin{aligned}
T^{\mu\nu} \propto T(\gamma_\mu) \cdot \gamma^\nu
\end{aligned}
\end{equation}

(with some probable adjustments to index positions).  Think this through later.

   %
% Copyright � 2012 Peeter Joot.  All Rights Reserved.
% Licenced as described in the file LICENSE under the root directory of this GIT repository.
%

%
%
\mychapter{Energy momentum tensor relation to Lorentz force}
\label{chap:PJenMtensor}
%\date{Feb 17, 2009.  enMTensor.tex}

\section{Motivation}
In \chapcite{PJstressEnergyLorentz} the energy momentum tensor was related
to the Lorentz force in STA form.  Work the same calculation strictly in
tensor form, to
%compare the difficulty of the algebra and
develop more comfort with tensor manipulation.  This should also serve
as a translation aid to compare signs due to metric tensor differences
in other reading.

\subsection{Definitions}

The energy momentum ``tensor'', really a four vector, is defined
in \citep{doran2003gap}
as

\begin{equation}\label{eqn:enMTensor:20}
\begin{aligned}
T(a) &=
\frac{\epsilon_0}{2} F a \tilde{F} = -\frac{\epsilon_0}{2} F a {F}
\end{aligned}
\end{equation}

We have seen that the divergence of the \(T(\gamma^\mu)\) vectors generate
the Lorentz force relations.

Let us expand this with respect to index lower basis vectors for use in the
divergence calculation.

\begin{equation}\label{eqn:enMTensor:40}
\begin{aligned}
T(\gamma^\mu) &=
\lr{ T(\gamma^\mu) \cdot \gamma^\nu } \gamma_\nu
\end{aligned}
\end{equation}

So we define
\begin{equation}\label{eqn:enMTensor:60}
\begin{aligned}
T^{\mu \nu}
&= T(\gamma^\mu) \cdot \gamma^\nu
\end{aligned}
\end{equation}

and can write these four vectors in tensor form as
\begin{equation}\label{eqn:enMTensor:80}
\begin{aligned}
T(\gamma^\mu) &= T^{\mu \nu} \gamma_\nu
\end{aligned}
\end{equation}

\subsection{Expanding out the tensor}

An expansion of \(T^{\mu\nu}\) was done in \chapcite{PJemstresstensor}, but looking
back that seems a peculiar way, using the four vector potential.

Let us try again in terms of \(F^{\mu\nu}\) instead.  Our field is

\begin{equation}\label{eqn:enMTensor:100}
\begin{aligned}
F
&= \inv{2} F^{\mu\nu} \gamma_{\mu} \wedge \gamma_\nu
%\\
%&= \inv{2} F_{\mu\nu} \gamma^{\mu} \wedge \gamma^\nu
\end{aligned}
\end{equation}

So our tensor components are
\begin{equation}\label{eqn:enMTensor:120}
\begin{aligned}
T^{\mu\nu}
&= T(\gamma^\mu) \cdot \gamma^\nu \\
&= -\frac{\epsilon_0}{8} F^{\lambda\sigma} F^{\alpha\beta}
\gpgradezero{ (\gamma_\lambda \wedge \gamma_\sigma) \gamma^\mu (\gamma_\alpha \wedge \gamma_\beta) \gamma^\nu } \\
\end{aligned}
\end{equation}

Or
\begin{equation}\label{eqn:enMTensor:140}
\begin{aligned}
-8{\inv{\epsilon_0}} T^{\mu\nu}
&=
F^{\lambda\sigma} F^{\alpha\beta}
\gpgradezero{
(\gamma_\lambda {\delta_\sigma}^\mu
-\gamma_\sigma {\delta_\lambda}^\mu)
(\gamma_\alpha {\delta_\beta}^\nu
-\gamma_\beta {\delta_\alpha}^\nu)
} \\
&+
F^{\lambda\sigma} F^{\alpha\beta}
\gpgradezero{ (\gamma_\lambda \wedge \gamma_\sigma \wedge \gamma^\mu) (\gamma_\alpha \wedge \gamma_\beta \wedge \gamma^\nu) } \\
\end{aligned}
\end{equation}

Expanding only the first term to start with
\begin{equation}\label{eqn:enMTensor:160}
\begin{aligned}
&
F^{\lambda\sigma} F^{\alpha\beta} (\gamma_\lambda {\delta_\sigma}^\mu) \cdot  (\gamma_\alpha {\delta_\beta}^\nu)
+F^{\lambda\sigma} F^{\alpha\beta} (\gamma_\sigma {\delta_\lambda}^\mu) \cdot (\gamma_\beta {\delta_\alpha}^\nu)  \\
&-F^{\lambda\sigma} F^{\alpha\beta} (\gamma_\lambda {\delta_\sigma}^\mu) \cdot (\gamma_\beta {\delta_\alpha}^\nu)
-F^{\lambda\sigma} F^{\alpha\beta} (\gamma_\sigma {\delta_\lambda}^\mu) \cdot (\gamma_\alpha {\delta_\beta}^\nu)  \\
&=
F^{\lambda\mu} F^{\alpha\nu} \gamma_\lambda \cdot \gamma_\alpha
+F^{\mu\sigma} F^{\nu\beta} \gamma_\sigma \cdot \gamma_\beta
-F^{\lambda\mu} F^{\nu\beta} \gamma_\lambda \cdot \gamma_\beta
-F^{\mu\sigma} F^{\alpha\nu} \gamma_\sigma \cdot \gamma_\alpha \\
&=
\eta_{\alpha\beta}
(
F^{\lambda\mu} F^{\alpha\nu} \gamma_\lambda \cdot \gamma^\beta
+
%\eta_{\alpha\beta}
F^{\mu\sigma} F^{\nu\alpha} \gamma_\sigma \cdot \gamma^\beta
-
%\eta_{\alpha\beta}
F^{\lambda\mu} F^{\nu\alpha} \gamma_\lambda \cdot \gamma^\beta
-
%\eta_{\alpha\beta}
F^{\mu\sigma} F^{\alpha\nu} \gamma_\sigma \cdot \gamma^\beta )
\\
&=
\eta_{\alpha\lambda} F^{\lambda\mu} F^{\alpha\nu} + \eta_{\alpha\sigma} F^{\mu\sigma} F^{\nu\alpha}
- \eta_{\alpha\lambda} F^{\lambda\mu} F^{\nu\alpha} - \eta_{\alpha\sigma} F^{\mu\sigma} F^{\alpha\nu}  \\
&=
2( \eta_{\alpha\lambda} F^{\lambda\mu} F^{\alpha\nu} + \eta_{\alpha\sigma} F^{\mu\sigma} F^{\nu\alpha} ) \\
&=
2( \eta_{\alpha\beta} F^{\beta\mu} F^{\alpha\nu} +
\eta_{\alpha\beta} F^{\mu\beta} F^{\nu\alpha} ) \\
&=
4 \eta_{\alpha\beta} F^{\beta\mu} F^{\alpha\nu}  \\
&= 4 F^{\beta\mu} {F_{\beta}}^{\nu} \\
&= 4 F^{\alpha\mu} {F_{\alpha}}^{\nu} \\
\end{aligned}
\end{equation}

For the second term after a shuffle of indices we have
\begin{equation}\label{eqn:enMTensor:180}
\begin{aligned}
F^{\lambda\sigma} F_{\alpha\beta}
\eta^{\mu\mu'} \gpgradezero{ (\gamma_\lambda \wedge \gamma_\sigma \wedge \gamma_\mu) (\gamma^\alpha \wedge \gamma^\beta \wedge \gamma^\nu) } \\
\end{aligned}
\end{equation}

This dot product is reducible with the identity
\begin{equation}\label{eqn:enMTensor:200}
\begin{aligned}
(a \wedge b \wedge c) \cdot (d \wedge e \wedge f) &=
(((a \wedge b \wedge c) \cdot d) \cdot e) \cdot f
\end{aligned}
\end{equation}

leaving a completely antisymmetized sum

\begin{equation}\label{eqn:enMTensor:220}
\begin{aligned}
&
F^{\lambda\sigma} F_{\alpha\beta}
\eta^{\mu\mu'}
(
{\delta_\lambda}^\nu {\delta_\sigma}^\beta {\delta_{\mu'}}^\alpha
-{\delta_\lambda}^\nu {\delta_\sigma}^\alpha {\delta_{\mu'}}^\beta
-{\delta_\lambda}^\beta {\delta_\sigma}^\nu {\delta_{\mu'}}^\alpha
+{\delta_\lambda}^\alpha {\delta_\sigma}^\nu {\delta_{\mu'}}^\beta
+{\delta_\lambda}^\beta {\delta_\sigma}^\alpha {\delta_{\mu'}}^\nu
-{\delta_\lambda}^\alpha {\delta_\sigma}^\beta {\delta_{\mu'}}^\nu
) \\
&=
  F^{\nu\beta} F_{{\mu'}\beta} \eta^{\mu\mu'}
- F^{\nu\alpha} F_{\alpha{\mu'}} \eta^{\mu\mu'}
- F^{\beta\nu} F_{{\mu'}\beta} \eta^{\mu\mu'}
+ F^{\alpha\nu} F_{\alpha{\mu'}} \eta^{\mu\mu'}
+ F^{\beta\alpha} F_{\alpha\beta} \eta^{\mu\mu'} {\delta_{\mu'}}^\nu
- F^{\alpha\beta} F_{\alpha\beta} \eta^{\mu\mu'} {\delta_{\mu'}}^\nu
 \\
&=
4 F^{\nu\alpha} F_{{\mu'}\alpha} \eta^{\mu\mu'}
+ 2 F^{\beta\alpha} F_{\alpha\beta} \eta^{\mu\mu'} {\delta_{\mu'}}^\nu
 \\
&=
4 F^{\nu\alpha} {F^{\mu}}_{\alpha}
+ 2 F^{\beta\alpha} F_{\alpha\beta} \eta^{\mu\nu}
 \\
\end{aligned}
\end{equation}

Combining these we have
\begin{equation}\label{eqn:enMTensor:240}
\begin{aligned}
T^{\mu\nu}
&=
-\frac{\epsilon_0}{8} \left(
 4 F^{\alpha\mu} {F_{\alpha}}^{\nu}
+ 4 F^{\nu\alpha} {F^{\mu}}_{\alpha}
+ 2 F^{\beta\alpha} F_{\alpha\beta} \eta^{\mu\nu}
\right) \\
&=
\frac{\epsilon_0}{8} \left(
- 4 F^{\alpha\mu} {F_{\alpha}}^{\nu}
+ 4 F^{\alpha\mu} {F^{\nu}}_{\alpha}
+ 2 F^{\alpha\beta} F_{\alpha\beta} \eta^{\mu\nu}
\right) \\
\end{aligned}
\end{equation}


If by some miracle all the index manipulation worked out, we have
\begin{equation}\label{eqn:stressEnTen:miracle}
\begin{aligned}
T^{\mu\nu} &= {\epsilon_0} \left( F^{\alpha\mu} {F^{\nu}}_{\alpha} + \inv{4} F^{\alpha\beta} F_{\alpha\beta} \eta^{\mu\nu} \right)
\end{aligned}
\end{equation}

\subsubsection{Justifying some of the steps}

For justification of some of the
index manipulations of the \(F\) tensor components it is
helpful to think back to the definitions in terms of four vector potentials

\begin{equation}\label{eqn:enMTensor:260}
\begin{aligned}
F &= \grad \wedge A \\
&= \partial^\mu A^\nu \gamma_\mu \wedge \gamma_\nu \\
&= \partial_\mu A_\nu \gamma^\mu \wedge \gamma^\nu \\
&= \partial_\mu A^\nu \gamma^\mu \wedge \gamma_\nu \\
&= \partial^\mu A_\nu \gamma_\mu \wedge \gamma^\nu \\
&= \inv{2}(\partial^\mu A^\nu -\partial^\nu A^\mu ) \gamma_\mu \wedge \gamma_\nu \\
&= \inv{2}(\partial_\mu A_\nu -\partial_\nu A_\mu ) \gamma^\mu \wedge \gamma^\nu \\
&= \inv{2}(\partial_\mu A^\nu -\partial^\nu A_\mu ) \gamma^\mu \wedge \gamma_\nu \\
&= \inv{2}(\partial^\mu A_\nu -\partial_\nu A^\mu ) \gamma_\mu \wedge \gamma^\nu
\end{aligned}
\end{equation}

So with the shorthand
\begin{equation}\label{eqn:enMTensor:280}
\begin{aligned}
F^{\mu\nu} &= \partial^\mu A^\nu -\partial^\nu A^\mu \\
F_{\mu\nu} &= \partial_\mu A_\nu -\partial_\nu A_\mu \\
{F_{\mu}}^{\nu} &= \partial_\mu A^\nu -\partial^\nu A_\mu \\
{F^{\mu}}_{\nu} &= \partial^\mu A_\nu -\partial_\nu A^\mu
\end{aligned}
\end{equation}

We have
\begin{equation}\label{eqn:enMTensor:300}
\begin{aligned}
F
&= \inv{2}F^{\mu\nu} \gamma_\mu \wedge \gamma_\nu \\
&= \inv{2}F_{\mu\nu} \gamma^\mu \wedge \gamma^\nu \\
&= \inv{2}{F_\mu}^\nu \gamma^\mu \wedge \gamma_\nu \\
&= \inv{2}{F^\mu}_\nu \gamma_\mu \wedge \gamma^\nu
\end{aligned}
\end{equation}

In particular, and perhaps not obvious without the definitions handy, the following was used above

\begin{equation}\label{eqn:enMTensor:320}
\begin{aligned}
{F^{\mu}}_{\nu} &= -{F_{\nu}}^{\mu}
\end{aligned}
\end{equation}

\subsection{The divergence}

What is our divergence in tensor form?  This would be

\begin{equation}\label{eqn:enMTensor:340}
\begin{aligned}
\grad \cdot T(\gamma^\mu)
&= (\gamma^\alpha \partial_\alpha ) \cdot (T^{\mu\nu} \gamma_\nu) \\
\end{aligned}
\end{equation}

So we have
\begin{equation}\label{eqn:enMTensor:360}
\begin{aligned}
\grad \cdot T(\gamma^\mu)
&= \partial_\nu T^{\mu\nu}
\end{aligned}
\end{equation}

Ignoring the \(\epsilon_0\) factor for now, chain rule gives us

\begin{equation}\label{eqn:enMTensor:380}
\begin{aligned}
(\partial_\nu &F^{\alpha\mu}) {F^{\nu}}_{\alpha} +
F^{\alpha\mu} (\partial_\nu {F^{\nu}}_{\alpha}) +
\inv{2} (\partial_\nu F^{\alpha\beta}) F_{\alpha\beta} \eta^{\mu\nu} \\
&=
(\partial_\nu F^{\alpha\mu}) {F^{\nu}}_{\alpha} +
{F_{\alpha}}^{\mu}
(\partial_\nu F^{\nu\alpha}) +
\inv{2} (\partial_\nu F^{\alpha\beta}) F_{\alpha\beta} \eta^{\mu\nu}
\end{aligned}
\end{equation}

Only this center term is recognizable in terms of current since we
have
\begin{equation}\label{eqn:enMTensor:400}
\begin{aligned}
\grad \cdot F &= J/\epsilon_0 c
\end{aligned}
\end{equation}

Where the LHS is
\begin{equation}\label{eqn:enMTensor:420}
\begin{aligned}
\grad \cdot F
&= \gamma^\alpha \partial_\alpha \cdot \left( \inv{2} F^{\mu\nu} \gamma_\mu \wedge \gamma_\nu \right) \\
&= \inv{2} \partial_\alpha F^{\mu\nu} ( {\delta^\alpha}_\mu \gamma_\nu -{\delta^\alpha}_\nu \gamma_\mu ) \\
&= \partial_\mu F^{\mu\nu} \gamma_\nu
\end{aligned}
\end{equation}

So we have

\begin{equation}\label{eqn:enMTensor:440}
\begin{aligned}
\partial_\mu F^{\mu\nu}
&= (J \cdot \gamma^\nu)/\epsilon_0 c \\
&= ((J^\alpha \gamma_\alpha) \cdot \gamma^\nu)/\epsilon_0 c \\
&= J^\nu/\epsilon_0 c
\end{aligned}
\end{equation}

Or
\begin{equation}\label{eqn:enMTensor:460}
\begin{aligned}
\partial_\mu F^{\mu\nu} &= J^\nu/\epsilon_0 c
\end{aligned}
\end{equation}
%\partial_\nu F^{\nu\alpha} &= J^\alpha/\epsilon_0 c

So we have

\begin{equation}\label{eqn:enMTensor:480}
\begin{aligned}
\grad \cdot T(\gamma^\mu)
&= \epsilon_0\left(
(\partial_\nu F^{\alpha\mu}) {F^{\nu}}_{\alpha} +
\inv{2} (\partial_\nu F^{\alpha\beta}) F_{\alpha\beta} \eta^{\mu\nu}
\right)
+
{F_{\alpha}}^{\mu} J^\alpha/c
\end{aligned}
\end{equation}

So, to get the expected result the remaining two derivative terms must somehow cancel.  How to reduce these?  Let us look at twice that

\begin{equation}\label{eqn:enMTensor:500}
\begin{aligned}
2 (\partial_\nu &F^{\alpha\mu}) {F^{\nu}}_{\alpha} + (\partial_\nu F^{\alpha\beta}) F_{\alpha\beta} \eta^{\mu\nu} \\
&= 2 (\partial^\nu F^{\alpha\mu}) F_{\nu\alpha} + (\partial^\mu F^{\alpha\beta}) F_{\alpha\beta} \\
&= (\partial^\nu F^{\alpha\mu}) (F_{\nu\alpha} -F_{\alpha\nu}) + (\partial^\mu F^{\alpha\beta}) F_{\alpha\beta} \\
&=
(\partial^\alpha F^{\beta\mu}) F_{\alpha\beta}
+(\partial^\beta F^{\mu\alpha}) F_{\alpha\beta}
+ (\partial^\mu F^{\alpha\beta}) F_{\alpha\beta} \\
&=
(\partial^\alpha F^{\beta\mu} +\partial^\beta F^{\mu\alpha} + \partial^\mu F^{\alpha\beta}) F_{\alpha\beta} \\
\end{aligned}
\end{equation}

Ah, there is the trivector term of Maxwell's equation hiding in there.

\begin{equation}\label{eqn:enMTensor:520}
\begin{aligned}
0
&= \grad \wedge F \\
&= \gamma_\mu \partial^\mu \wedge \left(\inv{2} F^{\alpha\beta} (\gamma_\alpha \wedge \gamma_\beta) \right) \\
&= \inv{2} (\partial^\mu F^{\alpha\beta}) (\gamma_\mu \wedge \gamma_\alpha \wedge \gamma_\beta) \\
&= \inv{3!}
\left(
\partial^\mu F^{\alpha\beta}
+\partial^\alpha F^{\beta\mu}
+\partial^\beta F^{\mu\alpha}
\right)
(\gamma_\mu \wedge \gamma_\alpha \wedge \gamma_\beta)
\end{aligned}
\end{equation}

Since this is zero, each component of this trivector must separately equal zero, and we have

\begin{equation}\label{eqn:enMTensor:540}
\begin{aligned}
\partial^\mu F^{\alpha\beta} +\partial^\alpha F^{\beta\mu} +\partial^\beta F^{\mu\alpha} = 0
\end{aligned}
\end{equation}

So, where \(T^{\mu\nu}\) is defined by \eqnref{eqn:stressEnTen:miracle}, the final result is

\begin{equation}\label{eqn:stressEnTen:covariantTensor}
\begin{aligned}
\partial_\nu T^{\mu\nu} &= F^{\alpha\mu} J_\alpha/c
\end{aligned}
\end{equation}

   %
% Copyright � 2012 Peeter Joot.  All Rights Reserved.
% Licenced as described in the file LICENSE under the root directory of this GIT repository.
%

%
%
\mychapter{DC Power consumption formula for resistive load}
\label{chap:dcPower}
%\date{Jan 06, 2009.  dcPower.tex}

\section{Motivation}

Despite a lot of recent study of electrodynamics, faced with a simple electrical problem:

``What capacity generator would be required for an arc welder on a 30 Amp breaker using a 220 volt circuit''.

I could not think of how to answer this off the top of my head.  Back in school without hesitation I would have
been able to plug into \(P = I V\) to get a capacity estimation for the generator.

Having forgotten the formula, let us see how we get that \(P = I V\) relationship from Maxwell's equations.

\section{}

Having just derived the Poynting energy momentum density relationship from Maxwell's equations, let that be the starting
point
%
\begin{equation}\label{eqn:dcPower:20}
\begin{aligned}
\frac{d}{dt}\left(\frac{\epsilon_0}{2}\left(\BE^2 + c^2\BB^2 \right) \right) = - \inv{\mu_0} \left(\BE \cross \BB \right) - \BE \cdot \Bj
\end{aligned}
\end{equation}
%
The left hand side is the energy density time variation, which is power per unit volume, so we can integrate this
over a volume to determine the power associated with a change in the field.
%
\begin{equation}\label{eqn:dcPower:40}
\begin{aligned}
P = -\int dV \left( \inv{\mu_0} \left(\BE \cross \BB \right) + \BE \cdot \Bj \right)
\end{aligned}
\end{equation}
%
As a reminder, let us write the magnetic and electric fields in terms of potentials.

In terms of the ``native'' four potential our field is
%
\begin{equation}\label{eqn:dcPower:60}
\begin{aligned}
F
&= \BE + ic \BB \\
&= \grad \wedge A \\
&= \gamma^0 \gamma_k \partial_0 A^k + \gamma^j \gamma_0 \partial_j A^0 + \gamma^m \wedge \gamma_n \partial_m A^n \\
\end{aligned}
\end{equation}
%
The electric field is
%
\begin{equation}\label{eqn:dcPower:80}
\begin{aligned}
\BE &= \sum_k (\grad \wedge A) \cdot (\gamma^0 \gamma^k) \gamma_k \gamma_0 \\
\end{aligned}
\end{equation}
%
From this, with \(\phi = A^0\), and \(\BA = \sigma_k A^k\) we have
\begin{equation}\label{eqn:dcPower:100}
\begin{aligned}
\BE &= -\inv{c} \PD{t}{\BA} - \grad \phi \\
i\BB &= \spacegrad \wedge \BA
\end{aligned}
\end{equation}
%
Now, the arc welder is (I think) a DC device, and to
get a rough idea of what it requires lets just assume that its a rectifier that outputs RMS DC.
So if we make this simplification, and assume that we have a
purely resistive load (ie: no inductance and therefore no magnetic fields) and a DC supply and constant current, then
we eliminate the vector potential terms.

This wipes out the \(\BB\) and the Poynting vector, and leaves our electric field specified in terms
of the potential difference across the load \(\BE = -\spacegrad \phi\).
%
That is
\begin{equation}\label{eqn:dcPower:120}
\begin{aligned}
P &= \int dV (\spacegrad \phi) \cdot \Bj
\end{aligned}
\end{equation}
%
Suppose we are integrating over the length of a uniformly resistive load with some fixed cross sectional area.  \(\Bj dV\) is then the magnitude of the current directed along the wire for its length.  This basically leaves us with a line integral over the length of the wire that we are measuring our potential drop over so we are left with just
%
\begin{equation}\label{eqn:dcPower:140}
\begin{aligned}
P &= (\delta \phi) I
\end{aligned}
\end{equation}
%
This \(\delta \phi\) is just our voltage drop \(V\), and this gives us the desired \(P = I V\) equation.
Now, I also recall from school
now that I think about it that \(P = I V\) also applied to inductive loads, but it required that \(I\) and \(V\) be phasors that
represented the sinusoidal currents and sources.  A good followup exercise would be to show from Maxwell's equations
that this is in fact valid.  Eventually I will know the origin of all the formulas from my old engineering courses.

   %
% Copyright � 2012 Peeter Joot.  All Rights Reserved.
% Licenced as described in the file LICENSE under the root directory of this GIT repository.
%

%
%
\mychapter{Rayleigh-Jeans Law Notes}
\label{chap:PJrayleighJeans}
\index{Rayleigh-Jeans law}
%\date{Dec 27, 2008.  rayleighJeans.tex}

\section{Motivation}

Fill in the gaps for a reading of the
initial parts of the Rayleigh-Jeans discussion of \citep{bohm1989qt}.

\section{2. Electromagnetic energy}

Energy of the field given to be:
%
\begin{equation}\label{eqn:rayleighJeans:20}
\begin{aligned}
E = \inv{8\pi} \int (\bcE^2 + \bcH^2)
\end{aligned}
\end{equation}
%
I still do not really know where this comes from.
Could perhaps justify this with a Hamiltonian of a field (although this is
uncomfortably abstract).

With the particle Hamiltonian we have
%
\begin{equation}\label{eqn:rayleighJeans:40}
\begin{aligned}
H = \qdot_i p_i -\LL
\end{aligned}
\end{equation}
%
What is the field equivalent of this?  Try to get the feel for this with some simple fields (such as the one dimensional vibrating string), and the Coulomb field.  For the physical case, do this with both the Hamiltonian approach and a physical limiting argument.

\section{3. Electromagnetic Potentials}

Bohm writes Maxwell's equations in non-SI units, and also, naturally, not in STA form which would be somewhat more natural for a gauge
discussion.
%
\begin{equation}\label{eqn:rayleighJeans:60}
\begin{aligned}
\spacegrad \cross \bcE &= -\inv{c} \partial_t \bcH \\
\spacegrad \cdot \bcE &= 4 \pi \rho \\
\spacegrad \cross \bcH &= \inv{c} \partial_t \bcE + 4 \pi \Bj \\
\spacegrad \cdot \bcH &= 0
\end{aligned}
\end{equation}
%
In STA form this is
%
\begin{equation}\label{eqn:rayleighJeans:80}
\begin{aligned}
\spacegrad \bcE &= - \partial_0 i\bcH + 4 \pi \rho \\
\spacegrad i\bcH &= -\partial_0 \bcE - 4 \pi \Bj \\
\end{aligned}
\end{equation}
%
Or
\begin{equation}\label{eqn:rayleigh_jeans:maxwellNotQuiteCovariant}
\begin{aligned}
\spacegrad (\bcE + i\bcH) + \partial_0 (\bcE + i\bcH) &= 4 \pi (\rho - \Bj)
\end{aligned}
\end{equation}
%
Left multiplying by \(\gamma_0\) gives
%
\begin{equation}\label{eqn:rayleighJeans:100}
\begin{aligned}
\gamma_0 \spacegrad
&= \gamma_0 \sum_k \sigma_k \partial_k \\
&= \gamma_0 \sum_k \gamma_k \gamma_0 \partial_k \\
&= -\sum_k \gamma_k \partial_k \\
&= \gamma^k \partial_k \\
\end{aligned}
\end{equation}
%
and
%
\begin{equation}\label{eqn:rayleighJeans:120}
\begin{aligned}
\gamma_0 \Bj
&= \sum_k \gamma_0 \sigma_k j^k \\
&= -\sum_k \gamma_k j^k,
\end{aligned}
\end{equation}
%
so with \(J^0 = \rho\), \(J^k = j^k\) and \(J = \gamma_\mu J^\mu\), we have
%
\begin{equation}\label{eqn:rayleighJeans:140}
\begin{aligned}
\gamma^\mu \partial_\mu (\bcE + i\bcH) &= 4 \pi J
\end{aligned}
\end{equation}
%
and finally with \(F = \bcE + i\bcH\), we have Maxwell's equation in covariant form
%
\begin{equation}\label{eqn:rayleighJeans:160}
\begin{aligned}
\grad F &= 4 \pi J.
\end{aligned}
\end{equation}
%
Next it is stated that general solutions can be expressed as
%
\begin{equation}\label{eqn:rayleigh_jeans:fieldsFromPotentials}
\begin{aligned}
\bcH &= \spacegrad \cross \Ba \\
\bcE &= - \inv{c} \PD{t}{\Ba} - \spacegrad \phi
\end{aligned}
\end{equation}
%
Let us double check that this jives with the bivector potential solution \(F = \grad \wedge A = \bcE + i\bcH\).  Let us split our bivector
into spacetime and spatial components by the conjugate operation
%
\begin{equation}\label{eqn:rayleighJeans:180}
\begin{aligned}
F^\conj &=\gamma_0 F \gamma_0 \\
&= \gamma_0 \gamma^\mu \wedge \gamma^\nu \partial_\mu A_\mu \gamma_0 \\
&=
\left\{
\begin{array}{l l}
0 & \quad \mbox{if \(\mu = \nu\)} \\
\gamma^\mu \gamma^\nu \partial_\mu A_\nu & \quad \mbox{if \(\mu \in \{1,2,3\}\), and \(\nu \in \{1,2,3\}\)} \\
-\gamma^\mu \gamma^\nu \partial_\mu A_\nu & \quad \mbox{one of \(\mu = 0\) or \(\nu = 0\) } \\
\end{array} \right.
\end{aligned}
\end{equation}
%
\begin{equation}\label{eqn:rayleighJeans:200}
\begin{aligned}
F
&= \bcE + i\bcH \\
&= \inv{2}(F - F^\conj) + \inv{2}(F + F^\conj) \\
&= \left(\gamma^k \wedge \gamma^0 \partial_k A_0 +\gamma^0 \wedge \gamma^k \partial_0 A_k\right) + \left(\gamma^a \wedge \gamma^b \partial_a A_b\right) \\
&= -\left(\sum_k \sigma_k \partial_k A^0 + \partial_0 \sigma_k A^k\right) + i\left(\epsilon_{abc}\sigma_a \partial_b A^c\right) \\
\end{aligned}
\end{equation}
%
So, with \(\Ba = \sigma_k A^k\), and \(\phi = A^0\), we do have equations \eqnref{eqn:rayleigh_jeans:fieldsFromPotentials} as identical to \(F = \grad \wedge A\).

Now how about the gauge variations of the fields?  Bohm writes that we can alter the potentials by
%
\begin{equation}\label{eqn:rayleigh_jeans:gauge}
\begin{aligned}
\Ba' &= \Ba - \spacegrad \psi \\
\phi' &= \phi + \inv{c}\PD{t}{\psi}
\end{aligned}
\end{equation}
%
How does this translate to an alteration of the four potential?  For the vector potential we have
%
\begin{equation}\label{eqn:rayleighJeans:220}
\begin{aligned}
\sigma_k {A^k}' &= \sigma_k A^k - \sigma_k \partial \psi \\
\gamma_k \gamma_0 {A^k}' &= \gamma_k \gamma_0 A^k - \gamma_k \gamma_0 \partial_k \psi \\
-\gamma_0 \gamma_k {A^k}' &= -\gamma_0 \gamma_k A^k - \gamma_0 \gamma^k \partial_k \psi \\
\gamma_k {A^k}' &= \gamma_k A^k + \gamma^k \partial_k \psi \\
\end{aligned}
\end{equation}
%
with \(\phi = A^0\), add in the \(\phi\) term
%
\begin{equation}\label{eqn:rayleighJeans:240}
\begin{aligned}
\gamma_0 \phi' &= \gamma_0 \phi + \gamma_0 \PD{x^0}{\psi} \\
\gamma_0 \phi' &= \gamma_0 \phi + \gamma^0 \PD{x^0}{\psi}
\end{aligned}
\end{equation}
%
For
\begin{equation}\label{eqn:rayleighJeans:260}
\begin{aligned}
\gamma_\mu {A^\mu}' &= \gamma_\mu A^\mu + \gamma^\mu \partial_\mu \psi \\
\end{aligned}
\end{equation}
%
Which is just a statement that we can add a spacetime gradient to our vector potential without altering the field equation:
%
\begin{equation}\label{eqn:rayleighJeans:280}
\begin{aligned}
A' &= A + \grad \psi
\end{aligned}
\end{equation}
%
Let us verify that this does in fact not alter Maxwell's equation.
%
\begin{equation}\label{eqn:rayleighJeans:300}
\begin{aligned}
\grad (\grad \wedge (A + \grad \psi) &= 4 \pi J
\grad (\grad \wedge A) + \grad (\grad \wedge \grad \psi) &=
\end{aligned}
\end{equation}
%
Since \(\grad \wedge \grad = 0\) we have
%
\begin{equation}\label{eqn:rayleighJeans:320}
\begin{aligned}
\grad (\grad \wedge A') = \grad (\grad \wedge A)
\end{aligned}
\end{equation}
%
Now the statement that \(\grad \wedge \grad\) as an operator equals zero, just by virtue of \(\grad\) being a vector is worth explicit
confirmation.  Let us expand that to verify
%
\begin{equation}\label{eqn:rayleighJeans:340}
\begin{aligned}
\grad \wedge \grad \psi
&= \gamma^\mu \wedge \gamma^\nu \partial_\mu \partial_\nu \psi \\
&= \left(\sum_{\mu < \nu} + \sum_{\nu < \mu}\right) \gamma^\mu \wedge \gamma^\nu \partial_\mu \partial_\nu \psi \\
&= \sum_{\mu < \nu} \gamma^\mu \wedge \gamma^\nu (\partial_\mu \partial_\nu \psi - \partial_\nu \partial_\mu \psi) \\
\end{aligned}
\end{equation}
%
So, we see that we additionally need the field variable \(\psi\) to be sufficiently continuous for mixed partial equality for the
statement that \(\grad \wedge \grad = 0\) to be valid.  Assuming that continuity is taken as a given the confirmation of the invariance under this transformation is thus complete.

Now, Bohm says it is possible to pick \(\spacegrad \cdot \Ba' = 0\).  From \eqnref{eqn:rayleigh_jeans:gauge} that implies
%
\begin{equation}\label{eqn:rayleighJeans:360}
\begin{aligned}
\spacegrad \cdot \Ba'
&= \spacegrad \cdot \Ba - \spacegrad \cdot \spacegrad \psi \\
&= \spacegrad \cdot \Ba - \spacegrad^2 \psi = 0 \\
\end{aligned}
\end{equation}
%
So, provided we can find a solution to the Poisson equation
%
\begin{equation}\label{eqn:rayleigh_jeans:psiForDivAPrimeEqZero}
\begin{aligned}
\spacegrad^2 \psi = \spacegrad \cdot \Ba
\end{aligned}
\end{equation}
%
one can find a \(\psi, \Ba\) gauge transformation that has the particular quality that \(\spacegrad \cdot \Ba' = 0\).

That solution, from \eqnref{eqn:rayleigh_jeans:laplacianOfPoisson} is
%
\begin{equation}\label{eqn:rayleighJeans:380}
\begin{aligned}
\psi(\Br) = -\inv{4\pi}\int (\spacegrad' \cdot \Ba(\Br')) dV' \inv{\Abs{\Br-\Br'}}
\end{aligned}
\end{equation}
%
The corollary to this
is that one may similarly impose a requirement that \(\spacegrad \cdot \Ba = 0\), since if that is not the case, some \(\Ba'\) can be added to the vector potential to make that the case.

FIXME: handwaving description here.  Show with a math statement with \(\Ba \rightarrow \Ba'\).

\subsection{Free space solutions}

From \eqnref{eqn:rayleigh_jeans:maxwellNotQuiteCovariant} and \eqnref{eqn:rayleigh_jeans:fieldsFromPotentials}
the free space solution to Maxwell's
equation must satisfy
%
\begin{equation}\label{eqn:rayleighJeans:400}
\begin{aligned}
0
&= \left(\spacegrad + \partial_0\right) (\bcE + i\bcH) \\
&= \left(\spacegrad + \partial_0\right) \left(- \partial_0{\Ba} - \spacegrad \phi + \spacegrad \wedge \Ba \right) \\
&= - \spacegrad \partial_0{\Ba} - \spacegrad^2 \phi + \spacegrad (\spacegrad \wedge \Ba)
 - \partial_{00}{\Ba} - \partial_0 \spacegrad \phi + \partial_0 (\spacegrad \wedge \Ba) \\
&= - \spacegrad \cdot \partial_0{\Ba} - \spacegrad^2 \phi + \spacegrad \cdot (\spacegrad \wedge \Ba)
 - \partial_{00}{\Ba} - \partial_0 \spacegrad \phi  \\
\end{aligned}
\end{equation}
%
Since the scalar and vector parts of this equation must separately equal zero we have
%
\begin{equation}\label{eqn:rayleighJeans:420}
\begin{aligned}
0 &= - \partial_0 \spacegrad \cdot {\Ba} - \spacegrad^2 \phi \\
0 &= \spacegrad \cdot (\spacegrad \wedge \Ba) - \partial_{00}{\Ba} - \partial_0 \spacegrad \phi  \\
\end{aligned}
\end{equation}
%
If one picks a gauge transformation such that \(\spacegrad \cdot \Ba = 0\) we then have
\begin{equation}\label{eqn:rayleighJeans:440}
\begin{aligned}
0 &= \spacegrad^2 \phi \\
0 &= \spacegrad^2 \Ba - \partial_{00}{\Ba} - \partial_0 \spacegrad \phi  \\
\end{aligned}
\end{equation}
%
For the first Bohm argues that ``It is well known that the only solution of this equation that is regular over all space is \(\phi = 0\)'', and anything else implies charge in the region.  What does regular mean here?  I suppose this seems like a reasonable enough statement, but I think the proper way to think about this is really that one has picked the covariant gauge \(\grad \cdot A = 0\) (that is simpler anyhow).  With an acceptance of the \(\phi =0\) argument one is left with the vector potential wave equation which was the desired goal of that section.

Note: The following 
\href{https://www.physicsforums.com/threads/electrodynamic-vector-potential-wave-equations-in-free-space.281874/}{physicsforums thread} discusses some of the confusion I had in this bit of text.

\subsection{Doing this all directly}

Now, the whole point of the gauge transformation appears to be to show that one can find the four wave equation solutions for
Maxwell's equation by picking a specific gauge.  This is actually trivial to do from the STA Maxwell equation:
%
\begin{equation}\label{eqn:rayleighJeans:460}
\begin{aligned}
\grad (\grad \wedge A) = \grad( \grad A - \grad \cdot A ) = \grad^2 A - \grad (\grad \cdot A) = 4 \pi J
\end{aligned}
\end{equation}
%
So, if one picks a gauge transformation with \(\grad \cdot A = 0\), one has
%
\begin{equation}\label{eqn:rayleighJeans:480}
\begin{aligned}
\grad^2 A = 4 \pi J
\end{aligned}
\end{equation}
%
This is precisely the four wave equations desired
\begin{equation}\label{eqn:rayleighJeans:500}
\begin{aligned}
\partial_\nu\partial^\nu A^\mu = 4 \pi J^\mu
\end{aligned}
\end{equation}
%
FIXME: show the precise gauge transformation \(A \rightarrow A'\) that leads to \(\grad \cdot A = 0\).

\section{Energy density.  Get the units right with these CGS equations}

We will want to calculate the equivalent of
%
\begin{equation}\label{eqn:rayleighJeans:520}
\begin{aligned}
U = \frac{\epsilon_0}{2} (\BE^2 + c^2 \BB^2)
\end{aligned}
\end{equation}
%
but are faced with the alternate units of Bohm's text.  Let us repeat the
derivation of the electric field energy from \chapcite{PJelectricFieldEnergy}
in the CGS units directly from Maxwell's equation
%
\begin{equation}\label{eqn:rayleigh_jeans:maxwell}
\begin{aligned}
F &= \bcE + i\bcH \\
J &= (\rho + \Bj) \gamma_0 \\
\grad F &= 4 \pi J
\end{aligned}
\end{equation}
%
to ensure we get it right.

To start with we our spacetime split of \eqnref{eqn:rayleigh_jeans:maxwell} is
%
\begin{equation}\label{eqn:rayleighJeans:540}
\begin{aligned}
( \partial_0 + \spacegrad ) (\bcE + \bcH) = 4 \pi (\rho - \Bj)
\end{aligned}
\end{equation}
%
The scalar part gives us Coulomb's law
%
\begin{equation}\label{eqn:rayleighJeans:560}
\begin{aligned}
\spacegrad \cdot \bcE = 4 \pi \rho
\end{aligned}
\end{equation}
%
Gauss's theorem applied to a spherical constant density charge distribution
gives us
\begin{equation}\label{eqn:rayleighJeans:580}
\begin{aligned}
\int \spacegrad \cdot \bcE dV &= 4 \pi \int \rho dV \\
\implies \\
\int {\bcE} \cdot \ncap dA &= 4 \pi Q \\
\implies \\
\Abs{\bcE} 4 \pi r^2 &= 4 \pi Q \\
\end{aligned}
\end{equation}
%
so we have the expected ``unitless'' Coulomb law force equation
%
\begin{equation}\label{eqn:rayleighJeans:600}
\begin{aligned}
{\BF} = q\bcE = \frac{q Q }{r^2} \rcap
\end{aligned}
\end{equation}
%
So far so good.  Next introduction of a potential.  For statics we do not care
about the four vectors and stick with the old fashion definition of the potential \(\phi\) indirectly in terms of \(\bcE\).  That is
%
\begin{equation}\label{eqn:rayleighJeans:620}
\begin{aligned}
\bcE = -\spacegrad \phi
\end{aligned}
\end{equation}
%
A line integral of this gives us \(\phi\) in terms of \(\bcE\)
\begin{equation}\label{eqn:rayleighJeans:640}
\begin{aligned}
-\int \bcE \cdot \Br
&= \int \spacegrad \phi \cdot d\Br \\
&= \phi - \phi_0 \\
\end{aligned}
\end{equation}
%
With \(\phi(\infty) = 0\) this is
%
\begin{equation}\label{eqn:rayleighJeans:660}
\begin{aligned}
\phi(d)
&= -\int_{r=\infty}^d \bcE \cdot d\Br  \\
&= -\int_{r=\infty}^d \frac{Q}{r^2} \rcap \cdot d\Br  \\
&= -\int_{r=\infty}^d \frac{Q}{r^2} dr  \\
&= \frac{Q}{d} \\
%&= -\int \frac{\rho dV}{r^2} dr
\end{aligned}
\end{equation}
%
Okay.  Now onto the electrostatic energy.  The work done to move one charge from infinite to some separation \(d\) of another like sign charge
is
%
\begin{equation}\label{eqn:rayleighJeans:680}
\begin{aligned}
\int_{r=\infty}^{d} F \cdot d\Br
&= \int_{r= \infty}^d \frac{q Q}{r^2} \rcap \cdot (-d\Br)  \\
&= -\int_{r= \infty}^d \frac{qQ}{r^2} dr  \\
&= \frac{qQ}{d} \\
&= q_1 \phi_2(d) \\
\end{aligned}
\end{equation}
%
%
For a distribution of discrete charges we have to sum over all pairs
%
\begin{equation}\label{eqn:rayleighJeans:700}
\begin{aligned}
W
&= \sum_{i \ne j} \frac{q_i q_j}{d_{ij}} \\
&= \sum_{i,j} \inv{2} \frac{q_i q_j}{d_{ij}} \\
\end{aligned}
\end{equation}
%
In a similar fashion we can do a continuous variation, employing a double summation over all space.  Note first
that we can also write
one of the charge densities in terms of the potential
%
\begin{equation}\label{eqn:rayleighJeans:720}
\begin{aligned}
\bcE &= - \spacegrad \phi \\
\implies \\
\spacegrad \cdot \bcE
&= - \spacegrad \cdot \spacegrad \phi \\
&= - \spacegrad^2 \phi \\
&= 4 \pi \rho
\end{aligned}
\end{equation}
%
\begin{equation}\label{eqn:rayleighJeans:740}
\begin{aligned}
W
&= \inv{2} \int \rho \phi(r) dV \\
&= -\inv{8\pi} \int \phi \spacegrad^2 \phi dV \\
&= \inv{8\pi} \int ( (\spacegrad \phi)^2 - \spacegrad \cdot (\phi \spacegrad \phi)) dV \\
&= \inv{8\pi} \int (-\bcE)^2 - \inv{8\pi} \int (\phi \spacegrad \phi) \cdot \ncap dA
\end{aligned}
\end{equation}
%
Here the one and two subscripts could be dropped with a switch to the total charge density and the potential from this complete charge superposition.
For our final result we have an energy density of
%
\begin{equation}\label{eqn:rayleighJeans:760}
\begin{aligned}
\frac{dW}{dV} &= \inv{8\pi} {\bcE}^2
\end{aligned}
\end{equation}
%
\section{Auxiliary details}

\subsection{Confirm Poisson solution to Laplacian}
\index{Laplacian!Poisson solution}

Bohm lists the solution for \eqnref{eqn:rayleigh_jeans:psiForDivAPrimeEqZero} (a Poisson integral), but I forget how one shows this.  I can not figure out how to integrate this Laplacian, but it is simple enough to confirm this by back substitution.

Suppose one has
%
\begin{equation}\label{eqn:rayleighJeans:780}
\begin{aligned}
\psi = \int \frac{\rho(\Br')}{\Abs{\Br - \Br'}} dV'
\end{aligned}
\end{equation}
%
We can take the Laplacian by direct differentiation under the integration sign
%
\begin{equation}\label{eqn:rayleighJeans:800}
\begin{aligned}
\spacegrad^2 \psi = \int {\rho(\Br')} dV' \spacegrad^2 \inv{\Abs{\Br - \Br'}}
\end{aligned}
\end{equation}
%
To evaluate the Laplacian we need
%
\begin{equation}\label{eqn:rayleighJeans:820}
\begin{aligned}
\PD{x_i}{\Abs{\Br - \Br'}^k}
&= \PD{x_i}{} \left(\sum_j (x_j - x_j')^2 \right)^{k/2} \\
&= k { 2\Abs{\Br - \Br'}^{k-2}} \PD{x_i}{} \left(\sum_j (x_j - x_j')^2 \right) \\
%&= k { 2\Abs{\Br - \Br'}^{k-2}} 2 (x_i - x_i')
&= k { \Abs{\Br - \Br'}^{k-2}} (x_i - x_i')
\end{aligned}
\end{equation}
%
So we have
\begin{equation}\label{eqn:rayleighJeans:840}
\begin{aligned}
\PD{x_i}{} \PD{x_i}{} {\Abs{\Br - \Br'}^{-1}}
&=
- (x_i - x_i') \PD{x_i}{}{ \inv{ \Abs{\Br - \Br'}^3} }
- \inv{ \Abs{\Br - \Br'}^3} \PD{x_i}{ (x_i - x_i') } \\
&=
3 (x_i - x_i')^2 { \inv{ \Abs{\Br - \Br'}^5} }
- \inv{ \Abs{\Br - \Br'}^3} \\
\end{aligned}
\end{equation}
%
So, provided \(\Br \ne \Br'\) we have
%
\begin{equation}\label{eqn:rayleighJeans:860}
\begin{aligned}
\spacegrad^2 \psi &=
3 (\Br - \Br')^2 \inv{ \Abs{\Br - \Br'}^5}
- 3 \inv{ \Abs{\Br - \Br'}^3} \\
&= 0
\end{aligned}
\end{equation}
%
Observe that this is true only for \R{3}.  Now, one is left with only an integral around a neighborhood around the point \(\Br\) which can be made small enough that \(\rho(\Br') = \rho(\Br)\) in that volume can be taken as constant.
%
\begin{equation}\label{eqn:rayleighJeans:880}
\begin{aligned}
\spacegrad^2 \psi
&= \rho(\Br) \int dV' \spacegrad^2 \inv{\Abs{\Br - \Br'}} \\
&= \rho(\Br) \int dV' \spacegrad \cdot \spacegrad \inv{\Abs{\Br - \Br'}} \\
&= -\rho(\Br) \int dV' \spacegrad \cdot \frac{(\Br -\Br')}{\Abs{\Br - \Br'}^3} \\
\end{aligned}
\end{equation}
%
Now, if the divergence in this integral was with respect to the primed variable that ranges over the infinitesimal volume, then this could be converted to a surface integral.
Observe that a radial expansion of this divergence allows for convenient change of variables to the primed \(x_i'\) coordinates
%
\begin{equation}\label{eqn:rayleighJeans:900}
\begin{aligned}
\spacegrad \cdot \frac{(\Br -\Br')}{\Abs{\Br - \Br'}^3}
&=
\left(\frac{\Br - \Br'}{\Abs{\Br-\Br'}} \PD{\Abs{\Br-\Br'}}{}\right) \cdot
\left(\frac{\Br - \Br'}{\Abs{\Br-\Br'}} \inv{\Abs{\Br-\Br'}^2}\right) \\
&=
\PD{\Abs{\Br'-\Br}}{} {\Abs{\Br'-\Br}^{-2}} \\
&=
\left(\frac{\Br' - \Br}{\Abs{\Br'-\Br}} \PD{\Abs{\Br'-\Br}}{}\right) \cdot
\left(\frac{\Br' - \Br}{\Abs{\Br'-\Br}} \inv{\Abs{\Br'-\Br}^2}\right) \\
&= \spacegrad' \cdot \frac{(\Br'-\Br)}{\Abs{\Br' - \Br}^3}
\end{aligned}
\end{equation}
%
Now, since \(\Br'-\Br\) is in the direction of the outwards normal the divergence theorem can be used
%
\begin{equation}\label{eqn:rayleighJeans:920}
\begin{aligned}
\spacegrad^2 \psi
&= -\rho(\Br) \int dV' \spacegrad' \cdot \frac{(\Br' -\Br)}{\Abs{\Br' - \Br}^3} \\
&= -\rho(\Br) \int_{\partial V'} dA' \inv{\Abs{\Br' - \Br}^2} \\
\end{aligned}
\end{equation}
%
Picking a spherical integration volume, for which the radius is constant \(R = \Abs{\Br'-\Br}\), we have
%
\begin{equation}\label{eqn:rayleighJeans:940}
\begin{aligned}
\spacegrad^2 \psi
&= -\rho(\Br) 4 \pi R^2 \inv{R^2} \\
\end{aligned}
\end{equation}
%
In summary this is
%
\begin{equation}\label{eqn:rayleigh_jeans:laplacianOfPoisson}
\begin{aligned}
\psi &= \int \frac{\rho(\Br')}{\Abs{\Br - \Br'}} dV' \\
\spacegrad^2 \psi &= - 4 \pi \rho(\Br)
\end{aligned}
\end{equation}
%
Having written this out I recall that the same approach was used in
\citep{schwartz1987pe} (there it was to calculate \(\spacegrad \cdot \BE\) in terms of the charge density, but the ideas are all the same.)

   %
% Copyright � 2012 Peeter Joot.  All Rights Reserved.
% Licenced as described in the file LICENSE under the root directory of this GIT repository.
%

%
%
%\input{../peeter_prologue.tex}

\mychapter{Energy and momentum for Complex electric and magnetic field phasors}
\index{phasor!energy}
\index{phasor!momentum}
\label{chap:complexFieldEnergy}

%\blogpage{http://sites.google.com/site/peeterjoot/math2009/complexFieldEnergy.pdf}
%\date{Dec 13, 2009}
%\revisionInfo{complexFieldEnergy.tex}

\beginArtWithToc
%\beginArtNoToc

\section{Motivation}

In \citep{jackson1975cew} a complex phasor representations of the electric and magnetic fields is used
%
%\begin{subequations}
\begin{equation}\label{eqn:complexFieldEnergy:1}
\begin{aligned}
\BE &= \bcE e^{-i\omega t} \\
\BB &= \bcB e^{-i\omega t}.
\end{aligned}
\end{equation}
%\end{subequations}
%
Here the vectors \(\bcE\) and \(\bcB\) are allowed to take on complex values.  Jackson uses the real part of these complex vectors as the true fields, so one is really interested in just these quantities
%
%\begin{subequations}
\begin{equation}\label{eqn:complexFieldEnergy:2}
\begin{aligned}
\Real \BE &= \bcE_r \cos(\omega t) + \bcE_i \sin(\omega t) \\
\Real \BB &= \bcB_r \cos(\omega t) + \bcB_i \sin(\omega t),
\end{aligned}
\end{equation}
%\end{subequations}
%
but carry the whole thing in manipulations to make things simpler.  It is stated that the energy for such complex vector fields takes the form (ignoring constant scaling factors and units)
%
\begin{equation}\label{eqn:complexFieldEnergy:3}
\begin{aligned}
\text{Energy} \propto \BE \cdot \conjugateStar{\BE} + \BB \cdot \conjugateStar{\BB}.
\end{aligned}
\end{equation}
%
In some ways this is an obvious generalization.  Less obvious is how this and the Poynting vector are related in their corresponding conservation relationships.

Here I explore this, employing a Geometric Algebra representation of the energy momentum tensor based on the real field representation found in \citep{doran2003gap}.  Given the complex valued fields and a requirement that both the real and imaginary parts of the field satisfy Maxwell's equation, it should be possible to derive the conservation relationship between the energy density and Poynting vector from first principles.

\section{Review of GA formalism for real fields}

In SI units the Geometric algebra form of Maxwell's equation is
%
\begin{equation}
\label{eqn:complexFieldEnergy:4}
\grad F = J/\epsilon_0 c,
\end{equation}
%
where one has for the symbols
%
%\begin{subequations}
\begin{equation}
\label{eqn:complexFieldEnergy:5}
\begin{aligned}
F &= \BE + c I \BB \\
I &= \gamma_0 \gamma_1 \gamma_2 \gamma_3 \\
\BE &= E^k \gamma_k \gamma_0  \\
\BB &= B^k \gamma_k \gamma_0  \\
(\gamma^0)^2 &= -(\gamma^k)^2 = 1 \\
\gamma^\mu \cdot \gamma_\nu &= {\delta^\mu}_\nu \\
J &= c \rho \gamma_0 + J^k \gamma_k \\
\grad &= \gamma^\mu \partial_\mu = \gamma^\mu \PDi{x^\mu}{}.
\end{aligned}
\end{equation}
%\end{subequations}
%
The symmetric electrodynamic energy momentum tensor for real fields \(\BE\) and \(\BB\) is
%
\begin{equation}\label{eqn:complexFieldEnergy:6}
T(a) = \frac{-\epsilon_0}{2} F a F = \frac{\epsilon_0}{2} F a \tilde{F}.
\end{equation}
%
It may not be obvious that this is in fact a four vector, but this can be seen since it can only have grade one and three components, and also equals its reverse implying that the grade three terms are all zero.  To illustrate this explicitly consider the components of \(T^{\mu 0}\)
%
\begin{equation}\label{eqn:complexFieldEnergy:90}
\begin{aligned}
\frac{2}{\epsilon_0} T\left(\gamma^0\right)
&= -\left(\BE + c I \BB\right) \gamma^0 \left(\BE + c I \BB\right) \\
&= \left(\BE + c I \BB\right) \left(\BE - c I \BB\right) \gamma^0 \\
&= \left(\BE^2 + c^2 \BB^2 + c I \left(\BB \BE - \BE \BB\right)\right) \gamma^0 \\
&= \left(\BE^2 + c^2 \BB^2\right) \gamma^0 + 2 c I \left( \BB \wedge \BE \right) \gamma^0 \\
&= \left(\BE^2 + c^2 \BB^2\right) \gamma^0 + 2 c \left( \BE \cross \BB \right) \gamma^0 \\
\end{aligned}
\end{equation}
%
Our result is a four vector in the Dirac basis as expected

%\begin{subequations}
%\label{eqn:complexFieldEnergy:7}
\begin{equation}\label{eqn:complexFieldEnergy:110}
\begin{aligned}
T\left(\gamma^0\right) &= T^{\mu 0} \gamma_\mu \\
T^{0 0} &= \frac{\epsilon_0}{2} \left(\BE^2 + c^2 \BB^2\right) \\
T^{k 0} &= c \epsilon_0 \left(\BE \cross \BB\right)_k
\end{aligned}
\end{equation}
%\end{subequations}
%
Similar expansions are possible for the general tensor components \(T^{\mu\nu}\) but lets defer this more general expansion until considering complex valued fields.  The main point here is to remind oneself how to express the energy momentum tensor in a fashion that is natural in a GA context.  We also know that one has a conservation relationship associated with the divergence of this tensor \(\grad \cdot T(a)\) (ie. \(\partial_\mu T^{\mu\nu}\)), and want to rederive this relationship after guessing what form the GA expression for the energy momentum tensor takes when one allows the field vectors to take complex values.

\section{Computing the conservation relationship for complex field vectors}

As in \eqnref{eqn:complexFieldEnergy:3}, if one wants
%
\begin{equation}\label{eqn:complexFieldEnergy:8}
\begin{aligned}
T^{0 0} \propto \BE \cdot \conjugateStar{\BE} + c^2 \BB \cdot \conjugateStar{\BB},
\end{aligned}
\end{equation}
%
it is reasonable to assume that our energy momentum tensor will take the form
%
\begin{equation}\label{eqn:complexFieldEnergy:9}
\begin{aligned}
T(a) &=
\frac{\epsilon_0}{4} \left( \conjugateStar{F} a \tilde{F} + \tilde{F} a \conjugateStar{F} \right)
= \frac{\epsilon_0}{2} \Real \left( \conjugateStar{F} a \tilde{F} \right)
\end{aligned}
\end{equation}
%
For real vector fields this reduces to the previous results and should produce the desired mix of real and imaginary dot products for the energy density term of the tensor.  This is also a real four vector even when the field is complex, so the energy density and power density terms will all be real valued, which seems desirable.

\subsection{Expanding the tensor.  Easy parts}

As with real fields expansion of \(T(a)\) in terms of \(\BE\) and \(\BB\) is simplest for \(a = \gamma^0\).  Let us start with that.
%
\begin{equation}\label{eqn:complexFieldEnergy:130}
\begin{aligned}
\frac{4}{\epsilon_0} T(\gamma^0) \gamma_0
&=
-(\conjugateStar{\BE} + c I \conjugateStar{\BB} )\gamma^0 (\BE + c I \BB) \gamma_0
-(\BE + c I \BB )\gamma^0 (\conjugateStar{\BE} + c I \conjugateStar{\BB} ) \gamma_0 \\
&=
(\conjugateStar{\BE} + c I \conjugateStar{\BB} ) (\BE - c I \BB)
+(\BE + c I \BB ) (\conjugateStar{\BE} - c I \conjugateStar{\BB} ) \\
&=
\conjugateStar{\BE} \BE + \BE \conjugateStar{\BE}
+ c^2 (\conjugateStar{\BB} \BB + \BB \conjugateStar{\BB} )
+ c I ( \conjugateStar{\BB} \BE - \conjugateStar{\BE} \BB + \BB \conjugateStar{\BE} - \BE \conjugateStar{\BB} ) \\
&=
2 \BE \cdot \conjugateStar{\BE} + 2 c^2 \BB \cdot \conjugateStar{\BB}
+ 2 c ( \BE \cross \conjugateStar{\BB} + \conjugateStar{\BE} \cross \BB ).
\end{aligned}
\end{equation}
%
This gives
%
\begin{equation}\label{eqn:complexFieldEnergy:20}
\begin{aligned}
T(\gamma^0)
&=
\frac{\epsilon_0}{2} \left( \BE \cdot \conjugateStar{\BE} + c^2 \BB \cdot \conjugateStar{\BB} \right) \gamma^0
+ \frac{\epsilon_0 c}{2} ( \BE \cross \conjugateStar{\BB} + \conjugateStar{\BE} \cross \BB ) \gamma^0
\end{aligned}
\end{equation}
%
The sum of \(\conjugateStar{F} a F\) and its conjugate has produced the desired energy density expression.  An implication of this is that one can form and take real parts of a complex Poynting vector \(\BS \propto \BE \cross \conjugateStar{\BB}\) to calculate the momentum density.  This is stated but not demonstrated in Jackson, perhaps considered too obvious or messy to derive.

Observe that the a choice to work with complex valued vector fields gives a nice consistency, and one has the same factor of \(1/2\) in both the energy and momentum terms.  While the energy term is obviously real, the momentum terms can be written in an explicitly real notation as well since one has a quantity plus its conjugate.  Using a more conventional four vector notation (omitting the explicit Dirac basis vectors), one can write this out as a strictly real quantity.
%
\begin{equation}\label{eqn:complexFieldEnergy:21}
T(\gamma^0)
=
\epsilon_0
\left( \inv{2}\left(\BE \cdot \conjugateStar{\BE} + c^2 \BB \cdot \conjugateStar{\BB}\right),
c \Real( \BE \cross \conjugateStar{\BB} ) \right)
\end{equation}
%
Observe that when the vector fields are restricted to real quantities, the conjugate and real part operators can be dropped and the real vector field result \eqnref{eqn:complexFieldEnergy:7} is recovered.

\subsection{Expanding the tensor.  Messier parts}

I intended here to compute \(T(\gamma^k)\), and my starting point was a decomposition of the field vectors into components that anticommute or commute with \(\gamma^k\)
%
%\begin{subequations}
\begin{equation}\label{eqn:complexFieldEnergy:22}
\begin{aligned}
\BE &= \BE_\parallel + \BE_\perp \\
\BB &= \BB_\parallel + \BB_\perp.
\end{aligned}
\end{equation}
%\end{subequations}
%
The components parallel to the spatial vector \(\sigma_k = \gamma_k \gamma_0\) are anticommuting \(\gamma^k \BE_\parallel = -\BE_\parallel \gamma^k\), whereas the perpendicular components commute \(\gamma^k \BE_\perp = \BE_\perp \gamma^k\).  The expansion of the tensor products is then
%
\begin{equation}\label{eqn:complexFieldEnergy:150}
\begin{aligned}
(\conjugateStar{F} \gamma^k \tilde{F} + \tilde{F} \gamma^k \conjugateStar{F}) \gamma_k
&=
- (\conjugateStar{\BE} + I c \conjugateStar{\BB}) \gamma^k ( \BE_\parallel + \BE_\perp + c I ( \BB_\parallel + \BB_\perp ) ) \gamma_k \\
&- (\BE + I c \BB) \gamma^k ( {\BE_\parallel}^\conj + {\BE_\perp}^\conj + c I ( {\BB_\parallel}^\conj + {\BB_\perp}^\conj ) ) \gamma_k \\
&=
 (\conjugateStar{\BE} + I c \conjugateStar{\BB}) ( \BE_\parallel - \BE_\perp + c I ( -\BB_\parallel + \BB_\perp ) ) \\
&+ (\BE + I c \BB) ( {\BE_\parallel}^\conj - {\BE_\perp}^\conj + c I ( -{\BB_\parallel}^\conj + {\BB_\perp}^\conj ) ) \\
\end{aligned}
\end{equation}
%
This is not particularly pretty to expand out.  I did attempt it, but my result looked wrong.  For the application I have in mind I do not actually need anything more than \(T^{\mu 0}\), so rather than show something wrong, I will just omit it (at least for now).
%.  For all the quadratic electric field products we have
%
%\begin{align*}
%&({\BE_\parallel}^\conj + {\BE_\perp}^\conj )( \BE_\parallel - \BE_\perp )
%+(\BE_\parallel + \BE_\perp)( {\BE_\parallel}^\conj - {\BE_\perp}^\conj )
%= \\
%&\qquad 2 \BE_\parallel \cdot {\BE_\parallel}^\conj
%- 2 \BE_\perp \cdot {\BE_\perp}^\conj
%+ 2 I ({\BE_\perp}^\conj \cross \BE_\parallel + {\BE_\perp} \cross {\BE_\parallel}^\conj)
%\end{align*}
%
%With a \(\BE \rightarrow c \BB\) substuition, this is also the form of the quadratic magnetic field products.  Omitting the \(c I\) factor, this leaves only the products
%
%\begin{align*}
%&({\BE_\parallel}^\conj + {\BE_\perp}^\conj ) ( -\BB_\parallel + \BB_\perp ) + ({\BB_\parallel}^\conj + {\BB_\perp}^\conj ) ( \BE_\parallel - \BE_\perp ) \\
%&\quad +
%(\BE_\parallel + \BE_\perp ) ( -{\BB_\parallel}^\conj + {\BB_\perp}^\conj ) + (\BB_\parallel + \BB_\perp ) ( {\BE_\parallel}^\conj - {\BE_\perp}^\conj )  \\
%&=
%2 I (
%\BB_\parallel \cross {\BE_\parallel}^\conj
%+{\BB_\parallel}^\conj \cross \BE_\parallel
%+{\BE_\perp}^\conj \cross \BB_\perp
%+\BE_\perp \cross {\BB_\perp}^\conj ) \\
%&\quad
%- 2 (
%+{\BE_\perp}^\conj \cdot \BB_\parallel
%+\BE_\perp \cdot {\BB_\parallel}^\conj
%)
%+ 2 (
%{\BE_\parallel}^\conj \cdot \BB_\perp
%+\BE_\parallel \cdot {\BB_\perp}^\conj
%)
%\end{align*}
%
%Putting all the pieces together we have
%
%\begin{align*}
%\frac{2}{\epsilon_0} T(\gamma^k) \gamma_k
%&=
%\BE_\parallel \cdot {\BE_\parallel}^\conj - \BE_\perp \cdot {\BE_\perp}^\conj
%+c^2 (\BB_\parallel \cdot {\BB_\parallel}^\conj ) - c^2 (\BB_\perp \cdot {\BB_\perp}^\conj ) \\
%&\quad + I ({\BE_\perp}^\conj \cross \BE_\parallel + {\BE_\perp} \cross {\BE_\parallel}^\conj)
%+ I c^2 ({\BB_\perp}^\conj \cross \BB_\parallel + {\BB_\perp} \cross {\BB_\parallel}^\conj) \\
%&\quad - c (
%\BB_\parallel \cross {\BE_\parallel}^\conj
%+{\BB_\parallel}^\conj \cross \BE_\parallel
%+{\BE_\perp}^\conj \cross \BB_\perp
%+\BE_\perp \cross {\BB_\perp}^\conj ) \\
%&\quad + c I (
%{\BE_\parallel}^\conj \cdot \BB_\perp
%+\BE_\parallel \cdot {\BB_\perp}^\conj
%-{\BE_\perp}^\conj \cdot \BB_\parallel
%-\BE_\perp \cdot {\BB_\parallel}^\conj ).
%\end{align*}
%
%Or
%\begin{align*}
%\frac{1}{\epsilon_0} T(\gamma^k) \gamma_k
%&=
%\inv{2} \Bigl( \BE_\parallel \cdot {\BE_\parallel}^\conj - \BE_\perp \cdot {\BE_\perp}^\conj
%+c^2 (\BB_\parallel \cdot {\BB_\parallel}^\conj ) - c^2 (\BB_\perp \cdot {\BB_\perp}^\conj ) \Bigr) \\
%&\quad + I \Real
%\Bigl(
%{\BE_\perp}^\conj \cross \BE_\parallel
%+ c^2 {\BB_\perp}^\conj \cross \BB_\parallel \Bigr) \\
%&\quad - c \Real \Bigl(
%\BB_\parallel \cross {\BE_\parallel}^\conj
%+{\BE_\perp}^\conj \cross \BB_\perp
%\Bigr) \\
%&\quad + c I \Real \Bigl(
%{\BE_\parallel}^\conj \cdot \BB_\perp
%-{\BE_\perp}^\conj \cdot \BB_\parallel
%\Bigr).
%\end{align*}
%
%FIXME: this does not look right.

\subsection{Calculating the divergence}

Working with \eqnref{eqn:complexFieldEnergy:9}, let us calculate the divergence and see what one finds for the corresponding conservation relationship.
%
\begin{equation}\label{eqn:complexFieldEnergy:170}
\begin{aligned}
\frac{4}{\epsilon_0} \grad \cdot T(a)
&=
\gpgradezero{ \grad ( \conjugateStar{F} a \tilde{F} + \tilde{F} a \conjugateStar{F} )} \\
&=
-\gpgradezero{ F \lrgrad \conjugateStar{F} a + \conjugateStar{F} \lrgrad F a } \\
&=
-\gpgradeone{ F \lrgrad \conjugateStar{F} + \conjugateStar{F} \lrgrad F } \cdot a \\
&=
-\gpgradeone{
F \rgrad \conjugateStar{F}
+F \lgrad \conjugateStar{F}
+ \conjugateStar{F} \lgrad F
+ \conjugateStar{F} \rgrad F
} \cdot a \\
&=
-\inv{\epsilon_0 c} \gpgradeone{
F \conjugateStar{J}
- J \conjugateStar{F}
- \conjugateStar{J} F
+ \conjugateStar{F} J
} \cdot a \\
&= \frac{2}{\epsilon_0 c} a \cdot (
J \cdot \conjugateStar{F}
+ \conjugateStar{J} \cdot F
) \\
&= \frac{4}{\epsilon_0 c} a \cdot \Real ( J \cdot \conjugateStar{F} ).
\end{aligned}
\end{equation}
%
We have then for the divergence
%
\begin{equation}\label{eqn:complexFieldEnergy:10}
\begin{aligned}
\grad \cdot T(a)
&= a \cdot \inv{ c } \Real \left( J \cdot \conjugateStar{F} \right).
\end{aligned}
\end{equation}
%
Lets write out \(J \cdot \conjugateStar{F}\) in the (stationary) observer frame where \(J = (c\rho + \BJ) \gamma_0\).  This is
%
\begin{equation}\label{eqn:complexFieldEnergy:190}
\begin{aligned}
J \cdot \conjugateStar{F}
&=
\gpgradeone{ (c\rho + \BJ) \gamma_0 ( \conjugateStar{\BE} + I c \conjugateStar{\BB} ) } \\
&=
- (\BJ \cdot \conjugateStar{\BE} ) \gamma_0
- c \left(
\rho \conjugateStar{\BE}
+ \BJ \cross \conjugateStar{\BB}
\right) \gamma_0
\end{aligned}
\end{equation}
%
Writing out the four divergence relationships in full one has
%
%\begin{subequations}
\begin{equation}\label{eqn:complexFieldEnergy:11}
\begin{aligned}
\grad \cdot T(\gamma^0) &= - \inv{ c } \Real( \BJ \cdot \conjugateStar{\BE} ) \\
\grad \cdot T(\gamma^k) &= - \Real \left( \rho \conjugateStar{(E^k)} + (\BJ \cross \conjugateStar{\BB})_k \right)
\end{aligned}
\end{equation}
%\end{subequations}
%
Just as in the real field case one has a nice relativistic split into energy density and force (momentum change) components, but one has to take real parts and conjugate half the terms appropriately when one has complex fields.

Combining the divergence relation for \(T(\gamma^0)\) with \eqnref{eqn:complexFieldEnergy:21} the conservation relation for this subset of the energy momentum tensor becomes
%
\begin{equation}\label{eqn:complexFieldEnergy:30}
\begin{aligned}
\inv{c} \PD{t}{}\frac{\epsilon_0}{2}(\BE \cdot \conjugateStar{\BE} + c^2 \BB \cdot \conjugateStar{\BB})
+ c \epsilon_0 \Real \spacegrad \cdot (\BE \cross \conjugateStar{\BB} )
=
- \inv{c} \Real( \BJ \cdot \conjugateStar{\BE} )
\end{aligned}
\end{equation}
%
Or
%
\begin{equation}\label{eqn:complexFieldEnergy:30b}
\begin{aligned}
\PD{t}{}\frac{\epsilon_0}{2}(\BE \cdot \conjugateStar{\BE} + c^2 \BB \cdot \conjugateStar{\BB})
+ \Real \spacegrad \cdot \inv{\mu_0} (\BE \cross \conjugateStar{\BB} )
+ \Real( \BJ \cdot \conjugateStar{\BE} )
= 0
\end{aligned}
\end{equation}
%
It is this last term that puts some meaning behind Jackson's treatment since we now know how the energy and momentum are related as a four vector quantity in this complex formalism.

While I have used geometric algebra to get to this final result, I would be interested to compare how the intermediate mess compares with the same complex field vector result obtained via traditional vector techniques.  I am sure I could try this myself, but am not interested enough to attempt it.

Instead, now that this result is obtained, proceeding on to application is now possible.  My intention is to try the vacuum electromagnetic energy density example from \citep{bohm1989qt} using complex exponential Fourier series instead of the doubled sum of sines and cosines that Bohm used.

%\EndArticle
%%\EndNoBibArticle

   %
% Copyright � 2012 Peeter Joot.  All Rights Reserved.
% Licenced as described in the file LICENSE under the root directory of this GIT repository.
%

%
%
%\input{../peeter_prologue.tex}

\mychapter{Electrodynamic field energy for vacuum}
\index{energy!electrodynamic field}
\label{chap:fourierMaxVac}

%\blogpage{http://sites.google.com/site/peeterjoot/math2009/fourierMaxVac.pdf}
%\date{Dec 16, 2009}
%\revisionInfo{fourierMaxVac.tex}

%\beginArtWithToc
\beginArtNoToc

\section{Motivation}

From \chapcite{complexFieldEnergy} how to formulate the energy momentum tensor for complex vector fields (ie. phasors) in the Geometric Algebra formalism is now understood.  To recap, for the field \(F = \BE + I c \BB\), where \(\BE\) and \(\BB\) may be complex vectors we have for Maxwell's equation
%
\begin{equation}\label{eqn:fourierMaxVac:1}
\grad F = J/\epsilon_0 c.
\end{equation}
%
This is a doubly complex representation, with the four vector pseudoscalar \(I = \gamma_0 \gamma_1 \gamma_2 \gamma_3\) acting as a non-commutatitive imaginary, as well as real and imaginary parts for the electric and magnetic field vectors.  We take the real part (not the scalar part) of any bivector solution \(F\) of Maxwell's equation as the actual solution, but allow ourself the freedom to work with the complex phasor representation when convenient.  In these phasor vectors, the imaginary \(i\), as in \(\BE = \Real(\BE) + i \Imag(\BE)\), is a commuting imaginary, commuting with all the multivector elements in the algebra.

The real valued, four vector, energy momentum tensor \(T(a)\) was found to be
%
\begin{equation}\label{eqn:fourierMaxVac:2}
T(a) = \frac{\epsilon_0}{4} \Bigl( \conjugateStar{F} a \tilde{F} + \tilde{F} a \conjugateStar{F} \Bigr) =
-\frac{\epsilon_0}{2} \Real \Bigl( \conjugateStar{F} a F \Bigr).
\end{equation}
%
To supply some context that gives meaning to this tensor the associated conservation relationship was found to be
%
\begin{equation}\label{eqn:fourierMaxVac:3}
\begin{aligned}
\grad \cdot T(a) &= a \cdot \inv{ c } \Real \left( J \cdot \conjugateStar{F} \right).
\end{aligned}
\end{equation}
%
and in particular for \(a = \gamma^0\), this four vector divergence takes the form
%
\begin{equation}\label{eqn:fourierMaxVac:4}
\begin{aligned}
\PD{t}{}\frac{\epsilon_0}{2}(\BE \cdot \conjugateStar{\BE} + c^2 \BB \cdot \conjugateStar{\BB})
+ \spacegrad \cdot \inv{\mu_0} \Real (\BE \cross \conjugateStar{\BB} )
+ \Real( \BJ \cdot \conjugateStar{\BE} )
= 0,
\end{aligned}
\end{equation}
%
relating the energy term \(T^{00} = T(\gamma^0) \cdot \gamma^0\) and the Poynting spatial vector \(T(\gamma^0) \wedge \gamma^0\) with the current density and electric field product that constitutes the energy portion of the Lorentz force density.

Let us apply this to calculating the energy associated with the field that is periodic within a rectangular prism as done by Bohm in \citep{bohm1989qt}.  We do not necessarily need the Geometric Algebra formalism for this calculation, but this will be a fun way to attempt it.

\section{Setup}

Let us assume a Fourier representation for the four vector potential \(A\) for the field \(F = \grad \wedge A\).  That is
%
\begin{equation}
\label{eqn:fourierMaxVac:5}
A = \sum_{\Bk} A_\Bk(t) e^{i \Bk \cdot \Bx},
\end{equation}
%
where summation is over all angular wave number triplets \(\Bk = 2 \pi (k_1/\lambda_1, k_2/\lambda_2, k_3/\lambda_3)\).  The Fourier coefficients \(A_\Bk = {A_\Bk}^\mu \gamma_\mu\) are allowed to be complex valued, as is the resulting four vector \(A\), and the associated bivector field \(F\).

Fourier inversion, with \(V = \lambda_1 \lambda_2 \lambda_3\), follows from
%
\begin{equation}\label{eqn:fourierMaxVac:6}
\delta_{\Bk', \Bk} =
\inv{ V }
\int_0^{\lambda_1}
\int_0^{\lambda_2}
\int_0^{\lambda_3}
e^{ i \Bk' \cdot \Bx}
e^{-i \Bk \cdot \Bx} dx^1 dx^2 dx^3,
\end{equation}
%
but only this orthogonality relationship and not the Fourier coefficients themselves
%
\begin{equation}
\label{eqn:fourierMaxVac:7}
A_\Bk =
\inv{ V }
\int_0^{\lambda_1}
\int_0^{\lambda_2}
\int_0^{\lambda_3} A(\Bx, t) e^{- i \Bk \cdot \Bx} dx^1 dx^2 dx^3,
\end{equation}
%
will be of interest here.  Evaluating the curl for this potential yields
%
\begin{equation}\label{eqn:fourierMaxVac:8}
F = \grad \wedge A
= \sum_{\Bk} \left( \inv{c} \gamma^0 \wedge \dot{A}_\Bk + \gamma^m \wedge A_\Bk \frac{2 \pi i k_m}{\lambda_m} \right) e^{i \Bk \cdot \Bx}.
\end{equation}
%
Since the four vector potential has been expressed using an explicit split into time and space components it will be natural to re express the bivector field in terms of scalar and (spatial) vector potentials, with the Fourier coefficients.  Writing \(\sigma_m = \gamma_m \gamma_0\) for the spatial basis vectors, \({A_\Bk}^0 = \phi_\Bk\), and \(\BA = A^k \sigma_k\), this is
%
\begin{equation}\label{eqn:fourierMaxVac:71}
A_\Bk = (\phi_\Bk + \BA_\Bk) \gamma_0.
\end{equation}
%
The Faraday bivector field \(F\) is then
%
\begin{equation}\label{eqn:fourierMaxVac:72}
F = \sum_\Bk \left( -\inv{c} \dot{\BA}_\Bk - i \Bk \phi_\Bk + i \Bk \wedge \BA_\Bk \right) e^{i \Bk \cdot \Bx}.
\end{equation}
%
This is now enough to express the energy momentum tensor \(T(\gamma^\mu)\)
%
\begin{equation}\label{eqn:fourierMaxVac:73}
\begin{aligned}
&T(\gamma^\mu)  \\
&= -\frac{\epsilon_0}{2} \sum_{\Bk,\Bk'}
\Real \left(
\left( -\inv{c} \conjugateStar{(\dot{\BA}_{\Bk'})} + i \Bk' \conjugateStar{\phi_{\Bk'}} - i \Bk' \wedge \conjugateStar{\BA_{\Bk'}} \right)
\gamma^\mu
\left( -\inv{c} \dot{\BA}_\Bk - i \Bk \phi_\Bk + i \Bk \wedge \BA_\Bk \right) e^{i (\Bk -\Bk') \cdot \Bx}
\right).
\end{aligned}
\end{equation}
%
It will be more convenient to work with a scalar plus bivector (spatial vector) form of this tensor, and right multiplication by \(\gamma_0\) produces such a split
%
\begin{equation}\label{eqn:fourierMaxVac:74}
T(\gamma^\mu) \gamma_0 = \gpgradezero{T(\gamma^\mu) \gamma_0} + \sigma_a \gpgradezero{ \sigma_a T(\gamma^\mu) \gamma_0 }
\end{equation}
%
The primary object of this treatment will be consideration of the \(\mu = 0\) components of the tensor, which provide a split into energy density \(T(\gamma^0) \cdot \gamma_0\), and Poynting vector (momentum density) \(T(\gamma^0) \wedge \gamma_0\).

Our first step is to integrate \autoref{eqn:fourierMaxVac:74} over the volume \(V\).  This integration and the orthogonality relationship \autoref{eqn:fourierMaxVac:6}, removes the exponentials, leaving
%
\begin{equation}\label{eqn:fourierMaxVac:262}
\begin{aligned}
\int T(\gamma^\mu) \cdot \gamma_0
&= -\frac{\epsilon_0 V}{2} \sum_{\Bk}
\Real \gpgradezero{
\left( -\inv{c} \conjugateStar{(\dot{\BA}_{\Bk})} + i \Bk \conjugateStar{\phi_{\Bk}} - i \Bk \wedge \conjugateStar{\BA_{\Bk}} \right)
\gamma^\mu
\left( -\inv{c} \dot{\BA}_\Bk - i \Bk \phi_\Bk + i \Bk \wedge \BA_\Bk \right)
\gamma_0 } \\
\int T(\gamma^\mu) \wedge \gamma_0
&= -\frac{\epsilon_0 V}{2} \sum_{\Bk}
\Real \sigma_a \gpgradezero{ \sigma_a
\left( -\inv{c} \conjugateStar{(\dot{\BA}_{\Bk})} + i \Bk \conjugateStar{\phi_{\Bk}} - i \Bk \wedge \conjugateStar{\BA_{\Bk}} \right)
\gamma^\mu
\left( -\inv{c} \dot{\BA}_\Bk - i \Bk \phi_\Bk + i \Bk \wedge \BA_\Bk \right) \gamma_0
}
\end{aligned}
\end{equation}
%
Because \(\gamma_0\) commutes with the spatial bivectors, and anticommutes with the spatial vectors, the remainder of the Dirac basis vectors in these expressions can be eliminated

\begin{subequations}
\label{eqn:fourierMaxVac:77}
\begin{align}
\int T(\gamma^0) \cdot \gamma_0
&= -\frac{\epsilon_0 V }{2} \sum_{\Bk}
\Real \gpgradezero{
\left( -\inv{c} \conjugateStar{(\dot{\BA}_{\Bk})} + i \Bk \conjugateStar{\phi_{\Bk}} - i \Bk \wedge \conjugateStar{\BA_{\Bk}} \right)
\left( \inv{c} \dot{\BA}_\Bk + i \Bk \phi_\Bk + i \Bk \wedge \BA_\Bk \right)
}
\label{eqn:fourierMaxVac:77a}
\\
\int T(\gamma^0) \wedge \gamma_0
&= -\frac{\epsilon_0 V}{2} \sum_{\Bk}
\Real \sigma_a \gpgradezero{ \sigma_a
\left( -\inv{c} \conjugateStar{(\dot{\BA}_{\Bk})} + i \Bk \conjugateStar{\phi_{\Bk}} - i \Bk \wedge \conjugateStar{\BA_{\Bk}} \right)
\left( \inv{c} \dot{\BA}_\Bk + i \Bk \phi_\Bk + i \Bk \wedge \BA_\Bk \right)
}
\label{eqn:fourierMaxVac:77b}
\\
\int T(\gamma^m) \cdot \gamma_0
&= \frac{\epsilon_0 V }{2} \sum_{\Bk}
\Real \gpgradezero{
\left( -\inv{c} \conjugateStar{(\dot{\BA}_{\Bk})} + i \Bk \conjugateStar{\phi_{\Bk}} - i \Bk \wedge \conjugateStar{\BA_{\Bk}} \right)
\sigma_m
\left( \inv{c} \dot{\BA}_\Bk + i \Bk \phi_\Bk + i \Bk \wedge \BA_\Bk \right)
}
\label{eqn:fourierMaxVac:77c}
\\
\int T(\gamma^m) \wedge \gamma_0
&= \frac{\epsilon_0 V}{2} \sum_{\Bk}
\Real \sigma_a \gpgradezero{ \sigma_a
\left( -\inv{c} \conjugateStar{(\dot{\BA}_{\Bk})} + i \Bk \conjugateStar{\phi_{\Bk}} - i \Bk \wedge \conjugateStar{\BA_{\Bk}} \right)
\sigma_m
\left( \inv{c} \dot{\BA}_\Bk + i \Bk \phi_\Bk + i \Bk \wedge \BA_\Bk \right)
}
\label{eqn:fourierMaxVac:77d}
.
\end{align}
\end{subequations}

\section{Expanding the energy momentum tensor components}

\subsection{Energy}
\index{energy}

In \autoref{eqn:fourierMaxVac:77a} only the bivector-bivector and vector-vector products produce any scalar grades.  Except for the bivector product this can be done by inspection.  For that part we utilize the identity
%
\begin{equation}\label{eqn:fourierMaxVac:90}
\gpgradezero{ (\Bk \wedge \Ba) (\Bk \wedge \Bb) }
= (\Ba \cdot \Bk) (\Bb \cdot \Bk) - \Bk^2 (\Ba \cdot \Bb).
\end{equation}
%
This leaves for the energy \(H = \int T(\gamma^0) \cdot \gamma_0\) in the volume
%
\begin{equation}\label{eqn:fourierMaxVac:80}
H =
\frac{\epsilon_0 V}{2} \sum_\Bk \left(
\inv{c^2} \Abs{\dot{\BA}_\Bk}^2
+\Bk^2 \left( \Abs{\phi_\Bk}^2 + \Abs{\BA_\Bk}^2 \right) - \Abs{\Bk \cdot \BA_\Bk}^2
+ \frac{2}{c} \Real \left( i \conjugateStar{\phi_\Bk} \cdot \dot{\BA}_\Bk \right)
\right)
\end{equation}
%
We are left with a completely real expression, and one without any explicit Geometric Algebra.  This does not look like the Harmonic oscillator Hamiltonian that was expected.  A gauge transformation to eliminate \(\phi_\Bk\) and an observation about when \(\Bk \cdot \BA_\Bk\) equals zero will give us that, but first lets get the mechanical jobs done, and reduce the products for the field momentum.

\subsection{Momentum}
\index{momentum}

Now move on to \autoref{eqn:fourierMaxVac:77b}.  For the factors other than \(\sigma_a\) only the vector-bivector products can contribute to the scalar product.  We have two such products, one of the form
%
\begin{equation}\label{eqn:fourierMaxVac:302}
\begin{aligned}
\sigma_a \gpgradezero{ \sigma_a \Ba (\Bk \wedge \Bc) }
&=
\sigma_a (\Bc \cdot \sigma_a) (\Ba \cdot \Bk) - \sigma_a (\Bk \cdot \sigma_a) (\Ba \cdot \Bc) \\
&=
\Bc (\Ba \cdot \Bk) - \Bk (\Ba \cdot \Bc),
\end{aligned}
\end{equation}
%
and the other
\begin{equation}\label{eqn:fourierMaxVac:322}
\begin{aligned}
\sigma_a \gpgradezero{ \sigma_a (\Bk \wedge \Bc) \Ba }
&=
\sigma_a (\Bk \cdot \sigma_a) (\Ba \cdot \Bc) - \sigma_a (\Bc \cdot \sigma_a) (\Ba \cdot \Bk) \\
&=
\Bk (\Ba \cdot \Bc) - \Bc (\Ba \cdot \Bk).
\end{aligned}
\end{equation}
%
The momentum \(\BP = \int T(\gamma^0) \wedge \gamma_0\) in this volume follows by computation of
%
\begin{equation}\label{eqn:fourierMaxVac:342}
\begin{aligned}
&\sigma_a \gpgradezero{ \sigma_a
\left( -\inv{c} \conjugateStar{(\dot{\BA}_{\Bk})} + i \Bk \conjugateStar{\phi_{\Bk}} - i \Bk \wedge \conjugateStar{\BA_{\Bk}} \right)
\left( \inv{c} \dot{\BA}_\Bk + i \Bk \phi_\Bk + i \Bk \wedge \BA_\Bk \right)
} \\
&=
  i \BA_\Bk \left( \left( -\inv{c} \conjugateStar{(\dot{\BA}_{\Bk})} + i \Bk \conjugateStar{\phi_{\Bk}} \right) \cdot \Bk \right)
- i \Bk \left( \left( -\inv{c} \conjugateStar{(\dot{\BA}_{\Bk})} + i \Bk \conjugateStar{\phi_{\Bk}} \right) \cdot \BA_\Bk \right)  \\
&- i \Bk \left( \left( \inv{c} \dot{\BA}_\Bk + i \Bk \phi_\Bk \right) \cdot \conjugateStar{\BA_\Bk} \right)
+ i \conjugateStar{\BA_{\Bk}} \left( \left( \inv{c} \dot{\BA}_\Bk + i \Bk \phi_\Bk \right) \cdot \Bk \right)
\end{aligned}
\end{equation}
%
All the products are paired in nice conjugates, taking real parts, and premultiplication with \(-\epsilon_0 V/2\) gives the desired result.  Observe that two of these terms cancel, and another two have no real part.  Those last are
%
\begin{equation}\label{eqn:fourierMaxVac:362}
\begin{aligned}
-\frac{\epsilon_0 V \Bk}{2 c} \Real \left( i
\conjugateStar{(\dot{\BA}_\Bk} \cdot \BA_\Bk
+
\dot{\BA}_\Bk \cdot \conjugateStar{\BA_\Bk}
\right)
&=
-\frac{\epsilon_0 V \Bk}{2 c} \Real \left( i \frac{d}{dt} \BA_\Bk \cdot \conjugateStar{\BA_\Bk} \right)
\end{aligned}
\end{equation}
%
Taking the real part of this pure imaginary \(i \Abs{\BA_\Bk}^2\) is zero, leaving just
%
\begin{equation}\label{eqn:fourierMaxVac:100}
\begin{aligned}
\BP &= \epsilon_0 V \sum_{\Bk}
\Real \left(
i \BA_\Bk \left( \inv{c} \conjugateStar{\dot{\BA}_\Bk} \cdot \Bk \right)
+
\Bk^2 \phi_\Bk \conjugateStar{ \BA_\Bk }
- \Bk \conjugateStar{\phi_\Bk} (\Bk \cdot \BA_\Bk)
\right)
\end{aligned}
\end{equation}
%
I am not sure why exactly, but I actually expected a term with \(\Abs{\BA_\Bk}^2\), quadratic in the vector potential.  Is there a mistake above?

\subsection{Gauge transformation to simplify the Hamiltonian}
\index{gauge transformation}
\index{Hamiltonian}

In \autoref{eqn:fourierMaxVac:80} something that looked like the Harmonic oscillator was expected.  On the surface this does not appear to be such a beast.  Exploitation of gauge freedom is required to make the simplification that puts things into the Harmonic oscillator form.

If we are to change our four vector potential \(A \rightarrow A + \grad \psi\), then Maxwell's equation takes the form
%
\begin{equation}\label{eqn:fourierMaxVac:30}
J/\epsilon_0 c = \grad (\grad \wedge (A + \grad \psi) = \grad (\grad \wedge A) + \grad (\mathLabelBox{\grad \wedge \grad \psi}{\(=0\)}),
\end{equation}
%
which is unchanged by the addition of the gradient to any original potential solution to the equation.  In coordinates this is a transformation of the form
%
\begin{equation}\label{eqn:fourierMaxVac:31}
A^\mu \rightarrow A^\mu + \partial_\mu \psi,
\end{equation}
%
and we can use this to force any one of the potential coordinates to zero.  For this problem, it appears that it is desirable to seek a \(\psi\) such that \(A^0 + \partial_0 \psi = 0\).  That is
%
\begin{equation}\label{eqn:fourierMaxVac:32}
\sum_\Bk \phi_\Bk(t) e^{i \Bk \cdot \Bx} + \inv{c} \partial_t \psi = 0.
\end{equation}
%
Or,
%
\begin{equation}\label{eqn:fourierMaxVac:33}
\psi(\Bx,t) = \psi(\Bx,0) -\inv{c} \sum_\Bk e^{i \Bk \cdot \Bx} \int_{\tau=0}^t \phi_\Bk(\tau).
\end{equation}
%
With such a transformation, the \(\phi_\Bk\) and \(\dot{\BA}_\Bk\) cross term in the Hamiltonian \autoref{eqn:fourierMaxVac:80} vanishes, as does the \(\phi_\Bk\) term in the four vector square of the last term, leaving just
%
\begin{equation}
\label{eqn:fourierMaxVac:17b}
H =
\frac{\epsilon_0}{c^2} V \sum_\Bk
\left(
\inv{2} \Abs{\dot{\BA}_\Bk}^2
+
\inv{2} \Bigl(
(c \Bk)^2 \Abs{\BA_\Bk}^2 + \Abs{ ( c \Bk) \cdot \BA_\Bk}^2
+ \Abs{ c \Bk \cdot \BA_\Bk}^2
\Bigr)
\right).
\end{equation}
%
Additionally, wedging \autoref{eqn:fourierMaxVac:5} with \(\gamma_0\) now does not loose any information so our potential Fourier series is reduced to just

\begin{subequations}
\label{eqn:fourierMaxVac:5b}
\begin{equation}\label{eqn:fourierMaxVac:402}
\begin{aligned}
\BA &= \sum_{\Bk} \BA_\Bk(t) e^{2 \pi i \Bk \cdot \Bx} \\
\BA_\Bk &=
\inv{ V }
\int_0^{\lambda_1}
\int_0^{\lambda_2}
\int_0^{\lambda_3} \BA(\Bx, t) e^{-i \Bk \cdot \Bx} dx^1 dx^2 dx^3.
\end{aligned}
\end{equation}
\end{subequations}
%
The desired harmonic oscillator form would be had in \autoref{eqn:fourierMaxVac:17b} if it were not for the \(\Bk \cdot \BA_\Bk\) term.  Does that vanish?  Returning to Maxwell's equation should answer that question, but first it has to be expressed in terms of the vector potential.  While \(\BA = A \wedge \gamma_0\), the lack of an \(A^0\) component means that this can be inverted as
%
\begin{equation}\label{eqn:fourierMaxVac:41}
A = \BA \gamma_0 = -\gamma_0 \BA.
\end{equation}
%
The gradient can also be factored scalar and spatial vector components
%
\begin{equation}\label{eqn:fourierMaxVac:42}
\grad = \gamma^0 ( \partial_0 + \spacegrad ) = ( \partial_0 - \spacegrad ) \gamma^0.
\end{equation}
%
So, with this \(A^0 = 0\) gauge choice the bivector field \(F\) is
%
\begin{equation}\label{eqn:fourierMaxVac:43}
F = \grad \wedge A = \inv{2} \left( \rgrad A - A \lgrad \right)
\end{equation}
%
From the left the gradient action on \(A\) is
%
\begin{equation}\label{eqn:fourierMaxVac:422}
\begin{aligned}
\rgrad A
&= ( \partial_0 - \spacegrad ) \gamma^0 (-\gamma_0 \BA) \\
&= ( -\partial_0 + \rspacegrad ) \BA,
%&= \partial_0 \BA
%+ \spacegrad \cdot \BA
%+ \spacegrad \wedge \BA
%\\
\end{aligned}
\end{equation}
%
and from the right
%
\begin{equation}\label{eqn:fourierMaxVac:442}
\begin{aligned}
A \lgrad
&=
\BA \gamma_0 \gamma^0 ( \partial_0 + \spacegrad ) \\
&=
\BA ( \partial_0 + \spacegrad ) \\
&=
\partial_0 \BA
+ \BA \lspacegrad
\end{aligned}
\end{equation}
%
Taking the difference we have
%
\begin{equation}\label{eqn:fourierMaxVac:462}
F
=
\inv{2} \Bigl( -\partial_0 \BA + \rspacegrad \BA -  \partial_0 \BA - \BA \lspacegrad \Bigr).
\end{equation}
%
Which is just
%
\begin{equation}\label{eqn:fourierMaxVac:44}
F = -\partial_0 \BA + \spacegrad \wedge \BA.
\end{equation}
%
For this vacuum case, premultiplication of Maxwell's equation by \(\gamma_0\) gives
%
\begin{equation}\label{eqn:fourierMaxVac:482}
\begin{aligned}
0
&= \gamma_0 \grad ( -\partial_0 \BA + \spacegrad \wedge \BA ) \\
&= (\partial_0 + \spacegrad)( -\partial_0 \BA + \spacegrad \wedge \BA ) \\
&= -\inv{c^2} \partial_{tt} \BA
- \partial_0 \spacegrad \cdot \BA
- \partial_0 \spacegrad \wedge \BA
+ \partial_0 ( \spacegrad \wedge \BA )
+ \mathLabelBox{\spacegrad \cdot ( \spacegrad \wedge \BA ) }{\(\spacegrad^2 \BA - \spacegrad (\spacegrad \cdot \BA)\)}
+
\mathLabelBox
[
   labelstyle={below of=m\themathLableNode, below of=m\themathLableNode}
]
{\spacegrad \wedge ( \spacegrad \wedge \BA )}{\(=0\)} \\
\end{aligned}
\end{equation}
%
The spatial bivector and trivector grades are all zero.  Equating the remaining scalar and vector components to zero separately yields a pair of equations in \(\BA\)

\begin{subequations}
\label{eqn:fourierMaxVac:45}
\begin{equation}\label{eqn:fourierMaxVac:502}
\begin{aligned}
0 &= \partial_t (\spacegrad \cdot \BA) \\
0 &= -\inv{c^2} \partial_{tt} \BA + \spacegrad^2 \BA + \spacegrad (\spacegrad \cdot \BA)
\end{aligned}
\end{equation}
\end{subequations}
%
If the divergence of the vector potential is constant we have just a wave equation.  Let us see what that divergence is with the assumed Fourier representation
%
\begin{equation}\label{eqn:fourierMaxVac:522}
\begin{aligned}
\spacegrad \cdot \BA
&=
\sum_{\Bk \ne (0,0,0)} {\BA_\Bk}^m 2 \pi i \frac{k_m}{\lambda_m} e^{i \Bk \cdot \Bx} \\
&=
i \sum_{\Bk \ne (0,0,0)} (\BA_\Bk \cdot \Bk) e^{i \Bk \cdot \Bx} \\
&=
i \sum_\Bk (\BA_\Bk \cdot \Bk) e^{i \Bk \cdot \Bx}
\end{aligned}
\end{equation}
%
Since \(\BA_\Bk = \BA_\Bk(t)\), there are two ways for \(\partial_t (\spacegrad \cdot \BA) = 0\).  For each \(\Bk\) there must be a requirement for either \(\BA_\Bk \cdot \Bk = 0\) or \(\BA_\Bk = \text{constant}\).  The constant \(\BA_\Bk\) solution to the first equation appears to represent a standing spatial wave with no time dependence.  Is that of any interest?

The more interesting seeming case is where we have some non-static time varying state.  In this case, if \(\BA_\Bk \cdot \Bk\), the second of these Maxwell's equations is just the vector potential wave equation, since the divergence is zero.  That is
%
\begin{equation}\label{eqn:fourierMaxVac:50}
0 = -\inv{c^2} \partial_{tt} \BA + \spacegrad^2 \BA
\end{equation}
%
Solving this is not really what is of interest, since the objective was just to determine if the divergence could be assumed to be zero.  This shows then, that if the transverse solution to Maxwell's equation is picked, the Hamiltonian for this field, with this gauge choice, becomes
%
\begin{equation}
\label{eqn:fourierMaxVac:17c}
H =
\frac{\epsilon_0}{c^2} V \sum_\Bk
\left(
\inv{2} \Abs{\dot{\BA}_\Bk}^2
+
\inv{2}
(c \Bk)^2 \Abs{\BA_\Bk}^2
\right).
\end{equation}
%
How does the gauge choice alter the Poynting vector?  From \autoref{eqn:fourierMaxVac:100}, all the \(\phi_\Bk\) dependence in that integrated momentum density is lost
%
\begin{equation}\label{eqn:fourierMaxVac:200}
\begin{aligned}
\BP &= \epsilon_0 V \sum_{\Bk}
\Real \left(
i \BA_\Bk \left( \inv{c} \conjugateStar{\dot{\BA}_\Bk} \cdot \Bk \right)
\right).
\end{aligned}
\end{equation}
%
The \(\BA_\Bk \cdot \Bk\) solutions to Maxwell's equation are seen to result in zero momentum for this infinite periodic field.  My expectation was something of the form \(c \BP = H \kcap\), so intuition is either failing me, or my math is failing me, or this contrived periodic field solution leads to trouble.

What do we really know about the energy and momentum components of \(T(\gamma^0)\)?  For vacuum, we have
%
\begin{equation}\label{eqn:fourierMaxVac:201}
\inv{c} \PD{t}{T(\gamma^0) \cdot \gamma_0} + \spacegrad \cdot \Bigl(T(\gamma^0) \wedge \gamma_0 \Bigr) = 0.
\end{equation}
%
However, integration over the volume has been performed.  That is different than integrating this four divergence.  What we can say is
%
\begin{equation}\label{eqn:fourierMaxVac:202}
\inv{c} \int d^3 \Bx \PD{t}{T(\gamma^0) \cdot \gamma_0} + \int d^3 \Bx \spacegrad \cdot \Bigl(T(\gamma^0) \wedge \gamma_0 \Bigr) = 0.
\end{equation}
%
It is not obvious that the integration and differentiation order can be switched in order to come up with an expression containing \(H\) and \(\BP\).  This is perhaps where intuition is failing me.

\section{Conclusions and followup}

The objective was met, a reproduction of Bohm's Harmonic oscillator result using a complex exponential Fourier series instead of separate sine and cosines.

The reason for Bohm's choice to fix zero divergence as the gauge choice upfront is now clear.  That automatically cuts complexity from the results.  Figuring out how to work this problem with complex valued potentials and also using the Geometric Algebra formulation probably also made the work a bit more difficult since blundering through both simultaneously was required instead of just one at a time.

This was an interesting exercise though, since doing it this way I am able to understand all the intermediate steps.  Bohm employed some subtler argumentation to eliminate the scalar potential \(\phi\) upfront, and I have to admit I did not follow his logic, whereas blindly following where the math leads me all makes sense.

As a bit of followup, I had like to consider the constant \(\BA_\Bk\) case in more detail, and any implications of the freedom to pick \(\BA_0\).
%I had also like to construct the Poynting vector \(T(\gamma^0) \wedge \gamma_0\), and see what the structure of that is with this Fourier representation.

The general calculation of \(T^{\mu\nu}\) for the assumed Fourier solution should be possible too, but was not attempted.  Doing that general calculation with a four dimensional Fourier series is likely tidier than working with scalar and spatial variables as done here.

Now that the math is out of the way (except possibly for the momentum which does not seem right), some discussion of implications and applications is also in order.  My preference is to let the math sink-in a bit first and mull over the momentum issues at leisure.



%FIXME: check units.  What is up with the \(1/c^2\) in the Hamiltonian?
%
%FIXME: discuss energy and momentum and Hamiltonian equations once done.  Perhaps repost since equation numbers and citations are shuffled.

%\EndArticle

   %
% Copyright � 2012 Peeter Joot.  All Rights Reserved.
% Licenced as described in the file LICENSE under the root directory of this GIT repository.
%

%
%
%\input{../peeter_prologue_widescreen.tex}

% reformated equations 63, 8, 16, 17 to break long lines.
%\input{../peeter_prologue_print.tex}

\mychapter{Fourier transform solutions and associated energy and momentum for the homogeneous Maxwell equation}
\index{Maxwell equation!Fourier transform}
\index{Maxwell equation!energy and momentum}
\label{chap:ftMaxVacuum}
%\blogpage{http://sites.google.com/site/peeterjoot/math2009/ftMaxVacuum.pdf}
%\date{Dec 21, 2009}
%\revisionInfo{ftMaxVacuum.tex}

%\beginArtWithToc
\beginArtNoToc

%FIXME: toggled from not using 'we have' to doing so.  Fix the inconsistencies.
\section{Motivation and notation}

In \chapcite{fourierMaxVac}, building on \chapcite{complexFieldEnergy} a derivation for the energy and momentum density was derived for an assumed Fourier series solution to the homogeneous Maxwell's equation.  Here we move to the continuous case examining Fourier transform solutions and the associated energy and momentum density.

A complex (phasor) representation is implied, so taking real parts when all is said and done is required of the fields.  For the energy momentum tensor the Geometric Algebra form, modified for complex fields, is used

\begin{equation}
\label{eqn:ftMaxVacuum:1}
T(a) = -\frac{\epsilon_0}{2} \Real \Bigl( \conjugateStar{F} a F \Bigr).
\end{equation}

The assumed four vector potential will be written

\begin{equation}
\label{eqn:ftMaxVacuum:2}
A(\Bx, t) = A^\mu(\Bx, t) \gamma_\mu = \inv{(\sqrt{2 \pi})^3} \int A(\Bk, t) e^{i \Bk \cdot \Bx } d^3 \Bk.
\end{equation}

Subject to the requirement that \(A\) is a solution of Maxwell's equation

\begin{equation}
\label{eqn:ftMaxVacuum:3}
\grad (\grad \wedge A) = 0.
\end{equation}

To avoid latex hell, no special notation will be used for the Fourier coefficients,

\begin{equation}
\label{eqn:ftMaxVacuum:3a}
A(\Bk, t) = \inv{(\sqrt{2 \pi})^3} \int A(\Bx, t) e^{-i \Bk \cdot \Bx } d^3 \Bx.
\end{equation}

When convenient and unambiguous, this \((\Bk,t)\) dependence will be implied.

Having picked a time and space representation for the field, it will be natural to express both the four potential and the gradient as scalar plus spatial vector, instead of using the Dirac basis.  For the gradient this is

\begin{equation}
\label{eqn:ftMaxVacuum:4}
\grad = \gamma^\mu \partial_\mu = (\partial_0 - \spacegrad) \gamma_0 = \gamma_0 (\partial_0 + \spacegrad),
\end{equation}

and for the four potential (or the Fourier transform functions), this is

\begin{equation}
\label{eqn:ftMaxVacuum:5}
A = \gamma_\mu A^\mu = (\phi + \BA) \gamma_0 = \gamma_0 (\phi - \BA).
\end{equation}

\section{Setup}

The field bivector \(F = \grad \wedge A\) is required for the energy momentum tensor.  This is

\begin{equation}\label{eqn:ftMaxVacuum:222}
\begin{aligned}
\grad \wedge A
&= \inv{2}\left( \rgrad A - A \lgrad \right) \\
&= \inv{2}\left( (\rpartial_0 - \rspacegrad) \gamma_0 \gamma_0 (\phi - \BA)
-
(\phi + \BA) \gamma_0 \gamma_0 (\lpartial_0 + \lspacegrad)
\right) \\
&= -\spacegrad \phi -\partial_0 \BA + \inv{2}(\rspacegrad \BA - \BA \lspacegrad)
\end{aligned}
\end{equation}

This last term is a spatial curl and the field is then

\begin{equation}
\label{eqn:ftMaxVacuum:6}
F = -\spacegrad \phi -\partial_0 \BA + \spacegrad \wedge \BA
\end{equation}

Applied to the Fourier representation this is

\begin{equation}
\label{eqn:ftMaxVacuum:7}
F =
\inv{(\sqrt{2 \pi})^3} \int
\left(
- \inv{c} \dot{\BA}
- i \Bk \phi
+ i \Bk \wedge \BA
\right)
e^{i \Bk \cdot \Bx } d^3 \Bk.
\end{equation}

It is only the real parts of this that we are actually interested in, unless physical meaning can be assigned to the complete complex vector field.

\section{Constraints supplied by Maxwell's equation}

A Fourier transform solution of Maxwell's vacuum equation \(\grad F = 0\) has been assumed.  Having expressed the Faraday bivector in terms of spatial vector quantities, it is more convenient to do this back substitution into after pre-multiplying Maxwell's equation by \(\gamma_0\), namely

\begin{equation}
\label{eqn:ftMaxVacuum:20}
\begin{aligned}
0
&= \gamma_0 \grad F \\
&= (\partial_0 + \spacegrad) F.
\end{aligned}
\end{equation}

Applied to the spatially decomposed field as specified in \autoref{eqn:ftMaxVacuum:6}, this is

\begin{equation}\label{eqn:ftMaxVacuum:302}
\begin{aligned}
0
&=
-\partial_0 \spacegrad \phi
-\partial_{00} \BA
+ \partial_0 \spacegrad \wedge \BA
-\spacegrad^2 \phi
- \spacegrad \partial_0 \BA
+ \spacegrad \cdot (\spacegrad \wedge \BA ) \\
&=
- \partial_0 \spacegrad \phi - \spacegrad^2 \phi
- \partial_{00} \BA
- \spacegrad \cdot \partial_0 \BA
+ \spacegrad^2 \BA
- \spacegrad (\spacegrad \cdot \BA ) \\
\end{aligned}
\end{equation}

All grades of this equation must simultaneously equal zero, and the bivector grades have canceled (assuming commuting space and time partials), leaving two equations of constraint for the system

\begin{subequations}
\label{eqn:ftMaxVacuum:22}
\begin{align}
0 &=
\spacegrad^2 \phi + \spacegrad \cdot \partial_0 \BA
\label{eqn:ftMaxVacuum:22a}
\\
0 &=
\partial_{00} \BA - \spacegrad^2 \BA
+ \spacegrad \partial_0 \phi
+ \spacegrad ( \spacegrad \cdot \BA )
\label{eqn:ftMaxVacuum:22b}
\end{align}
\end{subequations}

It is immediately evident that a gauge transformation could be immediately helpful to simplify things.  In \citep{bohm1989qt} the gauge choice \(\spacegrad \cdot \BA = 0\) is used.  From \autoref{eqn:ftMaxVacuum:22a} this implies that \(\spacegrad^2 \phi = 0\).  Bohm argues that for this current and charge free case this implies \(\phi = 0\), but he also has a periodicity constraint.  Without a periodicity constraint it is easy to manufacture non-zero counterexamples.  One is a linear function in the space and time coordinates

\begin{equation}
\label{eqn:ftMaxVacuum:23}
\phi = p x + q y + r z + s t
\end{equation}

This is a valid scalar potential provided that the wave equation for the vector potential is also a solution.  We can however, force \(\phi = 0\) by making the transformation \(A^\mu \rightarrow A^\mu + \partial^\mu \psi\), which in non-covariant notation is

\begin{equation}
\label{eqn:ftMaxVacuum:24}
\begin{aligned}
\phi &\rightarrow \phi + \inv{c} \partial_t \psi \\
\BA &\rightarrow \phi - \spacegrad \psi
\end{aligned}
\end{equation}

If the transformed field \(\phi' = \phi + \partial_t \psi/c\) can be forced to zero, then the complexity of the associated Maxwell equations are reduced.  In particular, antidifferentiation of \(\phi = -(1/c) \partial_t \psi\), yields

\begin{equation}
\label{eqn:ftMaxVacuum:25}
\psi(\Bx,t) = \psi(\Bx, 0) - c \int_{\tau=0}^t \phi(\Bx, \tau) d\tau.
\end{equation}

Dropping primes, the transformed Maxwell equations now take the form

\begin{subequations}
\label{eqn:ftMaxVacuum:26}
\begin{align}
0 &= \partial_t( \spacegrad \cdot \BA )
\label{eqn:ftMaxVacuum:26a}
\\
0 &=
\partial_{00} \BA - \spacegrad^2 \BA + \spacegrad (\spacegrad \cdot \BA ).
\label{eqn:ftMaxVacuum:26b}
\end{align}
\end{subequations}

There are two classes of solutions that stand out for these equations.  If the vector potential is constant in time \(\BA(\Bx,t) = \BA(\Bx)\), Maxwell's equations are reduced to the single equation

\begin{equation}
\label{eqn:ftMaxVacuum:28}
0
= - \spacegrad^2 \BA + \spacegrad (\spacegrad \cdot \BA ).
\end{equation}

Observe that a gradient can be factored out of this equation

\begin{equation}\label{eqn:ftMaxVacuum:442}
\begin{aligned}
- \spacegrad^2 \BA + \spacegrad (\spacegrad \cdot \BA )
&=
\spacegrad (-\spacegrad \BA + \spacegrad \cdot \BA ) \\
&=
-\spacegrad (\spacegrad \wedge \BA).
\end{aligned}
\end{equation}

The solutions are then those \(\BA\)s that satisfy both
\begin{subequations}
\begin{align}
\label{eqn:ftMaxVacuum:28b}
0 &= \partial_t \BA \\
0 &= \spacegrad (\spacegrad \wedge \BA).
\end{align}
\end{subequations}

In particular any non-time dependent potential \(\BA\) with constant curl provides a solution to Maxwell's equations.  There may be other solutions to \autoref{eqn:ftMaxVacuum:28} too that are more general.  Returning to \autoref{eqn:ftMaxVacuum:26} a second way to satisfy these equations stands out.  Instead of requiring of \(\BA\) constant curl, constant divergence with respect to the time partial eliminates \autoref{eqn:ftMaxVacuum:26a}.  The simplest resulting equations are those for which the divergence is a constant in time and space (such as zero).  The solution set are then spanned by the vectors \(\BA\) for which

\begin{subequations}
\label{eqn:ftMaxVacuum:29}
\begin{align}
\text{constant} &= \spacegrad \cdot \BA
\label{eqn:ftMaxVacuum:29a}
\\
0 &= \inv{c^2} \partial_{tt} \BA - \spacegrad^2 \BA.
\label{eqn:ftMaxVacuum:29b}
\end{align}
\end{subequations}

Any \(\BA\) that both has constant divergence and satisfies the wave equation will via \autoref{eqn:ftMaxVacuum:6} then produce a solution to Maxwell's equation.

%Also consider the gauge transformation here.  Note that the gauge transformation shows that the \(\Abs{\BA}^2\) belongs in the potential, while the \(\Abs{\dot{\BA}/c}^2\) belongs in the kinetic.  Assume that the remainder of the position only and velocity only terms are grouped that way and see where it leads.

\section{Maxwell equation constraints applied to the assumed Fourier solutions}

Let us consider Maxwell's equations in all three forms, \autoref{eqn:ftMaxVacuum:22}, \autoref{eqn:ftMaxVacuum:28b}, and \autoref{eqn:ftMaxVacuum:29} and apply these constraints to the assumed Fourier solution.

In all cases the starting point is a pair of Fourier transform relationships, where the Fourier transforms are the functions to be determined

\begin{subequations}
\label{eqn:ftMaxVacuum:40}
\begin{align}
\phi(\Bx, t) &= (2 \pi)^{-3/2} \int \phi(\Bk, t) e^{i \Bk \cdot \Bx } d^3 \Bk
\label{eqn:ftMaxVacuum:40a}
\\
\BA(\Bx, t) &= (2 \pi)^{-3/2} \int \BA(\Bk, t) e^{i \Bk \cdot \Bx } d^3 \Bk
\label{eqn:ftMaxVacuum:40b}
\end{align}
\end{subequations}

\subsection{Case I.  Constant time vector potential.  Scalar potential eliminated by gauge transformation}
\index{gauge transformation}
\index{vector potential}

From \autoref{eqn:ftMaxVacuum:40a} we require

\begin{equation}\label{eqn:ftMaxVacuum:50}
0 = (2 \pi)^{-3/2} \int \partial_t \BA(\Bk, t) e^{i \Bk \cdot \Bx } d^3 \Bk.
\end{equation}

So the Fourier transform also cannot have any time dependence, and we have

\begin{equation}\label{eqn:ftMaxVacuum:51}
\BA(\Bx, t) = (2 \pi)^{-3/2} \int \BA(\Bk) e^{i \Bk \cdot \Bx } d^3 \Bk
\end{equation}

What is the curl of this?  Temporarily falling back to coordinates is easiest for this calculation

\begin{equation}\label{eqn:ftMaxVacuum:522}
\begin{aligned}
\spacegrad \wedge \BA(\Bk) e^{i\Bk \cdot \Bx}
&=
\sigma_m \partial_m \wedge \sigma_n A^n(\Bk) e^{i \Bx \cdot \Bx} \\
&=
\sigma_m \wedge \sigma_n A^n(\Bk) i k^m e^{i \Bx \cdot \Bx} \\
&=
i\Bk \wedge \BA(\Bk) e^{i \Bx \cdot \Bx} \\
\end{aligned}
\end{equation}

This gives

\begin{equation}\label{eqn:ftMaxVacuum:52}
\spacegrad \wedge \BA(\Bx, t) = (2 \pi)^{-3/2} \int i \Bk \wedge \BA(\Bk) e^{i \Bk \cdot \Bx } d^3 \Bk.
\end{equation}

We want to equate the divergence of this to zero.  Neglecting the integral and constant factor this requires

\begin{equation}\label{eqn:ftMaxVacuum:542}
\begin{aligned}
0
&=
\spacegrad \cdot \left( i \Bk \wedge \BA e^{i\Bk \cdot \Bx} \right) \\
&=
\gpgradeone{ \sigma_m \partial_m i (\Bk \wedge \BA) e^{i\Bk \cdot \Bx} } \\
&=
-\gpgradeone{ \sigma_m (\Bk \wedge \BA) k^m e^{i\Bk \cdot \Bx} } \\
&=
-\Bk \cdot (\Bk \wedge \BA) e^{i\Bk \cdot \Bx} \\
\end{aligned}
\end{equation}

Requiring that the plane spanned by \(\Bk\) and \(\BA(\Bk)\) be perpendicular to \(\Bk\) implies that \(\BA \propto \Bk\).  The solution set is then completely described by functions of the form

\begin{equation}\label{eqn:ftMaxVacuum:53}
\BA(\Bx, t) = (2 \pi)^{-3/2} \int \Bk \psi(\Bk) e^{i \Bk \cdot \Bx } d^3 \Bk,
\end{equation}

where \(\psi(\Bk)\) is an arbitrary scalar valued function.  This is however, an extremely uninteresting solution since the curl is uniformly zero

\begin{equation}\label{eqn:ftMaxVacuum:562}
\begin{aligned}
F
&= \spacegrad \wedge \BA \\
&= (2 \pi)^{-3/2} \int (i \Bk) \wedge \Bk \psi(\Bk) e^{i \Bk \cdot \Bx } d^3 \Bk.
\end{aligned}
\end{equation}

Since \(\Bk \wedge \Bk = 0\), when all is said and done the \(\phi = 0\), \(\partial_t \BA = 0\) case appears to have no non-trivial (zero) solutions.  Moving on, ...

\subsection{Case II.  Constant vector potential divergence.  Scalar potential eliminated by gauge transformation}
\index{divergence}
\index{gauge transformation}

Next in the order of complexity is consideration of the case \autoref{eqn:ftMaxVacuum:29}.  Here we also have \(\phi = 0\), eliminated by gauge transformation, and are looking for solutions with the constraint

\begin{equation}\label{eqn:ftMaxVacuum:582}
\begin{aligned}
\text{constant}
&= \spacegrad \cdot \BA(\Bx, t) \\
&= (2 \pi)^{-3/2} \int i \Bk \cdot \BA(\Bk, t) e^{i \Bk \cdot \Bx } d^3 \Bk.
\end{aligned}
\end{equation}

How can this constraint be enforced?  The only obvious way is a requirement for \(\Bk \cdot \BA(\Bk, t)\) to be zero for all \((\Bk,t)\), meaning that our to be determined Fourier transform coefficients are required to be perpendicular to the wave number vector parameters at all times.

The remainder of Maxwell's equations, \autoref{eqn:ftMaxVacuum:29b} impose the addition constraint on the Fourier transform \(\BA(\Bk,t)\)

\begin{equation}\label{eqn:ftMaxVacuum:60}
0 = (2 \pi)^{-3/2} \int \left( \inv{c^2} \partial_{tt} \BA(\Bk, t) - i^2 \Bk^2 \BA(\Bk, t)\right) e^{i \Bk \cdot \Bx } d^3 \Bk.
\end{equation}

For zero equality for all \(\Bx\) it appears that we require the Fourier transforms \(\BA(\Bk)\) to be harmonic in time

\begin{equation}\label{eqn:ftMaxVacuum:61}
\partial_{tt} \BA(\Bk, t) = - c^2 \Bk^2 \BA(\Bk, t).
\end{equation}

This has the familiar exponential solutions

\begin{equation}\label{eqn:ftMaxVacuum:62}
\BA(\Bk, t) = \BA_{\pm}(\Bk) e^{ \pm i c \Abs{\Bk} t },
\end{equation}

also subject to a requirement that \(\Bk \cdot \BA(\Bk) = 0\).  Our field, where the \(\BA_{\pm}(\Bk)\) are to be determined by initial time conditions, is by \autoref{eqn:ftMaxVacuum:6} of the form

\begin{equation}\label{eqn:ftMaxVacuum:602}
\begin{aligned}
F(\Bx, t)
&=
\Real \frac{i}{(\sqrt{2\pi})^3} \int \Bigl( -\Abs{\Bk} \BA_{+}(\Bk) + \Bk \wedge \BA_{+}(\Bk) \Bigr) \exp(i \Bk \cdot \Bx + i c \Abs{\Bk} t) d^3 \Bk \\
&+ \Real \frac{i}{(\sqrt{2\pi})^3} \int \Bigl( \Abs{\Bk} \BA_{-}(\Bk) + \Bk \wedge \BA_{-}(\Bk) \Bigr) \exp(i \Bk \cdot \Bx - i c \Abs{\Bk} t) d^3 \Bk.
\end{aligned}
\end{equation}

Since \(0 = \Bk \cdot \BA_{\pm}(\Bk)\), we have \(\Bk \wedge \BA_{\pm}(\Bk) = \Bk \BA_{\pm}\).  This allows for factoring out of \(\Abs{\Bk}\).  The structure of the solution is not changed by incorporating the \(i (2\pi)^{-3/2} \Abs{\Bk}\) factors into \(\BA_{\pm}\), leaving the field having the general form

\begin{equation}\label{eqn:ftMaxVacuum:64}
\begin{aligned}
&F(\Bx, t) \\
&=
\Real \int ( \kcap - 1 ) \BA_{+}(\Bk) \exp(i \Bk \cdot \Bx + i c \Abs{\Bk} t) d^3 \Bk
+ \Real \int ( \kcap + 1 ) \BA_{-}(\Bk) \exp(i \Bk \cdot \Bx - i c \Abs{\Bk} t) d^3 \Bk.
\end{aligned}
\end{equation}

The original meaning of \(\BA_{\pm}\) as Fourier transforms of the vector potential is obscured by the tidy up change to absorb \(\Abs{\Bk}\), but the geometry of the solution is clearer this way.

It is also particularly straightforward to confirm that \(\gamma_0 \grad F = 0\) separately for either half of \autoref{eqn:ftMaxVacuum:64}.

\subsection{Case III.  Non-zero scalar potential.  No gauge transformation}

Now lets work from \autoref{eqn:ftMaxVacuum:22}.  In particular, a divergence operation can be factored from \autoref{eqn:ftMaxVacuum:22a}, for

\begin{equation}\label{eqn:ftMaxVacuum:70}
0 = \spacegrad \cdot (\spacegrad \phi + \partial_0 \BA).
\end{equation}

Right off the top, there is a requirement for

\begin{equation}\label{eqn:ftMaxVacuum:71}
\text{constant} = \spacegrad \phi + \partial_0 \BA.
\end{equation}

In terms of the Fourier transforms this is

\begin{equation}\label{eqn:ftMaxVacuum:72}
\text{constant} =
\inv{(\sqrt{2 \pi})^3} \int
\Bigl(
i \Bk \phi(\Bk, t) + \inv{c} \partial_t \BA(\Bk, t)
\Bigr)
e^{i \Bk \cdot \Bx } d^3 \Bk.
\end{equation}

Are there any ways for this to equal a constant for all \(\Bx\) without requiring that constant to be zero?  Assuming no for now, and that this constant must be zero, this implies a coupling between the \(\phi\) and \(\BA\) Fourier transforms of the form

\begin{equation}
\label{eqn:ftMaxVacuum:73}
\phi(\Bk, t) = -\inv{i c \Bk} \partial_t \BA(\Bk, t)
\end{equation}

A secondary implication is that \(\partial_t \BA(\Bk, t) \propto \Bk\) or else \(\phi(\Bk, t)\) is not a scalar.  We had a transverse solution by requiring via gauge transformation that \(\phi = 0\), and here we have instead the vector potential in the propagation direction.

A secondary confirmation that this is a required coupling between the scalar and vector potential can be had by evaluating the divergence equation of \autoref{eqn:ftMaxVacuum:70}

\begin{equation}\label{eqn:ftMaxVacuum:72b}
0 =
\inv{(\sqrt{2 \pi})^3} \int
\Bigl(
- \Bk^2 \phi(\Bk, t) + \frac{i\Bk}{c} \cdot \partial_t \BA(\Bk, t)
\Bigr)
e^{i \Bk \cdot \Bx } d^3 \Bk.
\end{equation}

Rearranging this also produces \autoref{eqn:ftMaxVacuum:73}.  We want to now substitute this relationship into \autoref{eqn:ftMaxVacuum:22b}.

Starting with just the \(\partial_0 \phi - \spacegrad \cdot \BA\) part we have

\begin{equation}\label{eqn:ftMaxVacuum:74}
\partial_0 \phi + \spacegrad \cdot \BA
=
\inv{(\sqrt{2 \pi})^3} \int
\Bigl(
\frac{i}{c^2 \Bk} \partial_{tt} \BA(\Bk, t) + i \Bk \cdot \BA
\Bigr)
e^{i \Bk \cdot \Bx } d^3 \Bk.
\end{equation}

Taking the gradient of this brings down a factor of \(i\Bk\) for

\begin{equation}\label{eqn:ftMaxVacuum:74b}
\spacegrad (\partial_0 \phi + \spacegrad \cdot \BA)
=
-\inv{(\sqrt{2 \pi})^3} \int
\Bigl(
\frac{1}{c^2} \partial_{tt} \BA(\Bk, t) + \Bk (\Bk \cdot \BA)
\Bigr)
e^{i \Bk \cdot \Bx } d^3 \Bk.
\end{equation}

\autoref{eqn:ftMaxVacuum:22b} in its entirety is now

\begin{equation}\label{eqn:ftMaxVacuum:75}
0 =
\inv{(\sqrt{2 \pi})^3} \int
\Bigl(
%\frac{2}{c^2} \partial_{tt} \BA(\Bk, t)
- (i\Bk)^2 \BA
+ \Bk (\Bk \cdot \BA)
\Bigr)
e^{i \Bk \cdot \Bx } d^3 \Bk.
\end{equation}

This is not terribly pleasant looking.  Perhaps going the other direction.  We could write

\begin{equation}\label{eqn:ftMaxVacuum:76}
\phi = \frac{i}{c \Bk} \PD{t}{\BA} = \frac{i}{c} \PD{t}{\psi},
\end{equation}

so that

\begin{equation}\label{eqn:ftMaxVacuum:77}
\BA(\Bk, t) = \Bk \psi(\Bk, t).
\end{equation}

\begin{equation}\label{eqn:ftMaxVacuum:642}
0
=
\inv{(\sqrt{2 \pi})^3} \int
\Bigl(
\inv{c^2} \Bk \psi_{tt}
- \spacegrad^2 \Bk \psi
+ \spacegrad \frac{i}{c^2} \psi_{tt}
+\spacegrad( \spacegrad \cdot (\Bk \psi) )
\Bigr)
e^{i \Bk \cdot \Bx } d^3 \Bk \\
\end{equation}

Note that the gradients here operate on everything to the right, including and especially the exponential.  Each application of the gradient brings down an additional \(i\Bk\) factor, and we have

\begin{equation}\label{eqn:ftMaxVacuum:662}
\inv{(\sqrt{2 \pi})^3} \int
\Bk \Bigl(
\inv{c^2} \psi_{tt}
- i^2 \Bk^2 \psi
+ \frac{i^2}{c^2} \psi_{tt}
+i^2 \Bk^2 \psi
\Bigr)
e^{i \Bk \cdot \Bx } d^3 \Bk.
\end{equation}

This is identically zero, so we see that this second equation provides no additional information.  That is somewhat surprising since there is not a whole lot of constraints supplied by the first equation.  The function \(\psi(\Bk, t)\) can be anything.  Understanding of this curiosity comes from computation of the Faraday bivector itself.  From \autoref{eqn:ftMaxVacuum:6}, that is

\begin{equation}\label{eqn:ftMaxVacuum:78}
F =
\inv{(\sqrt{2 \pi})^3} \int
\Bigl(
-i \Bk \frac{i}{c}\psi_t - \inv{c} \Bk \psi_t + i \Bk \wedge \Bk \psi
\Bigr)
e^{i \Bk \cdot \Bx } d^3 \Bk.
\end{equation}

All terms cancel, so we see that a non-zero \(\phi\) leads to \(F = 0\), as was the case when considering \autoref{eqn:ftMaxVacuum:40a} (a case that also resulted in \(\BA(\Bk) \propto \Bk\)).

Can this Fourier representation lead to a non-transverse solution to Maxwell's equation?  If so, it is not obvious how.

\section{The energy momentum tensor}

The energy momentum tensor is then

\begin{equation}\label{eqn:ftMaxVacuum:682}
\begin{aligned}
T(a) = -\frac{\epsilon_0}{2 (2 \pi)^3} \Real \iint &
\left(
- \inv{c} \conjugateStar{\dot{\BA}}(\Bk',t)
+ i \Bk' \conjugateStar{\phi}(\Bk', t)
- i \Bk' \wedge \conjugateStar{\BA}(\Bk', t)
\right) \\
&a
\left(
- \inv{c} \dot{\BA}(\Bk, t)
- i \Bk \phi(\Bk, t)
+ i \Bk \wedge \BA(\Bk, t)
\right)
e^{i (\Bk -\Bk') \cdot \Bx } d^3 \Bk d^3 \Bk'.
\end{aligned}
\end{equation}

Observing that \(\gamma_0\) commutes with spatial bivectors and anticommutes with spatial vectors, and writing \(\sigma_\mu = \gamma_\mu \gamma_0\), the tensor splits neatly into scalar and spatial vector components

%\begin{subequations}
%\label{eqn:ftMaxVacuum:16}
\begin{equation}\label{eqn:ftMaxVacuum:702}
\begin{aligned}
&T(\gamma_\mu) \cdot \gamma_0 = \frac{\epsilon_0}{2 (2 \pi)^3} \Real \iint
e^{i (\Bk -\Bk') \cdot \Bx } d^3 \Bk d^3 \Bk' \\
&\gpgradezero{
\left(
\inv{c} \conjugateStar{\dot{\BA}}(\Bk',t)
- i \Bk' \conjugateStar{\phi}(\Bk', t)
+ i \Bk' \wedge \conjugateStar{\BA}(\Bk', t)
\right)
\sigma_\mu
\left(
\inv{c} \dot{\BA}(\Bk, t)
+ i \Bk \phi(\Bk, t)
+ i \Bk \wedge \BA(\Bk, t)
\right)
} \\
&T(\gamma_\mu) \wedge \gamma_0 = \frac{\epsilon_0}{2 (2 \pi)^3} \Real \iint
e^{i (\Bk -\Bk') \cdot \Bx } d^3 \Bk d^3 \Bk' \\
&\gpgradeone{
\left(
\inv{c} \conjugateStar{\dot{\BA}}(\Bk',t)
- i \Bk' \conjugateStar{\phi}(\Bk', t)
+ i \Bk' \wedge \conjugateStar{\BA}(\Bk', t)
\right)
\sigma_\mu
\left(
\inv{c} \dot{\BA}(\Bk, t)
+ i \Bk \phi(\Bk, t)
+ i \Bk \wedge \BA(\Bk, t)
\right)
}.
\end{aligned}
\end{equation}
%\end{subequations}

In particular for \(\mu = 0\), we have

%\begin{subequations}
%\label{eqn:ftMaxVacuum:17}
\begin{equation}\label{eqn:ftMaxVacuum:722}
\begin{aligned}
H &\equiv
T(\gamma_0) \cdot \gamma_0 = \frac{\epsilon_0}{2 (2 \pi)^3} \Real \iint
e^{i (\Bk -\Bk') \cdot \Bx } d^3 \Bk d^3 \Bk' \\
&\left(
\left(
\inv{c} \conjugateStar{\dot{\BA}}(\Bk',t)
- i \Bk' \conjugateStar{\phi}(\Bk', t)
\right)
\cdot
\left(
\inv{c} \dot{\BA}(\Bk, t)
+ i \Bk \phi(\Bk, t)
\right)
- (\Bk' \wedge \conjugateStar{\BA}(\Bk', t)) \cdot (\Bk \wedge \BA(\Bk, t))
\right)
\\
\BP &\equiv
T(\gamma_\mu) \wedge \gamma_0 = \frac{\epsilon_0}{2 (2 \pi)^3} \Real \iint
e^{i (\Bk -\Bk') \cdot \Bx } d^3 \Bk d^3 \Bk' \\
&\left(
i
\left(
\inv{c} \conjugateStar{\dot{\BA}}(\Bk',t)
- i \Bk' \conjugateStar{\phi}(\Bk', t)
\right) \cdot
\left(
\Bk \wedge \BA(\Bk, t)
\right)
-i
\left(
\inv{c} \dot{\BA}(\Bk, t)
+ i \Bk \phi(\Bk, t)
\right)
\cdot
\left(
\Bk' \wedge \conjugateStar{\BA}(\Bk', t)
\right)
\right).
\end{aligned}
\end{equation}
%\end{subequations}

Integrating this over all space and identification of the delta function

\begin{equation}
\label{eqn:ftMaxVacuum:9}
\delta(\Bk) \equiv \inv{(2 \pi)^3} \int e^{i \Bk \cdot \Bx} d^3 \Bx,
\end{equation}

reduces the tensor to a single integral in the continuous angular wave number space of \(\Bk\).

\begin{equation}
\label{eqn:ftMaxVacuum:10}
\int T(a) d^3 \Bx = -\frac{\epsilon_0}{2} \Real \int
\left(
- \inv{c} \conjugateStar{\dot{\BA}}
+ i \Bk \conjugateStar{\phi}
- i \Bk \wedge \conjugateStar{\BA}
\right)
a
\left(
- \inv{c} \dot{\BA}
- i \Bk \phi
+ i \Bk \wedge \BA
\right)
d^3 \Bk.
\end{equation}

Or,

\begin{equation}
\label{eqn:ftMaxVacuum:12}
\int T(\gamma_\mu) \gamma_0 d^3 \Bx =
\frac{\epsilon_0}{2} \Real \int
\gpgrade{
\left(
\inv{c} \conjugateStar{\dot{\BA}}
- i \Bk \conjugateStar{\phi}
+ i \Bk \wedge \conjugateStar{\BA}
\right)
\sigma_\mu
\left(
\inv{c} \dot{\BA}
+ i \Bk \phi
+ i \Bk \wedge \BA
\right)
}{0,1}
d^3 \Bk.
\end{equation}

%%\begin{align}
%%\label{eqn:ftMaxVacuum:13b}
%%\PD{t}{H} + \spacegrad \cdot (c \BP) = 0.
%%\end{align}

Multiplying out \autoref{eqn:ftMaxVacuum:12} yields for \(\int H\)

\begin{equation}
\label{eqn:ftMaxVacuum:14}
\int H d^3 \Bx =
\frac{\epsilon_0}{2} \int d^3 \Bk \left(
\inv{c^2} \Abs{\dot{\BA}}^2 + \Bk^2 (\Abs{\phi}^2 + \Abs{\BA}^2 )
- \Abs{\Bk \cdot \BA}^2
+ 2 \frac{\Bk}{c} \cdot \Real( i \conjugateStar{\phi} \dot{\BA} )
\right)
\end{equation}

Recall that the only non-trivial solution we found for the assumed Fourier transform representation of \(F\) was for \(\phi = 0\), \(\Bk \cdot \BA(\Bk, t) = 0\).  Thus we have for the energy density integrated over all space, just

\begin{equation}
\label{eqn:ftMaxVacuum:14b}
\int H d^3 \Bx =
\frac{\epsilon_0}{2} \int d^3 \Bk \left(
\inv{c^2} \Abs{\dot{\BA}}^2 + \Bk^2 \Abs{\BA}^2
\right).
\end{equation}

Observe that we have the structure of a Harmonic oscillator for the energy of the radiation system.  What is the canonical momentum for this system?  Will it correspond to the Poynting vector, integrated over all space?

Let us reduce the vector component of \autoref{eqn:ftMaxVacuum:12}, after first imposing the \(\phi=0\), and \(\Bk \cdot \BA = 0\) conditions used to above for our harmonic oscillator form energy relationship.  This is

\begin{equation}\label{eqn:ftMaxVacuum:842}
\begin{aligned}
\int \BP d^3 \Bx
&=
\frac{\epsilon_0}{2 c} \Real
\int d^3 \Bk \left(
i \conjugateStar{\BA}_t \cdot (\Bk \wedge \BA)
+ i (\Bk \wedge \conjugateStar{\BA}) \cdot \BA_t
\right) \\
&=
\frac{\epsilon_0}{2 c} \Real
\int d^3 \Bk \left(
-i (\conjugateStar{\BA}_t \cdot \BA) \Bk
+ i \Bk (\conjugateStar{\BA} \cdot \BA_t)
\right)
\end{aligned}
\end{equation}

This is just

\begin{equation}
\label{eqn:ftMaxVacuum:15}
\int \BP d^3 \Bx
=
\frac{\epsilon_0}{c} \Real
i \int
\Bk (\conjugateStar{\BA} \cdot \BA_t) d^3 \Bk.
\end{equation}

Recall that the Fourier transforms for the transverse propagation case had the form \(\BA(\Bk, t) = \BA_{\pm}(\Bk) e^{\pm i c \Abs{\Bk} t}\), where the minus generated the advanced wave, and the plus the receding wave.  With substitution of the vector potential for the advanced wave into the energy and momentum results of \autoref{eqn:ftMaxVacuum:14b} and \autoref{eqn:ftMaxVacuum:15} respectively, we have

\begin{equation}\label{eqn:ftMaxVacuum:80}
\begin{aligned}
\int H d^3 \Bx   &= \epsilon_0 \int \Bk^2 \Abs{\BA(\Bk)}^2 d^3 \Bk \\
\int \BP d^3 \Bx &= \epsilon_0 \int \kcap \Bk^2 \Abs{\BA(\Bk)}^2 d^3 \Bk.
\end{aligned}
\end{equation}

After a somewhat circuitous route, this has the relativistic symmetry that is expected.  In particular the for the complete \(\mu=0\) tensor we have after integration over all space

\begin{equation}\label{eqn:ftMaxVacuum:81}
\int
T(\gamma_0) \gamma_0 d^3 \Bx = \epsilon_0 \int (1 + \kcap) \Bk^2 \Abs{\BA(\Bk)}^2 d^3 \Bk.
\end{equation}

The receding wave solution would give the same result, but directed as \(1 - \kcap\) instead.

Observe that we also have the four divergence conservation statement that is expected

\begin{equation}\label{eqn:ftMaxVacuum:82}
\PD{t}{} \int H d^3 \Bx + \spacegrad \cdot \int c \BP d^3 \Bx = 0.
\end{equation}

This follows trivially since both the derivatives are zero.  If the integration region was to be more specific instead of a \(0 + 0 = 0\) relationship, we would have the power flux \(\PDi{t}{H}\) equal in magnitude to the momentum change through a bounding surface.  For a more general surface the time and spatial dependencies should not necessarily vanish, but we should still have this radiation energy momentum conservation.

%%FIXME: Given an energy H and a set of generalized coordinates, how does one construct the canonical momenta without the Lagrangian.  How can one find the Lagrangian from the Hamiltonian, when the separate potential and kinetic terms are not known.  What is the potential and the Kinetic in this system (ie: \(H = E^2 + B^2\)).  Looks like there is a separation into velocity and non-velocity parts.  Using that could potentially produce a Lagrangian.

%\EndArticle


\part{Quantum Mechanics}
   %
% Copyright � 2012 Peeter Joot.  All Rights Reserved.
% Licenced as described in the file LICENSE under the root directory of this GIT repository.
%

%
%
\mychapter{Bohr Model}
\index{Bohr model}
\label{chap:bohr}
%\date{Dec 11, 2008.  bohr.tex}

\section{Motivation}

The Bohr model is taught as early as high school chemistry when the various orbitals are
discussed (or maybe it was high school physics).  I recall
that the first time I saw this I did not see where all the ideas came from.
With a bit more math under my belt now, reexamine these ideas as a lead up to
the proper wave mechanics.

\section{Calculations}

\subsection{Equations of motion}

A prerequisite to discussing electron orbits is first setting up the equations of motion
for the two charged particles (ie: the proton and electron).

With the proton position at \(\Br_p\), and the electron at \(\Br_e\), we have two equations, one
for the force on the proton from the electron and the other for the force on the proton from
the electron.  These are respectively
%
\begin{equation}\label{eqn:bohr:chargeEquations}
\begin{aligned}
  \K e^2 \frac { \Br_e - \Br_p } { \Abs{\Br_e - \Br_p}^3 } &= m_p \frac{d^2 \Br_p }{dt^2} \\
- \K e^2 \frac { \Br_e - \Br_p } { \Abs{\Br_e - \Br_p}^3 } &= m_e \frac{d^2 \Br_e }{dt^2}
\end{aligned}
\end{equation}
%
In lieu of a picture, setting \(\Br_p = 0\) works to check signs, leaving an inwards force on the electron as desired.

% FIXME: Add one.
%\begin{figure}[htp]
%\centering
%\includegraphics[totalheight=0.4\textheight]{picturepath}
%\caption{My Caption}\label{fig:pictlabel}
%\end{figure}
%
%... see \cref{fig:picturepath} ...

As usual for a two body problem, use of the difference vector and center of mass vector is desirable.  That is
%
\begin{equation}\label{eqn:bohr:20}
\begin{aligned}
\Bx &= \Br_e - \Br_p \\
M &= m_e + m_p \\
\BR &= \inv{M}(m_e \Br_e + m_p \Br_p)
\end{aligned}
\end{equation}
%
Solving for \(\Br_p\) and \(\Br_e\) in terms of \(\BR\) and \(\Bx\) we have
%
\begin{equation}\label{eqn:bohr:40}
\begin{aligned}
\Br_e &= \frac{m_p}{M} \Bx + \BR \\
\Br_p &= \frac{-m_e}{M} \Bx + \BR \\
\end{aligned}
\end{equation}
%
% check:
%r_e - r_p = M/M x
%m_e \Br_e + m_p \Br_p &= \frac{-m_e m_p}{M} \Bx + m_e \BR + \frac{m_p m_e}{M} \Bx + m_p \BR \\

Substitution back into \eqnref{eqn:bohr:chargeEquations} we have
%
\begin{equation}\label{eqn:bohr:60}
\begin{aligned}
  \K e^2 \frac {\Bx} { \Abs{\Bx}^3 } &= m_p \frac{d^2}{dt^2}\left( \frac{-m_e}{M} \Bx + \BR \right) \\
 -\K e^2 \frac {\Bx} { \Abs{\Bx}^3 } &= m_e \frac{d^2}{dt^2}\left( \frac{m_p}{M} \Bx + \BR \right),
\end{aligned}
\end{equation}
%
and sums and (scaled) differences of that give us our reduced mass equation and constant center-of-mass velocity equation
\begin{equation}\label{eqn:bohr:80}
\begin{aligned}
\frac{d^2 \Bx}{dt^2} &= -\K e^2 \frac {\Bx} { \Abs{\Bx}^3 } \left( \inv{m_e} + \inv{m_p} \right) \\
\frac{d^2 \BR}{dt^2} &= 0
\end{aligned}
\end{equation}
%
writing \(1/\mu = 1/m_e + 1/m_p\), and \(k = e^2/4 \pi \epsilon_0\), our difference vector equation is thus
%
\begin{equation}\label{eqn:bohr:reduceEOM}
\begin{aligned}
\mu \frac{d^2 \Bx}{dt^2} &= -k \frac {\Bx} { \Abs{\Bx}^3 }
\end{aligned}
\end{equation}
%
\subsection{Circular solution}

The Bohr model postulates that electron orbits are circular.  It is easy enough to verify that a circular orbit in the center of mass frame is a solution to equation
\eqnref{eqn:bohr:reduceEOM}.   Write the path in terms of the unit bivector for the plane of rotation \(i\) and an initial vector position \(\Bx_0\)
%
\begin{equation}\label{eqn:bohr:circular}
\begin{aligned}
\Bx = \Bx_0 e^{i \omega t}
\end{aligned}
\end{equation}
%
For constant \(i\) and \(\omega\), we have
%
\begin{equation}\label{eqn:bohr:100}
\begin{aligned}
\mu \Bx_0 (i\omega)^2 e^{i\omega t} = - k \frac{\Bx_0}{\Abs{\Bx_0}^3} e^{i\omega t}
\end{aligned}
\end{equation}
%
This provides the
angular velocity in terms of the reduced mass of the system and the charge constants
%
\begin{equation}\label{eqn:bohr:omegaSquared}
\begin{aligned}
\omega^2 = \frac{k}{\mu \Abs{\Bx_0}^3} = \frac{e^2}{4 \pi \epsilon_0 \mu \Abs{\Bx_0}^3}.
\end{aligned}
\end{equation}
%
Although not relevant to the quantum theme, it is hard not to call out the observation that this is
a Kepler's law like relation for the period of the circular orbit given the radial distance from the center of mass
%
\begin{equation}\label{eqn:bohr:120}
\begin{aligned}
T^2 = \frac{16 \pi^3 \epsilon_0 \mu}{e^2} \Abs{\Bx_0}^3
\end{aligned}
\end{equation}
%
Kepler's law also holds for elliptical orbits, but this takes more work to show.

\subsection{Angular momentum conservation}
\index{angular momentum conservation}

Now, the next step in the Bohr argument was that the angular momentum, a conserved quantity is also quantized.  To give real
meaning to the conservation statement we need the equivalent Lagrangian formulation of \eqnref{eqn:bohr:reduceEOM}.  Anti-differentiation
gives
%
\begin{equation}\label{eqn:bohr:140}
\begin{aligned}
\grad_\Bv \left( \inv{2} \mu \Bv^2 \right)
&= k \xcap \partial_x \inv{x} \\
&= - \grad_\Bx
\mathLabelBox
[
   labelstyle={below of=m\themathLableNode, below of=m\themathLableNode}
]
{\left(- k\inv{\Abs{\Bx}}\right)}{\(=\phi\)}
\end{aligned}
\end{equation}
%
So, our Lagrangian is
\begin{equation}\label{eqn:bohr:160}
\begin{aligned}
\LL = K - \phi = \inv{2} \mu \Bv^2 + k \inv{\Abs{\Bx}}
\end{aligned}
\end{equation}
%
The essence of the conservation argument, an application of
Noether's theorem,
is that a rotational transformation of the Lagrangian leaves this energy relationship unchanged.  Repeating
the angular momentum example from \citep{classicalmechanics:PJEulerLagrange} (which was done for the more general case of any radial potential), we
write \(\hat{B}\) for the unit bivector associated with a rotational plane.  The position vector is transformed by rotation in this plane as follows
%
\begin{equation}\label{eqn:bohr:180}
\begin{aligned}
\Bx &\rightarrow \Bx' \\
\Bx' &= R \Bx R^\dagger \\
R &= \exp{\hat{B}\theta/2}
\end{aligned}
\end{equation}
%
The magnitude of the position vector is rotation invariant
%
\begin{equation}\label{eqn:bohr:200}
\begin{aligned}
(\Bx')^2 &= R \Bx R^\dagger R \Bx R^\dagger = \Bx^2,
\end{aligned}
\end{equation}
%
as is our the square of the transformed velocity.  The transformed velocity is
%
\begin{equation}\label{eqn:bohr:220}
\begin{aligned}
\frac{d\Bx'}{dt} &= \dot{R} \Bx R + R \dot{\Bx} R^\dagger + R \Bx \dot{R}^\dagger
\end{aligned}
\end{equation}
%
but with \(\dot{\theta} = 0\), \(\dot{R} = 0\) its square is just
%
\begin{equation}\label{eqn:bohr:240}
\begin{aligned}
(\Bv')^2 &= R {\Bv} R^\dagger R \dot{\Bv} R^\dagger = \Bv^2.
\end{aligned}
\end{equation}
%
We therefore have a Lagrangian that is invariant under this rotational transformation
%
\begin{equation}\label{eqn:bohr:260}
\begin{aligned}
\LL \rightarrow \LL' = \LL,
\end{aligned}
\end{equation}
%
and by Noether's theorem (essentially application of the chain rule), we have
%
\begin{equation}\label{eqn:bohr:280}
\begin{aligned}
\frac{d\LL'}{d\theta}
&= \frac{d}{dt} \left(\frac{d\Bx'}{d\theta} \cdot \grad_{\Bv'} \LL \right) \\
&= \frac{d}{dt} \left( (\hat{B} \cdot \Bx') \cdot \mu \Bv' \right).
\end{aligned}
\end{equation}
%
But \(d\LL'/d\theta = 0\), so we have for any \(\hat{B}\)
%
\begin{equation}\label{eqn:bohr:300}
\begin{aligned}
(\hat{B} \cdot \Bx') \cdot (\mu \Bv') &= \hat{B} \cdot (\Bx' \wedge (\mu \Bv')) = \text{constant}
\end{aligned}
\end{equation}
%
Dropping primes this is
%
\begin{equation}\label{eqn:bohr:320}
\begin{aligned}
L = \Bx \wedge (\mu \Bv) = \text{constant},
\end{aligned}
\end{equation}
%
a constant bivector for the conserved center of mass (reduced-mass) angular momentum associated with the Lagrangian of this system.

\subsection{Quantized angular momentum for circular solution}

In terms of the circular solution of \eqnref{eqn:bohr:circular} the angular momentum bivector is
%
\begin{equation}\label{eqn:bohr:340}
\begin{aligned}
L = \Bx \wedge (\mu \Bv)
&= \gpgradetwo{ \Bx_0 e^{i \omega t} \mu \Bx_0 i \omega e^{i \omega t} } \\
&= \gpgradetwo{ e^{-i \omega t} \Bx_0 \mu \Bx_0 \omega e^{i \omega t} i } \\
&= (\Bx_0)^2 \mu \omega i \\
%&= i \frac{e \mu}{2} \sqrt{\frac{\Abs{\Bx_0}}{\pi \epsilon_0 \mu}} \\
&= i e \sqrt{\frac{\mu \Abs{\Bx_0}}{4 \pi \epsilon_0}}
\end{aligned}
\end{equation}
%
%\begin{align}\label{eqn:bohr:omegaSquared}
%\omega = \frac{e}{2 \sqrt{\pi \epsilon_0}} \Abs{\Bx_0}^{-3/2}

Now if this angular momentum is quantized with quantum magnitude \(l\) we have we have for the bivector angular momentum the values
%
\begin{equation}\label{eqn:bohr:360}
\begin{aligned}
L = i n l = i e \sqrt{\frac{\mu \Abs{\Bx_0}}{4 \pi \epsilon_0}}
\end{aligned}
\end{equation}
%
Which with \(l = \Hbar\) (where experiment in the form of the spectral hydrogen line values is required to fix this constant and relate it to Plank's black body constant)
is the momentum equation in terms of
the Bohr radius \(\Bx_0\) at each energy level.  Writing that radius \(r_n = \Abs{\Bx_0}\) explicitly as a function of n, we have
%
\begin{equation}\label{eqn:bohr:380}
\begin{aligned}
r_n = \frac{4 \pi \epsilon_0}{\mu} \left(\frac{n \Hbar}{e}\right)^2
\end{aligned}
\end{equation}
%
\subsubsection{Velocity}

One of the assumptions of this treatment is a \(\Abs{\Bv_e} << c\) requirement so that Coulombs law is valid (ie: slow enough that all the other Maxwell's equations can be neglected).
Let us evaluate the velocity numerically at the some of the quantization levels and see how this compares to the speed of light.

First we need an expression for the velocity itself.  This is
%
\begin{equation}\label{eqn:bohr:400}
\begin{aligned}
\Bv^2
&= ( \Bx_0 i \omega e^{i \omega t} )^2 \\
&= \frac{e^2}{4 \pi \epsilon_0 \mu r_n} \\
&= \frac{e^4}{(4 \pi \epsilon_0)^2 (n \Hbar)^2}.
\end{aligned}
\end{equation}
%
For
\begin{equation}\label{eqn:bohr:420}
\begin{aligned}
v_n
&= \frac{e^2}{4 \pi \epsilon_0 n \Hbar} \\
&= 2.1 \times 10^6 m/s
\end{aligned}
\end{equation}
%
This is the \(1/137\) of the speed of light value that one sees googling electron speed in hydrogen, and only decreases with quantum number so the non-relativistic speed approximation holds
(\(\gamma = 1.00002663\)).  This speed is still pretty zippy, even if it is not relativistic, so it is not unreasonable to attempt to repeat this treatment trying to incorporate the remainder
of Maxwell's equations.

Interestingly the velocity is not a function of the reduced mass at all, but just the charge and quantum numbers.  One also gets a good hint at why the Bohr theory breaks down
for larger atoms.  An electron in circular orbit around an ion of Gold would have a velocity of \(79/137\) the speed of light!

% google calculator:
%1/sqrt(1- ((elementary charge)^2 / 4 / pi / epsilon_0 /hbar/c)^2)

% - discuss connection to Sch. results?
% - try: proper maxwell's/Lorentz equations instead of just the Coulomb force.

   %
% Copyright � 2012 Peeter Joot.  All Rights Reserved.
% Licenced as described in the file LICENSE under the root directory of this GIT repository.
%

%
%
\chapter{Schr\"{o}dinger equation probability conservation}
\index{Schr\"{o}dinger equation!probability conservation}
\label{chap:schCurrent}
%\date{Jan 11, 2009.  schCurrent.tex}
\section{Motivation}

In \citep{mcmahon2005qmd} is a one dimensional probability conservation
derivation from
Schr\"{o}dinger's equation.  Do this for the three dimensional case.

\section{}

Consider the time rate of change of the probability as expressed
in terms of the wave function

\begin{equation}\label{eqn:schCurrent:20}
\begin{aligned}
\PD{t}{\rho}
&= \PD{t}{\psi^\conj \psi} \\
&= \PD{t}{\psi^\conj} \psi + \psi^\conj \PD{t}{\psi} \\
\end{aligned}
\end{equation}

This can be calculated from Schr\"{o}dinger's equation and its complex
conjugate

\begin{equation}\label{eqn:schCurrent:40}
\begin{aligned}
\partial_t \psi &= \left(-\frac{\Hbar}{2mi}\spacegrad^2 + \inv{i\Hbar} V \right) \psi \\
\partial_t \psi^\conj &= \left(\frac{\Hbar}{2mi}\spacegrad^2 - \inv{i\Hbar} V \right) \psi^\conj \\
\end{aligned}
\end{equation}

Multiplying by the conjugate wave functions and adding we have
\begin{equation}\label{eqn:schCurrent:60}
\begin{aligned}
\PD{t}{\rho}
&=
\psi^\conj \left(-\frac{\Hbar}{2mi}\spacegrad^2 + \inv{i\Hbar} V \right) \psi +
\psi \left(\frac{\Hbar}{2mi}\spacegrad^2 - \inv{i\Hbar} V \right) \psi^\conj \\
&=
\frac{\Hbar}{2mi} \left(
-\psi^\conj \spacegrad^2 \psi + \psi \spacegrad^2 \psi^\conj \right) \\
\end{aligned}
\end{equation}

So we have the following conservation law
\begin{equation}\label{eqn:sch_current:intermediate}
\begin{aligned}
\PD{t}{\rho} + \frac{\Hbar}{2mi} \left( \psi^\conj \spacegrad^2 \psi - \psi \spacegrad^2 \psi^\conj \right) = 0
\end{aligned}
\end{equation}

The text indicates that the second order terms here can be written as a divergence.  Somewhat loosely, by treating \(\psi\) as a scalar field one can show that this is the case

\begin{equation}\label{eqn:schCurrent:80}
\begin{aligned}
\spacegrad \cdot \left( \psi^\conj \spacegrad \psi - \psi \spacegrad \psi^\conj \right)
&=
\gpgradezero{
\spacegrad \left( \psi^\conj \spacegrad \psi - \psi \spacegrad \psi^\conj \right)
} \\
&=
\gpgradezero{
(\spacegrad \psi^\conj) (\spacegrad \psi) - (\spacegrad \psi) (\spacegrad \psi^\conj)
+\psi^\conj \spacegrad^2 \psi - \psi \spacegrad^2 \psi^\conj
} \\
&=
\gpgradezero{
2 (\spacegrad \psi^\conj) \wedge (\spacegrad \psi)
+\psi^\conj \spacegrad^2 \psi - \psi \spacegrad^2 \psi^\conj
} \\
&=
\psi^\conj \spacegrad^2 \psi - \psi \spacegrad^2 \psi^\conj
 \\
\end{aligned}
\end{equation}

Assuming that this procedure is justified.
\Eqnref{eqn:sch_current:intermediate} therefore can be written
in terms of a probability current very reminiscent of the current density vector of electrodynamics

\begin{equation}\label{eqn:sch_current:pcons}
\begin{aligned}
\BJ &= \frac{\Hbar}{2mi} \left( \psi^\conj \spacegrad \psi - \psi \spacegrad \psi^\conj \right) \\
0 &= \PD{t}{\rho} + \spacegrad \cdot \BJ
\end{aligned}
\end{equation}

Regarding justification, this should be revisited.
It appears to give the right answer, despite the fact that \(\psi\) is a complex (mixed grade) object, which
likely has some additional significance.

\section{}

Now, having calculated the probability conservation \eqnref{eqn:sch_current:pcons}, it is interesting to
note the similarity to the relativistic spacetime divergence from Maxwell's equation.

We can write
\begin{equation}\label{eqn:schCurrent:100}
\begin{aligned}
0 = \PD{t}{\rho} + \spacegrad \cdot \BJ &= \grad \cdot \left( c\rho \gamma_0 + \BJ \gamma_0 \right)
\end{aligned}
\end{equation}

and form something that has the appearance of a relativistic four vector, re-writing the conservation equation as

\begin{equation}\label{eqn:schCurrent:120}
\begin{aligned}
J &= c\rho \gamma_0 + \BJ \gamma_0 \\
0 &= \grad \cdot J
\end{aligned}
\end{equation}

Expanding this four component vector shows an interesting form:

\begin{equation}\label{eqn:schCurrent:140}
\begin{aligned}
J &= c \rho \gamma_0 +
\frac{\Hbar}{2mi} \left( \psi^\conj \spacegrad \psi - \psi \spacegrad \psi^\conj \right) \gamma_0 \\
\end{aligned}
\end{equation}

Now, if one assumes the wave function can be represented as a even grade object with the following complex
structure
\begin{equation}\label{eqn:schCurrent:160}
\begin{aligned}
\psi &= \alpha + \gamma^m \wedge \gamma^n \beta_{mn}
\end{aligned}
\end{equation}

then \(\gamma_0\) will commute with \(\psi\).  Noting that \(\spacegrad \gamma_0 = \sum_k \gamma_k \partial_k = -\gamma^k \partial_k\), we have

\begin{equation}\label{eqn:schCurrent:180}
\begin{aligned}
m J &= m c \psi^\conj \psi \gamma_0 +
\frac{i\Hbar}{2} \left( \psi^\conj \gamma^k \partial_k \psi - \psi \gamma^k \partial_k \psi^\conj \right)
\end{aligned}
\end{equation}

Now, this is an interesting form.  In particular compare this to the Dirac Lagrangian, as given in
the \href{http://en.wikipedia.org/wiki/Dirac_equation#Adjoint_equation_and_Dirac_current}{wikipedia Dirac equation} article.

\begin{equation}\label{eqn:schCurrent:200}
\begin{aligned}
L = mc \overbar{\psi}\psi - \frac{i\Hbar}{2}(\overbar{\psi}\gamma^\mu (\partial_\mu\psi) - (\partial_\mu\overbar{\psi})\gamma^\mu \psi)
\end{aligned}
\end{equation}

Although the Schr\"{o}dinger equation is a non-relativistic equation, it appears that the probability current,
when we add the \(\gamma^0 \partial_0\) term required to put this into a covariant form, is in fact the Lagrangian density
for the Dirac equation (when scaled by mass).

I do not know enough yet about QM to see what exactly the implications of this are, but I suspect that there is something
of some interesting significance to this particular observation.

\section{On the grades of the QM complex numbers}

To get to \eqnref{eqn:sch_current:intermediate}, no assumptions about the representation of the field variable \(\psi\) were
required.  However, to make the identification

\begin{equation}\label{eqn:schCurrent:220}
\begin{aligned}
\psi^\conj \spacegrad^2 \psi - \psi \spacegrad^2 \psi^\conj
&= \spacegrad \cdot \left( \psi^\conj \spacegrad^2 \psi - \psi \spacegrad^2 \psi^\conj \right)
\end{aligned}
\end{equation}

we need some knowledge or assumptions about the representation.  The assumption made initially was that we could treat
\(\psi\) as a scalar, but then we later see there is value trying to switch to the Dirac representation (which appears
to be the logical way to relativistically extend the probability current).

For example, with a geometric algebra multivector representation we have many ways to construct complex quantities.  Assuming a
Euclidean basis we can construct a complex number we can factor out one of the basis vectors

\begin{equation}\label{eqn:schCurrent:240}
\begin{aligned}
\sigma_1 x_1 + \sigma_2 x_2 = \sigma_1 ( x_1 + \sigma_1 \sigma_2 x_2 )
\end{aligned}
\end{equation}

However, this is not going to commute with vectors (ie: such as the gradient), unless that vector is perpendicular to the
plane spanned by this vector.  As an example

\begin{equation}\label{eqn:schCurrent:260}
\begin{aligned}
i = \sigma_1 \sigma_2
\end{aligned}
\end{equation}

\begin{equation}\label{eqn:schCurrent:280}
\begin{aligned}
i \sigma_1 &= -\sigma_1 i \\
i \sigma_2 &= -\sigma_2 i \\
i \sigma_3 &=  \sigma_3 i
\end{aligned}
\end{equation}

What would work is a complex representation using the \R{3} pseudoscalar (aka the Dirac pseudoscalar).

\begin{equation}\label{eqn:schCurrent:300}
\begin{aligned}
\psi = \alpha + \sigma_1 \sigma_2 \sigma_3 \beta = \alpha + \gamma_0 \gamma_1 \gamma_2 \gamma_3 \beta
\end{aligned}
\end{equation}

   %
% Copyright � 2012 Peeter Joot.  All Rights Reserved.
% Licenced as described in the file LICENSE under the root directory of this GIT repository.
%

%
%
\mychapter{Dirac Lagrangian}
\index{Dirac Lagrangian}
\label{chap:diracLagrangian}
%\date{Dec 21, 2008.  diracLagrangian.tex}

\section{Dirac Lagrangian with Feynman slash notation}

Wikipedia's \href{https://en.wikipedia.org/wiki/Lagrangian#Dirac_Lagrangian}{Dirac Lagrangian} entry lists the Lagrangian as

\begin{equation}\label{eqn:diracLagrangian:24}
\begin{aligned}
\LL = \overbar{\psi} (i \Hbar c \Dslash - mc^2) \psi
\end{aligned}
\end{equation}

"where \(\psi\!\) is a Dirac spinor, \(\overbar{\psi} = \psi^\dagger \gamma^0\) is its Dirac adjoint, \(D\!\) is the gauge covariant derivative, and \(\Dslash\) is Feynman slash notation|Feynman notation for \(\gamma^\sigma D_\sigma\!\)."

Let us decode this.  First, what is \(D_\sigma\)?

From \href{https://en.wikipedia.org/wiki/Gauge_covariant_derivative}{Gauge theory}

\begin{equation}\label{eqn:diracLagrangian:44}
\begin{aligned}
D_\mu := \partial_\mu - i e A_\mu
\end{aligned}
\end{equation}

where \(A_\mu\) is the electromagnetic vector potential.

So, in four-vector notation we have

\begin{equation}\label{eqn:diracLagrangian:64}
\begin{aligned}
\Dslash
&= \gamma^\mu \partial_\mu - i e \gamma^\mu A_\mu \\
&= \grad - i e A \\
\end{aligned}
\end{equation}

So our Lagrangian written out in full is left as

\begin{equation}\label{eqn:diracLag:lag1}
\begin{aligned}
\LL = \psi^\dagger \gamma^0 ( i \Hbar c \grad + \Hbar c e A - mc^2) \psi
\end{aligned}
\end{equation}

How about this \(\gamma^0 i \grad\) term?  If we assume that \(i = \gamma_0 \gamma_1 \gamma_2 \gamma_3\) is the four space pseudoscalar, then this is

\begin{equation}\label{eqn:diracLagrangian:84}
\begin{aligned}
\gamma^0 i \grad
&= - i \gamma^0 (\gamma^0 \partial_0 + \gamma^i \partial_i) \\
&= - i (\partial_0 + \sigma_i \partial_i) \\
\end{aligned}
\end{equation}

So, operationally, we have the dual of a quaternion like gradient operator.  If \(\psi\) is an even grade object, as I had guess can be implied by
the term spinor, then there is some sense to requiring a gradient operation that has scalar and spacetime bivector components.

Let us write this

\begin{equation}\label{eqn:diracLagrangian:104}
\begin{aligned}
\gamma^0 \grad &= \partial_0 + \sigma_i \partial_i = {\grad}_{0,2}
\end{aligned}
\end{equation}

Now, how about the meaning of \(\overbar{\psi} = \psi^\dagger \gamma^0\)?  I initially assumed that \(\psi^\dagger\) was the reverse operation.
However, looking in the quantum treatment of \citep{doran2003gap} and their earlier relativity content, I see that they explicitly avoid dagger as a reverse in a relativistic context since it is used for ``something-else'' in a quantum context.  It appears that their mapping from matrix algebra to Clifford
algebra is

\begin{equation}\label{eqn:diracLagrangian:124}
\begin{aligned}
\psi^\dagger \equiv \gamma_0 \tilde{\psi} \gamma_0,
\end{aligned}
\end{equation}

where tilde is used for the reverse operation.

This then implies that

\begin{equation}\label{eqn:diracLagrangian:144}
\begin{aligned}
\overbar{\psi} = \psi^\dagger \gamma^0 = \gamma_0 \tilde{\psi}
\end{aligned}
\end{equation}

We now have an expression of the Lagrangian in full in terms of geometric objects

\begin{equation}\label{eqn:diracLag:lag2}
\begin{aligned}
\LL = \gamma_0 \tilde{\psi} ( i \Hbar c \grad + \Hbar c e A - mc^2) \psi.
\end{aligned}
\end{equation}

Assuming that this is now the correct geometric interpretation of the Lagrangian, why bother having that first \(\gamma_0\) factor.  It should not change the field equations (just as a constant factor should not make a difference).  It seems more natural to instead write the Lagrangian as just

\begin{equation}\label{eqn:diracLag:lag3}
\begin{aligned}
\LL = \tilde{\psi} \left( i \grad + e A - \frac{mc}{\Hbar} \right) \psi,
\end{aligned}
\end{equation}

where both the constant vector factor \(\gamma_0\), the redundant common factor of \(c\) have been removed, and we divide throughout by \(\Hbar\) to tidy up a bit.  Perhaps this tidy up should be omitted since it sacrifices the
energy dimensionality of the original.

\subsection{Dirac adjoint field}

%Now, it is somewhat interesting to note that treating \(\psi^\dagger\) as a reverse operation resulted, after calculation of the field equations, in the quantity \(\gamma_0 \psi^\dagger \gamma_0\) showing up as a significant quanity.  It was almost like the math was auto-correcting itself attempting to show that the this was the real quantity of interest.

The reverse sandwich operation of \(\gamma_0 \tilde{\psi} \gamma_0\) to produce the Dirac adjoint field from \(\psi\)
can be recognized
as very similar to the mechanism used to split the Faraday bivector for the electromagnetic field into electric and magnetic terms.  There addition and subtraction of the sandwich'ed fields with the original acted as a spacetime split operation, producing separate electric field spacetime (Pauli) bivectors and pure spatial bivectors (magnetic components) from the total field.  Here we have a
quaternion like field variable with scalar and bivector terms.  Is there a physical (observables) significance
only for a subset of the six possible bivectors that make up the spinor field?
%only in the
%spacetime bivector parts (ie: Pauli matrix components) of the six possible bivectors,
If so, then this adjoint operation can be used as a filter to select only the desired components.

%Observe that we have an almost recognizable term in the effective field variable \(\gamma_0 \psi \gamma_0\) of
%\eqnref{eqn:diracLag:antimatter}.  In particular recall that the electric field and magnetic field components of the faraday bivector can be obtained from a
%\(\gamma^0 F \gamma_0\) spacetime split.
Recall that the Faraday bivector is

\begin{equation}\label{eqn:diracLagrangian:164}
\begin{aligned}
F
&= \BE + i c \BB \\
&= E^j \sigma_j + i c B^j \sigma_j \\
&= E^j \gamma_j \gamma_0 + i c B^j \gamma_j \gamma_0 \\
\end{aligned}
\end{equation}

So we have

\begin{equation}\label{eqn:diracLagrangian:184}
\begin{aligned}
\gamma_0 F \gamma_0
&= E^j \gamma_0 \gamma_j + \gamma_0 i c B^j \gamma_j \\
&= -E^j \sigma_j + i c B^j \sigma_j \\
&= -\BE + i c \BB
\end{aligned}
\end{equation}

So we have

\begin{equation}\label{eqn:diracLagrangian:204}
\begin{aligned}
\inv{2} \left(F - \gamma_0 F \gamma_0 \right) &= \BE \\
\inv{2i} \left(F + \gamma_0 F \gamma_0 \right) &= c \BB
\end{aligned}
\end{equation}

How does this sandwich operation act on other grade objects?

\begin{itemize}
\item scalar

\begin{equation}\label{eqn:diracLagrangian:224}
\begin{aligned}
\gamma_0 \alpha \gamma_0 = \alpha
\end{aligned}
\end{equation}

\item vector

\begin{equation}\label{eqn:diracLagrangian:244}
\begin{aligned}
\gamma_0 \gamma_\mu \gamma_0
&= \left(2 \gamma_0 \cdot \gamma_\mu - \gamma_\mu \gamma_0\right) \gamma_0 \\
&= 2 (\gamma_0 \cdot \gamma_\mu) \gamma_0 - \gamma_\mu \\
&=
\left\{
\begin{array}{l l}
\gamma_0 & \quad \mbox{if \(\mu = 0\)} \\
-\gamma_i & \quad \mbox{if \(\mu = i \ne 0\)} \\
\end{array} \right.
\end{aligned}
\end{equation}

\item trivector

For the duals of the vectors we have the opposite split, where for the dual of \(\gamma_0\) we have a sign
toggle

\begin{equation}\label{eqn:diracLagrangian:264}
\begin{aligned}
\gamma_0 \gamma_i \gamma_j \gamma_k \gamma_0 = -\gamma_i \gamma_j \gamma_k
\end{aligned}
\end{equation}

whereas for the duals of \(\gamma_k\) we have invariant sign under sandwich
\begin{equation}\label{eqn:diracLagrangian:284}
\begin{aligned}
\gamma_0 \gamma_i \gamma_j \gamma_0 \gamma_0 = \gamma_i \gamma_j \gamma_0
\end{aligned}
\end{equation}

\item pseudoscalar

\begin{equation}\label{eqn:diracLagrangian:304}
\begin{aligned}
\gamma_0 i \gamma_0
&= \gamma_0 \gamma_0 \gamma_1 \gamma_2 \gamma_3 \gamma_0 \\
&= -i
\end{aligned}
\end{equation}
\end{itemize}

Ah ha!  Recalling the conjugation results from \chapcite{PJDiracGamma}, one sees that this sandwich operation is in fact just the equivalent of the conjugate operation on Dirac matrix algebra elements.  So we can write

\begin{equation}\label{eqn:diracLagrangian:324}
\begin{aligned}
\psi^\conj \equiv \gamma_0 \psi \gamma_0
\end{aligned}
\end{equation}

and can thus identify \(\gamma_0 \tilde{\psi} \gamma_0 = \psi^\dagger\) as the reverse of that conjugate quantity.  That is

\begin{equation}\label{eqn:diracLagrangian:344}
\begin{aligned}
\psi^\dagger = (\psi^\conj)^{\tilde{}}
\end{aligned}
\end{equation}

This does not really help identify the significance of this term but this identification may prove useful later.

\subsection{Field equations}

Now, how to recover the field equation from \eqnref{eqn:diracLag:lag3}
%or \eqnref{eqn:diracLag:lag2}
?  If one assumes that the Euler-Lagrange field equations

\begin{equation}\label{eqn:diracLagrangian:364}
\begin{aligned}
\PD{\eta}{\LL} - \partial_\mu \PD{(\partial_\mu \eta)}{\LL} = 0
\end{aligned}
\end{equation}

hold for these even grade field variables \(\psi\), then treating \(\psi\) and \(\overbar{\psi}\) as separate field variables one has for the reversed field variable

\begin{equation}\label{eqn:diracLagrangian:384}
\begin{aligned}
\PD{\tilde{\psi}}{\LL} - \partial_\mu \PD{(\partial_\mu \tilde{\psi})}{\LL} &= 0 \\
\left( i \grad + e A - \frac{mc}{\Hbar}\right) \psi - (0) &= 0
\end{aligned}
\end{equation}

Or
\begin{equation}\label{eqn:diracLagrangian:404}
\begin{aligned}
\Hbar (i \grad + e A) \psi = mc \psi
\end{aligned}
\end{equation}

Except for the additional \(e A\) term here, this is the Dirac equation that we get from taking square roots of the Klein-Gordon equation.  Should \(A\) be considered a field variable?  More likely is that this is a supplied potential as in the \(V\) of the non-relativistic
Schr\"{o}dinger's equation.

Being so loose with the math here (ie: taking partials with respect to non-scalar variables) is somewhat disturbing but developing some intuition is worthwhile before getting the details down.

\subsection{Conjugate field equation}

Our Lagrangian is not at all symmetric looking, having derivatives of \(\psi\), but not \(\overbar{\psi}\).  Compare this to the Lagrangians for the
Schr\"{o}dinger's equation, and Klein-Gordon equation respectively, which are

\begin{equation}\label{eqn:diracLagrangian:424}
\begin{aligned}
\LL &= \frac{\Hbar^2}{2m}
(\spacegrad \psi) \cdot (\spacegrad \psi^\conj) + V \psi \psi^\conj + {i \Hbar} \left( \psi \partial_t \psi^\conj - \psi^\conj \partial_t \psi \right) \\
\LL &= -\partial^\nu \psi \partial_\nu \psi^\conj + \frac{m^2 c^2}{\Hbar^2} \psi \psi^\conj.
\end{aligned}
\end{equation}

With these Lagrangians one gets the field equation for \(\psi\), differentiating with respect to the conjugate field \(\psi^\conj\), and the conjugate equation with differentiation with respect to \(\psi\) (where \(\psi\) and \(\psi^\conj\) are treated as independent field variables).

It is not obvious that evaluating the Euler-Lagrange equations will produce a similarly regular result, so
%we would get a similarly nice symmetric reversal result with respect to \(\psi\) and \(\tilde{\psi}\).
let us compute the derivatives with respect to the \(\psi\) field variables to compute the equations for \(\overbar{\psi}\) or \(\tilde{\psi}\) to see what results.  Written out in coordinates so that we can apply the Euler-Lagrange equations, our Lagrangian (with \(A\) terms omitted) is

\begin{equation}\label{eqn:diracLag:lag4}
\begin{aligned}
\LL = \tilde{\psi}\left( i \gamma^\mu \partial_\mu + e A - \frac{m c}{\Hbar} \right) \psi
\end{aligned}
\end{equation}

Again abusing the Euler Lagrange equations, ignoring the possible issues with commuting partials taken
with respect to spinors (not scalar variables), blinding plugging into the formulas we have

\begin{equation}\label{eqn:diracLagrangian:444}
\begin{aligned}
\PD{\psi}{\LL} &= \partial_\mu \PD{\partial_\mu \psi}{\LL} \\
\tilde{\psi}\left( e A - \frac{m c}{\Hbar} \right) &= \partial_\mu \left(\tilde{\psi} i \gamma^\mu \right)
\end{aligned}
\end{equation}

reversing this entire equation we have

\begin{equation}\label{eqn:diracLagrangian:464}
\begin{aligned}
\left( e A - \frac{m c}{\Hbar} \right) \psi &= \gamma^\mu i \partial_\mu \psi = - i \grad \psi
\end{aligned}
\end{equation}

Or
\begin{equation}\label{eqn:diracLagrangian:484}
\begin{aligned}
\Hbar \left( i \grad + e A \right) \psi = m c \psi
\end{aligned}
\end{equation}

So we do in fact get the same field equation regardless of which of the two field variables one differentiates with.  That is not obvious looking at the Lagrangian.

\section{Alternate Dirac Lagrangian with antisymmetric terms}

Now, the wikipedia article
\href{https://en.wikipedia.org/wiki/Dirac_equation#Adjoint_equation_and_Dirac_current}{Adjoint equation and Dirac current} lists the Lagrangian as

\begin{equation}\label{eqn:diracLagrangian:504}
\begin{aligned}
\LL = mc \overbar{\psi}\psi - {\inv{2}i\Hbar}(\overbar{\psi}\gamma^\mu (\partial_\mu\psi) - (\partial_\mu\overbar{\psi})\gamma^\mu \psi)
\end{aligned}
\end{equation}

Computing the Euler Lagrange equations for this potential free Lagrangian we have

\begin{equation}\label{eqn:diracLagrangian:524}
\begin{aligned}
m c \psi - \inv{2} i \Hbar \gamma^\mu \partial_\mu \psi = \partial_\mu \left( \inv{2} i \Hbar \gamma^\mu \psi \right)
\end{aligned}
\end{equation}

Or,
\begin{equation}\label{eqn:diracLagrangian:544}
\begin{aligned}
m c \psi = i \Hbar \grad \psi
\end{aligned}
\end{equation}

And the same computation, treating \(\overbar{\psi}\) as the independent field variable of interest we have

\begin{equation}\label{eqn:diracLagrangian:564}
\begin{aligned}
m c \psi + \inv{2} i \Hbar \partial_\mu \overbar{\psi} \gamma^\mu = -\inv{2} i \Hbar \partial_\mu \overbar{\psi} \gamma^\mu
\end{aligned}
\end{equation}

which is

\begin{equation}\label{eqn:diracLagrangian:584}
\begin{aligned}
m c \overbar{\psi} &= - i \Hbar \partial_\mu \overbar{\psi} \gamma^\mu \\
m c \gamma_0 \tilde{\psi} &= - i \Hbar \partial_\mu \gamma_0 \tilde{\psi} \gamma^\mu \\
m c \tilde{\psi} &= i \Hbar \partial_\mu \tilde{\psi} \gamma^\mu \\
m c \psi &= \Hbar \grad \psi i \\
\end{aligned}
\end{equation}

Or,
\begin{equation}\label{eqn:diracLagrangian:604}
\begin{aligned}
i \Hbar \grad \psi &= -m c \psi
\end{aligned}
\end{equation}

FIXME: This differs in sign from the same calculation with the Lagrangian of \eqnref{eqn:diracLag:lag4}.  Based on
the possibility of both roots in the Klein-Gordon equation, I suspect I have made a sign error in the first
calculation.

\section{Appendix}

\subsection{Pseudoscalar reversal}
\index{pseudoscalar}

The pseudoscalar reverses to itself

\begin{equation}\label{eqn:diracLagrangian:624}
\begin{aligned}
\tilde{i}
&= \gamma_{3210} \\
&= -\gamma_{2103} \\
&= -\gamma_{1023} \\
&= \gamma_{0123} \\
&= i,
\end{aligned}
\end{equation}

\subsection{Form of the spinor}

The specific structure of the spinor has not been defined here.  It has been assumed to be quaternion like,
and contain only even grades, but in the Dirac/Minkowski algebra that gives us two possibilities

\begin{equation}\label{eqn:diracLagrangian:644}
\begin{aligned}
\psi
&= \alpha + P^a \gamma_a \gamma_0 \\
&= \alpha + P^a \sigma_a
\end{aligned}
\end{equation}

Or
\begin{equation}\label{eqn:diracLagrangian:664}
\begin{aligned}
\psi
&= \alpha + P^c \gamma_a \wedge \gamma_b \\
&= \alpha - P^c \sigma_a \wedge \sigma_b \\
&= \alpha - i \epsilon_{a b c} P^c \sigma_c \\
\end{aligned}
\end{equation}

Spinors in Doran/Lasenby appear to use the latter form of dual Pauli vectors (wedge products of the Pauli spatial basis elements).  This actually makes sense since one wants a spatial bivector for rotation (ie: ``spin''), and not the spacetime bivectors, which provide a Lorentz boost action.

   %
% Copyright � 2012 Peeter Joot.  All Rights Reserved.
% Licenced as described in the file LICENSE under the root directory of this GIT repository.
%

%
%
%\input{../peeter_prologue.tex}

%\mychapter{Geometric Algebra equivalants for Pauli Matrices}
\mychapter{Pauli Matrices}
\index{Pauli matrices}
\label{chap:pauliMatrix}
% does not work with underscore.
%%\blogpage{http://sites.google.com/site/peeterjoot/math/pauli_matrix.pdf}
%\date{Dec 06, 2008.  pauliMatrix.tex}
%\date{Dec 06, 2008}
%%\revisionInfo{\(RCSfile: pauliMatrix.tex,v \) Last \(Revision: 1.29 \) \(Date: 2009/08/01 22:12:18 \)}

\beginArtWithToc

\section{Motivation}

Having learned Geometric (Clifford) Algebra from \citep{doran2003gap}, \citep{hestenes1999nfc}, \citep{dorst2007gac}, and other sources before studying any quantum mechanics, trying to work with (and talk to people familiar with) the Pauli and Dirac matrix notation as used in traditional quantum mechanics becomes difficult.

The aim of these notes is to work through equivalents to many Clifford algebra expressions entirely in commutator and anticommutator notations.  This will show the mapping between the (generalized) dot product and the wedge product, and also show how the different grade elements of the Clifford algebra \(\Clifford{3}{0}\) manifest in their matrix forms.

\section{Pauli Matrices}

The matrices in question are:
%
\begin{equation}\label{eqn:pauliMatrix:20}
\begin{aligned}
\sigma_1 &= \PauliX \\
\sigma_2 &= \PauliY \\
\sigma_3 &= \PauliZ
\end{aligned}
\end{equation}
%
These all have positive square as do the traditional Euclidean unit vectors \(\Be_i\), and so can be used algebraically as a vector basis for \R{3}.  So any vector that we can write in coordinates
%
%
\begin{equation}\label{eqn:pauliMatrix:40}
\begin{aligned}
\Bx = x^i \Be_i,
\end{aligned}
\end{equation}
%
we can equivalently write (an isomorphism) in terms of the Pauli matrix's
%
\begin{equation}\label{eqn:pauliMat:vectorInPauliBasis}
\begin{aligned}
x = x^i \sigma_i.
\end{aligned}
\end{equation}
%
\subsection{Pauli Vector}
\index{Pauli vector}
%Pauli Matrix article
\citep{wiki:pauli} introduces the Pauli vector as a mechanism for mapping between a vector basis and this matrix basis
%
\begin{equation}\label{eqn:pauliMatrix:60}
\begin{aligned}
\Bsigma = \sum \sigma_i \Be_i
\end{aligned}
\end{equation}
%
This is a curious looking construct with products of \(2 x 2\) matrices and \R{3} vectors.  Obviously these are not the usual \(3 x 1\) column vector representations.  This Pauli vector is thus really a notational construct.  If one takes the dot product of a vector expressed using the standard orthonormal Euclidean basis \(\{\Be_i\}\) basis, and then takes the dot product with the Pauli matrix in a mechanical fashion
%
\begin{equation}\label{eqn:pauliMatrix:80}
\begin{aligned}
\Bx \cdot \Bsigma &=
(x^i \Be_i) \cdot \sum \sigma_j \Be_j \\
&= \sum_{i,j} x^i \sigma_j \Be_i \cdot \Be_j \\
&= x^i \sigma_i \\
\end{aligned}
\end{equation}
%
one arrives at the matrix representation of the vector in the Pauli basis \(\{\sigma_i\}\).  Does this construct have any value?  That I do not know, but for the rest of these notes the coordinate representation as in equation \eqnref{eqn:pauliMat:vectorInPauliBasis} will be used directly.

\subsection{Matrix squares}

It was stated that the Pauli matrices have unit square.  Direct calculation of this is straightforward, and confirms the assertion
%
\begin{equation}\label{eqn:pauliMatrix:100}
\begin{aligned}
{\sigma_1}^2 &= \PauliX \PauliX = \PauliI = I \\
{\sigma_2}^2 &= \PauliY \PauliY = i^2 \PauliYNoI \PauliYNoI = \PauliI = I \\
{\sigma_3}^2 &= \PauliZ \PauliZ = \PauliI = I \\
\end{aligned}
\end{equation}
%
Note that unlike the vector (Clifford) square the identity matrix and not a scalar.

\subsection{Length}
If we are to operate with Pauli matrices how do we express our most basic vector operation, the length?

Examining a vector lying along one direction, say, \(\Ba = \alpha\xcap\) we expect
%
\begin{equation}\label{eqn:pauliMatrix:120}
\begin{aligned}
\Ba^2 = \Ba \cdot \Ba = \alpha^2 \xcap \cdot \xcap = \alpha^2.
\end{aligned}
\end{equation}
%
Lets contrast this to the Pauli square for the same vector \(y = \alpha\sigma_1\)
%
\begin{equation}\label{eqn:pauliMatrix:140}
\begin{aligned}
y^2 = \alpha^2 {\sigma_1}^2 = \alpha^2 I
\end{aligned}
\end{equation}
%
The wiki article mentions trace, but no application for it.  Since \(\traceB{I} = 2\), an observable application is that the trace operator provides a mechanism to convert a diagonal matrix to a scalar.  In particular for this scaled unit vector \(y\) we have
%
\begin{equation}\label{eqn:pauliMatrix:160}
\begin{aligned}
\alpha^2 = \inv{2} \traceB{y^2}
\end{aligned}
\end{equation}
%
It is plausible to guess that the squared length will be related to the matrix square in the general case as well
%
\begin{equation}\label{eqn:pauliMatrix:180}
\begin{aligned}
\Abs{x}^2 = \inv{2}\traceB{x^2}
\end{aligned}
\end{equation}
%
Let us see if this works by performing the coordinate expansion
%
\begin{equation}\label{eqn:pauliMatrix:200}
\begin{aligned}
x^2
&= (x^i \sigma_i)(x^j \sigma_j) \\
&= x^i x^j \sigma_i \sigma_j \\
\end{aligned}
\end{equation}
%
A split into equal and different indices thus leaves
%
\begin{equation}\label{eqn:pauliMat:square}
\begin{aligned}
x^2
&= \sum_{i < j} x^i x^j (\sigma_i \sigma_j + \sigma_j \sigma_i) + \sum_i (x^i)^2 {\sigma_i}^2
\end{aligned}
\end{equation}
%
As an algebra that is isomorphic to the Clifford Algebra \(\Clifford{3}{0}\) it is expected that the \(\sigma_i \sigma_j\) matrices anticommute for \(i \ne j\).  Multiplying these out verifies this
%
\begin{equation}\label{eqn:pauliMatrix:220}
\begin{aligned}
\begin{array}{l l l l}
\sigma_1 \sigma_2 &= i \PauliX \PauliYNoI    &= i \PauliZ                                  &=  i \sigma_3 \\
\sigma_2 \sigma_1 &= i \PauliYNoI \PauliX    &= i \begin{bmatrix}-1 & 0 \\0 & 1\end{bmatrix} &= -i \sigma_3 \\
\sigma_3 \sigma_1 &= \PauliZ \PauliX         &=   \begin{bmatrix}0 & 1 \\-1 & 0\end{bmatrix} &=  i \sigma_2 \\
\sigma_1 \sigma_3 &= \PauliX \PauliZ         &=   \begin{bmatrix}0 & -1 \\1 & 0\end{bmatrix} &= -i \sigma_2 \\
\sigma_2 \sigma_3 &= i \PauliYNoI \PauliZ    &= i \PauliX                                  &=  i \sigma_1 \\
\sigma_3 \sigma_2 &= i \PauliZ \PauliYNoI    &= i \begin{bmatrix}0 & -1 \\-1 & 0\end{bmatrix} &= -i \sigma_3 \\
\end{array}.
\end{aligned}
\end{equation}
%
Thus in \eqnref{eqn:pauliMat:square} the sum over the \(\{i < j\} = \{12, 23, 13\}\) indices is zero.

Having computed this, our vector square leaves us with the vector length multiplied by the identity matrix
%
\begin{equation}\label{eqn:pauliMatrix:240}
\begin{aligned}
x^2 = \sum_i (x^i)^2 I.
\end{aligned}
\end{equation}
%
Invoking the trace operator will therefore extract just the scalar length desired
%
\begin{equation}\label{eqn:pauliMatrix:260}
\begin{aligned}
\Abs{x}^2 = \inv{2} \traceB{x^2} = \sum_i (x^i)^2.
\end{aligned}
\end{equation}
%
\subsubsection{Aside: Summarizing the multiplication table}

It is worth pointing out that the multiplication table above used to confirm the antisymmetric behavior of the Pauli basis can be summarized as
%
\begin{equation}\label{eqn:pauliMatrix:280}
\begin{aligned}
\sigma_a \sigma_b = 2 i \epsilon_{abc} \sigma_c
\end{aligned}
\end{equation}
%
\subsection{Scalar product}
\index{scalar product!Pauli matrix}

Having found the expression for the length of a vector in the Pauli basis, the next logical desirable identity is the dot product.  One can guess that this will be the trace of a scaled symmetric product, but can motivate this without guessing in the usual fashion, by calculating the length of an orthonormal sum.

Consider first the length of a general vector sum.  To calculate this we first wish to calculate the matrix square of this sum.
%
\begin{equation}\label{eqn:pauliMatrix:300}
\begin{aligned}
(x + y)^2 &= x^2 + y^2 + x y + y x \\
\end{aligned}
\end{equation}
%
If these vectors are perpendicular this equals \(x^2 + y^2\).  Thus orthonormality implies that
\begin{equation}\label{eqn:pauliMatrix:320}
\begin{aligned}
x y + y x &= 0 \\
\end{aligned}
\end{equation}
%
or,
%
\begin{equation}\label{eqn:pauliMatrix:340}
\begin{aligned}
y x &= - x y
\end{aligned}
\end{equation}
%
We have already observed this by direct calculation for the Pauli matrices themselves.  Now, this is not any different than the usual description of perpendicularity in a Clifford Algebra, and it is notable that there are not any references to matrices in this argument.  One only requires that a well defined vector product exists, where the squared vector has a length interpretation.

One matrix dependent observation that can be made is that since the left hand side and the \(x^2\), and \(y^2\) terms are all diagonal, this symmetric sum must also be diagonal.  Additionally, for the length of this vector sum we then have
%
\begin{equation}\label{eqn:pauliMatrix:360}
\begin{aligned}
{\Abs{x + y}}^2
%&= x^2 + y^2 + x y + y x \\
&= \Abs{x}^2 + \Abs{y}^2 + \inv{2}\traceB{x y + y x} \\
\end{aligned}
\end{equation}
%
For correspondence with the Euclidean dot product of two vectors we must then have
%
\begin{equation}\label{eqn:pauliMatrix:380}
\begin{aligned}
\scalarProduct{x}{y} &= \inv{4}\traceB{x y + y x}.
\end{aligned}
\end{equation}
%
Here \(x \bullet y\) has been used to denote this scalar product (ie: a plain old number), since \(x \cdot y\) will be used later for a matrix dot product (this times the identity matrix) which is more natural in many ways for this Pauli algebra.

Observe the symmetric product that is found embedded in this scalar selection operation.  In physics this is known as the anticommutator, where the commutator is the antisymmetric sum.  In the physics notation the anticommutator (symmetric sum) is
%
\begin{equation}\label{eqn:pauliMat:anticommutator}
\begin{aligned}
\symmetric{x}{y} &= x y + y x
\end{aligned}
\end{equation}
%
So this scalar selection can be written
%
\begin{equation}\label{eqn:pauliMatrix:400}
\begin{aligned}
\scalarProduct{ x}{ y} = \inv{4}\trace{\symmetric{x}{y}}
\end{aligned}
\end{equation}
%
Similarly, the commutator, an antisymmetric product, is denoted:
%
\begin{equation}\label{eqn:pauliMat:commutator}
\begin{aligned}
\antisymmetric{x}{y} &= x y - y x,
\end{aligned}
\end{equation}
%
A close relationship between this commutator and the wedge product of Clifford Algebra is expected.

\subsection{Symmetric and antisymmetric split}
\index{symmetric sum!Pauli matrix}
\index{antisymmetric sum!Pauli matrix}

As with the Clifford product, the symmetric and antisymmetric split of a vector product is a useful concept.  This can be used to write the product of two Pauli basis vectors in terms of the anticommutator and commutator products
%
\begin{equation}\label{eqn:pauliMatrix:420}
\begin{aligned}
x y &= \inv{2} \symmetric{x}{y} + \inv{2} \antisymmetric{x}{y} \\
y x &= \inv{2} \symmetric{x}{y} - \inv{2} \antisymmetric{x}{y}
\end{aligned}
\end{equation}
%
These follows from the definition of the anticommutator \eqnref{eqn:pauliMat:anticommutator} and commutator \eqnref{eqn:pauliMat:commutator} products above, and are the equivalents of the Clifford symmetric and antisymmetric split into dot and wedge products
%
\begin{equation}\label{eqn:pauliMatrix:440}
\begin{aligned}
x y &= {x} \cdot {y} + {x} \wedge {y} \\
y x &= {x} \cdot {y} - {x} \wedge {y}
\end{aligned}
\end{equation}
%
Where the dot and wedge products are respectively
%
\begin{equation}\label{eqn:pauliMatrix:460}
\begin{aligned}
x \cdot y &= \inv{2}(x y + y x) \\
x \wedge y &= \inv{2}(x y - y x)
\end{aligned}
\end{equation}
%
Note the factor of two differences in the two algebraic notations.  In particular very handy Clifford vector product reversal formula
%
\begin{equation}\label{eqn:pauliMatrix:480}
\begin{aligned}
y x = - x y + 2 x \cdot y
\end{aligned}
\end{equation}
%
has no factor of two in its Pauli anticommutator equivalent
%
\begin{equation}\label{eqn:pauliMatrix:500}
\begin{aligned}
y x = - x y + \symmetric{x}{y}
\end{aligned}
\end{equation}
%
\subsection{Vector inverse}

It has been observed that the square of a vector is diagonal in this matrix representation, and can therefore be inverted for any non-zero vector
%
\begin{equation}\label{eqn:pauliMatrix:520}
\begin{aligned}
x^2 &= \Abs{x}^2 I \\
(x^2)^{-1} &= \Abs{x}^{-2} I \\
\implies \\
x^2 (x^2)^{-1} &= I \\
\end{aligned}
\end{equation}
%
So it is therefore quite justifiable to define
\begin{equation}\label{eqn:pauliMatrix:540}
\begin{aligned}
x^{-2} = \inv{x^2} \equiv \Abs{x}^{-2} I \\
\end{aligned}
\end{equation}
%
This allows for the construction of a dual sided vector inverse operation.
%
\begin{equation}\label{eqn:pauliMatrix:560}
\begin{aligned}
x^{-1}
&\equiv \inv{\Abs{x}^2} x \\
&= \inv{x^2} x \\
&= x \inv{x^2} \\
\end{aligned}
\end{equation}
%
This inverse is a scaled version of the vector itself.

The diagonality of the squared matrix or the inverse of that allows for commutation with x.  This diagonality plays the same role as the scalar in a regular Clifford square.  In either case the square can commute with the vector, and that commutation allows the inverse to have both left and right sided action.

Note that like the Clifford vector inverse when the vector is multiplied with this inverse, the product resides outside of the proper \R{3} Pauli basis since the identity matrix is required.

\subsection{Coordinate extraction}

Given a vector in the Pauli basis, we can extract the coordinates using the scalar product
%
\begin{equation}\label{eqn:pauliMatrix:580}
\begin{aligned}
x = \sum_i \inv{4}\trace{\symmetric{x}{\sigma_i}} \sigma_i
\end{aligned}
\end{equation}
%
But do not need to convert to strict scalar form if we are multiplying by a Pauli matrix.  So in anticommutator notation this takes the form
%
\begin{equation}\label{eqn:pauliMat:fourier}
\begin{aligned}
x &= x^i \sigma_i = \sum_i \inv{2}\symmetric{x}{\sigma_i} \sigma_i \\
x^i &= \inv{2}\symmetric{x}{\sigma_i}
\end{aligned}
\end{equation}
%
\subsection{Projection and rejection}

The usual Clifford algebra trick for projective and rejective split maps naturally to matrix form.  Write
%
\begin{equation}\label{eqn:pauliMatrix:600}
\begin{aligned}
x
&= x a a^{-1} \\
&= (x a) a^{-1} \\
&= \left( \inv{2}\symmetric{x}{a} + \inv{2}\antisymmetric{x}{a} \right) a^{-1} \\
&= \left( \inv{2}\left(x a + a x \right) + \inv{2} \left(x a - a x \right) \right) a^{-1} \\
&= \inv{2}\left(x + a x a^{-1} \right) + \inv{2} \left(x - a x a^{-1} \right) \\
\end{aligned}
\end{equation}
%
Since \(\symmetric{x}{a}\) is diagonal, this first term is proportional to \(a^{-1}\), and thus lines in the direction of \(a\) itself.  The second term is perpendicular to \(a\).

These are in fact the projection of \(x\) in the direction of \(a\) and rejection of \(x\) from the direction of \(a\) respectively.
%
\begin{equation}\label{eqn:pauliMatrix:620}
\begin{aligned}
x &= x_\parallel + x_\perp \\
x_\parallel &= \Proj_a(x) = \inv{2}\symmetric{x}{a}a^{-1} = \inv{2}\left(x + a x a^{-1} \right) \\
x_\perp &= \RejName_a(x) = \inv{2} \antisymmetric{x}{a} a^{-1} = \inv{2} \left( x - a x a^{-1} \right) \\
\end{aligned}
\end{equation}
%
To complete the verification of this note that the perpendicularity of the \(x_\perp\) term can be verified by taking dot products
%
\begin{equation}\label{eqn:pauliMatrix:640}
\begin{aligned}
\inv{2}\symmetric{a}{x_\perp}
&= \inv{4}\left( a \left(x - a x a^{-1} \right) +\left(x - a x a^{-1} \right) a \right) \\
&= \inv{4}\left( a x - a a x a^{-1} + x a - a x a^{-1} a \right) \\
&= \inv{4}\left( a x - x a + x a - a x \right) \\
&= 0
\end{aligned}
\end{equation}
%
\subsection{Space of the vector product}

Expansion of the anticommutator and commutator in coordinate form shows that these entities lie in a different space than the vectors itself.

For real coordinate vectors in the Pauli basis, all the commutator values are imaginary multiples and thus not representable
%
\begin{equation}\label{eqn:pauliMatrix:660}
\begin{aligned}
\antisymmetric{x}{y}
&= x^a \sigma_a y^b \sigma_b - y^a \sigma_a x^b \sigma_b \\
&= (x^a y^b - y^a x^b) \sigma_a \sigma_b \\
&= 2 i (x^a y^b - y^a x^b) \epsilon_{a b c} \sigma_c
\end{aligned}
\end{equation}
%
Similarly, the anticommutator is diagonal, which also falls outside the Pauli vector basis:
%
\begin{equation}\label{eqn:pauliMatrix:680}
\begin{aligned}
\symmetric{x}{y}
&= x^a \sigma_a y^b \sigma_b + y^a \sigma_a x^b \sigma_b \\
&= (x^a y^b + y^a x^b) \sigma_a \sigma_b \\
&= (x^a y^b + y^a x^b) ( I \delta_{a b} + i \epsilon_{a b c} \sigma_c) \\
&= \sum_a (x^a y^a + y^a x^a) I
+\sum_{a<b}(x^a y^b + y^a x^b) i (\mathLabelBox{\epsilon_{a b c} + \epsilon_{b a c}}{\(=0\)}) \sigma_c \\
&= \sum_a (x^a y^a + y^a x^a) I \\
&= 2 \sum_a x^a y^a I,
\end{aligned}
\end{equation}
%
These correspond to the Clifford dot product being scalar (grade zero), and the wedge defining a grade two space, where grade expresses the minimal degree that a product can be reduced to.  By example a Clifford product of normal unit vectors such as
%
\begin{equation}\label{eqn:pauliMatrix:700}
\begin{aligned}
\Be_1 \Be_3 \Be_4 \Be_1 \Be_3 \Be_4 \Be_3 &\propto \Be_3 \\
\Be_2 \Be_3 \Be_4 \Be_1 \Be_3 \Be_4 \Be_3 \Be_5 &\propto \Be_1 \Be_2 \Be_3 \Be_5
\end{aligned}
\end{equation}
%
are grade one and four respectively.  The proportionality constant will be dependent on metric of the underlying vector space and the number of permutations required to group terms in pairs of matching indices.

\subsection{Completely antisymmetrized product of three vectors}

In a Clifford algebra no imaginary number is required to express the antisymmetric (commutator) product.  However, the bivector space can be enumerated using a dual basis defined by multiplication of the vector basis elements with the unit volume trivector.  That is also the case here and gives a geometrical meaning to the imaginaries of the Pauli formulation.

How do we even write the unit volume element in Pauli notation?  This would be
%
\begin{equation}\label{eqn:pauliMatrix:720}
\begin{aligned}
\sigma_1 \wedge \sigma_2 \wedge \sigma_3
&= (\sigma_1 \wedge \sigma_2) \wedge \sigma_3 \\
&= \inv{2} \antisymmetric{\sigma_1}{\sigma_2} \wedge \sigma_3 \\
&= \inv{4} \left( \antisymmetric{\sigma_1}{\sigma_2} \sigma_3 + \sigma_3 \antisymmetric{\sigma_1}{\sigma_2} \right) \\
\end{aligned}
\end{equation}
%
So we have
\begin{equation}\label{eqn:pauliMat:triplewedgeproduct}
\begin{aligned}
\sigma_1 \wedge \sigma_2 \wedge \sigma_3
&= \inv{8} \symmetricBladeVecPauli{\sigma_1}{\sigma_2}{\sigma_3}
\end{aligned}
\end{equation}
%
Similar expansion of \(\sigma_1 \wedge \sigma_2 \wedge \sigma_3 = \sigma_1 \wedge (\sigma_2 \wedge \sigma_3)\), or \(\sigma_1 \wedge \sigma_2 \wedge \sigma_3 = (\sigma_3 \wedge \sigma_1) \wedge \sigma_2\) shows that we must also have
%
\begin{equation}\label{eqn:pauliMatrix:740}
\begin{aligned}
\symmetricBladeVecPauli{\sigma_1}{\sigma_2}{\sigma_3}
= \symmetricVecBladePauli{\sigma_1}{\sigma_2}{\sigma_3}
= \symmetricBladeVecPauli{\sigma_3}{\sigma_1}{\sigma_2}
\end{aligned}
\end{equation}
%
Until now the differences in notation between the anticommutator/commutator and the dot/wedge product of the Pauli algebra and Clifford algebra respectively have only differed by factors of two, which is not much of a big deal.  However, having to express the naturally associative wedge product operation in the non-associative looking notation of equation \eqnref{eqn:pauliMat:triplewedgeproduct} is rather unpleasant seeming.  Looking at an expression of the form gives no mnemonic hint of the underlying associativity, and actually seems obfuscating.  I suppose that one could get used to it though.

We expect to get a three by three determinant out of the trivector product.  Let us verify this by expanding this in Pauli notation for three general coordinate vectors
%
\begin{equation}\label{eqn:pauliMatrix:760}
\begin{aligned}
\symmetricBladeVecPauli{x}{y}{z}
&= \symmetricBladeVecPauli{x^a \sigma_a}{y^b \sigma_b}{ z^c \sigma_c} \\
&= 2 i \epsilon_{a b d} x^a y^b z^c \symmetric{\sigma_d}{\sigma_c} \\
&= 4 i \epsilon_{a b d} x^a y^b z^c \delta_{c d} I \\
&= 4 i \epsilon_{a b c} x^a y^b z^c I \\
&= 4 i
%\DETuvwijk{x}{y}{z}{a}{b}{c}
\begin{vmatrix}
 x_a & x_b & x_c \\
 y_a & y_b & y_c \\
 z_a & z_b & z_c
\end{vmatrix} I
\end{aligned}
\end{equation}
%
In particular, our unit volume element is
%
\begin{equation}\label{eqn:pauliMatrix:780}
\begin{aligned}
\sigma_1 \wedge \sigma_2 \wedge \sigma_3
&= \inv{4} \symmetricBladeVecPauli{\sigma_1}{\sigma_2}{\sigma_3} = i I
\end{aligned}
\end{equation}
%
So one sees that the complex number \(i\) in the Pauli algebra can logically be replaced by the unit pseudoscalar \(i I\), and relations involving \(i\), like the commutator expansion of a vector product, is restored to the expected dual form of Clifford algebra
%
\begin{equation}\label{eqn:pauliMatrix:800}
\begin{aligned}
\sigma_a \wedge \sigma_b
&= \inv{2} \antisymmetric{\sigma_a}{\sigma_b} \\
&= i \epsilon_{a b c} \sigma_c \\
&= (\sigma_a \wedge \sigma_b \wedge \sigma_c) \sigma_c \\
\end{aligned}
\end{equation}
%
Or
\begin{equation}\label{eqn:pauliMat:dualBivector}
\begin{aligned}
\sigma_a \wedge \sigma_b &= (\sigma_a \wedge \sigma_b \wedge \sigma_c) \cdot \sigma_c
\end{aligned}
\end{equation}
%
\subsection{Duality}

We have seen that multiplication by \(i\) is a duality operation, which is expected since \(iI\) is the matrix equivalent of the unit pseudoscalar.  Logically this means that for a vector \(x\), the product \((iI) x\) represents a plane quantity (torque, angular velocity/momentum, ...).  Similarly if \(B\) is a plane object, then \((iI) B\) will have a vector interpretation.

In particular, for the antisymmetric (commutator) part of the vector product \(x y\)
%
\begin{equation}\label{eqn:pauliMatrix:820}
\begin{aligned}
\inv{2} \antisymmetric{x}{y}
&= \inv{2} x^a y^b \antisymmetric{\sigma_a}{\sigma_b} \\
&= x^a y^b i \epsilon_{a b c}{\sigma_c} \\
\end{aligned}
\end{equation}
%
a ``vector'' in the dual space spanned by \(\{i \sigma_a\}\) is seen to be more naturally interpreted as a plane quantity (bivector in Clifford algebra).

As in Clifford algebra, we can write the cross product in terms of the antisymmetric product
%
\begin{equation}\label{eqn:pauliMatrix:840}
\begin{aligned}
a \times b = \inv{2i} \antisymmetric{a}{b}.
\end{aligned}
\end{equation}
%
With the factor of \(2\) in the denominator here (like the exponential form of sine), it is interesting
to contrast this to the cross product in its trigonometric form
%
\begin{equation}\label{eqn:pauliMatrix:860}
\begin{aligned}
a \times b
&= \Abs{a}\Abs{b} \sin(\theta) \ncap \\
&= \Abs{a}\Abs{b} \inv{2i} ( e^{i\theta} - e^{-i\theta}) \ncap
\end{aligned}
\end{equation}
%
This shows we can make the curious identity
%
\begin{equation}\label{eqn:pauliMatrix:880}
\begin{aligned}
\antisymmetric{\hata}{\hatb} &= ( e^{i\theta} - e^{-i\theta}) \ncap
\end{aligned}
\end{equation}
%
If one did not already know about the dual sides half angle rotation formulation of Clifford algebra, this is a hint about how one could potentially work towards that.  We have the commutator (or wedge product) as a rotation operator that leaves the normal component of a vector untouched (commutes with the normal vector).

\subsection{Complete algebraic space}

Pauli equivalents for all the elements in the Clifford algebra have now been
determined.

\begin{itemize}
\item scalar
\begin{equation}\label{eqn:pauliMatrix:900}
\begin{aligned}
\alpha \rightarrow \alpha I
\end{aligned}
\end{equation}
%
\item vector
\begin{equation}\label{eqn:pauliMatrix:920}
\begin{aligned}
u^i \sigma_i &\rightarrow
\begin{bmatrix}
0 & u^1 \\
u^1 & 0 \\
\end{bmatrix}
+
\begin{bmatrix}
0 & -i u^2 \\
i u^2 & 0 \\
\end{bmatrix}
+
\begin{bmatrix}
u^3 & 0 \\
0 & -u^3 \\
\end{bmatrix} \\
&=
\begin{bmatrix}
u^3 & u^1 -i u^2 \\
u^1 + i u^2 & -u^3 \\
\end{bmatrix}
\end{aligned}
\end{equation}
%
\item bivector
\begin{equation}\label{eqn:pauliMatrix:940}
\begin{aligned}
\sigma_1\sigma_2\sigma_3 v^a \sigma_a &\rightarrow i v^a \sigma_a \\
&= \begin{bmatrix}
iv^3 & iv^1 + v^2 \\
iv^1 - v^2 & -i v^3 \\
\end{bmatrix}
\end{aligned}
\end{equation}
\item pseudoscalar
\begin{equation}\label{eqn:pauliMatrix:960}
\begin{aligned}
\beta \sigma_1\sigma_2\sigma_3 \rightarrow i \beta I
\end{aligned}
\end{equation}
\end{itemize}
%
Summing these we have the mapping from Clifford basis to Pauli matrix as follows
%
\begin{equation}\label{eqn:pauliMatrix:980}
\begin{aligned}
\alpha + \beta I
+ u^i \sigma_i
+ I v^a \sigma_a
&\rightarrow
\begin{bmatrix}
(\alpha + u^3) + i(\beta + v^3) & (u^1 + v^2) +i (-u^2 + v^1) \\
(u^1 - v^2) + i (u^2 - v^1) & (\alpha -u^3) + i(\beta - v^3)\\
\end{bmatrix}
\end{aligned}
\end{equation}
%
Thus for any given sum of scalar, vector, bivector, and trivector elements we can completely express this in Pauli form as a general \(2 x 2\) complex matrix.

Provided that one can also extract the coordinates for each of the grades involved, this also provides a complete Clifford algebra characterization of an arbitrary complex \(2 x 2\) matrix.

Computationally this has some nice looking advantages.  Given any canned complex matrix software, one should be able to easily cook up with little work a working \R{3} Clifford calculator.

As for the coordinate extraction, part of the work can be done by taking real and imaginary components.  Let an element of the general algebra be denoted
%
\begin{equation}\label{eqn:pauliMatrix:1000}
\begin{aligned}
P =
\begin{bmatrix}
z_{11} & z_{12} \\
z_{21} & z_{22} \\
\end{bmatrix}
\end{aligned}
\end{equation}
%
We therefore have
\begin{equation}\label{eqn:pauliMatrix:1020}
\begin{aligned}
\Real(P) &=
\begin{bmatrix}
\alpha + u^3   & u^1 + v^2 \\
u^1 - v^2      & \alpha -u^3 \\
\end{bmatrix} \\
\Imag(P) &=
\begin{bmatrix}
\beta + v^3 & -u^2 + v^1 \\
u^2 + v^1 & \beta - v^3 \\
\end{bmatrix}
\end{aligned}
\end{equation}
%
By inspection, symmetric and antisymmetric sums of the real and imaginary parts recovers the coordinates as follows
%
\begin{equation}\label{eqn:pauliMatrix:1040}
\begin{aligned}
\alpha   &= \inv{2} \Real( z_{11} + z_{22} ) \\
u^3      &= \inv{2} \Real( z_{11} - z_{22} ) \\
u^1      &= \inv{2} \Real( z_{12} + z_{21} ) \\
v^2      &= \inv{2} \Real( z_{12} - z_{21} ) \\
\beta    &= \inv{2} \Imag( z_{11} + z_{22} ) \\
v^3      &= \inv{2} \Imag( z_{11} - z_{22} ) \\
v^1      &= \inv{2} \Imag( z_{12} + z_{21} ) \\
u^2      &= \inv{2} \Imag( -z_{12} + z_{21} ) \\
\end{aligned}
\end{equation}
%
In terms of grade selection operations the decomposition by grade
%
\begin{equation}\label{eqn:pauliMatrix:1060}
\begin{aligned}
P = \gpgradezero{P} +\gpgradeone{P} +\gpgradetwo{P} +\gpgradethree{P},
\end{aligned}
\end{equation}
%
is
%
\begin{equation}\label{eqn:pauliMatrix:1080}
\begin{aligned}
\gpgradezero{P} &= \inv{2} \Real( z_{11} + z_{22} ) = \inv{2} \Real(\trace P) \\
\gpgradeone{P} &= \inv{2} \left(
\Real( z_{12} + z_{21} ) \sigma_1
+\Imag( -z_{12} + z_{21} ) \sigma_2
+\Real( z_{11} - z_{22} ) \sigma_3
\right) \\
\gpgradetwo{P}
%&= \inv{2} \left(
%\Imag( z_{12} + z_{21} ) I \sigma_1
%+\Real( z_{12} - z_{21} ) I \sigma_2
%+\Imag( z_{11} - z_{22} ) I \sigma_3
%\right) \\
&= \inv{2} \left(
\Imag( z_{12} + z_{21} ) \sigma_2 \wedge \sigma_3
+\Real( z_{12} - z_{21} ) \sigma_3 \wedge \sigma_1
+\Imag( z_{11} - z_{22} ) \sigma_1 \wedge \sigma_2
\right) \\
\gpgradethree{P} &= \inv{2} \Imag( z_{11} + z_{22} ) I = \inv{2} \Imag(\trace P) \sigma_1 \wedge \sigma_2 \wedge \sigma_3
\end{aligned}
\end{equation}
%
Employing \(\Imag(z) = \Real(-iz)\), and \(\Real(z) = \Imag(iz)\) this can be made slightly more symmetrical, with Real operations selecting the vector coordinates and imaginary operations selecting the bivector coordinates.
%
\begin{equation}\label{eqn:pauliMatrix:1100}
\begin{aligned}
\gpgradezero{P} &= \inv{2} \Real( z_{11} + z_{22} ) = \inv{2} \Real(\trace P) \\
\gpgradeone{P} &= \inv{2} \left(
\Real( z_{12} + z_{21} ) \sigma_1
+\Real( iz_{12} - iz_{21} ) \sigma_2
+\Real( z_{11} - z_{22} ) \sigma_3
\right) \\
\gpgradetwo{P}
&= \inv{2} \left(
\Imag( z_{12} + z_{21} ) \sigma_2 \wedge \sigma_3
+\Imag( iz_{12} - iz_{21} ) \sigma_3 \wedge \sigma_1
+\Imag( z_{11} - z_{22} ) \sigma_1 \wedge \sigma_2
\right) \\
\gpgradethree{P} &= \inv{2} \Imag( z_{11} + z_{22} ) I = \inv{2} \Imag(\trace P) \sigma_1 \wedge \sigma_2 \wedge \sigma_3
\end{aligned}
\end{equation}
%
Finally, returning to the Pauli algebra, this also provides the following split of the Pauli multivector matrix into its geometrically significant components \(P = \gpgradezero{P} +\gpgradeone{P} +\gpgradetwo{P} +\gpgradethree{P}\),
%
\begin{equation}\label{eqn:pauliMatrix:1120}
\begin{aligned}
\gpgradezero{P} &= \inv{2} \Real( z_{11} + z_{22} ) I \\
\gpgradeone{P} &= \inv{2} \left(
\Real( z_{12} + z_{21} ) \sigma_1
+\Real( iz_{12} - iz_{21} ) \sigma_2
+\Real( z_{11} - z_{22} ) \sigma_3
\right) \\
\gpgradetwo{P}
&= \inv{2} \left(
\Imag( z_{12} + z_{21} ) i\sigma_1
+\Imag( iz_{12} - iz_{21} ) i\sigma_2
+\Imag( z_{11} - z_{22} ) i\sigma_k
\right) \\
\gpgradethree{P} &= \inv{2} \Imag( z_{11} + z_{22} ) iI
\end{aligned}
\end{equation}
%
\subsection{Reverse operation}

The reversal operation switches the order of the product of perpendicular vectors.  This will change the sign of grade two and three terms in the Pauli algebra.  Since \(\sigma_2\) is imaginary, conjugation does not have the desired effect, but Hermitian conjugation (conjugate transpose) does the trick.

Since the reverse operation can be written as Hermitian conjugation, one can also define the anticommutator and commutator in terms of reversion in a way that seems particularly natural for complex matrices.  That is
%
\begin{equation}\label{eqn:pauliMatrix:1140}
\begin{aligned}
\symmetric{a}{b} &= ab + (ab)^\conj \\
\antisymmetric{a}{b} &= ab - (ab)^\conj \\
\end{aligned}
\end{equation}
%
\subsection{Rotations}

Rotations take the normal Clifford, dual sided quaterionic form.  A rotation about a unit normal \(n\) will be
%
\begin{equation}\label{eqn:pauliMatrix:1160}
\begin{aligned}
R(x) = e^{-i n \theta/2} x e^{i n \theta/2}
\end{aligned}
\end{equation}
%
The Rotor \(R = e^{-i n \theta/2}\) commutes with any component of the vector x that is parallel to the normal (perpendicular to the plane), whereas it anticommutes with the components
in the plane.  Writing the vector components perpendicular and parallel to the plane respectively as \(x = x_\perp + x_\parallel\), the essence of the rotation action is this selective commutation or anti-commutation behavior
%
\begin{equation}\label{eqn:pauliMatrix:1180}
\begin{aligned}
R x_\parallel R^\conj &= x_\parallel R^\conj \\
R x_\perp R^\conj &= x_\perp R R^\conj = x_\perp
\end{aligned}
\end{equation}
%
Here the exponential has the obvious meaning in terms of exponential series, so for this bivector case we have
%
\begin{equation}\label{eqn:pauliMatrix:1200}
\begin{aligned}
\exp(i\ncap\theta/2) &= \cos(\theta/2) I + i\ncap \sin(\theta/2)
\end{aligned}
\end{equation}
%
The unit bivector \(B = i\ncap\) can also be defined explicitly in terms of two vectors \(a\), and \(b\) in the plane
%
\begin{equation}\label{eqn:pauliMatrix:1220}
\begin{aligned}
B = \inv{\Abs{\antisymmetric{a}{b}}} \antisymmetric{a}{b}
\end{aligned}
\end{equation}
%
Where the bivector length is defined in terms of the conjugate square (bivector times bivector reverse)
%
\begin{equation}\label{eqn:pauliMatrix:1240}
\begin{aligned}
\Abs{\antisymmetric{a}{b}}^2 = \antisymmetric{a}{b} {\antisymmetric{a}{b}}^\conj
\end{aligned}
\end{equation}
%
Examples to complete this subsection would make sense.  As one of the most powerful and useful operations in the algebra, it is a shame in terms of completeness to skimp on this.  However, except for some minor differences like substitution of the Hermitian conjugate operation for reversal, the use of the identity matrix \(I\) in place of the scalar in the exponential expansion, the treatment is exactly the same as in the Clifford algebra.

\subsection{Grade selection}

Coordinate equations for grade selection were worked out above, but the observation that reversion and Hermitian conjugation are isomorphic operations can partially clean this up.  In particular a Hermitian conjugate symmetrization and anti-symmetrization of the general matrix provides a nice split into quaternion and dual quaternion parts (say \(P = Q + R\) respectively).  That is
%
\begin{equation}\label{eqn:pauliMatrix:1260}
\begin{aligned}
Q &= \gpgradezero{P} + \gpgradeone{P} = \inv{2}(P + P^\conj) \\
R &= \gpgradetwo{P} + \gpgradethree{P} = \inv{2}(P - P^\conj)
\end{aligned}
\end{equation}
%
Now, having done that, how to determine \(\gpgradezero{Q}\), \(\gpgradeone{Q}\), \(\gpgradetwo{R}\), and \(\gpgradethree{R}\) becomes the next question.  Once that is done, the individual coordinates can be picked off easily enough.  For the vector parts, a Fourier decomposition as in equation \eqnref{eqn:pauliMat:fourier} will retrieve the desired coordinates.

The dual vector coordinates can be picked off easily enough taking dot products with the dual basis vectors
%
\begin{equation}\label{eqn:pauliMat:dualfourier}
\begin{aligned}
B &= B^k i \sigma_k = \sum_k \inv{2}\symmetric{B}{\inv{i \sigma_k}} i\sigma_k \\
B^k &= \inv{2}\symmetric{B}{\inv{i\sigma_k}}
\end{aligned}
\end{equation}
%
For the quaternion part \(Q\) the aim is to figure out how to isolate or subtract out the scalar part.  This is the only tricky bit because the diagonal bits are all mixed up with the \(\sigma_3\) term which is also real, and diagonal.  Consideration of the sum
%
\begin{equation}\label{eqn:pauliMatrix:1280}
\begin{aligned}
a I + b \sigma_3 =
\begin{bmatrix}
a + b & 0 \\
0 & a - b \\
\end{bmatrix},
\end{aligned}
\end{equation}
%
shows that trace will recover the value \(2a\), so we have
%
\begin{equation}\label{eqn:pauliMatrix:1300}
\begin{aligned}
\gpgradezero{Q} &= \inv{2}\traceB{Q} I \\
\gpgradeone{Q} &= Q - \inv{2}\traceB{Q} I.
\end{aligned}
\end{equation}
%
Next is isolation of the \textAndIndex{pseudoscalar} part of the dual quaternion \(R\).  As with the scalar part, consideration of the sum of the \(i\sigma_3\) term and the \(iI\) term is required
%
\begin{equation}\label{eqn:pauliMatrix:1320}
\begin{aligned}
ia I + ib \sigma_3 =
\begin{bmatrix}
ia + ib & 0 \\
0 & ia - ib \\
\end{bmatrix},
\end{aligned}
\end{equation}
%
So the trace of the dual quaternion provides the \(2a\), leaving the bivector and pseudoscalar grade split
%
\begin{equation}\label{eqn:pauliMatrix:1340}
\begin{aligned}
\gpgradethree{R} &= \inv{2}\traceB{R} I \\
\gpgradetwo{R} &= R - \inv{2}\traceB{R} I.
\end{aligned}
\end{equation}
%
A final assembly of these results provides the following coordinate free grade selection operators
%
\begin{equation}\label{eqn:pauliMatrix:1360}
\begin{aligned}
\gpgradezero{P} &= \inv{4}\traceB{P + P^\conj} I \\
\gpgradeone{P} &= \inv{2} (P + P^\conj) - \inv{4}\traceB{P + P^\conj} I \\
\gpgradetwo{P} &= \inv{2} (P - P^\conj) - \inv{4}\traceB{P - P^\conj} I \\
\gpgradethree{P} &= \inv{4}\traceB{P - P^\conj} I
\end{aligned}
\end{equation}
%
\subsection{Generalized dot products}

Here the equivalent of the generalized Clifford bivector/vector dot product will be computed, as well as the associated distribution equation
%
\begin{equation}\label{eqn:pauliMatrix:1380}
\begin{aligned}
( a \wedge b ) \cdot c &= a (b \cdot c) - b (a \cdot c)
\end{aligned}
\end{equation}
%
To translate this write
%
\begin{equation}\label{eqn:pauliMat:cliffordBivectorVectorDot}
\begin{aligned}
( a \wedge b ) \cdot c &= \inv{2} \left( (a \wedge b) c - c (a \wedge b) \right)
\end{aligned}
\end{equation}
%
Then with the identifications
%
\begin{equation}\label{eqn:pauliMatrix:1400}
\begin{aligned}
a \cdot b &\equiv \inv{2} \symmetric{a}{b} \\
a \wedge b &\equiv \inv{2} \antisymmetric{a}{b}
\end{aligned}
\end{equation}
%
we have
%
\begin{equation}\label{eqn:pauliMatrix:1420}
\begin{aligned}
( a \wedge b ) \cdot c
&\equiv \inv{4} \antisymmetric{\antisymmetric{a}{b}}{c}  \\
&= \inv{2} \left( a \symmetric{b}{c} - \symmetric{b}{c} a \right)
\end{aligned}
\end{equation}
%
From this we also get the strictly Pauli algebra identity
%
\begin{equation}\label{eqn:pauliMatrix:1440}
\begin{aligned}
\antisymmetric{\antisymmetric{a}{b}}{c} &= {2} \left( a \symmetric{b}{c} - \symmetric{b}{c} a \right)
\end{aligned}
\end{equation}
%
But the geometric meaning of this is unfortunately somewhat obfuscated by the notation.

\subsection{Generalized dot and wedge product}

The fundamental definitions of dot and wedge products are in terms of grade
%
\begin{equation}\label{eqn:pauliMatrix:1460}
\begin{aligned}
\gpgrade{A}{r} \cdot \gpgrade{B}{s} = \gpgrade{AB}{{\Abs{r-s}}}
\end{aligned}
\end{equation}
\begin{equation}\label{eqn:pauliMatrix:1480}
\begin{aligned}
\gpgrade{A}{r} \wedge \gpgrade{B}{s} = \gpgrade{AB}{{r+s}}
\end{aligned}
\end{equation}
%
Use of the trace and Hermitian conjugate split grade selection operations above, we can calculate these for each of the four grades in the Pauli algebra.

\subsubsection{Grade zero}

There are three dot products consider, vector/vector, bivector/bivector, and trivector/trivector.  In each case we want to compute
%
\begin{equation}\label{eqn:pauliMatrix:1500}
\begin{aligned}
A \cdot B &= \gpgradezero{ A}{ B} \\
&= \inv{4}\traceB{ AB + (AB)^\conj} I \\
&= \inv{4}\traceB{ AB + B^\conj A^\conj} I
\end{aligned}
\end{equation}
%
For vectors we have \(a^\conj = a\), since the Pauli basis is Hermitian, whereas for bivectors and trivectors we have \(a^\conj = -a\).  Therefore, in all cases where \(A\), and \(B\) have equal grades we have
%
\begin{equation}\label{eqn:pauliMatrix:1520}
\begin{aligned}
A \cdot B &= \gpgradezero{A}{B} I \\
&= \inv{4}\traceB{ AB + B A} I \\
&= \inv{4}\trace{\symmetric{ A}{ B}} I
\end{aligned}
\end{equation}
%
\subsubsection{Grade one}

We have two dot products that produce vectors, bivector/vector, and trivector/bivector, and in each case we need to compute
%
\begin{equation}\label{eqn:pauliMatrix:1540}
\begin{aligned}
\gpgradeone{ A B} &= \inv{2}( AB + (AB)^\conj ) -\inv{4}\traceB{ AB + (AB)^\conj }
\end{aligned}
\end{equation}
%
For the bivector/vector dot product we have
%
\begin{equation}\label{eqn:pauliMatrix:1560}
\begin{aligned}
(Ba)^\conj = -a B
\end{aligned}
\end{equation}
%
For bivector \(B = i b^k \sigma_k\), and vector \(a = a^k \sigma_k\) our symmetric Hermitian sum in coordinates is
%
\begin{equation}\label{eqn:pauliMatrix:1580}
\begin{aligned}
B a + (Ba)^\conj
&= B a - a B \\
&= i b^k \sigma_k a^m \sigma_m - a^m \sigma_m i b^k \sigma_k \\
\end{aligned}
\end{equation}
%
Any \(m=k\) terms will vanish, leaving only the bivector terms, which are traceless.  We therefore have
%
\begin{equation}\label{eqn:pauliMatrix:1600}
\begin{aligned}
B \cdot a &= \gpgradeone{ B a} \\
&= \inv{2}( B a - a B ) \\
&= \inv{2} \antisymmetric{B}{a}.
\end{aligned}
\end{equation}
%
This result was borrowed without motivation from Clifford algebra in equation \eqnref{eqn:pauliMat:cliffordBivectorVectorDot}, and thus not satisfactory in terms of a logically derived sequence.

For a trivector \(T\) dotted with bivector \(B\) we have
%
\begin{equation}\label{eqn:pauliMatrix:1620}
\begin{aligned}
(BT)^\conj = (-T)(-B) = TB = BT.
\end{aligned}
\end{equation}
%
This is also traceless, and the trivector/bivector dot product is therefore reduced to just
%
\begin{equation}\label{eqn:pauliMatrix:1640}
\begin{aligned}
B \cdot T &= \gpgradeone{ B T} \\
&= \inv{2} \symmetric{B}{T} \\
&= {B}{T} \\
&= T B.
\end{aligned}
\end{equation}
%
This is the duality relationship for bivectors.  Multiplication by the unit pseudoscalar (or any multiple of it), produces a vector, the dual of the original bivector.

\subsubsection{Grade two}

We have two products that produce a grade two term, the vector wedge product, and the vector/trivector dot product.  For either case we must compute
%
\begin{equation}\label{eqn:pauliMat:gradeTwo}
\begin{aligned}
\gpgradetwo{ A B} &= \inv{2}( AB - (AB)^\conj ) -\inv{4}\traceB{ AB - (AB)^\conj }
\end{aligned}
\end{equation}
%
For a vector \(a\), and trivector \(T\) we need the antisymmetric Hermitian sum
%
\begin{equation}\label{eqn:pauliMatrix:1660}
\begin{aligned}
a T - (a T)^\conj &= a T + T a = 2 a T = 2 T a
\end{aligned}
\end{equation}
%
This is a pure bivector, and thus traceless, leaving just
%
\begin{equation}\label{eqn:pauliMatrix:1680}
\begin{aligned}
a \cdot T  &= \gpgradetwo{ a T} \\
&= a T \\
&= T a
\end{aligned}
\end{equation}
%
Again we have the duality relation, pseudoscalar multiplication with a vector produces a bivector, and is equivalent to the dot product of the two.

Now for the wedge product case, with vector \(a = a^m \sigma_m\), and \(b = b^k \sigma_k\) we must compute
%
\begin{equation}\label{eqn:pauliMatrix:1700}
\begin{aligned}
a b - (a b)^\conj
&= a b - b a \\
&= a^m \sigma_m b^k \sigma_k - b^k \sigma_k a^m \sigma_m
\end{aligned}
\end{equation}
%
All the \(m = n\) terms vanish, leaving a pure bivector which is traceless, so only the first term of \eqnref{eqn:pauliMat:gradeTwo} is relevant, and is in this case a commutator
%
\begin{equation}\label{eqn:pauliMatrix:1720}
\begin{aligned}
a \wedge b  &= \gpgradetwo{ a b} \\
&= \inv{2} \antisymmetric{ a}{b}
\end{aligned}
\end{equation}
%
\subsubsection{Grade three}

There are two ways we can produce a grade three term in the algebra.  One is a wedge of a vector with a bivector, and the other is the wedge product of three vectors.  The triple wedge product is the grade three term of the product of the three
%
\begin{equation}\label{eqn:pauliMatrix:1740}
\begin{aligned}
a \wedge b \wedge c
&= \gpgradethree{ a b c} \\
&= \inv{4} \traceB{ a b c - (a b c)^\conj} \\
&= \inv{4} \traceB{ a b c - c b a} \\
\end{aligned}
\end{equation}
%
With a split of the \(bc\) and \(cb\) terms into symmetric and antisymmetric terms we have
%
\begin{equation}\label{eqn:pauliMatrix:1760}
\begin{aligned}
a b c - c b a
&=
 \inv{2} (a \symmetric{b}{c} - \symmetric{c}{b} a)
+\inv{2} (a \antisymmetric{b}{c} - \antisymmetric{c}{b} a)
\end{aligned}
\end{equation}
%
The symmetric term is diagonal so it commutes (equivalent to scalar commutation with a vector in Clifford algebra), and this therefore vanishes.  Writing \(B = b \wedge c = \inv{2}\antisymmetric{b}{c}\), and noting that \(\antisymmetric{b}{c} = -\antisymmetric{c}{b}\) we therefore have
%
\begin{equation}\label{eqn:pauliMatrix:1780}
\begin{aligned}
a \wedge B
&= \gpgradethree{ a B} \\
&= \inv{4} \traceB{ a B + B a} \\
&= \inv{4} \trace{ \symmetric{a}{B}} \\
\end{aligned}
\end{equation}
%
In terms of the original three vectors this is
%
\begin{equation}\label{eqn:pauliMatrix:1800}
\begin{aligned}
a \wedge b \wedge c
&= \gpgradethree{ a B} \\
&= \inv{8} \trace{ \symmetricVecBladePauli{a}{b}{c}}.
%&= \inv{8} \trace{ \symmetricVecBladePauli{a}{b}{c}}.
\end{aligned}
\end{equation}
%
Since this could have been expanded by grouping \(ab\) instead of \(bc\) we also have
%
\begin{equation}\label{eqn:pauliMatrix:1820}
\begin{aligned}
a \wedge b \wedge c
&= \inv{8} \trace{ \symmetricBladeVecPauli{a}{b}{c}}.
\end{aligned}
\end{equation}
%symmetricBladeVecPauli
%
\subsection{Plane projection and rejection}

Projection of a vector onto a plane follows like the vector projection case.  In the Pauli notation this is
%
\begin{equation}\label{eqn:pauliMatrix:1840}
\begin{aligned}
x
&= x B \inv{B} \\
&= \inv{2} \symmetric{x}{B} \inv{B} + \inv{2} \antisymmetric{x}{B} \inv{B} \\
\end{aligned}
\end{equation}
%
Here the plane is a bivector, so if vectors \(a\), and \(b\) are in the plane, the orientation and attitude can be represented
by the commutator

%\newcommand{\symmetricVecBladePauli}[3]{\left\{{#1},\left[{#2},{#3}\right]\right\}}
%
So we have
\begin{equation}\label{eqn:pauliMatrix:1860}
\begin{aligned}
x
%&= \inv{2} \symmetricVecBladePauli{x}{a}{b} \inv{\antisymmetric{a}{b}} + \inv{2} \antisymmetric{x}{\antisymmetric{a}{b}} \inv{\antisymmetric{a}{b}} \\
&= \inv{2} \symmetricVecBladePauli{x}{a}{b} \inv{\antisymmetric{a}{b}} + \inv{2} \antisymmetric{x}{\antisymmetric{a}{b}} \inv{\antisymmetric{a}{b}} \\
\end{aligned}
\end{equation}
%
Of these the second term is our projection onto the plane, while the first is the normal component of the vector.

\section{Examples}

\subsection{Radial decomposition}

\subsubsection{Velocity and momentum}

A decomposition of velocity into radial and perpendicular components should be straightforward in the Pauli algebra as it is in the Clifford algebra.

With a radially expressed position vector
%
\begin{equation}\label{eqn:pauliMatrix:1880}
\begin{aligned}
x = \Abs{x} \hatx,
\end{aligned}
\end{equation}
%
velocity can be written by taking derivatives
\begin{equation}\label{eqn:pauliMatrix:1900}
\begin{aligned}
v = x' = \Abs{x}' \hatx + \Abs{x} \hatx'
\end{aligned}
\end{equation}
%
or as above in the projection calculation with
\begin{equation}\label{eqn:pauliMatrix:1920}
\begin{aligned}
v
&= v \inv{x} x \\
&= \inv{2}\symmetric{v}{\inv{x}} x + \inv{2}\antisymmetric{v}{\inv{x}} x \\
&= \inv{2}\symmetric{v}{\hatx} \hatx + \inv{2}\antisymmetric{v}{\hatx} \hatx
\end{aligned}
\end{equation}
%
By comparison we have
%
\begin{equation}\label{eqn:pauliMatrix:1940}
\begin{aligned}
\Abs{x}' &= \inv{2}\symmetric{v}{\hatx} \\
\hatx' &= \inv{2 \Abs{x}} \antisymmetric{v}{\hatx} \hatx
\end{aligned}
\end{equation}
%
In assembled form we have
%
\begin{equation}\label{eqn:pauliMatrix:1960}
\begin{aligned}
v &= \inv{2}\symmetric{v}{\hatx} \hatx + x \omega \\
\end{aligned}
\end{equation}
%
Here the commutator has been identified with the angular velocity bivector \(\omega\)
%
\begin{equation}\label{eqn:pauliMatrix:1980}
\begin{aligned}
\omega &= \inv{2 x^2}\antisymmetric{x}{v}.
\end{aligned}
\end{equation}
%
Similarly, the linear and angular momentum split of a momentum vector is
%
\begin{equation}\label{eqn:pauliMatrix:2000}
\begin{aligned}
p_\parallel &= \inv{2}\symmetric{p}{\hatx} \hatx \\
p_\perp &= \inv{2}\antisymmetric{p}{\hatx} \hatx
\end{aligned}
\end{equation}
%
and in vector form
%
\begin{equation}\label{eqn:pauliMatrix:2020}
\begin{aligned}
p &= \inv{2}\symmetric{p}{\hatx} \hatx + m x \omega \\
\end{aligned}
\end{equation}
%
Writing \(J = m x^2\) for the moment of inertia we have for our commutator
%
\begin{equation}\label{eqn:pauliMatrix:2040}
\begin{aligned}
L = \inv{2}\antisymmetric{x}{p} = m x^2 \omega = J \omega
\end{aligned}
\end{equation}
%
With the identification of the commutator with the angular momentum bivector \(L\) we have the total momentum as
%
\begin{equation}\label{eqn:pauliMatrix:2060}
\begin{aligned}
p &= \inv{2}\symmetric{p}{\hatx} \hatx + \inv{x} L \\
\end{aligned}
\end{equation}
%
\subsubsection{Acceleration and force}

Having computed velocity, and its radial split, the next logical thing to try is acceleration.

The acceleration will be
%
\begin{equation}\label{eqn:pauliMatrix:2080}
\begin{aligned}
a = v' = \Abs{x}'' \hatx + 2 \Abs{x}' \hatx' + \Abs{x} \hatx''
\end{aligned}
\end{equation}
%
We need to compute \(\hatx''\) to continue, which is
%
\begin{equation}\label{eqn:pauliMatrix:2100}
\begin{aligned}
\hatx''
&= \left(\frac{1}{2 \Abs{x}^3} \antisymmetric{v}{x} x \right)' \\
&=
\frac{-3}{2 \Abs{x}^4} \Abs{x}' \antisymmetric{v}{x} x
+ \frac{1}{2 \Abs{x}^3} \antisymmetric{a}{x} x
+ \frac{1}{2 \Abs{x}^3} \antisymmetric{v}{x} v \\
&=
\frac{-3}{4 \Abs{x}^5} \symmetric{v}{x} \antisymmetric{v}{x} x
+ \frac{1}{2 \Abs{x}^3} \antisymmetric{a}{x} x
+ \frac{1}{2 \Abs{x}^3} \antisymmetric{v}{x} v
\end{aligned}
\end{equation}
%
Putting things back together is a bit messy, but starting so gives
%
\begin{equation}\label{eqn:pauliMatrix:2120}
\begin{aligned}
a
&= \Abs{x}'' \hatx + 2
%\Abs{x}'
%\hatx'
\inv{4 \Abs{x}^4} \symmetric{v}{x} \antisymmetric{v}{x} x
% + \Abs{x} \hatx''
+\frac{-3}{4 \Abs{x}^4} \symmetric{v}{x} \antisymmetric{v}{x} x
+ \frac{1}{2 \Abs{x}^2} \antisymmetric{a}{x} x
+ \frac{1}{2 \Abs{x}^2} \antisymmetric{v}{x} v \\
&=
  \Abs{x}'' \hatx
- \frac{1}{4 \Abs{x}^4} \symmetric{v}{x} \antisymmetric{v}{x} x
+ \frac{1}{2 \Abs{x}^2} \antisymmetric{a}{x} x
+ \frac{1}{2 \Abs{x}^2} \antisymmetric{v}{x} v \\
&=
  \Abs{x}'' \hatx
+ \inv{4 \Abs{x}^4} \antisymmetric{v}{x} \left( -\symmetric{v}{x} x + 2 x^2 v \right)
+ \frac{1}{2 \Abs{x}^2} \antisymmetric{a}{x} x
\end{aligned}
\end{equation}
%
The anticommutator can be eliminated above using
%
\begin{equation}\label{eqn:pauliMatrix:2140}
\begin{aligned}
v x &= \inv{2} \symmetric{v}{x} + \inv{2}\antisymmetric{v}{x} \\
\implies \\
-\symmetric{v}{x} x + 2 x^2 v
&= -(2 v x - \antisymmetric{v}{x}) x + 2 x^2 v \\
&= \antisymmetric{v}{x} x
\end{aligned}
\end{equation}
%
Finally reassembly of the assembly is thus
%
\begin{equation}\label{eqn:pauliMatrix:2160}
\begin{aligned}
a
&= \Abs{x}'' \hatx
+ \inv{4 \Abs{x}^4} \antisymmetric{v}{x}^2 x
+ \frac{1}{2 \Abs{x}^2} \antisymmetric{a}{x} x \\
&= \Abs{x}'' \hatx
+ \omega^2 x
+ \frac{1}{2} \antisymmetric{a}{x} \inv{x} \\
\end{aligned}
\end{equation}
%
The second term is the inwards facing radially directed acceleration, while the last is the rejective component of the acceleration.

It is usual to express this last term as the rate of change of angular momentum (torque).  Because \(\antisymmetric{v}{v} = 0\), we have
%
\begin{equation}\label{eqn:pauliMatrix:2180}
\begin{aligned}
\frac{d \antisymmetric{x}{v}}{dt} = \antisymmetric{x}{a}
\end{aligned}
\end{equation}
%
So, for constant mass, we can write the torque as
%
\begin{equation}\label{eqn:pauliMatrix:2200}
\begin{aligned}
\tau
&= \frac{d}{dt} \left( \inv{2} \antisymmetric{x}{p} \right) \\
&= \frac{dL}{dt}
\end{aligned}
\end{equation}
%
and finally have for the force
%
\begin{equation}\label{eqn:pauliMatrix:2220}
\begin{aligned}
F
&= m \Abs{x}'' \hatx + m \omega^2 x + \inv{x} \frac{dL}{dt} \\
&= m \left(\Abs{x}'' - \frac{\Abs{\omega^2}}{\Abs{x}} \right) \hatx + \inv{x} \frac{dL}{dt}
\end{aligned}
\end{equation}
%
\section{Conclusion}

Although many of the GA references that can be found downplay the Pauli algebra as unnecessarily employing matrices as a basis, I believe this shows that there are some nice computational and logical niceties in the complete formulation of the \R{3} Clifford algebra in this complex matrix formulation.  If nothing else it takes some of the abstraction away, which is good for developing intuition.  All of the generalized dot and wedge product relationships are easily derived showing specific examples of the general pattern for the dot and blade product equations which are sometimes supplied as definitions instead of consequences.

Also, the matrix concepts (if presented right which I likely have not done) should also be accessible to most anybody out of high school these days since both matrix algebra and complex numbers are covered as basics these days (at least that is how I recall it from fifteen years back;)

Hopefully, having gone through the exercise of examining all the equivalent constructions will be useful in subsequent Quantum physics study to see how the matrix algebra that is used in that subject is tied to the classical geometrical vector constructions.

Expressions that were scary and mysterious looking like
%
\begin{equation}\label{eqn:pauliMatrix:2240}
\begin{aligned}
\antisymmetric{L_x}{L_y} = i \Hbar L_z
\end{aligned}
\end{equation}
%
are no longer so bad since some of the geometric meaning that backs this operator expression is now clear (this is a quantization of angular momentum in a specific plane, and encodes the plane orientation as well as the magnitude).  Knowing that \(\antisymmetric{a}{b}\) was an antisymmetric sum, but not realizing the connection between that and the wedge product previously made me wonder ``where the hell did the i come from''?

This commutator equation is logically and geometrically a plane operation.  It can therefore be expressed with a vector duality relationship employing the \R{3} unit pseudoscalar \(iI = \sigma_1 \sigma_2 \sigma_3\).  This is a good nice step towards taking some of the mystery out of the math behind the physics of the subject (which has enough intrinsic mystery without the mathematical language adding to it).

It is unfortunate that QM uses this matrix operator formulation and none of classical physics does.  By the time one gets to QM learning an entirely new language is required despite the fact that there are many powerful applications of this algebra in the classical domain, not just for rotations which is recognized (in \citep{goldstein1951cm} for example where he uses the Pauli algebra to express his rotation quaternions.)

%\EndArticle

   %
% Copyright � 2012 Peeter Joot.  All Rights Reserved.
% Licenced as described in the file LICENSE under the root directory of this GIT repository.
%

%
%
\mychapter{Gamma Matrices}
\label{chap:PJDiracGamma}
\index{gamma matrices}
%\date{Dec 13, 2008.  gamma.tex}

\section{Dirac matrices}

\index{Dirac!matrix}
\index{gamma matrix}
The Dirac matrices \(\gamma^\mu\) can be used as a Minkowski basis.  The basic defining relationship is the Minkowski metric, where the dot products satisfy
%
\begin{equation}\label{eqn:gamma:20}
\begin{aligned}
\scalarProduct{\gamma^\mu}{\gamma^\nu} &= \pm \delta_{\mu\nu} \\
(\scalarProduct{\gamma^0}{\gamma^0})(\scalarProduct{\gamma^a}{\gamma^a}) &= -1 \quad \text{where \(a \in \{1,2,3\}\)}
\end{aligned}
\end{equation}
%
There is freedom to pick the positive square for either \(\gamma^0\) or \(\gamma^a\), and both conventions are common.

One of the matrix representations for these vectors listed in the
\href{https://en.wikipedia.org/wiki/Gamma_matrices}{Dirac matrix wikipedia article}
is
%
\begin{equation}\label{eqn:gamma:basis}
\begin{aligned}
\gamma^0 &= \begin{bmatrix}
 1  &  0  &  0  &  0  \\
 0  &  1  &  0  &  0  \\
 0  &  0  &  -1  &  0  \\
 0  &  0  &  0  &  -1  \\
\end{bmatrix} \quad
\gamma^1 = \begin{bmatrix}
 0  &  0  &  0  &  1  \\
 0  &  0  &  1  &  0  \\
 0  &  -1  &  0  &  0  \\
 -1  &  0  &  0  &  0  \\
\end{bmatrix} \\
\gamma^2 &= \begin{bmatrix}
 0  &  0  &  0  &  -i  \\
 0  &  0  &  i  &  0  \\
 0  &  i  &  0  &  0  \\
 -i  &  0  &  0  &  0  \\
\end{bmatrix}
\quad \gamma^3 = \begin{bmatrix}
 0  &  0  &  1  &  0  \\
 0  &  0  &  0  &  -1  \\
 -1  &  0  &  0  &  0  \\
 0  &  1  &  0  &  0  \\
\end{bmatrix}
\end{aligned}
\end{equation}
%
For this particular basis we have a \(+---\) metric signature.  In the matrix form this takes the specific meaning that \((\gamma^0)^2 = I\), and \((\gamma^a)^2 = -I\).

A table of all the possible product variants of \eqnref{eqn:gamma:basis} can be found below in the appendix.

\subsection{anticommutator product}
\index{anticommutator}

Noting that the matrices square in the fashion just described and that they reverse sign when multiplication order is reversed allows for summarizing the dot products relationships as follows
%
\begin{equation}\label{eqn:gamma:symmetric}
\begin{aligned}
\symmetric{\gamma^\mu}{\gamma^\nu}
&= {\gamma^\mu}{\gamma^\nu} + {\gamma^\nu}{\gamma^\mu} \\
%&= 2 (\scalarProduct{\gamma^\mu}{\gamma^\nu}) I \\
&= 2 \eta^{\mu\nu} I,
\end{aligned}
\end{equation}
%
where the metric tensor \(\eta^{\mu\nu} = \scalarProduct{\gamma^\mu}{\gamma^\nu}\) is commonly summarized as coordinates of a matrix as in
%
\begin{equation}\label{eqn:gamma:40}
\begin{aligned}
\begin{bmatrix}
\eta^{\mu\nu}
\end{bmatrix}
&=
\begin{bmatrix}
1 & 0 & 0 & 0 \\
0 & -1 & 0 & 0 \\
0 & 0 & -1 & 0 \\
0 & 0 & 0 & -1 \\
\end{bmatrix}
\end{aligned}
\end{equation}
%
The relationship \eqnref{eqn:gamma:symmetric} is taken as the defining relationship for the Dirac matrices, but can be seen to be just a matricized statement of the Clifford vector dot product.

\subsection{Written as Pauli matrices}
\index{Pauli matrices}

Using the Pauli matrices
%
\begin{equation}\label{eqn:gamma:60}
\begin{aligned}
\sigma_1 = \PauliX \quad \sigma_2 = \PauliY \quad \sigma_3 = \PauliZ
\end{aligned}
\end{equation}
%
one can write the Dirac matrices and all their products (reading from the multiplication table) more concisely as
%
\begin{equation}\label{eqn:gamma:80}
\begin{aligned}
\gamma^0 &=
\begin{bmatrix}
I & 0 \\
0 & -I
\end{bmatrix} \\
\gamma^a &=
\begin{bmatrix}
0 & \sigma_a \\
-\sigma_a & 0 \\
\end{bmatrix} \\
\gamma^0 \gamma^a &=
\begin{bmatrix}
0 & \sigma_a \\
\sigma_a & 0 \\
\end{bmatrix} \\
\gamma^a \gamma^b &=
- i \epsilon_{a b c}
\begin{bmatrix}
\sigma_c & 0 \\
0 & \sigma_c \\
\end{bmatrix} \\
\gamma^1 \gamma^2 \gamma^3 &= i
\begin{bmatrix}
0 & -I \\
I & 0
\end{bmatrix} \\
\gamma^0 \gamma^1 \gamma^2 &= i
\begin{bmatrix}
-\sigma_1 & 0 \\
0 & \sigma_1 \\
\end{bmatrix} \\
\gamma^3 \gamma^0 \gamma^1 &= i
\begin{bmatrix}
\sigma_2 & 0 \\
0 & -\sigma_2 \\
\end{bmatrix} \\
\gamma^0 \gamma^1 \gamma^2 &= i
\begin{bmatrix}
-\sigma_3 & 0 \\
0 & \sigma_3 \\
\end{bmatrix}
\end{aligned}
\end{equation}
%
\subsection{Deriving properties using the Pauli matrices}

From the multiplication table a number of properties can be observed.  Using the Pauli matrices one can arrive at these more directly using the multiplication identity for those
matrices
%
\begin{equation}\label{eqn:gamma:100}
\begin{aligned}
\sigma_a \sigma_b = 2 i \epsilon_{abc} \sigma_c
\end{aligned}
\end{equation}
%
Actually taking the time to type this out in full does not seem worthwhile and is a fairly straightforward exercise.

\subsection{Conjugation behavior}
\index{conjugation}

Unlike the Pauli matrices, the Dirac matrices do not split nicely via conjugation.  Instead we have the time basis vector and its dual are Hermitian
%
\begin{equation}\label{eqn:gamma:120}
\begin{aligned}
(\gamma^0)^\conj &= \gamma^0 \\
(\gamma^1 \gamma^2 \gamma^3)^\conj &= \gamma^1 \gamma^2 \gamma^3
\end{aligned}
\end{equation}
%
whereas the spacelike basis vectors and their duals are all anti-Hermitian
%
\begin{equation}\label{eqn:gamma:140}
\begin{aligned}
(\gamma^a)^\conj &= -\gamma^a \\
(\gamma^a \gamma^b \gamma^c)^\conj &= - \gamma^a \gamma^b \gamma^c.
\end{aligned}
\end{equation}
%
For the scalar and the pseudoscalar parts we have a Hermitian split
%
\begin{equation}\label{eqn:gamma:160}
\begin{aligned}
I^\conj &= I \\
(\gamma^0 \gamma^1 \gamma^2 \gamma^3)^\conj &= -(\gamma^0 \gamma^1 \gamma^2 \gamma^3)^\conj
\end{aligned}
\end{equation}
%
and finally, also have a Hermitian split of the bivector parts into spacetime (relative vectors), and the purely spatial bivectors
%
\begin{equation}\label{eqn:gamma:180}
\begin{aligned}
(\gamma^0 \gamma^a)^\conj &= \gamma^0 \gamma^a \\
(\gamma^a \gamma^b)^\conj &= -\gamma^a \gamma^b
\end{aligned}
\end{equation}
%
Is there a logical and simple set of matrix operations that splits things nicely into scalar, vector, bivector, trivector, and pseudoscalar parts as there was with the Pauli
matrices?

\section{Appendix.  Table of all generated products}

A small C++ program using boost::numeric::ublas and std::complex,
plus some perl to generate part of that, was
written to generate the multiplication table for the gamma matrix products
for this particular basis.  The metric tensor and the antisymmetry of
the wedge products can be seen from these.

%% <GENERATED>


%
%
\begin{equation}\label{eqn:gamma:200}
\begin{aligned}
\gamma^0 \gamma^0 = \begin{bmatrix}
 1  &  0  &  0  &  0  \\
 0  &  1  &  0  &  0  \\
 0  &  0  &  1  &  0  \\
 0  &  0  &  0  &  1  \\
\end{bmatrix} \quad
\gamma^1 \gamma^1 = \begin{bmatrix}
 -1  &  0  &  0  &  0  \\
 0  &  -1  &  0  &  0  \\
 0  &  0  &  -1  &  0  \\
 0  &  0  &  0  &  -1  \\
\end{bmatrix}
\end{aligned}
\end{equation}
%
\begin{equation}\label{eqn:gamma:220}
\begin{aligned}
\gamma^2 \gamma^2 = \begin{bmatrix}
 -1  &  0  &  0  &  0  \\
 0  &  -1  &  0  &  0  \\
 0  &  0  &  -1  &  0  \\
 0  &  0  &  0  &  -1  \\
\end{bmatrix} \quad
\gamma^3 \gamma^3 = \begin{bmatrix}
 -1  &  0  &  0  &  0  \\
 0  &  -1  &  0  &  0  \\
 0  &  0  &  -1  &  0  \\
 0  &  0  &  0  &  -1  \\
\end{bmatrix}
\end{aligned}
\end{equation}
%
\begin{equation}\label{eqn:gamma:240}
\begin{aligned}
\gamma^0 \gamma^1 = \begin{bmatrix}
 0  &  0  &  0  &  1  \\
 0  &  0  &  1  &  0  \\
 0  &  1  &  0  &  0  \\
 1  &  0  &  0  &  0  \\
\end{bmatrix} \quad
\gamma^1 \gamma^0 = \begin{bmatrix}
 0  &  0  &  0  &  -1  \\
 0  &  0  &  -1  &  0  \\
 0  &  -1  &  0  &  0  \\
 -1  &  0  &  0  &  0  \\
\end{bmatrix}
\end{aligned}
\end{equation}
%
\begin{equation}\label{eqn:gamma:260}
\begin{aligned}
\gamma^0 \gamma^2 = \begin{bmatrix}
 0  &  0  &  0  &  -i  \\
 0  &  0  &  i  &  0  \\
 0  &  -i  &  0  &  0  \\
 i  &  0  &  0  &  0  \\
\end{bmatrix} \quad
\gamma^2 \gamma^0 = \begin{bmatrix}
 0  &  0  &  0  &  i  \\
 0  &  0  &  -i  &  0  \\
 0  &  i  &  0  &  0  \\
 -i  &  0  &  0  &  0  \\
\end{bmatrix}
\end{aligned}
\end{equation}
%
\begin{equation}\label{eqn:gamma:280}
\begin{aligned}
\gamma^0 \gamma^3 = \begin{bmatrix}
 0  &  0  &  1  &  0  \\
 0  &  0  &  0  &  -1  \\
 1  &  0  &  0  &  0  \\
 0  &  -1  &  0  &  0  \\
\end{bmatrix} \quad
\gamma^3 \gamma^0 = \begin{bmatrix}
 0  &  0  &  -1  &  0  \\
 0  &  0  &  0  &  1  \\
 -1  &  0  &  0  &  0  \\
 0  &  1  &  0  &  0  \\
\end{bmatrix}
\end{aligned}
\end{equation}
%
\begin{equation}\label{eqn:gamma:300}
\begin{aligned}
\gamma^1 \gamma^2 = \begin{bmatrix}
 -i  &  0  &  0  &  0  \\
 0  &  i  &  0  &  0  \\
 0  &  0  &  -i  &  0  \\
 0  &  0  &  0  &  i  \\
\end{bmatrix} \quad
\gamma^2 \gamma^1 = \begin{bmatrix}
 i  &  0  &  0  &  0  \\
 0  &  -i  &  0  &  0  \\
 0  &  0  &  i  &  0  \\
 0  &  0  &  0  &  -i  \\
\end{bmatrix}
\end{aligned}
\end{equation}
%
\begin{equation}\label{eqn:gamma:320}
\begin{aligned}
\gamma^1 \gamma^3 = \begin{bmatrix}
 0  &  1  &  0  &  0  \\
 -1  &  0  &  0  &  0  \\
 0  &  0  &  0  &  1  \\
 0  &  0  &  -1  &  0  \\
\end{bmatrix} \quad
\gamma^3 \gamma^1 = \begin{bmatrix}
 0  &  -1  &  0  &  0  \\
 1  &  0  &  0  &  0  \\
 0  &  0  &  0  &  -1  \\
 0  &  0  &  1  &  0  \\
\end{bmatrix}
\end{aligned}
\end{equation}
%
\begin{equation}\label{eqn:gamma:340}
\begin{aligned}
\gamma^2 \gamma^3 = \begin{bmatrix}
 0  &  -i  &  0  &  0  \\
 -i  &  0  &  0  &  0  \\
 0  &  0  &  0  &  -i  \\
 0  &  0  &  -i  &  0  \\
\end{bmatrix} \quad
\gamma^3 \gamma^2 = \begin{bmatrix}
 0  &  i  &  0  &  0  \\
 i  &  0  &  0  &  0  \\
 0  &  0  &  0  &  i  \\
 0  &  0  &  i  &  0  \\
\end{bmatrix}
\end{aligned}
\end{equation}
%
\begin{equation}\label{eqn:gamma:360}
\begin{aligned}
\gamma^1 \gamma^2 \gamma^3 = \begin{bmatrix}
 0  &  0  &  -i  &  0  \\
 0  &  0  &  0  &  -i  \\
 i  &  0  &  0  &  0  \\
 0  &  i  &  0  &  0  \\
\end{bmatrix} \quad
\gamma^2 \gamma^3 \gamma^0 = \begin{bmatrix}
 0  &  -i  &  0  &  0  \\
 -i  &  0  &  0  &  0  \\
 0  &  0  &  0  &  i  \\
 0  &  0  &  i  &  0  \\
\end{bmatrix}
\end{aligned}
\end{equation}
%
\begin{equation}\label{eqn:gamma:380}
\begin{aligned}
\gamma^3 \gamma^0 \gamma^1 = \begin{bmatrix}
 0  &  1  &  0  &  0  \\
 -1  &  0  &  0  &  0  \\
 0  &  0  &  0  &  -1  \\
 0  &  0  &  1  &  0  \\
\end{bmatrix} \quad
\gamma^0 \gamma^1 \gamma^2 = \begin{bmatrix}
 -i  &  0  &  0  &  0  \\
 0  &  i  &  0  &  0  \\
 0  &  0  &  i  &  0  \\
 0  &  0  &  0  &  -i  \\
\end{bmatrix}
\end{aligned}
\end{equation}
%
\begin{equation}\label{eqn:gamma:400}
\begin{aligned}
\gamma^0 \gamma^1 \gamma^2 \gamma^3 = \begin{bmatrix}
 0  &  0  &  -i  &  0  \\
 0  &  0  &  0  &  -i  \\
 -i  &  0  &  0  &  0  \\
 0  &  -i  &  0  &  0  \\
\end{bmatrix}
\end{aligned}
\end{equation}
%
%
%% </GENERATED>

   %
% Copyright � 2012 Peeter Joot.  All Rights Reserved.
% Licenced as described in the file LICENSE under the root directory of this GIT repository.
%

%
%
%\input{../peeter_prologue.tex}

\mychapter{Bivector form of quantum angular momentum operator}
\index{angular momentum operator}
\label{chap:qmAngularMom}

%\blogpage{http://sites.google.com/site/peeterjoot/math2009/qmAngularMom.pdf}
%%\date{July 27, 2009}
%%\revisionInfo{\(RCSfile: qmAngularMom.tex,v \) Last \(Revision: 1.15 \) \(Date: 2009/10/22 02:07:20 \)}

%\date{July 27, 2009.  \(RCSfile: qmAngularMom.tex,v \) Last \(Revision: 1.15 \) \(Date: 2009/10/22 02:07:20 \)}

\beginArtWithToc

\section{Spatial bivector representation of the angular momentum operator}

Reading \citep{bohm1989qt} on the angular momentum operator, the form of the operator is suggested by analogy where components of \(\Bx \cross \Bp\) with
the position representation \(\Bp \sim -i \Hbar \spacegrad\) used to expand the coordinate representation of the operator.

The result is the following coordinate representation of the operator

%\begin{align*}
%L_x &= -i \Hbar( y \partial_z - z \partial_y ) \\
%L_y &= -i \Hbar( z \partial_x - x \partial_z ) \\
%L_z &= -i \Hbar( x \partial_y - y \partial_x ) \\
%\end{align*}
\begin{equation}\label{eqn:qmAngularMom:37}
\begin{aligned}
L_1 &= -i \Hbar( x_2 \partial_3 - x_3 \partial_2 ) \\
L_2 &= -i \Hbar( x_3 \partial_1 - x_1 \partial_3 ) \\
L_3 &= -i \Hbar( x_1 \partial_2 - x_2 \partial_1 ) \\
\end{aligned}
\end{equation}
%
It is interesting to put these in vector form, and then employ the freedom to use for \(i = \sigma_1 \sigma_2 \sigma_3\) the spatial pseudoscalar.
%
\begin{equation}\label{eqn:qmAngularMom:57}
\begin{aligned}
\BL
&=
-\sigma_1 (\sigma_1 \sigma_2 \sigma_3) \Hbar( x_2 \partial_3 - x_3 \partial_2 )
-\sigma_2 (\sigma_2 \sigma_3 \sigma_1) \Hbar( x_3 \partial_1 - x_1 \partial_3 )
-\sigma_3 (\sigma_3 \sigma_1 \sigma_2) \Hbar( x_1 \partial_2 - x_2 \partial_1 ) \\
&=
-\sigma_2 \sigma_3 \Hbar( x_2 \partial_3 - x_3 \partial_2 )
-\sigma_3 \sigma_1 \Hbar( x_3 \partial_1 - x_1 \partial_3 )
-\sigma_1 \sigma_2 \Hbar( x_1 \partial_2 - x_2 \partial_1 ) \\
&=
-\Hbar ( \sigma_1 x_1 +\sigma_2 x_2 +\sigma_3 x_3 ) \wedge ( \sigma_1 \partial_1 +\sigma_2 \partial_2 +\sigma_3 \partial_3 ) \\
\end{aligned}
\end{equation}
%
The choice to use the pseudoscalar for this imaginary seems a logical one and the end result is a pure bivector representation of angular momentum operator
%
\begin{equation}\label{eqn:qmAngularMom:ang}
\BL = - \Hbar \Bx \wedge \spacegrad
\end{equation}
%
The choice to represent angular momentum as a bivector \(\Bx \wedge \Bp\) is also natural in classical mechanics (encoding the orientation of the plane and the magnitude of the momentum in the bivector), although its dual form the axial vector \(\Bx \cross \Bp\) is more common, at least in introductory mechanics.  Observe that there is no longer any explicit imaginary in \eqnref{eqn:qmAngularMom:ang}, since the bivector itself has an implicit complex structure.

\section{Factoring the gradient and Laplacian}

The form of \eqnref{eqn:qmAngularMom:ang} suggests a more direct way to extract the angular momentum operator from the Hamiltonian (i.e. from the Laplacian).  Bohm uses the spherical polar representation of the Laplacian as the starting point.  Instead let us project the gradient itself in a specific constant direction \(\Ba\), much as we can do to find the polar form angular velocity and acceleration components.

Write
%
\begin{equation}\label{eqn:qmAngularMom:77}
\begin{aligned}
\spacegrad
&=
\inv{\Ba} \Ba \spacegrad \\
&=
\inv{\Ba} (\Ba \cdot \spacegrad + \Ba \wedge \spacegrad) \\
\end{aligned}
\end{equation}
%
Or
\begin{equation}\label{eqn:qmAngularMom:97}
\begin{aligned}
\spacegrad
&=
\spacegrad \Ba \inv{\Ba} \\
&=
(\spacegrad \cdot \Ba + \spacegrad \wedge \Ba) \inv{\Ba} \\
&=
(\Ba \cdot \spacegrad - \Ba \wedge \spacegrad) \inv{\Ba} \\
\end{aligned}
\end{equation}
%
The Laplacian is therefore
%
\begin{equation}\label{eqn:qmAngularMom:117}
\begin{aligned}
\spacegrad^2
&=
\gpgradezero{ \spacegrad^2 } \\
&=
\gpgradezero{
(\Ba \cdot \spacegrad - \Ba \wedge \spacegrad) \inv{\Ba} \inv{\Ba} (\Ba \cdot \spacegrad + \Ba \wedge \spacegrad)
} \\
&=
\inv{\Ba^2} \gpgradezero{
(\Ba \cdot \spacegrad - \Ba \wedge \spacegrad) (\Ba \cdot \spacegrad + \Ba \wedge \spacegrad)
} \\
&=
\inv{\Ba^2} ((\Ba \cdot \spacegrad)^2 - (\Ba \wedge \spacegrad)^2 ) \\
\end{aligned}
\end{equation}
%
So we have for the Laplacian a representation in terms of projection and rejection components
%
\begin{equation}\label{eqn:qmAngularMom:137}
\begin{aligned}
\spacegrad^2
&=
(\acap \cdot \spacegrad)^2 - \inv{\Ba^2} (\Ba \wedge \spacegrad)^2 \\
&=
(\acap \cdot \spacegrad)^2 - (\acap \wedge \spacegrad)^2 \\
\end{aligned}
\end{equation}
%
The vector \(\Ba\) was arbitrary, and just needed to be constant with respect to the factorization operations.  Setting \(\Ba = \Bx\), the radial position from the origin one may guess that we have
%
\begin{equation}\label{eqn:qmAngularMom:wrong}
\spacegrad^2 = \frac{\partial^2 }{\partial r^2} - \inv{\Bx^2} (\Bx \wedge \spacegrad)^2
\end{equation}
%
however, with the switch to a non-constant position vector \(\Bx\), this cannot possibly be right.

\section{The Coriolis term}

The radial factorization of the gradient relied on the direction vector \(\Ba\) being constant.  If we evaluate \eqnref{eqn:qmAngularMom:wrong}, then there should be a non-zero remainder compared to the Laplacian.  Evaluation by coordinate expansion is one way to verify this, and should produce the difference.  Let us do this in two parts, starting with the scalar part of \((x \wedge \grad)^2\).  Summation will be implied by mixed indices, and for generality a general basis and associated reciprocal frame will be used.
%
\begin{equation}\label{eqn:qmAngularMom:157}
\begin{aligned}
\gpgradezero{ (x \wedge \grad)^2 } f
&=
((x^\mu \gamma_\mu) \wedge (\gamma_\nu \partial^\nu)) \cdot
((x_\alpha \gamma^\alpha) \wedge (\gamma^\beta \partial_\beta)) f \\
&=
(\gamma_\mu \wedge \gamma_\nu) \cdot (\gamma^\alpha \wedge \gamma^\beta) x^\mu \partial^\nu (x_\alpha \partial_\beta) f \\
&=
({\delta_\mu}^\beta {\delta_\nu}^\alpha -{\delta_\mu}^\alpha {\delta_\nu}^\beta) x^\mu \partial^\nu (x_\alpha \partial_\beta) f \\
&=
x^\mu \partial^\nu ((x_\nu \partial_\mu) - x_\mu \partial_\nu) f \\
&=
x^\mu (\partial^\nu x_\nu) \partial_\mu f - x^\mu (\partial^\nu x_\mu) \partial_\nu f \\
&+x^\mu x_\nu \partial^\nu \partial_\mu f - x^\mu x_\mu \partial^\nu \partial_\nu f \\
&=
(n-1) x \cdot \grad f +x^\mu x_\nu \partial^\nu \partial_\mu f - x^2 \grad^2 f \\
\end{aligned}
\end{equation}
%
For the dot product we have
\begin{equation}\label{eqn:qmAngularMom:177}
\begin{aligned}
\gpgradezero{ (x \cdot \grad)^2 } f
&=
x^\mu \partial_\mu( x^\nu \partial_\nu ) f \\
&=
x^\mu (\partial_\mu x^\nu) \partial_\nu  f + x^\mu x^\nu \partial_\mu \partial_\nu f \\
&=
x^\mu \partial_\mu f + x^\mu x_\nu \partial^\nu \partial_\mu f \\
&=
x \cdot \grad f + x^\mu x_\nu \partial^\nu \partial_\mu f \\
\end{aligned}
\end{equation}
%
So, forming the difference we have
%
\begin{equation}\label{eqn:qmAngularMom:197}
(x \cdot \grad)^2 f - \gpgradezero{(x \wedge \grad)^2} f =
-(n - 2) x \cdot \grad f + x^2 \grad^2 f \\
\end{equation}
%
Or
\begin{equation}\label{eqn:qmAngularMom:withScalar}
\grad^2 = \inv{x^2} (x \cdot \grad)^2 - \inv{x^2} \gpgradezero{(x \wedge \grad)^2} + (n - 2) \inv{x} \cdot \grad
\end{equation}
%
\section{On the bivector and quadvector components of the squared angular momentum operator}

The requirement for a scalar selection on all the \((x \wedge \grad)^2\) terms is a bit ugly, but omitting it would be incorrect for two reasons.  One reason is that this is a bivector operator and not a bivector (where the squaring operates on itself).  The other is that we derived a result for arbitrary dimension, and the product of two bivectors in a general space has grade 2 and grade 4 terms in addition to the scalar terms.  Without taking only the scalar parts, lets expand this product a bit more carefully, starting with
%
\begin{equation}\label{eqn:qmAngularMom:217}
(x \wedge \grad)^2
=
(\gamma_\mu \wedge \gamma_\nu) (\gamma^\alpha \wedge \gamma^\beta)
x^\mu \partial^\nu x_\alpha \partial_\beta
\end{equation}
%
Just expanding the multivector factor for now, we have
%
\begin{equation}\label{eqn:qmAngularMom:237}
\begin{aligned}
&2 (\gamma_\mu \wedge \gamma_\nu) (\gamma^\alpha \wedge \gamma^\beta) \\
&=
\gamma_\mu \gamma_\nu (\gamma^\alpha \wedge \gamma^\beta)
- \gamma_\nu \gamma_\mu (\gamma^\alpha \wedge \gamma^\beta) \\
&=
\gamma_\mu
\left(
{\delta_\nu}^\alpha \gamma^\beta - {\delta_\nu}^\beta \gamma^\alpha
+ \gamma_\nu \wedge \gamma^\alpha \wedge \gamma^\beta \right)
-
\gamma_\nu \left(
{\delta_\mu}^\alpha \gamma^\beta - {\delta_\mu}^\beta \gamma^\alpha
+ \gamma_\mu \wedge \gamma^\alpha \wedge \gamma^\beta \right)
\\
&=
{\delta_\nu}^\alpha {\delta_\mu}^\beta - {\delta_\nu}^\beta {\delta_\mu}^\alpha
-{\delta_\mu}^\alpha {\delta_\nu}^\beta + {\delta_\mu}^\beta {\delta_\nu}^\alpha \\
&+ \gamma_\mu \wedge \gamma_\nu \wedge \gamma^\alpha \wedge \gamma^\beta
- \gamma_\nu \wedge \gamma_\mu \wedge \gamma^\alpha \wedge \gamma^\beta \\
&+ \gamma_\mu \cdot (\gamma_\nu \wedge \gamma^\alpha \wedge \gamma^\beta )
-\gamma_\nu \cdot (\gamma_\mu \wedge \gamma^\alpha \wedge \gamma^\beta ) \\
\end{aligned}
\end{equation}
%
Our split into grades for this operator is then, the scalar
%
\begin{equation}\label{eqn:qmAngularMom:257}
\begin{aligned}
\gpgradezero{(x \wedge \grad)^2 }
&= (x \wedge \grad) \cdot (x \wedge \grad) \\
&= \lr{ {\delta_\nu}^\alpha {\delta_\mu}^\beta - {\delta_\nu}^\beta {\delta_\mu}^\alpha  }
x^\mu \partial^\nu x_\alpha \partial_\beta \\
\end{aligned}
\end{equation}
%
the pseudoscalar (or grade 4 term in higher than 4D spaces).
%
\begin{equation}\label{eqn:qmAngularMom:277}
\begin{aligned}
\gpgradefour{(x \wedge \grad)^2 }
&= (x \wedge \grad) \wedge (x \wedge \grad) \\
&= \lr{ \gamma_\mu \wedge \gamma_\nu \wedge \gamma^\alpha \wedge \gamma^\beta   }
x^\mu \partial^\nu x_\alpha \partial_\beta \\
\end{aligned}
\end{equation}
%
If we work in dimensions less than or equal to three, we will have no grade four term since this wedge product is zero (irrespective of the operator action), so in 3D we have only a bivector term in excess of the scalar part of this operator.

The bivector term deserves some reduction, but is messy to do so.  This has been done separately in (\chapcite{bivectorSelect})

We can now write for the squared operator
%
\begin{equation}\label{eqn:qmAngularMom:onGradeTwo3}
(x \wedge \grad)^2 =
(n-2)(x \wedge \grad)
+
(x \wedge \grad) \wedge (x \wedge \grad)
+(x \wedge \grad) \cdot (x \wedge \grad)
\end{equation}
%
and then eliminate the scalar selection from the \eqnref{eqn:qmAngularMom:withScalar}
%
\begin{equation}\label{eqn:qmAngularMom:onGradeTwo4}
\grad^2 = \inv{x^2} (x \cdot \grad)^2 + (n - 2) \inv{x} \cdot \grad
- \inv{x^2}
\left(
(x \wedge \grad)^2 - (n-2) (x \wedge \grad) - (x \wedge \grad) \wedge (x \wedge \grad)
\right)
\end{equation}
%
In 3D this is
%
\begin{equation}\label{eqn:qmAngularMom:onGradeTwo5}
\spacegrad^2 = \inv{\Bx^2} (\Bx \cdot \spacegrad)^2 + \inv{\Bx} \cdot \spacegrad
- \inv{\Bx^2} \lr{ \Bx \wedge \spacegrad - 1  } (\Bx \wedge \spacegrad)
\end{equation}
%
Wow, that was an ugly mess of algebra.  The worst of it for the bivector grades was initially incorrect and the correct handling omitted.  There is likely a more clever coordinate free way to do the same expansion.  We will see later that at least a partial verification of \eqnref{eqn:qmAngularMom:onGradeTwo5} can be obtained by considering of the Quantum eigenvalue problem, examining simultaneous eigenvalues of \(\Bx \wedge \spacegrad\), and \(\gpgradezero{\Bx \wedge \spacegrad)^2}\).  However, lets revisit this after examining the radial terms in more detail, and also after verifying that at least in the scalar selection form, this factorized Laplacian form has the same structure as the Laplacian in scalar \(r\), \(\theta\), and \(\phi\) operator form.

FIXME: the reduction of the scalar selection term doesn't look right: and appears to leave a bivector term in an otherwise scalar equation.  With that term in place, this doens't match the same identity \citep{sakurai2014modern} \texteqnref{6.16}, whereas \cref{eqn:qmSum:goo6} does.  Does that cancel out when \( \lr{ \Bx \wedge \spacegrad }^2 \) is expanded?

\section{Correspondence with explicit radial form}

We have seen above that we can factor the 3D Laplacian as
%
\begin{equation}\label{eqn:qmAngularMom:3dLaplacian}
\spacegrad^2 \psi = \inv{\Bx^2}( (\Bx \cdot \spacegrad)^2 + \Bx \cdot \spacegrad - \gpgradezero{ (\Bx \wedge \spacegrad)^2 } ) \psi
\end{equation}
%
Contrast this to the explicit \(r,\theta,\phi\) form as given in (Bohm's \citep{bohm1989qt}, 14.2)
%
\begin{equation}\label{eqn:qmAngularMom:LaplacianRTP}
\spacegrad^2 \psi = \inv{r} \frac{\partial^2}{\partial r^2} (r\psi) + \inv{r^2} \left(
\inv{\sin\theta} \partial_\theta \sin\theta \partial_\theta + \inv{\sin^2\theta} + \partial_{\phi \phi} \right) \psi
\end{equation}
%
Let us expand out the non-angular momentum operator terms explicitly as a partial verification of this factorization.  The radial term in Bohm's Laplacian formula expands out to
%
\begin{equation}\label{eqn:qmAngularMom:297}
\begin{aligned}
\inv{r} \frac{\partial^2}{\partial r^2} (r\psi)
&=
\inv{r} \partial_r (\partial_r r \psi) \\
&=
\inv{r} \partial_r (\psi + r\partial_r \psi) \\
&=
\inv{r} \partial_r \psi + \inv{r}( \partial_r \psi + r \partial_{rr} \psi) \\
&=
\frac{2}{r} \partial_r \psi + \partial_{rr} \psi \\
\end{aligned}
\end{equation}
%
On the other hand, with \(\Bx = r\rcap\), what we expect to correspond to the radial term in the vector factorization is
%
\begin{equation}\label{eqn:qmAngularMom:317}
\begin{aligned}
\inv{\Bx^2}( (\Bx \cdot \spacegrad)^2 + \Bx \cdot \spacegrad ) \psi
&=
\inv{r^2}( (r \rcap \cdot \spacegrad)^2 + r \rcap \cdot \spacegrad  ) \psi \\
&=
\inv{r^2}( (r \partial_r )^2 + r \partial_r  ) \psi \\
&=
\inv{r^2}( r \partial_r \psi + r^2 \partial_{rr} \psi + r \partial_r \psi ) \\
&=
\frac{2}{r} \partial_r \psi + \partial_{rr} \psi
\end{aligned}
\end{equation}
%
Okay, good.  It is a brute force way to verify things, but it works.  With \(\Bx \wedge \spacegrad = I (\Bx \cross \spacegrad)\) we can eliminate the wedge product from the factorization expression \eqnref{eqn:qmAngularMom:3dLaplacian} and express things completely in quantities that can be understood without any resort to Geometric Algebra.  That is
%
\begin{equation}\label{eqn:qmAngularMom:LaplacianRadialCross}
\spacegrad^2 \psi = \inv{r} \frac{\partial^2}{\partial r^2} (r\psi) + \inv{r^2} \gpgradezero{ (\Bx \cross \spacegrad)^2 } \psi
\end{equation}
%
Bohm resorts to analogy and an operatorization of \(L_c = \epsilon_{abc} (x_a p_b - x_b p_a)\), then later a spherical polar change of coordinates to match exactly the \(L^2\) expression with \eqnref{eqn:qmAngularMom:LaplacianRTP}.  With the GA formalism we see this a bit more directly, although it is not the least bit obvious that the operator \(\Bx \cross \spacegrad\) has no radial dependence.  Without resorting to a comparison with the explicit \(r,\theta,\phi\) form that would not be so easy to see.

\section{Raising and Lowering operators in GA form}

Having seen in (\chapcite{L1Associated}) that we have a natural GA form for the \(l=1\) spherical harmonic eigenfunctions \(\psi_1^{m}\), and that we have the vector angular momentum operator \(\Bx \cross \spacegrad\) showing up directly in a sort-of-radial factorization of the Laplacian, it is natural to wonder what the GA form of the raising and lowering operators are.  At least for the \(l=1\) harmonics use of \(i = I \Be_3\) (unit bivector for the \(x-y\) plane) for the imaginary ended up providing a nice geometric interpretation.

Let us see what that provides for the raising and lowering operators.  First we need to express \(L_x\) and \(L_y\) in terms of our bivector angular momentum operator.  Let us switch notations and drop the \(-i \Hbar\) factor from \eqnref{eqn:qmAngularMom:ang} writing just
%
\begin{equation}\label{eqn:qmAngularMom:Ang}
\BL = \Bx \wedge \spacegrad
\end{equation}
%
We can now write this in terms of components with respect to the basis bivectors \(I \Be_k\).  That is
%
\begin{equation}\label{eqn:qmAngularMom:BLprojected}
\BL = \sum_k \lr{ (\Bx \wedge \spacegrad) \cdot \inv{I \Be_k} } I \Be_k
\end{equation}
%
These scalar product results are expected to match the \(L_x\), \(L_y\), and \(L_z\) components at least up to a sign.  Let us check, picking \(L_z\) as representative
%
\begin{equation}\label{eqn:qmAngularMom:337}
\begin{aligned}
(\Bx \wedge \spacegrad) \cdot \inv{I \Be_3}
&=
(\sigma_m \wedge \sigma^k) \cdot {-\sigma_1 \sigma_2 \sigma_3 \sigma_3} x^m \partial_k \\
&=
(\sigma_m \wedge \sigma^k) \cdot {-\sigma_1 \sigma_2} x^m \partial_k \\
&=
-( x^2 \partial_1 - x^1 \partial_2 )
\end{aligned}
\end{equation}
%
With the \(-i\Hbar\) factors dropped this is \(L_z = L_3 = x^1 \partial_2 - x^2 \partial_1\), the projection of \(\BL\) onto the \(x-y\) plane \(I \Be_k\).  So, now how about the raising and lowering operators
%
\begin{equation}\label{eqn:qmAngularMom:357}
\begin{aligned}
L_x \pm i L_y
&=
L_x \pm I \Be_3 L_y \\
&=
\BL \cdot \inv{I\Be_1} \pm I \Be_3 \BL \cdot \inv{I\Be_2} \\
&=
-\Be_1 I \lr{ I \Be_1 \BL \cdot \inv{I\Be_1} \pm I \Be_2 \BL \cdot \inv{I\Be_2}  } \\
\end{aligned}
\end{equation}
%
Or
\begin{equation}\label{eqn:qmAngularMom:raisingLoweringBivector}
(I \Be_1) L_x \pm i L_y = I \Be_1 \BL \cdot \inv{I\Be_1} \pm I \Be_2 \BL \cdot \inv{I\Be_2}
\end{equation}
%
Compare this to the projective split of \(\BL\) \eqnref{eqn:qmAngularMom:BLprojected}.  We have projections of the bivector angular momentum operator onto the bivector directions \(I\Be_1\) and \(I\Be_2\) (really the bivectors for the planes perpendicular to the \(\xcap\) and \(\ycap\) directions).

We have the Laplacian in explicit vector form and have a clue how to vectorize (really bivectorize) the raising and lowering operators.  We have also seen how to geometrize the first spherical harmonics.  The next logical step is to try to apply this vector form of the raising and lowering operators to the vector form of the spherical harmonics.

\section{Explicit expansion of the angular momentum operator}

There is a couple of things to explore before going forward.  One is an explicit verification that \(\Bx \wedge \spacegrad\) has no radial dependence (something not obvious).  Another is that we should be able to compare the \(\Bx^{-2} (\Bx \wedge \spacegrad)^2\) (as done for the \(\Bx \cdot \spacegrad\) terms) the explicit \(r,\theta,\phi\) expression for the Laplacian to verify consistency and correctness.

For the spherical polar rotation we use the rotor
%
\begin{equation}\label{eqn:qmAngularMom:angExp1}
R = e^{\Be_{31}\theta/2} e^{\Be_{12}\phi/2}
\end{equation}
%
Our position vector and gradient in spherical polar coordinates are
%
\begin{equation}\label{eqn:qmAngularMom:angExp2}
\Bx = r \tilde{R} \Be_3 R
\end{equation}
%
\begin{equation}\label{eqn:qmAngularMom:angExp3}
\spacegrad = \rcap \partial_r + \thetacap \inv{r} \partial_\theta + \phicap \inv{r \sin\theta} \partial_\phi
\end{equation}
%
with the unit vectors translate from the standard basis as
%
\begin{equation}\label{eqn:qmAngularMom:angExp4}
\begin{pmatrix}
\rcap \\
\thetacap \\
\phicap \\
\end{pmatrix}
=
\tilde{R}
\begin{pmatrix}
\Be_3 \\
\Be_1 \\
\Be_2 \\
\end{pmatrix}
R
\end{equation}
%
This last mapping can be used to express the gradient unit vectors in terms of the standard basis, as we did for the position vector \(\Bx\).  That is
%
\begin{equation}\label{eqn:qmAngularMom:angExp5}
\spacegrad = \tilde{R} \lr{ \Be_3 R \partial_r + \Be_1 R \inv{r} \partial_\theta + \Be_2 R \inv{r \sin\theta} \partial_\phi  }
\end{equation}
%
Okay, we have now got all the pieces collected, ready to evaluate \(\Bx \wedge \spacegrad\)
%
\begin{equation}\label{eqn:qmAngularMom:377}
\begin{aligned}
\Bx \wedge \spacegrad
&=
r \gpgradetwo{\tilde{R} \Be_3 R
\tilde{R} \lr{ \Be_3 R \partial_r + \Be_1 R \inv{r} \partial_\theta + \Be_2 R \inv{r \sin\theta} \partial_\phi  } } \\
&=
r \gpgradetwo{\tilde{R} \lr{ R \partial_r + \Be_3 \Be_1 R \inv{r} \partial_\theta + \Be_3 \Be_2 R \inv{r \sin\theta} \partial_\phi  } } \\
\end{aligned}
\end{equation}
%
Observe that the \({\Be_3}^2\) contribution is only a scalar, so bivector selection of that is zero.  In the remainder we have cancellation of \(r/r\) factors, leaving just
%
\begin{equation}\label{eqn:qmAngularMom:angExp6}
\Bx \wedge \spacegrad
=
\tilde{R} \lr{ \Be_3 \Be_1 R \partial_\theta + \Be_3 \Be_2 R \inv{\sin\theta} \partial_\phi  }
\end{equation}
%
Using \eqnref{eqn:qmAngularMom:angExp4} this is
%
\begin{equation}\label{eqn:qmAngularMom:angExp7}
\Bx \wedge \spacegrad
=
\rcap \lr{ \thetacap \partial_\theta + \phicap \inv{\sin\theta} \partial_\phi  }
\end{equation}
%
As hoped, there is no explicit radial dependence here, taking care of the first of the desired verifications.

Next we want to square this operator.  It should be noted that in the original derivation where we ``factored'' the gradient operator with respect to the reference vector \(\Bx\) our Laplacian really followed by considering \((\Bx \wedge \spacegrad)^2 \equiv \gpgradezero{(\Bx \wedge \spacegrad)^2}\).  That is worth noting since a regular bivector would square to a negative constant, whereas the operator factors of the vectors in this expression do not intrinsically commute.

An additional complication for evaluating the square of \(\Bx \wedge \spacegrad\) using the result of \eqnref{eqn:qmAngularMom:angExp7} is that \(\thetacap\) and \(\rcap\) are functions of \(\theta\) and \(\phi\), so we would have to operate on those too.  Without that operator subtlety we get the wrong answer
%
\begin{equation}\label{eqn:qmAngularMom:397}
\begin{aligned}
-\gpgradezero{ (\Bx \wedge \spacegrad)^2 }
&=
\gpgradezero{
\tilde{R} \lr{ \Be_1 R \partial_\theta + \frac{\Be_2 R}{\sin\theta}\partial_\phi  }
\tilde{R} \lr{ \Be_1 R \partial_\theta + \frac{\Be_2 R}{\sin\theta}\partial_\phi  }
 } \\
&\ne
\partial_{\theta\theta} + \inv{\sin^2\theta} \partial_{\phi\phi}
\end{aligned}
\end{equation}
%
Equality above would only be if the unit vectors were fixed.  By comparison we also see that this is missing a \(\cot\theta \partial_\theta\) term.  That must come from the variation of the unit vectors with position in the second application of \(\Bx \wedge \spacegrad\).

\section{Derivatives of the unit vectors}

To properly evaluate the angular momentum square we will need to examine the \(\partial_\theta\) and \(\partial_\phi\) variation of the unit vectors \(\rcap\), \(\thetacap\), and \(\phicap\).  Some part of this question can be evaluated without reference to the specific vector or even which derivative is being evaluated.  Writing \(e\) for one of \(\Be_1\), \(\Be_2\), or \(\Be_k\), and \(\sigma = \tilde{R} e R\) for the mapping of this vector under rotation, and \(\partial\) for the desired \(\theta\) or \(\phi\) partial derivative, we have
%
\begin{equation}\label{eqn:qmAngularMom:rotationGen1}
\partial (\tilde{R} e R)
=
(\partial \tilde{R}) e R + \tilde{R} e (\partial R)
\end{equation}
%
Since \(\tilde{R} R = 1\), we have
%
\begin{equation}\label{eqn:qmAngularMom:417}
\begin{aligned}
0
&= \partial (\tilde{R} R) \\
&=
(\partial \tilde{R}) R + \tilde{R} (\partial R)
\end{aligned}
\end{equation}
%
So substitution of \((\partial \tilde{R}) = -\tilde{R} (\partial R) \tilde{R}\), back into \eqnref{eqn:qmAngularMom:rotationGen1} supplies
%
\begin{equation}\label{eqn:qmAngularMom:437}
\begin{aligned}
\partial (\tilde{R} e R)
&=
-\tilde{R} (\partial R) \tilde{R} e R + \tilde{R} e (\partial R) \\
&=
-\tilde{R} (\partial R) (\tilde{R} e R) + (\tilde{R} e R) \tilde{R} (\partial R) \\
&=
-\tilde{R} (\partial R) \sigma + \sigma \tilde{R} (\partial R) \\
\end{aligned}
\end{equation}
%
Writing the bivector term as
%
\begin{equation}\label{eqn:qmAngularMom:rotationGen2}
\Omega = \tilde{R} (\partial R)
\end{equation}
%
The change in the rotated vector is seen to be entirely described by the commutator of that vectors image under rotation with \(\Omega\).  That is
%
\begin{equation}\label{eqn:qmAngularMom:rotationGen3}
\partial \sigma = \antisymmetric{\sigma}{\Omega}
\end{equation}
%
Our spherical polar rotor was given by
%
\begin{equation}\label{eqn:qmAngularMom:rotationGen4}
R = e^{\Be_{31}\theta/2} e^{\Be_{12}\phi/2}
\end{equation}
%
Lets calculate the \(\Omega\) bivector for each of the \(\theta\) and \(\phi\) partials.  For \(\theta\) we have
%
\begin{equation}\label{eqn:qmAngularMom:457}
\begin{aligned}
\Omega_\theta &= \tilde{R} \partial_\theta R \\
&= \frac{1}{2} e^{-\Be_{12}\phi/2} e^{-\Be_{31}\theta/2} \Be_{31} e^{\Be_{31}\theta/2} e^{\Be_{12}\phi/2} \\
&= \frac{1}{2} e^{-\Be_{12}\phi/2} \Be_{31} e^{\Be_{12}\phi/2} \\
&= \frac{1}{2} \Be_3 e^{-\Be_{12}\phi/2} \Be_{1} e^{\Be_{12}\phi/2} \\
&= \frac{1}{2} \Be_{31} e^{\Be_{12}\phi} \\
\end{aligned}
\end{equation}
%
Explicitly, this is the bivector \(\Omega_\theta = (\Be_{31} \cos\theta + \Be_{32} \sin\theta)/2\), a wedge product of a vectors in \(\zcap\) direction with one in the perpendicular \(x-y\) plane (curiously a vector in the \(x-y\) plane rotated by polar angle \(\theta\), not the equatorial angle \(\phi\)).

FIXME: picture.  Draw this plane cutting through the sphere.

For the \(\phi\) partial variation of any of our unit vectors our bivector rotation generator is
%
\begin{equation}\label{eqn:qmAngularMom:477}
\begin{aligned}
\Omega_\phi &= \tilde{R} \partial_\phi R \\
&= \frac{1}{2} e^{-\Be_{12}\phi/2} e^{-\Be_{31}\theta/2} e^{\Be_{31}\theta/2} \Be_{12} e^{\Be_{12}\phi/2} \\
&= \frac{1}{2} \Be_{12} \\
\end{aligned}
\end{equation}
%
This one has no variation at all with angle whatsoever.  If this is all correct so far perhaps that is not surprising given the fact that we expect an extra \(\cot\theta\) in the angular momentum operator square, so a lack of \(\phi\) dependence in the rotation generator likely means that any additional \(\phi\) dependence will cancel out.  Next step is to take these rotation generator bivectors, apply them via commutator products to the \(\rcap\), \(\thetacap\), and \(\phicap\) vectors, and see what we get.

\section{Applying the vector derivative commutator (or not)}

Let us express the \(\thetacap\) and \(\phicap\) unit vectors explicitly in terms of the standard basis.  Starting with \(\thetacap\) we have
%
\begin{equation}\label{eqn:qmAngularMom:497}
\begin{aligned}
\thetacap
&= \tilde{R} \Be_1 R \\
&= e^{-\Be_{12}\phi/2} e^{-\Be_{31}\theta/2} \Be_1 e^{\Be_{31}\theta/2} e^{\Be_{12}\phi/2} \\
&= e^{-\Be_{12}\phi/2} \Be_1 e^{\Be_{31}\theta} e^{\Be_{12}\phi/2} \\
&= e^{-\Be_{12}\phi/2} (\Be_1 \cos\theta -\Be_3 \sin\theta) e^{\Be_{12}\phi/2} \\
&= \Be_1 \cos\theta e^{\Be_{12}\phi} -\Be_3 \sin\theta
\end{aligned}
\end{equation}
%
Explicitly in vector form, eliminating the exponential, this is \(\thetacap = \Be_1 \cos\theta \cos\phi + \Be_2 \cos\theta\sin\phi - \Be_3\sin\theta\), but it is more convenient to keep the exponential as is.
%
For \(\phicap\) we have
\begin{equation}\label{eqn:qmAngularMom:517}
\begin{aligned}
\phicap
&= \tilde{R} \Be_2 R \\
&= e^{-\Be_{12}\phi/2} e^{-\Be_{31}\theta/2} \Be_2 e^{\Be_{31}\theta/2} e^{\Be_{12}\phi/2} \\
&= e^{-\Be_{12}\phi/2} \Be_2 e^{\Be_{12}\phi/2} \\
&= \Be_2 e^{\Be_{12}\phi} \\
\end{aligned}
\end{equation}
%
Again, explicitly this is \(\phicap = \Be_2 \cos\phi - \Be_1 \sin\phi\), but we will use the exponential form above.  Last we want \(\rcap\)
%
\begin{equation}\label{eqn:qmAngularMom:537}
\begin{aligned}
\rcap
&= \tilde{R} \Be_3 R \\
&= e^{-\Be_{12}\phi/2} e^{-\Be_{31}\theta/2} \Be_3 e^{\Be_{31}\theta/2} e^{\Be_{12}\phi/2} \\
&= e^{-\Be_{12}\phi/2} \Be_3 e^{\Be_{31}\theta} e^{\Be_{12}\phi/2} \\
&= e^{-\Be_{12}\phi/2} (\Be_3 \cos\theta + \Be_1 \sin\theta) e^{\Be_{12}\phi/2} \\
&= \Be_3 \cos\theta + \Be_1 \sin\theta e^{\Be_{12}\phi} \\
\end{aligned}
\end{equation}
%
Summarizing we have
\begin{equation}\label{eqn:qmAngularMom:vecBivecDerivatives1}
\begin{aligned}
\thetacap &= \Be_1 \cos\theta e^{\Be_{12}\phi} -\Be_3 \sin\theta \\
\phicap &= \Be_2 e^{\Be_{12}\phi} \\
\rcap &= \Be_3 \cos\theta + \Be_1 \sin\theta e^{\Be_{12}\phi}
\end{aligned}
\end{equation}
%
Or without exponentials
\begin{equation}\label{eqn:qmAngularMom:vecBivecDerivatives2}
\begin{aligned}
\thetacap &= \Be_1 \cos\theta \cos\phi + \Be_2 \cos\theta\sin\phi - \Be_3\sin\theta \\
\phicap &= \Be_2 \cos\phi - \Be_1 \sin\phi \\
\rcap &= \Be_3 \cos\theta + \Be_1 \sin\theta \cos\phi + \Be_2 \sin\theta \sin\phi
\end{aligned}
\end{equation}
%
Now, having worked out the cool commutator result, it appears that it will actually be harder to use it, then to just calculate the derivatives directly (at least for the \(\phicap\) derivatives).  For those we have
%
\begin{equation}\label{eqn:qmAngularMom:557}
\begin{aligned}
\partial_\theta \phicap
&= \partial_\theta \Be_2 e^{\Be_{12}\phi} \\
&= 0
\end{aligned}
\end{equation}
%
and
%
\begin{equation}\label{eqn:qmAngularMom:577}
\begin{aligned}
\partial_\phi \phicap
&= \partial_\phi \Be_2 e^{\Be_{12}\phi} \\
&= \Be_2 \Be_{12} e^{\Be_{12}\phi} \\
&= -\Be_{12} \phicap
\end{aligned}
\end{equation}
%
This multiplication takes \(\phicap\) a vector in the \(x,y\) plane and rotates it 90 degrees, leaving an inwards facing radial unit vector in the x,y plane.

Now, having worked out the commutator method, lets at least verify that it works.
%
\begin{equation}\label{eqn:qmAngularMom:597}
\begin{aligned}
\partial_\theta \phicap
&= \antisymmetric{\phicap}{\Omega_\theta} \\
&= \phicap \Omega_\theta - \Omega_\theta \phicap \\
&= \inv{2} (\phicap \Be_{31} e^{\Be_{12}\phi} - \Be_{31} e^{\Be_{12}\phi} \phicap)  \\
&= \inv{2} (\Be_2 \Be_3 \Be_1 e^{-\Be_{12}\phi} e^{\Be_{12}\phi} - \Be_{3} \Be_{1} \Be_{2} e^{-\Be_{12}\phi} e^{\Be_{12}\phi})  \\
&= \inv{2} (-\Be_3 \Be_2 \Be_1 - \Be_{3} \Be_{1} \Be_{2} )  \\
&= 0
\end{aligned}
\end{equation}
%
Much harder this way compared to taking the derivative directly, but we at least get the right answer.  For the \(\phi\) derivative using the commutator we have
%
\begin{equation}\label{eqn:qmAngularMom:617}
\begin{aligned}
\partial_\phi \phicap
&= \antisymmetric{\phicap}{\Omega_\phi} \\
&= \phicap \Omega_\phi - \Omega_\phi \phicap \\
&= \inv{2} (\phicap \Be_{12} - \Be_{12} \phicap)  \\
&= \inv{2} (\Be_2 e^{\Be_{12}\phi} \Be_{12} - \Be_{12} \Be_2 e^{\Be_{12}\phi})  \\
&= \inv{2} (-\Be_{12} \Be_2 e^{\Be_{12}\phi} - \Be_{12} \Be_2 e^{\Be_{12}\phi})  \\
&= -\Be_{12} \phicap
\end{aligned}
\end{equation}
%
Good, also consistent with direct calculation.  How about our \(\thetacap\) derivatives?  Lets just calculate these directly without bothering at all with the commutator.  This is
%
\begin{equation}\label{eqn:qmAngularMom:637}
\begin{aligned}
\partial_\phi \thetacap
&= \Be_1 \cos\theta \Be_12 e^{\Be_{12}\phi}  \\
&= \Be_2 \cos\theta e^{\Be_{12}\phi}  \\
&= \cos\theta \phicap
\end{aligned}
\end{equation}
%
and
\begin{equation}\label{eqn:qmAngularMom:657}
\begin{aligned}
\partial_\theta \thetacap
&= -\Be_1 \sin\theta e^{\Be_{12}\phi} -\Be_3 \cos\theta \\
&= -\Be_{12} \sin\theta \phicap -\Be_3 \cos\theta \\
\end{aligned}
\end{equation}
%
Finally, last we have the derivatives of \(\rcap\).  Those are
%
\begin{equation}\label{eqn:qmAngularMom:677}
\begin{aligned}
\partial_\phi \rcap
&= \Be_2 \sin\theta e^{\Be_{12}\phi} \\
&= \sin\theta \phicap
\end{aligned}
\end{equation}
%
and
\begin{equation}\label{eqn:qmAngularMom:697}
\begin{aligned}
\partial_\theta \rcap
&= -\Be_3 \sin\theta + \Be_1 \cos\theta e^{\Be_{12}\phi} \\
&= -\Be_3 \sin\theta + \Be_{12} \cos\theta \phicap \\
\end{aligned}
\end{equation}
%
Summarizing, all the derivatives we need to evaluate the square of the angular momentum operator are
\begin{equation}\label{eqn:qmAngularMom:vecBivecDerivatives3}
\begin{aligned}
\partial_\theta \phicap &= 0 \\
\partial_\phi \phicap &= -\Be_{12} \phicap \\
\partial_\theta \thetacap &= -\Be_{12} \sin\theta \phicap -\Be_3 \cos\theta  \\
\partial_\phi \thetacap &= \cos\theta \phicap \\
\partial_\theta \rcap &= -\Be_3 \sin\theta + \Be_{12} \cos\theta \phicap \\
\partial_\phi \rcap &= \sin\theta \phicap
\end{aligned}
\end{equation}
%
Bugger.  We actually want the derivatives of the bivectors \(\rcap\thetacap\) and \(\rcap\phicap\) so we are not ready to evaluate the squared angular momentum.  There is three choices, one is to use these results and apply the chain rule, or start over and directly take the derivatives of these bivectors, or use the commutator result (which did not actually assume vectors and we can apply it to bivectors too if we really wanted to).

An attempt to use the chain rule get messy, but it looks like the bivectors reduce nicely, making it pointless to even think about the commutator method.  Introducing some notational conveniences, first write \(i = \Be_{12}\).  We will have to be a bit careful with this since it commutes with \(\Be_3\), but anticommutes with \(\Be_1\) or \(\Be_2\) (and therefore \(\phicap\)).  As usual we also write \(I = \Be_1 \Be_2 \Be_3\) for the Euclidean pseudoscalar (which commutes with all vectors and bivectors).
%
\begin{equation}\label{eqn:qmAngularMom:717}
\begin{aligned}
\rcap \thetacap
&= (\Be_3 \cos\theta + i \sin\theta \phicap)(\cos\theta i \phicap - \Be_3 \sin\theta) \\
&=
\Be_3 \cos^2\theta i \phicap -i \sin^2\theta \phicap \Be_3 +(i \phicap i \phicap -\Be_3 \Be_3 ) \cos\theta \sin\theta \\
&=
i \Be_3 (\cos^2\theta + \sin^2\theta) \phicap +(-\phicap i^2 \phicap - 1) \cos\theta \sin\theta \\
\end{aligned}
\end{equation}
%
This gives us just
%
\begin{equation}\label{eqn:qmAngularMom:vecBivecDerivatives4}
\rcap \thetacap = I \phicap
\end{equation}
%
and calculation of the bivector partials will follow exclusively from the \(\phicap\) partials tabulated above.
%
Our other bivector does not reduce quite as cleanly.  We have
\begin{equation}\label{eqn:qmAngularMom:737}
\rcap \phicap
=
(\Be_3 \cos\theta + i \sin\theta \phicap) \phicap
\end{equation}
%
So for this one we have
%
\begin{equation}\label{eqn:qmAngularMom:vecBivecDerivatives5}
\rcap \phicap = \Be_3 \phicap \cos\theta + i \sin\theta
\end{equation}
%
Tabulating all the bivector derivatives (details omitted) we have
%
\begin{equation}\label{eqn:qmAngularMom:vecBivecDerivatives6}
\begin{aligned}
\partial_\theta (\rcap \thetacap) &= 0 \\
\partial_\phi (\rcap \thetacap) &= \Be_3 \phicap \\
\partial_\theta (\rcap \phicap) &= -\Be_3 \phicap \sin\theta + i \cos\theta
= i e^{I\phicap\theta} \\
\partial_\phi (\rcap \phicap) &= -I \phicap \cos\theta
\end{aligned}
\end{equation}
%
Okay, we should now be armed to do the squaring of the angular momentum.

\section{Squaring the angular momentum operator}

It is expected that we have the equivalence of the squared bivector form of angular momentum with the classical scalar form in terms of spherical angles \(\phi\), and \(\theta\).  Specifically, if no math errors have been made playing around with this GA representation, we should have the following identity for the scalar part of the squared angular momentum operator
%
\begin{equation}\label{eqn:qmAngularMom:squaring1}
-\gpgradezero{ (\Bx \wedge \spacegrad)^2 } =
\inv{\sin\theta} \PD{\theta}{} \sin\theta \PD{\theta}{} + \inv{\sin^2\theta} \frac{\partial^2}{\partial \phi^2}
\end{equation}
%
To finally attempt to verify this we write the angular momentum operator in polar form, using \(i = \Be_1 \Be_2\) as
%
\begin{equation}\label{eqn:qmAngularMom:squaring0}
\Bx \wedge \spacegrad
=
\rcap \lr{ \thetacap \partial_\theta + \phicap \inv{\sin\theta} \partial_\phi  }
\end{equation}
%
Expressing the unit vectors in terms of \(\phicap\) and after some rearranging we have
%
\begin{equation}\label{eqn:qmAngularMom:squaring2}
\Bx \wedge \spacegrad
=
I \phicap \lr{ \partial_\theta + i e^{I\phicap \theta} \inv{\sin\theta} \partial_\phi  }
\end{equation}
%
Using this lets now compute the partials.  First for the \(\theta\) partials we have
%
\begin{equation}\label{eqn:qmAngularMom:757}
\begin{aligned}
\partial_\theta (\Bx \wedge \spacegrad)
&=
I \phicap \left(
\partial_{\theta\theta}
+ i I \phicap e^{I\phicap \theta} \inv{\sin\theta} \partial_\phi
+ i e^{I\phicap \theta} \frac{\cos\theta}{\sin^2\theta} \partial_\phi
+ i e^{I\phicap \theta} \inv{\sin\theta} \partial_{\theta\phi}
\right) \\
&=
I \phicap \left(
\partial_{\theta\theta}
+ i ( I \phicap e^{I\phicap \theta} \sin\theta
+  e^{I\phicap \theta} \cos\theta ) \inv{\sin^2\theta} \partial_\phi
+ i e^{I\phicap \theta} \inv{\sin\theta} \partial_{\theta\phi}
\right) \\
&=
I \phicap \left(
\partial_{\theta\theta}
+ i e^{2 I\phicap \theta} \inv{\sin^2\theta} \partial_\phi
+ i e^{I\phicap \theta} \inv{\sin\theta} \partial_{\theta\phi}
\right) \\
\end{aligned}
\end{equation}
%
Premultiplying by \(I\phicap\) and taking scalar parts we have the first part of the application of \eqnref{eqn:qmAngularMom:squaring2} on itself,
%
\begin{equation}\label{eqn:qmAngularMom:squaring3}
\gpgradezero{ I \phicap \partial_\theta (\Bx \wedge \spacegrad) } = -\partial_{\theta\theta}
\end{equation}
%
For the \(\phi\) partials it looks like the simplest option is using the computed bivector \(\phi\) partials \(\partial_\phi (\rcap \thetacap) = \Be_3 \phicap\), \(\partial_\phi (\rcap \phicap) = -I \phicap \cos\theta\).  Doing so we have
%
\begin{equation}\label{eqn:qmAngularMom:777}
\begin{aligned}
\partial_\phi (\Bx \wedge \spacegrad)
&=
\partial_\phi \lr{ \rcap \thetacap \partial_\theta + \rcap \phicap \inv{\sin\theta} \partial_\phi  }  \\
&=
\Be_3 \phicap \partial_\theta +
+\rcap \thetacap \partial_{\phi\theta}
-I \phicap \cot\theta \partial_\phi
+ \rcap \phicap \inv{\sin\theta} \partial_{\phi\phi}
\end{aligned}
\end{equation}
%
So the remaining terms of the squared angular momentum operator follow by premultiplying by \(\rcap \phicap/\sin\theta\), and taking scalar parts.  This is
%
\begin{equation}\label{eqn:qmAngularMom:797}
\begin{aligned}
\gpgradezero{ \rcap\phicap \inv{\sin\theta} \partial_\phi (\Bx \wedge \spacegrad) }
&=
\inv{\sin\theta} \gpgradezero{
-\rcap \Be_3 \partial_\theta +
-\phicap \thetacap \partial_{\phi\theta}
-\rcap I \cot\theta \partial_\phi
}
- \inv{\sin^2\theta} \partial_{\phi\phi}
\end{aligned}
\end{equation}
%
The second and third terms in the scalar selection have only bivector parts, but since \(\rcap = \Be_3 \cos\theta + \Be_1 \sin\theta e^{\Be_{12}\phi}\) has component in the \(\Be_3\) direction, we have
%
\begin{equation}\label{eqn:qmAngularMom:squaring4}
\gpgradezero{ \rcap\phicap \inv{\sin\theta} \partial_\phi (\Bx \wedge \spacegrad) }
=
-\cot\theta \partial_\theta - \inv{\sin^2\theta} \partial_{\phi\phi}
\end{equation}
%
Adding results from \eqnref{eqn:qmAngularMom:squaring3}, and \eqnref{eqn:qmAngularMom:squaring4} we have
%
\begin{equation}\label{eqn:qmAngularMom:squaring5}
-\gpgradezero{ (\Bx \wedge \spacegrad)^2 }
=
\partial_{\theta\theta} +\cot\theta \partial_\theta + \inv{\sin^2\theta} \partial_{\phi\phi}
\end{equation}
%
A final verification of \eqnref{eqn:qmAngularMom:squaring1} now only requires a simple calculus expansion
%
\begin{equation}\label{eqn:qmAngularMom:817}
\begin{aligned}
\inv{\sin\theta} \PD{\theta}{} \sin\theta \PD{\theta}{} \psi
&=
\inv{\sin\theta} \PD{\theta}{} \sin\theta \partial_\theta \psi \\
&=
\inv{\sin\theta} (\cos\theta \partial_\theta \psi + \sin\theta \partial_{\theta\theta} \psi) \\
&=
\cot\theta \partial_\theta \psi + \partial_{\theta\theta} \psi
\end{aligned}
\end{equation}
%
Voila.  This exercise demonstrating that what was known to have to be true, is in fact explicitly true, is now done.  There is no new or interesting results in this in and of itself, but we get some additional confidence in the new methods being experimented with.

\section{3D Quantum Hamiltonian}

Going back to the quantum Hamiltonian we do still have the angular momentum operator as one of the distinct factors of the Laplacian.  As operators we have something akin to the projection of the gradient onto the radial direction, as well as terms that project the gradient onto the tangential plane to the sphere at the radial point
%
\begin{equation}\label{eqn:qmAngularMom:837}
\begin{aligned}
-\frac{\Hbar^2}{2m} \spacegrad^2 + V
&=
-\frac{\Hbar^2}{2m} \lr{ \inv{\Bx^2} (\Bx \cdot \spacegrad)^2 - \inv{\Bx^2} \gpgradezero{(\Bx \wedge \spacegrad)^2} + \inv{\Bx} \cdot \spacegrad  } + V
\end{aligned}
\end{equation}
%
Using the result of \eqnref{eqn:qmAngularMom:onGradeTwo5} and the radial formulation for the rest, we can write this
%
\begin{equation}\label{eqn:qmAngularMom:857}
\begin{aligned}
0
&= \lr{ \spacegrad^2 - \frac{2m}{\Hbar^2} (V - E)  } \psi \\
&=
\inv{r}\frac{\partial}{\partial r} r \frac{\partial \psi}{\partial r}
- \inv{r^2} \lr{ \Bx \wedge \spacegrad - 1  } (\Bx \wedge \spacegrad) \psi
- \frac{2m}{\Hbar^2} (V - E) \psi \\
\end{aligned}
\end{equation}
%
If \(V = V(r)\), then a radial split by separation of variables is possible.  Writing \(\psi = R(r) Y\), we get
%
\begin{equation}\label{eqn:qmAngularMom:threeDqm1}
\begin{aligned}
\frac{r}{R} \frac{\partial}{\partial r} r \frac{\partial R}{\partial r} - \frac{2m r^2}{\Hbar^2} (V(r) - E)
= \inv{Y} \lr{ \Bx \wedge \spacegrad - 1  } (\Bx \wedge \spacegrad) Y = \text{constant}
\end{aligned}
\end{equation}
%
For the constant, lets use \(c\), and split this into a pair of equations
%
\begin{equation}\label{eqn:qmAngularMom:threeDqm2}
r \frac{\partial}{\partial r} r \frac{\partial R}{\partial r} - \frac{2m r^2 R}{\Hbar^2} (V(r) - E) = c R
\end{equation}
%
\begin{equation}\label{eqn:qmAngularMom:threeDqm3}
\lr{ \Bx \wedge \spacegrad - 1  } (\Bx \wedge \spacegrad) Y = c Y
\end{equation}
%
In this last we can examine simultaneous eigenvalues of \(\Bx \wedge \spacegrad\), and \(\gpgradezero{(\Bx \wedge \spacegrad)^2}\).  Suppose that
\(Y_\lambda\) is an eigenfunction of \(\Bx \wedge \spacegrad\) with eigenvalue \(\lambda\).  We then have
%
\begin{equation}\label{eqn:qmAngularMom:877}
\begin{aligned}
\gpgradezero{(\Bx \wedge \spacegrad)^2} Y_\lambda
&= \lr{ \Bx \wedge \spacegrad - 1  } (\Bx \wedge \spacegrad) Y_\lambda \\
&= \lr{ \Bx \wedge \spacegrad - 1  } \lambda Y_\lambda  \\
&= \lambda \lr{ \lambda - 1  } Y_\lambda
\end{aligned}
\end{equation}
%
We see immediately that \(Y_\lambda\) is then also an eigenfunction of \(\gpgradezero{(\Bx \wedge \spacegrad)^2}\), with eigenvalue
%
\begin{equation}\label{eqn:qmAngularMom:threeDqm4p}
\lambda \lr{ \lambda - 1  }
\end{equation}
%
Bohm gives results for simultaneous eigenfunctions of \(L_x, L_y\), or \(L_z\) with \(L^2\), in which case the eigenvalues match.  He also shows that eigenfunctions of raising and lowering operators, \(L_x \pm iL_y\) are also simultaneous eigenfunctions of \(L^2\), but having \(m(m \pm 1)\) eigenvalues.  This is something slightly different since we are not considering any specific components, but we still see that eigenfunctions of the bivector angular momentum operator \(\Bx \wedge \spacegrad\) are simultaneous eigenfunctions of the scalar squared angular momentum operator \(\gpgradezero{\Bx \wedge \spacegrad}\) (Q: is that identical to the scalar operator \(L^2\)).

Moving on, the next order of business is figuring out how to solve the multivector eigenvalue problem
%
\begin{equation}\label{eqn:qmAngularMom:threeDqm4}
(\Bx \wedge \spacegrad) Y_\lambda = \lambda Y_\lambda
\end{equation}
%
\section{Angular momentum polar form, factoring out the raising and lowering operators, and simultaneous eigenvalues}

After a bit more manipulation we find that the angular momentum operator polar form representation, again using \(i = \Be_1 \Be_2\), is
%
\begin{equation}\label{eqn:qmAngularMom:polar1}
\Bx \wedge \spacegrad = I \phicap ( \partial_\theta + i \cot\theta \partial_\phi + \Be_{23} e^{i\phi} \partial_\phi )
\end{equation}
%
Observe how similar the exponential free terms within the braces are to the raising operator as given in Bohm's equation (14.40)
%
\begin{equation}\label{eqn:qmAngularMom:polar2}
\begin{aligned}
L_x + i L_y &= e^{i\phi} (\partial_\theta + i \cot\theta \partial_\phi ) \\
L_z &= \inv{i} \partial_\phi
\end{aligned}
\end{equation}
%
In fact since \(\Be_{23}e^{i\phi} = e^{-i\phi} \Be_{23}\), the match can be made even closer
%
\begin{equation}\label{eqn:qmAngularMom:polar3}
\Bx \wedge \spacegrad = I \phicap e^{-i\phi} ( \mathLabelBox{e^{i\phi} (\partial_\theta + i \cot\theta \partial_\phi)}{\(= L_x + i L_y\)} + \Be_{13} \mathLabelBox{\inv{i} \partial_\phi}{\(=L_z\)} )
\end{equation}
%
This is a surprising factorization, but noting that \(\phicap = \Be_2 e^{i\phi}\) we have
%
\begin{equation}\label{eqn:qmAngularMom:polar4}
\Bx \wedge \spacegrad = \Be_{31} \lr{ e^{i\phi} (\partial_\theta + i \cot\theta \partial_\phi) + \Be_{13} \inv{i} \partial_\phi  }
\end{equation}
%
It appears that the factoring out from the left of a unit bivector (in this case \(\Be_{31}\)) from the bivector angular momentum operator, leaves as one of the remainders the raising operator.

Similarly, noting that \(\Be_{13}\) anticommutes with \(i = \Be_{12}\), we have the right factorization
%
\begin{equation}\label{eqn:qmAngularMom:polar5}
\Bx \wedge \spacegrad =
\lr{ e^{-i\phi} (\partial_\theta - i \cot\theta \partial_\phi) - \Be_{13} \inv{i} \partial_\phi  }
\Be_{31}
\end{equation}
%
Now in the remainder, we see the polar form representation of the lowering operator \(L_x - i L_y = e^{-i\phi}(\partial_\theta - i\cot\theta \partial_\phi)\).

I was not expecting the raising and lowering operators ``to fall out'' as they did by simply expressing the complete bivector operator in polar form.  This is actually fortuitous since it shows why this peculiar combination is of interest.

If we find a zero solution to the raising or lowering operator, that is also a solution of the eigenproblem \((\partial_\phi - \lambda) \psi = 0\), then this is necessarily also an eigensolution of \(\Bx \wedge \spacegrad\).  A secondary implication is that this is then also an eigensolution of \(\gpgradezero{(\Bx \wedge \spacegrad)^2} \psi = \lambda' \psi\).  This was the starting point in Bohm's quest for the spherical harmonics, but why he started there was not clear to me.

Saying this without the words, let us look for eigenfunctions for the non-raising portion of \eqnref{eqn:qmAngularMom:polar4}.  That is
%
\begin{equation}\label{eqn:qmAngularMom:polar6}
\Be_{31} \Be_{13} \inv{i} \partial_\phi f = \lambda f
\end{equation}
%
Since \(\Be_{31} \Be_{13} = 1\) we want solutions of
%
\begin{equation}\label{eqn:qmAngularMom:polar7}
\partial_\phi f = i \lambda f
\end{equation}
%
Solutions are
%
\begin{equation}\label{eqn:qmAngularMom:polar12}
f = \kappa(\theta) e^{i\lambda \phi}
\end{equation}
%
A demand that this is a zero eigenfunction for the raising operator, means we are looking for solutions of
%
\begin{equation}\label{eqn:qmAngularMom:polar8}
\Be_{31} e^{i\phi} (\partial_\theta + i \cot\theta \partial_\phi) \kappa(\theta) e^{i\lambda \phi} = 0
\end{equation}
%
It is sufficient to find zero eigenfunctions of
%
\begin{equation}\label{eqn:qmAngularMom:polar9}
(\partial_\theta + i \cot\theta \partial_\phi) \kappa(\theta) e^{i\lambda \phi} = 0
\end{equation}
%
Evaluation of the \(\phi\) partials and rearrangement leaves us with an equation in \(\theta\) only
%
\begin{equation}\label{eqn:qmAngularMom:polar10}
\frac{\partial \kappa }{\partial \theta} = \lambda \cot\theta \kappa
\end{equation}
%
This has solutions \(\kappa = A(\phi) (\sin\theta)^\lambda\), where because of the partial derivatives in \eqnref{eqn:qmAngularMom:polar10} we are free to make the integration constant a function of \(\phi\).  Since this is the functional dependence that is a zero of the raising operator, including this at the \(\theta\) dependence of \eqnref{eqn:qmAngularMom:polar12} means that we have a simultaneous zero of the raising operator, and an eigenfunction of eigenvalue \(\lambda\) for the remainder of the angular momentum operator.
%
\begin{equation}\label{eqn:qmAngularMom:polar11}
f(\theta,\phi) = (\sin\theta)^\lambda e^{i\lambda \phi}
\end{equation}
%
This is very similar seeming to the process of adding homogeneous solutions to specific ones, since we augment the specific eigenvalued solutions for one part of the operator by ones that produce zeros for the rest.

As a check lets apply the angular momentum operator to this as a test and see if the results match our expectations.
%
\begin{equation}\label{eqn:qmAngularMom:897}
\begin{aligned}
(\Bx \wedge \spacegrad ) (\sin\theta)^\lambda e^{i\lambda \phi}
&=
\rcap \lr{ \thetacap \partial_\theta + \phicap \inv{\sin\theta} \partial_\phi  }  (\sin\theta)^\lambda e^{i\lambda \phi} \\
&=
\rcap \lr{ \thetacap \lambda (\sin\theta)^{\lambda-1} \cos\theta + \phicap \inv{\sin\theta} (\sin\theta)^\lambda (i\lambda) } e^{i\lambda \phi} \\
&=
\lambda \rcap \lr{ \thetacap \cos\theta + \phicap i  } e^{i\lambda \phi} (\sin\theta)^{\lambda-1}  \\
\end{aligned}
\end{equation}
%
From \eqnref{eqn:qmAngularMom:vecBivecDerivatives5} we have
%
\begin{equation}\label{eqn:qmAngularMom:917}
\begin{aligned}
\rcap \phicap i
&= \Be_3 \phicap i \cos\theta - \sin\theta \\
&= \Be_{32} i e^{i\phi} \cos\theta - \sin\theta \\
&= \Be_{13} e^{i\phi} \cos\theta - \sin\theta \\
\end{aligned}
\end{equation}
%
and from \eqnref{eqn:qmAngularMom:vecBivecDerivatives4} we have
%
\begin{equation}\label{eqn:qmAngularMom:937}
\begin{aligned}
\rcap \thetacap
&= I \phicap  \\
&= \Be_{31} e^{i\phi}
\end{aligned}
\end{equation}
%
Putting these together shows that \((\sin\theta)^\lambda e^{i\lambda \phi}\) is an eigenfunction of \(\Bx \wedge \spacegrad\),
%
\begin{equation}\label{eqn:qmAngularMom:polar13}
(\Bx \wedge \spacegrad ) (\sin\theta)^\lambda e^{i\lambda \phi} = -\lambda (\sin\theta)^\lambda e^{i\lambda \phi}
\end{equation}
%
This negation surprised me at first, but I do not see any errors here in the arithmetic.  Observe that this provides a verification of messy algebra that led to \eqnref{eqn:qmAngularMom:onGradeTwo5}.  That was
%
\begin{equation}\label{eqn:qmAngularMom:polar14}
\gpgradezero{(\Bx \wedge \spacegrad)^2} \questionEquals \lr{ \Bx \wedge \spacegrad - 1  } (\Bx \wedge \spacegrad)
\end{equation}
%
Using this and \eqnref{eqn:qmAngularMom:polar13} the operator effect of \(\gpgradezero{(\Bx \wedge \spacegrad)^2}\) for the eigenvalue we have is
%
\begin{equation}\label{eqn:qmAngularMom:957}
\begin{aligned}
\gpgradezero{(\Bx \wedge \spacegrad)^2}
(\sin\theta)^\lambda e^{i\lambda \phi}
&=
\lr{ \Bx \wedge \spacegrad - 1  } (\Bx \wedge \spacegrad)
(\sin\theta)^\lambda e^{i\lambda \phi} \\
&=
((-\lambda)^2 - (-\lambda)) (\sin\theta)^\lambda e^{i\lambda \phi} \\
\end{aligned}
\end{equation}
%
So the eigenvalue is \(\lambda(\lambda + 1)\), consistent with results obtained with coordinate and scalar polar form tools.

\section{Summary}

Having covered a fairly wide range in the preceding Geometric Algebra exploration of the angular momentum operator, it seems worthwhile to attempt to summarize what was learned.

The exploration started with a simple observation that the use of the spatial pseudoscalar for the imaginary of the angular momentum operator in its coordinate form
%
\begin{equation}\label{eqn:qmSum:goo1}
\begin{aligned}
L_1 &= -i \Hbar( x_2 \partial_3 - x_3 \partial_2 ) \\
L_2 &= -i \Hbar( x_3 \partial_1 - x_1 \partial_3 ) \\
L_3 &= -i \Hbar( x_1 \partial_2 - x_2 \partial_1 )
\end{aligned}
\end{equation}
%
allowed for expressing the angular momentum operator in its entirety as a bivector valued operator
%
\begin{equation}\label{eqn:qmSum:goo2}
\BL = - \Hbar \Bx \wedge \spacegrad
\end{equation}
%
The bivector representation has an intrinsic complex behavior, eliminating the requirement for an explicitly imaginary \(i\) as is the case in the coordinate representation.

It was then assumed that the Laplacian can be expressed directly in terms of \(\Bx \wedge \spacegrad\).  This is not an unreasonable thought since we can factor the gradient with components projected onto and perpendicular to a constant reference vector \(\acap\) as
%
\begin{equation}\label{eqn:qmSum:goo3}
\spacegrad = \acap (\acap \cdot \spacegrad) + \acap (\acap \wedge \spacegrad)
\end{equation}
%
and this squares to a Laplacian expressed in terms of these constant reference directions
%
\begin{equation}\label{eqn:qmSum:goo4}
\spacegrad^2 = (\acap \cdot \spacegrad)^2 - (\acap \cdot \spacegrad)^2
\end{equation}
%
a quantity that has an angular momentum like operator with respect to a constant direction.  It was then assumed that we could find an operator representation of the form
%
\begin{equation}\label{eqn:qmSum:goo5}
\spacegrad^2 = \inv{\Bx^2} \lr{ (\Bx \cdot \spacegrad)^2 - \gpgradezero{(\Bx \cdot \spacegrad)^2} + f(\Bx, \spacegrad)  }
\end{equation}
%
Where \(f(\Bx, \spacegrad)\) was to be determined, and was found by subtraction.  Thinking ahead to relativistic applications this result was obtained for the n-dimensional Laplacian and was found to be
%
\begin{equation}\label{eqn:qmSum:goo6}
\grad^2 = \inv{x^2} \lr{ (n-2 + x \cdot \grad) (x \cdot \grad) - \gpgradezero{(x \wedge \grad)^2}  }
\end{equation}
%
For the 3D case specifically this is
%
\begin{equation}\label{eqn:qmSum:goo7}
\spacegrad^2 = \inv{\Bx^2} \lr{ (1 + \Bx \cdot \spacegrad) (\Bx \cdot \spacegrad) - \gpgradezero{(\Bx \wedge \spacegrad)^2}  }
\end{equation}
%
While the scalar selection above is good for some purposes, it interferes with observations about simultaneous eigenfunctions for the angular momentum operator and the scalar part of its square as seen in the Laplacian.  With some difficulty and tedium, by subtracting the bivector and quadvector grades from the squared angular momentum operator \((x \wedge \grad)^2\) it was eventually found in \cref{eqn:qmAngularMom:onGradeTwo5} that \eqnref{eqn:qmSum:goo6} can be written as
%
\begin{equation}\label{eqn:qmSum:goo8}
\grad^2 =
\inv{x^2} \left(
  (n-2 + x \cdot \grad) (x \cdot \grad)
+ (n-2 - x \wedge \grad) (x \wedge \grad)
+ (x \wedge \grad) \wedge (x \wedge \grad)
\right)
\end{equation}
%
In the 3D case the quadvector vanishes and \eqnref{eqn:qmSum:goo7} with the scalar selection removed is reduced to
%
\begin{equation}\label{eqn:qmSum:goo9}
\spacegrad^2 =
\inv{\Bx^2} \lr{ (1 + \Bx \cdot \spacegrad) (\Bx \cdot \spacegrad) + (1 - \Bx \wedge \spacegrad) (\Bx \wedge \spacegrad)  }
\end{equation}
%
FIXME: This doesn't look right, since we have a bivector \( \inv{\Bx^2} \Bx \wedge \spacegrad \) on the RHS and everything else is a scalar.

In 3D we also have the option of using the duality relation between the cross and the wedge \(\Ba \wedge \Bb = i (\Ba \cross \Bb)\) to express the Laplacian
%
\begin{equation}\label{eqn:qmSum:goo10}
\spacegrad^2 =
\inv{\Bx^2} \lr{ (1 + \Bx \cdot \spacegrad) (\Bx \cdot \spacegrad) + (1 - i (\Bx \cross \spacegrad)) i(\Bx \cross \spacegrad)  }
\end{equation}
%
Since it is customary to express angular momentum as \(\BL = -i \Hbar (\Bx \cross \spacegrad)\), we see here that the imaginary in this context should perhaps necessarily be viewed as the spatial pseudoscalar.  It was that guess that led down this path, and we come full circle back to this considering how to factor the Laplacian in vector quantities.  Curiously this factorization is in no way specific to Quantum Theory.

A few verifications of the Laplacian in \eqnref{eqn:qmSum:goo10} were made.  First it was shown that the directional derivative terms containing \(\Bx \cdot \spacegrad\), are equivalent to the radial terms of the Laplacian in spherical polar coordinates.   That is
%
\begin{equation}\label{eqn:qmSum:goo11}
\inv{\Bx^2} (1 + \Bx \cdot \spacegrad) (\Bx \cdot \spacegrad) \psi = \inv{r} \frac{\partial^2}{\partial r^2} (r\psi)
\end{equation}
%
Employing the quaternion operator for the spherical polar rotation
%
\begin{equation}\label{eqn:qmSum:goo12}
\begin{aligned}
R &= e^{\Be_{31}\theta/2} e^{\Be_{12}\phi/2} \\
\Bx &= r \tilde{R} \Be_3 R
\end{aligned}
\end{equation}
%
it was also shown that there was explicitly no radial dependence in the angular momentum operator which takes the form
%
\begin{equation}\label{eqn:qmSum:goo13}
\begin{aligned}
\Bx \wedge \spacegrad
&= \tilde{R} \lr{ \Be_3 \Be_1 R \partial_\theta + \Be_3 \Be_2 R \inv{\sin\theta} \partial_\phi  } \\
&= \rcap \lr{ \thetacap \partial_\theta + \phicap \inv{\sin\theta} \partial_\phi  }
\end{aligned}
\end{equation}
%
Because there is a \(\theta\), and \(\phi\) dependence in the unit vectors \(\rcap\), \(\thetacap\), and \(\phicap\), squaring the angular momentum operator in this form means that the unit vectors are also operated on.  Those vectors were given by the triplet
%
\begin{equation}\label{eqn:qmSum:goo14}
\begin{pmatrix}
\rcap \\
\thetacap \\
\phicap \\
\end{pmatrix}
=
\tilde{R}
\begin{pmatrix}
\Be_3 \\
\Be_1 \\
\Be_2 \\
\end{pmatrix}
R
\end{equation}
%
Using \(I = \Be_1 \Be_2 \Be_3\) for the spatial pseudoscalar, and \(i = \Be_1 \Be_2\) (a possibly confusing switch of notation) for the bivector of the x-y plane we can write the spherical polar unit vectors in exponential form as
%
\begin{equation}\label{eqn:qmSum:goo15}
\begin{pmatrix}
\phicap \\
\rcap \\
\thetacap \\
\end{pmatrix}
=
\begin{pmatrix}
\Be_2 e^{i\phi} \\
\Be_3 e^{I \phicap \theta} \\
i \phicap e^{I \phicap \theta} \\
\end{pmatrix}
\end{equation}
%
These or related expansions were used to verify (with some difficulty) that the scalar squared bivector operator is identical to the expected scalar spherical polar coordinates parts of the Laplacian
%
\begin{equation}\label{eqn:qmSum:goo16}
-\gpgradezero{ (\Bx \wedge \spacegrad)^2 } =
\inv{\sin\theta} \PD{\theta}{} \sin\theta \PD{\theta}{} + \inv{\sin^2\theta} \frac{\partial^2}{\partial \phi^2}
\end{equation}
%
Additionally, by left or right dividing a unit bivector from the angular momentum operator, we are able to find that the raising and lowering operators are left as one of the factors
%
\begin{equation}\label{eqn:qmSum:goo17}
\begin{aligned}
\Bx \wedge \spacegrad &= \Be_{31} \lr{ e^{i\phi} (\partial_\theta + i \cot\theta \partial_\phi) + \Be_{13} \inv{i} \partial_\phi  } \\
\Bx \wedge \spacegrad &= \lr{ e^{-i\phi} (\partial_\theta - i \cot\theta \partial_\phi) - \Be_{13} \inv{i} \partial_\phi  } \Be_{31}
\end{aligned}
\end{equation}
%
Both of these use \(i = \Be_1 \Be_2\), the bivector for the plane, and not the spatial pseudoscalar.  We are then able to see that in the context of the raising and lowering operator for the radial equation the interpretation of the imaginary should be one of a plane.

Using the raising operator factorization, it was calculated that \((\sin\theta)^\lambda e^{i\lambda \phi}\) was an eigenfunction of the bivector operator \(\Bx \wedge \spacegrad\) with eigenvalue \(-\lambda\).  This results in the simultaneous eigenvalue of \(\lambda(\lambda + 1)\) for this eigenfunction with the scalar squared angular momentum operator.

There are a few things here that have not been explored to their logical conclusion.

The bivector Fourier projections \(I \Be_k (\Bx \wedge \spacegrad ) \cdot (-I \Be_k)\) do not obey the commutation relations of the scalar angular momentum components, so an attempt to directly use these to construct raising and lowering operators does not produce anything useful.  The raising and lowering operators in a form that could be used to find eigensolutions were found by factoring out \(\Be_{13}\) from the bivector operator.  Making this particular factorization was a fluke and only because it was desirable to express the bivector operator entirely in spherical polar form.  It is curious that this results in raising and lowering operators for the x,y plane, and understanding this further would be nice.

In the eigen solutions for the bivector operator, no quantization condition was imposed.  I do not understand the argument that Bohm used to do so in the traditional treatment, and revisiting this once that is done is in order.

I am also unsure exactly how Bohm knows that the inner product for the eigenfunctions should be a surface integral.  This choice works, but what drives it.  Can that be related to anything here?

%\EndArticle

   %
% Copyright � 2012 Peeter Joot.  All Rights Reserved.
% Licenced as described in the file LICENSE under the root directory of this GIT repository.
%

%
%
%\input{../peeter_prologue.tex}

\chapter{Graphical representation of Spherical Harmonics for \texorpdfstring{\(l=1\)}{l equal 1}}
\index{spherical harmonics}
\label{chap:L1Associated}

%\blogpage{http://sites.google.com/site/peeterjoot/math2009/L1Associated.pdf}
%\date{Aug 16, 2009}
%\revisionInfo{\(RCSfile: L1Associated.tex,v \) Last \(Revision: 1.6 \) \(Date: 2009/10/22 02:07:20 \)}

%\beginArtWithToc
\beginArtNoToc

\section{First observations}

In Bohm's QT \citep{bohm1989qt}, 14.17), the properties of \(l=1\) associated Legendre polynomials are examined under rotation.  Wikipedia (\citep{wiki:sphericalHarm} calls these eigen functions the spherical harmonics.

The unnormalized eigenfunctions are given (eqn (14.47) in Bohm) for \(s \in [0,l]\), with \(\cos\theta = \zeta\) by

\begin{equation}\label{eqn:L1Associated:foo0}
\begin{aligned}
\psi_l^{l-s} = \frac{e^{i(l-s)\phi}}{(1-\zeta^2)^{(l-s)/2}} \frac{\partial^s}{\partial \zeta^s} (1-\zeta^2)^l
\end{aligned}
\end{equation}

The normalization is provided by a surface area inner product

\begin{equation}\label{eqn:L1Associated:foo2}
\begin{aligned}
(u,v) &= \int_{\theta=0}^{\pi} \int_{\phi=0}^{2\pi} u v^\conj \sin\theta d\theta d\phi
\end{aligned}
\end{equation}

Computing these for \(l=1\), and disregarding any normalization these eigenfunctions can be found to be

\begin{equation}\label{eqn:L1Associated:foo1}
\begin{aligned}
\psi_1^1 &= \sin\theta e^{i\phi} \\
\psi_1^0 &= \cos\theta \\
\psi_1^{-1} &= \sin\theta e^{-i\phi}
\end{aligned}
\end{equation}

There is a direct relationship between these eigenfunctions with a triple of vectors associated with a point on the unit sphere.  Referring to \cref{fig:L1Associated}, observe the three doubled arrow vectors, all associated with a point on the unit sphere \(\Bx = (x,y,z) = (\sin\theta \cos\phi, \sin\theta \cos\phi, \cos\theta)\).

\imageFigure{../figures/gabook/L1Associated}{Vectoring the \(l=1\) associated Legendre polynomials}{fig:L1Associated}{0.4}

The normal to the \(x,y\) plane from \(\Bx\), designated \(\Bn\) has the vectorial value

\begin{equation}\label{eqn:L1Associated:foo3}
\begin{aligned}
\Bn = \cos\theta \Be_3
\end{aligned}
\end{equation}

From the origin to the point of of the \(x,y\) plane intersection to the normal we have

\begin{equation}\label{eqn:L1Associated:foo4}
\begin{aligned}
\Brho = \sin\theta (\cos\phi \Be_1 + \sin\phi \Be_2) = \Be_1 \sin\theta e^{\Be_1 \Be_2 \phi}
\end{aligned}
\end{equation}

and finally in the opposite direction also in the plane and mirroring \(\Brho\) we have the last of this triplet of vectors

\begin{equation}\label{eqn:L1Associated:foo5}
\begin{aligned}
\Brho_{-} = \sin\theta (\cos\phi \Be_1 - \sin\phi \Be_2) = \Be_1 \sin\theta e^{-\Be_1 \Be_2 \phi}
\end{aligned}
\end{equation}

So, if we choose to use \(i=\Be_1 \Be_2\) (the bivector for the plane normal to the z-axis), then we can in fact vectorize these eigenfunctions.  The vectors \(\Brho\) (i.e. \(\psi_1^1)\), and \(\Brho_{-}\) (i.e. \(\psi_1^{-1}\)) are both normal to \(\Bn\) (i.e. \(\psi_1^0\)), but while the vectors \(\Brho\) and \(\Brho_{-}\) are both in the plane one is produced with a counterclockwise rotation of \(\Be_1\) by \(\phi\) in the plane and the other with an opposing rotation.

Summarizing, we can write the unnormalized vectors the relations

\begin{equation*}
\begin{array}{l l l}%\label{eqn:L1Associated:foo7}
\psi_1^1 &= \Be_1 \Brho &= \sin\theta e^{\Be_1\Be_2 \phi} \\
\psi_1^0 &= \Be_3 \Bn &= \cos\theta \\
\psi_1^{-1} &= \Be_1 \Brho_{-} &= \sin\theta e^{-\Be_1\Be_2 \phi}
\end{array}
\end{equation*}

%\begin{equation*}\label{eqn:L1Associated:foo7}
%\begin{array}{l l l}
%\Brho &= \Be_1 \psi_1^1 &= \sin\theta e^{\Be_1\Be_2 \phi} = \sin\theta (\cos\phi \Be_1 + \sin\phi \Be_2) \\
%\Bn &= \Be_3 \psi_1^0 &= \cos\theta \Be_3 \\
%\Brho_{-} &= \Be_1 \psi_1^{-1} &= \sin\theta e^{-\Be_1\Be_2 \phi} = \sin\theta (\cos\phi \Be_1 - \sin\phi \Be_2)
%\end{array}
%\end{equation*}

I have no familiarity yet with the \(l=2\) or higher Legendre eigenfunctions.  Do they also admit a geometric representation?

\section{Expressing Legendre eigenfunctions using rotations}

We can express a point on a sphere with a pair of rotation operators.  First rotating \(\Be_3\) towards \(\Be_1\) in the \(z,x\) plane by \(\theta\), then in the \(x,y\) plane by \(\phi\) we have the point \(\Bx\) in \cref{fig:L1Associated}

Writing the result of the first rotation as \(\Be_3'\) we have

\begin{equation}\label{eqn:L1Associated:foob1}
\begin{aligned}
\Be_3' = \Be_3 e^{\Be_{31}\theta} = e^{-\Be_{31}\theta/2} \Be_3 e^{\Be_{31}\theta/2}
\end{aligned}
\end{equation}

One more rotation takes \(\Be_3'\) to \(\Bx\).  That is

\begin{equation}\label{eqn:L1Associated:foob2}
\begin{aligned}
\Bx = e^{-\Be_{12}\phi/2} \Be_3' e^{\Be_{12}\phi/2}
\end{aligned}
\end{equation}

All together, writing \(R_\theta = e^{\Be_{31}\theta/2}\), and \(R_\phi = e^{\Be_{12}\phi/2}\), we have

\begin{equation}\label{eqn:L1Associated:foob3}
\begin{aligned}
\Bx = \tilde{R_\phi} \tilde{R_\theta} \Be_3 R_\theta R_\phi
\end{aligned}
\end{equation}

It is worth a quick verification that this produces the desired result.

\begin{equation}\label{eqn:L1Associated:27}
\begin{aligned}
\tilde{R_\phi} \tilde{R_\theta} \Be_3 R_\theta R_\phi
&= \tilde{R_\phi} \Be_3 e^{\Be_{31}\theta} R_\phi \\
&= e^{-\Be_{12}\phi/2} (\Be_3 \cos\theta + \Be_1 \sin\theta) e^{\Be_{12}\phi/2} \\
&=
\Be_3 \cos\theta + \Be_1 \sin\theta e^{\Be_{12}\phi} \\
\end{aligned}
\end{equation}

This is the expected result

\begin{equation}\label{eqn:L1Associated:foob4}
\begin{aligned}
\Bx = \Be_3 \cos\theta + \sin\theta (\Be_1 \sin\theta + \Be_2 \cos\theta)
\end{aligned}
\end{equation}

The projections onto the \(\Be_3\) and the \(x,y\) plane are then, respectively,

\begin{equation}\label{eqn:L1Associated:foob5}
\begin{aligned}
\Bx_z &= \Be_3 (\Be_3 \cdot \Bx) = \Be_3 \cos\theta  \\
\Bx_{x,y} &= \Be_3 (\Be_3 \wedge \Bx) = \sin\theta (\Be_1 \sin\theta + \Be_2 \cos\theta)
\end{aligned}
\end{equation}

So if \(\Bx_{\pm}\) is the point on the unit sphere associated with the rotation angles \(\theta,\pm\phi\), then we have for the \(l=1\) associated Legendre polynomials

\begin{equation}\label{eqn:L1Associated:foob6}
\begin{aligned}
\psi_1^0 &= \Be_3 \cdot \Bx \\
\psi_1^{\pm 1} &= \Be_1 \Be_3 (\Be_3 \wedge \Bx_{\pm})
\end{aligned}
\end{equation}

Note that the \(\pm\) was omitted from \(\Bx\) for \(\psi_1^0\) since either produces the same \(\Be_3\) component.  This gives us a nice geometric interpretation of these eigenfunctions.  We see that \(\psi_1^0\) is the biggest when \(\Bx\) is close to straight up, and when this occurs \(\psi_1^{\pm 1}\) are correspondingly reduced, but when \(\Bx\) is close to the \(x,y\) plane where \(\psi_1^{\pm 1}\) will be greatest the \(z\)-axis component is reduced.

%\EndArticle

   %
% Copyright � 2012 Peeter Joot.  All Rights Reserved.
% Licenced as described in the file LICENSE under the root directory of this GIT repository.
%

%
%
%\input{../peeter_prologue.tex}

\chapter{Bivector grades of the squared angular momentum operator}
\index{angular momentum operator!bivector grades}
\label{chap:bivectorSelect}

%\blogpage{http://sites.google.com/site/peeterjoot/math2009/bivectorSelect.pdf}
%\date{Sept 6, 2009}
%\revisionInfo{\(RCSfile: bivectorSelect.tex,v \) Last \(Revision: 1.3 \) \(Date: 2009/10/22 02:07:20 \)}

%\beginArtWithToc
\beginArtNoToc

\section{Motivation}

The aim here is to extract the bivector grades of the squared angular momentum operator

\begin{equation}\label{eqn:bivectorSelect:goo1}
\begin{aligned}
\gpgradetwo{ (x \wedge \grad)^2 } \questionEquals \cdots
\end{aligned}
\end{equation}

I had tried this before and believe gotten it wrong.  Take it super slow and dumb and careful.

\section{Non-operator expansion}

Suppose \(P\) is a bivector, \(P = (\gamma^k \wedge \gamma^m) P_{km}\), the grade two product with a different unit bivector is

\begin{equation}\label{eqn:bivectorSelect:33}
\begin{aligned}
&\gpgradetwo{ (\gamma_a \wedge \gamma_b) (\gamma^k \wedge \gamma^m) } P_{km} \\
&=
\gpgradetwo{ (\gamma_a \gamma_b - \gamma_a \cdot \gamma_b) (\gamma^k \wedge \gamma^m) } P_{km} \\
&=
\gpgradetwo{ \gamma_a (\gamma_b \cdot (\gamma^k \wedge \gamma^m)) } P_{km}
+ \gpgradetwo{ \gamma_a (\gamma_b \wedge (\gamma^k \wedge \gamma^m)) } P_{km}
- (\gamma_a \cdot \gamma_b) (\gamma^k \wedge \gamma^m) P_{km} \\
&=
(\gamma_a \wedge \gamma^m) P_{b m} -(\gamma_a \wedge \gamma^k) P_{k b} - (\gamma_a \cdot \gamma_b) (\gamma^k \wedge \gamma^m) P_{km} \\
&+ (\gamma_a \cdot \gamma_b) (\gamma^k \wedge \gamma^m) P_{km}
- (\gamma_b \wedge \gamma^m) P_{a m}
+ (\gamma_b \wedge \gamma^k) P_{k a}
\\
&=
(\gamma_a \wedge \gamma^c) (P_{b c} -P_{c b})
+ (\gamma_b \wedge \gamma^c) (P_{c a} -P_{a c} )
\\
\end{aligned}
\end{equation}

This same procedure will be used for the operator square, but we have the complexity of having the second angular momentum operator change the first bivector result.

\section{Operator expansion}

In the first few lines of the bivector product expansion above, a blind replacement \(\gamma_a \rightarrow x\), and \(\gamma_b \rightarrow \grad\) gives us

\begin{equation}\label{eqn:bivectorSelect:53}
\begin{aligned}
&\gpgradetwo{ (x \wedge \grad) (\gamma^k \wedge \gamma^m) } P_{km}  \\
&=
\gpgradetwo{ (x \grad - x \cdot \grad) (\gamma^k \wedge \gamma^m) } P_{km} \\
&=
\gpgradetwo{ x (\grad \cdot (\gamma^k \wedge \gamma^m)) } P_{km}
+ \gpgradetwo{ x (\grad \wedge (\gamma^k \wedge \gamma^m)) } P_{km}
- (x \cdot \grad) (\gamma^k \wedge \gamma^m) P_{km} \\
\end{aligned}
\end{equation}

Using \(P_{km} = x_k \partial_m\), eliminating the coordinate expansion we have an intermediate result that gets us partway to the desired result

\begin{equation}\label{eqn:bivectorSelect:goo2}
\begin{aligned}
\gpgradetwo{ (x \wedge \grad)^2 }
&=
\gpgradetwo{ x (\grad \cdot (x \wedge \grad)) }
+ \gpgradetwo{ x (\grad \wedge (x \wedge \grad)) }
- (x \cdot \grad) (x \wedge \grad)
\end{aligned}
\end{equation}

An expansion of the first term should be easier than the second.  Dropping back to coordinates we have

\begin{equation}\label{eqn:bivectorSelect:73}
\begin{aligned}
\gpgradetwo{ x (\grad \cdot (x \wedge \grad)) }
&=
\gpgradetwo{ x (\grad \cdot (\gamma^k \wedge \gamma^m)) } x_k \partial_m \\
&=
\gpgradetwo{ x (\gamma_a \partial^a \cdot (\gamma^k \wedge \gamma^m)) } x_k \partial_m \\
&=
\gpgradetwo{ x \gamma^m \partial^k } x_k \partial_m
-\gpgradetwo{ x \gamma^k \partial^m } x_k \partial_m  \\
&=
x \wedge (\partial^k x_k \gamma^m \partial_m )
- x \wedge (\partial^m \gamma^k x_k \partial_m ) \\
\end{aligned}
\end{equation}

Okay, a bit closer.  Backpedaling with the reinsertion of the complete vector quantities we have

\begin{equation}\label{eqn:bivectorSelect:goo3}
\begin{aligned}
\gpgradetwo{ x (\grad \cdot (x \wedge \grad)) } &= x \wedge (\partial^k x_k \grad ) - x \wedge (\partial^m x \partial_m )
\end{aligned}
\end{equation}

Expanding out these two will be conceptually easier if the functional operation is made explicit.  For the first

\begin{equation}\label{eqn:bivectorSelect:93}
\begin{aligned}
x \wedge (\partial^k x_k \grad ) \phi
&=
x \wedge x_k \partial^k (\grad \phi)
+x \wedge ((\partial^k x_k) \grad) \phi \\
&=
x \wedge ((x \cdot \grad) (\grad \phi))
+ n (x \wedge \grad) \phi
\end{aligned}
\end{equation}

In operator form this is

\begin{equation}\label{eqn:bivectorSelect:goo4}
\begin{aligned}
x \wedge (\partial^k x_k \grad ) &= n (x \wedge \grad) + x \wedge ((x \cdot \grad) \grad )
\end{aligned}
\end{equation}

Now consider the second half of \eqnref{eqn:bivectorSelect:goo3}.  For that we expand

\begin{equation}\label{eqn:bivectorSelect:113}
\begin{aligned}
x \wedge (\partial^m x \partial_m ) \phi
&=
x \wedge (x \partial_m \partial^m \phi)
+ x \wedge ((\partial^m x) \partial_m \phi)
\end{aligned}
\end{equation}

Since \(x \wedge x = 0\), and \(\partial^m x = \partial^m x_k \gamma^k = \gamma^m\), we have

\begin{equation}\label{eqn:bivectorSelect:133}
\begin{aligned}
x \wedge (\partial^m x \partial_m ) \phi
&=
x \wedge (\gamma^m \partial_m ) \phi \\
&=
(x \wedge \grad) \phi
\end{aligned}
\end{equation}

Putting things back together we have for \eqnref{eqn:bivectorSelect:goo3}

%\gpgradetwo{ x (\grad \cdot (x \wedge \grad)) } &= x \wedge (\partial^k x_k \grad ) - x \wedge (\partial^m x \partial_m )
%x \wedge (\partial^k x_k \grad ) &= n (x \wedge \grad) + x \wedge ((x \cdot \grad) \grad )
%x \wedge (\partial^m x \partial_m ) \phi &= (x \wedge \grad) \phi

\begin{equation}\label{eqn:bivectorSelect:goo5}
\begin{aligned}
\gpgradetwo{ x (\grad \cdot (x \wedge \grad)) } &= (n-1) (x \wedge \grad) + x \wedge ((x \cdot \grad) \grad )
\end{aligned}
\end{equation}

This now completes a fair amount of the bivector selection, and a substitution back into \eqnref{eqn:bivectorSelect:goo2} yields

\begin{equation}\label{eqn:bivectorSelect:goo6}
\begin{aligned}
\gpgradetwo{ (x \wedge \grad)^2 }
&=
(n-1 - x \cdot \grad) (x \wedge \grad) + x \wedge ((x \cdot \grad) \grad )
+ x \cdot (\grad \wedge (x \wedge \grad))
\end{aligned}
\end{equation}

The remaining task is to explicitly expand the last vector-trivector dot product.  To do that we use the basic alternation expansion identity

\begin{equation}\label{eqn:bivectorSelect:goo7}
\begin{aligned}
a \cdot (b \wedge c \wedge d)
&=
(a \cdot b) (c \wedge d)
-(a \cdot c) (b \wedge d)
+(a \cdot d) (b \wedge c)
\end{aligned}
\end{equation}

To see how to apply this to the operator case lets write that explicitly but temporarily in coordinates

\begin{equation}\label{eqn:bivectorSelect:153}
\begin{aligned}
x \cdot ((\grad \wedge (x \wedge \grad)) \phi
&=
(x^\mu \gamma_\mu) \cdot ((\gamma^\nu \partial_\nu ) \wedge (x_\alpha \gamma^\alpha \wedge (\gamma^\beta \partial_\beta))) \phi \\
&=
x \cdot \grad (x \wedge \grad) \phi
-
x \cdot \gamma^\alpha \grad \wedge x_\alpha \grad \phi
+
x^\mu \grad \wedge x \gamma_\mu \cdot \gamma^\beta \partial_\beta  \phi \\
&=
x \cdot \grad (x \wedge \grad) \phi
-
x^\alpha \grad \wedge x_\alpha \grad \phi
+
x^\mu \grad \wedge x \partial_\mu  \phi
\end{aligned}
\end{equation}

Considering this term by term starting with the second one we have

\begin{equation}\label{eqn:bivectorSelect:173}
\begin{aligned}
x^\alpha \grad \wedge x_\alpha \grad \phi
&=
x_\alpha (\gamma^\mu \partial_\mu) \wedge x^\alpha \grad \phi \\
&=
x_\alpha \gamma^\mu \wedge (\partial_\mu x^\alpha) \grad \phi
+x_\alpha \gamma^\mu \wedge x^\alpha \partial_\mu \grad \phi  \\
&=
x_\mu \gamma^\mu \wedge \grad \phi
+x_\alpha x^\alpha \gamma^\mu \wedge \partial_\mu \grad \phi  \\
&=
x \wedge \grad \phi
+x^2 \grad \wedge \grad \phi  \\
\end{aligned}
\end{equation}

The curl of a gradient is zero, since summing over an product of antisymmetric and symmetric indices \(\gamma^\mu \wedge \gamma^\nu \partial_{\mu\nu}\) is zero.  Only one term remains to evaluate in the vector-trivector dot product now

\begin{equation}\label{eqn:bivectorSelect:goo8}
\begin{aligned}
x \cdot (\grad \wedge x \wedge \grad)
&=
(-1 + x \cdot \grad )(x \wedge \grad)
+
x^\mu \grad \wedge x \partial_\mu
\end{aligned}
\end{equation}

Again, a completely dumb and brute force expansion of this is

\begin{equation}\label{eqn:bivectorSelect:193}
\begin{aligned}
x^\mu \grad \wedge x \partial_\mu \phi
&=
x^\mu (\gamma^\nu \partial_\nu) \wedge (x^\alpha \gamma_\alpha) \partial_\mu \phi \\
&=
x^\mu \gamma^\nu \wedge (\partial_\nu (x^\alpha \gamma_\alpha)) \partial_\mu \phi
+x^\mu \gamma^\nu \wedge (x^\alpha \gamma_\alpha) \partial_\nu \partial_\mu \phi \\
&=
x^\mu (\gamma^\alpha \wedge \gamma_\alpha) \partial_\mu \phi
+x^\mu \gamma^\nu \wedge x \partial_\nu \partial_\mu \phi \\
\end{aligned}
\end{equation}

With \(\gamma^\mu = \pm \gamma_\mu\), the wedge in the first term is zero, leaving

\begin{equation}\label{eqn:bivectorSelect:213}
\begin{aligned}
x^\mu \grad \wedge x \partial_\mu \phi
&=
-x^\mu x \wedge \gamma^\nu \partial_\nu \partial_\mu \phi \\
&=
-x^\mu x \wedge \gamma^\nu \partial_\mu \partial_\nu \phi \\
&=
-x \wedge x^\mu \partial_\mu \gamma^\nu \partial_\nu \phi \\
\end{aligned}
\end{equation}

In vector form we have finally

\begin{equation}\label{eqn:bivectorSelect:goo9}
\begin{aligned}
x^\mu \grad \wedge x \partial_\mu \phi &= -x \wedge (x \cdot \grad) \grad \phi
\end{aligned}
\end{equation}

The final expansion of the vector-trivector dot product is now

\begin{equation}\label{eqn:bivectorSelect:goo10}
\begin{aligned}
x \cdot (\grad \wedge x \wedge \grad)
&=
(-1 + x \cdot \grad )(x \wedge \grad)
-x \wedge (x \cdot \grad) \grad \phi
\end{aligned}
\end{equation}

This was the last piece we needed for the bivector grade selection.  Incorporating this into \eqnref{eqn:bivectorSelect:goo6}, both the \(x \cdot \grad x \wedge \grad\), and the \(x \wedge (x \cdot \grad) \grad\) terms cancel leaving the surprising simple result

\begin{equation}\label{eqn:bivectorSelect:goo11}
\begin{aligned}
\gpgradetwo{ (x \wedge \grad)^2 }
%&=
%(n-1 - x \cdot \grad) (x \wedge \grad) + x \wedge ((x \cdot \grad) \grad )
%%+ x \cdot (\grad \wedge (x \wedge \grad))
%+(-1 + x \cdot \grad )(x \wedge \grad)
%-x \wedge (x \cdot \grad) \grad \phi
&=
(n-2) (x \wedge \grad)
\end{aligned}
\end{equation}

The power of this result is that it allows us to write the scalar angular momentum operator from the Laplacian as

\begin{equation}\label{eqn:bivectorSelect:233}
\begin{aligned}
\gpgradezero{ (x \wedge \grad)^2 }
&= (x \wedge \grad)^2 - \gpgradetwo{ (x \wedge \grad)^2 } - (x \wedge \grad) \wedge (x \wedge \grad) \\
&= (x \wedge \grad)^2 - (n-2) (x \wedge \grad) - (x \wedge \grad) \wedge (x \wedge \grad) \\
&= (-(n-2) + (x \wedge \grad) - (x \wedge \grad) \wedge ) (x \wedge \grad)
\end{aligned}
\end{equation}

The complete Laplacian is

\begin{equation}\label{eqn:bivectorSelect:goo12}
\begin{aligned}
\grad^2 &= \inv{x^2} (x \cdot \grad)^2 + (n - 2) \inv{x} \cdot \grad
- \inv{x^2}
\left(
(x \wedge \grad)^2 - (n-2) (x \wedge \grad) - (x \wedge \grad) \wedge (x \wedge \grad)
\right)
\end{aligned}
\end{equation}

In particular in less than four dimensions the quad-vector term is necessarily zero.  The 3D Laplacian becomes

\begin{equation}\label{eqn:bivectorSelect:goo13}
\begin{aligned}
\spacegrad^2 &= \inv{\Bx^2} (1 + \Bx \cdot \spacegrad)(\Bx \cdot \spacegrad)
+ \inv{\Bx^2} (1 - \Bx \wedge \spacegrad) (\Bx \wedge \spacegrad)
\end{aligned}
\end{equation}

So any eigenfunction of the bivector angular momentum operator \(\Bx \wedge \spacegrad\) is necessarily a simultaneous eigenfunction of the scalar operator.

%%\EndArticle
%\EndNoBibArticle


\part{Fourier treatments}
   %
% Copyright � 2012 Peeter Joot.  All Rights Reserved.
% Licenced as described in the file LICENSE under the root directory of this GIT repository.
%

%
%
\mychapter{Fourier Solutions to Heat and Wave equations}
\label{chap:PJheatFourier}
\index{heat equation!Fourier transform}
\index{wave equation!Fourier transform}
\index{Fourier transform!heat equation}
\index{Fourier transform!wave equation}
%\date{Jan 19, 2009.  heatFourier.tex}

\section{Motivation}

Stanford iTunesU has some Fourier transform lectures by Prof. Brad Osgood.
He starts with Fourier series and by Lecture 5 has covered this and
the solution of the Heat equation on a ring as an example.

Now, for these lectures I get only sound on my ipod.  I can listen along and
pick up most of the lectures since this is review material, but here is some
notes to firm things up.

Since this heat equation

\begin{equation}\label{eqn:heatFourier:20}
\begin{aligned}
\grad^2 u = \kappa \partial_t u
\end{aligned}
\end{equation}

is also the Schr\"{o}dinger equation for a free particle in one
dimension (once the
constant is fixed appropriately), we can also apply the Fourier
technique to a particle
constrained to a circle.  It would be interesting afterwards to
contrast this with Susskind's solution of the
same problem (where he used the Fourier transform and algebraic techniques
instead).

\section{Preliminaries}

\subsection{Laplacian}
\index{Laplacian}

Osgood wrote the heat equation for the ring as

\begin{equation}\label{eqn:heatFourier:40}
\begin{aligned}
\inv{2} u_{xx} = u_t
\end{aligned}
\end{equation}

where \(x\) represented an angular position on the ring, and where
he set the heat diffusion constant to \(1/2\) for convenience.
To apply this to the Schr\"{o}dinger equation retaining all the desired
units we want to be a bit more careful, so let us start with the Laplacian
in polar coordinates.

In polar coordinates our gradient is

\begin{equation}\label{eqn:heatFourier:60}
\begin{aligned}
\grad = \thetacap \inv{r} \PD{\theta}{} +\rcap \PD{r}{}
\end{aligned}
\end{equation}

squaring this we have

\begin{equation}\label{eqn:heatFourier:80}
\begin{aligned}
\grad^2 = \grad \cdot \grad
&=
\thetacap \inv{r} \PD{\theta}{} \cdot \left(\thetacap \inv{r} \PD{\theta}{}\right)
 +
\rcap \PD{r}{} \cdot \left(\rcap \PD{r}{} \right) \\
&=
\frac{-1}{r^3} \PD{\theta}{r} \PD{\theta}{}
+\inv{r^2} \PDSq{\theta}{}
+ \PDSq{r}{}
\\
&= \inv{r^2} \PDSq{\theta}{} + \PDSq{r}{} \\
\end{aligned}
\end{equation}

So for the circularly constrained where \(r\) is constant case we have simply

\begin{equation}\label{eqn:heatFourier:100}
\begin{aligned}
\grad^2 = \inv{r^2} \PDSq{\theta}{}
\end{aligned}
\end{equation}

and our heat equation to solve becomes

\begin{equation}\label{eqn:heatFourier:120}
\begin{aligned}
\PDSq{\theta}{u(\theta, t)} = (r^2 \kappa) \PD{t}{u(\theta, t)}
\end{aligned}
\end{equation}

\subsection{Fourier series}
\index{Fourier series}

Now we also want Fourier series for a given period.  Assuming the absence of the "Rigor Police" as Osgood puts it
we write for a periodic function \(f(x)\) known on the interval \(I = [a, a+T]\)

\begin{equation}\label{eqn:heatFourier:140}
\begin{aligned}
f(x) = \sum c_k e^{2\pi i k x/T}
\end{aligned}
\end{equation}

\begin{equation}\label{eqn:heatFourier:160}
\begin{aligned}
\int_{\partial I} f(x) e^{- 2 \pi i n x /T}
&= \sum c_k \int_{\partial I} e^{2\pi i (k -n) x/T} \\
&= c_n T
\end{aligned}
\end{equation}

So our Fourier coefficient is
\begin{equation}\label{eqn:heatFourier:180}
\begin{aligned}
\hat{f}(n) = c_n = \inv{T} \int_{\partial I} f(x) e^{- 2 \pi i n x /T}
\end{aligned}
\end{equation}

\section{Solution of heat equation}

\subsection{Basic solution}

Now we are ready to solve the radial heat equation

\begin{equation}\label{eqn:heat_fourier:heatRadial}
\begin{aligned}
u_{\theta\theta} = r^2 \kappa u_t,
\end{aligned}
\end{equation}

by assuming a Fourier series solution.

Suppose

\begin{equation}\label{eqn:heatFourier:200}
\begin{aligned}
u(\theta, t)
&= \sum c_n(t) e^{2 \pi i n \theta / T} \\
&= \sum c_n(t) e^{i n \theta} \\
\end{aligned}
\end{equation}

Taking derivatives of this assumed solution we have
\begin{equation}\label{eqn:heatFourier:220}
\begin{aligned}
u_{\theta\theta} &= \sum (i n)^2 c_n e^{i n \theta} \\
u_{t} &= \sum c_n' e^{i n \theta}
\end{aligned}
\end{equation}

Substituting this back into \eqnref{eqn:heat_fourier:heatRadial} we have

\begin{equation}\label{eqn:heatFourier:240}
\begin{aligned}
\sum - n^2 c_n e^{ i n \theta} = \sum c_n' r^2 \kappa e^{i n \theta}
\end{aligned}
\end{equation}

equating components we have

\begin{equation}\label{eqn:heatFourier:260}
\begin{aligned}
c_n' = - \frac{n^2}{ r^2 \kappa } c_n
\end{aligned}
\end{equation}

which is also just an exponential.

\begin{equation}\label{eqn:heatFourier:280}
\begin{aligned}
c_n = A_n \exp\left(- \frac{n^2}{ r^2 \kappa } t \right)
\end{aligned}
\end{equation}

Reassembling we have the time variation of the solution now fixed and can write

\begin{equation}\label{eqn:heatFourier:300}
\begin{aligned}
u(\theta, t) = \sum A_n \exp\left(- \frac{n^2}{ r^2 \kappa } t + i n \theta\right)
\end{aligned}
\end{equation}

\subsection{As initial value problem}

For the heat equation case, we can assume a known initial heat distribution
\(f(\theta)\).
For an initial time \(t=0\) we can then write

\begin{equation}\label{eqn:heatFourier:320}
\begin{aligned}
u(\theta, 0) = \sum A_n e^{i n \theta} = f(\theta)
\end{aligned}
\end{equation}

This is just another Fourier series, with Fourier coefficients

\begin{equation}\label{eqn:heatFourier:340}
\begin{aligned}
A_n = \inv{2\pi} \int_{\partial I} f(v) e^{-i n v} dv
\end{aligned}
\end{equation}

Final reassembly of the results gives us

\begin{equation}\label{eqn:heatFourier:360}
\begin{aligned}
u(\theta, t) = \sum \exp\left(- \frac{n^2}{ r^2 \kappa } t + i n \theta\right) \inv{2\pi} \int_{\partial I} f(v) e^{-i n v} dv
\end{aligned}
\end{equation}

\subsection{Convolution}
\index{convolution}

Osgood's next step, also with the rigor police in hiding, was to exchange orders of integration and summation, to write

\begin{equation}\label{eqn:heatFourier:380}
\begin{aligned}
u(\theta, t)
&=
\int_{\partial I} f(v) dv \inv{2 \pi} \sum_{n=-\infty}^{\infty} \exp\left(- \frac{n^2}{ r^2 \kappa } t -i n (v -\theta)\right) \\
\end{aligned}
\end{equation}

Introducing a Green's function \(g(v, t)\), we then have the complete solution in terms of convolution

\begin{equation}\label{eqn:heat_fourier:seriesGreens}
\begin{aligned}
g( v , t ) &= \inv{2 \pi} \sum_{n=-\infty}^\infty \exp\left(- \frac{n^2}{ r^2 \kappa } t -i n v \right) \\
u(\theta, t) &= \int_{\partial I} f(v) g(v - \theta, t) dv
\end{aligned}
\end{equation}

Now, this Green's function is fairly interesting.  By summing over paired negative and positive indices, we have a set of weighted Gaussians.

\begin{equation}\label{eqn:heatFourier:400}
\begin{aligned}
g( v , t ) &= \inv{2 \pi} + \sum_{n=1}^\infty \exp\left(- \frac{n^2}{ r^2 \kappa } t \right) \frac{\cos(n v )}{\pi} \\
\end{aligned}
\end{equation}

Recalling that the delta function can be expressed as a limit of a \(\sinc\) function, seeing something similar in this Green's function is not entirely unsurprising seeming.

\section{Wave equation}

The QM equation for a free particle is

\begin{equation}\label{eqn:heat_fourier:schro}
\begin{aligned}
-\frac{\Hbar^2}{2m} \grad^2 \psi = i \Hbar \partial_t \psi
\end{aligned}
\end{equation}

This has the same form of the heat equation, so for the free particle on a circle our wave equation is

\begin{equation}\label{eqn:heatFourier:420}
\begin{aligned}
\psi_{\theta\theta} = - \frac{2 m i r^2 }{\Hbar} \partial_t \psi \quad \mbox{ ie: \(\kappa = - 2 m i /\Hbar\) }
\end{aligned}
\end{equation}

So, if the wave equation was known at an initial time \(\psi(\theta, 0) = \phi(\theta)\), we therefore have by comparison the time evolution of the particle's wave function is

\begin{equation}\label{eqn:heatFourier:440}
\begin{aligned}
g( w, t ) &= \inv{2 \pi} + \sum_{n=1}^\infty \exp\left(- \frac{i \Hbar n^2 t}{ 2 m r^2 } \right) \frac{\cos(n w )}{\pi} \\
\psi(\theta, t) &= \int_{\partial I} \phi(v) g(v - \theta, t) dv
\end{aligned}
\end{equation}

%TODO: contrast this to a Fourier transform solution.  Also write this in terms of circular angular momentum since that appears natually in the Green's function.

\section{Fourier transform solution}

% also see example (brief on details)
%\href{http://zakuski.utsa.edu/~gokhman/ftp//courses/notes/heat.pdf}{ example of Fourier tx solution. }
Now, lets try this one dimensional heat problem with a Fourier transform instead to compare.  Here we do not try to start with an
assumed solution, but instead take the Fourier transform of both sides of the equation directly.

\begin{equation}\label{eqn:heatFourier:460}
\begin{aligned}
\calF(u_{xx}) = \kappa \calF(u_t)
\end{aligned}
\end{equation}

Let us start with the left hand side, where we can evaluate by integrating by parts

\begin{equation}\label{eqn:heatFourier:480}
\begin{aligned}
\calF(u_{xx})
&=
%\inv{\sqrt{2\pi}}
\IIinf u_{xx}(x, t) e^{- 2 \pi i s x } dx \\
&=
%\inv{\sqrt{2\pi}}
\IIinf \PD{x}{u_x(x, t)} e^{- 2 \pi i s x } dx \\
&=
%\inv{\sqrt{2\pi}}
\left(
{\left. u_x(x, t) e^{- 2 \pi i s x } \right\vert}_{x= -\infty}^\infty
-( - 2 \pi i s ) \IIinf u_x(x, t) e^{- 2 \pi i s x } dx
\right) \\
\end{aligned}
\end{equation}

So if we assume (or require) that the derivative of our unknown function \(u\) is zero at infinity, and then similarly
require the function itself to be zero there, we have

\begin{equation}\label{eqn:heatFourier:500}
\begin{aligned}
\calF(u_{xx})
&=
%\inv{\sqrt{2\pi}}
( 2 \pi i s ) \IIinf \PD{x}{u_x(x, t)} e^{- 2 \pi i s x } dx  \\
&=
%\inv{\sqrt{2\pi}}
( 2 \pi i s )^2 \IIinf u(x, t) e^{- 2 \pi i s x } dx  \\
&= ( 2 \pi i s )^2 \calF(u)
\end{aligned}
\end{equation}

Now, for the time derivative.  We want

\begin{equation}\label{eqn:heatFourier:520}
\begin{aligned}
\calF(u_t) &=
%\inv{\sqrt{2\pi}}
\IIinf u_t(x, t) e^{- 2 \pi i s x } dx \\
\end{aligned}
\end{equation}

But can pull the derivative out of the integral for
\begin{equation}\label{eqn:heatFourier:540}
\begin{aligned}
\calF(u_t)
&= \PD{t}{} \left(
%\inv{\sqrt{2\pi}}
\IIinf u(x, t) e^{- 2 \pi i s x } dx \right) \\
&= \PD{t}{\calF(u)}
\end{aligned}
\end{equation}

So, now we have an equation relating time derivatives only of the Fourier transformed solution.

Writing \(\calF(u) = \hat{u}\) this is

\begin{equation}\label{eqn:heat_fourier:toSolveFreq}
\begin{aligned}
( 2 \pi i s )^2 \hat{u} = \kappa \PD{t}{\hat{u}}
\end{aligned}
\end{equation}

With a solution of

\begin{equation}\label{eqn:heatFourier:560}
\begin{aligned}
\hat{u} = A(s) e^{ -4 \pi^2 s^2 t/ \kappa }
\end{aligned}
\end{equation}

Here \(A(s)\) is an arbitrary constant in time integration constant, which may depend on \(s\) since it is a solution of our simpler frequency domain partial differential equation
\eqnref{eqn:heat_fourier:toSolveFreq}.

Performing an inverse transform to recover \(u(x,t)\) we thus have

\begin{equation}\label{eqn:heatFourier:580}
\begin{aligned}
u(x,t)
&=
%\inv{\sqrt{2\pi}}
\IIinf \hat{u} e^{2 \pi i x s } ds  \\
&=
%\inv{\sqrt{2\pi}}
\IIinf A(s) e^{ -4 \pi^2 s^2 t/ \kappa } e^{2 \pi i x s } ds  \\
\end{aligned}
\end{equation}

Now, how about initial conditions.  Suppose we have \(u(x,0) = f(x)\), then

\begin{equation}\label{eqn:heatFourier:600}
\begin{aligned}
f(x) &=
%\inv{\sqrt{2\pi}}
\IIinf A(s) e^{2 \pi i x s } ds \\
\end{aligned}
\end{equation}

Which is just an inverse Fourier transform in terms of the integration ``constant'' \(A(s)\).  We can therefore write the \(A(s)\) in terms of the
initial time domain conditions.

\begin{equation}\label{eqn:heatFourier:620}
\begin{aligned}
A(s) &=
%\inv{\sqrt{2\pi}}
\IIinf f(x) e^{-2 \pi i s x } dx \\
&= \hat{f}(s)
\end{aligned}
\end{equation}

and finally have a complete solution of the one dimensional Heat equation.  That is

\begin{equation}\label{eqn:heatFourier:640}
\begin{aligned}
u(x,t) &=
%\inv{\sqrt{2\pi}}
\IIinf \hat{f}(s) e^{ -4 \pi^2 s^2 t/ \kappa } e^{2 \pi i x s } ds  \\
\end{aligned}
\end{equation}

\subsection{With Green's function?}
\index{Green's function}

If we put in the integral for \(\hat{f}(s)\) explicitly and switch the order as was done with the Fourier series will we get a similar result?   Let us try

\begin{equation}\label{eqn:heatFourier:660}
\begin{aligned}
u(x,t)
&=
%\inv{\sqrt{2\pi}}
\IIinf \left(
%\inv{\sqrt{2\pi}}
\IIinf f(u) e^{-2 \pi i s u } du \right) e^{ -4 \pi^2 s^2 t/ \kappa } e^{2 \pi i x s } ds  \\
&=
%\inv{\sqrt{2\pi}}
\IIinf du f(u)
%\inv{\sqrt{2\pi}}
\IIinf e^{ -4 \pi^2 s^2 t/ \kappa } e^{2 \pi i (x - u) s } ds  \\
\end{aligned}
\end{equation}

Cool.  So, with the introduction of a Green's function \(g(w,t)\) for the fundamental solution of the heat equation, we therefore have our solution in terms of convolution with the initial conditions.  It does not get any more general than this!

\begin{equation}\label{eqn:heatFourier:680}
\begin{aligned}
g(w,t) &=
%\inv{{2\pi}}
\IIinf \exp\left( -\frac{4 \pi^2 s^2 t}{\kappa} + 2 \pi i w s \right) ds \\
u(x,t) &= \IIinf f(u) g( x - u, t) du
\end{aligned}
\end{equation}

Compare this to \eqnref{eqn:heat_fourier:seriesGreens}, the solution in terms of Fourier series.  The form is almost identical, but the requirement for periodicity has been removed by switch to the continuous frequency domain!

\subsection{Wave equation}
\index{wave equation}

With only a change of variables, setting \(\kappa = - 2 m i /\Hbar\) we have the general solution to the one dimensional zero potential wave equation \eqnref{eqn:heat_fourier:schro} in terms of an initial wave function.  However, we have a form of the Fourier transform that obscures the physics has been picked here unfortunately.  Let us start over in super speed mode directly from the wave equation, using the form of the Fourier transform that substitutes \(2\pi s \rightarrow k\) for wave number.

We want to solve
\begin{equation}\label{eqn:heatFourier:700}
\begin{aligned}
-\frac{\Hbar^2}{2m} \psi_{xx} = i \Hbar \psi_t
\end{aligned}
\end{equation}

Now calculate
\begin{equation}\label{eqn:heatFourier:720}
\begin{aligned}
\calF(\psi_{xx})
&= \inv{2\pi} \IIinf \psi_{xx}(x,t) e^{-i k x} dx \\
&=
\inv{2\pi} {\left.\psi_{x}(x,t) e^{-i k x} \right\vert}_{-\infty}^\infty
-(- i k) \inv{2\pi} \IIinf \psi_{x}(x,t) e^{-i k x} dx \\
&= \cdots \\
&= \inv{2\pi} (i k)^2 \hat{\psi}(k)
\end{aligned}
\end{equation}

So we have

\begin{equation}\label{eqn:heatFourier:740}
\begin{aligned}
-\frac{\Hbar^2}{2m} (ik)^2 \hat{\psi}(k,t) = i \Hbar \PD{t}{\hat{\psi}(k,t)}
\end{aligned}
\end{equation}

This provides us the fundamental solutions to the wave function in the wave
number domain

\begin{equation}\label{eqn:heatFourier:760}
\begin{aligned}
\hat{\psi}(k,t) &= A(k) \exp\left( -\frac{i \Hbar k^2}{ 2 m } t \right) \\
{\psi}(x,t) &=
\inv{\sqrt{2\pi}} \IIinf A(k) \exp\left( -\frac{i \Hbar k^2}{ 2 m } t \right)
\exp( i k x ) dk \\
\end{aligned}
\end{equation}

In particular, as before, with an initial time wave function \(\psi(x,0) = \phi(x)\) we have

\begin{equation}\label{eqn:heatFourier:780}
\begin{aligned}
\phi(x) = {\psi}(x,0)
&= \inv{\sqrt{2\pi}} \IIinf A(k) \exp( i k x ) dk \\
&= \calF^{-1}(A(k))
\end{aligned}
\end{equation}

So, \(A(k) = \hat{\phi}\), and we have

\begin{equation}\label{eqn:heatFourier:800}
\begin{aligned}
{\psi}(x,t) &=
\inv{\sqrt{2\pi}} \IIinf \hat{\phi}(k) \exp\left( -\frac{i \Hbar k^2}{ 2 m } t \right) \exp( i k x ) dk \\
\end{aligned}
\end{equation}

So, ending the story we have finally, the general solution for the time evolution of our one dimensional wave function given
initial conditions

\begin{equation}\label{eqn:heatFourier:820}
\begin{aligned}
{\psi}(x,t) &= \calF^{-1}\left( \hat{\phi}(k) \exp\left( -\frac{i \Hbar k^2}{ 2 m } t \right) \right)
\end{aligned}
\end{equation}

or, alternatively, in terms of momentum via \(k = p/\Hbar\) we have

\begin{equation}\label{eqn:heatFourier:840}
\begin{aligned}
{\psi}(x,t) &= \calF^{-1}\left( \hat{\phi}(p) \exp\left( -\frac{i p^2}{ 2 m \Hbar } t \right) \right)
\end{aligned}
\end{equation}

Pretty cool!  Observe that in the wave number or momentum domain the time evolution of the wave function is just a continual phase shift relative to the initial conditions.

%Our solution is
%
%\begin{align}
%g(w,t) &=
%%\inv{{2\pi}}
%\IIinf \exp\left(
%2 \pi i s w
%-\frac{2 \pi^2 s^2 i t \Hbar}{m}
%\right) ds \\
%u(x,t) &= \IIinf f(u) g( x - u, t) du
%\end{align}

\subsection{Wave function solutions by Fourier transform for a particle on a circle}

Now, thinking about how to translate this Fourier transform method to the
wave equation for a particle on a circle (as done by Susskind in his online lectures) makes me realize that one
is free to use any sort of integral transform method appropriate for the problem (Fourier, Laplace, ...).
It does not have to be the Fourier transform.  Now, if we happen to pick an integral transform with \(\theta \in [0, \pi]\) bounds, what do we have?  This is nothing more than the inner product for the Fourier series, and we come full circle!

Now, the next thing to work out in detail is how to translate from the transform methods to the algebraic bra ket notation.  This
looks like it will follow immediately if one calls out the inner product in use explicitly, but that is an exploration for a
different day.

   %
% Copyright � 2012 Peeter Joot.  All Rights Reserved.
% Licenced as described in the file LICENSE under the root directory of this GIT repository.
%

%
%
%\input{../peeter_prologue_print.tex}

\chapter{Poisson and retarded Potential Green's functions from Fourier kernels}\label{chap:PJpoisson}
\index{retarded potential!Green's function}
\index{Poisson potential!Green's function}
%\date{Feb 18, 2009.  poisson.tex}

\beginArtWithToc
\section{Motivation}

Having recently attempted a number of Fourier solutions to the Heat, Schr\"{o}dinger, Maxwell vacuum, and inhomogeneous Maxwell equation, a reading
of \citep{mjPerryElectrodynamics} inspired me to have another go.  In particular, he writes the Poisson equation solution explicitly in terms of a Green's
function.

The Green's function for the Poisson equation

\begin{equation}\label{eqn:poisson:50}
\begin{aligned}
G(\mathbf{x} - \Bx') = \frac{1}{4\pi\lvert{\mathbf{x} -\Bx'}\rvert}
\end{aligned}
\end{equation}

is not really derived, rather is just pointed out.  However, it is a nice closed form that does not have any integrals.  Contrast this to the Fourier transform method, where one ends up with a messy threefold integral that is not particularly obvious how to integrate.

In the PF thread 
\href{https://www.physicsforums.com/threads/fourier-transform-solution-to-electrostatics-poisson-equation.293550/}{Fourier transform solution to electrostatics Poisson equation?} I asked if anybody knew how to reduce this integral to the potential kernel of electrostatics.  Before getting any answer from PF I found it in \citep{byron1992mca}, a book recently purchased, but not yet read.

Go through this calculation here myself in full detail to get more comfort with the ideas.  Some of these ideas can probably also be applied to previous incomplete Fourier solution attempts.  In particular, the retarded time potential solutions likely follow.  Can these same ideas be applied to the STA form of the Maxwell equation, explicitly inverting it, as \citep{doran2003gap} indicate is possible (but do not spell out).

\section{Poisson equation}
\subsection{Setup}

As often illustrated with the Heat equation, we seek a Fourier transform solution of the electrostatics Poisson equation

\begin{equation}\label{eqn:poisson:LaplacianPhi}
\nabla^2 \phi = -\rho/\epsilon_0
\end{equation}

Our 3D Fourier transform pairs are defined as
\begin{equation}\label{eqn:poisson:70}
\begin{aligned}
\hat{f}(\mathbf{k}) &= \frac{1}{(\sqrt{2\pi})^3} \iiint f(\mathbf{x}) e^{-i \mathbf{k} \cdot \mathbf{x} } d^3 x \\
{f}(\mathbf{x}) &= \frac{1}{(\sqrt{2\pi})^3} \iiint \hat{f}(\mathbf{k}) e^{i \mathbf{k} \cdot \mathbf{x} } d^3 k \\
\end{aligned}
\end{equation}

Applying the transform we get

\begin{equation}\label{eqn:poisson:poissonSolution}
\begin{aligned}
\phi(\mathbf{x}) &= \frac{1}{\epsilon_0} \iiint \rho(\mathbf{x}') G(\mathbf{x-x'}) d^3 x' \\
G(\mathbf{x}) &= \frac{1}{(2 \pi)^3} \iiint \frac{1}{\mathbf{k}^2} e^{ i \mathbf{k} \cdot \mathbf{x} } d^3 k
\end{aligned}
\end{equation}

Green's functions are usually defined by their delta function operational properties.  Doing so, as defined above we have

\begin{equation}\label{eqn:poisson:greens}
\nabla^2 G(\Bx) = - 4 \pi \delta^3(\Bx)
\end{equation}

(note that there are different sign conventions for this delta function identification.)

Application to the Poisson equation \eqnref{eqn:poisson:LaplacianPhi} gives

\begin{equation}\label{eqn:poisson:10}
\int \nabla^2 G(\Bx - \Bx') \phi(\Bx') = \int (- 4 \pi \delta^3(\Bx - \Bx')) \phi(\Bx') = - 4 \pi \phi(\Bx)
\end{equation}

and with expansion in the alternate sequence
\begin{equation}\label{eqn:poisson:20}
\int \nabla^2 G(\Bx - \Bx') \phi(\Bx') = \int G(\Bx - \Bx') ({\nabla'}^2 \phi(\Bx')) = -\inv{\epsilon_0}\int G(\Bx - \Bx') \rho(\Bx')
\end{equation}

With prior knowledge of electrostatics we should therefore find

\begin{equation}\label{eqn:poisson:30}
G(\Bx) = \inv{4 \pi \Abs{\Bx}}.
\end{equation}

Our task is to actually compute this from the Fourier integral.

\subsection{Evaluating the convolution kernel integral}

\subsubsection{Some initial thoughts}

Now it seems to me that this integral \(G\) only has to be evaluated around a small neighborhood of the origin.  For example if one evaluates one of the integrals

\begin{equation}\label{eqn:poisson:90}
\begin{aligned}
\int_{-\infty}^\infty \frac{1}{{k_1}^2 + {k_2}^2 + {k_3}^3 } e^{ i k_1 x_1 } dk_1
\end{aligned}
\end{equation}

using a an upper half plane contour the result is zero unless \(k_2 = k_3 = 0\).  So one is left with something loosely like

\begin{equation}\label{eqn:poisson:110}
\begin{aligned}
G(\mathbf{x}) &= \lim_{\epsilon \rightarrow 0} \frac{1}{(2 \pi)^3}
\int_{k_1 = -\epsilon}^{\epsilon} dk_1
\int_{k_2 = -\epsilon}^{\epsilon} dk_2
\int_{k_3 = -\epsilon}^{\epsilon} dk_3
 \frac{1}{\mathbf{k}^2} e^{ i \mathbf{k} \cdot \mathbf{x} }
\end{aligned}
\end{equation}

How to reduce this?  Somehow it must be possible to take this Fourier convolution kernel and somehow evaluate the integral to produce the electrostatics potential.

\subsubsection{An attempt}

The answer of how to do so, as pointed out above, was found in \citep{byron1992mca}.  Instead of trying to evaluate this integral which has a pole at the origin, they cleverly evaluate a variant of it

\begin{equation}\label{eqn:poisson:130}
\begin{aligned}
I = \iiint \frac{1}{\mathbf{k}^2 + a^2} e^{ i \mathbf{k} \cdot \mathbf{x} } d^3 k
\end{aligned}
\end{equation}

which splits and shifts the repeated pole into two first order poles away from the origin.
After a change to spherical polar coordinates, the new integral can be evaluated, and the Poisson Green's function in potential form follows by
letting \(a\) tend to zero.

Very cool.  It seems worthwhile to go through the motions of this myself, omitting no details I would find valuable.

First we want the volume element in spherical polar form, and our vector.  That is

\begin{equation}\label{eqn:poisson:150}
\begin{aligned}
\rho &= k \cos\phi \\
dA &= (\rho d\theta)(k d\phi) \\
d^3 k &= dk dA = k^2 \cos\phi d\theta d\phi dk \\
\Bk
&= (\rho \cos\theta, \rho \sin\theta, k \sin\theta) \\
&= k (\cos\phi \cos\theta, \cos\phi \sin\theta, \sin\phi) \\
\end{aligned}
\end{equation}

FIXME: scan picture to show angle conventions picked.

This produces
\begin{equation}\label{eqn:poisson:170}
\begin{aligned}
I &= \int_{\theta=0}^{2\pi} \int_{\phi=-\pi/2}^{\pi/2}
k^2
\int_{k=0}^\infty
\cos\phi d\theta d\phi dk  \\
&\qquad \frac{1}{k^2 + a^2} \exp\left( i k (\cos\phi \cos\theta x_1 + \cos\phi \sin\theta + x_2 + \sin\phi x_3) \right)
\end{aligned}
\end{equation}

Now, this is a lot less tractable than the Byron/Fuller treatment.  In particular they were able to make a \(t = \cos\phi\) substitution, and if I try this I get

\begin{equation}\label{eqn:poisson:190}
\begin{aligned}
I = -\int_{\theta=0}^{2\pi} \int_{t=-1}^{1} \int_{k=0}^\infty \frac{1}{k^2 + a^2} \exp\left( i k (t \cos\theta x_1 + t \sin\theta x_2 + \sqrt{1-t^2} x_3) \right)
k^2 dt d\theta dk \\
\end{aligned}
\end{equation}

Now, this is still a whole lot different, and in particular it has \(ik (t\sin\theta x_2 + \sqrt{1-t^2} x_3)\) in the exponential.  I puzzled over this for a while, but it becomes clear on writing.  Freedom to orient the axis along a preferable direction has been used, and some basis for which \(\Bx = x_j \Be^j + = x \Be^1\) has been used!  We are now left with

\begin{equation}\label{eqn:poisson:210}
\begin{aligned}
I
&= -\int_{\theta=0}^{2\pi} \int_{t=-1}^{1} \int_{k=0}^\infty \frac{1}{k^2 + a^2} \exp\left( i k t \cos\theta x \right) k^2 dt d\theta dk \\
&= -\int_{\theta=0}^{2\pi} \int_{k=0}^\infty \frac{2}{(k^2 + a^2) \cos\theta} \sin\left( k t \cos\theta x \right) k d\theta dk \\
&= -\int_{\theta=0}^{2\pi} \int_{k=-\infty}^\infty \frac{1}{(k^2 + a^2) \cos\theta} \sin\left( k t \cos\theta x \right) k d\theta dk \\
\end{aligned}
\end{equation}

Here the fact that our integral kernel is even in \(k\) has been used to double the range and half the kernel.

However, looking at this, one can see that there is trouble.  In particular, we have \(\cos\theta\) in the denominator, with a range that allows zeros.  How did the text avoid this trouble?

\subsection{Take II}

After mulling it over for a bit, it appears that aligning \(\Bx\) with the x-axis is causing the trouble.  Aligning
with the
z-axis will work out much better, and leave only one trig term in the exponential.  Essentially we need to
use a volume of rotation about the z-axis, integrating along all sets of constant \(\Bk \cdot \Bx\).
This is a set of integrals over concentric ring volume elements (FIXME: picture).

Our volume element, measuring \(\theta \in [0, \pi]\) from the z-axis, and \(\phi\) as our position on the ring

\begin{equation}\label{eqn:poisson:230}
\begin{aligned}
\Bk \cdot \Bx &= k x \cos\theta \\
\rho &= k \sin\theta \\
dA &= (\rho d\phi)(k d\theta) \\
d^3 k &= dk dA = k^2 \sin\theta d\theta d\phi dk \\
\end{aligned}
\end{equation}

This gives us

\begin{equation}\label{eqn:poisson:250}
\begin{aligned}
I &= \int_{\theta=0}^{\pi} \int_{\phi=0}^{2\pi} \int_{k=0}^\infty \frac{1}{k^2 + a^2} \exp\left( i k x \cos\theta \right)
k^2 \sin\theta d\theta d\phi dk \\
\end{aligned}
\end{equation}

Now we can integrate immediately over \(\phi\), and make a \(t = \cos\theta\) substitution (\(dt = -\sin\theta d\theta\))

\begin{equation}\label{eqn:poisson:270}
\begin{aligned}
I
&= - 2\pi \int_{t=1}^{-1} \int_{k=0}^\infty \frac{1}{k^2 + a^2} \exp\left( i k x t \right) k^2 dt dk \\
&= - \frac{2\pi}{ix} \int_{k=0}^\infty \frac{1}{k^2 + a^2}
\left(
e^{-i k x } -e^{i k x }
\right)
k dk \\
&= \frac{2\pi}{ix} \int_{k=0}^\infty \frac{1}{k^2 + a^2}
e^{i k x }
k dk
- \frac{2\pi}{ix} \int_{k=-0}^{-\infty} \frac{1}{k^2 + a^2}
e^{i k x }
(-k) (-dk) \\
&= \frac{2\pi}{ix} \int_{k=-\infty}^\infty \frac{1}{k^2 + a^2} e^{i k x } k dk \\
&= \frac{2\pi}{ix} \int_{k=-\infty}^\infty \inv{k - ia} \frac{ k e^{ i k x } }{(k + ia)} dk \\
\end{aligned}
\end{equation}

Now we have something that is in form for contour integration.  In the upper half plane we have a pole at \(k= ia\).  Assuming that the
integral over the big semicircular arc vanishes, we can just pick up the residue at that pole contributing.  The assumption that
this vanishes is actually non-trivial looking since the \(k/(k+ia)\) term at a big radius \(R\) tends to \(1\).  This is probably where
Jordan's lemma comes in, so some study to understand that looks well justified.

\begin{equation}\label{eqn:poisson:290}
\begin{aligned}
0
&= I - 2 \pi i {\left. \frac{2\pi}{ix} \frac{ k e^{ i k x } }{(k + ia)} \right\vert}_{k= ia} \\
&= I - 2 \pi i \frac{2\pi}{ix} \frac{ e^{ - a x } }{2} \\
\end{aligned}
\end{equation}

So we have
\begin{equation}\label{eqn:poisson:310}
\begin{aligned}
I = \frac{2 \pi^2}{x} e^{ - a x }
\end{aligned}
\end{equation}

Now that we have this, the Green's function of \eqnref{eqn:poisson:poissonSolution} is

\begin{equation}\label{eqn:poisson:330}
\begin{aligned}
G(\mathbf{x})
&= \lim_{a \rightarrow 0} \inv{(2\pi)^3} \frac{2 \pi^2}{x} e^{ - a x } \\
&= \inv{4\pi\Abs{\Bx}}
\end{aligned}
\end{equation}

Which gives
\begin{equation}\label{eqn:poisson:350}
\begin{aligned}
\phi(\Bx) &= \inv{4\pi \epsilon_0} \int \frac{\rho(\Bx')}{\Abs{\Bx - \Bx'}} dV'
\end{aligned}
\end{equation}

Awesome!  All following from the choice to set \(\BE = -\spacegrad \phi\), we have a solution for \(\phi\) following directly
from the divergence equation \(\spacegrad \cdot \BE = \rho/\epsilon_0\) via Fourier transformation of this equation.

\section{Retarded time potentials for the 3D wave equation}

\subsection{Setup}

If we look at the general inhomogeneous Maxwell equation

\begin{equation}\label{eqn:poisson:370}
\begin{aligned}
\grad F = J/\epsilon_0 c
\end{aligned}
\end{equation}

In terms of potential \(F = \grad \wedge A\) and employing in the Lorentz gauge \(\grad \cdot A = 0\), we have

\begin{equation}\label{eqn:poisson:390}
\begin{aligned}
\grad^2 A = \left(\inv{c^2} \partial_{tt} - \sum \partial_{jj}\right) A = J/\epsilon_0 c
\end{aligned}
\end{equation}

As scalar equations with \(A = A^\mu \gamma_\mu\), \(J = J^\nu \gamma_\nu\) we have four equations all of the same form.

A Green's function form for such wave equations was previously calculated in \chapcite{PJfourierMaxwellSecondOrder}.  That was

\begin{equation}\label{eqn:poisson:410}
\begin{aligned}
\left( \inv{c^2} \PDSq{t}{} -\sum_j \PDSq{x^j}{} \right)\psi &= g
\end{aligned}
\end{equation}

\begin{equation}\label{eqn:poisson:430}
\begin{aligned}
{\psi}(\Bx, t) &= \IIinf \IIinf \IIinf \IIinf g(\Bx', t') G(\Bx - \Bx', t - t') d^3 x' dt' \\
G(\Bx, t) &= \theta(t) \IIinf \IIinf \IIinf \frac{c}{(2\pi)^3 \Abs{\Bk}} \sin( \Abs{\Bk} c t ) \exp\left( i \Bk \cdot \Bx \right) d^3 k
\end{aligned}
\end{equation}

Here \(\theta(t)\) is the unit step function, which meant we only sum the time contributions of the charge density for \(t - t' > 0\), or \(t' < t\).  That is the causal variant of the solution, which was arbitrary mathematically (\(t > t'\) would have also worked).

\subsection{Reducing the Green's function integral}

Let us see if the spherical polar method works to reduce this equation too.  In particular we want to evaluate

\begin{equation}\label{eqn:poisson:450}
\begin{aligned}
I = \IIinf \IIinf \IIinf \inv{\Abs{\Bk}} \sin( \Abs{\Bk} c t ) \exp\left( i \Bk \cdot \Bx \right) d^3 k
\end{aligned}
\end{equation}

Will we have a requirement to introduce a pole off the origin as above?  Perhaps like
\begin{equation}\label{eqn:poisson:470}
\begin{aligned}
\IIinf \IIinf \IIinf \inv{\Abs{\Bk} + \alpha} \sin( \Abs{\Bk} c t ) \exp\left( i \Bk \cdot \Bx \right) d^3 k
\end{aligned}
\end{equation}

Let us omit it for now, but make the same spherical polar substitution used successfully above, writing
%\begin{align*}
%I = \IIinf \IIinf \IIinf \inv{\Abs{\Bk} } \sin( \Abs{\Bk} c t ) \exp\left( i \Bk \cdot \Bx \right) d^3 k
%\end{align*}

\begin{equation}\label{eqn:poisson:490}
\begin{aligned}
I
&= \int_{\theta=0}^{\pi} \int_{\phi=0}^{2\pi} \int_{k=0}^\infty \frac{1}{k }
\sin\left( k c t \right) \exp\left( i k x \cos\theta \right)
k^2 \sin\theta d\theta d\phi dk \\
&= 2 \pi \int_{\theta=0}^{\pi} \int_{k=0}^\infty %\frac{1}{k }
\sin\left( k c t \right) \exp\left( i k x \cos\theta \right)
k \sin\theta d\theta dk \\
\end{aligned}
\end{equation}

Let \(\tau = \cos\theta\), \(-d\tau = \sin\theta d\theta\), for

\begin{equation}\label{eqn:poisson:510}
\begin{aligned}
I
&= 2 \pi \int_{\tau=1}^{-1} \int_{k=0}^\infty %\frac{1}{k }
\sin\left( k c t \right) \exp\left( i k x \tau \right)
k (-d\tau) dk \\
&= -2 \pi \int_{k=0}^\infty %\frac{1}{k }
\sin\left( k c t \right)
\frac{2}{2i k x} \left( {\exp\left( -i k x \right) } -{\exp\left( i k x \right) } \right)
k dk \\
&= \frac{4 \pi }{x} \int_{k=0}^\infty %\frac{1}{k }
\sin\left( k c t \right)
\sin\left( k x \right)
dk \\
&= \frac{2 \pi }{x} \int_{k=0}^\infty %\frac{1}{k }
\left( \cos\left( k (x-c t) \right) -\cos\left( k (x+c t) \right) \right)
 dk \\
\end{aligned}
\end{equation}

Okay, this is much simpler, but still not in a form that is immediately obvious how to apply contour integration to, since it has no poles.
The integral
kernel here is however an even function,
so we can use the trick of doubling the integral range.

\begin{equation}\label{eqn:poisson:530}
\begin{aligned}
I &= \frac{\pi }{x} \int_{k=-\infty}^\infty \left( \cos\left( k (x-c t) \right) -\cos\left( k (x+c t) \right) \right) dk \\
\end{aligned}
\end{equation}

Having done this, this integral is not really any more well defined.  With the Rigor police on holiday, let us assume we want the
principle value of this integral
\begin{equation}\label{eqn:poisson:550}
\begin{aligned}
I
&= \lim_{R \rightarrow \infty} \frac{\pi }{x} \int_{k=-R}^R \left( \cos\left( k (x-c t) \right) -\cos\left( k (x+c t) \right) \right) dk \\
&= \lim_{R \rightarrow \infty} \frac{\pi }{x} \int_{k=-R}^R d \left( \frac{\sin\left( k (x-c t) \right)}{ x - c t} - \frac{\sin\left( k (x+c t) \right)}{x + c t} \right) \\
&= \lim_{R \rightarrow \infty} \frac{2 \pi^2 }{x} \left( \frac{\sin\left( R (x-c t) \right)}{ \pi(x - c t)} - \frac{\sin\left( R (x+c t) \right)}{\pi(x + c t)} \right) \\
\end{aligned}
\end{equation}

This \(\sinc\) limit has been seen before being functionally identified with the delta function (the wikipedia article calls these ``nascent delta function''), so we can write

\begin{equation}\label{eqn:poisson:570}
\begin{aligned}
I &= \frac{2 \pi^2 }{x} \left( \delta(x-c t) - \delta(x+c t) \right)
\end{aligned}
\end{equation}

For our Green's function we now have

\begin{equation}\label{eqn:poisson:590}
\begin{aligned}
G(\Bx, t)
&= \theta(t) \frac{c }{(2\pi)^3} \frac{2 \pi^2 }{\Abs{\Bx}} \left( \delta(x-c t) - \delta(x+c t) \right) \\
&= \theta(t) \frac{c }{4\pi \Abs{\Bx}} \left( \delta(x-c t) - \delta(x+c t) \right) \\
\end{aligned}
\end{equation}

And finally, our wave function (switching variables to convolve with the charge density) instead of the Green's function

\begin{equation}\label{eqn:poisson:610}
\begin{aligned}
{\psi}(\Bx, t)
&=
\IIinf \IIinf \IIinf \IIinf g(\Bx - \Bx', t - t') \theta(t') \frac{c }{4\pi \Abs{\Bx'}} \delta(\Abs{\Bx'} - c t') d^3 x' dt' \\
&-\IIinf \IIinf \IIinf \IIinf g(\Bx - \Bx', t - t') \theta(t') \frac{c }{4\pi \Abs{\Bx'}} \delta(\Abs{\Bx'} + c t') d^3 x' dt' \\
\end{aligned}
\end{equation}

Let us break these into two parts

\begin{equation}\label{eqn:poisson:almostThere}
\begin{aligned}
{\psi}(\Bx, t) &= {\psi}_{-}(\Bx, t) +{\psi}_{+}(\Bx, t)
\end{aligned}
\end{equation}

Where the first part, \(\psi_{-}\) is for the \(-ct'\) delta function and one \(\psi_{-}\) for the \(+ct'\).
Making a \(\tau = t-t'\) change of variables, this first portion is

\begin{equation}\label{eqn:poisson:630}
\begin{aligned}
{\psi}_{-}(\Bx, t)
&= -\IIinf \IIinf \IIinf \IIinf g(\Bx - \Bx', \tau) \theta(t-\tau) \frac{c }{4\pi \Abs{\Bx'}} \delta(\Abs{\Bx'} - c t + c\tau) d^3 x' d\tau \\
&= -\IIinf \IIinf \IIinf g(\Bx - \Bx'', t - \Abs{\Bx''}/c) \frac{c }{4\pi \Abs{\Bx''}} d^3 x'' \\
\end{aligned}
\end{equation}

One more change of variables, \(\Bx' = \Bx - \Bx''\), \(d^3 x'' = -d^3 x\), gives the final desired retarded potential result.  The \(\psi_{+}\) result is similar (see below), and assembling all we have

\begin{equation}\label{eqn:poisson:650}
\begin{aligned}
{\psi}_{-}(\Bx, t) &= \IIinf \IIinf \IIinf g(\Bx', t - \Abs{\Bx - \Bx'}/c) \frac{c }{4\pi \Abs{\Bx - \Bx'}} d^3 x' \\
{\psi}_{+}(\Bx, t) &= -\IIinf \IIinf \IIinf g(\Bx, t + \Abs{\Bx -\Bx'}/c) \frac{c }{4\pi \Abs{\Bx -\Bx'}} d^3 x'
\end{aligned}
\end{equation}

It looks like my initial interpretation of the causal nature of the unit step in the original functional form was not really right.  It is not
until the Green's function is ``integrated'' do we get this causal and non-causal split into two specific solutions.
In the first of these solutions is only charge contributions at the position in space offset by the wave propagation speed effects the
potential (this is the causal case).  On the other hand we have a second specific solution to the wave equation
summing the charge contributions at all the future positions, this time offset by the time it takes a wave to propagate backwards from that future spacetime

The final mathematical result is consistent with statements seen elsewhere, such as in \citep{feynman1963flp}, although it is
likely that the path taken by others to get this result was less ad-hoc than mine.
It is been a couple years since seeing this for the first time in Feynman's text.
It was not clear to me how somebody could possibly come up with those starting with Maxwell's equations.
Here by essentially applying undergrad Engineering Fourier methods, we get the result in an admittedly ad-hoc fashion, but at least the result is not
pulled out of a magic hat.

\subsection{Omitted Details.  Advanced time solution}

Similar to the above for \(\psi_{+}\) we have
\begin{equation}\label{eqn:poisson:670}
\begin{aligned}
{\psi}_{+}(\Bx, t)
&= -\IIinf \IIinf \IIinf \IIinf g(\Bx - \Bx', t - t') \theta(t') \frac{c }{4\pi \Abs{\Bx'}} \delta(\Abs{\Bx'} + c t') d^3 x' dt' \\
&= \IIinf \IIinf \IIinf \IIinf g(\Bx - \Bx', \tau) \theta(t -\tau) \frac{c }{4\pi \Abs{\Bx'}} \delta(\Abs{\Bx'} + c (t -\tau)) d^3 x' d\tau \\
&= \IIinf \IIinf \IIinf \IIinf g(\Bx - \Bx', \tau) \theta(t -\tau) \frac{c }{4\pi \Abs{\Bx'}} \delta(\Abs{\Bx'} + c t -c\tau) d^3 x' d\tau \\
&= \IIinf \IIinf \IIinf g(\Bx - \Bx', t + \Abs{\Bx'}/c) \frac{c }{4\pi \Abs{\Bx'}} d^3 x' \\
&= -\IIinf \IIinf \IIinf g(\Bx, t + \Abs{\Bx -\Bx'}/c) \frac{c }{4\pi \Abs{\Bx -\Bx'}} d^3 x' \\
\end{aligned}
\end{equation}

Is there an extra factor of \(-1\) here?

\section{1D wave equation}

It is somewhat irregular seeming to treat the 3D case before what should be the simpler 1D case, so let us
try evaluating the Green's function for the 1D wave equation too.

We have found that Fourier transforms applied to the forced wave equation
\begin{equation}\label{eqn:poisson:690}
\begin{aligned}
\left( \inv{v^2} \partial_{tt} -\partial_{xx} \right)\psi = g(x,t)
\end{aligned}
\end{equation}

result in the following integral solution.

\begin{equation}\label{eqn:poisson:oneDimResult}
\begin{aligned}
{\psi}(x, t)
&=
\int_{x'=-\infty}^\infty
\int_{t' = 0}^\infty {g}(x-x', t-t') G(x', t') dx' dt' \\
G(x, t) &=
\int_{k = -\infty}^\infty
\frac{v}{2\pi {k}}
\sin( {k} v t )
\exp( i k x )
dk
\end{aligned}
\end{equation}

As in the 3D case above can this reduced to something that does not involve such an unpalatable integral.
Given the 3D result, it would be reasonable to get a result involving \(g(x \pm vt)\) terms.

First let us get rid of the sine term, and express \(G\) entirely in exponential form.  That is

\begin{equation}\label{eqn:poisson:710}
\begin{aligned}
G(x, t)
&=
\int_{k = -\infty}^\infty
\frac{v}{4\pi k i }
\left(\exp( {k} v t ) -\exp( -{k} v t )\right) \exp( i k x )
dk \\
&=
\int_{k = -\infty}^\infty
\frac{v}{4\pi k i }
\left(e^{ {k} (x + v t ) } - e^{ {k} (x - v t) }\right)
dk \\
\end{aligned}
\end{equation}

Using the unit step function identification from \eqnref{eqn:poisson:heavysideUnitIntegral}, we have

\begin{equation}\label{eqn:poisson:730}
\begin{aligned}
G(x, t) &= \frac{v}{2} \left(\theta(x + v t )  - \theta(x - v t) \right)
\end{aligned}
\end{equation}

If this identification works our solution then becomes

\begin{equation}\label{eqn:poisson:750}
\begin{aligned}
{\psi}(x, t)
&=
\int_{x'=-\infty}^\infty
\int_{t'= 0}^\infty
{g}(x-x', t-t')
%G(x', t')
\frac{v}{2} \left(\theta(x' + v t' )  - \theta(x' - v t') \right)
dx' dt' \\
&=
\int_{x'=-\infty}^\infty
\int_{s= 0}^\infty
{g}(x-x', t-s/v)
\frac{1}{2} \left(\theta(x' + s )  - \theta(x' - s) \right)
dx' ds
\end{aligned}
\end{equation}

This is already much simpler than the original, but additional reduction should be possible by breaking this down into specific intervals.  An
alternative, perhaps is to use integration by parts and the delta function as the derivative of the unit step identification.
%  Let us see
%if that works.

Let us try a pair of variable changes

\begin{equation}\label{eqn:poisson:770}
\begin{aligned}
{\psi}(x, t)
&=
% x' + s = u
% x' = u - s
\int_{u=-\infty}^\infty
\int_{s= 0}^\infty
{g}(x-u+s, t-s/v)
\frac{1}{2} \theta(u )
du ds \\
% x' - s = u
% x' = u + s
&-\int_{u=-\infty}^\infty
\int_{s= 0}^\infty
{g}(x-u -s, t-s/v)
\frac{1}{2} \theta(u)
du ds \\
\end{aligned}
\end{equation}

Like the retarded time potential solution to the 3D wave equation, we now have the wave function solution entirely specified by a
weighted sum of the driving function

\begin{equation}\label{eqn:poisson:790}
\begin{aligned}
{\psi}(x, t)
&=
\frac{1}{2}
\int_{u = 0}^\infty
\int_{s = 0}^\infty
\left( {g}(x-u+s, t-s/v) - {g}(x-u -s, t-s/v) \right)
du ds
\end{aligned}
\end{equation}

Can this be tidied at all?  Let us do a change of variables here, writing \(-\tau = t -s/v\).

% -\tau = t - s/v
% s/v = t + \tau
% s = v(t + \tau)
% ds = d\tau
% \tau(0) = -t
%
%
% \tau = t - s/v
% s/v = t - \tau
% s = v(t - \tau)
% ds = -d\tau
% \tau(0) = t
%
\begin{equation}\label{eqn:poisson:810}
\begin{aligned}
{\psi}(x, t)
&=
\frac{1}{2}
\int_{u = 0}^\infty
\int_{\tau = -t}^\infty
%\left( {g}(x-u+v(t + \tau), \tau) - {g}(x-u -v(t+\tau), \tau) \right)
\left( {g}(x+vt -(u - v\tau), \tau) - {g}(x-v t -(u +v\tau), \tau) \right)
du d\tau \\
&=
\frac{1}{2}
\int_{u = 0}^\infty
\int_{\tau = -\infty}^{t}
\left( {g}(x+vt -(u + v\tau), -\tau) - {g}(x-v t -(u -v\tau), -\tau) \right)
du d\tau \\
\end{aligned}
\end{equation}

Is that any better?  I am not so sure, and intuition says there is a way to reduce this to a single integral summing
only over spatial variation.

\subsection{Followup to verify}

There has been a lot of guessing and loose mathematics here.  However, if this is a valid solution despite all that, we should be
able to apply the wave function operator \(\inv{v^2} \partial_{tt} + \partial_{xx}\) as a
consistency check and get back \(g(x,t)\) by differentiating
under the integral sign.

FIXME: First have to think about how exactly to do this differentiation.

\section{Appendix}
\subsection{Integral form of unit step function}

The wiki article on the Heaviside unit step function lists an integral form

\begin{equation}\label{eqn:poisson:830}
\begin{aligned}
I_\epsilon &= \inv{2\pi i} \PV \IIinf \frac{e^{ix\tau}}{\tau - i\epsilon} d\tau \\
\theta(x) &= \lim_{\epsilon \rightarrow 0} I_\epsilon
\end{aligned}
\end{equation}

How does this make sense?  For \(x>0\) we can evaluate this with an upper half plane semi-circular contour (FIXME: picture).  Along the arc \(z = R e^{i\phi}\) we have

\begin{equation}\label{eqn:poisson:850}
\begin{aligned}
\Abs{I_\epsilon}
&= \Abs{\inv{2\pi i} \int_{\phi=0}^{\pi} \frac{e^{i R (\cos\phi + i \sin\phi)}}{ R e^{i\phi} - i \epsilon} R i e^{i\phi} d\phi} \\
&\approx \Abs{\inv{2\pi} \int_{\phi=0}^{\pi} e^{i R \cos\phi } e^{- R \sin\phi } d\phi} \\
&\le \inv{2\pi} \int_{\phi=0}^{\pi} e^{- R \sin\phi } d\phi \\
&\le \inv{2\pi} \int_{\phi=0}^{\pi} e^{- R } d\phi \\
&= \inv{2} e^{- R } \\
\end{aligned}
\end{equation}

This tends to zero as \(R \rightarrow \infty\), so evaluating the residue, we have for \(x > 0\)

\begin{equation}\label{eqn:poisson:870}
\begin{aligned}
{I_\epsilon}
&= -(-2 \pi i) \inv{2\pi i} {\left. {e^{ix\tau}} \right\vert}_{\tau = i\epsilon} \\
&= e^{-x\epsilon}
\end{aligned}
\end{equation}

Now for \(x<0\) an upper half plane contour will diverge, but the lower half plane can be used.  This gives us \(I_\epsilon = 0\) in that region.
All that remains is the \(x=0\) case.  There we have

\begin{equation}\label{eqn:poisson:890}
\begin{aligned}
I_\epsilon(0)
&= \inv{2\pi i} \PV \IIinf \inv{\tau - i\epsilon} d\tau \\
&= \inv{2\pi i} \lim_{R \rightarrow \infty} \ln\left( \frac{R - i\epsilon} {-R - i\epsilon} \right) \\
& \rightarrow \inv{2\pi i} \ln\left( -1 \right) \\
&= \inv{2\pi i} i \pi \\
\end{aligned}
\end{equation}

Summarizing we have

\begin{equation}\label{eqn:poisson:910}
\begin{aligned}
I_\epsilon(x) =
\left\{
\begin{array}{l l}
e^{-x\epsilon} & \quad \mbox{if \(x > 0\)} \\
\inv{2} & \quad \mbox{if \(x = 0\)} \\
0 & \quad \mbox{if \(x < 0\)} \\
\end{array}
\right.
\end{aligned}
\end{equation}

So in the limit this does work as an integral formulation of the unit step.  This will be used to (very loosely) identify

\begin{equation}\label{eqn:poisson:heavysideUnitIntegral}
\begin{aligned}
\theta(x) \sim \inv{2\pi i} \PV \IIinf \frac{e^{ix\tau}}{\tau} d\tau
\end{aligned}
\end{equation}
%\EndArticle

   %
% Copyright � 2012 Peeter Joot.  All Rights Reserved.
% Licenced as described in the file LICENSE under the root directory of this GIT repository.
%

%
%
\mychapter{Fourier transform solutions to the wave equation}
\index{Fourier transform!wave equation}
\index{wave equation!Fourier transform}
\label{chap:PJwaveFourier}
%\date{Jan 26, 2009.  waveFourier.tex}

\section{Mechanical wave equation solution}

We want to solve
%
\begin{equation}\label{eqn:wave_fourier:waveTwoDim}
\begin{aligned}
\left(\inv{v^2} \partial_{tt} - \partial_{xx}\right) \psi = 0
\end{aligned}
\end{equation}
%
A separation of variables treatment of this has been done in
\chapcite{PJwaveFourVector}, and some logical followup for that done in
\chapcite{PJemWave} in the context of Maxwell's equation for the vacuum field.

Here the Fourier transform will be used as a tool.

\section{One dimensional case}

Following the heat equation treatment in \chapcite{PJheatFourier}, we take Fourier transforms
of both parts of \eqnref{eqn:wave_fourier:waveTwoDim}.
%
\begin{equation}\label{eqn:waveFourier:23}
\begin{aligned}
\calF\left( \inv{v^2} \partial_{tt} \psi \right) = \calF\left( \partial_{xx} \psi \right)
\end{aligned}
\end{equation}
%
For the \(x\) derivatives we can integrate by parts twice
%
\begin{equation}\label{eqn:waveFourier:43}
\begin{aligned}
\calF\left( \partial_{xx} \psi \right)
&= \inv{\sqrt{2 \pi}} \IIinf \left( \partial_{xx} \psi \right) \exp\left( -i k x \right) dx \\
&= -\inv{\sqrt{2 \pi}} \IIinf \left( \partial_{x} \psi \right) \partial_x\left(\exp\left( -i k x \right) \right) dx \\
&= -\frac{-i k}{\sqrt{2 \pi}} \IIinf \left( \partial_{x} \psi \right) \exp\left( -i k x \right) dx \\
&= \frac{(-i k)^2}{\sqrt{2 \pi}} \IIinf \psi \exp\left( -i k x \right) dx \\
\end{aligned}
\end{equation}
%
Note that this integration by parts requires that \(\partial_x \psi = \psi = 0\) at \(\pm \infty\).  We are left with
%
\begin{equation}\label{eqn:waveFourier:63}
\begin{aligned}
\calF\left( \partial_{xx} \psi \right) &= -k^2 \hat{\psi}(k, t)
\end{aligned}
\end{equation}
%
Now, for the left hand side, for the Fourier transform of the time partials we can pull the derivative operation out of the
integral
%
\begin{equation}\label{eqn:waveFourier:83}
\begin{aligned}
\calF\left( \inv{v^2} \partial_{tt} \psi \right)
&= \inv{\sqrt{2 \pi}} \IIinf \left( \inv{v^2} \partial_{tt} \psi \right) \exp\left( -i k x \right) dx \\
&= \inv{v^2} \partial_{tt} \hat{\psi}(k,t) \\
\end{aligned}
\end{equation}
%
We are left with our harmonic oscillator differential equation for the transformed wave function
%
\begin{equation}\label{eqn:waveFourier:103}
\begin{aligned}
\inv{v^2} \partial_{tt} \hat{\psi}(k,t) &= -k^2 \hat{\psi}(k, t).
\end{aligned}
\end{equation}
%
Since we have a partial differential equation, for the integration constant we are free to pick any function of \(k\).  The solutions of this are therefore of the form
%
\begin{equation}\label{eqn:waveFourier:123}
\begin{aligned}
\hat{\psi}(k,t) &= A(k) \exp\left( \pm i v k t \right)
\end{aligned}
\end{equation}
%
Performing an inverse Fourier transform we now have the wave equation expressed in terms of this unknown (so far) frequency domain function \(A(k)\).  That is
%
\begin{equation}\label{eqn:wave_fourier:almostThere}
\begin{aligned}
{\psi}(x,t) &= \inv{\sqrt{2\pi}} \IIinf A(k) \exp\left( \pm i v k t + i k x \right) dk
\end{aligned}
\end{equation}
%
Now, suppose we fix the boundary value conditions by employing a known value of the wave function at \(t = 0\), say \(\psi(x,0) = \phi(x)\).  We then have
%
\begin{equation}\label{eqn:waveFourier:143}
\begin{aligned}
{\phi}(x) &= \inv{\sqrt{2\pi}} \IIinf A(k) \exp\left( i k x \right) dk
\end{aligned}
\end{equation}
%
From which we have \(A(k)\) in terms of \(\phi\) by inverse transform
%
\begin{equation}\label{eqn:wave_fourier:freqInitial}
\begin{aligned}
A(k) &= \inv{\sqrt{2\pi}} \IIinf \phi(x) \exp\left( -i k x \right) dx
\end{aligned}
\end{equation}
%
One could consider the problem fully solved at this point, but it can be carried further.  Let us substitute
\eqnref{eqn:wave_fourier:freqInitial} back into \eqnref{eqn:wave_fourier:almostThere}.  This is
%
\begin{equation}\label{eqn:waveFourier:163}
\begin{aligned}
{\psi}(x,t) &= \inv{\sqrt{2\pi}} \IIinf \left( \inv{\sqrt{2\pi}} \IIinf \phi(u) \exp\left( -i k u \right) du \right) \exp\left( \pm i v k t + i k x \right) dk
\end{aligned}
\end{equation}
%
With the Rigor police on holiday, exchange the order of integration
%
\begin{equation}\label{eqn:waveFourier:183}
\begin{aligned}
{\psi}(x,t)
&= \IIinf \phi(u) du \inv{{2\pi}} \IIinf \exp\left( -i k u \pm i v k t + i k x \right) dk \\
&= \IIinf \phi(u) du \inv{{2\pi}} \IIinf \exp\left( i k (x - u \pm v t ) \right) dk \\
\end{aligned}
\end{equation}
%
The principle value of this inner integral is
%
\begin{equation}\label{eqn:waveFourier:203}
\begin{aligned}
\PV \inv{{2\pi}} \IIinf \exp\left( i k (x - u \pm v t ) \right) dk
&= \lim_{R\rightarrow \infty} \inv{2\pi} \int_{-R}^R \exp\left( i k (x - u \pm v t ) \right) dk  \\
&= \lim_{R\rightarrow \infty} \frac{\sin\left( R (x - u \pm v t ) \right) }{ \pi (x - u \pm v t) } \\
\end{aligned}
\end{equation}
%
And here we make the usual identification with the delta function \(\delta( x - u \pm v t )\).  We are left with
%
\begin{equation}\label{eqn:waveFourier:223}
\begin{aligned}
{\psi}(x,t)
&= \IIinf \phi(u) \delta( x - u \pm v t ) du \\
&= \phi( x \pm v t )
\end{aligned}
\end{equation}
%
We find, amazingly enough, just by application of the Fourier transform, that the time evolution of the
wave function follows propagation of the initial wave packet down the x-axis in one of the two directions with velocity \(v\).

This is a statement well known to any first year student taking a vibrations and waves course, but it is nice to see
it follow from the straightforward application of transform techniques straight out of the Engineer's toolbox.

\section{Two dimensional case}

Next, using a two dimensional Fourier transform
%
\begin{equation}\label{eqn:waveFourier:243}
\begin{aligned}
\hat{f}(k,m) &= \inv{(\sqrt{2\pi})^2} \IIinf f(x,y) \exp\left( -i k x - i m y \right) dx dy \\
{f}(x,y) &= \inv{(\sqrt{2\pi})^2} \IIinf \hat{f}(k,m) \exp\left( i k x + i m y \right) dk dm,
\end{aligned}
\end{equation}
%
let us examine the two dimensional wave equation
%
\begin{equation}\label{eqn:waveFourier:263}
\begin{aligned}
\calF\left( \left(\inv{v^2} \partial_{tt} -\partial_{xx} -\partial_{yy} \right)\psi = 0 \right)
\end{aligned}
\end{equation}
%
Applying the same technique as above we have
%
\begin{equation}\label{eqn:waveFourier:283}
\begin{aligned}
\inv{v^2}\partial_{tt}\hat{\psi}(k,m,t) = \left((-i k)^2 + (-i m)^2\right) \hat{\psi}(k,m,t)
\end{aligned}
\end{equation}
%
With a solution
%
\begin{equation}\label{eqn:waveFourier:303}
\begin{aligned}
\hat{\psi}(k,m,t) = A(k,m) \exp\left( \pm i \sqrt{k^2 + m^2} v t \right).
\end{aligned}
\end{equation}
%
Inverse transforming we have our spatial domain function
%
\begin{equation}\label{eqn:waveFourier:323}
\begin{aligned}
{\psi}(x,y,t) = \inv{(\sqrt{2\pi})^2} \IIinf A(k,m) \exp\left( i k x + i m y \pm i \sqrt{k^2 + m^2} v t \right) dk dm
\end{aligned}
\end{equation}
%
Again introducing an initial value function \(\psi(x,y,0) = \phi(x,y)\) we have
%
\begin{equation}\label{eqn:waveFourier:343}
\begin{aligned}
A(k,m)
&= \hat{\phi}(k,m) \\
&= \inv{(\sqrt{2\pi})^2} \IIinf \phi(u,w) \exp\left( -i k u - i m w \right) du dw
\end{aligned}
\end{equation}
%
From which we can produce a final solution for the time evolution of an initial wave function, in terms of a Green's function
for the wave equation.
%
\begin{equation}\label{eqn:wave_fourier:greensSolution}
\begin{aligned}
{\psi}(x,y,t) &= \IIinf \phi(u,w) G( x-u, y-w, t) du dw \\
G(x,y,t) &= \inv{({2\pi})^2} \IIinf \exp\left( i k x + i m y \pm i \sqrt{k^2 + m^2} v t \right) dk dm
\end{aligned}
\end{equation}
%
Pretty cool even if it is incomplete.

\subsection{A (busted) attempt to reduce this Green's function to deltas}

Now, for this inner integral kernel in \eqnref{eqn:wave_fourier:greensSolution}, our Green's function, or fundamental solution for the wave equation,
we expect to have the action of a delta function.  If it weare not for that root term we could make that
identification easily since it could be factored into independent bits:
%
\begin{equation}\label{eqn:waveFourier:363}
\begin{aligned}
\inv{({2\pi})^2} &\IIinf \exp\left( i k (x-u) + i m (y-w) \right) dk dm  \\
&=
\left(\inv{{2\pi}} \IIinf \exp\left( i k (x-u) \right) dk \right)
\left(\inv{{2\pi}} \IIinf \exp\left( i m (y-w) \right) dm \right) \\
&\sim \delta( x - u)\delta( y - w)
\end{aligned}
\end{equation}
%
Having seen previously that functions of the form \(f(\kcap \cdot \Bx -v t)\) are general solutions to the wave equation in higher
dimensions suggests rewriting the integral kernel of the wave function in the following form
%
\begin{equation}\label{eqn:waveFourier:383}
\begin{aligned}
\inv{({2\pi})^2} &\IIinf \exp\left( i k (x-u) + i m (y-w) \pm i \sqrt{k^2 + m^2} v t \right) dk dm \\
&=
\inv{{2\pi}} \IIinf dk \exp\left( i k (x - u \pm v t) \right) \\
&\times \inv{{2\pi}} \IIinf dm \exp\left( i m (y - w \pm v t) \right) \\
&\times \exp\left( \pm i v t ( \sqrt{k^2 + m^2} -k - m ) \right) \\
\end{aligned}
\end{equation}
%
Now, the first two integrals have the form that we associate with one dimensional delta functions, and one can see that when either
\(k\) or \(m\) separately large (and positive) relative to the other than the third factor is approximately zero.  In a loose fashion one
can guesstimate that this combined integral has the following delta action
%
\begin{equation}\label{eqn:waveFourier:403}
\begin{aligned}
\inv{({2\pi})^2} &\IIinf \exp\left( i k (x-u) + i m (y-w) \pm i \sqrt{k^2 + m^2} v t \right) dk dm \\
&\sim
\delta( x -u \pm vt )
\delta( y -w \pm vt )
\end{aligned}
\end{equation}
%
If that is the case then our final solution becomes
%
\begin{equation}\label{eqn:waveFourier:423}
\begin{aligned}
\psi(x,y,t)
&= \IIinf \phi(u,w) \delta( x -u \pm vt ) \delta( y -w \pm vt ) du dw \\
&= \phi( x \pm vt, y \pm vt ) \\
\end{aligned}
\end{equation}
%
This is a bit different seeming than the unit wave number dot product form, but lets see if it works.  We want to expand
%
\begin{equation}\label{eqn:waveFourier:443}
\begin{aligned}
\left(\inv{v^2} \partial_{tt} -\partial_{xx} -\partial_{yy} \right)\psi
%&=
%\inv{v^2} (\pm v)^2 \partial_{xx} \phi
%+\inv{v^2} (\pm v)^2 \partial_{yy} \phi
%-\partial_{xx} \phi
%-\partial_{yy} \phi
\end{aligned}
\end{equation}
%
Let us start with the time partials
%
\begin{equation}\label{eqn:waveFourier:463}
\begin{aligned}
\partial_{tt} \phi( x \pm vt, y \pm vt )
&=
\partial_{t} \partial_t \phi( x \pm vt, y \pm vt ) \\
&= \partial_{t} ( \partial_x \phi (\pm v) +\partial_y \phi (\pm v) ) \\
&= (\pm v)( \partial_{x} \partial_t \phi  + \partial_y \partial_t \phi ) \\
&= (\pm v)^2( \partial_{x} (\partial_x \phi + \partial_y \phi)  + \partial_y (\partial_x \phi + \partial_y \phi)) \\
&= (\pm v)^2( \partial_{xx} \phi +\partial_{yy} \phi +\partial_{yx} \phi +\partial_{xy} \phi )
\end{aligned}
\end{equation}
%
So application of this test solution to the original wave equation is not zero, since these cross partials are not necessarily zero
%
\begin{equation}\label{eqn:waveFourier:483}
\begin{aligned}
\left(\inv{v^2} \partial_{tt} -\partial_{xx} -\partial_{yy} \right)\psi
&= \partial_{yx} \phi + \partial_{xy} \phi
\end{aligned}
\end{equation}
%
This indicates that an incorrect guess was made about the delta function action of the integral kernel found
via this Fourier transform technique.  The remainder of that root term does not in fact cancel out, which appeared may
occur, but was just too convenient.  Oh well.

\section{Three dimensional wave function}

It is pretty clear that a three dimensional Fourier transform
%
\begin{equation}\label{eqn:waveFourier:503}
\begin{aligned}
\hat{f}(k,m,n) &= \inv{(\sqrt{2\pi})^3} \IIinf f(x,y,z) \exp\left( -i k x - i m y - i n z \right) dx dy dz \\
{f}(x,y,z) &= \inv{(\sqrt{2\pi})^3} \IIinf \hat{f}(k,m,n) \exp\left( i k x + i m y + i n z \right) dk dm dn,
\end{aligned}
\end{equation}
%
applied to a three dimensional wave equation
%
\begin{equation}\label{eqn:waveFourier:523}
\begin{aligned}
\calF\left( \left(\inv{v^2} \partial_{tt} -\partial_{xx} -\partial_{yy} -\partial_{zz} \right)\psi = 0 \right)
\end{aligned}
\end{equation}
%
will lead to the similar results, but since this result did not work, it is not worth perusing this more general case just yet.

Despite the failure in the hopeful attempt to reduce the Green's function to a product of delta functions, one still gets a general solution
from this approach for the three dimensional case.
%
\begin{equation}\label{eqn:wave_fourier:greensSolution3d}
\begin{aligned}
{\psi}(x,y,z,t) &= \IIinf \phi(u,w,r) G( x-u, y-w, z-r, t) du dw dr \\
G(x,y,z,t) &= \inv{({2\pi})^3} \IIinf \exp\left( i k x + i m y + i n z \pm i \sqrt{k^2 + m^2 + n^2} v t \right) dk dm dn
\end{aligned}
\end{equation}
%
So, utilizing this or reducing it to the familiar \(f(\kcap \cdot \Bx \pm vt)\) solutions becomes the next step.  Intuition says that we
need to pick a different inner product to get that solution.  For the two dimensional case that likely has to be an inner product
with a circular contour, and for the three dimensional case a spherical surface inner product of some sort.

Now, also interestingly, one can see hints here of the non-vacuum Maxwell retarded time potential wave solutions.
This inspires an attempt to try to tackle that too.

   %
% Copyright � 2012 Peeter Joot.  All Rights Reserved.
% Licenced as described in the file LICENSE under the root directory of this GIT repository.
%
%
%
\mychapter{Fourier transform solutions to Maxwell's equation}
\label{chap:PJfourierMaxwellSecondOrder}
\index{Maxwell's equation!Fourier transform}
%\date{Jan 29, 2009.  fourierMaxwell.tex}

\section{Motivation}

In \chapcite{PJwaveFourier} a Green's function solution to the homogeneous wave equation
%
\begin{equation}\label{eqn:fourier_maxwell:wave}
\begin{aligned}
\left(\inv{v^2} \partial_{tt} -\partial_{xx} -\partial_{yy} -\partial_{zz} \right)\psi = 0
\end{aligned}
\end{equation}
%
was found to be
%
\begin{equation}\label{eqn:fourier_maxwell:greensSolution3d}
\begin{aligned}
{\psi}(x,y,z,t) &= \IIinf \phi(u,w,r) G( x-u, y-w, z-r, t) du d\tau dr \\
G(x,y,z,t) &= \inv{({2\pi})^3} \IIinf \exp\left( i k x + i m y + i n z \pm i \sqrt{k^2 + m^2 + n^2} v t \right) dk dm dn
\end{aligned}
\end{equation}
%
The aim of this set of notes is to explore the same ideas to the forced wave equations for the four vector potentials of the Lorentz gauge Maxwell equation.

Such solutions can be used to find the Faraday bivector or its associated tensor components.

Note that the specific form of the Fourier transform used in these notes continues to be
%
\begin{equation}\label{eqn:fourierMaxwell:23}
\begin{aligned}
%\hat{f}(\Bk) &= \inv{(\sqrt{2\pi})^n} \IIinf f(\Bx) \exp\left( -i k_j x^j \right) d^n x \\
%{f}(\Bx) &= \inv{(\sqrt{2\pi})^n} \IIinf \hat{f}(\Bk) \exp\left( i k_j x^j \right) d^n k
\hat{f}(\Bk) &= \inv{(\sqrt{2\pi})^n} \IIinf f(\Bx) \exp\left( -i \Bk \cdot \Bx \right) d^n x \\
{f}(\Bx) &= \inv{(\sqrt{2\pi})^n} \IIinf \hat{f}(\Bk) \exp\left( i \Bk \cdot \Bx \right) d^n k
\end{aligned}
\end{equation}
%
\section{Forced wave equation}
\index{wave equation!forced}
\subsection{One dimensional case}
A good starting point is the reduced complexity one dimensional forced wave equation
\begin{equation}\label{eqn:fourierMaxwell:43}
\left( \inv{v^2} \partial_{tt} -\partial_{xx} \right)\psi = g
\end{equation}
%
Fourier transforming to the wave number domain, with application of integration by parts twice (each toggling the sign of the spatial derivative term) we have
\begin{equation}\label{eqn:fourierMaxwell:63}
\inv{v^2}\hat{\psi}_{tt} - (-i k)^2 \hat{\psi} = \hat{g}.
\end{equation}
%
This leaves us with a linear differential equation of the following form to solve
\begin{equation}\label{eqn:fourier_maxwell:waveNumEquationToSolve}
f'' + \alpha^2 f = h.
\end{equation}
%
Out of line solution of this can be found below in \eqnref{eqn:fourier_maxwell:solutionToWaveNumberDomainEquation}, where we have \(f = \hat{\psi}\), \(\alpha = k v\), and \(h = \hat{g} v^2\).  Our solution for the wave function in the wave number domain is now completely
specified
\begin{equation}\label{eqn:fourierMaxwell:83}
\hat{\psi}(k, t) = \Abs{\frac{v}{k}} \int_{u=t_0(k)}^t \hat{g}(u) \sin( \Abs{k v} (t-u) ) du.
\end{equation}
%
Because of the partial differentiation we have the flexibility to make the initial time a function of the wave number \(k\), but it is probably more natural to just set \(t_0 = -\infty\).  Also let us explicitly pick \(v > 0\) so that absolutes are only required on the factors of \(k\)
\begin{equation}\label{eqn:fourierMaxwell:103}
\begin{aligned}
\hat{\psi}(k, t) = \frac{v}{\Abs{k}} \int_{u = -\infty}^t \hat{g}(k, u) \sin( \Abs{k} v (t-u) ) du
\end{aligned}
\end{equation}
%
But seeing the integral in this form suggests a change of variables \(\tau = t-u\), which gives us our final wave function in the wave number domain with all the time
dependency removed from the integration limits
%
\begin{equation}\label{eqn:fourierMaxwell:123}
\begin{aligned}
\hat{\psi}(k, t) = \frac{v}{\Abs{k}} \int_{\tau = 0}^\infty \hat{g}(k, t-\tau) \sin( \Abs{k} v \tau ) d\tau
\end{aligned}
\end{equation}
%
With this our wave function is
%
\begin{equation}\label{eqn:fourierMaxwell:143}
\begin{aligned}
{\psi}(x, t)
&=
\inv{\sqrt{2\pi}} \IIinf
\left(
\frac{v}{k} \int_{\tau = 0}^\infty \hat{g}(k, t-\tau) \sin( \Abs{k} v \tau ) d\tau
\right) \exp( i k x ) dk \\
\end{aligned}
\end{equation}
%
%
But we also have
%
\begin{equation}\label{eqn:fourierMaxwell:163}
\begin{aligned}
\hat{g}(k,t) &= \inv{\sqrt{2\pi}} \int_{-\infty}^\infty {g}(x, t) \exp( -i k x ) dx
\end{aligned}
\end{equation}
%
Reassembling we have
%
\begin{equation}\label{eqn:fourierMaxwell:183}
\begin{aligned}
{\psi}(x, t)
&=
\int_{k = -\infty}^\infty
\int_{\tau = 0}^\infty
\int_{y=-\infty}^\infty
\frac{v}{ 2 \pi \Abs{k}}
{g}(y, t-\tau)
\sin( \Abs{k} v \tau )
\exp( i k (x-y) )
dy
d\tau
dk
\end{aligned}
\end{equation}
%
Rearranging a bit, and noting that \(\sinc(\Abs{k}x) = \sinc(kx)\) we have
%
\begin{equation}\label{eqn:fourier_maxwell:oneDimResult}
\begin{aligned}
{\psi}(x, t)
&=
\int_{x'=-\infty}^\infty
\int_{t' = 0}^\infty {g}(x-x', t-t') G(x', t') dx' dt' \\
G(x, t) &=
\int_{k = -\infty}^\infty
\frac{v}{2\pi {k}}
\sin( {k} v t )
\exp( i k x )
dk
\end{aligned}
\end{equation}
%
We see that our charge density summed over all space contributes to the wave function, but it is the charge density at that spatial location as it existed at a specific previous time.

The Green's function that we convolve with in \eqnref{eqn:fourier_maxwell:oneDimResult} is a rather complex looking function.  As seen later in \chapcite{PJpoisson} it was possible to evaluate a 3D variant of such an integral in ad-hoc methods to produce a form in terms of retarded time and advanced time delta functions.  A similar reduction, also in \chapcite{PJpoisson}, of the Green's function above yields a unit step function identification
%
\begin{equation}\label{eqn:fourierMaxwell:203}
\begin{aligned}
G(x, t) &= \frac{v}{2} \left(\theta(x + v t )  - \theta(x - v t) \right)
\end{aligned}
\end{equation}
%
(This has to be verified more closely to see if it works).

\subsection{Three dimensional case}

Now, lets move on to the 3D case that is of particular interest for electrodynamics.  Our wave equation is now of the form
%
\begin{equation}\label{eqn:fourierMaxwell:223}
\begin{aligned}
\left( \inv{v^2} \PDSq{t}{} -\sum_j \PDSq{x^j}{} \right)\psi = g
\end{aligned}
\end{equation}
%
and our Fourier transformation produces almost the same result, but we have a wave number contribution from each of the three dimensions
%
\begin{equation}\label{eqn:fourierMaxwell:243}
\begin{aligned}
\inv{v^2}\hat{\psi}_{tt} + \Bk^2 \hat{\psi} = \hat{g}
\end{aligned}
\end{equation}
%
Our wave number domain solution is therefore
\begin{equation}\label{eqn:fourier_maxwell:waveNumDomainSolution}
\begin{aligned}
\hat{\psi}(\Bk, t) = \frac{v}{\Abs{\Bk}} \int_{\tau = 0}^\infty \hat{g}(\Bk, t-\tau) \sin( \Abs{\Bk} v \tau ) d\tau
\end{aligned}
\end{equation}
%
But our wave number domain charge density is
%
\begin{equation}\label{eqn:fourierMaxwell:263}
\begin{aligned}
\hat{g}(\Bk, t) &= \inv{(\sqrt{2\pi})^3} \IIinf g(\Bx, t) \exp\left( -i \Bk \cdot \Bx \right) d^3 x \\
\end{aligned}
\end{equation}
%
Our wave number domain result in terms of the charge density is therefore
%
%\hat{g}(\Bk, t-\tau) &= \inv{(\sqrt{2\pi})^3} \IIinf g(\Br, t-\tau) \exp\left( -i \Bk \cdot \Br \right) d^3 r \\
\begin{equation}\label{eqn:fourierMaxwell:283}
\begin{aligned}
\hat{\psi}(\Bk, t) =
\frac{v}{\Abs{\Bk}} \int_{\tau = 0}^\infty
%\hat{g}(\Bk, t-\tau) &=
\left(
\inv{(\sqrt{2\pi})^3} \IIinf g(\Br, t-\tau) \exp\left( -i \Bk \cdot \Br \right) d^3 r
\right)
\sin( \Abs{\Bk} v \tau ) d\tau
\end{aligned}
\end{equation}
%
And finally inverse transforming back to the spatial domain we have a complete solution for the inhomogeneous wave equation in terms of the spatial and temporal charge density distribution
%
\begin{equation}\label{eqn:fourierMaxwell:303}
\begin{aligned}
{\psi}(\Bx, t)
%&= \IIinf \int_{t' = 0}^\infty g(\Br, t-t') G(\Bx -\Br) d^3 r dt' \\
% x'_j = x_j - r_j
% r_j = x_j - x'_j
% dr_j = -dx'_j
% III dr_1 dr_2 dr_3 = (-1)^3 III d^3 x'
&= \IIinf \int_{t' = 0}^\infty g(\Bx -\Bx', t-t') G(\Bx', t') d^3 x' dt' \\
G(\Bx, t)
&= \IIinf
\frac{v}{(2\pi)^3 \Abs{\Bk}}
\sin( \Abs{\Bk} v t )
\exp\left( i \Bk \cdot \Bx \right)
d^3 k
\end{aligned}
\end{equation}
%
For computational purposes we are probably much better off using
\eqnref{eqn:fourier_maxwell:waveNumDomainSolution}, however,
from an abstract point of form this expression is much prettier.

One can also see the elements of the traditional retarded time expressions for the potential hiding in there.  See
\chapcite{PJpoisson} for an evaluation of this integral (in an ad-hoc
non-rigorous fashion) eventually producing the retarded time solution.

\subsubsection{Tweak this a bit to put into proper Green's function form}

Now, it makes sense to redefine \(G(\Bx,t)\) above so that we can integrate uniformly over all space and time.  To do so we can add a unit step function into the definition, so that \(G(\Bx,t<0) = 0\).  Additionally, if we express this convolution it is slightly tidier (and consistent with the normal Green's function notation) to put the parameter differences in the kernel term.  Such a change of variables will alter the sign of the integral limits by a factor of \((-1)^4\), but we also have a \((-1)^4\) term from the differentials.  After making these final adjustments we have a final variation of our integral solution

%
% y_j = x_j - x'_j
% dy_j = x_j - dx'_j
% t' = t - \tau
% dt' = t - d\tau
%
%&= \int_x'1 \int_x'2 \int_x'3 \int_tau g(\Bx -\Bx', t-\tau) G(\Bx', \tau) d^3 x' d\tau \\
%
% =>
%
%&= (-1)^4 \int_y1 \int_y2 \int_y3 \int_t' g(\By, t') G(\Bx - \By, t - t') (-1)^4 d^3 y dt' \\
%
%
\begin{equation}\label{eqn:fourierMaxwell:323}
\begin{aligned}
{\psi}(\Bx, t)
&= \IIinf g(\Bx', t') G(\Bx - \Bx', t - t') d^3 x' dt' \\
G(\Bx, t)
&= \theta(t) \IIinf
\frac{v}{(2\pi)^3 \Abs{\Bk}}
\sin( \Abs{\Bk} v t )
\exp\left( i \Bk \cdot \Bx \right)
d^3 k
\end{aligned}
\end{equation}
%
Now our inhomogeneous solution is expressed nicely as the convolution of our current density over all space and time with an integral kernel.  That integral kernel is precisely the Green's function for this forced wave equation.

This solution comes with a large number of assumptions.  Along the way we have the assumption that both our wave function and the charge density was Fourier transformable, and that the wave number domain products were inverse transformable.  We also had an assumption that the wave function is sufficiently small at the limits of integration that the intermediate contributions from the integration by parts vanished, and finally the big assumption that we were perfectly free to interchange integration order in an extremely ad-hoc and non-rigorous fashion!

\section{Maxwell equation solution}

Having now found Green's function form for the forced wave equation, we can now move to Maxwell's equation
%
\begin{equation}\label{eqn:fourierMaxwell:343}
\begin{aligned}
\grad F = J/\epsilon_0 c
\end{aligned}
\end{equation}
%
In terms of potentials we have \(F = \grad \wedge A\), and may also impose the Lorentz gauge \(\grad \cdot A = 0\), to give us our four charge/current forced wave equations
%
\begin{equation}\label{eqn:fourierMaxwell:363}
\begin{aligned}
\grad^2 A = J/\epsilon_0 c
\end{aligned}
\end{equation}
%
As scalar equations these are
%
\begin{equation}\label{eqn:fourierMaxwell:383}
\begin{aligned}
\left( \inv{c^2} \PDSq{t}{} -\sum_j \PDSq{x^j}{} \right) A^\mu = \frac{J^\mu}{\epsilon_0 c}
\end{aligned}
\end{equation}
%
So, from above, also writing \(x^0 = ct\), we have
%
\begin{equation}\label{eqn:fourier_maxwell:fourVectorPotentials}
\begin{aligned}
{A^\mu}(x)
&= \inv{\epsilon_0 c} \int J^\mu(x') G(x - x') d^4 x' \\
G(x)
&= \theta(x^0) \int
\frac{1}{(2\pi)^3 \Abs{\Bk}}
\sin( \Abs{\Bk} x^0 )
\exp\left( i \Bk \cdot \Bx \right)
d^3 k
\end{aligned}
\end{equation}
%
\subsection{Four vector form for the Green's function}
\index{Green's function}

Can we put the sine and exponential product in a more pleasing form?  It would be nice to merge the \(\Bx\) and \(ct\) terms into a
single four vector form.  One possibility is merging the two
%
\begin{equation}\label{eqn:fourierMaxwell:403}
\begin{aligned}
\sin( \Abs{\Bk} x^0 ) &\exp\left( i \Bk \cdot \Bx \right) \\
&=
\inv{2i} \left(
\exp\left( i \left( \Bk \cdot \Bx +\Abs{\Bk} x^0 \right) \right)
-\exp\left( i \left( \Bk \cdot \Bx -\Abs{\Bk} x^0 \right) \right)
\right) \\
&=
\inv{2i} \left(
\exp\left( i \Abs{\Bk} \left( \kcap \cdot \Bx + x^0 \right) \right)
-\exp\left( i \Abs{\Bk} \left( \kcap \cdot \Bx - x^0 \right) \right)
\right)
\end{aligned}
\end{equation}
%
Here we have a sort of sine like conjugation in the two exponentials.  Can we tidy this up?  Let us write the unit wave number
vector in terms of direction cosines
%
\begin{equation}\label{eqn:fourierMaxwell:423}
\begin{aligned}
\kcap
&= \sum_m \sigma_m \alpha_m \\
&= \sum_m \gamma_m \gamma_0 \alpha_m \\
\end{aligned}
\end{equation}
%
Allowing us to write
%
\begin{equation}\label{eqn:fourierMaxwell:443}
\begin{aligned}
\sum_m \gamma^m \alpha_m &= -\kcap \gamma_0
\end{aligned}
\end{equation}
%
This gives us
\begin{equation}\label{eqn:fourierMaxwell:463}
\begin{aligned}
\kcap \cdot \Bx + x^0
&= \alpha_m x^m + x^0 \\
&= (\alpha_m \gamma^m) \cdot (\gamma_j x^j) + \gamma^0 \cdot \gamma_0 x^0 \\
&= (-\kcap \gamma_0 + \gamma_0) \cdot \gamma_\mu x^\mu \\
&= (-\kcap \gamma_0 + \gamma_0) \cdot x
\end{aligned}
\end{equation}
%
Similarly we have
%
\begin{equation}\label{eqn:fourierMaxwell:483}
\begin{aligned}
\kcap \cdot \Bx - x^0  &= (-\kcap \gamma_0 - \gamma_0) \cdot x
\end{aligned}
\end{equation}
%
and can now put \(G\) in explicit four vector form
%
\begin{equation}\label{eqn:fourierMaxwell:503}
\begin{aligned}
G(x)
&=
%\theta(x \cdot \gamma_0) \int
%\frac{1}{(2\pi)^3 2 i \Abs{\Bk}}
%\left(
%\exp\left( i ((\Abs{\Bk} -\Bk)\gamma_0) \cdot x \right)
%-\exp\left( -i ((\Abs{\Bk} +\Bk)\gamma_0) \cdot x \right)
%\right)
%d^3 k
\frac{\theta(x \cdot \gamma_0)}{
(2\pi)^3 2 i
} \int
\left(
\exp\left( i ((\Abs{\Bk} -\Bk)\gamma_0) \cdot x \right)
-\exp\left( -i ((\Abs{\Bk} +\Bk)\gamma_0) \cdot x \right)
\right)
\frac{d^3 k}{ \Abs{\Bk} }
\end{aligned}
\end{equation}
%
Hmm, is that really any better?  Intuition says that this whole thing
can be written as sine with some sort of geometric product conjugate
terms.

I get as far as writing
%
\begin{equation}\label{eqn:fourierMaxwell:523}
\begin{aligned}
i ( \Bk \cdot \Bx \pm \Abs{\Bk} x^0 )
&=
(i \gamma_0) \wedge ( \Bk \pm \Abs{\Bk} ) \cdot x
\end{aligned}
\end{equation}
%
But that does not quite have the conjugate form I was looking for (or does it)?  Have to go back and look at Hestenes's multivector conjugation operation.  Think it had something to do with reversion, but do not recall.

Failing that tidy up the following
%
\begin{equation}\label{eqn:fourierMaxwell:543}
\begin{aligned}
G(x)
&=
\frac{\theta(x \cdot \gamma_0)}{ (2\pi)^3 }
\int
\sin( \Abs{\Bk} x \cdot \gamma_0 )
\exp\left( -i (\Bk \gamma_0) \cdot x \right)
\frac{d^3 k}{ \Abs{\Bk} }
\end{aligned}
\end{equation}
%
is probably about as good as it gets for now.  Note the interesting feature
that we end up essentially integrating over a unit ball in our wave number
space.  This suggests the possibility of simplification using the
divergence theorem.

\subsection{Faraday tensor}
\index{Faraday tensor}

Attempting to find a tidy four vector form for the four vector potentials was in preparation for taking derivatives.
Specifically, applied to \eqnref{eqn:fourier_maxwell:fourVectorPotentials} we have
%
\begin{equation}\label{eqn:fourierMaxwell:563}
\begin{aligned}
F^{\mu\nu} = \partial^\mu A^\nu - \partial^\nu A^\mu
\end{aligned}
\end{equation}
%
subject to the Lorentz gauge constraint
\begin{equation}\label{eqn:fourierMaxwell:583}
\begin{aligned}
0 = \partial_\mu A^\mu
\end{aligned}
\end{equation}
%
If we switch the convolution indices for our potentials
%
\begin{equation}\label{eqn:fourierMaxwell:603}
\begin{aligned}
{A^\mu}(\Bx, t) &= \inv{\epsilon_0 c} \int J^\mu(x - x') G(x') d^4 x' \\
\end{aligned}
\end{equation}
%
Then the Lorentz gauge condition, after differentiation under the integral sign, is
%
\begin{equation}\label{eqn:fourierMaxwell:623}
\begin{aligned}
0 = \partial_\mu A^\mu &= \inv{\epsilon_0 c} \int \left(\partial_\mu J^\mu(x - x') \right) G(x') d^4 x' \\
\end{aligned}
\end{equation}
%
So we see that the Lorentz gauge seems to actually imply the continuity equation
%
\begin{equation}\label{eqn:fourierMaxwell:643}
\begin{aligned}
\partial_\mu J^\mu(x) = 0
\end{aligned}
\end{equation}
%
Similarly, it appears that we can write our tensor components in terms of current density derivatives
%
\begin{equation}\label{eqn:fourierMaxwell:663}
\begin{aligned}
F^{\mu\nu}
%&= \partial^\mu A^\nu - \partial^\nu A^\mu \\
&= \inv{\epsilon_0 c} \int \left(\partial^\mu J^\nu(x - x') - \partial^\nu J^\mu(x - x') \right) G(x') d^4 x'
\end{aligned}
\end{equation}
%
Logically, I suppose that one can consider the entire problem solved here, pending the completion of this calculus exercise.

In terms of tidiness, it would be nicer seeming use the original convolution, and take derivative differences of the Green's function.  However, how to do this is not clear to me since this function has no defined derivative at the \(t=0\) points due to the unit step.

\section{Appendix.  Mechanical details}

\subsection{Solving the wave number domain differential equation}

We wish to solve equation the inhomogeneous \eqnref{eqn:fourier_maxwell:waveNumEquationToSolve}.  Writing this in terms of a linear operator equation this is
%
\begin{equation}\label{eqn:fourierMaxwell:683}
\begin{aligned}
L(y) &= y'' + \alpha^2 y \\
L(y) &= h
\end{aligned}
\end{equation}
%
The solutions of this equation will be formed from linear combinations of the homogeneous problem plus a specific solution of the inhomogeneous problem

By inspection the homogeneous problem has solutions in \(\Span \{ e^{ i \alpha x }, e^{ -i \alpha x }\}\).
We can find a solution to the inhomogeneous problem using the variation of parameters method, assuming a solution of the form
%
\begin{equation}\label{eqn:fourierMaxwell:703}
\begin{aligned}
y  = u e^{ i \alpha x } + v e^{ -i \alpha x }
\end{aligned}
\end{equation}
%
Taking derivatives we have
\begin{equation}\label{eqn:fourierMaxwell:723}
\begin{aligned}
y' = u' e^{ i \alpha x } + v' e^{ -i \alpha x } + i \alpha (u e^{ i \alpha x } - v e^{ -i \alpha x })
\end{aligned}
\end{equation}
%
The trick to solving this is to employ the freedom to set the \(u'\), and \(v'\) terms above to zero
%
\begin{equation}\label{eqn:fourier_maxwell:firstConstraint}
\begin{aligned}
u' e^{ i \alpha x } + v' e^{ -i \alpha x } = 0
\end{aligned}
\end{equation}
%
Given this choice we then have
\begin{equation}\label{eqn:fourierMaxwell:743}
\begin{aligned}
y' &= i \alpha (u e^{ i \alpha x } - v e^{ -i \alpha x }) \\
y'' &=
(i \alpha)^2 (u e^{ i \alpha x } + v e^{ -i \alpha x })
i \alpha (u' e^{ i \alpha x } - v' e^{ -i \alpha x })
\end{aligned}
\end{equation}
%
So we have
\begin{equation}\label{eqn:fourierMaxwell:763}
\begin{aligned}
L(y)
&=
(i \alpha)^2 (u e^{ i \alpha x } + v e^{ -i \alpha x })  \\
&+i \alpha (u' e^{ i \alpha x } - v' e^{ -i \alpha x })
+ (\alpha)^2 (u e^{ i \alpha x } + v e^{ -i \alpha x })  \\
&=
i \alpha (u' e^{ i \alpha x } - v' e^{ -i \alpha x })
\end{aligned}
\end{equation}
%
With this and \eqnref{eqn:fourier_maxwell:firstConstraint} we have a set of simultaneous first order linear differential equations to solve
%
\begin{equation}\label{eqn:fourierMaxwell:783}
\begin{aligned}
\begin{bmatrix}
u' \\
v' \\
\end{bmatrix}
&=
{\begin{bmatrix}
 e^{ i \alpha x } &- e^{ -i \alpha x } \\
 e^{ i \alpha x } &  e^{ -i \alpha x } \\
\end{bmatrix}}^{-1}
\begin{bmatrix}
{h}/{i \alpha} \\
0 \\
\end{bmatrix} \\
&=
\inv{2}
{\begin{bmatrix}
 e^{ -i \alpha x } & e^{ -i \alpha x } \\
 -e^{ i \alpha x } &  e^{ i \alpha x } \\
\end{bmatrix}}
\begin{bmatrix}
{h}/{i \alpha} \\
0 \\
\end{bmatrix} \\
&=
\frac{h}{2 i \alpha}
{\begin{bmatrix}
 e^{ -i \alpha x } \\
 -e^{ i \alpha x } \\
\end{bmatrix}}
\end{aligned}
\end{equation}
%
Substituting back into the assumed solution we have
\begin{equation}\label{eqn:fourierMaxwell:803}
\begin{aligned}
y
&= \frac{1}{2 i \alpha} \left(
  e^{ i \alpha x } \int h e^{ -i \alpha x }
- e^{ -i \alpha x } \int h e^{ i \alpha x }
\right) \\
&= \frac{1}{2 i \alpha} \int_{u=x_0}^x h(u) \left( e^{ -i \alpha (u-x) } -e^{ i \alpha (u-x) } \right) du \\
\end{aligned}
\end{equation}
%
So our solution appears to be
%
\begin{equation}\label{eqn:fourier_maxwell:solutionToWaveNumberDomainEquation}
\begin{aligned}
y &= \frac{1}{\alpha} \int_{u=x_0}^x h(u) \sin( \alpha(x-u) ) du
\end{aligned}
\end{equation}
%
A check to see if this is correct is in order to verify this.  Differentiating using \eqnref{eqn:fourier_maxwell:diffInt} we have
%
\begin{equation}\label{eqn:fourierMaxwell:823}
\begin{aligned}
y'
&=
{\left.
\frac{1}{\alpha}
h(u) \sin( \alpha(x-u) ) \right\vert}_{u=x}
+\frac{1}{\alpha} \int_{u=x_0}^x \PD{x}{} h(u) \sin( \alpha(x-u) ) du \\
&= \int_{u=x_0}^x h(u) \cos( \alpha(x-u) ) du \\
\end{aligned}
\end{equation}
%
and for the second derivative we have
%
\begin{equation}\label{eqn:fourierMaxwell:843}
\begin{aligned}
y''
&=
{\left. h(u) \cos( \alpha(x-u) ) \right\vert}_{u=x}
- \alpha \int_{u=x_0}^x h(u) \sin( \alpha(x-u) ) du \\
&= h(x) - \alpha^2 y(x)
\end{aligned}
\end{equation}
%
Excellent, we have \(y'' + \alpha^2 y = h\) as desired.

\subsection{Differentiation under the integral sign}

Given an function that is both a function of the integral limits and the integrals kernel
%
\begin{equation}\label{eqn:fourierMaxwell:863}
\begin{aligned}
f(x) = \int_{u = a(x)}^{b(x)} G(x,u) du,
\end{aligned}
\end{equation}
%
lets recall how to differentiate the beastie.  First let \(G(x,u) = \PDi{u}{F(x,u)}\) so we have
%
\begin{equation}\label{eqn:fourierMaxwell:883}
\begin{aligned}
f(x) = F(x,b(x)) - F(x,a(x))
\end{aligned}
\end{equation}
%
and our derivative is
\begin{equation}\label{eqn:fourierMaxwell:903}
\begin{aligned}
f'(x)
&=
\PD{x}{F}(x,b(x))
\PD{u}{F}(x,b(x)) b'
-\PD{x}{F}(x,a(x))
-\PD{u}{F}(x,a(x)) a' \\
&=
G(x,b(x)) b'
-G(x,a(x)) a'
+\PD{x}{F}(x,b(x))
-\PD{x}{F}(x,a(x))
\\
\end{aligned}
\end{equation}
%
Now, we want \(\PDi{x}{F}\) in terms of \(G\), and to get there, assuming sufficient continuity, we have from the definition
%
\begin{equation}\label{eqn:fourierMaxwell:923}
\begin{aligned}
\PD{x}{} G(x,u)
&= \PD{x}{} \PD{u}{F(x,u)} \\
&= \PD{u}{} \PD{x}{F(x,u)} \\
\end{aligned}
\end{equation}
%
Integrating both sides with respect to \(u\) we have
%
\begin{equation}\label{eqn:fourierMaxwell:943}
\begin{aligned}
\int \PD{x}{G} du
&= \int \PD{u}{} \PD{x}{F(x,u)} du \\
%&= \int \PD{u}{} \left( \PD{x}{F(x,u)} + A(x) \right) du \\
&= \PD{x}{F(x,u)}
\end{aligned}
\end{equation}
%
This allows us to write
%
\begin{equation}\label{eqn:fourierMaxwell:963}
\begin{aligned}
\PD{x}{F}(x,b(x))
-\PD{x}{F}(x,a(x))
&=
\int_{a}^b \PD{x}{G}(x,u) du
\end{aligned}
\end{equation}
%
and finally
%
\begin{equation}\label{eqn:fourier_maxwell:diffInt}
\begin{aligned}
\frac{d}{dx} \int_{u = a(x)}^{b(x)} G(x,u) du
&=
G(x,b(x)) b'
-G(x,a(x)) a'
+ \int_{a(x)}^{b(x)} \PD{x}{G}(x,u) du
\end{aligned}
\end{equation}
%
\subsubsection{Argument logic error above to understand}

Is the following not also true
%
\begin{equation}\label{eqn:fourierMaxwell:983}
\begin{aligned}
\int \PD{x}{G} du
&= \int \PD{u}{} \PD{x}{F(x,u)} du \\
&= \int \PD{u}{} \left( \PD{x}{F(x,u)} + A(x) \right) du \\
&= \PD{x}{F(x,u)} + A(x)u + B
\end{aligned}
\end{equation}
%
In this case we have
\begin{equation}\label{eqn:fourierMaxwell:1003}
\begin{aligned}
\PD{x}{F}(x,b(x)) -\PD{x}{F}(x,a(x)) &= \int_{a}^b \PD{x}{G}(x,u) du - A(x) ( b(x) - a(x))
\end{aligned}
\end{equation}
%
How to reconcile this with the answer I expect (and having gotten it, I believe matches my recollection)?

   %
% Copyright � 2012 Peeter Joot.  All Rights Reserved.
% Licenced as described in the file LICENSE under the root directory of this GIT repository.
%

%
%
\mychapter{First order Fourier transform solution of Maxwell's equation}
\label{chap:PJfirstOrderMaxwell}
\index{Maxwell's equation!Fourier transform}
%\date{Jan 31, 2009.  firstorderFourierMaxwell.tex}

\section{Motivation}

In \chapcite{PJfourierMaxwellSecondOrder} solutions of Maxwell's equation
via Fourier transformation of the four potential forced wave equations were
explored.

Here a first order solution is attempted, by directly Fourier transforming
the Maxwell's equation in bivector form.

\section{Setup}

Again using a 3D spatial Fourier transform, we want to put Maxwell's equation into an explicit time dependent form, and can do so by
premultiplying by our observer's time basis vector \(\gamma_0\)

\begin{equation}\label{eqn:firstorderFourierMaxwell:20}
\begin{aligned}
\gamma_0 \grad F &= \gamma_0 \frac{J }{\epsilon_0 c}
\end{aligned}
\end{equation}

On the left hand side we have
\begin{equation}\label{eqn:firstorderFourierMaxwell:40}
\begin{aligned}
\gamma_0 \grad
&= \gamma_0 \left( \gamma^0 \partial_0 + \gamma^k \partial_k \right) \\
&= \partial_0 - \gamma^k \gamma_0 \partial_k \\
&= \partial_0 + \sigma^k \partial_k \\
&= \partial_0 + \spacegrad \\
\end{aligned}
\end{equation}

and on the right hand side we have
\begin{equation}\label{eqn:firstorderFourierMaxwell:60}
\begin{aligned}
\gamma_0 \frac{J }{\epsilon_0 c}
&= \gamma_0 \frac{c \rho \gamma_0 + J^k \gamma_k }{\epsilon_0 c} \\
&= \frac{c \rho - J^k \sigma_k }{\epsilon_0 c} \\
&= \frac{\rho}{\epsilon_0} - \frac{\Bj}{\epsilon_0 c} \\
\end{aligned}
\end{equation}

Both the spacetime gradient and the current density four vector have been put in a quaternion-ic form with scalar and bivector grades in the
STA basis.  This leaves us with the time centric formulation of Maxwell's equation

\begin{equation}\label{eqn:firstorderFourierMaxwell:80}
\begin{aligned}
\left(\partial_0 + \spacegrad\right) F &= \frac{\rho}{\epsilon_0} - \frac{\Bj}{\epsilon_0 c}
\end{aligned}
\end{equation}

Except for the fact that we have objects of various grades here, and that this is a first instead of second order equation,
these equations have the same form as in the previous Fourier transform attacks.
Those were Fourier transform application for the homogeneous and inhomogeneous wave equations, and the heat and
Schr\"{o}dinger equation.

\section{Fourier transforming a mixed grade object}

Now, here we make the assumption that we can apply 3D Fourier transform pairs to mixed grade objects, as in

\begin{equation}\label{eqn:firstorder_fourier_maxwell:FourierTxDefinition}
\begin{aligned}
\hat{\psi}(\Bk, t) &= \inv{(\sqrt{2\pi})^3} \IIinf \psi(\Bx, t) \exp\left( -i \Bk \cdot \Bx \right) d^3 x \\
{\psi}(\Bx, t) &= \PV \inv{(\sqrt{2\pi})^3} \IIinf \hat{\psi}(\Bk, t) \exp\left( i \Bk \cdot \Bx \right) d^3 k
\end{aligned}
\end{equation}

Now, because of linearity, is it clear enough that this will work, provided this is a valid transform pair for any specific grade.
We do however want to be careful of the order of the factors since we want the flexibility to use any particular convenient representation
of \(i\), in particular \(i = \gamma_0 \gamma_1 \gamma_2 \gamma_3 = \sigma_1 \sigma_2 \sigma_3\).

Let us repeat our an ad-hoc verification that this transform pair works as desired, being careful with the order of products and specifically
allowing for \(\psi\) to be a non-scalar function.
Writing \(\Bk = k_m \sigma^m\), \(\Br = \sigma_m r^m\), \(\Bx = \sigma_m x^m\), that is an expansion of

\begin{equation}\label{eqn:firstorderFourierMaxwell:100}
\begin{aligned}
\PV &\inv{(\sqrt{2\pi})^3} \int
\left( \inv{(\sqrt{2\pi})^3} \int \psi(\Br, t) \exp\left( -i \Bk \cdot \Br \right) d^3 r \right)
\exp\left( i \Bk \cdot \Bx \right) d^3 k \\
&= \int \psi(\Br, t) d^3 r \PV \inv{({2\pi})^3} \int \exp\left( i \Bk \cdot (\Bx -\Br) \right) d^3 k \\
&= \int \psi(\Br, t) d^3 r \prod_{m=1}^3 \PV \inv{{2\pi}} \int \exp\left( i k_m (x^m -r^m) \right) dk_m \\
&= \int \psi(\Br, t) d^3 r \prod_{m=1}^3 \lim_{R\rightarrow \infty} \frac{\sin\left( R (x^m -r^m) \right)}{\pi(x^m - r^m)} \\
&\sim \int \psi(\Br, t) \delta(x^1-r^1) \delta(x^2-r^2) \delta(x^3-r^3) d^3 r \\
&= \psi(\Bx, t)
\end{aligned}
\end{equation}

In the second last step above we make the ad-hoc identification of that \(\sinc\) limit with the Dirac delta function, and recover
our original function as desired (the Rigor police are on holiday again).

\subsection{Rotor form of the Fourier transform?}

Although the formulation picked above appears to work, it is not the only choice to potentially make for the Fourier transform
of multivector.  Would it be more natural to pick an explicit Rotor formulation?  This perhaps makes more sense since it is then
automatically grade preserving.

\begin{equation}\label{eqn:firstorder_fourier_maxwell:rotorFourier}
\begin{aligned}
\hat{\psi}(\Bk, t) &= \inv{(\sqrt{2\pi})^n} \IIinf \exp\left( \inv{2} i \Bk \cdot \Bx \right) \psi(\Bx, t) \exp\left( - \inv{2} i \Bk \cdot \Bx \right) d^n x \\
{\psi}(\Bx, t) &= \PV \inv{(\sqrt{2\pi})^n} \IIinf \exp\left( -\inv{2} i \Bk \cdot \Bx \right) \hat{\psi}(\Bk, t) \exp\left( \inv{2} i \Bk \cdot \Bx \right) d^n k
\end{aligned}
\end{equation}

This is not a moot question since I later
tried to make an assumption that the grade of a transformed object equals the original grade.  That does not work with the
Fourier transform definition that has been picked in \eqnref{eqn:firstorder_fourier_maxwell:FourierTxDefinition}.  It may be necessary to revamp the complete treatment, but
for now at least an observation that the grades of transform pairs do not necessarily match is required.

Does the transform pair work?  For the \(n=1\) case this is

\begin{equation}\label{eqn:firstorderFourierMaxwell:120}
\begin{aligned}
\calF(f) = \hat{f}(k) &= \inv{\sqrt{2\pi}} \IIinf \exp\left( \inv{2} i k x \right) f(x) \exp\left( - \inv{2} i k x \right) dx \\
\calF^{-1}(\hat{f}) = {f}(x) &= \PV \inv{\sqrt{2\pi}} \IIinf \exp\left( -\inv{2} i k x \right) \hat{f}(k) \exp\left( \inv{2} i k x \right) dk
\end{aligned}
\end{equation}

Will the computation of \(\calF^{-1}(\calF(f(x)))\) produce \(f(x)\)?  Let us try

\begin{equation}\label{eqn:firstorderFourierMaxwell:140}
\begin{aligned}
&\calF^{-1}(\calF(f(x))) \\
&=
\PV \inv{{2\pi}} \IIinf \exp\left( -\inv{2} i k x \right)
\left(
\IIinf \exp\left( \inv{2} i k u \right) f(u) \exp\left( - \inv{2} i k u \right) du
\right)
\exp\left( \inv{2} i k x \right) dk \\
&=
\PV \inv{{2\pi}} \IIinf \exp\left( -\inv{2} i k (x -u) \right) f(u) \exp\left( \inv{2} i k (x -u) \right) du dk \\
\end{aligned}
\end{equation}

Now, this expression can not obviously be identified with the delta function as in the single sided transformation.  Suppose we decompose \(f\) into grades that
commute and anticommute with \(i\).  That is

\begin{equation}\label{eqn:firstorderFourierMaxwell:160}
\begin{aligned}
f &= f_\parallel + f_\perp \\
f_\parallel i &= i f_\parallel  \\
f_\perp  i &= -i f_\perp
\end{aligned}
\end{equation}

This is also sufficient to determine how these components of \(f\) behave with respect to the exponentials.  We have

\begin{equation}\label{eqn:firstorderFourierMaxwell:180}
\begin{aligned}
e^{i\theta}
&= \sum_m \frac{(i\theta)^m}{m!} \\
&= \cos(\theta) + i\sin(\theta)
\end{aligned}
\end{equation}

So we also have

\begin{equation}\label{eqn:firstorderFourierMaxwell:200}
\begin{aligned}
f_\parallel e^{i\theta} &= e^{i\theta} f_\parallel  \\
f_\perp e^{i\theta} &= e^{-i\theta} f_\perp
\end{aligned}
\end{equation}

This gives us
\begin{equation}\label{eqn:firstorderFourierMaxwell:220}
\begin{aligned}
\calF^{-1}(\calF(f(x)))
&=
\PV \inv{{2\pi}} \IIinf f_\parallel(u) du dk
+\PV \inv{{2\pi}} \IIinf f_\perp(u) \exp\left( i k (x -u) \right) du dk \\
&=
\inv{{2\pi}} \IIinf dk \IIinf f_\parallel(u) du +\IIinf f_\perp(u) \delta( x -u ) du \\
\end{aligned}
\end{equation}

So, we see two things.  First is that any \(f_\parallel \ne 0\) produces an unpleasant infinite result.  One could, in a vague sense, allow for odd valued \(f_\parallel\), however, if we were to apply this inversion transformation pair to a function time varying multivector function \(f(x,t)\), this would then require that the function is odd for all times.  Such a function must be zero valued. % or some odd construction such as a function that is zero everywhere except at some denumerable set of points.

The second thing that we see is that if \(f\) entirely anticommutes with \(i\), we do recover it with this transform pair, obtaining \(f_\perp(x)\).

With respect to Maxwell's equation
this immediately means that this double sided transform pair is of no use, since our pseudoscalar \(i = \gamma_0 \gamma_1\gamma_2 \gamma_3\) commutes with our grade two
field bivector \(F\).

\section{Fourier transforming the spacetime split gradient equation}

Now, suppose we have a Maxwell like equation of the form

\begin{equation}\label{eqn:firstorder_fourier_maxwell:spacetimeGradientEquation}
\begin{aligned}
\left(\partial_0 + \spacegrad \right) \psi = g
\end{aligned}
\end{equation}

Let us take the Fourier transform of this equation.  This gives us

\begin{equation}\label{eqn:firstorderFourierMaxwell:240}
\begin{aligned}
\partial_0 \hat{\psi} + \sigma^m \calF(\partial_m \psi) = \hat{g}
\end{aligned}
\end{equation}

Now, we need to look at the middle term in a bit more detail.  For the wave, and heat equations this was evaluated with just an integration
by parts.  Was there any commutation assumption in that previous treatment?  Let us write this out in full to make sure we are cool.

\begin{equation}\label{eqn:firstorderFourierMaxwell:260}
\begin{aligned}
\calF(\partial_m \psi)
&= \inv{(\sqrt{2\pi})^3} \int \left(\partial_m \psi(\Bx, t)\right) \exp\left( -i \Bk \cdot \Bx \right) d^3 x
\end{aligned}
\end{equation}

Let us also expand the integral completely, employing a permutation of indices \(\pi(1,2,3) = (m,n,p)\).

\begin{equation}\label{eqn:firstorderFourierMaxwell:280}
\begin{aligned}
\calF(\partial_m \psi)
&=
\inv{(\sqrt{2\pi})^3}
\int_{\partial x^p} dx^p
\int_{\partial x^n} dx^n
\int_{\partial x^m} dx^m
\left(\partial_m \psi(\Bx, t)\right) \exp\left( -i \Bk \cdot \Bx \right) \\
\end{aligned}
\end{equation}

Okay, now we are ready for the integration by parts.  We want a derivative substitution, based on

\begin{equation}\label{eqn:firstorderFourierMaxwell:300}
\begin{aligned}
\partial_m &\left( \psi(\Bx, t) \exp\left( -i \Bk \cdot \Bx \right) \right) \\
&= (\partial_m \psi(\Bx, t)) \exp\left( -i \Bk \cdot \Bx \right) + \psi(\Bx, t) \partial_m \exp\left( -i \Bk \cdot \Bx \right) \\
&= (\partial_m \psi(\Bx, t)) \exp\left( -i \Bk \cdot \Bx \right) + \psi(\Bx, t) ( -i k_m ) \exp\left( -i \Bk \cdot \Bx \right) \\
\end{aligned}
\end{equation}

Observe that we do not wish to assume that the pseudoscalar \(i\) commutes with anything except the exponential term, so we have to leave
it sandwiched or on the far right.  We also must take care to not necessarily commute the exponential itself with \(\psi\) or its derivative.
Having noted this we can rearrange as desired for the integration by parts

\begin{equation}\label{eqn:firstorderFourierMaxwell:320}
\begin{aligned}
(\partial_m \psi(\Bx, t)) \exp\left( -i \Bk \cdot \Bx \right)
&=
\partial_m \left( \psi(\Bx, t) \exp\left( -i \Bk \cdot \Bx \right) \right) - \psi(\Bx, t) ( -i k_m ) \exp\left( -i \Bk \cdot \Bx \right) \\
\end{aligned}
\end{equation}

and substitute back into the integral

\begin{equation}\label{eqn:firstorderFourierMaxwell:340}
\begin{aligned}
\sigma^m \calF(\partial_m \psi)
&=
\inv{(\sqrt{2\pi})^3}
\int_{\partial x^p} dx^p
\int_{\partial x^n} dx^n
{\left. {\left(\sigma^m \psi(\Bx, t) \exp\left( -i \Bk \cdot \Bx \right) \right)} \right\vert}_{\partial x^m} \\
&-
\inv{(\sqrt{2\pi})^3}
\int_{\partial x^p} dx^p
\int_{\partial x^n} dx^n
\int_{\partial x^m} dx^m
\sigma^m \psi(\Bx, t) ( -i k_m )
\exp\left( -i \Bk \cdot \Bx \right)
\\
\end{aligned}
\end{equation}

So, we find that the Fourier transform of our spatial gradient is

\begin{equation}\label{eqn:firstorderFourierMaxwell:360}
\begin{aligned}
\calF(\grad \psi) = \Bk \hat{\psi} i
\end{aligned}
\end{equation}

This has the specific ordering of the vector products for our possibility of non-commutative factors.

From this, without making any assumptions about grade, we have the wave number domain equivalent
for the spacetime split of the gradient \eqnref{eqn:firstorder_fourier_maxwell:spacetimeGradientEquation}

\begin{equation}\label{eqn:firstorder_fourier_maxwell:waveDomainGeneral}
\begin{aligned}
\partial_0 \hat{\psi} + \Bk \hat{\psi} i = \hat{g}
\end{aligned}
\end{equation}

\section{Back to specifics.  Maxwell's equation in wave number domain}

For Maxwell's equation our field variable \(F\) is grade two in the STA basis, and our specific transform pair is:

\begin{equation}\label{eqn:firstorderFourierMaxwell:380}
\begin{aligned}
\left(\partial_0 + \spacegrad \right) F &= \gamma_0 J/\epsilon_0 c \\
\partial_0 \hat{F} + \Bk \hat{F} i &= \gamma_0 \hat{J}/ \epsilon_0 c
\end{aligned}
\end{equation}

Now, \(\exp(i\theta)\) and \(i\) commute, and \(i\) also commutes with both \(F\) and \(\Bk\).  This is true since our field \(F\) as well as the spatial vector \(\Bk\) are grade two in the STA basis.
Two sign interchanges occur as we commute with each vector factor of these
bivectors.

This allows us to write our transformed equation in the slightly tidier form

\begin{equation}\label{eqn:firstorder_fourier_maxwell:toSolve}
\begin{aligned}
\partial_0 \hat{F} + (i \Bk) \hat{F} &= \gamma_0 \hat{J}/ \epsilon_0 c
\end{aligned}
\end{equation}

We want to find a solution to this equation.  If the objects in question were all scalars this would be simple enough, and is a problem of the form

\begin{equation}\label{eqn:firstorder_fourier_maxwell:firstOrder}
\begin{aligned}
B' + A B &= Q
\end{aligned}
\end{equation}

For our electromagnetic field our transform is a summation of the following
form

\begin{equation}\label{eqn:firstorderFourierMaxwell:400}
\begin{aligned}
(\BE + i c \BB) (\cos\theta + i \sin\theta)
&=
(\BE \cos\theta - c \BB \sin\theta) +
i (\BE \sin\theta + c \BB \cos\theta)
\end{aligned}
\end{equation}

The summation of the integral itself will not change the grades, so \(\hat{F}\)
is also a grade two multivector.  The dual of our spatial wave number
vector \(i\Bk\) is also grade two with basis bivectors \(\gamma_m \gamma_n\) very
much like the magnetic field portions of our field vector \(i c \BB\).

Having figured out the grades of all the terms in \eqnref{eqn:firstorder_fourier_maxwell:toSolve}, what
does a grade split of this equation yield?  For the equation to be true
do we not need it to be true for all grades?  Our grade zero, four, and two
terms respectively are

\begin{equation}\label{eqn:firstorderFourierMaxwell:420}
\begin{aligned}
(i \Bk) \cdot \hat{F} &= \hat{\rho}/ \epsilon_0 \\
(i \Bk) \wedge \hat{F} &= 0 \\
\partial_0 \hat{F} + (i \Bk) \times \hat{F} &= -\hat{\Bj}/ \epsilon_0 c
\end{aligned}
\end{equation}

Here the (antisymmetric) commutator product \(\gpgradetwo{ab} = a \times b = (a b - ba)/2\) has been used in the last equation for this bivector product.

It is kind of interesting that an unmoving charge density contributes nothing
to the time variation of the field in the wave number domain, instead
only the current density (spatial) vectors contribute to our differential
equation.

\subsection{Solving this first order inhomogeneous problem}

We want to solve the inhomogeneous scalar equation
\eqnref{eqn:firstorder_fourier_maxwell:firstOrder} but do so in a fashion that is also valid for
the grades for the Maxwell equation problem.

Application of variation of parameters produces the desired result.  Let us write this equation in operator form

\begin{equation}\label{eqn:firstorderFourierMaxwell:440}
\begin{aligned}
L(B) &= B' + A B
\end{aligned}
\end{equation}

and start with the solution of the
homogeneous problem

\begin{equation}\label{eqn:firstorderFourierMaxwell:460}
\begin{aligned}
L(B) = 0
\end{aligned}
\end{equation}

This is

\begin{equation}\label{eqn:firstorderFourierMaxwell:480}
\begin{aligned}
B' = -A B
\end{aligned}
\end{equation}

so we expect exponential solutions will do the trick, but have to get the ordering right due to the possibility of non-commutative factors.  How about one of

\begin{equation}\label{eqn:firstorderFourierMaxwell:500}
\begin{aligned}
B &= C e^{-At} \\
B &= e^{-At} C
\end{aligned}
\end{equation}

Where \(C\) is constant, but not necessarily a scalar, and does not have to commute with \(A\).  Taking derivatives of the first we have

\begin{equation}\label{eqn:firstorderFourierMaxwell:520}
\begin{aligned}
B' = C (-A) e^{-At}
\end{aligned}
\end{equation}

This does not have the desired form unless \(C\) and \(A\) commute.  How about the second possibility?  That one has the derivative

\begin{equation}\label{eqn:firstorderFourierMaxwell:540}
\begin{aligned}
B'
&= (-A) e^{-At} C \\
&= -A B
\end{aligned}
\end{equation}

which is what we want.  Now, for the inhomogeneous problem we want to use a test solution replacing C with an function to be determined.  That is

\begin{equation}\label{eqn:firstorderFourierMaxwell:560}
\begin{aligned}
B &= e^{-At} U
\end{aligned}
\end{equation}

For this we have
\begin{equation}\label{eqn:firstorderFourierMaxwell:580}
\begin{aligned}
L(B)
&= (-A) e^{-At} U + e^{-At} U' + A B  \\
&= e^{-At} U'
\end{aligned}
\end{equation}

Our inhomogeneous problem \(L(B) = Q\) is therefore reduced to

\begin{equation}\label{eqn:firstorderFourierMaxwell:600}
\begin{aligned}
e^{-At} U' &= Q
\end{aligned}
\end{equation}

Or
\begin{equation}\label{eqn:firstorderFourierMaxwell:620}
\begin{aligned}
U &= \int e^{A t} Q(t) dt
\end{aligned}
\end{equation}

As an indefinite integral this gives us

\begin{equation}\label{eqn:firstorderFourierMaxwell:640}
\begin{aligned}
B(t)
&= e^{-At} \int e^{A t} Q(t) dt \\
\end{aligned}
\end{equation}

And finally in definite integral form,
if all has gone well, we have a specific solution to the forced problem

\begin{equation}\label{eqn:firstorder_fourier_maxwell:linearSolved}
\begin{aligned}
B(t) &= \int_{t_0}^{t} e^{-A (t -\tau)} Q(\tau) d\tau
\end{aligned}
\end{equation}

\subsubsection{Verify}

With differentiation under the integral sign we have

\begin{equation}\label{eqn:firstorderFourierMaxwell:660}
\begin{aligned}
\frac{dB}{dt}
&={\left. e^{-A (t -\tau)} Q(\tau) \right\vert}_{\tau=t} + \int_{t_0}^{t} -A e^{-A (t -\tau)} Q(\tau) d\tau \\
&= Q(t) - A B
\end{aligned}
\end{equation}

Great!

\subsection{Back to Maxwell's}

Switching to explicit time derivatives we have

\begin{equation}\label{eqn:firstorderFourierMaxwell:680}
\begin{aligned}
\partial_t \hat{F} + (i c \Bk) \hat{F} &= \gamma_0 \hat{J}/ \epsilon_0
\end{aligned}
\end{equation}

By \eqnref{eqn:firstorder_fourier_maxwell:linearSolved}, this has, respectively, homogeneous and inhomogeneous solutions

\begin{equation}\label{eqn:firstorder_fourier_maxwell:waveNumberDomain}
\begin{aligned}
\hat{F}(\Bk,t) &= e^{-i c \Bk t} C(\Bk) \\
\hat{F}(\Bk,t) &= \inv{\epsilon_0} \int_{t_0(\Bk)}^{t} e^{-(i c \Bk) (t -\tau)} \gamma_0 \hat{J}(\Bk,\tau) d\tau
\end{aligned}
\end{equation}

For the homogeneous term at \(t=0\) we have

\begin{equation}\label{eqn:firstorderFourierMaxwell:700}
\begin{aligned}
\hat{F}(\Bk,0) &= C(\Bk) \\
\end{aligned}
\end{equation}

So, \(C(\Bk)\) is just the Fourier transform of an initial wave packet.  Reassembling all the bits in terms of fully specified Fourier and inverse Fourier transforms we have

\begin{equation}\label{eqn:firstorderFourierMaxwell:720}
\begin{aligned}
F(\Bx,t)
&=
%F^(\Bk, t) = \inv{(\sqrt{2\pi})^3} \int e^{-ic \Bk t} F(\Bx,0) e^{-i \Bk \cdot \Bx} d^3 x
\inv{(\sqrt{2\pi})^3} \int
%\hat{F}(\Bk,t)
\left(
\inv{(\sqrt{2\pi})^3} \int e^{-ic \Bk t} F(\Bu,0) e^{-i \Bk \cdot \Bu} d^3 u
\right)
e^{i \Bk \cdot \Bx} d^3 k \\
&= \inv{({2\pi})^3} \int e^{-ic \Bk t} F(\Bu,0) e^{i \Bk \cdot (\Bx - \Bu)} d^3 u d^3 k \\
\end{aligned}
\end{equation}

We have something like a double sided Green's function, with which we do a spatial convolution over all space with to produce a function of wave number.  One more integration over all wave numbers gives us our inverse Fourier transform.  The final result is a beautiful closed form solution for the time evolution of an arbitrary wave packet for the field specified at some specific initial time.

Now, how about that forced term?  We want to inverse Fourier transform our \(\hat{J}\) based equation in \eqnref{eqn:firstorder_fourier_maxwell:waveNumberDomain}.  Picking our \(t_0 = -\infty\) this is

\begin{equation}\label{eqn:firstorderFourierMaxwell:740}
\begin{aligned}
F(\Bx,t)
&=
\inv{(\sqrt{2\pi})^3} \int
\left( \inv{\epsilon_0} \int_{\tau = -\infty}^{t} e^{-(i c \Bk) (t -\tau)} \gamma_0 \hat{J}(\Bk,\tau) d\tau  \right) e^{i \Bk \cdot \Bx} d^3 k \\
&=
\inv{\epsilon_0 ({2\pi})^3} \int
\int_{\tau = -\infty}^{t} e^{-(i c \Bk) (t -\tau)} \gamma_0 {J}(\Bu,\tau)
e^{i \Bk \cdot (\Bx-\Bu)}
d\tau
d^3 u
d^3 k
\end{aligned}
\end{equation}

Again we have a double sided Green's function.  We require a convolution summing the four vector current density contributions over all space and for all times less than \(t\).

Now we can combine the vacuum and charge present solutions for a complete solution to Maxwell's equation.  This is

\begin{equation}\label{eqn:firstorderFourierMaxwell:760}
\begin{aligned}
F(\Bx,t)
&=
\inv{({2\pi})^3} \int
e^{-i c \Bk t}
\left(
F(\Bu, 0) + \inv{\epsilon_0} \int_{\tau = -\infty}^{t} e^{i c \Bk \tau } \gamma_0 J(\Bu,\tau)  d\tau
\right)
e^{i \Bk \cdot (\Bx-\Bu)}
d^3 u
d^3 k
\end{aligned}
\end{equation}

Now, this may not be any good for actually computing with, but it sure is pretty!

There is a lot of verification required to see if all this math actually works out, and
also a fair amount of followup required to play with this and see what other goodies fall out if this is used.  I had expect that this result ought to be usable to show familiar
results like the Biot-Savart law.

How do our energy density and Poynting energy momentum density conservation relations, and the stress energy tensor terms, look given a closed form expression for \(F\)?

It is also kind of interesting to see the time phase term coupled to the current density here in the forcing term.  That looks awfully similar to some QM expressions, although it
could be coincidental.

   %
% Copyright � 2012 Peeter Joot.  All Rights Reserved.
% Licenced as described in the file LICENSE under the root directory of this GIT repository.
%

%
%
\chapter{4D Fourier transforms applied to Maxwell's equation}\label{chap:PJ4dFourier}
\index{Fourier transform!4D}
%\date{Feb 1, 2009.  4dFourier.tex}

\section{Notation}

Please see \chapcite{notationTable} for a summary of much of the notation used here.

\section{Motivation}

In \chapcite{PJfirstOrderMaxwell}, a solution of the first order Maxwell equation

\begin{equation}\label{eqn:4dFourier:23}
\begin{aligned}
\grad F &= \frac{J }{\epsilon_0 c}
\end{aligned}
\end{equation}

was found to be

\begin{equation}\label{eqn:4dFourier:43}
\begin{aligned}
F(\Bx,t)
&=
\inv{({2\pi})^3} \int
e^{-i c \Bk t}
\left(
F(\Bu, 0) + \inv{\epsilon_0} \int_{\tau = -\infty}^{t} e^{i c \Bk \tau } \gamma_0 J(\Bu,\tau)  d\tau
\right)
e^{i \Bk \cdot (\Bx-\Bu)}
d^3 u
d^3 k
\end{aligned}
\end{equation}

This does not have the spacetime uniformity that is expected for a solution of a Lorentz invariant equation.

Similarly, in \chapcite{PJfourierMaxwellSecondOrder} solutions of the second order Maxwell equation in the Lorentz gauge
\(\grad \cdot A = 0\)

\begin{equation}\label{eqn:4dFourier:63}
\begin{aligned}
F &= \grad \wedge A \\
\grad^2 A &= J/\epsilon_0 c
\end{aligned}
\end{equation}

were found to be
\begin{equation}\label{eqn:4d_fourier:fourVectorPotentials}
\begin{aligned}
{A^\mu}(x)
&= \inv{\epsilon_0 c} \int J^\mu(x') G(x - x') d^4 x' \\
G(x)
&=
\frac{u(x \cdot \gamma_0)}{ (2\pi)^3 }
\int
\sin( \Abs{\Bk} x \cdot \gamma_0 )
\exp\left( -i (\Bk \gamma_0) \cdot x \right)
\frac{d^3 k}{ \Abs{\Bk} }
\end{aligned}
\end{equation}

Here our convolution kernel \(G\) also does not exhibit a uniform four vector form that one could logically expect.

In these notes an attempt to rework these problems using a 4D spacetime Fourier transform will be made.

\section{4D Fourier transform}

As before we want a multivector friendly Fourier transform pair, and choose the following

\begin{equation}\label{eqn:4d_fourier:FourierTxDefinition}
\begin{aligned}
\hat{\psi}(k) &= \inv{(\sqrt{2\pi})^4} \IIinf \psi(x) \exp\left( -i k \cdot x \right) d^4 x \\
{\psi}(x) &= \PV \inv{(\sqrt{2\pi})^4} \IIinf \hat{\psi}(k) \exp\left( i k \cdot x \right) d^4 k
\end{aligned}
\end{equation}

Here we use \(i = \gamma_0 \gamma_1 \gamma_2 \gamma_3\) as our pseudoscalar, and have to therefore be careful of order
of operations since this does not necessarily commute with multivector \(\psi\) or \(\hat{\psi}\) functions.

For our dot product and vectors, with summation over matched upstairs downstairs indices implied, we write

\begin{equation}\label{eqn:4dFourier:83}
\begin{aligned}
x &= x^\mu \gamma_\mu = x_\mu \gamma^\mu \\
k &= k^\mu \gamma_\mu = k_\mu \gamma^\mu \\
x \cdot k &= x^\mu k_\mu = x_\mu k^\mu
\end{aligned}
\end{equation}

Finally our differential volume elements are defined to be

\begin{equation}\label{eqn:4dFourier:103}
\begin{aligned}
d^4 x &= dx^0 dx^1 dx^2 dx^3 \\
d^4 k &= dk_0 dk_1 dk_2 dk_3 \\
\end{aligned}
\end{equation}

Note the opposite pairing of upstairs and downstairs indices in the coordinates.

\section{Potential equations}

\subsection{Inhomogeneous case}

First for the attack is the Maxwell potential equations.  As well as using a 4D transform, having learned how to do Fourier
transformations of multivectors, we will attack this one in vector form as well.  Our equation to invert is

\begin{equation}\label{eqn:4dFourier:123}
\begin{aligned}
\grad^2 A = J/\epsilon_0 c
\end{aligned}
\end{equation}

There is nothing special to do for the transformation of the current term, but the left hand side will require two integration
parts

\begin{equation}\label{eqn:4dFourier:143}
\begin{aligned}
\calF(\grad^2 A )
&= \inv{(2 \pi)^2} \IIinf \left(\left(\partial_{00} - \sum_m \partial_{mm}\right) A \right) e^{ -i k_\mu x^\mu } d^4 x \\
&= \inv{(2 \pi)^2} \IIinf A \left( (-i k_0)^2 - \sum_m (-i k_m)^2 \right) e^{ -i k_\mu x^\mu } d^4 x \\
\end{aligned}
\end{equation}

As usual it is required that \(A\) and \(\partial_\mu A\) vanish at infinity.  Now for the scalar in the interior we have

\begin{equation}\label{eqn:4dFourier:163}
\begin{aligned}
(-i k_0)^2 - \sum_m (-i k_m)^2
&= -(k_0)^2 + \sum_m (k_m)^2 \\
\end{aligned}
\end{equation}

But this is just the (negation) of the square of our wave number vector
\begin{equation}\label{eqn:4dFourier:183}
\begin{aligned}
k^2
&= k_\mu \gamma^\mu \cdot k_\nu \gamma^\nu \\
&= k_\mu k_\nu \gamma^\mu \cdot \gamma^\nu \\
&=
k_0 k_0 \gamma_0 \cdot \gamma^0
-\sum_{a,b} k_a k_b \gamma_a \cdot \gamma^b \\
&= (k_0)^2 - \sum_a (k_a)^2
\end{aligned}
\end{equation}

Putting things back together we have for our potential vector in the wave number domain

\begin{equation}\label{eqn:4dFourier:203}
\begin{aligned}
\hat{A} &= \frac{\hat{J}}{- k^2 \epsilon_0 c}
\end{aligned}
\end{equation}

Inverting, and substitution for \(\hat{J}\) gives us our spacetime domain potential vector in one fell swoop

\begin{equation}\label{eqn:4dFourier:223}
\begin{aligned}
A(x)
&=
\inv{(\sqrt{2\pi})^4} \IIinf
\left(
\inv{- k^2 \epsilon_0 c} \inv{(\sqrt{2\pi})^4} \IIinf {J}(x') e^{-i k \cdot x' } d^4 x'
\right)
e^{ i k \cdot x } d^4 k \\
&=
\inv{({2\pi})^4} \IIinf {J}(x') \inv{- k^2 \epsilon_0 c} e^{ i k \cdot (x - x') } d^4 k d^4 x'
\\
\end{aligned}
\end{equation}

This allows us to write this entire specific solution to the forced wave equation problem as a convolution integral

\begin{equation}\label{eqn:4d_fourier:potentialInHomogeneous}
\begin{aligned}
A(x) &= \inv{\epsilon_0 c} \IIinf {J}(x') G(x-x') d^4 x' \\
G(x) &= \frac{-1}{ ({2\pi})^4} \IIinf \frac{e^{ i k \cdot x }}{ k^2 } d^4 k
\end{aligned}
\end{equation}

Pretty slick looking, but actually also problematic if one thinks about it.  Since \(k^2\) is null in some cases
\(G(x)\) may blow up in some conditions.  My assumption however, is that a well defined meaning can be associated
with this integral, I just do not know what it is yet.  A way to define this more exactly may require
picking a more specific orthonormal basis once the exact character of \(J\) is known.

FIXME: In \chapcite{PJpoisson} I worked through how to evaluate such an integral (expanding on a too brief treatment found in \citep{byron1992mca}).  To apply such a technique here, where our Green's function has precisely the same form as the Green's function for the Poisson's equation, a way to do the equivalent of a spherical polar parametrization will be required.  How would that be done in 4D?  Have seen such treatments in \citep{flanders1989dfa} for hypervolume and surface integration, but they did not make much sense then.  Perhaps they would now?

\subsection{The homogeneous case}

The missing element here is the addition of any allowed homogeneous solutions to the wave equation.
The form of such solutions cannot be obtained with the 4D transform since that produces

\begin{equation}\label{eqn:4dFourier:243}
\begin{aligned}
-k^2 \hat{A} = 0
\end{aligned}
\end{equation}

and no meaningful inversion of that is possible.

For the homogeneous problem we are forced to re-express the spacetime Laplacian with an explicit bias towards either time or a specific direction in space, and attack with a Fourier transform on the remaining coordinates.  This has been done previously, but we can
revisit this using our new vector transform.

Now we switch to a spatial Fourier transform

\begin{equation}\label{eqn:4d_fourier:3DFourierTxDefinition}
\begin{aligned}
\hat{\psi}(\Bk, t) &= \inv{(\sqrt{2\pi})^3} \IIinf \psi(\Bx, t) \exp\left( -i \Bk \cdot \Bx \right) d^3 x \\
{\psi}(\Bx, t) &= \PV \inv{(\sqrt{2\pi})^3} \IIinf \hat{\psi}(\Bk, t) \exp\left( i \Bk \cdot \Bx \right) d^3 k
\end{aligned}
\end{equation}

Using a spatial transform we have

\begin{equation}\label{eqn:4dFourier:263}
\begin{aligned}
\calF((\partial_{00} - \sum_m \partial_{mm}) A)
&= \partial_{00} \hat{A} - \sum_m \hat{A} (-i k_m)^2
\end{aligned}
\end{equation}

Carefully keeping the pseudoscalar factors all on the right of our vector as the integration by parts was performed does not make a difference since we just end up with a scalar in the end.  Our equation in the wave number domain is then just

\begin{equation}\label{eqn:4dFourier:283}
\begin{aligned}
\partial_{tt} \hat{A}(\Bk,t) + (c^2 \Bk^2) \hat{A}(\Bk,t) &= 0 %c \hat{J}(\Bk,t)/\epsilon_0
\end{aligned}
\end{equation}

with exponential solutions

\begin{equation}\label{eqn:4dFourier:303}
\begin{aligned}
\hat{A}(\Bk, t) &= C(\Bk) \exp(\pm i c \Abs{\Bk} t)
\end{aligned}
\end{equation}

In particular, for \(t = 0\) we have

\begin{equation}\label{eqn:4dFourier:323}
\begin{aligned}
\hat{A}(\Bk, 0) &= C(\Bk)
\end{aligned}
\end{equation}

Reassembling then gives us our homogeneous solution

\begin{equation}\label{eqn:4dFourier:343}
\begin{aligned}
{A}(\Bx, t)
&=
\inv{(\sqrt{2\pi})^3} \IIinf
\left( \inv{(\sqrt{2\pi})^3} \IIinf A(\Bx', 0) e^{ -i \Bk \cdot \Bx' } d^3 x' \right) e^{\pm i c \Abs{\Bk} t}
e^{ i \Bk \cdot \Bx } d^3 k
\end{aligned}
\end{equation}

This is

\begin{equation}\label{eqn:4d_fourier:potentialHomogeneous}
\begin{aligned}
{A}(\Bx, t) &= \IIinf A(\Bx', 0) G( \Bx - \Bx' ) d^3 x' \\
G(\Bx) &= \inv{({2\pi})^3} \IIinf \exp\left( i \Bk \cdot \Bx \pm i c \Abs{\Bk} t \right) d^3 k
\end{aligned}
\end{equation}

Here also we have to be careful to keep the Green's function on the right hand side of \(A\) since they will not generally commute.

\subsection{Summarizing}

Assembling both the homogeneous and inhomogeneous parts for a complete solution we have for the Maxwell
four vector potential

\begin{equation}\label{eqn:4dFourier:363}
\begin{aligned}
A(x) &= \IIinf \left( A(\Bx', 0) H( \Bx - \Bx' ) + \inv{\epsilon_0 c} \IIinf {J}(x') G(x-x') dx^0 \right) dx^1 dx^2 dx^3 \\
H(\Bx) &= \inv{({2\pi})^3} \IIinf \exp\left( i \Bk \cdot \Bx  \pm i c \Abs{\Bk} t \right) d^3 k \\
G(x) &= \frac{-1}{ ({2\pi})^4} \IIinf \frac{e^{ i k \cdot x }}{ k^2 } d^4 k
\end{aligned}
\end{equation}

Here for convenience both four vectors and spatial vectors were used with

\begin{equation}\label{eqn:4dFourier:383}
\begin{aligned}
x &= x^\mu \gamma_\mu \\
\Bx &= x^m \sigma_m = x \wedge \gamma_0
\end{aligned}
\end{equation}

As expected, operating where possible in a Four vector context does produce a simpler convolution kernel for the vector potential.

\section{First order Maxwell equation treatment}

Now we want to Fourier transform Maxwell's equation directly.  That is

\begin{equation}\label{eqn:4dFourier:403}
\begin{aligned}
\calF(\grad F = J/\epsilon_0 c)
\end{aligned}
\end{equation}

For the LHS we have

\begin{equation}\label{eqn:4dFourier:423}
\begin{aligned}
\calF(\grad F)
&= \calF(\gamma^\mu \partial_\mu F) \\
&= \gamma^\mu \inv{(2\pi)^2} \IIinf (\partial_\mu F) e^{ - i k \cdot x } d^4 x \\
&= -\gamma^\mu \inv{(2\pi)^2} \IIinf F \partial_\mu (e^{ - i k_\sigma x^\sigma }) d^4 x \\
&= -\gamma^\mu \inv{(2\pi)^2} \IIinf F (-i k_\mu ) e^{ - i k \cdot x } d^4 x \\
&= -i \gamma^\mu k_\mu \inv{(2\pi)^2} \IIinf F e^{ - i k \cdot x } d^4 x \\
&= -i k \hat{F}
\end{aligned}
\end{equation}

This gives us

\begin{equation}\label{eqn:4dFourier:443}
\begin{aligned}
-i k \hat{F} = \hat{J}/\epsilon_0 c
\end{aligned}
\end{equation}

So to solve the forced Maxwell equation we have only to inverse transform the following

\begin{equation}\label{eqn:4dFourier:463}
\begin{aligned}
\hat{F} = \inv{ -i k \epsilon_0 c} \hat{J}
\end{aligned}
\end{equation}

This is

\begin{equation}\label{eqn:4dFourier:483}
\begin{aligned}
{F}
&= \inv{(\sqrt{2\pi})^4} \IIinf \inv{ -i k \epsilon_0 c} \left( \inv{(\sqrt{2\pi})^4} \IIinf J(x') e^{ -i k \cdot x' } d^4 x' \right) e^{ i k \cdot x } d^4 k \\
\end{aligned}
\end{equation}

Adding to this a solution to the homogeneous equation we now have a complete solution in terms of the given four current density and an
initial field wave packet

\begin{equation}\label{eqn:4dFourier:503}
\begin{aligned}
{F} &=
\inv{({2\pi})^3} \int e^{ -i c \Bk t } F(\Bx', 0) e^{ i \Bk \cdot (\Bx-\Bx') } d^3 x' d^3 k
+
\inv{ ({2\pi})^4 \epsilon_0 c} \int \frac{i}{ k } J(x') e^{ i k \cdot (x - x') } d^4 k d^4 x' \\
\end{aligned}
\end{equation}

Observe that we can not make a single sided Green's function to convolve \(J\) with since the vectors \(k\) and \(J\) may not commute.

As expected working in a relativistic context for our inherently relativistic equation turns out to be much simpler and produce a simpler result.  As before
trying to actually evaluate these integrals is a different story.


   %
% Copyright � 2012 Peeter Joot.  All Rights Reserved.
% Licenced as described in the file LICENSE under the root directory of this GIT repository.
%

%
%
\chapter{Fourier series Vacuum Maxwell's equations}\label{chap:PJFourierVacuum}
\index{Maxwell's equations!Fourier series}
%\date{Feb 03, 2009.  fourierSeriesMaxwell.tex}

\section{Motivation}

In \citep{bohm1989qt},
after finding a formulation of Maxwell's equations that he likes, his next
step is to assume the electric and magnetic fields can be expressed in
a 3D Fourier series form, with periodicity in some repeated volume
of space, and then proceeds to evaluate the energy of the
field.

\subsection{Notation}

See the notational table \chapcite{notationTable} for much of the notation
assumed here.

\section{Setup}

Let us try this.  Instead of using the sine and cosine Fourier series
which looks more complex than it ought to be, use of a complex exponential
ought to be cleaner.

\subsection{3D Fourier series in complex exponential form}

For a multivector function \(f(\Bx, t)\), periodic in some rectangular spatial volume, let us assume that we have a
3D Fourier series representation.

Define the element of volume for our fundamental wavelengths to be the region bounded by three intervals in the \(x^1, x^2, x^3\) directions respectively

\begin{equation}\label{eqn:fourierSeriesMaxwell:20}
\begin{aligned}
I_1 &= [ a^1, a^1 + \lambda_1 ] \\
I_2 &= [ a^2, a^2 + \lambda_2 ] \\
I_3 &= [ a^3, a^3 + \lambda_3 ] \\
\end{aligned}
\end{equation}

Our assumed Fourier representation is then

\begin{equation}\label{eqn:fourierSeriesMaxwell:40}
\begin{aligned}
f(\Bx, t) &= \sum_{\Bk} \hat{f}_{\Bk}(t) \exp\left( - \sum_j \frac{2 \pi i k_j x^j}{\lambda_j} \right)
\end{aligned}
\end{equation}

Here \(\hat{f}_{\Bk} = \hat{f}_{\{k_1, k_2, k_3\}}\) is indexed over a triplet of integer values, and the \(k_1, k_2, k_3\) indices take on all integer values in the \([-\infty, \infty]\) range.

Note that we also wish to allow \(i\) to not just be a generic complex number, but allow for the use of either the Euclidean or Minkowski pseudoscalar

\begin{equation}\label{eqn:fourierSeriesMaxwell:60}
\begin{aligned}
i = \gamma_0 \gamma_1 \gamma_2 \gamma_3 = \sigma_1 \sigma_2 \sigma_3
\end{aligned}
\end{equation}

Because of this we should not assume that we can commute \(i\), or our exponentials with the functions \(f(\Bx,t)\), or \(\hat{f}_{\Bk}(t)\).

\begin{equation}\label{eqn:fourierSeriesMaxwell:80}
\begin{aligned}
\int_{x^1 = \partial I_1} &\int_{x^2 = \partial I_2} \int_{x^3 = \partial I_3} f(\Bx, t)
e^{ 2 \pi i m_j x^j/\lambda_j}
dx^1 dx^2 dx^3 \\
&= \sum_{\Bk} \hat{f}_{\Bk}(t) \int_{x^1 = \partial I_1} \int_{x^2 = \partial I_2} \int_{x^3 = \partial I_3} dx^1 dx^2 dx^3 e^{ 2 \pi i (m_j - k_j) x^j/\lambda_j} dx^1 dx^2 dx^3
\end{aligned}
\end{equation}

But each of these integrals is just \(\delta_{\Bk,\Bm} \lambda_1 \lambda_2 \lambda_3\), giving us

\begin{equation}\label{eqn:fourierSeriesMaxwell:100}
\begin{aligned}
\hat{f}_{\Bk}(t)
&= \inv{\lambda_1 \lambda_2 \lambda_3 } \int_{x^1 = \partial I_1} \int_{x^2 = \partial I_2} \int_{x^3 = \partial I_3} f(\Bx, t) \exp\left( \sum_j \frac{2 \pi i k_j x^j}{\lambda_j} \right) dx^1 dx^2 dx^3 \\
\end{aligned}
\end{equation}

To tidy things up
lets invent (or perhaps abuse) some notation to tidy things up.  As a subscript on our Fourier coefficients we have used \(\Bk\) as an index.
Let us also use it as a vector, and define

\begin{equation}\label{eqn:fourierSeriesMaxwell:120}
\begin{aligned}
\Bk \equiv 2 \pi \sum_m \frac{\sigma^m k_m}{\lambda_m}
\end{aligned}
\end{equation}

With our spatial vector \(\Bx\) written

\begin{equation}\label{eqn:fourierSeriesMaxwell:140}
\begin{aligned}
\Bx = \sum_m \sigma_m x^m
\end{aligned}
\end{equation}

We now have a \(\Bk \cdot \Bx\) term in the exponential, and can remove when desirable the coordinate summation.  If we write \(V = \lambda_1 \lambda_2 \lambda_3\)
it leaves a nice tidy notation for the 3D Fourier series over the volume

\begin{equation}\label{eqn:fourierSeriesMaxwell:160}
\begin{aligned}
f(\Bx, t) &= \sum_{\Bk} \hat{f}_{\Bk}(t) e^{ - i \Bk \cdot \Bx } \\
\hat{f}_{\Bk}( t) &= \inv{V} \int f(\Bx, t) e^{ i \Bk \cdot \Bx } d^3 x
\end{aligned}
\end{equation}

This allows us to proceed without caring about the specifics of the lengths of the sides of the rectangular prism that defines the periodicity of the signal
in question.

\subsection{Vacuum equation}

Now that we have a desirable seeming Fourier series representation, we
want to apply this to Maxwell's equation for the vacuum.  We will use the
STA formulation of Maxwell's equation, but use the unit convention of Bohm's
book.

In \chapcite{PJrayleighJeans} the STA equivalent to Bohm's notation
for Maxwell's equations was found to be

\begin{equation}\label{eqn:fourier_series_maxwell:maxwell}
\begin{aligned}
F &= \bcE + i\bcH \\
J &= (\rho + \Bj) \gamma_0 \\
\grad F &= 4 \pi J
\end{aligned}
\end{equation}

This is the CGS form of Maxwell's equation, but with the old style \(\bcH\) for \(c\BB\), and \(\bcE\) for \(\BE\).  In more recent texts \(\bcE\) (as a non-vector) is reserved for electromotive flux.  In this set of notes I use Bohm's notation, since the aim is to clarify for myself aspects of his treatment.

For the vacuum equation, we make an explicit spacetime split by premultiplying with \(\gamma_0\)

\begin{equation}\label{eqn:fourierSeriesMaxwell:180}
\begin{aligned}
\gamma_0 \grad
&= \gamma_0
\lr{ \gamma^0 \partial_0 + \gamma^k \partial_k } \\
&= \partial_0 - \gamma^k \gamma_0 \partial_k \\
&= \partial_0 + \gamma_k \gamma_0 \partial_k \\
&= \partial_0 + \sigma_k \partial_k \\
&= \partial_0 + \spacegrad \\
\end{aligned}
\end{equation}

So our vacuum equation is just

\begin{equation}\label{eqn:fourier_series_maxwell:vacuumMaxwell}
\begin{aligned}
(\partial_0 + \spacegrad) F = 0
\end{aligned}
\end{equation}

\section{First order vacuum solution with Fourier series}

\subsection{Basic solution in terms of undetermined coefficients}

Now that a notation for the 3D Fourier series has been established, we
can assume a series solution for our field of the form

\begin{equation}\label{eqn:fourier_series_maxwell:assumed}
\begin{aligned}
F(\Bx,t) = \sum_{\Bk} \hat{F}_{\Bk}(t) e^{-2\pi i k_j x^j/\lambda_j}
\end{aligned}
\end{equation}

can now apply this to the vacuum Maxwell equation \eqnref{eqn:fourier_series_maxwell:vacuumMaxwell}.
This gives us

\begin{equation}\label{eqn:fourierSeriesMaxwell:200}
\begin{aligned}
\sum_{\Bk} \left(\partial_t \hat{F}_{\Bk}(t) \right) e^{-2\pi i k_j x^j/\lambda_j}
&= -c \sum_{\Bk, m} \sigma^m \hat{F}_{\Bk}(t) \PD{x^m}{} e^{-2\pi i k_j x^j/\lambda_j} \\
&= -c \sum_{\Bk, m} \sigma^m \hat{F}_{\Bk}(t) \left(-2 \pi \frac{k_m}{\lambda_m}\right) e^{-2\pi i k_j x^j/\lambda_j} \\
&= 2 \pi c \sum_{\Bk} \sum_m \frac{\sigma^m k_m}{\lambda_m} \hat{F}_{\Bk}(t) i e^{-2\pi i k_j x^j/\lambda_j} \\
\end{aligned}
\end{equation}


Note that \(i\) commutes with \(\Bk\) and since \(F\) is also an STA bivector \(i\) commutes with \(F\).  Putting all this together we have

\begin{equation}\label{eqn:fourierSeriesMaxwell:220}
\begin{aligned}
\sum_{\Bk} \left(\partial_t \hat{F}_{\Bk}(t) \right) e^{-i \Bk \cdot \Bx }
&= i c \sum_{\Bk} \Bk \hat{F}_{\Bk}(t) e^{- i \Bk \cdot \Bx } \\
\end{aligned}
\end{equation}

Term by term we now have a (big ass, triple infinite) set of very simple first order differential equations, one for each \(\Bk\) triplet of indices.  Specifically this is

\begin{equation}\label{eqn:fourierSeriesMaxwell:240}
\begin{aligned}
\hat{F}_{\Bk}' &= i c \Bk \hat{F}_{\Bk}
\end{aligned}
\end{equation}

With solutions

\begin{equation}\label{eqn:fourierSeriesMaxwell:260}
\begin{aligned}
\hat{F}_{0} &= C_{0} \\
\hat{F}_{\Bk} &= \exp\left(i c \Bk t \right) C_{\Bk} \\
\end{aligned}
\end{equation}

Here \(C_{\Bk}\) is an undetermined STA bivector.  For now we keep this undetermined coefficient on the right hand side of the exponential since no demonstration that it commutes with a factor of the form \(\exp(i\Bk\phi)\).  Substitution back into our assumed solution sum we have a solution to Maxwell's equation, in terms of a set of as yet undetermined (bivector) coefficients

\begin{equation}\label{eqn:fourierSeriesMaxwell:280}
\begin{aligned}
F(\Bx,t) = C_0 + \sum_{\Bk \ne 0} \exp\left(i c \Bk t \right) C_{\Bk} \exp(-i \Bk \cdot \Bx )
\end{aligned}
\end{equation}

The special case of \(\Bk = 0\) is now seen to be not so special and can be brought into the sum.

\begin{equation}\label{eqn:fourierSeriesMaxwell:300}
\begin{aligned}
F(\Bx,t) = \sum_{\Bk} \exp\left(i c \Bk t \right) C_{\Bk} \exp(-i \Bk \cdot \Bx )
\end{aligned}
\end{equation}

We can also
take advantage of the bivector nature of \(C_{\Bk}\), which implies the complex exponential can commute to the left, since the two fold commutation with the pseudoscalar with change sign twice.
%  A similar right commutation of the \(i\Bk\) exponential cannot be justified, and without more thought I am unsure if it can be allowed?

\begin{equation}\label{eqn:fourier_series_maxwell:undetermined}
\begin{aligned}
F(\Bx,t) = \sum_{\Bk}
\exp\left(i \Bk c t \right)
\exp\left(-i \Bk \cdot \Bx \right)
C_{\Bk}
\end{aligned}
\end{equation}

\subsection{Solution as time evolution of initial field}

Now, observe the form of this sum for \(t=0\).  This is

\begin{equation}\label{eqn:fourierSeriesMaxwell:320}
\begin{aligned}
F(\Bx,0)
&= \sum_{\Bk} C_{\Bk} \exp(-i \Bk \cdot \Bx ) \\
\end{aligned}
\end{equation}

So, the \(C_k\) coefficients are precisely the Fourier coefficients of \(F(\Bx,0)\).  This is to be expected having repeatedly seen similar results in the Fourier transform treatments of
\chapcite{PJfourierMaxwellSecondOrder}, \chapcite{PJfirstOrderMaxwell}, and \chapcite{PJ4dFourier}.
We then have an equation for the complete time evolution of any spatially periodic electrodynamic field in terms of the field value at all points in the region at some initial time.  Summarizing so far this is

\begin{equation}\label{eqn:fourierSeriesMaxwell:340}
\begin{aligned}
F(\Bx,t) &= \sum_{\Bk} \exp\left(i c \Bk t \right)
C_{\Bk}
\exp(-i \Bk \cdot \Bx) \\
C_{\Bk}
&= \inv{V} \int F(\Bx', 0) \exp\left( i\Bk \cdot \Bx' \right) d^3 x'
\end{aligned}
\end{equation}

Regrouping slightly we can write this as a convolution with a Fourier kernel (a Green's function).  That is

\begin{equation}\label{eqn:fourier_series_maxwell:bivectorSolNonGreens}
\begin{aligned}
F(\Bx,t) &= \inv{V} \int \sum_{\Bk} \exp\left( i \Bk ct \right) \exp\left( i \Bk \cdot (\Bx' - \Bx) \right) F(\Bx', 0) d^3 x'
\end{aligned}
\end{equation}

Or
\begin{equation}\label{eqn:fourier_series_maxwell:bivectorSolution}
\begin{aligned}
F(\Bx,t) &= \int G(\Bx - \Bx', t) F(\Bx', 0) d^3 x' \\
G(\Bx,t) &= \inv{V} \sum_{\Bk} \exp\left( i \Bk ct \right) \exp\left( -i \Bk \cdot \Bx \right)
\end{aligned}
\end{equation}

Okay, that is cool.  We have now got the basic periodicity result directly from Maxwell's equation in one shot.  No need to drop down to
potentials, or even the separate electric or magnetic components of our field \(F = \bcE + i \bcH\).

\subsection{Prettying it up?  Questions of commutation}

Now, it is tempting here to write
\eqnref{eqn:fourier_series_maxwell:undetermined}
as a single exponential

% k = kcappa g_0 |k|
% kcap g_0 = kcappa
\begin{equation}\label{eqn:fourier_series_maxwell:isItValid}
\begin{aligned}
F(\Bx,t)
%&= \sum_{\Bk} e^{i \Abs{\Bk}( \kcap c t - \kcap \cdot \Bx)} C_{\Bk} \\
&= \sum_{\Bk} \exp\left(i \Bk c t - i\Bk \cdot \Bx \right) C_{\Bk} \quad\quad \mbox{VALID?}
\end{aligned}
\end{equation}

This would probably allow for a prettier four vector form in terms of \(x = x^\mu \gamma_\mu\) replacing the separate \(\Bx\) and \(x^0 = ct\) terms.
However,
such a grouping is not allowable unless one first demonstrates that \(e^{i \Bu }\), and \(e^{i \alpha }\), for spatial vector \(\Bu\) and scalar \(\alpha\) commute!

To demonstrate that this is in fact the case
note that exponential
of this dual spatial vector can be written

\begin{equation}\label{eqn:fourierSeriesMaxwell:360}
\begin{aligned}
\exp( i \Bu ) &= \cos( \Bu ) + i \sin( \Bu ) \\
\end{aligned}
\end{equation}

This spatial vector cosine, \(\cos(\Bu)\), is a scalar (even powers only), and our sine, \(\sin(\Bu) \propto \Bu\), is a spatial vector in the direction of \(\Bu\) (odd powers leaves a vector times a scalar).  Spatial vectors commute with \(i\) (toggles sign twice percolating its way through), therefore pseudoscalar exponentials also commute with \(i\).

This will simplify a lot, and it shows that \eqnref{eqn:fourier_series_maxwell:isItValid} is in fact a valid representation.

Now, there is one more question of commutation here.  Namely, does a dual spatial vector exponential commute with the field itself
(or equivalently, one of the Fourier coefficients).

Expanding such a product and attempting term by term commutation should show

\begin{equation}\label{eqn:fourierSeriesMaxwell:380}
\begin{aligned}
e^{i\Bu} F
&= (\cos \Bu + i\sin\Bu) (\bcE + i\bcH) \\
&= i\sin\Bu (\bcE + i\bcH) + (\bcE + i\bcH) \cos\Bu \\
&= i (\sin\Bu) \bcE - (\sin\Bu) \bcH + F \cos\Bu \\
&= i (-\bcE \sin\Bu + 2 \bcE \cdot \sin\Bu ) + (\bcH \sin\Bu - 2 \bcH \cdot \sin\Bu ) + F \cos\Bu \\
&= 2 \sin\Bu \cdot (\bcE - \bcH) + F (\cos\Bu -i\sin\Bu) \\
\end{aligned}
\end{equation}

That is
\begin{equation}\label{eqn:fourier_series_maxwell:anticommutes}
\begin{aligned}
e^{i\Bu} F &= 2 \sin\Bu \cdot (\bcE - \bcH) + F e^{-i\Bu}
\end{aligned}
\end{equation}

This exponential has one anticommuting term, but also has a scalar component introduced by the portions of the electric
and magnetic fields that are colinear with the spatial vector \(\Bu\).

%
%Would this look any tidier in terms of unit wave number vector \(\Bk = \Abs{\Bk} \kcap\)?  Let us see
%
%\begin{align*}
%F(\Bx,t) = \sum_{\Bk} \exp\left(-i \Abs{\Bk}(\kcap \cdot \Bx - \kcap c t) \right) C_{\Bk}
%\end{align*}
%FIXME:EXP: above.
%
%Perhaps not.
%
%One thing we may do however, is perform a summation swap and sum over all
%triplets \(-\Bk\) instead, with a redefition of the undetermined
%coefficients \(C_{\Bk}\) as \(C_{-\Bk}\) (incorporating the effects of that sign swap into the value of these coefficients).  This takes the sign out of the exponential and pretties it up slightly.
%
%\begin{align*}
%F(\Bx,t) = \sum_{\Bk} e^{i \Bk \cdot \Bx - i\Bk c t} C_{\Bk}
%\end{align*}
%FIXME:EXP: above.
%
%There was also the notational trick noticed in the Fourier transform treatment
%where a conversion of these separate space and time exponential factors into a single
%four vector dot product was possible.  That should work here a bit better than in the Fourier transform case.
%
%FIXME: detail that here.  Want to see what that implies for a Lorentz transformation of the field.

\section{Field Energy and momentum}

Given that we have the same structure for our four vector potential solutions as the complete bivector field, it does not appear that there is much
reason to work in the second order quantities.  Following Bohm we should now be prepared to express the field energy density and
momentum density in terms of the Fourier coefficients, however unlike Bohm, let us try this using the first order
solutions found above.

In CGS units (see \chapcite{PJrayleighJeans} for verification) these field energy and momentum densities (Poynting vector \(\BP\)) are, respectively

\begin{equation}\label{eqn:fourierSeriesMaxwell:400}
\begin{aligned}
E &= \inv{8\pi}
\lr{ {\bcE}^2 + \bcH^2 } \\
\BP &= \inv{4\pi} (\bcE \cross \bcH )
\end{aligned}
\end{equation}

Given that we have a complete field equation without an explicit separation of electric and magnetic components, perhaps this
is easier to calculate from the stress energy four vector for energy/momentum.  In CGS units this must be

\begin{equation}\label{eqn:fourierSeriesMaxwell:420}
\begin{aligned}
T(\gamma_0) &= \inv{8\pi} F \gamma_0 \tilde{F}
\end{aligned}
\end{equation}

An expansion of this to verify the CGS conversion seems worthwhile.

\begin{equation}\label{eqn:fourierSeriesMaxwell:440}
\begin{aligned}
T(\gamma_0)
&= \inv{8\pi} F \gamma_0 \tilde{F} \\
&= \frac{-1}{8\pi} (\bcE + i\bcH) \gamma_0 (\bcE + i\bcH) \\
&= \frac{1}{8\pi} (\bcE + i\bcH) (\bcE - i\bcH) \gamma_0 \\
&= \frac{1}{8\pi} \left( \bcE^2 - (i\bcH)^2 + i(\bcH \bcE - \bcE \bcH) \right) \gamma_0 \\
&= \frac{1}{8\pi} \left( \bcE^2 + \bcH^2 + 2 i^2 \bcH \cross \bcE \right) \gamma_0 \\
&= \frac{1}{8\pi} \left( \bcE^2 + \bcH^2 \right) \gamma_0 + \inv{4 \pi} \left(\bcE \cross \bcH \right) \gamma_0 \\
\end{aligned}
\end{equation}

Good, as expected we have

\begin{equation}\label{eqn:fourierSeriesMaxwell:460}
\begin{aligned}
E &= T(\gamma_0) \cdot \gamma_0 \\
\BP &= T(\gamma_0) \wedge \gamma_0
\end{aligned}
\end{equation}

FIXME: units here for \(\BP\) are off by a factor of \(c\).  This does not matter
so much in four vector form \(T(\gamma_0)\) where the units naturally take care
of themselves.

Okay, let us apply this to our field \eqnref{eqn:fourier_series_maxwell:bivectorSolNonGreens}, and try to percolate the \(\gamma_0\) through all the terms of \(\tilde{F}(\Bx,t)\)

\begin{equation}\label{eqn:fourierSeriesMaxwell:480}
\begin{aligned}
\gamma_0 \tilde{F}(\Bx,t)
&= -\gamma_0 F(\Bx,t) \\
&= -\gamma_0 \inv{V} \int \sum_{\Bk} \exp\left( i \Bk ct \right) \exp\left( i \Bk \cdot (\Bx' -\Bx) \right) F(\Bx', 0) d^3 x' \\
\end{aligned}
\end{equation}

Taking one factor at a time

\begin{equation}\label{eqn:fourierSeriesMaxwell:500}
\begin{aligned}
\gamma_0 \exp\left( i \Bk ct \right)
&= \gamma_0 (\cos\left( \Bk ct \right) + i \sin\left( \Bk ct \right) ) \\
&= \cos\left( \Bk ct \right) \gamma_0 - i \gamma_0 \sin\left( \Bk ct \right) ) \\
&= \cos\left( \Bk ct \right) \gamma_0 - i \sin\left( \Bk ct \right) ) \gamma_0 \\
&= \exp\left( -i \Bk ct \right) \gamma_0
\end{aligned}
\end{equation}


Next, percolate \(\gamma_0\) through the pseudoscalar exponential.

\begin{equation}\label{eqn:fourierSeriesMaxwell:520}
\begin{aligned}
\gamma_0 e^{i\phi}
&= \gamma_0 (\cos\phi + i \sin\phi) \\
&= \cos\phi \gamma_0 - i \gamma_0 \sin\phi \\
&= e^{-i\phi} \gamma_0
\end{aligned}
\end{equation}

Again, the percolation produces a conjugate effect.  Lastly, as noted previously \(F\) commutes with \(i\).  We have therefore

\begin{equation}\label{eqn:fourierSeriesMaxwell:540}
\begin{aligned}
\tilde{F}(\Bx,t) \gamma_0 {F}(\Bx,t) \gamma_0
&=
\frac{1}{V^2} \int \sum_{\Bk,\Bm}
F(\Ba, 0)
e^{i \Bk \cdot (\Ba -\Bx) }
e^{ i \Bk ct }
e^{ -i \Bm ct } e^{ -i \Bm \cdot (\Bb -\Bx) } F(\Bb, 0) d^3 a d^3 b \\
&= \frac{1}{V^2} \int \sum_{\Bk,\Bm} F(\Ba, 0) e^{ i \Bk \cdot \Ba -i \Bm \cdot \Bb + i (\Bk -\Bm) ct -i (\Bk - \Bm) \cdot \Bx } F(\Bb, 0) d^3 a d^3 b \\
&= \frac{1}{V^2} \int \sum_{\Bk} F(\Ba, 0) F(\Bb, 0) e^{ i \Bk \cdot (\Ba - \Bb) } d^3 a d^3 b \\
&\quad + \frac{1}{V^2} \int \sum_{\Bk \ne \Bm} F(\Ba, 0) e^{ i \Bk \cdot \Ba -i \Bm \cdot \Bb + i (\Bk -\Bm) ct -i (\Bk - \Bm) \cdot \Bx } F(\Bb, 0) d^3 a d^3 b \\
&= \frac{1}{V^2} \int \sum_{\Bk} F(\Ba, 0) F(\Bb, 0) e^{ i \Bk \cdot (\Ba - \Bb) } d^3 a d^3 b \\
&\quad + \frac{1}{V^2} \int \sum_{\Bm, \Bk \ne 0} F(\Ba, 0) e^{
i \Bm \cdot (\Ba -\Bb)
+i \Bk \cdot (\Ba -\Bx)
+ i \Bk ct
} F(\Bb, 0) d^3 a d^3 b \\
\end{aligned}
\end{equation}

Hmm.  Messy.  The scalar bits of the above are our energy.  We have a \(F^2\) like term in the first integral (like the Lagrangian density), but it is at different points, and
we have to integrate those with a sort of vector convolution.  Given the reciprocal relationships between convolution and multiplication moving between the frequency and time domains in Fourier transforms I had expect that this first integral can somehow be turned into the sum of the squares of all the Fourier coefficients

\begin{equation}\label{eqn:fourierSeriesMaxwell:560}
\begin{aligned}
\sum_{\Bk} C_{\Bk}^2
\end{aligned}
\end{equation}

which is very much like a discrete version of the Rayleigh energy theorem as derived in \chapcite{PJqmFourier}, and is in this case
a constant (not a function of time or space) and is dependent on only the initial field.
That would mean that the remainder is the Poynting vector,
which looks reasonable since it has the appearance of being somewhat antisymmetric.

Hmm, having mostly figured it out without doing the math in this case, the answer pops out.  This first integral can be separated cleanly since the pseudoscalar
exponentials commute with the bivector field.  We then have

\begin{equation}\label{eqn:fourierSeriesMaxwell:580}
\begin{aligned}
\frac{1}{V^2} &\int \sum_{\Bk} F(\Ba, 0) F(\Bb, 0) e^{ i \Bk \cdot (\Ba - \Bb) } d^3 a d^3 b \\
&= \frac{1}{V} \int \sum_{\Bk} F(\Ba, 0) e^{ i \Bk \cdot \Ba } d^3 a \int F(\Bb, 0) e^{ -i \Bk \cdot \Bb } d^3 b \\
&= \sum_{\Bk} \hat{F}_{-\Bk} \hat{F}_{\Bk} \\
\end{aligned}
\end{equation}

A side note on subtle notational sneakiness here.  In the assumed series
solution of \eqnref{eqn:fourier_series_maxwell:assumed} \(\hat{F}_{\Bk}(t)\) was the \(\Bk\) Fourier coefficient of \(F(\Bx,t)\), whereas here the use of \(\hat{F}_{\Bk}\) has been used to denote the \(\Bk\) Fourier coefficient of \(F(\Bx,0)\).
An alternative considered and rejected was something messier like \(\widehat{F(t=0)}_{\Bk}\), or the use of the original, less physically significant, \(C_{\Bk}\) coefficients.

The second term could also use a simplification, and it looks like we can separate these \(\Ba\) and \(\Bb\) integrals too

\begin{equation}\label{eqn:fourierSeriesMaxwell:600}
\begin{aligned}
\frac{1}{V^2} &\int \sum_{\Bm, \Bk \ne 0} F(\Ba, 0) e^{
i \Bm \cdot (\Ba -\Bb)
+i \Bk \cdot (\Ba -\Bx)
+ i \Bk ct
} F(\Bb, 0) d^3 a d^3 b \\
&=\frac{1}{V} \int \sum_{\Bm, \Bk \ne 0} F(\Ba, 0) e^{ i (\Bm + \Bk) \cdot \Ba } d^3 a
e^{ i \Bk ct -i \Bk \cdot \Bx }
\inv{V} \int F(\Bb, 0)
e^{-i \Bm \cdot \Bb}
d^3 b
 \\
&= \sum_{\Bm} \sum_{\Bk \ne 0} \hat{F}_{-\Bm -\Bk} e^{ i \Bk ct -i \Bk \cdot \Bx } \hat{F}_{\Bm} \\
\end{aligned}
\end{equation}

Making an informed guess that the first integral is a scalar, and the second is a spatial vector, our energy and momentum densities (Poynting vector) respectively are

\begin{equation}\label{eqn:fourier_series_maxwell:energyMomentum}
\begin{aligned}
U &
\questionEquals
 \inv{8 \pi} \sum_{\Bk} \hat{F}_{-\Bk} \hat{F}_{\Bk} \\
\BP &
\questionEquals
 \inv{8 \pi} \sum_{\Bm} \sum_{\Bk \ne 0} \hat{F}_{-\Bm -\Bk} e^{ i \Bk ct -i \Bk \cdot \Bx } \hat{F}_{\Bm}
\end{aligned}
\end{equation}

Now that much of the math is taken care of, more consideration about the physics implications is required.  In particular, relating these
abstract quantities to the frequencies and the harmonic oscillator model as Bohm did is desirable (that was the whole point of the exercise).

On the validity of \eqnref{eqn:fourier_series_maxwell:energyMomentum}, it is not unreasonable to expect that
\(\PDi{t}{U} = 0\), and \(\spacegrad \cdot \BP = 0\) separately in these current free conditions from the energy momentum conservation relation

\begin{equation}\label{eqn:fourierSeriesMaxwell:620}
\begin{aligned}
\PD{t}{}\frac{1}{8\pi} \left(\bcE^2 + \bcH^2\right) + \inv{4\pi} \spacegrad \cdot (\bcE \cross \bcH) &= -\bcE \cdot \Bj
\end{aligned}
\end{equation}

Note that an SI derivation of this relation can be found in \chapcite{PJpoynting}.  So it therefore makes some sense that all the time dependence ends
up in what has been labeled as the Poynting vector.  A proof that the spatial divergence of this quantity is zero would help validate
the guess made (or perhaps invalidate it).

Hmm.  Again on the validity of identifying the first sum with the energy.  It does not appear to work for the \(\Bk = 0\) case, since that gives you

\begin{equation}\label{eqn:fourierSeriesMaxwell:640}
\begin{aligned}
\inv{8 \pi V^2} \int F(\Ba, 0) F(\Bb, 0) d^3 a d^3b
\end{aligned}
\end{equation}

That is only a scalar if the somehow all the non-scalar parts of that product somehow magically cancel out.
Perhaps it is true that the second sum has no scalar part, and if that is the case one would have

\begin{equation}\label{eqn:fourierSeriesMaxwell:660}
\begin{aligned}
U
\questionEquals
 \inv{8 \pi} \sum_{\Bk} \gpgradezero{\hat{F}_{-\Bk} \hat{F}_{\Bk}} \\
\end{aligned}
\end{equation}

An explicit calculation of \(T(\gamma_0) \cdot \gamma_0\) is probably justified
to discarding all other grades, and get just the energy.

So, instead of optimistically hoping that the scalar and spatial vector terms will automatically fall out, it appears
that we have to explicitly calculate the dot and wedge products, as in

\begin{equation}\label{eqn:fourierSeriesMaxwell:680}
\begin{aligned}
U &= -\frac{1}{16\pi}( F \gamma_0 F \gamma_0 + \gamma_0 F \gamma_0 F ) \\
\BP &= -\frac{1}{16\pi}( F \gamma_0 F \gamma_0 - \gamma_0 F \gamma_0 F )
\end{aligned}
\end{equation}

and then substitute our Fourier series solution for \(F\) to get the desired result.  This appears to be getting more complex instead of
less so unfortunately, but hopefully following this to a logical conclusion will show in retrospect a faster way to the desired result.
A first attempt to do so shows that we have to return to our assumed Fourier solution and revisit some of the assumptions made.

\section{Return to the assumed solutions to Maxwell's equation}

An initial attempt to expand \eqnref{eqn:fourier_series_maxwell:energyMomentum} properly
given the Fourier specification of the Maxwell solution
gets into trouble.  Consideration of some special cases for specific values
of \(\Bk\) shows that there is a problem with the grades of the solution.

Let us reexamine the assumed solution of \eqnref{eqn:fourier_series_maxwell:bivectorSolNonGreens} with respect to grade
\begin{equation}\label{eqn:fourierSeriesMaxwell:700}
\begin{aligned}
F(\Bx,t) &= \inv{V} \int \sum_{\Bk} \exp\left( i \Bk ct \right) \exp\left( i \Bk \cdot (\Bx' - \Bx) \right) F(\Bx', 0) d^3 x'
\end{aligned}
\end{equation}

For scalar Fourier approximations we are used to the ability to select a subset of the Fourier terms to approximate the field, but
except for the \(\Bk = 0\) term it appears that a term by term approximation actually introduces noise in the form of non-bivector grades.

Consider first the \(\Bk = 0\) term.  This gives us a first order approximation of the field which is

\begin{equation}\label{eqn:fourierSeriesMaxwell:720}
\begin{aligned}
F(\Bx,t) &\approx \inv{V} \int F(\Bx', 0) d^3 x'
\end{aligned}
\end{equation}

As summation is grade preserving this spatial average of the initial field conditions does have the required grade as desired.
Next consider a non-zero Fourier term such as \(\Bk = \{1,0,0\}\).  For this single term approximation of the field let us write
out the field term as

\begin{equation}\label{eqn:fourierSeriesMaxwell:740}
\begin{aligned}
F_{\Bk}(\Bx,t)
&= \inv{V} \int e^{ i \kcap \Abs{\Bk} ct + i \Bk \cdot (\Bx' - \Bx) } (\bcE(\Bx', 0) + i\bcH(\Bx', 0)) d^3 x'
\end{aligned}
\end{equation}

Now, let us expand the exponential.  This was shorthand for the product of the exponentials, which seemed to be a reasonable
shorthand since we showed they commute.  Expanded out this is

\begin{equation}\label{eqn:fourierSeriesMaxwell:760}
\begin{aligned}
\exp&( i \kcap \Abs{\Bk} ct + i \Bk \cdot (\Bx' - \Bx) ) \\
&= (\cos( {\Bk} ct ) + i \kcap \sin( \Abs{\Bk} ct ))( \cos( \Bk \cdot (\Bx' - \Bx) ) + i \sin(\Bk \cdot (\Bx' - \Bx) )) \\
\end{aligned}
\end{equation}

For ease of manipulation write \(\Bk \cdot (\Bx' - \Bx) = k \Delta x\), and \(\Bk c t = \Bomega t\), we have

\begin{equation}\label{eqn:fourierSeriesMaxwell:780}
\begin{aligned}
%(\cos( \Bomega t ) + i \sin( \Bomega t ))( \cos( k \Delta x ) ) + i \sin( k \Delta x ) ))
\exp( i \Bomega t + i k \Delta x )
&= \cos( \Bomega t ) \cos( k \Delta x ) +i \cos( \Bomega t ) \sin( k \Delta x ) \\
&+i \sin( \Bomega t ) \cos( k \Delta x ) - \sin( \Bomega t ) \sin( k \Delta x )  \\
\end{aligned}
\end{equation}

Note that \(\cos(\Bomega t)\) is a scalar, whereas \(\sin(\Bomega t)\) is a (spatial) vector in the direction of \(\Bk\).
Multiplying this out with the initial time field \(F(\Bx',0) = \bcE(\Bx', 0) + i\bcH(\Bx',0) = \bcE' + i\bcH'\) we can separate into grades.

\begin{equation}\label{eqn:fourierSeriesMaxwell:800}
\begin{aligned}
\exp&( i \Bomega t + i k \Delta x ) (\bcE' + i\bcH') \\
&= \cos( \Bomega t ) (\bcE' \cos( k \Delta x ) -\bcH' \sin( k \Delta x ) ) +  \sin( \Bomega t ) \cross ( \bcH' \sin( k \Delta x ) - \bcE' \cos( k \Delta x ) ) \\
&+i \cos( \Bomega t ) (\bcE' \sin( k \Delta x ) + \bcH' \cos( k \Delta x ) ) -i \sin( \Bomega t ) \cross (\bcE' \sin( k \Delta x ) + \bcH' \cos( k \Delta x ) ) \\
&-  \sin( \Bomega t ) \cdot (\bcE' \sin( k \Delta x ) +\bcH' \cos( k \Delta x ) ) \\
&+i( \sin( \Bomega t ) \cdot (\bcE' \cos( k \Delta x ) - \bcH' \sin( k \Delta x )) \\
\end{aligned}
\end{equation}

The first two lines, once integrated, produce the electric and magnetic fields, but the last two are rogue scalar and pseudoscalar terms.  These
are allowed in so far as they are still solutions to the differential equation, but do not have the desired physical meaning.

If one explicitly sums over pairs of \(\{\Bk,-\Bk\}\) of index triplets then some cancellation occurs.  The cosine cosine products and sine sine products double
and the sine cosine terms cancel.  We therefore have

\begin{equation}\label{eqn:fourierSeriesMaxwell:820}
\begin{aligned}
\inv{2} &\exp( i \Bomega t + i k \Delta x ) (\bcE' + i\bcH') \\
&= \cos( \Bomega t ) \bcE' \cos( k \Delta x ) +  \sin( \Bomega t ) \cross \bcH' \sin( k \Delta x ) \\
&+i \cos( \Bomega t ) \bcH' \cos( k \Delta x ) -i \sin( \Bomega t ) \cross \bcE' \sin( k \Delta x ) \\
&-  \sin( \Bomega t ) \cdot \bcE' \sin( k \Delta x ) \\
&-i \sin( \Bomega t ) \cdot \bcH' \sin( k \Delta x ) \\
&= (\bcE' + i\bcH') \cos( \Bomega t ) \cos( k \Delta x )
 -i \sin( \Bomega t ) \cross (\bcE' + i \bcH') \sin( k \Delta x ) \\
&-  \sin( \Bomega t ) \cdot (\bcE'+i\bcH) \sin( k \Delta x ) \\
\end{aligned}
\end{equation}

Here for grouping purposes \(i\) is treated as a scalar, which should be justifiable in this specific case.  A final grouping produces

\begin{equation}\label{eqn:fourierSeriesMaxwell:840}
\begin{aligned}
\inv{2} \exp( i \Bomega t + i k \Delta x ) (\bcE' + i\bcH')
&= (\bcE' + i\bcH') \cos( \Bomega t ) \cos( k \Delta x )  \\
&-i \kcap \cross (\bcE' + i \bcH') \sin( \Abs{\Bomega} t ) \sin( k \Delta x ) \\
&-  \sin( \Bomega t ) \cdot (\bcE'+i\bcH') \sin( k \Delta x ) \\
\end{aligned}
\end{equation}

Observe that despite the grouping of the summation over the pairs of complementary sign index triplets we still have a pure scalar and pure pseudoscalar
term above.  Allowable by the math since the differential equation had no way of encoding the grade of the desired solution.  That only came from the
initial time specification of \(F(\Bx',0)\), but that is not enough.

Now, from above, we can see that one way to reconcile this grade requirement is to require both \(\kcap \cdot \bcE' = 0\), and \(\kcap \cdot \bcH' = 0\).
How can such a requirement make sense given that \(\Bk\) ranges over all directions in space, and that both \(\bcE'\) and \(\bcH'\) could conceivably
range over many different directions in the volume of periodicity.

With no other way out, it seems that we have to impose two requirements, one on the allowable wavenumber vector directions (which in turn means we can only
pick specific orientations of the Fourier volume), and another on the field directions themselves.  The electric and magnetic fields must therefore
be directed only perpendicular to the wave number vector direction.  Wow, that is a pretty severe implication following strictly from a grade requirement!

Thinking back to \eqnref{eqn:fourier_series_maxwell:anticommutes}, it appears that an implication of this is that we have

\begin{equation}\label{eqn:fourierSeriesMaxwell:860}
\begin{aligned}
e^{i\Bomega t} F(\Bx',0) &= F(\Bx',0) e^{-i\Bomega t}
\end{aligned}
\end{equation}

Knowing this is a required condition should considerably simplify the energy and momentum questions.

%\section{FIXME}
%
%Caught myself in these notes abusing notation and probably made mistakes by combining exponentials that probably do not commute into single argument exponentials.  Go back and review all other recent previous Fourier treatments and check for and fix this if neccessary
% ... TURNS OUT THEY DID COMMUTE .... should still review other work.

   %
% Copyright � 2012 Peeter Joot.  All Rights Reserved.
% Licenced as described in the file LICENSE under the root directory of this GIT repository.
%
%
%
\mychapter{Plane wave Fourier series solutions to the Maxwell vacuum equation}
\label{chap:PJplaneWave}
\index{Maxwell equation!Fourier series}
%\date{Feb 08, 2009.  planewave.tex}
\section{Motivation}
In \chapcite{PJFourierVacuum} an exploration of spatially periodic solutions to the electrodynamic vacuum equation was performed using a multivector formulation
of a 3D Fourier series.
Here a summary of the results obtained will be presented in a more
coherent fashion, followed by an attempt to build on them.
In particular a complete
description of the field energy and momentum is desired.

A conclusion from the first analysis was that the
orientation of both the electric and magnetic field components
must be perpendicular to the angular velocity and wave number vectors
within the entire spatial volume.  This was a requirement for the field
solutions to retain a bivector grade (STA/Dirac basis).

Here a specific orientation of the Fourier volume so that two of the axis
lie in the direction of the initial time electric and magnetic fields will be
used.  This is expected to simplify the treatment.

Also note that having obtained some results in a first attempt hindsight
now allows a few choices of variables that will be seen to be appropriate.
The natural motivation for any such choices can be found in the initial
treatment.
\subsection{Notation}
Conventions, definitions, and notation used here will largely follow
\chapcite{PJFourierVacuum}.
Also of possible aid in that document is a table of symbols and their definitions.
\section{A concise review of results}
\subsection{Fourier series and coefficients}
A notation for a 3D Fourier series for a spatially periodic function and its Fourier coefficients was developed
\begin{equation}\label{eqn:planewave:20}
\begin{aligned}
f(\Bx) &= \sum_{\Bk} \hat{f}_{\Bk} e^{ - i \Bk \cdot \Bx } \\
\hat{f}_{\Bk} &= \inv{V} \int f(\Bx) e^{ i \Bk \cdot \Bx } d^3 x
\end{aligned}
\end{equation}
%
In the vector context \(\Bk\) is
%
\begin{equation}\label{eqn:planewave:40}
\begin{aligned}
\Bk = 2 \pi \sum_m \sigma^m \frac{k_m}{\lambda_m}
\end{aligned}
\end{equation}
%
Where \(\lambda_m\) are the dimensions of the volume of integration,
\(V = \lambda_1 \lambda_2 \lambda_3\) is the volume, and
in an index context \(\Bk = \{k_1, k_2, k_3\}\) is a triplet of integers,
positive, negative or zero.

\subsection{Vacuum solution and constraints}

We want to find (STA) bivector solutions \(F\) to the vacuum Maxwell equation
%
\begin{equation}\label{eqn:planewave:maxwell}
\begin{aligned}
\grad F = \gamma_0 (\partial_0 + \spacegrad) F = 0
\end{aligned}
\end{equation}
%
We start by assuming a Fourier series solution of the form
%
\begin{equation}\label{eqn:planewave:60}
\begin{aligned}
F(\Bx,t) &= \sum_{\Bk} \hat{F}_{\Bk}(t) e^{-i \Bk \cdot \Bx}
\end{aligned}
\end{equation}
%
For a solution term by term identity is required
%
\begin{equation}\label{eqn:planewave:80}
\begin{aligned}
\PD{t}{} \hat{F}_{\Bk}(t) e^{-i \Bk \cdot \Bx}
&= -c \sigma^m \hat{F}_{\Bk}(t) \PD{x^m}{} \exp\left(-i 2 \pi \frac{k_j x^j}{ \lambda_j}\right) \\
&= i c \Bk \hat{F}_{\Bk}(t) e^{-i \Bk \cdot \Bx}
\end{aligned}
\end{equation}
%
With \(\Bomega = c \Bk\), we have a simple first order single variable differential equation
%
\begin{equation}\label{eqn:planewave:100}
\begin{aligned}
\hat{F}_{\Bk}'(t) = i \Bomega \hat{F}_{\Bk}(t)
\end{aligned}
\end{equation}
%
with solution
%
\begin{equation}\label{eqn:planewave:120}
\begin{aligned}
\hat{F}_{\Bk}(t) = e^{i \Bomega t} \hat{F}_{\Bk}
\end{aligned}
\end{equation}
%
Here, the constant was written as \(\hat{F}_{\Bk}\) given prior knowledge that this is will be the Fourier coefficient of the
initial time field.  Our assumed solution is now
%
\begin{equation}\label{eqn:planewave:assummedSolutionNewStartingPoint}
\begin{aligned}
F(\Bx,t) &= \sum_{\Bk} e^{i \Bomega t} \hat{F}_{\Bk} e^{-i \Bk \cdot \Bx}
\end{aligned}
\end{equation}
%
Observe that for \(t = 0\), we have
%
\begin{equation}\label{eqn:planewave:140}
\begin{aligned}
F(\Bx,0) &= \sum_{\Bk} \hat{F}_{\Bk} e^{-i \Bk \cdot \Bx}
\end{aligned}
\end{equation}
%
which is confirmation of the Fourier coefficient role of \(\hat{F}_{\Bk}\)
%
\begin{equation}\label{eqn:planewave:160}
\begin{aligned}
\hat{F}_{\Bk} &= \inv{V} \int F(\Bx', 0) e^{ i \Bk \cdot \Bx' } d^3 x'
\end{aligned}
\end{equation}
%
\begin{equation}\label{eqn:planewave:solutionFromInitialConditions}
\begin{aligned}
F(\Bx,t) &= \inv{V} \sum_{\Bk} \int e^{i \Bomega t} F(\Bx', 0) e^{i \Bk \cdot (\Bx'-\Bx) } d^3 x'
\end{aligned}
\end{equation}
%
It is straightforward to show that \(F(\Bx, 0)\), and pseudoscalar exponentials commute.  Specifically we have
%
\begin{equation}\label{eqn:planewave:180}
\begin{aligned}
F e^{ i \Bk \cdot \Bx } = e^{ i \Bk \cdot \Bx } F
\end{aligned}
\end{equation}
%
This follows from the (STA) bivector nature of \(F\).

Another commutativity relation of note is between our time phase exponential and the pseudoscalar exponentials.  This one is also straightforward to show
and will not be done again here
%
\begin{equation}\label{eqn:planewave:200}
\begin{aligned}
e^{ i \Bomega t} e^{ i \Bk \cdot \Bx } = e^{ i \Bk \cdot \Bx } e^{ i \Bomega t}
\end{aligned}
\end{equation}
%
Lastly, and most importantly of the commutativity relations,
it was also found that the initial field \(F(\Bx,0)\) must have both electric and magnetic field components perpendicular to all \(\Bomega \propto \Bk\) at all points
\(\Bx\) in the integration volume.
This was because the vacuum Maxwell equation \eqnref{eqn:planewave:maxwell} by itself does not impose any grade requirement on the solution in isolation.  An
additional requirement that the solution have bivector only values imposes this inherent planar nature in a charge free region, at least for solutions
with spatial periodicity.  Some revisiting of previous Fourier transform solutions attempts at the vacuum equation is required since similar constraints are
expected there too.

The planar constraint can be expressed in terms of dot products of the field components, but an alternate way of expressing the same thing was seen to be
a statement of conjugate commutativity between this dual spatial vector exponential and the complete field
%
\begin{equation}\label{eqn:planewave:restriction}
\begin{aligned}
e^{ i \Bomega t} F &= F e^{ -i \Bomega t}
\end{aligned}
\end{equation}
%
The set of Fourier coefficients considered in the sum must be restricted to those values that \eqnref{eqn:planewave:restriction} holds.  An effective
way to achieve this is to
pick a specific orientation of the coordinate system so the angular
velocity bivector is quantized in the same plane as the field.  This means that
the angular velocity takes on integer multiples \(k\) of this value
%
\begin{equation}\label{eqn:planewave:220}
\begin{aligned}
i \Bomega_k = 2 \pi i c k \frac{\Bsigma}{\lambda}
\end{aligned}
\end{equation}
%
Here \(\Bsigma\) is a unit vector describing the perpendicular to the plane of the field, or equivalently via a duality relationship \(i \Bsigma\) is a unit bivector with the same orientation as the field.

\subsection{Conjugate operations}

In order to tackle expansion of energy and momentum in terms of Fourier coefficients, some conjugation operations will be required.

Such a conjugation is found when computing electric and magnetic field components and also in the \(T(\gamma_0) \propto F \gamma_0 F\) energy
momentum four vector.  In both cases it involves products with \(\gamma_0\).

\subsection{Electric and magnetic fields}

From the total field one can
obtain the electric and magnetic fields via coordinates as in

%\begin{align*}
%F = E^m \sigma_m + i H^m \sigma_m
%\end{align*}
%
%From which we could write
\begin{equation}\label{eqn:planewave:240}
\begin{aligned}
%E^m &= F \cdot \sigma_m \\
%H^m &= (-i F) \cdot \sigma_m
\bcE &= \sigma^m (F \cdot \sigma_m) \\
\bcH &= \sigma^m ((-i F) \cdot \sigma_m)
\end{aligned}
\end{equation}
%
However, due to the conjugation effect of \(\gamma_0\)
(a particular observer's time basis vector)
on \(F\),
we can compute the electric and magnetic field components without resorting to coordinates
%
\begin{equation}\label{eqn:planewave:260}
\begin{aligned}
\bcE &= \inv{2}(F - \gamma_0 F \gamma_0 ) \\
\bcH &= \inv{2i}(F + \gamma_0 F \gamma_0 )
\end{aligned}
\end{equation}
%
Such a split is expected to show up when examining the energy and momentum of our Fourier expressed field in detail.
%Observe the conjugation effect of the multiplications with \(\gamma_0\).

\subsection{Conjugate effects on the exponentials}

Now, since \(\gamma_0\) anticommutes with \(i\) we have a conjugation operation on percolation of \(\gamma_0\) through the products of an exponential
%
\begin{equation}\label{eqn:planewave:280}
\begin{aligned}
\gamma_0 e^{i \Bk \cdot \Bx} = e^{-i \Bk \cdot \Bx} \gamma_0
\end{aligned}
\end{equation}
%
However, since \(\gamma_0\) also anticommutes with any spatial basis vector \(\sigma_k = \gamma_k \gamma_0\), we have for a dual spatial vector exponential
%
\begin{equation}\label{eqn:planewave:300}
\begin{aligned}
\gamma_0 e^{i \Bomega t} &= e^{i \Bomega t} \gamma_0
\end{aligned}
\end{equation}
%
We should now be armed to consider the energy momentum questions that were the desired goal of the initial treatment.

\section{Plane wave Energy and Momentum in terms of Fourier coefficients}

\subsection{Energy momentum four vector}
\index{energy momentum}

To obtain the energy component \(U\) of the energy momentum four
vector (given here in CGS units)
%
\begin{equation}\label{eqn:planewave:320}
\begin{aligned}
T(\gamma_0) &= \inv{8\pi} F \gamma_0 \tilde{F} = \frac{-1}{8\pi}(F \gamma_0 F)
\end{aligned}
\end{equation}
%
we want a calculation of the field energy for the plane wave solutions of Maxwell's equation
%
\begin{equation}\label{eqn:planewave:340}
\begin{aligned}
U
&= T(\gamma_0) \cdot \gamma_0 \\
&= -\inv{16 \pi} ( F \gamma_0 F \gamma_0 + \gamma_0 F \gamma_0 F )
\end{aligned}
\end{equation}
%
Given the
observed commutativity relationships, at
least some parts of this calculation can be performed by direct multiplication of
\eqnref{eqn:planewave:solutionFromInitialConditions} summed over two sets of wave
number vector indices as in.
%
\begin{equation}\label{eqn:planewave:360}
\begin{aligned}
F(\Bx,t)
&= \inv{V} \sum_{\Bk} \int e^{i \Bomega_k t + i \Bk \cdot (\Ba-\Bx) } F(\Ba, 0) d^3 a \\
%&= \inv{V} \sum_{\Bk} \int F(\Ba, 0) e^{-i \Bomega_k t + i \Bk \cdot (\Ba-\Bx) } d^3 a \\
&= \inv{V} \sum_{\Bm} \int e^{i \Bomega_m t + i \Bm \cdot (\Bb-\Bx) } F(\Bb, 0) d^3 b
%&= \inv{V} \sum_{\Bm} \int F(\Bb,0) e^{-i \Bomega_m t + i \Bm \cdot (\Bb-\Bx) } F(\Bb, 0) d^3 b
\end{aligned}
\end{equation}
%
However, this gets messy fast.  Looking for an alternate approach requires some mechanism for encoding the effect
of the \(\gamma_0\) sandwich on the Fourier coefficients of the field bivector.  It has been observed that this operation has a conjugate
effect.  The form of the stress energy four vector suggests that a natural conjugate definition will be
%
\begin{equation}\label{eqn:planewave:380}
\begin{aligned}
F^\dagger &= \gamma_0 \tilde{F} \gamma_0
\end{aligned}
\end{equation}
%
where \(\tilde{F}\) is the multivector reverse operation.

This notation for conjugation is in fact what
, for Quantum Mechanics, \citep{doran2003gap} calls the Hermitian adjoint.

In this form our stress energy vector is
%
\begin{equation}\label{eqn:planewave:400}
\begin{aligned}
T(\gamma_0) &= \inv{8 \pi} F F^\dagger \gamma_0
\end{aligned}
\end{equation}
%
While the trailing \(\gamma_0\) term here may look a bit out of place,
the energy density and the Poynting vector end up with a very complementary structure
%
\begin{equation}\label{eqn:planewave:420}
\begin{aligned}
U &= \inv{16 \pi} \left(F F^\dagger + (F F^\dagger)^{\tilde{}} \right) \\
\BP &= \inv{16 \pi c} \left(F F^\dagger - (F F^\dagger)^{\tilde{}} \right)
\end{aligned}
\end{equation}
%
Having this conjugate operation defined it can also be applied to the
spacetime split of the electric and the magnetic fields.  That can also now be written in a form
that calls out the inherent complex nature of the fields
%
\begin{equation}\label{eqn:planewave:440}
\begin{aligned}
\bcE &= \inv{2}(F + F^\dagger ) \\
\bcH &= \inv{2i}(F - F^\dagger )
\end{aligned}
\end{equation}
%
\subsection{Aside.  Applications for the conjugate in non-QM contexts}

Despite the existence of the QM notation, it does not appear used in the text or ptIII notes outside of that context.
For example,
in addition to the
stress energy tensor and the spacetime split of the fields, an additional non-QM example
where the conjugate operation could be used, is in the ptIII hout8 where Rotors that satisfy
%
\begin{equation}\label{eqn:planewave:460}
\begin{aligned}
v \cdot \gamma_0 = \gpgradezero{\gamma_0 R \gamma_0 \tilde{R}} = \gpgradezero{R^\dagger R} > 0
\end{aligned}
\end{equation}
%
are called proper orthochronous.  There are likely other places involving a time centric projections where this
conjugation operator would have a natural fit.


\subsection{Energy density. Take II}
\index{energy density}

For the Fourier coefficient energy calculation we now take \eqnref{eqn:planewave:assummedSolutionNewStartingPoint} as the starting point.

We will need the conjugate of the field
%
\begin{equation}\label{eqn:planewave:480}
\begin{aligned}
F^\dagger
&= \gamma_0
\left(
\sum_{\Bk}
e^{i \Bomega t}
\hat{F}_{\Bk}
e^{-i \Bk \cdot \Bx}
\right)
^{\tilde{}}
\gamma_0 \\
&= \gamma_0 \sum_{\Bk}
(e^{-i \Bk \cdot \Bx})^{\tilde{}}
(-\hat{F}_{\Bk})
(e^{i \Bomega t})^{\tilde{}}
\gamma_0 \\
&= -\gamma_0 \sum_{\Bk}
e^{-i \Bk \cdot \Bx}
\hat{F}_{\Bk}
e^{-i \Bomega t}
\gamma_0 \\
&= -\sum_{\Bk}
e^{i \Bk \cdot \Bx}
\gamma_0
\hat{F}_{\Bk}
\gamma_0
e^{-i \Bomega t}
\\
\end{aligned}
\end{equation}
% reverse i \Bk = \Bk~ i~ = -\Bk i = -i\Bk
%
This is
\begin{equation}\label{eqn:planewave:conjugateField}
\begin{aligned}
F^\dagger
&= \sum_{\Bk}
e^{i \Bk \cdot \Bx}
(\hat{F}_{\Bk})^{\dagger}
e^{-i \Bomega t}
\end{aligned}
\end{equation}
%
So for the energy we have
%
\begin{equation}\label{eqn:planewave:500}
\begin{aligned}
calF^\dagger + F^\dagger F
&=
\sum_{\Bm, \Bk}
e^{i \Bomega_m t}
\hat{F}_{\Bm}
%e^{-i \Bm \cdot \Bx}
e^{i (\Bk-\Bm) \cdot \Bx}
(\hat{F}_{\Bk})^{\dagger}
e^{-i \Bomega_k t}
+
e^{i \Bk \cdot \Bx}
(\hat{F}_{\Bk})^{\dagger}
%e^{-i \Bomega_k t}
e^{i (\Bomega_m-\Bomega_k) t}
\hat{F}_{\Bm}
e^{-i \Bm \cdot \Bx}
\\
&=
\sum_{\Bm, \Bk}
e^{i \Bomega_m t}
\hat{F}_{\Bm}
(\hat{F}_{\Bk})^{\dagger}
e^{i (\Bk-\Bm) \cdot \Bx -i \Bomega_k t}
+
e^{i \Bk \cdot \Bx}
(\hat{F}_{\Bk})^{\dagger}
\hat{F}_{\Bm}
e^{-i (\Bomega_m-\Bomega_k) t -i \Bm \cdot \Bx}
\\
&=
\sum_{\Bm, \Bk}
\hat{F}_{\Bm}
(\hat{F}_{\Bk})^{\dagger}
e^{i (\Bk-\Bm) \cdot \Bx -i (\Bomega_k-\Bomega_m) t}
+
(\hat{F}_{\Bk})^{\dagger}
\hat{F}_{\Bm}
e^{i (\Bomega_k-\Bomega_m) t +i (\Bk -\Bm) \cdot \Bx}
\\
&=
\sum_{\Bk}
\hat{F}_{\Bk}
(\hat{F}_{\Bk})^{\dagger}
+
(\hat{F}_{\Bk})^{\dagger}
\hat{F}_{\Bk} \\
&\quad+
\sum_{\Bm \ne \Bk}
\hat{F}_{\Bm}
(\hat{F}_{\Bk})^{\dagger}
e^{i (\Bk-\Bm) \cdot \Bx -i (\Bomega_k-\Bomega_m) t}
+
(\hat{F}_{\Bk})^{\dagger}
\hat{F}_{\Bm}
e^{i (\Bomega_k-\Bomega_m) t +i (\Bk -\Bm) \cdot \Bx}
\\
\end{aligned}
\end{equation}
%
In the first sum all the time dependence and all the spatial dependence
that is not embedded in the Fourier coefficients themselves has been eliminated.
What is left is something that looks like it is a real quantity (to be verified)
Assuming (also to be verified) that \(\hat{F}_{\Bk}\) commutes with its conjugate
we have something that looks
like a discrete version of what \citep{haykin1994cs} calls
the Rayleigh energy theorem
%
\begin{equation}\label{eqn:planewave:520}
\begin{aligned}
\IIinf f(x)f^\conj(x) dx &= \IIinf \hat{f}(k)\hat{f}^\conj(k) dk
\end{aligned}
\end{equation}
%
Here \(\hat{f}(k)\) is the Fourier transform of \(f(x)\).

Before going on it is expected that
the \(\Bk \ne \Bm\) terms all cancel.
Having restricted the orientations of the allowed angular velocity bivectors to scalar multiples of the plane formed by the (wedge of) the electric
and magnetic fields, we have only a single set of indices to sum over (ie: \(\Bk = 2 \pi \sigma k/ \lambda\)).
In particular we can sum over \(k < m\), and \(k > m\) cases separately and add these
with expectation of cancellation.  Let us see if this works out.

Write \(\Bomega = 2 \pi \sigma / \lambda\), \(\Bomega_k = k \Bomega\), and \(\Bk = \Bomega/c\) then we have for these terms
%
\begin{equation}\label{eqn:planewave:540}
\begin{aligned}
%\sum_{\{m < k \} \cup \{ k < m \}} &
\sum_{\Bm \ne \Bk}
e^{ i (k-m) \Bomega \cdot \Bx/c } \left(
\hat{F}_{\Bm} (\hat{F}_{\Bk})^{\dagger} e^{ -i (k-m) \Bomega t }
+
(\hat{F}_{\Bk})^{\dagger} \hat{F}_{\Bm} e^{ i (k-m)\Bomega t }
\right)
\\
\end{aligned}
\end{equation}
%
\subsubsection{Hermitian conjugate identities}
\index{Hermitian conjugate}

To get comfortable with the required manipulations, let us
find the Hermitian conjugate equivalents to some of the familiar complex number relationships.

Not all of these will be the same as in ``normal'' complex numbers.  For instance, while for complex numbers, the identities
%
\begin{equation}\label{eqn:planewave:560}
\begin{aligned}
z + \overbar{z} &= 2 \Re(z) \\
\inv{i}( z - \overbar{z} ) &= 2 \Im(z)
\end{aligned}
\end{equation}
%
are both real numbers, we have seen for the electric and magnetic fields that we do not get scalars from the Hermitian conjugates, instead get a spatial vector where we would get a real number in complex arithmetic.  Similarly we get a (bi)vector in the dual
space for the field minus its conjugate.

Some properties:
\begin{itemize}
\item Hermitian conjugate of a product
%
\begin{equation}\label{eqn:planewave:580}
\begin{aligned}
(ab)^\dagger
&= \gamma_0 (ab)^{\tilde{}} \gamma_0 \\
&= \gamma_0
(b)^{\tilde{}}
(a)^{\tilde{}}
\gamma_0 \\
&= \left(\gamma_0 (b)^{\tilde{}} \gamma0\right) \left(\gamma_0 (a)^{\tilde{}} \gamma_0\right) \\
\end{aligned}
\end{equation}
%
This is our familiar conjugate of a product is the inverted order product of conjugates.
\begin{equation}\label{eqn:planewave:600}
\begin{aligned}
(ab)^\dagger &= b^\dagger a^\dagger
\end{aligned}
\end{equation}
%
\item conjugate of a pure pseudoscalar exponential
%
\begin{equation}\label{eqn:planewave:620}
\begin{aligned}
\left(e^{i\alpha}\right)^\dagger
&=
\gamma_0
\left(
\cos(\alpha)
+ i \sin(\alpha)
\right)^{\tilde{}}
\gamma_0 \\
&=
\cos(\alpha)
- i
\gamma_0
\sin(\alpha)
\gamma_0 \\
\end{aligned}
\end{equation}
%
But that is just
\begin{equation}\label{eqn:planewave:640}
\begin{aligned}
\left(e^{i\alpha}\right)^\dagger &= e^{-i\alpha}
\end{aligned}
\end{equation}
%
Again in sync with complex analysis.  Good.
%
\item conjugate of a dual spatial vector exponential
\begin{equation}\label{eqn:planewave:660}
\begin{aligned}
\left(e^{i\Bk}\right)^\dagger
&=
\gamma_0
\left(
\cos(\Bk)
+ i \sin(\Bk)
\right)^{\tilde{}}
\gamma_0 \\
&=
\gamma_0
\left(
\cos(\Bk)
- \sin(\Bk) i
\right)
\gamma_0 \\
&=
\cos(\Bk)
- i\sin(\Bk)
\end{aligned}
\end{equation}
%
So, we have
%
\begin{equation}\label{eqn:planewave:680}
\begin{aligned}
\left(e^{i\Bk}\right)^\dagger &= e^{-i\Bk}
\end{aligned}
\end{equation}
%
Again, consistent with complex numbers for this type of multivector object.

\item dual spatial vector exponential product with a conjugate.
%
\begin{equation}\label{eqn:planewave:700}
\begin{aligned}
F^\dagger  e^{i\Bk}
&= \gamma_0 \tilde{F} \gamma_0 e^{i\Bk} \\
&= \gamma_0 \tilde{F} e^{i\Bk} \gamma_0 \\
&= \gamma_0 e^{-i\Bk} \tilde{F} \gamma_0 \\
&= e^{i\Bk} \gamma_0 \tilde{F} \gamma_0 \\
\end{aligned}
\end{equation}
%
So we have conjugate commutation for both the field and its conjugate
\begin{equation}\label{eqn:planewave:720}
\begin{aligned}
F^\dagger e^{i\Bk} &= e^{-i\Bk} F^\dagger \\
F e^{i\Bk} &= e^{-i\Bk} F
\end{aligned}
\end{equation}
%
\item pseudoscalar exponential product with a conjugate.

For scalar \(\alpha\)
%
\begin{equation}\label{eqn:planewave:740}
\begin{aligned}
F^\dagger  e^{i\alpha}
&= \gamma_0 \tilde{F} \gamma_0 e^{i\alpha} \\
&= \gamma_0 \tilde{F} e^{-i\alpha} \gamma_0 \\
&= \gamma_0 e^{-i\alpha} \tilde{F} \gamma_0 \\
&= e^{i\alpha} \gamma_0 \tilde{F} \gamma_0 \\
\end{aligned}
\end{equation}
%
In opposition to the dual spatial vector exponential, the plain old pseudoscalar exponentials commute with
both the field and its conjugate.
%
\begin{equation}\label{eqn:planewave:760}
\begin{aligned}
F^\dagger e^{i\alpha} &= e^{i\alpha} F^\dagger \\
F e^{i\alpha} &= e^{i\alpha} F
\end{aligned}
\end{equation}
%
\item Pauli vector conjugate.
%
\begin{equation}\label{eqn:planewave:780}
\begin{aligned}
(\sigma_k)^\dagger &= \gamma_0 \gamma_0 \gamma_k \gamma_0 = \sigma_k
\end{aligned}
\end{equation}
%
Jives with the fact that these in matrix form are called Hermitian.

\item pseudoscalar conjugate.
%
\begin{equation}\label{eqn:planewave:800}
\begin{aligned}
i^\dagger = \gamma_{0} i \gamma_0 = -i
\end{aligned}
\end{equation}
%
\item Field Fourier coefficient conjugate.
%
\begin{equation}\label{eqn:planewave:820}
\begin{aligned}
(\hat{F}_{\Bk})^\dagger
&=
\inv{V} \int e^{-i \Bk \cdot \Bx } F^\dagger(\Bx,0) d^3x = \widehat{F^\dagger}_{-\Bk}
\end{aligned}
\end{equation}
%
The conjugate of the \(\Bk\) Fourier coefficient is the \(-\Bk\) Fourier coefficient of the conjugate field.

\end{itemize}

Observe that the first three of these properties would have allowed for calculation of
\eqnref{eqn:planewave:conjugateField} by inspection.

\subsection{Products of Fourier coefficient with another conjugate coefficient}

To progress a relationship between the conjugate products of Fourier coefficients may be required.

\section{FIXME: finish this}

I am getting tired of trying to show (using Latex as a tool and also on paper)
that the \(\Bk \ne \Bm\) terms vanish and am going to take a break, and move on for a bit.  Come back to this later, but start
with a electric field and magnetic field expansion of the $
(\hat{F}_\Bk)^\dagger \hat{F}_\Bk
+
\hat{F}_\Bk (\hat{F}_\Bk)^\dagger
$ term to verify that this ends up being a scalar as desired and expected
(this is perhaps an easier first step than showing the cross terms are zero).

   %
% Copyright � 2012 Peeter Joot.  All Rights Reserved.
% Licenced as described in the file LICENSE under the root directory of this GIT repository.
%

%
%
\mychapter{Lorentz Gauge Fourier Vacuum potential solutions}
\index{Lorentz gauge}
\index{Fourier transform!vacuum potential}
\label{chap:potentialFourier}
%\date{Feb 07, 2009.  potentialFourier.tex}

\section{Motivation}

In \chapcite{PJFourierVacuum} a first order Fourier solution of the Vacuum
Maxwell equation was performed.  Here a comparative potential solution
is obtained.

\subsection{Notation}

The 3D Fourier series notation developed for this treatment can be found
in the original notes \chapcite{PJFourierVacuum}.  Also included there is a
table of notation, much of which is also used here.

\section{Second order treatment with potentials}

\subsection{With the Lorentz gauge}

Now, it appears that Bohm's use of potentials allows a nice comparison with the harmonic oscillator.  Let us also try a Fourier solution of the
potential equations.  Again, use STA instead of the traditional vector equations, writing \(A = (\phi + \Ba)\gamma_0\), and employing the Lorentz gauge
\(\grad \cdot A = 0\) we have for \(F = \grad \wedge A\) in CGS units

FIXME: Add \(\Ba\), and \(\psi\) to notational table below with definitions in terms of \(\bcE\), and \(\bcH\) (or the other way around).

\begin{equation}\label{eqn:potentialFourier:20}
\begin{aligned}
\grad^2 A = 4 \pi J
\end{aligned}
\end{equation}

Again with a spacetime split of the gradient

\begin{equation}\label{eqn:potentialFourier:40}
\begin{aligned}
\grad = \gamma^0(\partial_0 + \spacegrad) = (\partial_0 - \spacegrad) \gamma_0
\end{aligned}
\end{equation}

our four Laplacian can be written

\begin{equation}\label{eqn:potentialFourier:60}
\begin{aligned}
(\partial_0 - \spacegrad) \gamma_0 \gamma^0(\partial_0 + \spacegrad)
&= (\partial_0 - \spacegrad) (\partial_0 + \spacegrad) \\
&= \partial_{00} - \spacegrad^2
\end{aligned}
\end{equation}

Our vacuum field equation for the potential is thus
\begin{equation}\label{eqn:potentialFourier:80}
\begin{aligned}
\partial_{tt} A = c^2 \spacegrad^2 A
\end{aligned}
\end{equation}

Now, as before assume a Fourier solution and see what follows.  That is

\begin{equation}\label{eqn:potential_fourier:assumedPotential}
\begin{aligned}
A(\Bx, t) &= \sum_{\Bk} \hat{A}_{\Bk}(t) e^{ -i \Bk \cdot \Bx}
\end{aligned}
\end{equation}

Applied to each component this gives us
\begin{equation}\label{eqn:potentialFourier:100}
\begin{aligned}
\hat{A}_{\Bk}'' e^{ -i \Bk \cdot \Bx}
&= c^2 \hat{A}_{\Bk}(t) \sum_m \PDsq{x^m}{} e^{ - 2 \pi i \sum_j k_j x^j /\lambda_j} \\
&= c^2 \hat{A}_{\Bk}(t) \sum_m (- 2 \pi i k_m/\lambda_m)^2 e^{ - i \Bk \cdot \Bx } \\
&= -c^2 \Bk^2 \hat{A}_{\Bk} e^{ - i \Bk \cdot \Bx }
\end{aligned}
\end{equation}

So we are left with another big ass set of simplest equations to solve

\begin{equation}\label{eqn:potentialFourier:120}
\begin{aligned}
\hat{A}_{\Bk}'' &= -c^2 \Bk^2 \hat{A}_{\Bk}
\end{aligned}
\end{equation}

Note that again the origin point \(\Bk = (0,0,0)\) is a special case.  Also of note this time is that \(\hat{A}_{\Bk}\) has vector and trivector parts, unlike \(\hat{F}_{\Bk}\) which being derived from dual and non-dual components of a bivector was still a bivector.

It appears that solutions can be found with either left or right handed
vector valued integration constants

\begin{equation}\label{eqn:potentialFourier:140}
\begin{aligned}
\hat{A}_{\Bk}(t) &= \exp(\pm i c \Bk t) C_{\Bk} \\
                 &= D_{\Bk} \exp(\pm i c \Bk t)
\end{aligned}
\end{equation}

Since these are equal at \(t=0\), it appears to imply that these commute with the
complex exponentials as was the case for the bivector field.

For the \(\Bk = 0\) special case we have solutions
\begin{equation}\label{eqn:potentialFourier:160}
\begin{aligned}
\hat{A}_{\Bk}(t) &= D_0 t + C_0
\end{aligned}
\end{equation}

It does not seem unreasonable to require \(D_0 = 0\).  Otherwise this time dependent DC Fourier component will blow up at large and small values, while periodic
solutions are sought.

Putting things back together we have %either of

\begin{equation}\label{eqn:potentialFourier:180}
\begin{aligned}
A(\Bx, t) &= \sum_{\Bk} \exp(\pm i c \Bk t) C_{\Bk} \exp( -i \Bk \cdot \Bx ) \\
%          &= \sum_{\Bk} C_{\Bk} \exp(\pm i c \Bk t) \exp( -i \Bk \cdot \Bx )
\end{aligned}
\end{equation}

Here again for \(t=0\), our integration constants are found to be determined completely by the initial conditions

\begin{equation}\label{eqn:potentialFourier:200}
\begin{aligned}
A(\Bx, 0) &= \sum_{\Bk} C_{\Bk} e^{ -i \Bk \cdot \Bx}
\end{aligned}
\end{equation}

So we can write

\begin{equation}\label{eqn:potentialFourier:220}
\begin{aligned}
C_{\Bk} = \inv{V} \int A(\Bx', 0) e^{ i \Bk \cdot \Bx'} d^3 x'
\end{aligned}
\end{equation}

In integral form this is

\begin{equation}\label{eqn:potential_fourier:potentialSolution}
\begin{aligned}
A(\Bx, t) &= \int \sum_{\Bk} \exp(\pm i \Bk c t ) A(\Bx', 0) \exp( i \Bk \cdot (\Bx -\Bx') )
\end{aligned}
\end{equation}

This, somewhat surprisingly, is strikingly similar to what we had for the bivector field.  That was:

\begin{equation}\label{eqn:potential_fourier:bivectorSolution}
\begin{aligned}
F(\Bx,t) &= \int G(\Bx - \Bx', t) F(\Bx', 0) d^3 x' \\
G(\Bx,t) &= \inv{V} \sum_{\Bk} \exp\left( i \Bk ct \right) \exp\left( -i \Bk \cdot \Bx \right)
\end{aligned}
\end{equation}

We cannot however
commute the time phase term to construct a one sided Green's function for this
potential solution (or perhaps we can but if so shown or attempted to show that this is possible).  We also have a
plus or minus variation in the phase term due to the second order nature of the harmonic oscillator equations for our Fourier coefficients.

\subsection{Comparing the first and second order solutions}

A consequence of working in the Lorentz gauge (\(\grad \cdot A = 0\)) is that our field solution should be a gradient

\begin{equation}\label{eqn:potentialFourier:240}
\begin{aligned}
F
&= \grad \wedge A \\
&= \grad A \\
%&= \int A(\Bx', 0) \left(\grad G_A(\Bx - \Bx', t) \right) d^3 x' \\
\end{aligned}
\end{equation}

%Or with the opposite convolution
%\begin{align*}
%F &= \grad \int A(\Bx - \Bx', 0) G_A(\Bx', t) d^3 x' \\
%\end{align*}

FIXME: expand this out using \eqnref{eqn:potential_fourier:potentialSolution} to compare to the first order solution.

