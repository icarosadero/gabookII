%
% Copyright � 2012 Peeter Joot.  All Rights Reserved.
% Licenced as described in the file LICENSE under the root directory of this GIT repository.
%

%
%
\mychapter{Fourier Solutions to Heat and Wave equations}
\label{chap:PJheatFourier}
\index{heat equation!Fourier transform}
\index{wave equation!Fourier transform}
\index{Fourier transform!heat equation}
\index{Fourier transform!wave equation}
%\date{Jan 19, 2009.  heatFourier.tex}

\section{Motivation}

Stanford iTunesU has some Fourier transform lectures by Prof. Brad Osgood.
He starts with Fourier series and by Lecture 5 has covered this and
the solution of the Heat equation on a ring as an example.

Now, for these lectures I get only sound on my ipod.  I can listen along and
pick up most of the lectures since this is review material, but here is some
notes to firm things up.

Since this heat equation

\begin{equation}\label{eqn:heatFourier:20}
\begin{aligned}
\grad^2 u = \kappa \partial_t u
\end{aligned}
\end{equation}

is also the Schr\"{o}dinger equation for a free particle in one
dimension (once the
constant is fixed appropriately), we can also apply the Fourier
technique to a particle
constrained to a circle.  It would be interesting afterwards to
contrast this with Susskind's solution of the
same problem (where he used the Fourier transform and algebraic techniques
instead).

\section{Preliminaries}

\subsection{Laplacian}
\index{Laplacian}

Osgood wrote the heat equation for the ring as

\begin{equation}\label{eqn:heatFourier:40}
\begin{aligned}
\inv{2} u_{xx} = u_t
\end{aligned}
\end{equation}

where \(x\) represented an angular position on the ring, and where
he set the heat diffusion constant to \(1/2\) for convenience.
To apply this to the Schr\"{o}dinger equation retaining all the desired
units we want to be a bit more careful, so let us start with the Laplacian
in polar coordinates.

In polar coordinates our gradient is

\begin{equation}\label{eqn:heatFourier:60}
\begin{aligned}
\grad = \thetacap \inv{r} \PD{\theta}{} +\rcap \PD{r}{}
\end{aligned}
\end{equation}

squaring this we have

\begin{equation}\label{eqn:heatFourier:80}
\begin{aligned}
\grad^2 = \grad \cdot \grad
&=
\thetacap \inv{r} \PD{\theta}{} \cdot \left(\thetacap \inv{r} \PD{\theta}{}\right)
 +
\rcap \PD{r}{} \cdot \left(\rcap \PD{r}{} \right) \\
&=
\frac{-1}{r^3} \PD{\theta}{r} \PD{\theta}{}
+\inv{r^2} \PDSq{\theta}{}
+ \PDSq{r}{}
\\
&= \inv{r^2} \PDSq{\theta}{} + \PDSq{r}{} \\
\end{aligned}
\end{equation}

So for the circularly constrained where \(r\) is constant case we have simply

\begin{equation}\label{eqn:heatFourier:100}
\begin{aligned}
\grad^2 = \inv{r^2} \PDSq{\theta}{}
\end{aligned}
\end{equation}

and our heat equation to solve becomes

\begin{equation}\label{eqn:heatFourier:120}
\begin{aligned}
\PDSq{\theta}{u(\theta, t)} = (r^2 \kappa) \PD{t}{u(\theta, t)}
\end{aligned}
\end{equation}

\subsection{Fourier series}
\index{Fourier series}

Now we also want Fourier series for a given period.  Assuming the absence of the "Rigor Police" as Osgood puts it
we write for a periodic function \(f(x)\) known on the interval \(I = [a, a+T]\)

\begin{equation}\label{eqn:heatFourier:140}
\begin{aligned}
f(x) = \sum c_k e^{2\pi i k x/T}
\end{aligned}
\end{equation}

\begin{equation}\label{eqn:heatFourier:160}
\begin{aligned}
\int_{\partial I} f(x) e^{- 2 \pi i n x /T}
&= \sum c_k \int_{\partial I} e^{2\pi i (k -n) x/T} \\
&= c_n T
\end{aligned}
\end{equation}

So our Fourier coefficient is
\begin{equation}\label{eqn:heatFourier:180}
\begin{aligned}
\hat{f}(n) = c_n = \inv{T} \int_{\partial I} f(x) e^{- 2 \pi i n x /T}
\end{aligned}
\end{equation}

\section{Solution of heat equation}

\subsection{Basic solution}

Now we are ready to solve the radial heat equation

\begin{equation}\label{eqn:heat_fourier:heatRadial}
\begin{aligned}
u_{\theta\theta} = r^2 \kappa u_t,
\end{aligned}
\end{equation}

by assuming a Fourier series solution.

Suppose

\begin{equation}\label{eqn:heatFourier:200}
\begin{aligned}
u(\theta, t)
&= \sum c_n(t) e^{2 \pi i n \theta / T} \\
&= \sum c_n(t) e^{i n \theta} \\
\end{aligned}
\end{equation}

Taking derivatives of this assumed solution we have
\begin{equation}\label{eqn:heatFourier:220}
\begin{aligned}
u_{\theta\theta} &= \sum (i n)^2 c_n e^{i n \theta} \\
u_{t} &= \sum c_n' e^{i n \theta}
\end{aligned}
\end{equation}

Substituting this back into \eqnref{eqn:heat_fourier:heatRadial} we have

\begin{equation}\label{eqn:heatFourier:240}
\begin{aligned}
\sum - n^2 c_n e^{ i n \theta} = \sum c_n' r^2 \kappa e^{i n \theta}
\end{aligned}
\end{equation}

equating components we have

\begin{equation}\label{eqn:heatFourier:260}
\begin{aligned}
c_n' = - \frac{n^2}{ r^2 \kappa } c_n
\end{aligned}
\end{equation}

which is also just an exponential.

\begin{equation}\label{eqn:heatFourier:280}
\begin{aligned}
c_n = A_n \exp\left(- \frac{n^2}{ r^2 \kappa } t \right)
\end{aligned}
\end{equation}

Reassembling we have the time variation of the solution now fixed and can write

\begin{equation}\label{eqn:heatFourier:300}
\begin{aligned}
u(\theta, t) = \sum A_n \exp\left(- \frac{n^2}{ r^2 \kappa } t + i n \theta\right)
\end{aligned}
\end{equation}

\subsection{As initial value problem}

For the heat equation case, we can assume a known initial heat distribution
\(f(\theta)\).
For an initial time \(t=0\) we can then write

\begin{equation}\label{eqn:heatFourier:320}
\begin{aligned}
u(\theta, 0) = \sum A_n e^{i n \theta} = f(\theta)
\end{aligned}
\end{equation}

This is just another Fourier series, with Fourier coefficients

\begin{equation}\label{eqn:heatFourier:340}
\begin{aligned}
A_n = \inv{2\pi} \int_{\partial I} f(v) e^{-i n v} dv
\end{aligned}
\end{equation}

Final reassembly of the results gives us

\begin{equation}\label{eqn:heatFourier:360}
\begin{aligned}
u(\theta, t) = \sum \exp\left(- \frac{n^2}{ r^2 \kappa } t + i n \theta\right) \inv{2\pi} \int_{\partial I} f(v) e^{-i n v} dv
\end{aligned}
\end{equation}

\subsection{Convolution}
\index{convolution}

Osgood's next step, also with the rigor police in hiding, was to exchange orders of integration and summation, to write

\begin{equation}\label{eqn:heatFourier:380}
\begin{aligned}
u(\theta, t)
&=
\int_{\partial I} f(v) dv \inv{2 \pi} \sum_{n=-\infty}^{\infty} \exp\left(- \frac{n^2}{ r^2 \kappa } t -i n (v -\theta)\right) \\
\end{aligned}
\end{equation}

Introducing a Green's function \(g(v, t)\), we then have the complete solution in terms of convolution

\begin{equation}\label{eqn:heat_fourier:seriesGreens}
\begin{aligned}
g( v , t ) &= \inv{2 \pi} \sum_{n=-\infty}^\infty \exp\left(- \frac{n^2}{ r^2 \kappa } t -i n v \right) \\
u(\theta, t) &= \int_{\partial I} f(v) g(v - \theta, t) dv
\end{aligned}
\end{equation}

Now, this Green's function is fairly interesting.  By summing over paired negative and positive indices, we have a set of weighted Gaussians.

\begin{equation}\label{eqn:heatFourier:400}
\begin{aligned}
g( v , t ) &= \inv{2 \pi} + \sum_{n=1}^\infty \exp\left(- \frac{n^2}{ r^2 \kappa } t \right) \frac{\cos(n v )}{\pi} \\
\end{aligned}
\end{equation}

Recalling that the delta function can be expressed as a limit of a \(\sinc\) function, seeing something similar in this Green's function is not entirely unsurprising seeming.

\section{Wave equation}

The QM equation for a free particle is

\begin{equation}\label{eqn:heat_fourier:schro}
\begin{aligned}
-\frac{\Hbar^2}{2m} \grad^2 \psi = i \Hbar \partial_t \psi
\end{aligned}
\end{equation}

This has the same form of the heat equation, so for the free particle on a circle our wave equation is

\begin{equation}\label{eqn:heatFourier:420}
\begin{aligned}
\psi_{\theta\theta} = - \frac{2 m i r^2 }{\Hbar} \partial_t \psi \quad \mbox{ ie: \(\kappa = - 2 m i /\Hbar\) }
\end{aligned}
\end{equation}

So, if the wave equation was known at an initial time \(\psi(\theta, 0) = \phi(\theta)\), we therefore have by comparison the time evolution of the particle's wave function is

\begin{equation}\label{eqn:heatFourier:440}
\begin{aligned}
g( w, t ) &= \inv{2 \pi} + \sum_{n=1}^\infty \exp\left(- \frac{i \Hbar n^2 t}{ 2 m r^2 } \right) \frac{\cos(n w )}{\pi} \\
\psi(\theta, t) &= \int_{\partial I} \phi(v) g(v - \theta, t) dv
\end{aligned}
\end{equation}

%TODO: contrast this to a Fourier transform solution.  Also write this in terms of circular angular momentum since that appears natually in the Green's function.

\section{Fourier transform solution}

% also see example (brief on details)
%\href{http://zakuski.utsa.edu/~gokhman/ftp//courses/notes/heat.pdf}{ example of Fourier tx solution. }
Now, lets try this one dimensional heat problem with a Fourier transform instead to compare.  Here we do not try to start with an
assumed solution, but instead take the Fourier transform of both sides of the equation directly.

\begin{equation}\label{eqn:heatFourier:460}
\begin{aligned}
\calF(u_{xx}) = \kappa \calF(u_t)
\end{aligned}
\end{equation}

Let us start with the left hand side, where we can evaluate by integrating by parts

\begin{equation}\label{eqn:heatFourier:480}
\begin{aligned}
\calF(u_{xx})
&=
%\inv{\sqrt{2\pi}}
\IIinf u_{xx}(x, t) e^{- 2 \pi i s x } dx \\
&=
%\inv{\sqrt{2\pi}}
\IIinf \PD{x}{u_x(x, t)} e^{- 2 \pi i s x } dx \\
&=
%\inv{\sqrt{2\pi}}
\left(
{\left. u_x(x, t) e^{- 2 \pi i s x } \right\vert}_{x= -\infty}^\infty
-( - 2 \pi i s ) \IIinf u_x(x, t) e^{- 2 \pi i s x } dx
\right) \\
\end{aligned}
\end{equation}

So if we assume (or require) that the derivative of our unknown function \(u\) is zero at infinity, and then similarly
require the function itself to be zero there, we have

\begin{equation}\label{eqn:heatFourier:500}
\begin{aligned}
\calF(u_{xx})
&=
%\inv{\sqrt{2\pi}}
( 2 \pi i s ) \IIinf \PD{x}{u_x(x, t)} e^{- 2 \pi i s x } dx  \\
&=
%\inv{\sqrt{2\pi}}
( 2 \pi i s )^2 \IIinf u(x, t) e^{- 2 \pi i s x } dx  \\
&= ( 2 \pi i s )^2 \calF(u)
\end{aligned}
\end{equation}

Now, for the time derivative.  We want

\begin{equation}\label{eqn:heatFourier:520}
\begin{aligned}
\calF(u_t) &=
%\inv{\sqrt{2\pi}}
\IIinf u_t(x, t) e^{- 2 \pi i s x } dx \\
\end{aligned}
\end{equation}

But can pull the derivative out of the integral for
\begin{equation}\label{eqn:heatFourier:540}
\begin{aligned}
\calF(u_t)
&= \PD{t}{} \left(
%\inv{\sqrt{2\pi}}
\IIinf u(x, t) e^{- 2 \pi i s x } dx \right) \\
&= \PD{t}{\calF(u)}
\end{aligned}
\end{equation}

So, now we have an equation relating time derivatives only of the Fourier transformed solution.

Writing \(\calF(u) = \hat{u}\) this is

\begin{equation}\label{eqn:heat_fourier:toSolveFreq}
\begin{aligned}
( 2 \pi i s )^2 \hat{u} = \kappa \PD{t}{\hat{u}}
\end{aligned}
\end{equation}

With a solution of

\begin{equation}\label{eqn:heatFourier:560}
\begin{aligned}
\hat{u} = A(s) e^{ -4 \pi^2 s^2 t/ \kappa }
\end{aligned}
\end{equation}

Here \(A(s)\) is an arbitrary constant in time integration constant, which may depend on \(s\) since it is a solution of our simpler frequency domain partial differential equation
\eqnref{eqn:heat_fourier:toSolveFreq}.

Performing an inverse transform to recover \(u(x,t)\) we thus have

\begin{equation}\label{eqn:heatFourier:580}
\begin{aligned}
u(x,t)
&=
%\inv{\sqrt{2\pi}}
\IIinf \hat{u} e^{2 \pi i x s } ds  \\
&=
%\inv{\sqrt{2\pi}}
\IIinf A(s) e^{ -4 \pi^2 s^2 t/ \kappa } e^{2 \pi i x s } ds  \\
\end{aligned}
\end{equation}

Now, how about initial conditions.  Suppose we have \(u(x,0) = f(x)\), then

\begin{equation}\label{eqn:heatFourier:600}
\begin{aligned}
f(x) &=
%\inv{\sqrt{2\pi}}
\IIinf A(s) e^{2 \pi i x s } ds \\
\end{aligned}
\end{equation}

Which is just an inverse Fourier transform in terms of the integration ``constant'' \(A(s)\).  We can therefore write the \(A(s)\) in terms of the
initial time domain conditions.

\begin{equation}\label{eqn:heatFourier:620}
\begin{aligned}
A(s) &=
%\inv{\sqrt{2\pi}}
\IIinf f(x) e^{-2 \pi i s x } dx \\
&= \hat{f}(s)
\end{aligned}
\end{equation}

and finally have a complete solution of the one dimensional Heat equation.  That is

\begin{equation}\label{eqn:heatFourier:640}
\begin{aligned}
u(x,t) &=
%\inv{\sqrt{2\pi}}
\IIinf \hat{f}(s) e^{ -4 \pi^2 s^2 t/ \kappa } e^{2 \pi i x s } ds  \\
\end{aligned}
\end{equation}

\subsection{With Green's function?}
\index{Green's function}

If we put in the integral for \(\hat{f}(s)\) explicitly and switch the order as was done with the Fourier series will we get a similar result?   Let us try

\begin{equation}\label{eqn:heatFourier:660}
\begin{aligned}
u(x,t)
&=
%\inv{\sqrt{2\pi}}
\IIinf \left(
%\inv{\sqrt{2\pi}}
\IIinf f(u) e^{-2 \pi i s u } du \right) e^{ -4 \pi^2 s^2 t/ \kappa } e^{2 \pi i x s } ds  \\
&=
%\inv{\sqrt{2\pi}}
\IIinf du f(u)
%\inv{\sqrt{2\pi}}
\IIinf e^{ -4 \pi^2 s^2 t/ \kappa } e^{2 \pi i (x - u) s } ds  \\
\end{aligned}
\end{equation}

Cool.  So, with the introduction of a Green's function \(g(w,t)\) for the fundamental solution of the heat equation, we therefore have our solution in terms of convolution with the initial conditions.  It does not get any more general than this!

\begin{equation}\label{eqn:heatFourier:680}
\begin{aligned}
g(w,t) &=
%\inv{{2\pi}}
\IIinf \exp\left( -\frac{4 \pi^2 s^2 t}{\kappa} + 2 \pi i w s \right) ds \\
u(x,t) &= \IIinf f(u) g( x - u, t) du
\end{aligned}
\end{equation}

Compare this to \eqnref{eqn:heat_fourier:seriesGreens}, the solution in terms of Fourier series.  The form is almost identical, but the requirement for periodicity has been removed by switch to the continuous frequency domain!

\subsection{Wave equation}
\index{wave equation}

With only a change of variables, setting \(\kappa = - 2 m i /\Hbar\) we have the general solution to the one dimensional zero potential wave equation \eqnref{eqn:heat_fourier:schro} in terms of an initial wave function.  However, we have a form of the Fourier transform that obscures the physics has been picked here unfortunately.  Let us start over in super speed mode directly from the wave equation, using the form of the Fourier transform that substitutes \(2\pi s \rightarrow k\) for wave number.

We want to solve
\begin{equation}\label{eqn:heatFourier:700}
\begin{aligned}
-\frac{\Hbar^2}{2m} \psi_{xx} = i \Hbar \psi_t
\end{aligned}
\end{equation}

Now calculate
\begin{equation}\label{eqn:heatFourier:720}
\begin{aligned}
\calF(\psi_{xx})
&= \inv{2\pi} \IIinf \psi_{xx}(x,t) e^{-i k x} dx \\
&=
\inv{2\pi} {\left.\psi_{x}(x,t) e^{-i k x} \right\vert}_{-\infty}^\infty
-(- i k) \inv{2\pi} \IIinf \psi_{x}(x,t) e^{-i k x} dx \\
&= \cdots \\
&= \inv{2\pi} (i k)^2 \hat{\psi}(k)
\end{aligned}
\end{equation}

So we have

\begin{equation}\label{eqn:heatFourier:740}
\begin{aligned}
-\frac{\Hbar^2}{2m} (ik)^2 \hat{\psi}(k,t) = i \Hbar \PD{t}{\hat{\psi}(k,t)}
\end{aligned}
\end{equation}

This provides us the fundamental solutions to the wave function in the wave
number domain

\begin{equation}\label{eqn:heatFourier:760}
\begin{aligned}
\hat{\psi}(k,t) &= A(k) \exp\left( -\frac{i \Hbar k^2}{ 2 m } t \right) \\
{\psi}(x,t) &=
\inv{\sqrt{2\pi}} \IIinf A(k) \exp\left( -\frac{i \Hbar k^2}{ 2 m } t \right)
\exp( i k x ) dk \\
\end{aligned}
\end{equation}

In particular, as before, with an initial time wave function \(\psi(x,0) = \phi(x)\) we have

\begin{equation}\label{eqn:heatFourier:780}
\begin{aligned}
\phi(x) = {\psi}(x,0)
&= \inv{\sqrt{2\pi}} \IIinf A(k) \exp( i k x ) dk \\
&= \calF^{-1}(A(k))
\end{aligned}
\end{equation}

So, \(A(k) = \hat{\phi}\), and we have

\begin{equation}\label{eqn:heatFourier:800}
\begin{aligned}
{\psi}(x,t) &=
\inv{\sqrt{2\pi}} \IIinf \hat{\phi}(k) \exp\left( -\frac{i \Hbar k^2}{ 2 m } t \right) \exp( i k x ) dk \\
\end{aligned}
\end{equation}

So, ending the story we have finally, the general solution for the time evolution of our one dimensional wave function given
initial conditions

\begin{equation}\label{eqn:heatFourier:820}
\begin{aligned}
{\psi}(x,t) &= \calF^{-1}\left( \hat{\phi}(k) \exp\left( -\frac{i \Hbar k^2}{ 2 m } t \right) \right)
\end{aligned}
\end{equation}

or, alternatively, in terms of momentum via \(k = p/\Hbar\) we have

\begin{equation}\label{eqn:heatFourier:840}
\begin{aligned}
{\psi}(x,t) &= \calF^{-1}\left( \hat{\phi}(p) \exp\left( -\frac{i p^2}{ 2 m \Hbar } t \right) \right)
\end{aligned}
\end{equation}

Pretty cool!  Observe that in the wave number or momentum domain the time evolution of the wave function is just a continual phase shift relative to the initial conditions.

%Our solution is
%
%\begin{align}
%g(w,t) &=
%%\inv{{2\pi}}
%\IIinf \exp\left(
%2 \pi i s w
%-\frac{2 \pi^2 s^2 i t \Hbar}{m}
%\right) ds \\
%u(x,t) &= \IIinf f(u) g( x - u, t) du
%\end{align}

\subsection{Wave function solutions by Fourier transform for a particle on a circle}

Now, thinking about how to translate this Fourier transform method to the
wave equation for a particle on a circle (as done by Susskind in his online lectures) makes me realize that one
is free to use any sort of integral transform method appropriate for the problem (Fourier, Laplace, ...).
It does not have to be the Fourier transform.  Now, if we happen to pick an integral transform with \(\theta \in [0, \pi]\) bounds, what do we have?  This is nothing more than the inner product for the Fourier series, and we come full circle!

Now, the next thing to work out in detail is how to translate from the transform methods to the algebraic bra ket notation.  This
looks like it will follow immediately if one calls out the inner product in use explicitly, but that is an exploration for a
different day.
