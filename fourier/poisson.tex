%
% Copyright � 2012 Peeter Joot.  All Rights Reserved.
% Licenced as described in the file LICENSE under the root directory of this GIT repository.
%

%
%
%\input{../peeter_prologue_print.tex}

\chapter{Poisson and retarded Potential Green's functions from Fourier kernels}\label{chap:PJpoisson}
\index{retarded potential!Green's function}
\index{Poisson potential!Green's function}
%\date{Feb 18, 2009.  poisson.tex}

\beginArtWithToc
\section{Motivation}

Having recently attempted a number of Fourier solutions to the Heat, Schr\"{o}dinger, Maxwell vacuum, and inhomogeneous Maxwell equation, a reading
of \citep{mjPerryElectrodynamics} inspired me to have another go.  In particular, he writes the Poisson equation solution explicitly in terms of a Green's
function.

The Green's function for the Poisson equation

\begin{equation}\label{eqn:poisson:50}
\begin{aligned}
G(\mathbf{x} - \Bx') = \frac{1}{4\pi\lvert{\mathbf{x} -\Bx'}\rvert}
\end{aligned}
\end{equation}

is not really derived, rather is just pointed out.  However, it is a nice closed form that does not have any integrals.  Contrast this to the Fourier transform method, where one ends up with a messy threefold integral that is not particularly obvious how to integrate.

In the PF thread 
\href{https://www.physicsforums.com/threads/fourier-transform-solution-to-electrostatics-poisson-equation.293550/}{Fourier transform solution to electrostatics Poisson equation?} I asked if anybody knew how to reduce this integral to the potential kernel of electrostatics.  Before getting any answer from PF I found it in \citep{byron1992mca}, a book recently purchased, but not yet read.

Go through this calculation here myself in full detail to get more comfort with the ideas.  Some of these ideas can probably also be applied to previous incomplete Fourier solution attempts.  In particular, the retarded time potential solutions likely follow.  Can these same ideas be applied to the STA form of the Maxwell equation, explicitly inverting it, as \citep{doran2003gap} indicate is possible (but do not spell out).

\section{Poisson equation}
\subsection{Setup}

As often illustrated with the Heat equation, we seek a Fourier transform solution of the electrostatics Poisson equation

\begin{equation}\label{eqn:poisson:LaplacianPhi}
\nabla^2 \phi = -\rho/\epsilon_0
\end{equation}

Our 3D Fourier transform pairs are defined as
\begin{equation}\label{eqn:poisson:70}
\begin{aligned}
\hat{f}(\mathbf{k}) &= \frac{1}{(\sqrt{2\pi})^3} \iiint f(\mathbf{x}) e^{-i \mathbf{k} \cdot \mathbf{x} } d^3 x \\
{f}(\mathbf{x}) &= \frac{1}{(\sqrt{2\pi})^3} \iiint \hat{f}(\mathbf{k}) e^{i \mathbf{k} \cdot \mathbf{x} } d^3 k \\
\end{aligned}
\end{equation}

Applying the transform we get

\begin{equation}\label{eqn:poisson:poissonSolution}
\begin{aligned}
\phi(\mathbf{x}) &= \frac{1}{\epsilon_0} \iiint \rho(\mathbf{x}') G(\mathbf{x-x'}) d^3 x' \\
G(\mathbf{x}) &= \frac{1}{(2 \pi)^3} \iiint \frac{1}{\mathbf{k}^2} e^{ i \mathbf{k} \cdot \mathbf{x} } d^3 k
\end{aligned}
\end{equation}

Green's functions are usually defined by their delta function operational properties.  Doing so, as defined above we have

\begin{equation}\label{eqn:poisson:greens}
\nabla^2 G(\Bx) = - 4 \pi \delta^3(\Bx)
\end{equation}

(note that there are different sign conventions for this delta function identification.)

Application to the Poisson equation \eqnref{eqn:poisson:LaplacianPhi} gives

\begin{equation}\label{eqn:poisson:10}
\int \nabla^2 G(\Bx - \Bx') \phi(\Bx') = \int (- 4 \pi \delta^3(\Bx - \Bx')) \phi(\Bx') = - 4 \pi \phi(\Bx)
\end{equation}

and with expansion in the alternate sequence
\begin{equation}\label{eqn:poisson:20}
\int \nabla^2 G(\Bx - \Bx') \phi(\Bx') = \int G(\Bx - \Bx') ({\nabla'}^2 \phi(\Bx')) = -\inv{\epsilon_0}\int G(\Bx - \Bx') \rho(\Bx')
\end{equation}

With prior knowledge of electrostatics we should therefore find

\begin{equation}\label{eqn:poisson:30}
G(\Bx) = \inv{4 \pi \Abs{\Bx}}.
\end{equation}

Our task is to actually compute this from the Fourier integral.

\subsection{Evaluating the convolution kernel integral}

\subsubsection{Some initial thoughts}

Now it seems to me that this integral \(G\) only has to be evaluated around a small neighborhood of the origin.  For example if one evaluates one of the integrals

\begin{equation}\label{eqn:poisson:90}
\begin{aligned}
\int_{-\infty}^\infty \frac{1}{{k_1}^2 + {k_2}^2 + {k_3}^3 } e^{ i k_1 x_1 } dk_1
\end{aligned}
\end{equation}

using a an upper half plane contour the result is zero unless \(k_2 = k_3 = 0\).  So one is left with something loosely like

\begin{equation}\label{eqn:poisson:110}
\begin{aligned}
G(\mathbf{x}) &= \lim_{\epsilon \rightarrow 0} \frac{1}{(2 \pi)^3}
\int_{k_1 = -\epsilon}^{\epsilon} dk_1
\int_{k_2 = -\epsilon}^{\epsilon} dk_2
\int_{k_3 = -\epsilon}^{\epsilon} dk_3
 \frac{1}{\mathbf{k}^2} e^{ i \mathbf{k} \cdot \mathbf{x} }
\end{aligned}
\end{equation}

How to reduce this?  Somehow it must be possible to take this Fourier convolution kernel and somehow evaluate the integral to produce the electrostatics potential.

\subsubsection{An attempt}

The answer of how to do so, as pointed out above, was found in \citep{byron1992mca}.  Instead of trying to evaluate this integral which has a pole at the origin, they cleverly evaluate a variant of it

\begin{equation}\label{eqn:poisson:130}
\begin{aligned}
I = \iiint \frac{1}{\mathbf{k}^2 + a^2} e^{ i \mathbf{k} \cdot \mathbf{x} } d^3 k
\end{aligned}
\end{equation}

which splits and shifts the repeated pole into two first order poles away from the origin.
After a change to spherical polar coordinates, the new integral can be evaluated, and the Poisson Green's function in potential form follows by
letting \(a\) tend to zero.

Very cool.  It seems worthwhile to go through the motions of this myself, omitting no details I would find valuable.

First we want the volume element in spherical polar form, and our vector.  That is

\begin{equation}\label{eqn:poisson:150}
\begin{aligned}
\rho &= k \cos\phi \\
dA &= (\rho d\theta)(k d\phi) \\
d^3 k &= dk dA = k^2 \cos\phi d\theta d\phi dk \\
\Bk
&= (\rho \cos\theta, \rho \sin\theta, k \sin\theta) \\
&= k (\cos\phi \cos\theta, \cos\phi \sin\theta, \sin\phi) \\
\end{aligned}
\end{equation}

FIXME: scan picture to show angle conventions picked.

This produces
\begin{equation}\label{eqn:poisson:170}
\begin{aligned}
I &= \int_{\theta=0}^{2\pi} \int_{\phi=-\pi/2}^{\pi/2}
k^2
\int_{k=0}^\infty
\cos\phi d\theta d\phi dk  \\
&\qquad \frac{1}{k^2 + a^2} \exp\left( i k (\cos\phi \cos\theta x_1 + \cos\phi \sin\theta + x_2 + \sin\phi x_3) \right)
\end{aligned}
\end{equation}

Now, this is a lot less tractable than the Byron/Fuller treatment.  In particular they were able to make a \(t = \cos\phi\) substitution, and if I try this I get

\begin{equation}\label{eqn:poisson:190}
\begin{aligned}
I = -\int_{\theta=0}^{2\pi} \int_{t=-1}^{1} \int_{k=0}^\infty \frac{1}{k^2 + a^2} \exp\left( i k (t \cos\theta x_1 + t \sin\theta x_2 + \sqrt{1-t^2} x_3) \right)
k^2 dt d\theta dk \\
\end{aligned}
\end{equation}

Now, this is still a whole lot different, and in particular it has \(ik (t\sin\theta x_2 + \sqrt{1-t^2} x_3)\) in the exponential.  I puzzled over this for a while, but it becomes clear on writing.  Freedom to orient the axis along a preferable direction has been used, and some basis for which \(\Bx = x_j \Be^j + = x \Be^1\) has been used!  We are now left with

\begin{equation}\label{eqn:poisson:210}
\begin{aligned}
I
&= -\int_{\theta=0}^{2\pi} \int_{t=-1}^{1} \int_{k=0}^\infty \frac{1}{k^2 + a^2} \exp\left( i k t \cos\theta x \right) k^2 dt d\theta dk \\
&= -\int_{\theta=0}^{2\pi} \int_{k=0}^\infty \frac{2}{(k^2 + a^2) \cos\theta} \sin\left( k t \cos\theta x \right) k d\theta dk \\
&= -\int_{\theta=0}^{2\pi} \int_{k=-\infty}^\infty \frac{1}{(k^2 + a^2) \cos\theta} \sin\left( k t \cos\theta x \right) k d\theta dk \\
\end{aligned}
\end{equation}

Here the fact that our integral kernel is even in \(k\) has been used to double the range and half the kernel.

However, looking at this, one can see that there is trouble.  In particular, we have \(\cos\theta\) in the denominator, with a range that allows zeros.  How did the text avoid this trouble?

\subsection{Take II}

After mulling it over for a bit, it appears that aligning \(\Bx\) with the x-axis is causing the trouble.  Aligning
with the
z-axis will work out much better, and leave only one trig term in the exponential.  Essentially we need to
use a volume of rotation about the z-axis, integrating along all sets of constant \(\Bk \cdot \Bx\).
This is a set of integrals over concentric ring volume elements (FIXME: picture).

Our volume element, measuring \(\theta \in [0, \pi]\) from the z-axis, and \(\phi\) as our position on the ring

\begin{equation}\label{eqn:poisson:230}
\begin{aligned}
\Bk \cdot \Bx &= k x \cos\theta \\
\rho &= k \sin\theta \\
dA &= (\rho d\phi)(k d\theta) \\
d^3 k &= dk dA = k^2 \sin\theta d\theta d\phi dk \\
\end{aligned}
\end{equation}

This gives us

\begin{equation}\label{eqn:poisson:250}
\begin{aligned}
I &= \int_{\theta=0}^{\pi} \int_{\phi=0}^{2\pi} \int_{k=0}^\infty \frac{1}{k^2 + a^2} \exp\left( i k x \cos\theta \right)
k^2 \sin\theta d\theta d\phi dk \\
\end{aligned}
\end{equation}

Now we can integrate immediately over \(\phi\), and make a \(t = \cos\theta\) substitution (\(dt = -\sin\theta d\theta\))

\begin{equation}\label{eqn:poisson:270}
\begin{aligned}
I
&= - 2\pi \int_{t=1}^{-1} \int_{k=0}^\infty \frac{1}{k^2 + a^2} \exp\left( i k x t \right) k^2 dt dk \\
&= - \frac{2\pi}{ix} \int_{k=0}^\infty \frac{1}{k^2 + a^2}
\left(
e^{-i k x } -e^{i k x }
\right)
k dk \\
&= \frac{2\pi}{ix} \int_{k=0}^\infty \frac{1}{k^2 + a^2}
e^{i k x }
k dk
- \frac{2\pi}{ix} \int_{k=-0}^{-\infty} \frac{1}{k^2 + a^2}
e^{i k x }
(-k) (-dk) \\
&= \frac{2\pi}{ix} \int_{k=-\infty}^\infty \frac{1}{k^2 + a^2} e^{i k x } k dk \\
&= \frac{2\pi}{ix} \int_{k=-\infty}^\infty \inv{k - ia} \frac{ k e^{ i k x } }{(k + ia)} dk \\
\end{aligned}
\end{equation}

Now we have something that is in form for contour integration.  In the upper half plane we have a pole at \(k= ia\).  Assuming that the
integral over the big semicircular arc vanishes, we can just pick up the residue at that pole contributing.  The assumption that
this vanishes is actually non-trivial looking since the \(k/(k+ia)\) term at a big radius \(R\) tends to \(1\).  This is probably where
Jordan's lemma comes in, so some study to understand that looks well justified.

\begin{equation}\label{eqn:poisson:290}
\begin{aligned}
0
&= I - 2 \pi i {\left. \frac{2\pi}{ix} \frac{ k e^{ i k x } }{(k + ia)} \right\vert}_{k= ia} \\
&= I - 2 \pi i \frac{2\pi}{ix} \frac{ e^{ - a x } }{2} \\
\end{aligned}
\end{equation}

So we have
\begin{equation}\label{eqn:poisson:310}
\begin{aligned}
I = \frac{2 \pi^2}{x} e^{ - a x }
\end{aligned}
\end{equation}

Now that we have this, the Green's function of \eqnref{eqn:poisson:poissonSolution} is

\begin{equation}\label{eqn:poisson:330}
\begin{aligned}
G(\mathbf{x})
&= \lim_{a \rightarrow 0} \inv{(2\pi)^3} \frac{2 \pi^2}{x} e^{ - a x } \\
&= \inv{4\pi\Abs{\Bx}}
\end{aligned}
\end{equation}

Which gives
\begin{equation}\label{eqn:poisson:350}
\begin{aligned}
\phi(\Bx) &= \inv{4\pi \epsilon_0} \int \frac{\rho(\Bx')}{\Abs{\Bx - \Bx'}} dV'
\end{aligned}
\end{equation}

Awesome!  All following from the choice to set \(\BE = -\spacegrad \phi\), we have a solution for \(\phi\) following directly
from the divergence equation \(\spacegrad \cdot \BE = \rho/\epsilon_0\) via Fourier transformation of this equation.

\section{Retarded time potentials for the 3D wave equation}

\subsection{Setup}

If we look at the general inhomogeneous Maxwell equation

\begin{equation}\label{eqn:poisson:370}
\begin{aligned}
\grad F = J/\epsilon_0 c
\end{aligned}
\end{equation}

In terms of potential \(F = \grad \wedge A\) and employing in the Lorentz gauge \(\grad \cdot A = 0\), we have

\begin{equation}\label{eqn:poisson:390}
\begin{aligned}
\grad^2 A = \left(\inv{c^2} \partial_{tt} - \sum \partial_{jj}\right) A = J/\epsilon_0 c
\end{aligned}
\end{equation}

As scalar equations with \(A = A^\mu \gamma_\mu\), \(J = J^\nu \gamma_\nu\) we have four equations all of the same form.

A Green's function form for such wave equations was previously calculated in \chapcite{PJfourierMaxwellSecondOrder}.  That was

\begin{equation}\label{eqn:poisson:410}
\begin{aligned}
\left( \inv{c^2} \PDSq{t}{} -\sum_j \PDSq{x^j}{} \right)\psi &= g
\end{aligned}
\end{equation}

\begin{equation}\label{eqn:poisson:430}
\begin{aligned}
{\psi}(\Bx, t) &= \IIinf \IIinf \IIinf \IIinf g(\Bx', t') G(\Bx - \Bx', t - t') d^3 x' dt' \\
G(\Bx, t) &= \theta(t) \IIinf \IIinf \IIinf \frac{c}{(2\pi)^3 \Abs{\Bk}} \sin( \Abs{\Bk} c t ) \exp\left( i \Bk \cdot \Bx \right) d^3 k
\end{aligned}
\end{equation}

Here \(\theta(t)\) is the unit step function, which meant we only sum the time contributions of the charge density for \(t - t' > 0\), or \(t' < t\).  That is the causal variant of the solution, which was arbitrary mathematically (\(t > t'\) would have also worked).

\subsection{Reducing the Green's function integral}

Let us see if the spherical polar method works to reduce this equation too.  In particular we want to evaluate

\begin{equation}\label{eqn:poisson:450}
\begin{aligned}
I = \IIinf \IIinf \IIinf \inv{\Abs{\Bk}} \sin( \Abs{\Bk} c t ) \exp\left( i \Bk \cdot \Bx \right) d^3 k
\end{aligned}
\end{equation}

Will we have a requirement to introduce a pole off the origin as above?  Perhaps like
\begin{equation}\label{eqn:poisson:470}
\begin{aligned}
\IIinf \IIinf \IIinf \inv{\Abs{\Bk} + \alpha} \sin( \Abs{\Bk} c t ) \exp\left( i \Bk \cdot \Bx \right) d^3 k
\end{aligned}
\end{equation}

Let us omit it for now, but make the same spherical polar substitution used successfully above, writing
%\begin{align*}
%I = \IIinf \IIinf \IIinf \inv{\Abs{\Bk} } \sin( \Abs{\Bk} c t ) \exp\left( i \Bk \cdot \Bx \right) d^3 k
%\end{align*}

\begin{equation}\label{eqn:poisson:490}
\begin{aligned}
I
&= \int_{\theta=0}^{\pi} \int_{\phi=0}^{2\pi} \int_{k=0}^\infty \frac{1}{k }
\sin\left( k c t \right) \exp\left( i k x \cos\theta \right)
k^2 \sin\theta d\theta d\phi dk \\
&= 2 \pi \int_{\theta=0}^{\pi} \int_{k=0}^\infty %\frac{1}{k }
\sin\left( k c t \right) \exp\left( i k x \cos\theta \right)
k \sin\theta d\theta dk \\
\end{aligned}
\end{equation}

Let \(\tau = \cos\theta\), \(-d\tau = \sin\theta d\theta\), for

\begin{equation}\label{eqn:poisson:510}
\begin{aligned}
I
&= 2 \pi \int_{\tau=1}^{-1} \int_{k=0}^\infty %\frac{1}{k }
\sin\left( k c t \right) \exp\left( i k x \tau \right)
k (-d\tau) dk \\
&= -2 \pi \int_{k=0}^\infty %\frac{1}{k }
\sin\left( k c t \right)
\frac{2}{2i k x} \left( {\exp\left( -i k x \right) } -{\exp\left( i k x \right) } \right)
k dk \\
&= \frac{4 \pi }{x} \int_{k=0}^\infty %\frac{1}{k }
\sin\left( k c t \right)
\sin\left( k x \right)
dk \\
&= \frac{2 \pi }{x} \int_{k=0}^\infty %\frac{1}{k }
\left( \cos\left( k (x-c t) \right) -\cos\left( k (x+c t) \right) \right)
 dk \\
\end{aligned}
\end{equation}

Okay, this is much simpler, but still not in a form that is immediately obvious how to apply contour integration to, since it has no poles.
The integral
kernel here is however an even function,
so we can use the trick of doubling the integral range.

\begin{equation}\label{eqn:poisson:530}
\begin{aligned}
I &= \frac{\pi }{x} \int_{k=-\infty}^\infty \left( \cos\left( k (x-c t) \right) -\cos\left( k (x+c t) \right) \right) dk \\
\end{aligned}
\end{equation}

Having done this, this integral is not really any more well defined.  With the Rigor police on holiday, let us assume we want the
principle value of this integral
\begin{equation}\label{eqn:poisson:550}
\begin{aligned}
I
&= \lim_{R \rightarrow \infty} \frac{\pi }{x} \int_{k=-R}^R \left( \cos\left( k (x-c t) \right) -\cos\left( k (x+c t) \right) \right) dk \\
&= \lim_{R \rightarrow \infty} \frac{\pi }{x} \int_{k=-R}^R d \left( \frac{\sin\left( k (x-c t) \right)}{ x - c t} - \frac{\sin\left( k (x+c t) \right)}{x + c t} \right) \\
&= \lim_{R \rightarrow \infty} \frac{2 \pi^2 }{x} \left( \frac{\sin\left( R (x-c t) \right)}{ \pi(x - c t)} - \frac{\sin\left( R (x+c t) \right)}{\pi(x + c t)} \right) \\
\end{aligned}
\end{equation}

This \(\sinc\) limit has been seen before being functionally identified with the delta function (the wikipedia article calls these ``nascent delta function''), so we can write

\begin{equation}\label{eqn:poisson:570}
\begin{aligned}
I &= \frac{2 \pi^2 }{x} \left( \delta(x-c t) - \delta(x+c t) \right)
\end{aligned}
\end{equation}

For our Green's function we now have

\begin{equation}\label{eqn:poisson:590}
\begin{aligned}
G(\Bx, t)
&= \theta(t) \frac{c }{(2\pi)^3} \frac{2 \pi^2 }{\Abs{\Bx}} \left( \delta(x-c t) - \delta(x+c t) \right) \\
&= \theta(t) \frac{c }{4\pi \Abs{\Bx}} \left( \delta(x-c t) - \delta(x+c t) \right) \\
\end{aligned}
\end{equation}

And finally, our wave function (switching variables to convolve with the charge density) instead of the Green's function

\begin{equation}\label{eqn:poisson:610}
\begin{aligned}
{\psi}(\Bx, t)
&=
\IIinf \IIinf \IIinf \IIinf g(\Bx - \Bx', t - t') \theta(t') \frac{c }{4\pi \Abs{\Bx'}} \delta(\Abs{\Bx'} - c t') d^3 x' dt' \\
&-\IIinf \IIinf \IIinf \IIinf g(\Bx - \Bx', t - t') \theta(t') \frac{c }{4\pi \Abs{\Bx'}} \delta(\Abs{\Bx'} + c t') d^3 x' dt' \\
\end{aligned}
\end{equation}

Let us break these into two parts

\begin{equation}\label{eqn:poisson:almostThere}
\begin{aligned}
{\psi}(\Bx, t) &= {\psi}_{-}(\Bx, t) +{\psi}_{+}(\Bx, t)
\end{aligned}
\end{equation}

Where the first part, \(\psi_{-}\) is for the \(-ct'\) delta function and one \(\psi_{-}\) for the \(+ct'\).
Making a \(\tau = t-t'\) change of variables, this first portion is

\begin{equation}\label{eqn:poisson:630}
\begin{aligned}
{\psi}_{-}(\Bx, t)
&= -\IIinf \IIinf \IIinf \IIinf g(\Bx - \Bx', \tau) \theta(t-\tau) \frac{c }{4\pi \Abs{\Bx'}} \delta(\Abs{\Bx'} - c t + c\tau) d^3 x' d\tau \\
&= -\IIinf \IIinf \IIinf g(\Bx - \Bx'', t - \Abs{\Bx''}/c) \frac{c }{4\pi \Abs{\Bx''}} d^3 x'' \\
\end{aligned}
\end{equation}

One more change of variables, \(\Bx' = \Bx - \Bx''\), \(d^3 x'' = -d^3 x\), gives the final desired retarded potential result.  The \(\psi_{+}\) result is similar (see below), and assembling all we have

\begin{equation}\label{eqn:poisson:650}
\begin{aligned}
{\psi}_{-}(\Bx, t) &= \IIinf \IIinf \IIinf g(\Bx', t - \Abs{\Bx - \Bx'}/c) \frac{c }{4\pi \Abs{\Bx - \Bx'}} d^3 x' \\
{\psi}_{+}(\Bx, t) &= -\IIinf \IIinf \IIinf g(\Bx, t + \Abs{\Bx -\Bx'}/c) \frac{c }{4\pi \Abs{\Bx -\Bx'}} d^3 x'
\end{aligned}
\end{equation}

It looks like my initial interpretation of the causal nature of the unit step in the original functional form was not really right.  It is not
until the Green's function is ``integrated'' do we get this causal and non-causal split into two specific solutions.
In the first of these solutions is only charge contributions at the position in space offset by the wave propagation speed effects the
potential (this is the causal case).  On the other hand we have a second specific solution to the wave equation
summing the charge contributions at all the future positions, this time offset by the time it takes a wave to propagate backwards from that future spacetime

The final mathematical result is consistent with statements seen elsewhere, such as in \citep{feynman1963flp}, although it is
likely that the path taken by others to get this result was less ad-hoc than mine.
It is been a couple years since seeing this for the first time in Feynman's text.
It was not clear to me how somebody could possibly come up with those starting with Maxwell's equations.
Here by essentially applying undergrad Engineering Fourier methods, we get the result in an admittedly ad-hoc fashion, but at least the result is not
pulled out of a magic hat.

\subsection{Omitted Details.  Advanced time solution}

Similar to the above for \(\psi_{+}\) we have
\begin{equation}\label{eqn:poisson:670}
\begin{aligned}
{\psi}_{+}(\Bx, t)
&= -\IIinf \IIinf \IIinf \IIinf g(\Bx - \Bx', t - t') \theta(t') \frac{c }{4\pi \Abs{\Bx'}} \delta(\Abs{\Bx'} + c t') d^3 x' dt' \\
&= \IIinf \IIinf \IIinf \IIinf g(\Bx - \Bx', \tau) \theta(t -\tau) \frac{c }{4\pi \Abs{\Bx'}} \delta(\Abs{\Bx'} + c (t -\tau)) d^3 x' d\tau \\
&= \IIinf \IIinf \IIinf \IIinf g(\Bx - \Bx', \tau) \theta(t -\tau) \frac{c }{4\pi \Abs{\Bx'}} \delta(\Abs{\Bx'} + c t -c\tau) d^3 x' d\tau \\
&= \IIinf \IIinf \IIinf g(\Bx - \Bx', t + \Abs{\Bx'}/c) \frac{c }{4\pi \Abs{\Bx'}} d^3 x' \\
&= -\IIinf \IIinf \IIinf g(\Bx, t + \Abs{\Bx -\Bx'}/c) \frac{c }{4\pi \Abs{\Bx -\Bx'}} d^3 x' \\
\end{aligned}
\end{equation}

Is there an extra factor of \(-1\) here?

\section{1D wave equation}

It is somewhat irregular seeming to treat the 3D case before what should be the simpler 1D case, so let us
try evaluating the Green's function for the 1D wave equation too.

We have found that Fourier transforms applied to the forced wave equation
\begin{equation}\label{eqn:poisson:690}
\begin{aligned}
\left( \inv{v^2} \partial_{tt} -\partial_{xx} \right)\psi = g(x,t)
\end{aligned}
\end{equation}

result in the following integral solution.

\begin{equation}\label{eqn:poisson:oneDimResult}
\begin{aligned}
{\psi}(x, t)
&=
\int_{x'=-\infty}^\infty
\int_{t' = 0}^\infty {g}(x-x', t-t') G(x', t') dx' dt' \\
G(x, t) &=
\int_{k = -\infty}^\infty
\frac{v}{2\pi {k}}
\sin( {k} v t )
\exp( i k x )
dk
\end{aligned}
\end{equation}

As in the 3D case above can this reduced to something that does not involve such an unpalatable integral.
Given the 3D result, it would be reasonable to get a result involving \(g(x \pm vt)\) terms.

First let us get rid of the sine term, and express \(G\) entirely in exponential form.  That is

\begin{equation}\label{eqn:poisson:710}
\begin{aligned}
G(x, t)
&=
\int_{k = -\infty}^\infty
\frac{v}{4\pi k i }
\left(\exp( {k} v t ) -\exp( -{k} v t )\right) \exp( i k x )
dk \\
&=
\int_{k = -\infty}^\infty
\frac{v}{4\pi k i }
\left(e^{ {k} (x + v t ) } - e^{ {k} (x - v t) }\right)
dk \\
\end{aligned}
\end{equation}

Using the unit step function identification from \eqnref{eqn:poisson:heavysideUnitIntegral}, we have

\begin{equation}\label{eqn:poisson:730}
\begin{aligned}
G(x, t) &= \frac{v}{2} \left(\theta(x + v t )  - \theta(x - v t) \right)
\end{aligned}
\end{equation}

If this identification works our solution then becomes

\begin{equation}\label{eqn:poisson:750}
\begin{aligned}
{\psi}(x, t)
&=
\int_{x'=-\infty}^\infty
\int_{t'= 0}^\infty
{g}(x-x', t-t')
%G(x', t')
\frac{v}{2} \left(\theta(x' + v t' )  - \theta(x' - v t') \right)
dx' dt' \\
&=
\int_{x'=-\infty}^\infty
\int_{s= 0}^\infty
{g}(x-x', t-s/v)
\frac{1}{2} \left(\theta(x' + s )  - \theta(x' - s) \right)
dx' ds
\end{aligned}
\end{equation}

This is already much simpler than the original, but additional reduction should be possible by breaking this down into specific intervals.  An
alternative, perhaps is to use integration by parts and the delta function as the derivative of the unit step identification.
%  Let us see
%if that works.

Let us try a pair of variable changes

\begin{equation}\label{eqn:poisson:770}
\begin{aligned}
{\psi}(x, t)
&=
% x' + s = u
% x' = u - s
\int_{u=-\infty}^\infty
\int_{s= 0}^\infty
{g}(x-u+s, t-s/v)
\frac{1}{2} \theta(u )
du ds \\
% x' - s = u
% x' = u + s
&-\int_{u=-\infty}^\infty
\int_{s= 0}^\infty
{g}(x-u -s, t-s/v)
\frac{1}{2} \theta(u)
du ds \\
\end{aligned}
\end{equation}

Like the retarded time potential solution to the 3D wave equation, we now have the wave function solution entirely specified by a
weighted sum of the driving function

\begin{equation}\label{eqn:poisson:790}
\begin{aligned}
{\psi}(x, t)
&=
\frac{1}{2}
\int_{u = 0}^\infty
\int_{s = 0}^\infty
\left( {g}(x-u+s, t-s/v) - {g}(x-u -s, t-s/v) \right)
du ds
\end{aligned}
\end{equation}

Can this be tidied at all?  Let us do a change of variables here, writing \(-\tau = t -s/v\).

% -\tau = t - s/v
% s/v = t + \tau
% s = v(t + \tau)
% ds = d\tau
% \tau(0) = -t
%
%
% \tau = t - s/v
% s/v = t - \tau
% s = v(t - \tau)
% ds = -d\tau
% \tau(0) = t
%
\begin{equation}\label{eqn:poisson:810}
\begin{aligned}
{\psi}(x, t)
&=
\frac{1}{2}
\int_{u = 0}^\infty
\int_{\tau = -t}^\infty
%\left( {g}(x-u+v(t + \tau), \tau) - {g}(x-u -v(t+\tau), \tau) \right)
\left( {g}(x+vt -(u - v\tau), \tau) - {g}(x-v t -(u +v\tau), \tau) \right)
du d\tau \\
&=
\frac{1}{2}
\int_{u = 0}^\infty
\int_{\tau = -\infty}^{t}
\left( {g}(x+vt -(u + v\tau), -\tau) - {g}(x-v t -(u -v\tau), -\tau) \right)
du d\tau \\
\end{aligned}
\end{equation}

Is that any better?  I am not so sure, and intuition says there is a way to reduce this to a single integral summing
only over spatial variation.

\subsection{Followup to verify}

There has been a lot of guessing and loose mathematics here.  However, if this is a valid solution despite all that, we should be
able to apply the wave function operator \(\inv{v^2} \partial_{tt} + \partial_{xx}\) as a
consistency check and get back \(g(x,t)\) by differentiating
under the integral sign.

FIXME: First have to think about how exactly to do this differentiation.

\section{Appendix}
\subsection{Integral form of unit step function}

The wiki article on the Heaviside unit step function lists an integral form

\begin{equation}\label{eqn:poisson:830}
\begin{aligned}
I_\epsilon &= \inv{2\pi i} \PV \IIinf \frac{e^{ix\tau}}{\tau - i\epsilon} d\tau \\
\theta(x) &= \lim_{\epsilon \rightarrow 0} I_\epsilon
\end{aligned}
\end{equation}

How does this make sense?  For \(x>0\) we can evaluate this with an upper half plane semi-circular contour (FIXME: picture).  Along the arc \(z = R e^{i\phi}\) we have

\begin{equation}\label{eqn:poisson:850}
\begin{aligned}
\Abs{I_\epsilon}
&= \Abs{\inv{2\pi i} \int_{\phi=0}^{\pi} \frac{e^{i R (\cos\phi + i \sin\phi)}}{ R e^{i\phi} - i \epsilon} R i e^{i\phi} d\phi} \\
&\approx \Abs{\inv{2\pi} \int_{\phi=0}^{\pi} e^{i R \cos\phi } e^{- R \sin\phi } d\phi} \\
&\le \inv{2\pi} \int_{\phi=0}^{\pi} e^{- R \sin\phi } d\phi \\
&\le \inv{2\pi} \int_{\phi=0}^{\pi} e^{- R } d\phi \\
&= \inv{2} e^{- R } \\
\end{aligned}
\end{equation}

This tends to zero as \(R \rightarrow \infty\), so evaluating the residue, we have for \(x > 0\)

\begin{equation}\label{eqn:poisson:870}
\begin{aligned}
{I_\epsilon}
&= -(-2 \pi i) \inv{2\pi i} {\left. {e^{ix\tau}} \right\vert}_{\tau = i\epsilon} \\
&= e^{-x\epsilon}
\end{aligned}
\end{equation}

Now for \(x<0\) an upper half plane contour will diverge, but the lower half plane can be used.  This gives us \(I_\epsilon = 0\) in that region.
All that remains is the \(x=0\) case.  There we have

\begin{equation}\label{eqn:poisson:890}
\begin{aligned}
I_\epsilon(0)
&= \inv{2\pi i} \PV \IIinf \inv{\tau - i\epsilon} d\tau \\
&= \inv{2\pi i} \lim_{R \rightarrow \infty} \ln\left( \frac{R - i\epsilon} {-R - i\epsilon} \right) \\
& \rightarrow \inv{2\pi i} \ln\left( -1 \right) \\
&= \inv{2\pi i} i \pi \\
\end{aligned}
\end{equation}

Summarizing we have

\begin{equation}\label{eqn:poisson:910}
\begin{aligned}
I_\epsilon(x) =
\left\{
\begin{array}{l l}
e^{-x\epsilon} & \quad \mbox{if \(x > 0\)} \\
\inv{2} & \quad \mbox{if \(x = 0\)} \\
0 & \quad \mbox{if \(x < 0\)} \\
\end{array}
\right.
\end{aligned}
\end{equation}

So in the limit this does work as an integral formulation of the unit step.  This will be used to (very loosely) identify

\begin{equation}\label{eqn:poisson:heavysideUnitIntegral}
\begin{aligned}
\theta(x) \sim \inv{2\pi i} \PV \IIinf \frac{e^{ix\tau}}{\tau} d\tau
\end{aligned}
\end{equation}
%\EndArticle
