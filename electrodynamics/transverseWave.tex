%
% Copyright � 2012 Peeter Joot.  All Rights Reserved.
% Licenced as described in the file LICENSE under the root directory of this GIT repository.
%

%
%
%\input{../peeter_prologue.tex}

\chapter{Comparing phasor and geometric transverse solutions to the Maxwell equation}
\index{Maxwell equation!transverse solution}
\label{chap:transverseWave}

%\blogpage{http://sites.google.com/site/peeterjoot/math2009/transverseWave.pdf}
%\date{August 6, 2009}
%\revisionInfo{\(RCSfile: transverseWave.tex,v \) Last \(Revision: 1.10 \) \(Date: 2009/10/22 02:07:20 \)}

\beginArtWithToc

\section{Motivation}

In (\chapcite{maxwellVacuum}) a phasor like form of the transverse wave equation was found by considering Fourier solutions of the Maxwell equation.  This will be called the ``geometric phasor'' since it is hard to refer and compare it without giving it a name.  Curiously no perpendicularity condition for \(\BE\) and \(\BB\) seemed to be required for this geometric phasor.  Why would that be the case?  In Jackson's treatment, which employed the traditional dot and cross product form of Maxwell's equations, this followed by back substituting the assumed phasor solution back into the equations.  This back substitution was not done in (\chapcite{maxwellVacuum}).  If we attempt this we should find the same sort of additional mutual perpendicularity constraints on the fields.

Here we start with the equations from Jackson (\citep{jackson1975cew}, ch7), expressed in GA form.  Using the same assumed phasor form we should get the same results using GA.  Anything else indicates a misunderstanding or mistake, so as an intermediate step we should at least recover the Jackson result.

After using a more traditional phasor form (where one would have to take real parts) we revisit the geometric phasor found in (\chapcite{maxwellVacuum}).  It will be found that the perpendicular constraints of the Jackson phasor solution lead to a representation where the geometric phasor is reduced to the Jackson form with a straight substitution of the imaginary \(i\) with the pseudoscalar \(I = \sigma_1\sigma_2\sigma_3\).  This representation however, like the more general geometric phasor requires no selection of real or imaginary parts to construct a ``physical'' solution.

\section{With assumed phasor field}

Maxwell's equations in absence of charge and current ((7.1) of Jackson) can be summarized by

\begin{equation}\label{eqn:transverseWave:foo1}
\begin{aligned}
0 &= (\spacegrad + \sqrt{\mu\epsilon}\partial_0) F
\end{aligned}
\end{equation}

The \(F\) above is a composite electric and magnetic field merged into a single multivector.  In the spatial basic the electric field component \(\BE\) is a vector, and the magnetic component \(I\BB\) is a bivector (in the Dirac basis both are bivectors).

\begin{equation}\label{eqn:transverseWave:foo0}
\begin{aligned}
F &= \BE + I \BB/\sqrt{\mu\epsilon}
\end{aligned}
\end{equation}

With an assumed phasor form

\begin{equation}\label{eqn:transverseWave:foo2}
\begin{aligned}
F = \calF e^{ i(\Bk \cdot \Bx - \omega t) } = (\bcE + I\bcB/\sqrt{\mu\epsilon}) e^{ i(\Bk \cdot \Bx - \omega t) }
\end{aligned}
\end{equation}

Although there are many geometric multivectors that square to -1, we do not assume here that the imaginary \(i\) has any specific geometric meaning, and in fact commutes with all multivectors.  Because of this we have to take the real parts later when done.

Operating on \(F\) with Maxwell's equation we have

\begin{equation}\label{eqn:transverseWave:foo3}
\begin{aligned}
0 = (\spacegrad + \sqrt{\mu\epsilon}\partial_0) F = i \left( \Bk - \sqrt{\mu\epsilon}\frac{\omega}{c} \right) F
\end{aligned}
\end{equation}

Similarly, left multiplication of Maxwell's equation by the conjugate operator \(\spacegrad - \sqrt{\mu\epsilon}\partial_0\), we have the wave equation

\begin{equation}\label{eqn:transverseWave:foo4}
\begin{aligned}
0 &= \left(\spacegrad^2 - \frac{\mu\epsilon}{c^2}\frac{\partial^2}{\partial t^2}\right) F
\end{aligned}
\end{equation}

and substitution of the assumed phasor solution gives us

\begin{equation}\label{eqn:transverseWave:foo7}
\begin{aligned}
0 = \lr{\spacegrad^2 - {\mu\epsilon}\partial_{00}} F = -\left( \Bk^2 - {\mu\epsilon}\frac{\omega^2}{c^2} \right) F
\end{aligned}
\end{equation}

This provides the relation between the magnitude of \(\Bk\) and \(\omega\), namely

\begin{equation}\label{eqn:transverseWave:foo8}
\begin{aligned}
\Abs{\Bk} = \pm \sqrt{\mu\epsilon}\frac{\omega}{c}
\end{aligned}
\end{equation}

Without any real loss of generality we can pick the positive root, so the result of the Maxwell equation operator on the phasor is

\begin{equation}\label{eqn:transverseWave:foo5}
\begin{aligned}
0 = (\spacegrad + \sqrt{\mu\epsilon}\partial_0) F = i \sqrt{\mu\epsilon}\frac{\omega}{c} \left( \kcap - 1\right) F
\end{aligned}
\end{equation}

Rearranging we have the curious property that the field \(F\) can ``swallow'' a left multiplication by the propagation direction unit vector

\begin{equation}\label{eqn:transverseWave:fooA}
\begin{aligned}
\kcap F = F
\end{aligned}
\end{equation}

Selection of the scalar and pseudoscalar grades of this equation shows that the electric and magnetic fields \(\BE\) and \(\BB\) are both completely transverse to the propagation direction \(\kcap\).  For the scalar grades we have

\begin{equation}\label{eqn:transverseWave:28}
\begin{aligned}
0 &= \gpgradezero{\kcap F - F} \\
  &= \kcap \cdot \BE
\end{aligned}
\end{equation}

and for the pseudoscalar
\begin{equation}\label{eqn:transverseWave:48}
\begin{aligned}
0 &= \gpgradethree{\kcap F - F} \\
  &= I \kcap \cdot \BB
\end{aligned}
\end{equation}

From this we have \(\kcap \cdot \BB = \kcap \cdot \BB = 0\).  Because of this transverse property we see that the \(\kcap\) multiplication of \(F\) in \eqnref{eqn:transverseWave:fooA} serves to map electric field (vector) components into bivectors, and the magnetic bivector components into vectors.  For the result to be the same means we must have an additional coupling between the field components.  Writing out \eqnref{eqn:transverseWave:fooA} in terms of the field components we have

\begin{equation}\label{eqn:transverseWave:68}
\begin{aligned}
\BE + I\BB/\sqrt{\mu\epsilon}
&= \kcap (\BE + I\BB/\sqrt{\mu\epsilon} ) \\
&= \kcap \wedge \BE + I (\kcap \wedge \BB)/\sqrt{\mu\epsilon}  \\
&= I \kcap \cross \BE + I^2 (\kcap \cross \BB)/\sqrt{\mu\epsilon}
\end{aligned}
\end{equation}

Equating left and right hand grades we have

\begin{equation}\label{eqn:transverseWave:fooD}
\begin{aligned}
\BE &= -(\kcap \cross \BB)/\sqrt{\mu\epsilon} \\
\BB &= \sqrt{\mu\epsilon} (\kcap \cross \BE)
\end{aligned}
\end{equation}

Since \(\BE\) and \(\BB\) both have the same phase relationships we also have

\begin{equation}\label{eqn:transverseWave:fooDD}
\begin{aligned}
\bcE &= -(\kcap \cross \bcB)/\sqrt{\mu\epsilon} \\
\bcB &= \sqrt{\mu\epsilon} (\kcap \cross \bcE)
\end{aligned}
\end{equation}

With phasors as used in electrical engineering it is usual to allow the fields to have complex values.  Assuming this is allowed here too, taking real parts of \(F\), and separating by grade, we have for the electric and magnetic fields

\begin{equation}\label{eqn:transverseWave:fooX}
\begin{aligned}
\begin{pmatrix}
\BE \\
\BB
\end{pmatrix}
=
\Real
\begin{pmatrix}
\bcE \\
\bcB
\end{pmatrix}
\cos(\Bk \cdot \Bx - \omega t)
+\Imag
\begin{pmatrix}
\bcE \\
\bcB
\end{pmatrix}
\sin(\Bk \cdot \Bx - \omega t)
\end{aligned}
\end{equation}

We will find a slightly different separation into electric and magnetic fields with the geometric phasor.

\section{Geometrized phasor}

Translating from SI units to the CGS units of Jackson the geometric phasor representation of the field was found previously to be

\begin{equation}\label{eqn:transverseWave:fooE}
\begin{aligned}
F = e^{ -I \kcap \omega t } e^{ I \Bk \cdot \Bx } (\bcE + I\bcB/\sqrt{\mu\epsilon})
\end{aligned}
\end{equation}

As above the transverse requirement \(\bcE \cdot \Bk = \bcB \cdot \Bk = 0\) was required.  Application of Maxwell's equation operator should show if we require any additional constraints.  That is

\begin{equation}\label{eqn:transverseWave:88}
\begin{aligned}
0
&= (\spacegrad + \sqrt{\mu\epsilon}\partial_0) F \\
&=
(\spacegrad + \sqrt{\mu\epsilon}\partial_0) e^{ -I \kcap \omega t } e^{ I \Bk \cdot \Bx } (\bcE + I\bcB/\sqrt{\mu\epsilon}) \\
&=
\sum \sigma_m e^{ -I \kcap \omega t } (I k^m) e^{ I \Bk \cdot \Bx } (\bcE + I\bcB/\sqrt{\mu\epsilon})
-I \kcap \sqrt{\mu\epsilon} \frac{\omega}{c} e^{ -I \kcap \omega t } e^{ I \Bk \cdot \Bx } (\bcE + I\bcB/\sqrt{\mu\epsilon}) \\
&=
I \left(\Bk - \kcap \sqrt{\mu\epsilon} \frac{\omega}{c} \right) e^{ -I \kcap \omega t } e^{ I \Bk \cdot \Bx } (\bcE + I\bcB/\sqrt{\mu\epsilon})
\end{aligned}
\end{equation}

This is zero for any combinations of \(\bcE\) or \(\bcB\) since \(\Bk = \kcap \sqrt{\mu\epsilon} \omega/c\).  It therefore appears that this geometric phasor has a fundamentally different nature than the non-geometric version.  We have two exponentials that commute, but due to the difference in grades of the arguments, it does not appear that there is any easy way to express this as an single argument exponential.  Multiplying these out, and using the trig product to sum identities helps shed some light on the differences between the geometric phasor and the one using a generic imaginary.  Starting off we have

\begin{equation}\label{eqn:transverseWave:108}
\begin{aligned}
&e^{ -I \kcap \omega t } e^{ I \Bk \cdot \Bx } \\
&=
(\cos(\omega t) -I\kcap \sin(\omega t)) (\cos(\Bk \cdot \Bx) +I\sin(\Bk \cdot \Bx)) \\
&=
\cos(\omega t)\cos(\Bk \cdot \Bx)
+ \kcap \sin(\omega t)\sin(\Bk \cdot \Bx)
-I\kcap \sin(\omega t)\cos(\Bk \cdot \Bx)
+I \cos(\omega t) \sin(\Bk \cdot \Bx) \\
\end{aligned}
\end{equation}

In this first expansion we see that this product of exponentials has scalar, vector, bivector, and pseudoscalar grades, despite the fact that we have only
vector and bivector terms in the end result.  That will be seen to be due to the transverse nature of \(\calF\) that we multiply with.  Before performing that final multiplication, writing \(C_{-} = \cos(\omega t - \Bk \cdot \Bx)\), \(C_{+} = \cos(\omega t + \Bk \cdot \Bx)\), \(S_{-} = \sin(\omega t - \Bk \cdot \Bx)\), and \(S_{+} = \sin(\omega t + \Bk \cdot \Bx)\), we have

\begin{equation}\label{eqn:transverseWave:fooH}
\begin{aligned}
e^{ -I \kcap \omega t } e^{ I \Bk \cdot \Bx }
&=
\inv{2}
\left(
(C_{-} + C_{+})
+\kcap (C_{-} - C_{+})
-I \kcap (S_{-} + S_{+})
-I (S_{-} - S_{+})
\right)
\end{aligned}
\end{equation}

As an operator the left multiplication of \(\kcap\) on a transverse vector has the action

\begin{equation}\label{eqn:transverseWave:128}
\begin{aligned}
\kcap ( \cdot )
&= \kcap \wedge (\cdot) \\
&= I (\kcap \cross (\cdot)) \\
\end{aligned}
\end{equation}

This gives
\begin{equation}\label{eqn:transverseWave:fooI}
\begin{aligned}
e^{ -I \kcap \omega t } e^{ I \Bk \cdot \Bx }
&=
\inv{2}
\left(
(C_{-} + C_{+})
+(C_{-} - C_{+}) I \kcap \cross
+(S_{-} + S_{+}) \kcap \cross
-I (S_{-} - S_{+})
\right)
\end{aligned}
\end{equation}

Now, lets apply this to the field with \(\calF = \bcE + I\bcB/\sqrt{\mu\epsilon}\).  To avoid dragging around the \(\sqrt{\mu\epsilon}\) factors, let us also temporarily
work with units where \(\mu\epsilon = 1\).  We then have

\begin{equation}\label{eqn:transverseWave:148}
\begin{aligned}
2 e^{ -I \kcap \omega t } e^{ I \Bk \cdot \Bx } \calF
&= (C_{-} + C_{+}) (\bcE + I\bcB) \\
&+ (C_{-} - C_{+}) (I (\kcap \cross \bcE) - \kcap \cross \bcB) \\
&+ (S_{-} + S_{+}) (\kcap \cross \bcE +I (\kcap \cross \bcB))  \\
&+ (S_{-} - S_{+}) (-I \bcE + \bcB)
\end{aligned}
\end{equation}

Rearranging explicitly in terms of the electric and magnetic field components this is
\begin{equation}\label{eqn:transverseWave:168}
\begin{aligned}
2 e^{ -I \kcap \omega t } e^{ I \Bk \cdot \Bx } \calF
&=
 (C_{-} + C_{+}) \bcE
-(C_{-} - C_{+}) (\kcap \cross \bcB)
+(S_{-} + S_{+}) (\kcap \cross \bcE)
+(S_{-} - S_{+}) \bcB \\
&+{I}
\left(
 (C_{-} + C_{+}) \bcB
+(C_{-} - C_{+}) (\kcap \cross \bcE)
+(S_{-} + S_{+}) (\kcap \cross \bcB)
-(S_{-} - S_{+}) \bcE
\right) \\
\end{aligned}
\end{equation}

Quite a mess!  A first observation is that the application of the perpendicularity conditions \eqnref{eqn:transverseWave:fooDD} we have a remarkable reduction in complexity.  That is
%\bcE &= -(\kcap \cross \bcB)/\sqrt{\mu\epsilon} \\
%\bcB &= \sqrt{\mu\epsilon} (\kcap \cross \bcE)
\begin{equation}\label{eqn:transverseWave:188}
\begin{aligned}
2 e^{ -I \kcap \omega t } e^{ I \Bk \cdot \Bx } \calF
&=
 (C_{-} + C_{+}) \bcE
+(C_{-} - C_{+}) \bcE
+(S_{-} + S_{+}) \bcB
+(S_{-} - S_{+}) \bcB
\\
&+{I}
\left(
 (C_{-} + C_{+}) \bcB
+(C_{-} - C_{+}) \bcB
-(S_{-} + S_{+}) \bcE
-(S_{-} - S_{+}) \bcE
\right) \\
\end{aligned}
\end{equation}

This wipes out the receding wave terms leaving only the advanced wave terms, leaving

\begin{equation}\label{eqn:transverseWave:208}
\begin{aligned}
e^{ -I \kcap \omega t } e^{ I \Bk \cdot \Bx } \calF
&=
 C_{-} \bcE
+S_{-} (\kcap \cross \bcE)
+{I}
\left(
 C_{-} \bcB +S_{-} \kcap \cross \bcB
\right) \\
&=
 C_{-} (\bcE + I\bcB)
+S_{-} (\bcB -I\bcE) \\
&=
( C_{-} -I S_{-} ) (\bcE + I\bcB) \\
\end{aligned}
\end{equation}
%\bcE &= -(\kcap \cross \bcB)/\sqrt{\mu\epsilon} \\
%\bcB &= \sqrt{\mu\epsilon} (\kcap \cross \bcE)

We see therefore for this special case of mutually perpendicular (equ-magnitude) field components, our geometric phasor has only the advanced wave term

\begin{equation}\label{eqn:transverseWave:fooG}
\begin{aligned}
e^{ -I \kcap \omega t } e^{ I \Bk \cdot \Bx } \calF &= e^{-I(\omega t - \Bk \cdot \Bx)} \calF
\end{aligned}
\end{equation}

If we pick this as the starting point for the assumed solution, it is clear that the same perpendicularity constraints will follow as in Jackson's treatment, or the GA version of it above.  We have something that is slightly different though, for we have no requirement to take real parts of this simplified geometric phasor, since the result already contains just the vector and bivector terms of the electric and magnetic fields respectively.

A small aside, before continuing.  Having made this observation that we can write the assumed phasor for this transverse field in the form of \eqnref{eqn:transverseWave:fooG} an easier way to demonstrate that the product of exponentials reduces only to the advanced wave term is now clear.  Instead of using \eqnref{eqn:transverseWave:fooDD} we could start back at \eqnref{eqn:transverseWave:fooH} and employ the absorption property \(\kcap \calF = \calF\).  That gives

\begin{equation}\label{eqn:transverseWave:228}
\begin{aligned}
e^{ -I \kcap \omega t } e^{ I \Bk \cdot \Bx } \calF
&=
\inv{2}
\left(
(C_{-} + C_{+})
+(C_{-} - C_{+})
-I (S_{-} + S_{+})
-I (S_{-} - S_{+})
\right) \calF \\
&=
\left( C_{-} -I S_{-} \right) \calF
\end{aligned}
\end{equation}

That is the same result, obtained in a slicker manner.

\section{Explicit split of geometric phasor into advanced and receding parts}

For a more general split of the geometric phasor into advanced and receding wave terms, will there be interdependence between the electric and magnetic field components?   Going back to \eqnref{eqn:transverseWave:fooH}, and rearranging, we have

\begin{equation}\label{eqn:transverseWave:248}
\begin{aligned}
2 e^{ -I \kcap \omega t } e^{ I \Bk \cdot \Bx }
&=
(C_{-} -I S_{-})
+\kcap (C_{-} -I S_{-} )
+(C_{+} +I S_{+})
-\kcap (C_{+} +I S_{+}) \\
\end{aligned}
\end{equation}

So we have

\begin{equation}\label{eqn:transverseWave:fooJ}
\begin{aligned}
e^{ -I \kcap \omega t } e^{ I \Bk \cdot \Bx }
&=
\inv{2}(1 + \kcap)e^{-I(\omega t - \Bk \cdot \Bx)}
+\inv{2}(1 - \kcap)e^{I(\omega t + \Bk \cdot \Bx)}
\end{aligned}
\end{equation}

As observed if we have \(\kcap \calF = \calF\), the result is only the advanced wave term

\begin{equation}\label{eqn:transverseWave:268}
\begin{aligned}
e^{ -I \kcap \omega t } e^{ I \Bk \cdot \Bx } \calF = e^{-I(\omega t - \Bk \cdot \Bx)} \calF
\end{aligned}
\end{equation}

Similarly, with absorption of \(\kcap\) with the opposing sign \(\kcap \calF = -\calF\), we have only the receding wave

\begin{equation}\label{eqn:transverseWave:288}
\begin{aligned}
e^{ -I \kcap \omega t } e^{ I \Bk \cdot \Bx } \calF = e^{I(\omega t + \Bk \cdot \Bx)} \calF
\end{aligned}
\end{equation}

Either of the receding or advancing wave solutions should independently satisfy the Maxwell equation operator.  Let us verify both of these, and verify that for either the \(\pm\) cases the following is a solution and examine the constraints for that to be the case.

\begin{equation}\label{eqn:transverseWave:fooK}
\begin{aligned}
F = \inv{2}(1 \pm \kcap) e^{\pm I(\omega t \pm \Bk \cdot \Bx)} \calF
\end{aligned}
\end{equation}

Now we wish to apply the Maxwell equation operator \(\spacegrad + \sqrt{\mu\epsilon}\partial_0\) to this assumed solution.  That is

\begin{equation}\label{eqn:transverseWave:308}
\begin{aligned}
0
&= (\spacegrad + \sqrt{\mu\epsilon}\partial_0) F \\
&=
\sigma_m \inv{2}(1 \pm \kcap) (\pm I \pm k^m) e^{\pm I(\omega t \pm \Bk \cdot \Bx)} \calF
+ \inv{2}(1 \pm \kcap) (\pm I \sqrt{\mu\epsilon}\omega/c) e^{\pm I(\omega t \pm \Bk \cdot \Bx)} \calF \\
&=
\frac{\pm I}{2}\left(\pm \kcap + \sqrt{\mu\epsilon}\frac{\omega}{c}\right)(1 \pm \kcap) e^{\pm I(\omega t \pm \Bk \cdot \Bx)} \calF
\end{aligned}
\end{equation}

By left multiplication with the conjugate of the Maxwell operator \(\grad - \sqrt{\mu\epsilon}\partial_0\) we have the wave equation operator, and applying that, we have as before, a magnitude constraint on the wave number \(\Bk\)

\begin{equation}\label{eqn:transverseWave:328}
\begin{aligned}
0
&= (\spacegrad - \sqrt{\mu\epsilon}\partial_0) (\spacegrad + \sqrt{\mu\epsilon}\partial_0) F \\
&= (\spacegrad^2 - {\mu\epsilon}\partial_{00}) F \\
&= \frac{-1}{2}(1 \pm \kcap) \left( \Bk^2 - \mu\epsilon\frac{\omega^2}{c^2}\right) e^{\pm I(\omega t \pm \Bk \cdot \Bx)} \calF
\end{aligned}
\end{equation}

So we have as before \(\Abs{\Bk} = \sqrt{\mu\epsilon}\omega/c\).  Substitution into the first order operator result we have
\begin{equation}\label{eqn:transverseWave:348}
\begin{aligned}
0
&= (\spacegrad + \sqrt{\mu\epsilon}\partial_0) F \\
&=
\frac{\pm I}{2}\sqrt{\mu\epsilon}\frac{\omega}{c}\left(\pm \kcap + 1\right)(1 \pm \kcap) e^{\pm I(\omega t \pm \Bk \cdot \Bx)} \calF
\end{aligned}
\end{equation}

Observe that the multivector \(1 \pm \kcap\), when squared is just a multiple of itself

\begin{equation}\label{eqn:transverseWave:368}
\begin{aligned}
(1 \pm \kcap)^2 = 1 + \kcap^2 \pm 2 \kcap = 2 (1 \pm \kcap)
\end{aligned}
\end{equation}

So we have

\begin{equation}\label{eqn:transverseWave:388}
\begin{aligned}
0
&= (\spacegrad + \sqrt{\mu\epsilon}\partial_0) F \\
&=
{\pm I}\sqrt{\mu\epsilon}\frac{\omega}{c}(1 \pm \kcap) e^{\pm I(\omega t \pm \Bk \cdot \Bx)} \calF
\end{aligned}
\end{equation}

So we see that the constraint again on the individual assumed solutions is again that of absorption.  Separately the advanced or receding parts of the geometric phasor as expressed in \eqnref{eqn:transverseWave:fooK} are solutions provided

\begin{equation}\label{eqn:transverseWave:fooN}
\begin{aligned}
\kcap F = \mp F
\end{aligned}
\end{equation}

The geometric phasor is seen to be curious superposition of both advancing and receding states.  Independently we have something pretty much like the standard transverse phasor wave states.  Is this superposition state physically meaningful.  It is a solution to the Maxwell equation (without any constraints on \(\bcE\) and \(\bcB\)).

%\EndArticle
