%
% Copyright � 2012 Peeter Joot.  All Rights Reserved.
% Licenced as described in the file LICENSE under the root directory of this GIT repository.
%

%
%
\mychapter{Charge line element}
\index{line element}
\label{chap:chargeLineElement}
%\date{Nov 23, 2008.  chargeLineElement.tex}

\section{Motivation}

In \citep{purcell1963eam} the electric field for an infinite length charged line element is derived in two ways.  First using summation directly, then with Gauss's law.  Associated with the first was the statement that the field must be radial by symmetry.  This was not obvious to me when initially taking my E\&M course, so I thought it was worth revisiting.

\section{Calculation of electric field for non-infinite length line element}

\imageFigure{../figures/gabook/charge_line_element_figure}{Charge on wire}{fig:chargeLineElement}{0.4}

This calculation will be done with a thickness neglected wire running up and down along the \(y\) axis as illustrated in \cref{fig:chargeLineElement}, where the field is being measured at \(P = r \Be_1\), and the field contributions due to all charge elements \(dq = \lambda dy\) are to be summed.

We want to sum each of the field contributions along the line, so with
%
\begin{equation}\label{eqn:chargeLineElement:20}
\begin{aligned}
d\BE &= \frac{dq \ucap(\theta)}{4 \pi \epsilon_0 R^2} \\
r/R &= \cos\theta \\
dy &= r d(\tan\theta) = r \sec^2 \theta \\
\ucap(\theta) &= \Be_1 e^{i \theta} \\
i &= \Be_1 \Be_2
\end{aligned}
\end{equation}
%
% check:
% 3 pi/2 : e1 e^i\theta = e1 \cos 3\pi/2 + e2\sin 3\pi/2 = -e2
%   pi/2 : e1 e^i\theta = e1 \cos \pi/2 + e2\sin \pi/2   = e2
%      0 : e1 e^i\theta = e1 \cos 0 + e2\sin 0           = e1
%
% 3 pi/2 : e^i\theta = \cos 3\pi/2 + e1e2\sin 3\pi/2     = -e1e2
% pi/2   : e^i\theta = \cos \pi/2 + e1e2\sin \pi/2       = e1e2
% 0      : e^i\theta = \cos 0 + e1e2\sin 0               = 1

Putting things together we have
%
\begin{equation}\label{eqn:chargeLineElement:40}
\begin{aligned}
d\BE
&= \frac{\lambda r \sec^2 \theta \Be_1 e^{i\theta} d\theta}{4 \pi \epsilon_0 r^2 \sec^2 \theta} \\
&= \frac{\lambda \Be_1 e^{i\theta} d\theta}{4 \pi \epsilon_0 r} \\
&= -\frac{\lambda \Be_1 i d(e^{i\theta})}{4 \pi \epsilon_0 r} \\
\end{aligned}
\end{equation}
%
Thus the total field is
\begin{equation}\label{eqn:chargeLineElement:60}
\begin{aligned}
\BE
&= \int d\BE \\
&= -\frac{\lambda \Be_2}{4 \pi \epsilon_0 r} \int d(e^{i\theta}) \\
\end{aligned}
\end{equation}
%
We see that the integration, which has the value
%
\begin{equation}\label{eqn:chargeLineElement:80}
\begin{aligned}
\BE = -\frac{\lambda} {4 \pi \epsilon_0 r} \Be_2 e^{i\delta\theta}
\end{aligned}
\end{equation}
%
The integration range for the infinite wire is \(\theta \in [3\pi/2, \pi/2]\)
so the field for the infinite wire is
%
\begin{equation}\label{eqn:chargeLineElement:100}
\begin{aligned}
\BE
&= -\frac{\lambda} {4 \pi \epsilon_0 r} \Be_2 \left. e^{i\theta} \right\vert^{\theta = \pi/2}_{\theta = 3\pi/2} \\
&= -\frac{\lambda} {4 \pi \epsilon_0 r} \Be_2 (e^{i\pi/2} - e^{3i\pi/2}) \\
&= -\frac{\lambda} {4 \pi \epsilon_0 r} \Be_2 (\Be_1 \Be_2 - (-\Be_1 \Be_2)) \\
&= \frac{\lambda} {2 \pi \epsilon_0 r} \Be_1 \\
\end{aligned}
\end{equation}
% 3 pi/2 : e^i\theta = \cos 3\pi/2 + e1e2\sin 3\pi/2     = -e1e2
% pi/2   : e^i\theta = \cos \pi/2 + e1e2\sin \pi/2       = e1e2

%and \(e^{-i\pi}\)
%\ucap(\theta) &= \Be_1 e^{i \theta} \quad \theta \in [3\pi/2, \pi/2] \\

Invoking symmetry was done in order to work with coordinates, but working with the vector quantities directly
avoids this requirement and gives the general result for any subset of angles.

For a finite length wire all that is required is an angle parametrization of that wire's length
%
\begin{equation}\label{eqn:chargeLineElement:120}
\begin{aligned}
%[y_1, y_2] = r[\tan\theta_1, \tan\theta_2].
[\theta_1, \theta_2] = [\tan^{-1}(y_1/r), \tan^{-1}(y_2/r)]
\end{aligned}
\end{equation}
%
For such a range the exponential difference for the integral is
%
\begin{equation}\label{eqn:chargeLineElement:140}
\begin{aligned}
\left. e^{i\theta} \right \vert_{\theta_1}^{\theta_2}
&= e^{i\theta_2} - e^{i\theta_1} \\
&= e^{i(\theta_1 + \theta_2)/2} \left( e^{i(\theta_2 - \theta_1)/2} -e^{i(\theta_2 - \theta_1)/2} \right) \\
&= 2 i e^{i(\theta_1 + \theta_2)/2} \sin((\theta_2 - \theta_1)/2) \\
\end{aligned}
\end{equation}
%
thus the associated field is
%
\begin{equation}\label{eqn:chargeLineElement:160}
\begin{aligned}
\BE
&= -\frac{\lambda} {2 \pi \epsilon_0 r} \Be_2 i e^{i(\theta_1 + \theta_2)/2} \sin((\theta_2 - \theta_1)/2) \\
&= \frac{\lambda} {2 \pi \epsilon_0 r} \Be_1 e^{i(\theta_1 + \theta_2)/2} \sin((\theta_2 - \theta_1)/2) \\
\end{aligned}
\end{equation}
