%
% Copyright � 2012 Peeter Joot.  All Rights Reserved.
% Licenced as described in the file LICENSE under the root directory of this GIT repository.
%

%
%
\chapter{Expressing wave equation exponential solutions using four vectors}\label{chap:PJwaveFourVector}
\index{wave equation}
\index{four vector}
%\date{Nov 30, 2008.  waveEqn.tex}

\section{Mechanical Wave equation Solutions}

For the unforced wave equation in 3D one wants solutions to

\begin{equation}\label{eqn:wave_eqn:waveEquation}
\begin{aligned}
\left( \inv{\Bv^2} \partial_{tt} - \sum_{j=1}^3 \partial_{jj}\right) \phi = 0
\end{aligned}
\end{equation}

For the single spatial variable case one can verify that
\(\phi = f( \Bx \pm \Abs{\Bv} t)\) is a solution for any function \(f\).  In particular \(\phi = \exp(i (\pm \Abs{\Bv} t + x))\) is a solution.  Similarly
\(\phi = \exp(i (\pm \Abs{\Bv} t + \kcap \cdot \Bx))\) is a solution in the 3D case.

Can the relativistic four vector notation be used to put this in a more symmetric form with respect to time and position?  For the four
vector

\begin{equation}\label{eqn:waveEqn:20}
\begin{aligned}
x = x^\mu \gamma_\mu
\end{aligned}
\end{equation}

Lets try the following as a possible solution to \eqnref{eqn:wave_eqn:waveEquation}

\begin{equation}\label{eqn:waveEqn:40}
\begin{aligned}
\phi = \exp(i k \cdot x)
\end{aligned}
\end{equation}

verifying that this can be a solution, and determining the constraints required on the four vector \(k\).

Observe that

\begin{equation}\label{eqn:waveEqn:60}
\begin{aligned}
x \cdot k = x^\mu k_\mu
\end{aligned}
\end{equation}

so

\begin{equation}\label{eqn:waveEqn:80}
\begin{aligned}
\phi_\mu &= i k_\mu \\
\phi_{\mu\mu} &= (i k_\mu)^2 \phi = -(k_\mu)^2 \phi
\end{aligned}
\end{equation}

Since \(\partial_t = c\partial_0\), we have \(\phi_tt = c^2 \phi_{00}\), and

\begin{equation}\label{eqn:waveEqn:100}
\begin{aligned}
\left( \inv{\Bv^2} \partial_{tt} - \sum_{j=1}^3 \partial_{jj}\right) \phi &=
\left( -\inv{\Bv^2} c^2 {k_0}^2 - \sum_{j=1}^3 -(k_j)^2\right) \phi \\
\end{aligned}
\end{equation}

For equality with zero, and \(\Bbeta = \Bv/c\), we require

\begin{equation}\label{eqn:waveEqn:120}
\begin{aligned}
\Bbeta^2 = \frac{(k_0)^2}{\sum_j (k_j)^2}
\end{aligned}
\end{equation}

Now want the components of \(k = k_\mu \gamma^\mu\) in terms of \(k\) directly.  First

\begin{equation}\label{eqn:waveEqn:140}
\begin{aligned}
k_0 = k \cdot \gamma_0
\end{aligned}
\end{equation}

The spacetime relative vector for \(k\) is

\begin{equation}\label{eqn:waveEqn:160}
\begin{aligned}
\Bk &= k \wedge \gamma_0 = \sum k_\mu \gamma^\mu \wedge \gamma_0 = (\gamma_1)^2 \sum_j k_j \sigma_j \\
\Bk^2 &= (\pm 1)^2 \sum_j (k_j)^2
\end{aligned}
\end{equation}

So the constraint on the four vector parameter \(k\) is
\begin{equation}\label{eqn:waveEqn:180}
\begin{aligned}
\Bbeta^2
&= \frac{(k_0)^2}{\sum_j (k_j)^2} \\
&= \frac{(k \cdot \gamma_0)^2}{(k \wedge \gamma_0)^2} \\
\end{aligned}
\end{equation}

It is interesting to compare this to the relative spacetime bivector for \(x\)

\begin{equation}\label{eqn:waveEqn:200}
\begin{aligned}
v &= \frac{dx}{d\tau} = c \frac{dt}{d\tau} \gamma_0 + \frac{dx^i}{d\tau} \gamma_i \\
v \cdot \gamma^0 &= \frac{dx}{d\tau} \cdot \gamma^0 = c \frac{dt}{d\tau} \\
v \wedge \gamma_0 &= \frac{dx}{d\tau} \wedge \gamma_0 \\
&= \frac{dx^i}{d\tau} \sigma_i \\
&= \frac{dx^i}{dt} \frac{dt}{d\tau} \sigma_i \\
\end{aligned}
\end{equation}

\begin{equation}\label{eqn:waveEqn:220}
\begin{aligned}
\Bv/c
&= \frac{d (x^i \sigma_i) }{dt} \\
&= \frac{v \wedge \gamma_0}{ v \cdot \gamma^0 }
\end{aligned}
\end{equation}

So, for \(\phi = \exp(i k \cdot x)\) to be a solution to the wave equation for a wave traveling with velocity \(\Abs{\Bv}\), the constraint on k
in terms of proper velocity \(v\) is

\begin{equation}\label{eqn:waveEqn:240}
\begin{aligned}
\Abs{\frac{k \wedge \gamma_0}{ k \cdot \gamma^0 }}^{-1} &=
\Abs{\frac{v \wedge \gamma_0}{ v \cdot \gamma^0 }}
\end{aligned}
\end{equation}

So we see the relative spacetime vector of \(k\) has an inverse relationship with the relative spacetime velocity vector \(v = dx/d\tau\).
