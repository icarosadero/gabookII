%
% Copyright � 2012 Peeter Joot.  All Rights Reserved.
% Licenced as described in the file LICENSE under the root directory of this GIT repository.
%

%
%
\chapter{Vector forms of Maxwell's equations as projection and rejection operations}\label{chap:PJMaxwellProj}
\index{Maxwell's equations!projection}
\index{Maxwell's equations!rejection}
%\date{Sept 9, 2008.  vectorMaxwellsProjection.tex}

\section{Vector form of Maxwell's equations}

%FIXME: used \(i\) here as a pseudoscalar and an index depending on context.  Switched to \(I\) instead for the pseudoscalar for clarity ... hope I got them all.

We saw how to extract the tensor formulation of Maxwell's equations
from \(\grad F = J\).  A little bit of play shows how to pick off the divergence
equations we are used to as well.

The end result is that we can pick off two of the eight coordinate equations
with specific product operations.

It is helpful in the following to write \(\grad F\) in index notation
% See: em_bivector_metric_dependencies.pdf
%\eqnref{eqn:vecMaxProj:Fcomp}

\begin{equation}
\grad F = \PD{x^\mu}{E^i} {\gamma^{\mu}}_{i 0} - \epsilon_{i j k} c \PD{x^\mu}{B^i} {\gamma^{\mu}}_{j k}
\end{equation}

In particular, look at the span of the vector, or trivector multiplicands of
the partials of the electric and magnetic field coordinates

\begin{equation}\label{eqn:vecMaxProj:spanpartialE}
{\gamma^{\mu}}_{i 0} \in \Span \{ \gamma_{\mu}, \gamma_{0 i j} \}
\end{equation}

\begin{equation}\label{eqn:vecMaxProj:spanpartialB}
{\gamma^{\mu}}_{j k} \in \Span \{ \gamma_{i j \mu}, \gamma_i \}
\end{equation}

\subsection{Gauss's law for electrostatics}
\index{Gauss's law}
\index{electrostatics}

For extract Gauss's law for electric fields that operation is to take the scalar
parts of the product with \(\gamma^0\).

Dotting with \(\gamma^0\) will pick off the \(\rho\) term from
\(J\)

\begin{equation*}
\frac{J}{\epsilon_0 c} \cdot \gamma^0 = \rho/\epsilon_0,
\end{equation*}

We see that dotting
with \(\gamma_0\) will leave bivector parts contributed by the trivectors in
the span of \eqnref{eqn:vecMaxProj:spanpartialE}.  Similarly the magnetic partials
will contribute bivectors and scalars with this product.  Therefore to
get an equation with strictly scalar parts equal to \(\rho/\epsilon_0\) we need
to compute

\begin{equation}\label{eqn:vectorMaxwellsProjection:20}
\begin{aligned}
\gpgradezero{\left(\grad F - J/\epsilon_0 c\right) \gamma^0}
&= \gpgradezero{\grad \BE \gamma^0} - \rho/\epsilon_0 \\
&= \gpgradezero{\grad E^k {\gamma_{k0}}^0} - \rho/\epsilon_0 \\
&= \gpgradezero{ \gamma^j \partial_{j} E^k \gamma_{k} } - \rho/\epsilon_0 \\
&= {\delta^j}_k \partial_{j} E^k - \rho/\epsilon_0 \\
&= \partial_{k} E^k - \rho/\epsilon_0 \\
\end{aligned}
\end{equation}

This is Gauss's law for electrostatics:
\begin{equation}\label{eqn:vecMaxProj:gausselectro}
\gpgradezero{\left(\grad F - J/\epsilon_0 c\right) \gamma^0} = \spacegrad \cdot \BE - \rho/\epsilon_0 = 0
\end{equation}

\subsection{Gauss's law for magnetostatics}
\index{Gauss's law}
\index{magnetostatics}

Here we are interested in just the trivector terms that are equal to zero that we saw before in \(\grad \wedge \grad \wedge A = 0\).

The divergence like equation of these four can be obtained by dotting with \(\gamma_{123} = \gamma^0 I\).  From the span enumerated
in \eqnref{eqn:vecMaxProj:spanpartialB}, we see that only the \(\BB\) field contributes such a trivector.  An addition scalar part selection is used
to eliminate the bivector that \(J\) contributes.

\begin{equation}\label{eqn:vectorMaxwellsProjection:40}
\begin{aligned}
\gpgradezero{\left(\grad F - J/\epsilon_0 c\right) \cdot \left(\gamma^0 I\right)}
&= (\grad I c \BB) \cdot \left(\gamma^0 I\right) \\
&= \gpgradezero{ \grad I c \BB \gamma^0 I } \\
&= \gpgradezero{ I \grad I c \BB \gamma^0 } \\
&= - c \gpgradezero{ I^2 \grad \BB \gamma^0 } \\
&= c \gpgradezero{ \grad \BB \gamma^0 } \\
&= c \gpgradezero{ \gamma^{\mu} \partial_{\mu} B^k \gamma_k } \\
&= c {\delta^{\mu}}_k \partial_{\mu} B^k \\
&= c \partial_{k} B^k \\
&= 0
\end{aligned}
\end{equation}

This is just the divergence, and therefore yields Gauss's law for magnetostatics:

\begin{equation}\label{eqn:vecMaxProj:gaussmagnet}
\left(\grad F - J/\epsilon_0 c\right) \cdot \left(\gamma^0 I / c \right) = \spacegrad \cdot \BB = 0
\end{equation}

\subsection{Faraday's Law}
\index{Faraday's law}

We have three more trivector equal zero terms to extract from our field equation.

Taking dot products for those remaining three trivectors we have

\begin{equation}\label{eqn:vectorMaxwellsProjection:60}
\begin{aligned}
( \grad F - J/\epsilon_0 c ) \cdot (\gamma^j I)
\end{aligned}
\end{equation}

This will leave a contribution from \(J\), so to exclude that we want to calculate

\begin{equation}\label{eqn:vectorMaxwellsProjection:80}
\begin{aligned}
\gpgradezero{( \grad F - J/\epsilon_0 c ) \cdot (\gamma^j I)}
\end{aligned}
\end{equation}

The electric field contribution gives us
\begin{equation}\label{eqn:vectorMaxwellsProjection:100}
\begin{aligned}
\partial_{\mu} E^k \gpgradezero{ \gamma^{\mu} \gamma_{k0} {\gamma^j}_{0123} }
&=
-\partial_{\mu} E^k (\gamma_0)^2 \gpgradezero{ \gamma^{\mu} \gamma_{k} {\gamma^j}_{123} }
\end{aligned}
\end{equation}

the terms \(\mu = 0\) will not produce a scalar, so this leaves
\begin{equation}\label{eqn:vectorMaxwellsProjection:120}
\begin{aligned}
-\partial_{i} E^k (\gamma_0)^2 \gpgradezero{ \gamma^{i} \gamma_{k} {\gamma^j}_{123} }
&= -\partial_{i} E^k (\gamma_0)^2 (\gamma_k)^2 \epsilon_{jki} \\
&= \partial_{i} E^k \epsilon_{j k i} \\
&= -\partial_{i} E^k \epsilon_{jik} \\
\end{aligned}
\end{equation}

Now, for the magnetic field contribution we have
\begin{equation}\label{eqn:vectorMaxwellsProjection:140}
\begin{aligned}
c \partial_{\mu} B^k \gpgradezero{ \gamma^{\mu} I \gamma_{k0} {\gamma^j} I }
&= - c \partial_{\mu} B^k \gpgradezero{ I \gamma^{\mu} \gamma_{k0} {\gamma^j} I } \\
&= - c \partial_{\mu} B^k \gpgradezero{ I^2 \gamma^{\mu} \gamma_{k0} {\gamma^j} } \\
&= c \partial_{\mu} B^k \gpgradezero{ \gamma^{\mu} \gamma_{k0} {\gamma^j} } \\
\end{aligned}
\end{equation}

For a scalar part we need \(\mu = 0\) leaving
\begin{equation}\label{eqn:vectorMaxwellsProjection:160}
\begin{aligned}
c \partial_{0} B^k \gpgradezero{ \gamma^{0} \gamma_{k0} {\gamma^j} }
&= -\partial_{t} B^k \gpgradezero{ \gamma_{k} {\gamma^j} } \\
&= -\partial_{t} B^k {\delta_{k}}^j \\
&= -\partial_{t} B^j
\end{aligned}
\end{equation}

Combining the results and summing as a vector we have:
\begin{equation}\label{eqn:vectorMaxwellsProjection:180}
\begin{aligned}
\sum \sigma_j \gpgradezero{( \grad F - J/\epsilon_0 c ) \cdot (\gamma^j I)}
&= -\partial_{i} E^k \epsilon_{jik} \sigma_j -\partial_{t} B^j \sigma_j \\
&= -\partial_{j} E^k \epsilon_{i j k} \sigma_i -\partial_{t} B^i \sigma_i \\
&= -\spacegrad \cross \BE - \PD{t}{\BB} \\
&= 0
\end{aligned}
\end{equation}

Moving one term to the opposite side of the equation yields the familiar vector form for Faraday's law

\begin{equation}
\spacegrad \cross \BE = -\PD{t}{\BB}
\end{equation}

\subsection{Ampere Maxwell law}
\index{Ampere Maxwell law}

For the last law, we want the current density, so to extract the Ampere Maxwell law we must have to wedge with \(\gamma^0\).  Such a wedge will eliminate all the trivectors from the span of \eqnref{eqn:vecMaxProj:spanpartialE}, but can contribute pseudoscalar components from the trivectors in \eqnref{eqn:vecMaxProj:spanpartialB}.  Therefore the desired calculation is

\begin{equation}\label{eqn:vectorMaxwellsProjection:200}
\begin{aligned}
\gpgradetwo{\left(\grad F - J/\epsilon_0 c\right) \wedge \gamma^0}
&= \gpgradetwo{ (({\gamma^{\mu}}_{j0}) \wedge \gamma^0 \partial_{\mu} E^j + (\grad I c B) \wedge \gamma^0 } - (\gamma_0)^2 \BJ/\epsilon_0 c \\
&= \gpgradetwo{ -(({\gamma^{0}}_{0j}) \wedge \gamma^0 \partial_{0} E^j + (\grad I c B) \wedge \gamma^0 } - (\gamma_0)^2 \BJ/\epsilon_0 c \\
&= -{\gamma_{j}}^0 \inv{c} \partial_t E^j + \gpgradetwo{ (\grad I c B) \wedge \gamma^0 } - (\gamma_0)^2 \BJ/\epsilon_0 c \\
&= - \frac{(\gamma_0)^2}{c} \PD{t}{\BE} + c \gpgradeone{ \grad I B} \wedge \gamma^0 - (\gamma_0)^2 \BJ/\epsilon_0 c \\
\end{aligned}
\end{equation}

Let us take just that middle term

\begin{equation}\label{eqn:vectorMaxwellsProjection:220}
\begin{aligned}
\gpgradeone{ \grad I B } \wedge \gamma^0
&= -\gpgradeone{ I \gamma^{\mu} \partial_{\mu} B^k \gamma_{k0} } \wedge \gamma^0 \\
&= - \partial_{\mu} B^k \gpgradeone{ \gamma_{0123} \gamma^{\mu} \gamma_{k0} } \wedge \gamma^0 \\
&= \partial_{\mu} B^k \left(\gpgradetwo{ \gamma_{0123} \gamma^{\mu} \gamma_{0} } \cdot \gamma_k\right) \wedge \gamma^0
\end{aligned}
\end{equation}

Here \(\mu \ne 0\) since that leaves just a pseudoscalar in the grade two selection.
\begin{equation}\label{eqn:vectorMaxwellsProjection:240}
\begin{aligned}
\gpgradeone{ \grad I B } \wedge \gamma^0
&= \partial_{j} B^k \left(\gpgradetwo{ \gamma_{0123} \gamma^{j} \gamma_{0} } \cdot \gamma_k\right) \wedge \gamma^0 \\
&= (\gamma_0)^2 \partial_{j} B^k \left(\gpgradetwo{ \gamma_{123} \gamma^{j} } \cdot \gamma_k\right) \wedge \gamma^0 \\
&= (\gamma_0)^2 \partial_{j} B^k \left(\gpgradetwo{ \epsilon^{hkj}\gamma_{hkj} \gamma^{j} } \cdot \gamma_k\right) \wedge \gamma^0 \\
&= \partial_{j} B^k \epsilon^{hkj} (\gamma_0)^2 (\gamma_k)^2 {\gamma_{h}}^0 \\
&= - (\gamma_0)^2 \partial_{j} B^k \epsilon^{hkj} \sigma_h \\
&= (\gamma_0)^2 \spacegrad \cross \BB
\end{aligned}
\end{equation}

%\partial_{j} B^k \epsilon^{hkj} \sigma_h = - \spacegrad \cross \BB
Putting things back together and factoring out the common metric dependent \((\gamma_0)^2\) term we have

\begin{equation}\label{eqn:vectorMaxwellsProjection:260}
\begin{aligned}
- \inv{c} \PD{t}{\BE} + c \spacegrad \cross \BB - \BJ/\epsilon_0 c &= 0 \\
\implies \\
- \inv{c^2} \PD{t}{\BE} + \spacegrad \cross \BB - \BJ/\epsilon_0 c^2 &= 0
\end{aligned}
\end{equation}

With \(\inv{c^2} = \mu_0 \epsilon_0\) this is the Ampere Maxwell law

\begin{equation}
\spacegrad \cross \BB = \mu_0 \left(\BJ + \epsilon_0 \PD{t}{\BE} \right)
\end{equation}

which we can put in the projection form of \eqnref{eqn:vecMaxProj:gausselectro} and \eqnref{eqn:vecMaxProj:gaussmagnet} as:

\begin{equation}\label{eqn:vecMaxProj:amperemaxwell}
\gpgradetwo{\left(\grad F - J/\epsilon_0 c\right) \wedge (\gamma_0/c)} =
\spacegrad \cross \BB - \mu_0 \left(\BJ + \epsilon_0 \PD{t}{\BE} \right) = 0
\end{equation}

\section{Summary of traditional Maxwell's equations as projective operations on Maxwell Equation}

\begin{equation}\label{eqn:vectorMaxwellsProjection:280}
\begin{aligned}
\gpgradezero{\left(\grad F - J/\epsilon_0 c\right) \gamma^0} &= \spacegrad \cdot \BE - \rho/\epsilon_0 = 0 \\
\gpgradezero{\left(\grad F - J/\epsilon_0 c\right) \cdot \left(\gamma^0 I / c \right)} &= \spacegrad \cdot \BB = 0 \\
\sum \sigma_j \gpgradezero{( \grad F - J/\epsilon_0 c ) \cdot (\gamma^j I)} &= -\spacegrad \cross \BE - \PD{t}{\BB} = 0 \\
\gpgradetwo{\left(\grad F - J/\epsilon_0 c\right) \wedge (\gamma_0/c)} &= \spacegrad \cross \BB - \mu_0 \left(\BJ + \epsilon_0 \PD{t}{\BE} \right) = 0
\end{aligned}
\end{equation}

Faraday's law requiring a sum suggests that this can likely be written instead using a rejective operation.  Will leave that as a possible future followup.
