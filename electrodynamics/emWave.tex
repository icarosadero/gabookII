%
% Copyright � 2012 Peeter Joot.  All Rights Reserved.
% Licenced as described in the file LICENSE under the root directory of this GIT repository.
%

%
%
\mychapter{Electrodynamic wave equation solutions}
\label{chap:PJemWave}
\index{wave equation}
%\date{Jan 25, 2009.  emWave.tex}

\section{Motivation}

In \chapcite{PJwaveFourVector} four vector solutions to the mechanical wave
equations were explored.  What was obviously missing from that
was consideration of the special case for \(\Bv^2 = c^2\).

Here solutions to the electrodynamic wave equation will be examined.
Consideration of such solutions in more detail will is expected
to be helpful
as background for the more complex study of quantum (matter) wave equations.

\section{Electromagnetic wave equation solutions}

For electrodynamics our equation to solve is
%
\begin{equation}\label{eqn:emWave:20}
\begin{aligned}
\grad F = J/\epsilon_0 c
\end{aligned}
\end{equation}
%
For the unforced (vacuum) solutions, with
\(F = \grad \wedge A\), and the Coulomb gauge \(\grad \cdot A = 0\) this
reduces to
%
\begin{equation}\label{eqn:emWave:40}
\begin{aligned}
0
&= \left((\gamma^\mu)^2 \partial_{\mu\mu}\right) A  \\
&= \left( \inv{c^2}\partial_{tt} -\partial_{jj} \right) A
\end{aligned}
\end{equation}
%
These equations have the same form as the mechanical wave equation
where the wave velocity \(\Bv^2 = c^2\) is the speed of light
%
\begin{equation}\label{eqn:em_wave:waveEquation}
\begin{aligned}
\left( \inv{\Bv^2} \partial_{tt} - \sum_{j=1}^3 \partial_{jj}\right) \psi = 0
\end{aligned}
\end{equation}
%
\subsection{Separation of variables solution of potential equations}
\index{separation of variables}

Let us solve this using separation of variables, and write \(A^\nu = X Y Z T = \Pi_{\mu} X^{\mu}\)

From this we have
%
\begin{equation}\label{eqn:emWave:60}
\begin{aligned}
\sum_\mu (\gamma^\mu)^2 \frac{(X^\mu)''}{X^\mu} = 0
\end{aligned}
\end{equation}
%
and can proceed with the normal procedure of assuming that a solution can be
found by separately equating each term to a constant.  Writing those
constants explicitly as \((m_\mu)^2\), which we allow to be potentially complex
we have (no sum)
%
\begin{equation}\label{eqn:emWave:80}
\begin{aligned}
X^\mu = \exp\left( \pm \sqrt{(\gamma^\mu)^2} m_\mu x^\mu \right)
\end{aligned}
\end{equation}
%
Now, let \(k_\mu = \pm \sqrt{(\gamma^\mu)^2} m_\mu\), folding any sign variation
and complex factors into these constants.  Our complete solution
is thus
%
\begin{equation}\label{eqn:emWave:100}
\begin{aligned}
\Pi_\mu X^\mu = \exp\left( \sum k_\mu x^\mu \right)
\end{aligned}
\end{equation}
%
However, for this to be a solution, the wave equation imposes the constraint
%
\begin{equation}\label{eqn:emWave:120}
\begin{aligned}
\sum_\mu (\gamma^\mu)^2 (k_\mu)^2 = 0
\end{aligned}
\end{equation}
%
Or
\begin{equation}\label{eqn:emWave:140}
\begin{aligned}
(k_0)^2 - \sum_j (k_j)^2 = 0
\end{aligned}
\end{equation}
%
Summarizing each potential term has a solution expressible in terms of
null "wave-number" vectors \(K_\nu\)
%
\begin{equation}\label{eqn:em_wave:potentialSolution}
\begin{aligned}
A_\nu &= \exp\left( K_\nu \cdot x \right)  \\
\Abs{K_\nu} &= 0
\end{aligned}
\end{equation}
%
\subsection{Faraday bivector and tensor from the potential solutions}
\index{Faraday bivector}

From the components of the potentials
\eqnref{eqn:em_wave:potentialSolution}
we can compute the curl for the complete
field.  That is
%
\begin{equation}\label{eqn:emWave:160}
\begin{aligned}
F &= \grad \wedge A \\
A &= \gamma^\nu \exp\left( K_\nu \cdot x \right)  \\
\end{aligned}
\end{equation}
%
This is
%
\begin{equation}\label{eqn:emWave:180}
\begin{aligned}
F
&= \left(\gamma^\mu \wedge \gamma^\nu\right) \partial_\mu \exp\left( K_\nu \cdot x \right) \\
&= \left(\gamma^\mu \wedge \gamma^\nu\right) \partial_\mu \exp\left( \gamma^\alpha K_{\nu\alpha} \cdot \gamma_\sigma x^\sigma \right) \\
&= \left(\gamma^\mu \wedge \gamma^\nu\right) \partial_\mu \exp\left( K_{\nu\sigma} x^\sigma \right) \\
%&= \left(\gamma^\mu \wedge \gamma^\nu\right) K_{\nu\sigma} \delta_\mu\sigma \exp\left( K_{\nu\sigma} x^\sigma \right) \\
&= \left(\gamma^\mu \wedge \gamma^\nu\right) K_{\nu\mu} \exp\left( K_{\nu\sigma} x^\sigma \right) \\
&= \left(\gamma^\mu \wedge \gamma^\nu\right) K_{\nu\mu} \exp\left( K_{\nu} \cdot x \right) \\
&= \left(\gamma^\mu \wedge \gamma^\nu\right)
\inv{2} \left( K_{\nu\mu} \exp\left( K_{\nu} \cdot x \right) - K_{\mu\nu} \exp\left( K_{\mu} \cdot x \right) \right) \\
\end{aligned}
\end{equation}
%
Writing our field in explicit tensor form
%
\begin{equation}\label{eqn:emWave:200}
\begin{aligned}
F = F_{\mu\nu} \gamma^\mu \wedge \gamma^\nu
\end{aligned}
\end{equation}
%
our vacuum solution is therefore
%
\begin{equation}\label{eqn:emWave:220}
\begin{aligned}
F_{\mu\nu} &= \inv{2} \left( K_{\nu\mu} \exp\left( K_{\nu} \cdot x \right) - K_{\mu\nu} \exp\left( K_{\mu} \cdot x \right) \right)
\end{aligned}
\end{equation}
%
but subject to the null wave number and Lorentz gauge constraints
%
\begin{equation}\label{eqn:emWave:240}
\begin{aligned}
\Abs{K_\mu} &= 0 \\
\grad \cdot \left(\gamma^\mu \exp\left( K_\mu \cdot x \right)\right) &= 0
\end{aligned}
\end{equation}
%
\subsection{Examine the Lorentz gauge constraint}
\index{Lorentz gauge}

That Lorentz gauge constraint on the potential is a curious looking beastie.  Let us expand that out in full to examine it closer
%
\begin{equation}\label{eqn:emWave:260}
\begin{aligned}
\grad \cdot \left(\gamma^\mu \exp\left( K_\mu \cdot x \right)\right)
&= \gamma^\alpha \partial_\alpha \cdot \left(\gamma^\mu \exp\left( K_\mu \cdot x \right)\right)  \\
&= \sum_\mu (\gamma^\mu)^2 \partial_\mu \exp\left( K_\mu \cdot x \right) \\
&= \sum_\mu (\gamma^\mu)^2 \partial_\mu \exp\left( \sum \gamma^\nu K_{\mu\nu} \cdot \gamma_\alpha x^\alpha \right) \\
&= \sum_\mu (\gamma^\mu)^2 \partial_\mu \exp\left( \sum K_{\mu\alpha} x^\alpha \right) \\
&= \sum_\mu (\gamma^\mu)^2 K_{\mu\mu} \exp\left( K_\mu \cdot x \right)
\end{aligned}
\end{equation}
%
If this must be zero for any \(x\) it must also be zero for \(x =0\), so the Lorentz gauge imposes an additional restriction on the
wave number four vectors \(K_\mu\)
%
\begin{equation}\label{eqn:emWave:280}
\begin{aligned}
\sum_\mu (\gamma^\mu)^2 K_{\mu\mu} = 0
\end{aligned}
\end{equation}
%
Expanding in time and spatial coordinates this is
%
\begin{equation}\label{eqn:emWave:300}
\begin{aligned}
K_{00} - \sum_j K_{jj} = 0
\end{aligned}
\end{equation}
%
One obvious way to satisfy this is to require that the tensor \(K_{\mu\nu}\) be diagonal, but since we also have the null vector requirement
on each of the \(K_\mu\) four vectors it is not clear that this is an acceptable choice.

\subsection{Summarizing so far}

We have found that our field solution has the form
%
\begin{equation}\label{eqn:em_wave:WorkingSolution}
\begin{aligned}
F_{\mu\nu} &= \inv{2} \left( K_{\nu\mu} \exp\left( K_{\nu} \cdot x \right) - K_{\mu\nu} \exp\left( K_{\mu} \cdot x \right) \right)
\end{aligned}
\end{equation}
%
Where the vectors \(K_\mu\) have coordinates
\begin{equation}\label{eqn:em_wave:WorkingSolutionDefinitions}
\begin{aligned}
K_\mu &= \gamma^\nu K_{\mu\nu}
%x &= \gamma_\mu x^\mu
\end{aligned}
\end{equation}
%
This last allows us to write the field tensor completely in tensor formalism
%
\begin{equation}\label{eqn:emWave:320}
\begin{aligned}
F_{\mu\nu} &= \inv{2} \left( K_{\nu\mu} \exp\left( K_{\nu\sigma} x^\sigma \right) - K_{\mu\nu} \exp\left( K_{\mu\sigma} x^\sigma \right) \right)
\end{aligned}
\end{equation}
%
Note that we also require the constraints
%
\begin{equation}\label{eqn:em_wave:WorkingSolutionConstraints}
\begin{aligned}
0 &= \sum_\mu (\gamma^\mu)^2 K_{\mu\mu} \\
0 &= \sum_\mu (\gamma^\mu)^2 (K_{\nu\mu})^2
\end{aligned}
\end{equation}
%
Alternately, calling out the explicit space time split of the constraint, we can
remove the explicit \(\gamma^\mu\) factors
%
\begin{equation}\label{eqn:emWave:340}
\begin{aligned}
0 = K_{00} - \sum_j K_{jj} = (K_{00})^2 - \sum_j (K_{jj} )^2
\end{aligned}
\end{equation}
%
%If each of \(K_{\mu0} \ne 0\) we could alternately remove the explicit \(\gamma^\mu\) factors and write
%
%\begin{align*}
%1 = \sum_j \frac{K_{jj}}{K_{00}} = \sum_j \left(\frac{K_{jj}}{K_{00}}\right)^2
%\end{align*}
%
%Is this any better?

\section{Looking for more general solutions}

\subsection{Using mechanical wave solutions as a guide}

In the mechanical wave equation, we had exponential solutions of the form
%
\begin{equation}\label{eqn:emWave:360}
\begin{aligned}
f(\Bx,t) = \exp\left( \Bk \cdot \Bx + \omega t \right)
\end{aligned}
\end{equation}
%
which were solutions to \eqnref{eqn:em_wave:waveEquation} provided that
%
\begin{equation}\label{eqn:emWave:380}
\begin{aligned}
\inv{\Bv^2} \omega^2 - \Bk^2 = 0.
\end{aligned}
\end{equation}
%
This meant that
\begin{equation}\label{eqn:emWave:400}
\begin{aligned}
\omega = \pm \Abs{\Bv} \Abs{\Bk}
\end{aligned}
\end{equation}
%
and our function takes the (hyperbolic) form, or (sinusoidal) form respectively
%
\begin{equation}\label{eqn:emWave:420}
\begin{aligned}
f(\Bx,t) &= \exp\left( \Abs{\Bk}\left( \kcap \cdot \Bx \pm \Abs{\Bv} t \right) \right) \\
f(\Bx,t) &= \exp\left( i \Abs{\Bk}\left( \kcap \cdot \Bx \pm \Abs{\Bv} t \right) \right)
\end{aligned}
\end{equation}
%
Fourier series superposition of the latter solutions can be used to express any spatially periodic function, while Fourier transforms
can be used to express the non-periodic cases.

These superpositions, subject to boundary value conditions, allow for writing solutions to the wave equation in the form
%
\begin{equation}\label{eqn:em_wave:generalWaveSolution}
\begin{aligned}
f(\Bx,t) &= g\left( \kcap \cdot \Bx \pm \Abs{\Bv} t \right)
\end{aligned}
\end{equation}
%
Showing this logically follows from the original separation of variables approach has not been done.   However, despite this,
it is
simple enough to confirm that,
this more general function does satisfy the unforced wave equation \eqnref{eqn:em_wave:waveEquation}.

TODO: as followup here would like to go through the exercise of showing
that the solution of \eqnref{eqn:em_wave:generalWaveSolution} follows from a Fourier transform superposition.  Intuition says this is
possible, and I have said so without backing up the statement.

\subsection{Back to the electrodynamic case}

Using the above generalization argument as a guide we should be able to do something similar for the electrodynamic wave solution.

We want to solve for \(F\) the following gradient equation for the field in free space
%
\begin{equation}\label{eqn:em_wave:maxwell}
\begin{aligned}
\grad F = 0
\end{aligned}
\end{equation}
%
Let us suppose that the following is a solution and find the required constraints
%
\begin{equation}\label{eqn:em_wave:testSol}
\begin{aligned}
F = \gamma^\mu \wedge \gamma^\nu \left( K_{\mu\nu} f( x \cdot K_\mu ) -K_{\nu\mu} f( x \cdot K_\nu ) \right)
\end{aligned}
\end{equation}
%
We have two different grade equations built into Maxwell's equation \eqnref{eqn:em_wave:maxwell}, one of which is the vector equation, and the other
trivector.  Those are respectively
%
\begin{equation}\label{eqn:emWave:440}
\begin{aligned}
\grad \cdot F &= 0 \\
\grad \wedge F &= 0
\end{aligned}
\end{equation}
%
\subsubsection{zero wedge}

For the grade three term we have we can substitute \eqnref{eqn:em_wave:testSol} and see what comes out
%
\begin{equation}\label{eqn:emWave:460}
\begin{aligned}
\grad \wedge F
&=
\left( \gamma^\alpha \wedge \gamma^\mu \wedge \gamma^\nu\right)
\partial_\alpha \left( K_{\mu\nu} f( x \cdot K_\mu ) -K_{\nu\mu} f( x \cdot K_\nu ) \right) \\
\end{aligned}
\end{equation}
%
For the partial we will want the following
%
\begin{equation}\label{eqn:emWave:480}
\begin{aligned}
\partial_\mu ( x \cdot K_\beta )
&= \partial_\mu ( x^\nu \gamma_\nu \cdot K_{\beta\sigma} \gamma^\sigma ) \\
&= \partial_\mu ( x^\sigma K_{\beta\sigma} \\
&= K_{\beta\mu}
\end{aligned}
\end{equation}
%
and application of this with the chain rule we have
%
\begin{equation}\label{eqn:emWave:500}
\begin{aligned}
\grad \wedge F
&=
%\partial_\alpha ( x \cdot K_\beta ) &= K_{\beta\alpha}
\left( \gamma^\alpha \wedge \gamma^\mu \wedge \gamma^\nu\right)
\left( K_{\mu\nu} K_{\mu\alpha} f'( x \cdot K_\mu ) -K_{\nu\mu} K_{\nu\alpha} f'( x \cdot K_\nu ) \right) \\
&=
2\left( \gamma^\alpha \wedge \gamma^\mu \wedge \gamma^\nu\right) K_{\mu\nu} K_{\mu\alpha} f'( x \cdot K_\mu )
\\
\end{aligned}
\end{equation}
%
So, finally for this to be zero uniformly for all \(f\), we require
%
\begin{equation}\label{eqn:emWave:520}
\begin{aligned}
K_{\mu\nu} K_{\mu\alpha} = 0
\end{aligned}
\end{equation}
%
\subsubsection{zero divergence}
\index{divergence}

Now for the divergence term, corresponding to the current four vector condition \(J = 0\), we have
%
\begin{equation}\label{eqn:emWave:540}
\begin{aligned}
&\grad \cdot F \\
&= \gamma^\alpha \cdot
\left(\gamma^\mu \wedge \gamma^\nu\right) \partial_\alpha \left( K_{\mu\nu} f( x \cdot K_\mu ) -K_{\nu\mu} f( x \cdot K_\nu ) \right) \\
&=
(\gamma_\alpha)^2
%\gamma_\alpha \cdot \left(\gamma^\mu \wedge \gamma^\nu\right)
\left( \gamma^\nu {\delta_\alpha}^\mu -\gamma^\mu {\delta_\alpha}^\nu \right)
\partial_\alpha \left( K_{\mu\nu} f( x \cdot K_\mu ) -K_{\nu\mu} f( x \cdot K_\nu ) \right) \\
&=
\left(
(\gamma_\mu)^2 \gamma^\nu \partial_\mu
-(\gamma_\nu)^2 \gamma^\mu \partial_\nu
\right)
\left( K_{\mu\nu} f( x \cdot K_\mu ) -K_{\nu\mu} f( x \cdot K_\nu ) \right) \\
&=
(\gamma_\mu)^2 \gamma^\nu \partial_\mu \left( K_{\mu\nu} f( x \cdot K_\mu ) -K_{\nu\mu} f( x \cdot K_\nu ) \right)
-(\gamma_\mu)^2 \gamma^\nu \partial_\mu \left( K_{\nu\mu} f( x \cdot K_\nu ) -K_{\mu\nu} f( x \cdot K_\mu ) \right) \\
&= 2 (\gamma_\mu)^2 \gamma^\nu \partial_\mu \left( K_{\mu\nu} f( x \cdot K_\mu ) -K_{\nu\mu} f( x \cdot K_\nu ) \right) \\
\end{aligned}
\end{equation}
%
Application of the chain rule, and \(\partial_\mu ( x \cdot K_\beta ) = K_{\beta\mu}\), gives us
%
\begin{equation}\label{eqn:emWave:560}
\begin{aligned}
\grad \cdot F
&= 2 (\gamma_\mu)^2 \gamma^\nu \left( K_{\mu\nu} K_{\mu\mu} f'( x \cdot K_\mu ) -K_{\nu\mu} K_{\nu\mu} f'( x \cdot K_\nu ) \right) \\
\end{aligned}
\end{equation}
%
For \(\mu = \nu\) this is zero, which is expected since that should follow from the wedge product itself, but for the \(\mu \ne \nu\)
case it is not clear cut.

Damn.  On paper I missed some terms and it all canceled out nicely giving only a condition on \(K_{\mu\nu}\) from the wedge term.  The only
conclusion possible is that we require \(x \cdot K_\nu = x \cdot K_\mu\) for this form of solution, and therefore need to restrict the
test solution to a fixed spacetime direction.

\section{Take II.  A bogus attempt at a less general plane wave like solution}

Let us try instead
%
\begin{equation}\label{eqn:em_wave:solutionTensor}
\begin{aligned}
F = \gamma^\mu \wedge \gamma^\nu A_{\mu\nu} f( x \cdot k )
\end{aligned}
\end{equation}
%
and see if we can find conditions on the vector \(k\), and the tensor \(A\) that make this a solution to the unforced Maxwell equation \eqnref{eqn:em_wave:maxwell}.

\subsection{curl term}
\index{curl}

Taking the curl is straightforward
%
\begin{equation}\label{eqn:emWave:580}
\begin{aligned}
\grad \wedge F
&= \gamma^\alpha \wedge \gamma^\mu \wedge \gamma^\nu \partial_\alpha A_{\mu\nu} f( x \cdot k ) \\
&= \gamma^\alpha \wedge \gamma^\mu \wedge \gamma^\nu A_{\mu\nu} \partial_\alpha f( x^\sigma k_\sigma ) \\
&= \gamma^\alpha \wedge \gamma^\mu \wedge \gamma^\nu A_{\mu\nu} k_\alpha f'( x \cdot k ) \\
&= \inv{2} \gamma^\alpha \wedge \gamma^\mu \wedge \gamma^\nu (A_{\mu\nu} - A_{\nu\mu} ) k_\alpha f'( x \cdot k ) \\
\end{aligned}
\end{equation}
%
Curiously, the only condition that this yields is that we have
%
\begin{equation}\label{eqn:emWave:600}
\begin{aligned}
A_{\mu\nu} - A_{\nu\mu} = 0
\end{aligned}
\end{equation}
%
which is a symmetry requirement for the tensor
%
\begin{equation}\label{eqn:emWave:620}
\begin{aligned}
A_{\mu\nu} = A_{\nu\mu}
\end{aligned}
\end{equation}
%
\subsection{divergence term}

Now for the divergence
%
\begin{equation}\label{eqn:emWave:640}
\begin{aligned}
\grad \cdot F
&= \gamma_\alpha \cdot (\gamma^\mu \wedge \gamma^\nu) \partial^\alpha A_{\mu\nu} f( x_\sigma k^\sigma ) \\
&= \left( {\delta_\alpha}^\mu \gamma^\nu -{\delta_\alpha}^\nu \gamma^\mu \right) k^\alpha A_{\mu\nu} f'( x \cdot k ) \\
&=
 \gamma^\nu k^\mu A_{\mu\nu} f'( x \cdot k )
-\gamma^\mu k^\nu A_{\mu\nu} f'( x \cdot k )
\\
&= \gamma^\nu k^\mu (A_{\mu\nu} -A_{\nu\mu}) f'( x \cdot k )
\end{aligned}
\end{equation}
%
So, again, as in the divergence part of Maxwell's equation for the vacuum (\(\grad F = 0\)), we require, and it is sufficient that
%
\begin{equation}\label{eqn:emWave:660}
\begin{aligned}
A_{\mu\nu} -A_{\nu\mu} = 0,
\end{aligned}
\end{equation}
%
for \eqnref{eqn:em_wave:solutionTensor} to be a solution.  This is somewhat surprising since I would not have expected a symmetric tensor to fall out of
the analysis.

Actually, this is more than surprising and amounts to a requirement that the field solution is zero.  Going back to the proposed solution we have
%
\begin{equation}\label{eqn:emWave:680}
\begin{aligned}
F
&= \gamma^\mu \wedge \gamma^\nu A_{\mu\nu} f( x \cdot k ) \\
&= \gamma^\mu \wedge \gamma^\nu \inv{2} (A_{\mu\nu} - A_{\nu\mu})f( x \cdot k ) \\
\end{aligned}
\end{equation}
%
So, any symmetric components of the tensor \(A\) automatically cancel out.

\section{Summary}

A few dead ends have been chased and I am left with the original attempt summarized by
\eqnref{eqn:em_wave:WorkingSolution},
\eqnref{eqn:em_wave:WorkingSolutionDefinitions}, and
\eqnref{eqn:em_wave:WorkingSolutionConstraints}.

It appears that the TODO noted above to attempt the Fourier transform treatment will likely be required to put these exponentials into a more general form.
I had also intended to try to cover phase and group velocities for myself here but took too much time chasing the dead ends.  Will have to leave that
to another day.
