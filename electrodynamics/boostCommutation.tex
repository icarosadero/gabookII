%
% Copyright � 2012 Peeter Joot.  All Rights Reserved.
% Licenced as described in the file LICENSE under the root directory of this GIT repository.
%

%
%
%\input{../peeter_prologue_print.tex}
%\input{../peeter_prologue_widescreen.tex}

\mychapter{Multivector commutators and Lorentz boosts}
\index{commutator}
\index{Lorentz boost}
\label{chap:boostCommutation}

%\blogpage{http://sites.google.com/site/peeterjoot/math2010/boostCommutation.pdf}
%\date{Oct 30, 2010}
%\revisionInfo{boostCommutation.tex}

\beginArtWithToc
%\beginArtNoToc

\section{Motivation}

In some reading there I found that the electrodynamic field components transform in a reversed sense to that of vectors, where instead of the perpendicular to the boost direction remaining unaffected, those are the parts that are altered.

To explore this, look at the Lorentz boost action on a multivector, utilizing symmetric and antisymmetric products to split that vector into portions effected and unaffected by the boost.  For the bivector (electrodynamic case) and the four vector case, examine how these map to dot and wedge (or cross) products.

The underlying motivator for this boost consideration is an attempt to see where equation (6.70) of \citep{desai2009quantum} comes from.  We get to this by the very end.

\section{Guts}
\subsection{Structure of the bivector boost}

Recall that we can write our Lorentz boost in exponential form with
%
\begin{equation}\label{eqn:boostCommutator:1}
\begin{aligned}
L &= e^{\alpha \Bsigma/2} \\
X' &= L^\dagger X L,
\end{aligned}
\end{equation}
%
where \(\Bsigma\) is a spatial vector.  This works for our bivector field too, assuming the composite transformation is an outermorphism of the transformed four vectors.  Applying the boost to both the gradient and the potential our transformed field is then
%
\begin{equation}\label{eqn:boostCommutation:72}
\begin{aligned}
F'
&= \grad' \wedge A' \\
&= (L^\dagger \grad L) \wedge (L^\dagger A L) \\
&= \inv{2} \left(
(L^\dagger \rgrad L) (L^\dagger A L)
-
(L^\dagger A L) (L^\dagger \lgrad L)
\right) \\
&= \inv{2} L^\dagger \left( \rgrad A - A \lgrad \right) L  \\
&= L^\dagger (\grad \wedge A) L.
\end{aligned}
\end{equation}
%
Note that arrows were used briefly to indicate that the partials of the gradient are still acting on \(A\) despite their vector components being to one side.  We are left with the very simple transformation rule
%
\begin{equation}\label{eqn:boostCommutator:2}
\begin{aligned}
F' = L^\dagger F L,
\end{aligned}
\end{equation}
%
which has exactly the same structure as the four vector boost.

\subsection{Employing the commutator and anticommutator to find the parallel and perpendicular components}

If we apply the boost to a four vector, those components of the four vector that commute with the spatial direction \(\Bsigma\) are unaffected.  As an example, which also serves to ensure we have the sign of the rapidity angle \(\alpha\) correct, consider \(\Bsigma = \Bsigma_1\).  We have
%
\begin{equation}\label{eqn:boostCommutator:3}
\begin{aligned}
X' =
e^{-\alpha \Bsigma/2} (
x^0 \gamma_0 +
x^1 \gamma_1 +
x^2 \gamma_2 +
x^3 \gamma_3 ) (\cosh \alpha/2 + \gamma_1 \gamma_0 \sinh \alpha/2 )
\end{aligned}
\end{equation}
%
We observe that the scalar and \(\Bsigma_1 = \gamma_1 \gamma_0\) components of the exponential commute with \(\gamma_2\) and \(\gamma_3\) since there is no vector in common, but that \(\Bsigma_1\) anticommutes with \(\gamma_0\) and \(\gamma_1\).  We can therefore write
%
\begin{equation}\label{eqn:boostCommutation:92}
\begin{aligned}
X'
&=
x^2 \gamma_2 +
x^3 \gamma_3
+(
x^0 \gamma_0 +
x^1 \gamma_1 +
) (\cosh \alpha + \gamma_1 \gamma_0 \sinh \alpha ) \\
&=
x^2 \gamma_2 +
x^3 \gamma_3
+
\gamma_0 ( x^0 \cosh\alpha - x^1 \sinh \alpha )
+
\gamma_1 ( x^1 \cosh\alpha - x^0 \sinh \alpha )
\end{aligned}
\end{equation}
%
reproducing the familiar matrix result should we choose to write it out.  How can we express the commutation property without resorting to components.  We could write the four vector as a spatial and timelike component, as in
%
\begin{equation}\label{eqn:boostCommutator:4}
\begin{aligned}
X = x^0 \gamma_0 + \Bx \gamma_0,
\end{aligned}
\end{equation}
%
and further separate that into components parallel and perpendicular to the spatial unit vector \(\Bsigma\) as
%
\begin{equation}\label{eqn:boostCommutator:5}
\begin{aligned}
X = x^0 \gamma_0 + (\Bx \cdot \Bsigma) \Bsigma \gamma_0 + (\Bx \wedge \Bsigma) \Bsigma \gamma_0.
\end{aligned}
\end{equation}
%
However, it would be nicer to group the first two terms together, since they are ones that are affected by the transformation.  It would also be nice to not have to resort to spatial dot and wedge products, since we get into trouble too easily if we try to mix dot and wedge products of four vector and spatial vector components.

What we can do is employ symmetric and antisymmetric products (the anticommutator and commutator respectively).  Recall that we can write any multivector product this way, and in particular
%
\begin{equation}\label{eqn:boostCommutator:6}
\begin{aligned}
M \Bsigma = \inv{2} (M \Bsigma  + \Bsigma M) + \inv{2} (M \Bsigma - \Bsigma M).
\end{aligned}
\end{equation}
%
Left multiplying by the unit spatial vector \(\Bsigma\) we have
\begin{equation}\label{eqn:boostCommutator:7}
\begin{aligned}
M = \inv{2} (M + \Bsigma M \Bsigma) + \inv{2} (M - \Bsigma M \Bsigma) =
\inv{2} \symmetric{M}{\Bsigma} \Bsigma + \inv{2} \antisymmetric{M}{\Bsigma} \Bsigma.
\end{aligned}
\end{equation}
%
When \(M = \Ba\) is a spatial vector this is our familiar split into parallel and perpendicular components with the respective projection and rejection operators
%
\begin{equation}\label{eqn:boostCommutator:8}
\begin{aligned}
\Ba = \inv{2} \symmetric{\Ba}{\Bsigma} \Bsigma + \inv{2} \antisymmetric{\Ba}{\Bsigma} \Bsigma = (\Ba \cdot \Bsigma) \Bsigma + (\Ba \wedge \Bsigma) \Bsigma.
\end{aligned}
\end{equation}
%
However, the more general split employing symmetric and antisymmetric products in \eqnref{eqn:boostCommutator:7}, is something we can use for our four vector and bivector objects too.

Observe that we have the commutation and anti-commutation relationships
%
\begin{equation}\label{eqn:boostCommutator:9}
\begin{aligned}
\left( \inv{2} \symmetric{M}{\Bsigma} \Bsigma \right) \Bsigma &= \Bsigma \left( \inv{2} \symmetric{M}{\Bsigma} \Bsigma \right) \\
\left( \inv{2} \antisymmetric{M}{\Bsigma} \Bsigma \right) \Bsigma &= -\Bsigma \left( \inv{2} \antisymmetric{M}{\Bsigma} \Bsigma \right).
\end{aligned}
\end{equation}
%
This split therefore serves to separate the multivector object in question nicely into the portions that are acted on by the Lorentz boost, or left unaffected.

\subsection{Application of the symmetric and antisymmetric split to the bivector field}

Let us apply \eqnref{eqn:boostCommutator:7} to the spacetime event \(X\) again with an x-axis boost \(\sigma = \sigma_1\).  The anticommutator portion of X in this boost direction is
%
\begin{equation}\label{eqn:boostCommutation:112}
\begin{aligned}
\inv{2} \symmetric{X}{\Bsigma_1} \Bsigma_1
&=
\inv{2} \left(
\left(
x^0 \gamma_0 +
x^1 \gamma_1 +
x^2 \gamma_2 +
x^3 \gamma_3 \right)
+
\gamma_1 \gamma_0
\left(
x^0 \gamma_0 +
x^1 \gamma_1 +
x^2 \gamma_2 +
x^3 \gamma_3 \right)
\gamma_1 \gamma_0
\right) \\
&=
x^2 \gamma_2 + x^3 \gamma_3,
\end{aligned}
\end{equation}
%
whereas the commutator portion gives us
\begin{equation}\label{eqn:boostCommutation:132}
\begin{aligned}
\inv{2} \antisymmetric{X}{\Bsigma_1} \Bsigma_1
&=
\inv{2} \left(
\left(
x^0 \gamma_0 +
x^1 \gamma_1 +
x^2 \gamma_2 +
x^3 \gamma_3 \right)
-
\gamma_1 \gamma_0
\left(
x^0 \gamma_0 +
x^1 \gamma_1 +
x^2 \gamma_2 +
x^3 \gamma_3 \right)
\gamma_1 \gamma_0
\right) \\
&=
x^0 \gamma_0 + x^1 \gamma_1.
\end{aligned}
\end{equation}
%
We have seen that only these commutator portions are acted on by the boost.  We have therefore found the desired logical grouping of the four vector \(X\) into portions that are left unchanged by the boost and those that are affected.  That is
%
\begin{equation}\label{eqn:boostCommutator:15}
\begin{aligned}
\inv{2} \antisymmetric{X}{\Bsigma} \Bsigma &= x^0 \gamma_0 + (\Bx \cdot \Bsigma) \Bsigma \gamma_0  \\
\inv{2} \symmetric{X}{\Bsigma} \Bsigma &= (\Bx \wedge \Bsigma) \Bsigma \gamma_0
\end{aligned}
\end{equation}
%
Let us now return to the bivector field \(F = \grad \wedge A = \BE + I c \BB\), and split that multivector into boostable and unboostable portions with the commutator and anticommutator respectively.

Observing that our pseudoscalar \(I\) commutes with all spatial vectors we have for the anticommutator parts that will not be affected by the boost
%
\begin{equation}\label{eqn:boostCommutator:20}
\begin{aligned}
\inv{2} \symmetric{\BE + I c \BB}{\Bsigma} \Bsigma &= (\BE \cdot \Bsigma) \Bsigma + I c (\BB \cdot \Bsigma) \Bsigma,
\end{aligned}
\end{equation}
%
and for the components that will be boosted we have
\begin{equation}\label{eqn:boostCommutator:21}
\begin{aligned}
\inv{2} \antisymmetric{\BE + I c \BB}{\Bsigma} \Bsigma &= (\BE \wedge \Bsigma) \Bsigma + I c (\BB \wedge \Bsigma) \Bsigma.
\end{aligned}
\end{equation}
%
For the four vector case we saw that the components that lay ``perpendicular'' to the boost direction, were unaffected by the boost.  For the field we see the opposite, and the components of the individual electric and magnetic fields that are parallel to the boost direction are unaffected.
%
Our boosted field is therefore
\begin{equation}\label{eqn:boostCommutator:152}
\begin{aligned}
F' =
(\BE \cdot \Bsigma) \Bsigma + I c (\BB \cdot \Bsigma) \Bsigma
+
\left(
(\BE \wedge \Bsigma) \Bsigma + I c (\BB \wedge \Bsigma) \Bsigma
\right) \left( \cosh \alpha + \Bsigma \sinh \alpha \right)
\end{aligned}
\end{equation}
%
Focusing on just the non-parallel terms we have
\begin{equation}\label{eqn:boostCommutation:172}
\begin{aligned}
&\left(
(\BE \wedge \Bsigma) \Bsigma + I c (\BB \wedge \Bsigma) \Bsigma
\right) \left( \cosh \alpha + \Bsigma \sinh \alpha \right) \\
&=
(\BE_\perp + I c \BB_\perp ) \cosh\alpha
+(I \BE \cross \Bsigma - c \BB \cross \Bsigma ) \sinh\alpha \\
&=
\BE_\perp \cosh\alpha - c (\BB \cross \Bsigma ) \sinh\alpha
+ I ( c \BB_\perp \cosh\alpha + (\BE \cross \Bsigma) \sinh\alpha ) \\
&=
\gamma \left(
\BE_\perp - c (\BB \cross \Bsigma ) \Abs{\Bv}/c
+ I ( c \BB_\perp + (\BE \cross \Bsigma) \Abs{\Bv}/c)
\right)
\end{aligned}
\end{equation}
%
A final regrouping gives us
%
\begin{equation}\label{eqn:boostCommutator:50}
\begin{aligned}
F'
&=
\BE_\parallel + \gamma \left( \BE_\perp - \BB \cross \Bv \right)
+I c \left( \BB_\parallel + \gamma \left( \BB_\perp + \BE \cross \Bv/c^2 \right) \right)
\end{aligned}
\end{equation}
%
In particular when we consider the proton, electron system as in equation (6.70) of \citep{desai2009quantum} where it is stated that the electron will feel a magnetic field given by
%
\begin{equation}\label{eqn:boostCommutator:51}
\begin{aligned}
\BB = - \frac{\Bv}{c} \cross \BE
\end{aligned}
\end{equation}
%
we can see where this comes from.  If \(F = \BE + I c (0)\) is the field acting on the electron, then application of a \(\Bv\) boost to the electron perpendicular to the field (ie: radial motion), we get
%
\begin{equation}\label{eqn:boostCommutator:52}
\begin{aligned}
F' = \gamma \BE + I c \gamma \BE \cross \Bv/c^2 =
\gamma \BE + -I c \gamma \frac{\Bv}{c^2} \cross \BE
\end{aligned}
\end{equation}
%
We also have an additional \(1/c\) factor in our result, but that is a consequence of the choice of units where the dimensions of \(\BE\) match \(c \BB\), whereas in the text we have \(\BE\) and \(\BB\) in the same units.  We also have an additional \(\gamma\) factor, so we must presume that \(\Abs{\Bv} << c\) in this portion of the text.  That is actually a requirement here, for if the electron was already in motion, we would have to boost a field that also included a magnetic component.  A consequence of this is that the final interaction Hamiltonian of (6.75) is necessarily non-relativistic.

%\EndArticle
