%
% Copyright � 2012 Peeter Joot.  All Rights Reserved.
% Licenced as described in the file LICENSE under the root directory of this GIT repository.
%

%
%
% ointclockwise, ointctrclockwise
%\usepackage{txfonts}

\mychapter{Application of Stokes Integrals to Maxwell's Equation}
\index{Stokes theorem!Maxwell's equation}
\label{chap:stokesMaxwellApplication}
%\date{Sept 26, 2008.  stokesMaxwellApplication.tex}

\section{Putting Maxwell's equation in curl form}

These notes contain an application of the bivector Stokes equations
detailed in
\chapcite{PJStokes1}
.  Background of interest can also be found
in \citep{DenkerWire}, which contained the core statement of
the multivector form of Stokes equation and Biot-Savart like application
of it.  Also informative as background is the following excellent
\citep{DenkerMaxwell}.
introduction to the STA form of Maxwell's equation.

Stokes equation applied to a bivector takes the following form

\begin{equation}\label{eqn:stokesMax:summaryStokesVolume}
\iiint (\grad \wedge F) \cdot d^3\Bx = \oiintclockwise F \cdot d^2\Bx,
\end{equation}

where we will write \(F\) as the electromagnetic field bivector, and apply
it to Maxwell's equation

\begin{equation}\label{eqn:stokesMax:maxwell}
\grad F = J/\epsilon_0 c.
\end{equation}

Taking vector and trivector parts we have two equations
\begin{equation}\label{eqn:stokesMax:maxwellv}
\grad \cdot F = J/\epsilon_0 c,
\end{equation}

and
\begin{equation}\label{eqn:stokesMax:maxwellt}
\grad \wedge F = 0.
\end{equation}

\subsection{Trivector equation part}

The second of these, \eqnref{eqn:stokesMax:maxwellt}, we can apply Stokes to
directly:

\begin{equation}
\iiint (\grad \wedge F) \cdot d^3 \Bx = \oiintclockwise F \cdot d^2\Bx = 0.
\end{equation}

This area integral is a flux like quantity.  Suppose we call this the field flux, then this says
says
the flux of the combined electromagnetic field through any surface is zero
independent of the charge or current densities.
Note that here \(d^3\Bx\)
can be a regular spatial volume trivector element, but one can also pick
a spacetime (area times time) ``volume'' to integrate over, in which case
\(d^2\Bx\) are the oriented ``surfaces'' of such a spacetime volume.

This does not seem like a result that I am familiar with based on the traditional
vector forms of Maxwell's equation.  Perhaps it is recognizable in terms of
\(\BE\) and \(\BB\) explicitly:

\begin{equation}\label{eqn:stokesMax:gaussmagnetostatics}
\oiintclockwise \BE \cdot d^2\Bx = - c \oiintclockwise \BB \cdot (d^2\Bx I)
\end{equation}

On the surface this does not look like a familiar identity.  It is in fact Gauss's law for magneto-statics, which will be shown
later.

Note also the subtle difference from traditional vector treatments where
\(\BE\) and \(\BB\) were spatial vectors.  Here they are written as spacetime
bivectors,
\(\BE = E^i \sigma_i = E^i \gamma_i \wedge \gamma_0\),
\(\BB = B^i \sigma_i = B^i \gamma_i \wedge \gamma_0\).


\subsection{Vector part}

Moving on to the charge and current dependent vector terms of Maxwell's equation, we want express \eqnref{eqn:stokesMax:maxwellv} as a spacetime curl so that we can apply stokes to it.

We can do this by temporarily writing our field in terms of a potential as well its dual bivector.

\begin{equation}\label{eqn:stokesMaxwellApplication:20}
\begin{aligned}
F = \grad \wedge A = I D
\end{aligned}
\end{equation}
\begin{equation}\label{eqn:stokesMaxwellApplication:40}
\begin{aligned}
\grad F
&= \grad (\grad \wedge A) \\
&= \grad \cdot (\grad \wedge A) + \grad \wedge (\grad \wedge A) \\
&= \grad \cdot (I D) \\
&= \gpgradeone{ \grad I D } \\
&= -\gpgradeone{ I (\mathLabelBox{\grad \cdot D}{1 vector} +
\mathLabelBox
[
   labelstyle={below of=m\themathLableNode, below of=m\themathLableNode}
]
{\grad \wedge D}{3 vector}) } \\
&= - I (\grad \wedge D) \\
\end{aligned}
\end{equation}

or
\begin{equation}\label{eqn:stokesMaxwellApplication:60}
\begin{aligned}
I \grad F = \grad \wedge D.
\end{aligned}
\end{equation}

Applying stokes we have
\begin{equation}\label{eqn:stokesMaxwellApplication:80}
\begin{aligned}
\int (\grad \wedge D) \cdot d^3\Bx &= \oiintclockwise D \cdot d^2\Bx \\
\int (I \grad F) \cdot d^3\Bx
&= \oiintclockwise (-I F) \cdot d^2\Bx \\
&= \oiintclockwise \gpgradezero{ - F d^2\Bx I } \\
&= -\oiintclockwise F \cdot (d^2\Bx I) \\
\inv{\epsilon_0 c} \int (I J) \cdot d^3\Bx &= \\
\inv{\epsilon_0 c} \int \gpgradezero{ I J d^3\Bx } &= \\
\inv{\epsilon_0 c} \int \gpgradezero{ J d^3\Bx I } &= \\
\inv{\epsilon_0 c} \int J \cdot (d^3\Bx I) &= \\
\end{aligned}
\end{equation}

Or
\begin{equation}\label{eqn:stokesMax:maxwellint}
\oiintctrclockwise F \cdot (d^2\Bx I) = \int \frac{J}{\epsilon_0 c} \cdot (d^3\Bx I)
\end{equation}

This is the integral form of the vector part of Maxwell's equation \eqnref{eqn:stokesMax:maxwell}.
This does not look terribly familiar, but we are not used to
seeing Maxwell's equations in a non-disassembled form.
Hiding in there should be a subset of the
traditional eight Maxwell's equations in integral form.  It will be
possible to extract these by considering variations of current and charge density and different
volume and surface integration regions.

\section{Extracting the vector integral forms of Maxwell's equations}

One can extract the integral forms of Maxwell's equations
from \eqnref{eqn:stokesMax:maxwell}, by first extracting the differential vector
equations, and then using the spatial
divergence and stokes equations.
However, having formulated Stokes equation in its bivector form
we can go directly to those equations by appropriate selection of spatial
or spacetime volumes.
Of course we also now have new tools to work with the field in its entirety,
but lets use this as an exercise to verify that all the previous computation
that led to Stokes equation gives us the expected results.  In particular
this should be a good way to verify that
sign mistakes or other similar small errors (which would not be too hard)
have not been made.

\subsection{Zero current density. Gauss's law for Electrostatics}

With \(J = c \rho \gamma_0\), the integral form of Maxwell's equation becomes

\begin{equation}\label{eqn:stokesMaxwellApplication:100}
\begin{aligned}
\oiintctrclockwise F \cdot (d^2\Bx I)
&= \int \frac{\rho}{\epsilon_0} \gpgradezero{\gamma_0 d^3\Bx I} \\
&= \int \frac{\rho}{\epsilon_0} \gpgradezero{\gamma_{0123} \gamma_0 d^3\Bx} \\
&= -\inv{\epsilon_0} {\gamma_0}^2 \int \frac{\rho}{\epsilon_0} \gpgradezero{\gamma_{123} d^3\Bx} \\
\end{aligned}
\end{equation}

From this we see that, in the absence of currents the LHS integral must be zero unless the volume is purely spatial.  Denoting the boundary of a spacetime volume as \(\partial A c t\), this is

\begin{equation}\label{eqn:stokesMaxwellApplication:120}
\begin{aligned}
\oiintctrclockwise_{\partial {A c t}} F \cdot (d^2\Bx I) &= 0.
\end{aligned}
\end{equation}

For a purely spatial volume the dual surfaces \(d^2\Bx I\) always includes a spacetime bivector, therefore the magnetic field contributes nothing

\begin{equation*}
\oiintctrclockwise_{\partial V} I c B \cdot (d^2\Bx I) =
-c \oiintctrclockwise_{\partial V} B \cdot d^2\Bx = 0
\end{equation*}

Although this looks similar to the integral equivalent of \(\spacegrad \cdot B = 0\), we should look elsewhere for that since
that is true for the non-zero current density case too.

That leaves

\begin{equation}\label{eqn:stokesMaxwellApplication:140}
\begin{aligned}
\oiintctrclockwise E \cdot (d^2\Bx I) &= -\inv{\epsilon_0} {\gamma_0}^2 \int_{V} \rho \gpgradezero{\gamma_{123} d^3\Bx} \\
\end{aligned}
\end{equation}

Letting \(d^3 \Bx = dx^1 dx^2 dx^3 \gamma_{123}\).  Within the charge integral becomes

\begin{equation}\label{eqn:stokesMaxwellApplication:160}
\begin{aligned}
-\inv{\epsilon_0} {\gamma_0}^2 \int_{V} \rho \gpgradezero{\gamma_{123} d^3\Bx}
&=
\inv{\epsilon_0}
\mathLabelBox{{\gamma_0}^2 {\gamma_1}^2}{\(=-1\)}
\mathLabelBox
[
   labelstyle={below of=m\themathLableNode, below of=m\themathLableNode}
]
{{\gamma_2}^2 {\gamma_3}^2}{\(=(\pm 1)^2\)}
 \int_{V} \rho dx^1 dx^2 dx^3
&= -\inv{\epsilon_0} \int_{V} \rho dx^1 dx^2 dx^3
\end{aligned}
\end{equation}

To put this in correspondence with the forms we are used to consider the surfaces separately.  For the dual to the
front surface (see: \chapcite{PJStokes1})
we have

\begin{equation}\label{eqn:stokesMaxwellApplication:180}
\begin{aligned}
d^2 \Bx I
&= dx^1 dx^2 \gamma_{12} I \\
&= dx^1 dx^2 \gamma_{120123} \\
&= dx^1 dx^2 \gamma_{112023} \\
&= -dx^1 dx^2 \gamma_{112203} \\
&= -(\pm 1)^2 dx^1 dx^2 \gamma_{03} \\
&= dx^1 dx^2 \sigma_3
\end{aligned}
\end{equation}

For the left surface
\begin{equation}\label{eqn:stokesMaxwellApplication:200}
\begin{aligned}
d^2 \Bx I
&= dx^3 dx^2 \gamma_{32} I \\
&= dx^3 dx^2 \gamma_{320123} \\
&= dx^3 dx^2 \gamma_{332012} \\
&= dx^3 dx^2 \gamma_{332201} \\
&= dx^3 dx^2 (\pm 1)^2 \gamma_{01} \\
&= -dx^3 dx^2 \sigma_1 \\
\end{aligned}
\end{equation}

and for the top
\begin{equation}\label{eqn:stokesMaxwellApplication:220}
\begin{aligned}
d^2 \Bx I
&= dx^1 dx^3 \gamma_{13} I \\
&= dx^1 dx^3 \gamma_{130123} \\
&= dx^1 dx^3 \gamma_{113023} \\
&= dx^1 dx^3 \gamma_{113302} \\
&= -dx^1 dx^3 \sigma_2 \\
\end{aligned}
\end{equation}

Assembling results, writing \((x^1, x^2, x^3) = (x,y,z)\) we have
%\begin{align*}
%\iint
%(E_x(x, y, z_0) - E_x(x, y, z_1)) dx dy \\
%-\iint
%(E_y(x_1, y, z) - E_y(x_0, y, z)) dy dz \\
%-\iint
%(E_z(x, y_1, z) - E_z(x, y_0, z)) dx dz \\
%&= -\inv{\epsilon_0} \int_{V} \rho dx dy dz
%\end{align*}
\begin{equation}\label{eqn:stokesMaxwellApplication:240}
\begin{aligned}
\inv{\epsilon_0} \int_{V} \rho dx dy dz
&=
\iint
(E_x(x, y, z_1) - E_x(x, y, z_0)) dx dy \\
&+\iint
(E_y(x_1, y, z) - E_y(x_0, y, z)) dy dz \\
&+\iint
(E_z(x, y_1, z) - E_z(x, y_0, z)) dx dz \\
\end{aligned}
\end{equation}

This is Gauss's law for electrostatics in integral form

\begin{equation}
\iint \BE \cdot \ncap dA = \iiint \frac{\rho}{\epsilon_0} dV
\end{equation}

Although this extraction method is easy to understand, it is apparent that having only a pictorial way of enumerating the
oriented bivector
area elements is not efficient for high level computation.  Revisiting the stokes derivation with a more algebraic enumeration
of the surfaces should be done!

\subsection{Gauss's law for magneto-statics}
\index{Gauss's law}
\index{magnetostatics}

Return now to \eqnref{eqn:stokesMax:gaussmagnetostatics}, which resulted from considering the trivector part of Maxwell's equation

\begin{equation}
\oiintclockwise \BE \cdot d^2\Bx = - c \oiintclockwise \BB \cdot (d^2\Bx I).
\end{equation}

To start some observations can be made.

Only the spacetime surfaces of the volume
contribute to the LHS integral since \(\sigma_i \cdot (\gamma_j \wedge \gamma_k) = 0\).

For the RHS, only the purely spatial surfaces contribute to that \(\BB\) integral, since the dual surface \(d^2\Bx I\) must have a spacetime component for that dot product to be non-zero.  We have also just enumerated these dual surface area elements \(d^2 \Bx I\) for a purely
spatial surface, therefore with a \(E,B\) substitution we must have

\begin{equation}\label{eqn:stokesMaxwellApplication:260}
\begin{aligned}
0 &=
\iint
(B_x(x, y, z_1) - B_x(x, y, z_0)) dx dy  \\
&+\iint
(B_y(x_1, y, z) - B_y(x_0, y, z)) dy dz  \\
&+\iint
(B_z(x, y_1, z) - B_z(x, y_0, z)) dx dz
\end{aligned}
\end{equation}

or, more compactly

\begin{equation}
\iint \BB \cdot \ncap dA = 0
\end{equation}

For any current or charge distribution.  We have therefore obtained two of the eight Maxwell's equations.

\subsection{Zero charge.  Current density in single direction}
\index{current density}

Next to consider is \(J = j^i \gamma_i\).  For simplicity, consider current in only one direction,
taking \(J = j^1 \gamma_1\).  The exercise will be to compute the integrals of \eqnref{eqn:stokesMax:maxwellint}.

\begin{equation}\label{eqn:stokesMaxwellApplication:280}
\begin{aligned}
\oiintclockwise F \cdot (I d^2\Bx)
&= \int \frac{J}{\epsilon_0 c} \cdot (I d^3\Bx) \\
&= \int \frac{j^1}{\epsilon_0 c} \gamma_1 \cdot (I d^3\Bx) \\
\end{aligned}
\end{equation}

Unlike the calculations for the Gauss's law equations above, this one
will be done using the area orientation
methods from \chapcite{PJStokes2} since algebraically enumerating the surfaces
should make life easier.  The two Gauss's law results above were done without this, which was not too bad for a purely spatial volume, but with spacetime
volumes this is probably confusing in addition to being harder.

Starting with the volume element, one can observe that the current density
will not contribute to the boundary integral unless \(d^3\Bx\) has no \(\gamma_1\)
component, thus for a rectangular prism integration spacetime volume let
\(d^3 \Bx = \gamma_{023} dx^0 dx^2 dx^3\)

\begin{equation}\label{eqn:stokesMaxwellApplication:300}
\begin{aligned}
\gamma_1 \cdot (I d^3\Bx)
&= \gamma_1 \cdot \gamma_{0123023} dx^0 dx^2 dx^3 \\
&= \gamma_1 \cdot \gamma_{0012233} dx^0 dx^2 dx^3 \\
&= \gamma_1 \cdot \gamma_{111} dx^0 dx^2 dx^3 \\
&= -\gamma_1 \cdot \gamma^{1} dx^0 dx^2 dx^3 \\
&= - dx^0 dx^2 dx^3 \\
\end{aligned}
\end{equation}

Now for all the surfaces we want to calculate
\(I d^2\Bx\) for each of the surfaces.
For each of \(\mu \in \{0, 2, 3\}\), calculation of \(I (d^2\Bx)_\mu\) is required where

\begin{equation}\label{eqn:stokesMaxwellApplication:320}
\begin{aligned}
(d^2 \Bx)_\mu &= d^3 \Bx \cdot \Br^{\mu} \\
\Br &= x^i \gamma_i \\
\Br_\mu
&= \PD{x^{\mu}}{\Br} \\
&= \gamma_\mu \\
\Br^\mu &= \gamma^\mu
\end{aligned}
\end{equation}

Calculating the surfaces
\begin{equation}\label{eqn:stokesMaxwellApplication:340}
\begin{aligned}
I (d^2 \Bx)_\mu \frac{dx^{\mu}}{dx^0 dx^2 dx^3}
&= \gpgradetwo{ \gamma_{0123} (\gamma_{023} \cdot \gamma^{\mu}) } \\
&= \inv{2} \gpgradetwo{ \gamma_{0123} ( \gamma_{023} \gamma^{\mu} + \gamma^{\mu} \gamma_{023} ) } \\
&= \inv{2} \gpgradetwo{ \gamma_{0123} ( \gamma_{023} \gamma^{\mu} + \gamma_{023} \gamma^{\mu} ) } \\
&= \gpgradetwo{ \gamma_{0012233} \gamma^{\mu} } \\
&= -\gpgradetwo{ \gamma_{133} \gamma^{\mu} } \\
&= -\gpgradetwo{ \gamma^{1} \gamma^{\mu} } \\
&= \gamma^{\mu} \wedge \gamma^{1} \\
\end{aligned}
\end{equation}

Putting things back together we have

\begin{equation}\label{eqn:stokesMaxwellApplication:360}
\begin{aligned}
-\int j^1 dx^0 dx^2 dx^3 = \int \sum_{\mu = 0,2,3}
\left. F \cdot \left(\gamma^{\mu} \wedge \gamma^{1}\right) \right\vert_{\partial x^{\mu}}
\frac{dx^0 dx^2 dx^3}{dx^{\mu}}
\end{aligned}
\end{equation}

Now, for \(\mu=0\) we pick up the electric field component of the field

\begin{equation}\label{eqn:stokesMaxwellApplication:380}
\begin{aligned}
F \cdot \gamma^{01}
&= \left( E^i \gamma_{i0} -\epsilon_{ijk} c B^k \gamma_{ij} \right) \cdot \gamma^{01} \\
&= E^i,
\end{aligned}
\end{equation}

and for \(\mu=2,3\) we pick up magnetic field components
\begin{equation}\label{eqn:stokesMaxwellApplication:400}
\begin{aligned}
F \cdot \gamma^{\mu1}
&= \left( E^i \gamma_{i0} -\epsilon_{ijk} c B^k \gamma_{ij} \right) \cdot \gamma^{\mu1} \\
&= -\epsilon_{1 \mu k} c B^k \gamma_{1\mu} \cdot \gamma^{\mu1}.
\end{aligned}
\end{equation}

For \(\mu=2\) this is \(- c B^3\), and for \(\mu=3\), \(-\epsilon_{132} c B^2 = c B^2\), so we have

\begin{equation}\label{eqn:stokesMaxwellApplication:420}
\begin{aligned}
0
&= \int \frac{j^1}{c\epsilon_0} dx^0 dx^2 dx^3
+ \int \left. E^1 dx^2 dx^3 \right\vert_{\partial x^{0}}
+c\int \left. B^2 dx^0 dx^2 \right\vert_{\partial x^{3}}
-c\int \left. B^3 dx^0 dx^3 \right\vert_{\partial x^{2}} \\
&= \int \frac{j^1}{c\epsilon_0} dx^0 dx^2 dx^3
+ \int \PD{x^0}{E^1} dx^0 dx^2 dx^3
+c\int \PD{x^3}{B^2} dx^3 dx^2 dx^0
-c\int \PD{x^2}{B^3} dx^2 dx^0 dx^3  \\
&= \int dx^0 \int dx^2 dx^3 \left(\frac{j^1}{c\epsilon_0} + \inv{c}\PD{t}{E^1} +c\PD{x^3}{B^2} -c\PD{x^2}{B^3} \right) \\
\end{aligned}
\end{equation}

If this is zero for all time intervals, then the inner integral is also zero.  Utilizing \(c^2\mu_0\epsilon_0 = 1\) this is

\begin{equation}\label{eqn:stokesMaxwellApplication:440}
\begin{aligned}
0
&= \int dx^2 dx^3 \left( \mu_0 \left( j^1 + \epsilon_0 \PD{t}{E^1} \right) + \left( \PD{x^3}{B^2} -\PD{x^2}{B^3} \right) \right).
\end{aligned}
\end{equation}

Writing \(d\BA = \sigma_1 dx^2 dx^3\), \(\Bj = j^1 \sigma_1\), \(\BE = E^1 \sigma_1\), and \(\BB = B^i \sigma_i\) we can pick off
the differential form of the Maxwell-Ampere equation

\begin{equation}
\spacegrad \cross \BB = \mu_0 \left( \Bj + \epsilon_0 \PD{t}{\BE} \right),
\end{equation}

as well as the integral form
\begin{equation}
\int (\spacegrad \cross \BB) \cdot d\BA
= \mu_0 \left(\int \Bj \cdot d\BA + \epsilon_0 \int \PD{t}{\BE} \cdot d\BA \right)
\end{equation}

Both of these forms come straight from the application of the generalized Stokes equation integrating an appropriate
spacetime volume.

Now it is normal to have the spatial curl of \(\BB\) written as a closed loop integral.  Stokes can be employed
again to get exactly that form.  This really just undoes the fact that the partials to used as a convenience enumerate
exactly those loop boundaries (although they were originally oriented area boundaries).

\begin{equation}\label{eqn:stokesMaxwellApplication:460}
\begin{aligned}
\int \PD{x^3}{B^2} dx^3 &= B^2(t, x, y, z_1) - B^2(t, x, y, {z}_0) \\
\int \PD{x^2}{B^3} dx^2 &= B^3(t, x, {y}_1, z) - B^3(t, x, {y}_0, z)
\end{aligned}
\end{equation}

Also observe that this whole treatment was
done with \(J = j^1 \gamma_1\) only.  It is not hard to see that doing the same with \(j^i\) and summing over \(\sigma_i\)
will produce the same result.  Of course more care is required to handle the more abstract symbolic indices since a nice
hard-coded number is easier.
On the other hand the usual dodge, employing freedom to orient the coordinate system along the
\(\gamma_1\) direction makes the more general algebraic approach a less interesting exercise.

\subsection{Faraday's law}
\index{Faraday's law}

We have five of the eight Maxwell's equations.  Gauss's law for electrostatics
from the vector part of \eqnref{eqn:stokesMax:maxwellv}, integrating over a spatial
volume, and the Maxwell-Ampere equation from the same, integrating over
a spacetime volume.  Gauss's law for magneto-statics from the trivector part
of \eqnref{eqn:stokesMax:maxwellv}, integrating over a spatial volume.  This suggests
that our remaining three (one three-vector) equation will come from
integrating the trivector parts over a spacetime volume.

Stokes' gives us

\begin{equation*}
\int_V (\grad \wedge F) \cdot d^3 \Bx = \int_{\partial V} F \cdot (d^2\Bx)
\end{equation*}

Picking a spacetime volume element, and corresponding area elements

\begin{equation}\label{eqn:stokesMaxwellApplication:480}
\begin{aligned}
d^3 \Bx &= \gamma_{0ij} dx^0 dx^i dx^j \\
(d^2 \Bx)_\mu &= (\gamma_{0ij} \cdot \gamma^\mu) \frac{dx^0 dx^i dx^j}{dx^\mu}
\end{aligned}
\end{equation}

Our area integral (expanding boundaries as one more integral of partials) is
\begin{equation*}
\int \sum_{\mu = 0,i,j} dx^0 dx^i dx^j \left( \PD{x^\mu}{F} \cdot (\gamma_{0ij}\cdot\gamma^\mu) \right).
\end{equation*}

For the dot products of the area elements we have

\begin{equation*}
\left\{
\begin{array}{l l}
\gamma_{ij} & \quad \mbox{if \(\mu = 0\)} \\
\gamma_{0i} = -\sigma_i & \quad \mbox{if \(\mu = j\)} \\
-\gamma_{0j} = \sigma_j & \quad \mbox{if \(\mu = i\)} \\
\end{array} \right.
\end{equation*}

Our field derivatives in coordinates are

\begin{equation*}
\PD{x^\mu}{F}
= \PD{x^\mu}{E^m} \sigma_m - \epsilon_{klm} c \PD{x^\mu}{B^m} \gamma_{kl}
\end{equation*}

Observe that \(\mu\ne0\) selects only the electric field components, and \(\mu=0\) only the magnetic field components are selected.  Specifically

\begin{equation*}
\PD{x^\mu}{F} =
\left\{
\begin{array}{l l l}
-\epsilon_{jim} c \PD{x^0}{B^m} (\gamma_i)^2(\gamma_j)^2 &= \epsilon_{ijk}\PD{t}{B^k} & \quad \mbox{if \(\mu = 0\)} \\
\PD{x^j}{E^m}\sigma_m \cdot (-\sigma_i) &= -\PD{x^j}{E^i} & \quad \mbox{if \(\mu = j\)} \\
\PD{x^i}{E^m}\sigma_m \cdot (\sigma_j) &= \PD{x^i}{E^j} & \quad \mbox{if \(\mu = i\)} \\
\end{array} \right.
\end{equation*}

Reassembling the integral we have

\begin{equation}\label{eqn:stokesMaxwellApplication:500}
\begin{aligned}
0
&= \int dx^0 dx^i dx^j \left( \PD{x^i}{E^j} -\PD{x^j}{E^i} + \epsilon_{ijk} \PD{t}{B^k} \right) \\
&= \int dx^0 \epsilon_{ijk} \int dx^i dx^j \sigma_k \left( \sigma_k \epsilon_{ijk} \left(\PD{x^i}{E^j} -\PD{x^j}{E^i}\right) + \sigma_k \PD{t}{B^k} \right)
\end{aligned}
\end{equation}

Summing over \(k\), we can pick out the differential form of Faraday's law

\begin{equation}
0 = \PD{t}{\BB} + \spacegrad \cross \BE
\end{equation}

as well as the integral form

\begin{equation}\label{eqn:stokesMaxwellApplication:520}
\begin{aligned}
0
&=
\sum_k
\int dx^i dx^j \sigma_k \left( \sigma_k \epsilon_{ijk} \left(\PD{x^i}{E^j} -\PD{x^j}{E^i}\right) + \sigma_k \PD{t}{B^k} \right) \\
&=
\sum_k \epsilon_{ijk} \int dx^j \left.E^j\right\vert_{\partial x^i}
-\sum_k \epsilon_{ijk} \int dx^i \left.E^i\right\vert_{\partial x^j}
+ \int \PD{t}{\BB} \cdot \ncap d\BA
\end{aligned}
\end{equation}

which is
\begin{equation}
0 = \ointctrclockwise \BE \cdot d\Br + \int \PD{t}{\BB} \cdot d\BA.
\end{equation}

\section{Conclusion}

In the treatment of these notes, the
traditional integral form of Maxwell's equations are
obtained directly from the STA Maxwell's equation
using the bivector Stokes equation, and various spacetime integration volumes.

\subsection{Summary of results}

We started with the bivector form of Stokes law

\begin{equation}
\iiint (\grad \wedge F) \cdot d^3\Bx = \oiintclockwise F \cdot d^2\Bx,
\end{equation}

and the multivector Maxwell equation

\begin{equation}\label{eqn:stokesMax:summaryMaxwell}
\grad F = J/\epsilon_0 c.
\end{equation}

The trivector parts of this can be integrated directly.  This integral is always zero for all spacetime or spatial surfaces

\begin{equation}\label{eqn:stokesMaxwellApplication:540}
\begin{aligned}
\int \left(\grad \wedge F\right) \cdot d^3 \Bx &= 0
\end{aligned}
\end{equation}

Duality relations were used to put the vector parts of \eqnref{eqn:stokesMax:summaryMaxwell} into a form that Stokes can be applied to.  This gives us

\begin{equation}\label{eqn:stokesMaxwellApplication:560}
\begin{aligned}
\oiintctrclockwise F \cdot (d^2\Bx I) &= \int \frac{J}{\epsilon_0 c} \cdot (d^3\Bx I).
\end{aligned}
\end{equation}

Integration of the trivector parts

\begin{equation}
\iiint (\grad \wedge F) \cdot d^3 \Bx = \oiintclockwise F \cdot d^2\Bx = 0,
\end{equation}

produces a combined electric and magnetic field form of a Faraday's law and Gauss' magneto-statics law that does not look terribly familiar

\begin{equation}\label{eqn:stokesMax:summaryGaussMandFaraday}
\oiintclockwise \BE \cdot d^2\Bx = - c \oiintclockwise \BB \cdot (d^2\Bx I),
\end{equation}

but integration of this using a spatial volume produces the familiar Gauss's magneto-static law

\begin{equation}\label{eqn:stokesMaxwellApplication:580}
\begin{aligned}
\iint \BB \cdot d\BA &= 0 \\
\spacegrad \cdot \BB  &= 0.
\end{aligned}
\end{equation}

Integration and summation of the same trivector parts in \eqnref{eqn:stokesMax:summaryGaussMandFaraday} over each of the possible three spacetime volumes gives us Faraday's law
in its familiar forms
\begin{equation}\label{eqn:stokesMaxwellApplication:600}
\begin{aligned}
\PD{t}{\BB} + \spacegrad \cross \BE &= 0 \\
\ointctrclockwise \BE \cdot d\Br + \int \PD{t}{\BB} \cdot d\BA &= 0.
\end{aligned}
\end{equation}

Now, the vector parts of Maxwell's multivector equation integrated over a spatial volume produces Gauss's law for electrostatics

\begin{equation}\label{eqn:stokesMaxwellApplication:620}
\begin{aligned}
\iint \BE \cdot d\BA &= \int \frac{\rho}{\epsilon_0} dV \\
\spacegrad \cdot \BE &= \frac{\rho}{\epsilon_0}.
\end{aligned}
\end{equation}

Finally, integration of the same with summation over all spacetime volumes gives us the famous Maxwell-Ampere equation
\begin{equation}\label{eqn:stokesMaxwellApplication:640}
\begin{aligned}
\spacegrad \cross \BB &= \mu_0 \left( \Bj + \epsilon_0 \PD{t}{\BE} \right) \\
\ointctrclockwise \BB \cdot d\Br &= \mu_0 \left(\int \Bj \cdot d\BA + \epsilon_0 \int \PD{t}{\BE} \cdot d\BA \right).
\end{aligned}
\end{equation}

In the process of arriving at these results it appears that some of the use of Stokes equation was actually superfluous.  One of the first things
that was done once the area elements were established was to undo the boundary integral writing things once more in terms of the partials over those boundaries.
Doing all this with just the volume integrals would possibly have been simpler.  That said, as an exercise to validate the generalized Stokes equation
formulation it worked well!

Conceptually the idea that integration of Maxwell's equation over various volumes produces all the traditional vector differential and integral forms
that we are used to is quite nice.  It seems less arbitrary than trying to figure out the exactly what specific projection like operations, as done in
\chapcite{PJMaxwellProj}, will produce the various traditional vector differential equations.  Of course those can be used once found to develop the integral relations,
but here we get them all in one shot.
%The old 1966 Britannica article uses the integral forms

\subsection{Getting a glimpse of how the pieces fit together?}

I think I am starting to see a bit of the big picture for electrodynamics.
In \chapcite{PJMaxwell2}, an earlier treatment of Maxwell's equations in a GA context, I used
dimensional analysis to group electric and magnetic fields in a logical way, and employs the spatial pseudoscalar to combine divergence and curl terms.  This
I thought was a good motivation for the STA form of the equation, using
ideas familiar from school.  Similar treatments can be found elsewhere
such as in \citep{doran2003gap}
but understanding that takes a lot more work.

Once the STA form is taken as more fundamental, one can take that and show
the types of spacetime projection operations, as in
\chapcite{PJMaxwellProj}, and produce the various traditional vector
differential forms of Maxwell's equations.  Alternatively, as in
\chapcite{PJMaxwellTensor}, we can extract the traditional tensor
form of the equations.

From an even higher level point of view we can relate the STA Maxwell's
equations to the least action principles, as done in
\citep{classicalmechanics:PJSrLagrangian}, to
find the Lorentz force law in STA form using the Euler-Lagrange equations,
and finally in \citep{classicalmechanics:PJMaxwellLagrangian} where the STA
form of Maxwell's equation is obtained directly from a complex valued
field Lagrangian.

Goldstein
\citep{goldstein1951cm}
has an interesting treatment of a combined Lagrangian for both
the Lorentz force law and the field equations (using spatial delta functions).  Minimization of the action for that Lagrangian with respect to the potential
produces the field equations, and with respect to coordinates produces the
Lorentz force law.  Have to work through that in a covariant form to see
how this relates to my previous treatments.

\subsection{Followup}

It would be interesting to see if any of the problems in a Maxwell's equation
text like
\citep{fleisch2007ssg}
would be any easier with a combined field as
is possible in the STA formulation (ie: the ones based on just current
or charge distributions).

There is also some interesting looking treatments of complex number residue like integrals for the field equation
in references such as
\citep{HestenesFormsGA}.
I re-encountered that paper after writing up these notes.  I had seen it before but
those parts that cover (tersely) the same material as above did not make much sense until I had independently worked it all
out in detail myself.
Perhaps I am dense, but I find that many academic papers are ironically not very good at all for learning from!

I believe these residue/green's function ideas both relate to the
Biot-Savart law, as mentioned in
\citep{HestenesFormsGA}, \citep{doran2003gap}, and
\citep{DenkerWire}.  All of those are either too terse or have details missing
that indicate I need to study the ideas in more depth to understand.

%\bibliographystyle{plain}
