%
% Copyright � 2012 Peeter Joot.  All Rights Reserved.
% Licenced as described in the file LICENSE under the root directory of this GIT repository.
%

%
%
%\input{../peeter_prologue.tex}

\mychapter{Electromagnetic Gauge invariance}
\index{gauge invariance}
\label{chap:jackson12Dash1Gauge}

%\blogpage{http://sites.google.com/site/peeterjoot/math2009/jackson12Dash1Gauge.pdf}
%\date{Sept 24, 2009}
%\revisionInfo{\(RCSfile: jackson12Dash1Gauge.tex,v \) Last \(Revision: 1.5 \) \(Date: 2009/10/22 02:07:20 \)}

%\beginArtWithToc
\beginArtNoToc

At the end of section 12.1 in Jackson \citep{jackson1975cew} he states that it is obvious that the Lorentz force equations are gauge invariant.

\begin{equation}\label{eqn:jacksonGaugeInv:foo1}
\begin{aligned}
\frac{d \Bp}{dt} &= e \left( \BE + \frac{\Bu}{c} \cross \BB \right) \\
\frac{d E}{dt} &= e \Bu \cdot \BE
\end{aligned}
\end{equation}

Since I did not remember what Gauge invariance was, it was not so obvious.  But if I looking ahead to one of the problem 12.2 on this invariance we have a Gauge transformation defined in four vector form as

\begin{equation}\label{eqn:jacksonGaugeInv:foo2}
\begin{aligned}
A^\alpha \rightarrow A^\alpha + \partial^\alpha \psi
\end{aligned}
\end{equation}

In vector form with \(A = \gamma_\alpha A^\alpha\), this gauge transformation can be written

\begin{equation}\label{eqn:jacksonGaugeInv:foo3}
\begin{aligned}
A \rightarrow A + \grad \psi
\end{aligned}
\end{equation}

so this is really a statement that we add a spacetime gradient of something to the four vector potential.  Given this, how does the field transform?

\begin{equation}\label{eqn:jackson12Dash1Gauge:27}
\begin{aligned}
F
&= \grad \wedge A \\
&\rightarrow \grad \wedge (A + \grad \psi) \\
&= F + \grad \wedge \grad \psi
\end{aligned}
\end{equation}

But \(\grad \wedge \grad \psi = 0\) (assuming partials are interchangeable) so the field is invariant regardless of whether we are talking about the Lorentz force

\begin{equation}\label{eqn:jacksonGaugeInv:foo4}
\begin{aligned}
\grad F = J/\epsilon_0 c
\end{aligned}
\end{equation}

or the field equations themselves

\begin{equation}\label{eqn:jacksonGaugeInv:foo5}
\begin{aligned}
\frac{dp}{d\tau} = e F \cdot v/c
\end{aligned}
\end{equation}

So, once you know the definition of the gauge transformation in four vector form, yes this justifiably obvious, however, to anybody who is not familiar with Geometric Algebra, perhaps this is still not so obvious.  How does this translate to the more common place tensor or space time vector notations?  The tensor four vector translation is the easier of the two, and there we have

\begin{equation}\label{eqn:jackson12Dash1Gauge:47}
\begin{aligned}
F^{\alpha\beta}
&= \partial^\alpha A^\beta -\partial^\beta A^\alpha \\
&\rightarrow \partial^\alpha (A^\beta + \partial^\beta \psi) -\partial^\beta (A^\alpha + \partial^\alpha \psi) \\
&= F^{\alpha\beta} + \partial^\alpha \partial^\beta \psi -\partial^\beta \partial^\alpha \psi \\
\end{aligned}
\end{equation}

As required for \(\grad \wedge \grad \psi = 0\) interchange of partials means the field components \(F^{\alpha\beta}\) are unchanged by adding this gradient.  Finally, in plain old spatial vector form, how is this gauge invariance expressed?

In components we have

\begin{equation}\label{eqn:jacksonGaugeInv:foo6}
\begin{aligned}
A^0 &\rightarrow A^0 + \partial^0 \psi = \phi + \inv{c}\frac{\partial \psi}{\partial t} \\
A^k &\rightarrow A^k + \partial^k \psi = A^k - \frac{\partial \psi}{\partial x^k}
\end{aligned}
\end{equation}

This last in vector form is \(\BA \rightarrow \BA - \spacegrad \psi\), where the sign inversion comes from \(\partial^k = -\partial_k = -\partial/\partial x^k\), assuming a \(+---\) metric.

We want to apply this to the electric and magnetic field components

\begin{equation}\label{eqn:jacksonGaugeInv:foo7}
\begin{aligned}
\BE &= -\spacegrad \phi - \inv{c}\frac{\partial \BA}{\partial t} \\
\BB &= \spacegrad \cross \BA
\end{aligned}
\end{equation}

The electric field transforms as

\begin{equation}\label{eqn:jackson12Dash1Gauge:67}
\begin{aligned}
\BE &\rightarrow -\spacegrad \left( \phi + \inv{c}\frac{\partial \psi}{\partial t}\right) - \inv{c}\frac{\partial }{\partial t} \left( \BA - \spacegrad \psi \right) \\
&= \BE -\inv{c} \spacegrad \frac{\partial \psi}{\partial t} + \inv{c}\frac{\partial }{\partial t} \spacegrad \psi
\end{aligned}
\end{equation}

With partial interchange this is just \(\BE\).  For the magnetic field we have

\begin{equation}\label{eqn:jackson12Dash1Gauge:87}
\begin{aligned}
\BB
&\rightarrow \spacegrad \cross \left( \BA - \spacegrad \psi \right) \\
&= \BB  - \spacegrad \cross \spacegrad \psi
\end{aligned}
\end{equation}

Again since the partials interchange we have \(\spacegrad \cross \spacegrad \psi = 0\), so this is just the magnetic field.

Alright.  Worked this in three different ways, so now I can say its obvious.

%\EndArticle
%%\EndNoBibArticle
