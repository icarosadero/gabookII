%
% Copyright � 2012 Peeter Joot.  All Rights Reserved.
% Licenced as described in the file LICENSE under the root directory of this GIT repository.
%

%
%
\mychapter{Four vector potential}
\index{vector potential}
\index{four potential}
\label{chap:emPotential}
%\date{August 15, 2008}

\section{}

Goldstein's classical mechanics, and many other texts, will introduce the four potential starting with
Maxwell's equation in scalar, vector, bivector, trivector expanded form:

\begin{equation}\label{eqn:emPotential:20}
\begin{aligned}
\spacegrad \cdot \BE &= \frac{\rho}{\epsilon_0} \\
\spacegrad \cdot \BB &= 0 \\
\spacegrad \cross \BE &= - \frac{\partial \BB}{\partial t} \\
\spacegrad \cross \BB &= \mu_0\left(\BJ + \epsilon_0 \frac{\partial \BE}{\partial t}\right) \\
\end{aligned}
\end{equation}

ie: E can not be a gradient, since it has a curl, but B can be the curl of something since it has zero
divergence, so we have \(\BB = \spacegrad \cross \BA\).  Faraday's law above gives:

\begin{equation}\label{eqn:emPotential:40}
\begin{aligned}
0 &= \spacegrad \cross \BE + \frac{\partial \spacegrad \cross \BA}{\partial t} \\
&= \spacegrad \cross \left(\BE + \frac{\partial \BA}{\partial t}\right) \\
\end{aligned}
\end{equation}

Because this curl is zero, one can write it as a gradient, say \(-\spacegrad \phi\).

The end result are the equations:

\begin{align}
\BE &= - \left( \spacegrad \phi + \partial_t \BA \right) \label{eqn:fourPot:BE} \\
\BB &= \spacegrad \cross \BA \label{eqn:fourPot:BB}
\end{align}

Looking at what Goldstein does with this (which I re-derived above to put in the SI form I am used to), my
immediate question is how would the combined bivector field look when expressed using an STA basis, and
then once that is resolved, how would his Lagrangian for a charged point particle look in explicit four
vector form?

Intuition says that this is all going to work out to be a spacetime gradient of a four vector, but
I am not sure how the Lorentz gauge freedom will turn out.  Here is an exploration of this.

\subsection{}

Forming as usual

\begin{equation}\label{eqn:fourPot:BF}
\BF = \BE + i c \BB
\end{equation}

We can combine the equations \eqnref{eqn:fourPot:BE} and \eqnref{eqn:fourPot:BB} into bivector form

\begin{equation}\label{eqn:fourPot:BFA}
\BF = - \left( \spacegrad \phi + \partial_t \BA \right) + c \spacegrad \wedge \BA
\end{equation}

\subsection{Dimensions}

Let us do a dimensional check before continuing:

\Eqnref{eqn:fourPot:BF} gives:

\begin{equation*}
[\BE] = \frac{[m][d]}{[q][t]^2}
\end{equation*}

That and \eqnref{eqn:fourPot:BFA} gives
\begin{equation*}
[\phi] = \frac{[m][d]^2}{[q][t]^2}
\end{equation*}

And the two \(\BA\) terms of \eqnref{eqn:fourPot:BFA} both give:
\begin{equation*}
[\BA] = \frac{[m][d]}{[q][t]}.
\end{equation*}

Therefore if we create a four vector out of \(\phi\), and \(\BA\) in SI units we will need that factor \(c\) with \(\BA\) with velocity dimensions to fix things up.

\subsection{Expansion in terms of components.  STA split}

\begin{equation}\label{eqn:emPotential:80}
\begin{aligned}
\BF
&= - \left( \spacegrad \phi + \partial_t \BA \right) + c \spacegrad \wedge \BA \\
&= - \sum \gamma_i \gamma_0 \partial_{x^i}\phi - \sum \gamma_i \gamma_0 \partial_t A^i
+ c \left(\sum \sigma_i \partial_{x^i}\right) \wedge \left(\sum \sigma_j A^j\right) \\
&= \sum \gamma^i \partial_{x^i} (\gamma_0 \phi) + \sum \gamma_0 \partial_{ct} c \gamma_i A^i
- \left(\sum \gamma_i \partial_{x^i}\right) \wedge \left(\sum \gamma_j c A^j\right) \\
&= \sum \gamma^i \wedge \gamma_0 \partial_{x^i} \phi + \sum \gamma^0 \wedge \gamma_i \partial_{x^0} c A^i
+ \sum \gamma^i \wedge \gamma_j \partial_{x^i} c A^j \\
&= \left(\sum \gamma^i \partial_{x^i}\right) \wedge \left(\gamma_0 \phi + \gamma_i c A^i \right) \\
&= \grad \wedge \left(\gamma_0 \phi + \sum \gamma_i c A^i \right)
\end{aligned}
\end{equation}

Once the electric and magnetic fields are treated as one entity, the separate equations of
\eqnref{eqn:fourPot:BE} and \eqnref{eqn:fourPot:BB} become nothing more than a statement that the bivector field \(\BF\) is the spacetime curl
of a four vector potential \(A = \gamma_0 \phi + \sum \gamma_i c A^i\).

This original choice of components \(A^i\), defined such that \(\BB = \spacegrad \cross \BA\) is a bit unfortunate in SI
units.  Setting \(\calA^i = cA^i\), and \(\calA^0 = \phi\), one then has the more symmetric form.

\begin{equation*}
A = \sum \gamma_{\mu} \calA^{\mu}.
\end{equation*}

Of course the same thing could be achieved with \(c=1\) ;)

Anyways, substitution of this back into Maxwell's equation gives:

\begin{equation*}
\grad (\grad \wedge A) = \grad \cdot (\grad \wedge A) + \mathLabelBox{\grad \wedge \grad \wedge A}{\(=0\)} = J
\end{equation*}

One can see an immediate simplification possible if one requires:

\begin{equation*}
\grad \cdot A = 0.
\end{equation*}

Then we are left with a forced wave equation to solve for the four potential:

\begin{equation*}
\grad^2 A = -\left(\sum \partial_{x^i x^i} - \inv{c^2}\partial_{tt}\right) A = J.
\end{equation*}

Now, without all this mess of algebra, I could have gone straight to this end result (and had done so previously).  I just
wanted to see where I would get applying the STA basis to the classical vector+scalar four vector ideas.

\subsection{Lorentz gauge}
\index{Lorentz gauge}

Looking at \(\grad \cdot A = 0\), I was guessing that this was what I recalled being called the Lorentz gauge, but in a
slightly different form.

If one expands this you get:

\begin{equation}\label{eqn:emPotential:100}
\begin{aligned}
0
&= \grad \cdot A \\
&= \sum \gamma^{\mu} \partial_{\mu} \cdot \left(\gamma_0 \phi + c \sum \gamma_j A^j \right) \\
&= \partial_{ct} \phi + c \sum \partial_{x^i}A^i \\
&= \partial_{ct} \phi + c \spacegrad \cdot \BA
\end{aligned}
\end{equation}

Or,

\begin{equation}
\spacegrad \cdot \BA = -\inv{c^2}\partial_{t} \phi
\end{equation}

Checked my Feynman book.  Yes, this is the Lorentz Gauge.

Another note.  Again the SI units make things ugly.  With the above modification of components that hide this, where one sets \(A = \sum \gamma_i \calA^i\), this gauge equation also takes a simpler form:

\begin{equation*}
0 = \grad \cdot A = \left(\sum \gamma^{\mu} \partial_{x^{\mu}}\right) \cdot \left(\sum \gamma_{\nu} \calA^{\nu}\right) = \sum \partial_{x^{\mu}} \calA^{\mu}.
\end{equation*}

\section{Appendix}

\subsection{wedge of spacetime bivector basis elements}

For \(i \ne j\):

\begin{equation}\label{eqn:emPotential:120}
\begin{aligned}
\sigma_i \wedge \sigma_j
&= \inv{2}\left( \sigma_i \sigma_j - \sigma_j \sigma_i \right) \\
&= \inv{2}\left( \gamma_{i0j0} - \gamma_{j0i0} \right) \\
&= \inv{2}\left( -\gamma_{ij} +\gamma_{ji} \right) \\
&= \gamma_{ji}
\end{aligned}
\end{equation}
