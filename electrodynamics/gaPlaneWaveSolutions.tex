%
% Copyright � 2012 Peeter Joot.  All Rights Reserved.
% Licenced as described in the file LICENSE under the root directory of this GIT repository.
%
%\input{../blogpost.tex}
%\renewcommand{\basename}{gaPlaneWaveSolutions}
%\renewcommand{\dirname}{notes/gabook/electrodynamics/}
%\newcommand{\keywords}{Plane wave, Maxwell's equation, geometric algebra, geometric product, wave equation, phasor, wave number, angular frequency, scalar, vector, bivector, pseudoscalar, imaginary, real, multivector, cross product, wedge product}

%\input{../peeter_prologue_print2.tex}
%\beginArtNoToc

%\generatetitle{Plane wave solutions of Maxwell's equation using Geometric Algebra}
%\mychapter{A griffiths problem related to electromagnetic reflection and transmission}
\mychapter{Plane wave solutions of Maxwell's equation using Geometric Algebra}
\index{plane wave}
\index{Maxwell's equation}
\label{chap:gaPlaneWaveSolutions}
\section{Motivation}

Study of reflection and transmission of radiation in isotropic, charge and current free, linear matter utilizes the plane wave solutions to Maxwell's equations.  These have the structure of phasor equations, with some specific constraints on the components and the exponents.

These constraints are usually derived starting with the plain old vector form of Maxwell's equations, and it is natural to wonder how this is done directly using Geometric Algebra.  \citep{doran2003gap} provides one such derivation, using the covariant form of Maxwell's equations.  Here's a slightly more pedestrian way of doing the same.

\section{Maxwell's equations in media}

We start with Maxwell's equations for linear matter as found in \citep{griffiths1999introduction}
%
\begin{subequations}
\begin{equation}\label{eqn:gaPlaneWaveSolutions:10}
\spacegrad \cdot \BE = 0
\end{equation}
\begin{equation}\label{eqn:gaPlaneWaveSolutions:30}
\spacegrad \cross \BE = -\PD{t}{\BB}
\end{equation}
\begin{equation}\label{eqn:gaPlaneWaveSolutions:50}
\spacegrad \cdot \BB = 0
\end{equation}
\begin{equation}\label{eqn:gaPlaneWaveSolutions:70}
\spacegrad \cross \BB = \mu\epsilon \PD{t}{\BE}.
\end{equation}
\end{subequations}
%
We merge these using the geometric identity
%
\begin{equation}\label{eqn:gaPlaneWaveSolutions:90}
\spacegrad \cdot \Ba + I \spacegrad \cross \Ba = \spacegrad \Ba,
\end{equation}
%
where \(I\) is the 3D pseudoscalar \(I = \Be_1 \Be_2 \Be_3\), to find
%
\begin{subequations}
\begin{equation}\label{eqn:gaPlaneWaveSolutions:110}
\spacegrad \BE = -I \PD{t}{\BB}
\end{equation}
\begin{equation}\label{eqn:gaPlaneWaveSolutions:130}
\spacegrad \BB = I \mu\epsilon \PD{t}{\BE}.
\end{equation}
\end{subequations}
%
We want dimensions of \(1/L\) for the derivative operator on the RHS of \eqnref{eqn:gaPlaneWaveSolutions:130}, so we divide through by \(\sqrt{\mu\epsilon} I\) for
%
\begin{equation}\label{eqn:gaPlaneWaveSolutions:130b}
-I \inv{\sqrt{\mu\epsilon}} \spacegrad \BB = \sqrt{\mu\epsilon} \PD{t}{\BE}.
\end{equation}
%
This can now be added to \eqnref{eqn:gaPlaneWaveSolutions:110} for
%
\begin{equation}\label{eqn:gaPlaneWaveSolutions:150}
\left(
\spacegrad + \sqrt{\mu\epsilon} \PD{t}{} \right)
\left( \BE +
\frac{I}{\sqrt{\mu\epsilon}} \BB
\right)
= 0.
\end{equation}
%
This is Maxwell's equation in linear isotropic charge and current free matter in Geometric Algebra form.

\section{Phasor solutions}

We write the electromagnetic field as
%
\begin{equation}\label{eqn:gaPlaneWaveSolutions:170}
F =
\left( \BE +
\frac{I}{\sqrt{\mu\epsilon}} \BB
\right),
\end{equation}
%
so that for vacuum where \(1/\sqrt{\mu \epsilon} = c\) we have the usual \(F = \BE + I c \BB\).  Assuming a phasor solution of
%
\begin{equation}\label{eqn:gaPlaneWaveSolutions:190}
\tilde{F} = F_0 e^{i (\Bk \cdot \Bx - \omega t)}
\end{equation}
%
where \(F_0\) is allowed to be complex, and the actual field is obtained by taking the real part
%
\begin{equation}\label{eqn:gaPlaneWaveSolutions:210}
F = \Real \tilde{F} =
\Real(F_0) \cos(\Bk \cdot \Bx - \omega t)
-\Imag(F_0) \sin(\Bk \cdot \Bx - \omega t).
\end{equation}
%
Note carefully that we are using a scalar imaginary \(i\), as well as the multivector (pseudoscalar) \(I\), despite the fact that both have the square to scalar minus one property.

We now seek the constraints on \(\Bk\), \(\omega\), and \(F_0\) that allow \(\tilde{F}\) to be a solution to \eqnref{eqn:gaPlaneWaveSolutions:150}
%
\begin{equation}\label{eqn:gaPlaneWaveSolutions:230}
0 =
\left(
\spacegrad + \sqrt{\mu\epsilon} \PD{t}{}
\right)
\tilde{F}.
\end{equation}
%
As usual in the non-geometric algebra treatment, we observe that any such solution \(\tilde{F}\) to Maxwell's equation is also a wave equation solution.  In GA we can do so by right multiplying an operator that has a conjugate form,
%
\begin{equation}\label{eqn:gaPlaneWaveSolutions:250}
\begin{aligned}
0
&=
\left(
\spacegrad + \sqrt{\mu\epsilon} \PD{t}{}
\right)
\tilde{F} \\
&=
\left(
\spacegrad - \sqrt{\mu\epsilon} \PD{t}{}
\right)
\left(
\spacegrad + \sqrt{\mu\epsilon} \PD{t}{}
\right)
\tilde{F} \\
&=
\left( \spacegrad^2 - \mu\epsilon \frac{\partial^2}{\partial t^2} \right) \tilde{F} \\
&=
\left( \spacegrad^2 - \inv{v^2} \frac{\partial^2}{\partial t^2} \right) \tilde{F},
\end{aligned}
\end{equation}
%
where \(v = 1/\sqrt{\mu\epsilon}\) is the speed of the wave described by this solution.

Inserting the exponential form of our assumed solution \eqnref{eqn:gaPlaneWaveSolutions:190} we find
%
\begin{equation}\label{eqn:gaPlaneWaveSolutions:270}
0 = -(\Bk^2 - \omega^2/v^2) F_0 e^{i (\Bk \cdot \Bx - \omega t)},
\end{equation}
%
which implies that the wave number vector \(\Bk\) and the angular frequency \(\omega\) are related by
%
\begin{equation}\label{eqn:gaPlaneWaveSolutions:290}
v^2 \Bk^2 = \omega^2.
\end{equation}
%
Our assumed solution must also satisfy the first order system \eqnref{eqn:gaPlaneWaveSolutions:230}
%
\begin{equation}\label{eqn:gaPlaneWaveSolutions:310}
\begin{aligned}
0
&=
\left(
\spacegrad + \sqrt{\mu\epsilon} \PD{t}{}
\right)
F_0
e^{i (\Bk \cdot \Bx - \omega t)} \\
&=
i
\left(
\Be_m k_m - \frac{\omega}{v}
\right)
F_0
e^{i (\Bk \cdot \Bx - \omega t)} \\
&=
i k ( \kcap - 1 ) F_0 e^{i (\Bk \cdot \Bx - \omega t)}.
\end{aligned}
\end{equation}
%
The constraints on \(F_0\) must then be given by
%
\begin{equation}\label{eqn:gaPlaneWaveSolutions:330}
0 = \left( \kcap - 1 \right) F_0.
\end{equation}
%
With
%
\begin{equation}\label{eqn:gaPlaneWaveSolutions:350}
F_0 = \BE_0 + I v \BB_0,
\end{equation}
%
we must then have all grades of the multivector equation equal to zero
%
\begin{equation}\label{eqn:gaPlaneWaveSolutions:370}
0 =
( \kcap - 1 )
\left(
\BE_0 + I v \BB_0
\right).
\end{equation}
%
Writing out all the geometric products, grouping into columns by grade, we have
%
\begin{equation}\label{eqn:gaPlaneWaveSolutions:390}
\begin{array}{l l l l l}
0 &= \kcap \cdot \BE_0 & - \BE_0                   & + \kcap \wedge \BE_0 & I v \kcap \cdot \BB_0 \\
  &                    & + I v \kcap \wedge \BB_0  & + I v \BB_0          &
\end{array}
\end{equation}
%
We've made use of the fact that \(I\) commutes with all of \(\kcap\), \(\BE_0\), and \(\BB_0\) and employed the identity \(\Ba \Bb = \Ba \cdot \Bb + \Ba \wedge \Bb\).

Collecting the scalar, vector, bivector, and pseudoscalar grades and using \(\Ba \wedge \Bb = I \Ba \cross \Bb\) again, we have a set of constraints resulting from the first order system

\begin{subequations}
\label{eqn:gaPlaneWaveSolutions:405}
\begin{equation}\label{eqn:gaPlaneWaveSolutions:410}
0 = \kcap \cdot \BE_0
\end{equation}
\begin{equation}\label{eqn:gaPlaneWaveSolutions:430}
\BE_0 =- \kcap \cross v \BB_0
\end{equation}
\begin{equation}\label{eqn:gaPlaneWaveSolutions:450}
v \BB_0 = \kcap \cross \BE_0
\end{equation}
\begin{equation}\label{eqn:gaPlaneWaveSolutions:470}
0 = \kcap \cdot \BB_0.
\end{equation}
\end{subequations}
%
This and \eqnref{eqn:gaPlaneWaveSolutions:290} describe all the constraints on our phasor that are required for it to be a solution.  Note that only one of the two cross product equations in \eqnref{eqn:gaPlaneWaveSolutions:405} are required because the two are not independent (problem \ref{ch:gaPlaneWaveSolutions:pr1}).

Writing out the complete expression for \(F_0\) we have
%
\begin{equation}\label{eqn:gaPlaneWaveSolutions:510}
\begin{aligned}
F_0
&=
\BE_0 + I v \BB_0 \\
&=
\BE_0 + I \kcap \cross \BE_0 \\
&=
\BE_0 + \kcap \wedge \BE_0.
\end{aligned}
\end{equation}
%
Since \(\kcap \cdot \BE_0 = 0\), this is
%
\begin{equation}\label{eqn:gaPlaneWaveSolutions:530}
F_0 = (1 + \kcap) \BE_0.
\end{equation}
%
Had we been clever enough this could have been deduced directly from the \eqnref{eqn:gaPlaneWaveSolutions:330} directly, since we require a product that is killed by left multiplication with \(\kcap - 1\).  Our complete plane wave solution to Maxwell's equation is therefore given by
%
\begin{equation}\label{eqn:gaPlaneWaveSolutions:550}
\begin{aligned}
F &= \Real(\tilde{F}) = \BE + \frac{I}{\sqrt{\mu\epsilon}} \BB \\
\tilde{F} &= (1 \pm \kcap) \BE_0 e^{i (\Bk \cdot \Bx \mp \omega t)} \\
0 &= \kcap \cdot \BE_0 \\
\Bk^2 &= \omega^2 \mu \epsilon.
\end{aligned}
\end{equation}
%
%%
% Copyright � 2013 Peeter Joot.  All Rights Reserved.
% Licenced as described in the file LICENSE under the root directory of this GIT repository.
%
\makeproblem{Electrodynamic plane wave constraints}{ch:gaPlaneWaveSolutions:pr1}{

It was claimed that
\begin{subequations}
\begin{equation}\label{eqn:gaPlaneWaveSolutionsProblems:430a}
\BE_0 =- \kcap \cross v \BB_0
\end{equation}
\begin{equation}\label{eqn:gaPlaneWaveSolutionsProblems:450b}
v \BB_0 = \kcap \cross \BE_0
\end{equation}
\end{subequations}

%equations \eqnref{eqn:gaPlaneWaveSolutionsProblems:430} and \eqnref{eqn:gaPlaneWaveSolutionsProblems:450}
relating the electric and magnetic field of electrodynamic plane waves were dependent.  Show this.
}
\makeanswer{ch:gaPlaneWaveSolutions:pr1}{
This can be shown by crossing \(\kcap\) with \eqnref{eqn:gaPlaneWaveSolutionsProblems:430a} and using the identity

\begin{equation}\label{eqn:gaPlaneWaveSolutionsProblems:490}
\Ba \cross (\Ba \cross \Bb) = - \Ba^2 \Bb + \Ba (\Ba \cdot \Bb).
\end{equation}

This gives
\begin{equation}\label{eqn:gaPlaneWaveSolutionsProblems:490c}
\begin{aligned}
\kcap \cross \BE_0
&= - \kcap \cross (\kcap \cross v \BB_0 ) \\
&= \kcap^2 v \BB_0 - \kcap ( \cancel{\kcap \cdot \BE_0} ) \\
&= v \BB_0.
\end{aligned}
\end{equation}
}
\makeproblem{Proving that the wavevectors are all coplanar}{ch:gaPlaneWaveSolutions:2}{

\citep{griffiths1999introduction} poses the following simple but excellent problem, related to the relationship between the incident, transmission and reflection phasors, which he states has the following form

\begin{equation}\label{eqn:gaPlaneWaveSolutionsProblems:510}
() e^{i (\Bk_i \cdot \Bx - \omega t)}
+ () e^{i (\Bk_r \cdot \Bx - \omega t)}
= () e^{i (\Bk_t \cdot \Bx - \omega t)},
\end{equation}

He poses the problem (9.15)

Suppose \(A e^{i a x} + B e^{i b x} = C e^{i c x}\) for some nonzero constants \(A\), \(B\), \(C\), \(a\), \(b\), \(c\), and for all \(x\).  Prove that \(a = b = c\) and \(A + B = C\).
}

\makeanswer{ch:gaPlaneWaveSolutions:2}{
If this relation holds for all \(x\), then for \(x = 0\), we have \(A + B = C\).  We are left to show that

\begin{equation}\label{eqn:gaPlaneWaveSolutionsProblems:530}
A \left( e^{i a x} - e^{i c x} \right)
+ B \left( e^{i b x} - e^{i c x} \right) = 0.
\end{equation}

Let \(a = c + \delta\) and \(b = c + \epsilon\), so that

\begin{equation}\label{eqn:gaPlaneWaveSolutionsProblems:550}
A \left( e^{i \delta x} - 1 \right)
+ B \left( e^{i \epsilon x} - 1 \right) = 0.
\end{equation}

Now consider some special values of \(x\).  For \(x = 2 \pi/\epsilon\) we have

\begin{equation}\label{eqn:gaPlaneWaveSolutionsProblems:570}
A \left( e^{2 \pi i \delta/\epsilon} - 1 \right) = 0,
\end{equation}

and because \(A \ne 0\), we must conclude that \(\delta/\epsilon\) is an integer.

Similarily, for \(x = 2 \pi/\delta\), we have

\begin{equation}\label{eqn:gaPlaneWaveSolutionsProblems:590}
B \left( e^{2 \pi i \epsilon/\delta} - 1 \right) = 0,
\end{equation}

and this time must conclude that \(\epsilon/\delta\) is an integer.  These ratios must therefore take one of the values \(0, 1, -1\).  Consider the points \(x = 2 n \pi/\epsilon\) or \(x = 2 m \pi/\delta\) we find that \(n \delta/\epsilon\) and \(m \epsilon/\delta\) must be integers for any integers \(m, n\).  This only leaves \(\epsilon = \delta = 0\), or \(a = b = c\) as possibilities.
}


%\vcsinfo
%\EndArticle
