%
% Copyright � 2012 Peeter Joot.  All Rights Reserved.
% Licenced as described in the file LICENSE under the root directory of this GIT repository.
%

%
%
\chapter{Maxwell's equations expressed with Geometric Algebra}
\index{Maxwell's equations}
\label{chap:maxwellsGa}
%\date{January 29, 2008.  maxwellsGa.tex}

\section{On different ways of expressing Maxwell's equations}

One of the most striking applications of the geometric product is the ability to formulate the eight Maxwell's equations in a coherent fashion as a single equation.

This is not a new idea, and this has been done historically using formulations based on quaternions (~1910.  dig up citation).  A formulation in terms of antisymmetric second rank tensors \(F_{\mu \nu}\) and \(G_{\mu \nu}\) (See: wiki:Formulation of Maxwell's equations in special relativity) reduces the eight equations to two, but also introduces complexity and obfuscates the connection to the physically measurable quantities.

A formulation in terms of differential forms (See: wiki:Maxwell's equations) is also possible.  This does not have the complexity of the tensor formulation, but requires the electromagnetic field to be expressed as a differential form.  This is arguably strange given a traditional vector calculus education.  One also does not have to integrate a field in any fashion, so what meaning should be given to a electrodynamic field as a differential form?

\subsection{Introduction of complex vector electromagnetic field}

To explore the ideas, the starting point is the traditional set of Maxwell's equations

\begin{dmath}\label{eqn:maxwellsGa:20}
\nabla \cdot \BE  = \frac {\rho} {\epsilon_0}
\end{dmath}
\begin{dmath}\label{eqn:maxwellsGa:40}
\nabla \cdot \BB  = 0
\end{dmath}
\begin{dmath}\label{eqn:maxwellsGa:60}
\nabla \times \BE  +\frac{\partial \BB } {\partial t} = 0
\end{dmath}
\begin{dmath}\label{eqn:maxwellsGa:80}
c^2 \nabla \times \BB  - \frac{\partial \BE } {\partial t}
= \frac{\BJ}{\epsilon_0}
\end{dmath}

It is customary in relativistic treatments of electrodynamics to introduce a four vector \((x, y, z, ict)\).  Using this as a hint, one can write the time partials in terms of \(ict\) and regrouping slightly

\begin{dmath}\label{eqn:maxwellsGa:100}
\nabla \cdot \BE  = \frac {\rho} {\epsilon_0}
\end{dmath}
\begin{dmath}\label{eqn:maxwellsGa:120}
\nabla \cdot (ic\BB ) = 0
\end{dmath}
\begin{dmath}\label{eqn:maxwellsGa:140}
\nabla \times \BE  +\frac{\partial (ic\BB )} {\partial (ict)} = 0
\end{dmath}
\begin{dmath}\label{eqn:maxwellsGa:160}
\nabla \times (ic\BB ) + \frac{\partial \BE } {\partial (ict)}
= i\frac{\BJ}{\epsilon_0 c}
\end{dmath}

There is no use of geometric or wedge products here, but the opposing signs in the two sets of curl and time partial equations is removed.  The pairs of equations can be added together without loss of information since the original equations can be recovered by taking real and imaginary parts.
\begin{dmath}\label{eqn:maxwellsGa:180}
\nabla \cdot (\BE + ic \BB) = \frac {\rho} {\epsilon_0}
\end{dmath}
\begin{dmath}\label{eqn:maxwellsGa:200}
\nabla \times (\BE + ic \BB) + \frac{\partial (\BE + ic \BB)} {\partial (ict)}
= i\frac{\BJ}{\epsilon_0 c}
\end{dmath}

It is thus natural to define a combined electrodynamic field as a complex vector, expressing the natural orthogonality of the electric and magnetic fields
\begin{dmath}\label{eqn:maxwellsGa:220}
\BF = \BE + ic \BB.
\end{dmath}

The electric and magnetic fields can be recovered from this composite field by taking real and imaginary parts respectively, and we can now write Maxwell's equations in terms of this single electrodynamic field
\begin{dmath}\label{eqn:maxwellsGa:240}
\nabla \cdot \BF = \frac {\rho} {\epsilon_0}
\end{dmath}
\begin{dmath}\label{eqn:maxwellsGa:260}
\nabla \times \BF + \frac{\partial \BF} {\partial (ict)}
= i\frac{\BJ}{\epsilon_0 c}
\end{dmath}

\subsection{Converting the curls in the pair of Maxwell's equations for the electrodynamic field to wedge and geometric products}

The above manipulations didn't make any assumptions about the structure of the ``imaginary'' denoted \(i\) above.  What was implied was a requirement that \(i^2 = -1\), and that \(i\) commutes with vectors.  Both of these conditions are met by the use of the pseudoscalar for 3D Euclidean space \(\Be_1 \Be_2 \Be_3\).  This is usually denoted \(I\) and we'll now switch notations for clarity.
XX
With multiplication of the second by a \(I\) factor to convert to a wedge product representation the remaining pair of equations can be written

\begin{dmath}\label{eqn:maxwellsGa:280}
\nabla \cdot \BF = \frac {\rho} {\epsilon_0}
\end{dmath}
\begin{dmath}\label{eqn:maxwellsGa:300}
I\nabla \times \BF + \frac{1}{c} \frac{\partial \BF}{\partial t}
= -\frac{\BJ}{\epsilon_0 c}
\end{dmath}

This last, in terms of the geometric product is,
\begin{dmath}\label{eqn:maxwellsGa:320}
\nabla \wedge \BF + \frac{1}{c} \frac{\partial \BF}{\partial t}
= -\frac{\BJ}{\epsilon_0 c}
\end{dmath}

These equations can be added without loss

\begin{dmath}\label{eqn:maxwellsGa:340}
\nabla \cdot \BF + \nabla \wedge \BF + \frac{1}{c} \frac{\partial \BF}{\partial t} = \frac {\rho} {\epsilon_0} - \frac{\BJ}{\epsilon_0 c}
\end{dmath}

Leading to the end result

\begin{dmath}\label{eqn:maxwellsGa:5}
\left(\frac{1}{c} \frac{\partial}{\partial t} + \nabla\right)\BF = \frac {1} {\epsilon_0}\left(\rho - \frac{\BJ}{c}\right)
\end{dmath}

Here we have all of Maxwell's equations as a single differential equation.
This gives a hint why it is hard to separately solve these equations for the electric or magnetic field components (the partials of which are scattered across the original eight different equations.)  Logically the electric and magnetic field components have to be kept together.

Solution of this equation will require some new tools.  Minimally, some relearning of existing vector calculus tools is required.

\subsection{Components of the geometric product Maxwell equation}

Explicit expansion of this equation, again using \(I={\Be}_1{\Be}_2{\Be}_3\), will yield a scalar, vector, bivector, and pseudoscalar components, and is an interesting exercise to verify the simpler field equation really describes the same thing.

\begin{dmath}\label{eqn:maxwellsGa:360}
\left(\frac{1}{c} \frac{\partial}{\partial t} + \nabla\right)\BF
= \frac{1}{c} \frac{\partial \BE}{\partial t} + I\frac{1}{c} \frac{\partial \BB}{\partial t}
+ \nabla \cdot \BE + \nabla \wedge \BE + \nabla \cdot I \BB + \nabla \wedge I \BB
\end{dmath}

The imaginary part of the field can be multiplied out as bivector components explicitly

\begin{equation}\label{eqn:maxwellsGa:620}
\begin{aligned}
I \BB &= \Be _1 \Be _2 \Be _3 ( \Be _1 B_1 + \Be _2 B_2 + \Be _3 B_3 ) \\
&= \Be _2 \Be _3 B_1 + \Be _3 \Be _1 B_2 + \Be _1 \Be _2 B_3
\end{aligned}
\end{equation}

which allows for direct calculation of the following

\begin{dmath}\label{eqn:maxwellsGa:380}
\nabla \wedge I\BB = I\nabla \cdot \BB
\end{dmath}
\begin{dmath}\label{eqn:maxwellsGa:400}
\nabla \cdot I\BB = -\nabla \times \BB.
\end{dmath}

These can be demonstrated by reducing \( \gpgradethree{ \nabla I \BB } \), and \( \gpgradeone{ \nabla I \BB } \) respectively.  Using these identities and writing the electric field curl term in terms of the cross product

\begin{dmath}\label{eqn:maxwellsGa:420}
\nabla \wedge \BE = I \nabla \times \BE,
\end{dmath}

allows for grouping of real and imaginary scalar and real and imaginary vector (bivector) components

\begin{dmath}\label{eqn:maxwellsGa:440}
   \left(\nabla \cdot \BE\right) + I\left(\nabla \cdot \BB\right)
+
   \left(\frac{1}{c} \frac{\partial \BE}{\partial t} - \nabla \times \BB\right)
+ I\left(\frac{1}{c} \frac{\partial \BB}{\partial t} + \nabla \times \BE\right)
\end{dmath}
\begin{dmath}\label{eqn:maxwellsGa:460}
= \frac{\rho}{\epsilon_0} + I\left(0\right) + \left(-\frac{\BJ}{\epsilon_0 c}\right) + I \Bzero.
\end{dmath}

Comparing each of the left and right side components recovers the original set of four (or eight depending on your point of view) Maxwell's equations.

%\section{Future: comparison to gravitation?}
%
%% Wed 03/05/2008
%
%The high school electrostatics equation, where \(\rho\) is either a continuous distribution or a spatial delta function for point masses:
%
%\begin{dmath}\label{eqn:maxwellsGa:480}
%\BE(\Br) = \inv{4 \pi \epsilon_0}\int{\rho(\Br') \frac{(\Br -\Br')}{(\Br -\Br')^2}}dV'
%\end{dmath}
%
%As a field equation this is written:
%\begin{dmath}\label{eqn:maxwellsGa:500}
%\nabla \cdot \BE(\Br) = \frac{\rho(\Br)}{\epsilon_0}
%\end{dmath}
%
%but this is both not relativistically correct nor does is
%include the propagation effects for ``electrostatics'' interactions
%which occur at the speed of light.
%
%We need the other three components of the Maxwell's
%equation \eqnref{eqn:maxwellsGa:5}, to get the propagation and relativistic
%corrections.
%
%Compare this to newton's gravitational field equation:
%
%\begin{dmath}\label{eqn:maxwellsGa:520}
%\BG(\Br) = -G\int{\rho(\Br') \frac{(\Br -\Br')}{(\Br -\Br')^2}}dV'
%\end{dmath}
%
%% G = 1/4 pi e
%% 4piG = 1/e
%
%which can be written as a field equation as:
%\begin{dmath}\label{eqn:maxwellsGa:540}
%\nabla \cdot \BG(\Br) = 4\pi G \rho(\Br).
%\end{dmath}
%
%If one assumes that electrodynamics and gravitation
%have the same form then is the corrected form of the gravitational field
%equation with respect to relativity and propagation at the speed of light
%as follows:
%
%\begin{equation}\label{eqn:maxwellsGa:600}
%\left(\frac{1}{c} \frac{\partial}{\partial t} + \nabla\right)\BG(\Br) =
%4 \pi G \rho(\Br)
%\end{equation}
%
%Is this correct in any sense?  Perhaps it matches the special relativity results
%but not the general relativity ones?
