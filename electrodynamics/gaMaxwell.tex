%
% Copyright � 2012 Peeter Joot.  All Rights Reserved.
% Licenced as described in the file LICENSE under the root directory of this GIT repository.
%

%
%
\mychapter{Back to Maxwell's equations}
\label{chap:PJMaxwell2}
\index{Maxwell's equations}
%\date{ July 12, 2008. Revised \(Date: 2009/06/04 13:13:27 \)}

%\(Id: gaMaxwell.tex,v 1.12 2009/06/04 13:13:27 Peeter Exp \)

\section{}

Having observed and demonstrated that the Lorentz transformation is a natural consequence of requiring the electromagnetic wave equation retains the
form of the wave equation under change of space and time variables that includes a velocity change in one spacial direction.

Lets step back and look at Maxwell's equations in more detail.  In particular looking at how we get from integral to differential
to GA form.  Some of this is similar to the approach in GAFP, but that text is intended for more mathematically sophisticated readers.

We start with the equations in SI units:

\begin{equation}\label{eqn:gaMaxwell:20}
\begin{aligned}
\int_{S(\text{closed boundary of V})} \BE \cdot \ncap dA &= \inv{\epsilon_0} \int_V \rho dV \\
\int_{S(\text{any closed surface})} \BB \cdot \ncap dA &= 0 \\
\int_{C(\text{boundary of S})} \BE \cdot d\Bx &= - \int_{S} \frac{\partial \BB}{\partial t} \cdot \ncap dA \\
\int_{C(\text{boundary of S})} \BB \cdot d\Bx &= \mu_0\left(I + \epsilon_0\int_{S} \frac{\partial \BE}{\partial t} \cdot \ncap dA\right) \\
\end{aligned}
\end{equation}

As the surfaces and corresponding loops or volumes are made infinitely small, these equations (FIXME: demonstrate), can be written in differential form:

\begin{equation}\label{eqn:gaMaxwell:40}
\begin{aligned}
\spacegrad \cdot \BE &= \frac{\rho}{\epsilon_0} \\
\spacegrad \cdot \BB &= 0 \\
\spacegrad \cross \BE &= - \frac{\partial \BB}{\partial t} \\
\spacegrad \cross \BB &= \mu_0\left(\BJ + \epsilon_0 \frac{\partial \BE}{\partial t}\right) \\
\end{aligned}
\end{equation}

These are respectively, Gauss's Law for E, Gauss's Law for B, Faraday's Law, and the Ampere/Maxwell's Law.

This differential form can be manipulated to derive the wave equation for free space, or the wave equation with charge and current forcing terms in other space.

\subsection{Regrouping terms for dimensional consistency}

Derivation of the wave equation can be done nicely using geometric algebra, but first is it helpful to put these equations in a more dimensionally pleasant form.
Lets relate the dimensions of the electric and magnetic fields and the constants \(\mu_0, \epsilon_0\).

From Faraday's equation we can relate the dimensions of
\(\BB\), and \(\BE\):

\begin{equation}
\frac{[\BE]}{[d]} = \frac{[\BB]}{[t]}
\end{equation}

We therefore see that \(\BB\), and \(\BE\) are related dimensionally by a velocity factor.

Looking at the dimensions of the displacement current density in the Ampere/Maxwell equation we see:

\begin{equation}
\frac{[\BB]}{[d]} = [\mu_0\epsilon_0] \frac{[\BE]}{[t]}
\end{equation}

From the two of these the dimensions of the \(\mu_0\epsilon_0\) product can be seen to be:

\begin{equation}
[\mu_0\epsilon_0] = \frac{{[t]}^2}{{[d]}^2}
\end{equation}

So, we see that we have a velocity factor relating \(\BE\), and \(\BB\), and we also see that we have a squared velocity coefficient in Ampere/Maxwell's law.  Let us factor this out explicitly so that \(\BE\) and \(\BB\) take dimensionally consistent form:

\begin{equation}\label{eqn:gaMaxwell:60}
\begin{aligned}
\tau &= \frac{t}{\sqrt{\mu_0\epsilon_0}}  \\
\spacegrad \cdot \BE &= \frac{\rho}{\epsilon_0} \\
\spacegrad \cdot \frac{\BB}{\sqrt{\mu_0\epsilon_0}} &= 0 \\
\spacegrad \cross \BE &= - \frac{\partial}{\partial \tau} \frac{\BB}{\sqrt{\mu_0\epsilon_0}} \\
\spacegrad \cross \frac{\BB}{\sqrt{\mu_0\epsilon_0}} &= \sqrt{\frac{\mu_0}{\epsilon_0}} \BJ + \frac{\partial \BE}{\partial \tau}
\end{aligned}
\end{equation}

\subsection{Refactoring the equations with the geometric product}

Now that things are dimensionally consistent, we are ready to group these equations using the geometric product

\begin{equation}
\BA \BB = \BA \cdot \BB + \BA \wedge \BB = \BA \cdot \BB + i \BA \cross \BB
\end{equation}

where \(i = \Be_1\Be_2\Be_3\) is the spatial pseudoscalar.  By grouping the divergence and curl terms for each of \(\BB\), and \(\BE\) we can write vector gradient equations
for each of the Electric and Magnetic fields:

\begin{align}
\spacegrad \BE = \frac{\rho}{\epsilon_0} - i \frac{\partial}{\partial \tau} \frac{\BB}{\sqrt{\mu_0\epsilon_0}} \label{eqn:gaMax:grad_e} \\
\spacegrad \frac{\BB}{\sqrt{\mu_0\epsilon_0}} = i\sqrt{\frac{\mu_0}{\epsilon_0}} \BJ + i\frac{\partial \BE}{\partial \tau} \label{eqn:gaMax:grad_b}
\end{align}

Multiplication of \eqnref{eqn:gaMax:grad_b} with \(i\), and adding to \eqnref{eqn:gaMax:grad_e}, we have Maxwell's equations consolidated into:

\begin{equation}\label{eqn:gaMax:maxwelleb}
\spacegrad \left(\BE + i \frac{\BB}{\sqrt{\mu_0\epsilon_0}}\right) =
\left(\frac{\rho}{\epsilon_0} - \sqrt{\frac{\mu_0}{\epsilon_0}} \BJ\right)
- \frac{\partial}{\partial \tau} \left(\BE + \frac{i\BB}{\sqrt{\mu_0\epsilon_0}} \right)
\end{equation}

We see that we have a natural combined Electrodynamic field:

\begin{equation}
\BF = \epsilon_0\left(\BE + i \frac{\BB}{\sqrt{\mu_0\epsilon_0}}\right) = \epsilon_0\left(\BE + i c \BB\right)
\end{equation}

Note that here the \(\epsilon_0\) factor has been included as a convenience to remove it from the charge and current density terms later.  We have also looked ahead slightly and written:

\begin{equation}
c = \inv{\sqrt{\mu_0\epsilon_0}}
\end{equation}

The dimensional analysis above showed that this had dimensions of velocity.  This velocity is in fact the speed of light, and
we will see this more exactly when looking at the wave equation for electrodynamics.  Until that this can be viewed as a
nothing more than a convenient shorthand.

We use this to write (Maxwell's) \eqnref{eqn:gaMax:maxwelleb} as:

\begin{equation}\label{eqn:gaMax:maxwellvs}
\left(\spacegrad + \inv{c}\frac{\partial}{\partial t} \right) \BF = \rho - \frac{\BJ}{c}.
\end{equation}

These are still four equations, and the originals can be recovered by taking scalar, vector, bivector and trivector parts.  However, in this
consolidated form, we are able to see the structure more easily.

\subsection{Grouping by charge and current density}
\index{charge density}
\index{current density}

Before moving on to the wave equation, lets put equations \eqnref{eqn:gaMax:grad_e} and \eqnref{eqn:gaMax:grad_b} in a slightly more symmetric form,
grouping by charge and current density respectively:

\begin{align}
\spacegrad \BE + \frac{\partial ic\BB}{\partial ct} &= \frac{\rho}{\epsilon_0} \label{eqn:gaMax:gradE} \\
\spacegrad ic\BB + \frac{\partial \BE}{\partial ct} &= -\frac{\BJ}{\epsilon_0 c} \label{eqn:gaMax:gradB}
\end{align}

Here we see how spatial electric field variation and magnetic field time variation are related to charge density.  We also see the
opposite pairing, where
spatial magnetic field variation and electric field variation with time are related to current density.

TODO: examine Lorentz transformations of the coordinates here.

Perhaps the most interesting feature here is how the spacetime gradient ends up split across the \(\BE\) and \(\BB\) fields, but
it may not be worth revisiting this.  Let us move on.

\subsection{Wave equation for light}
\index{wave equation!light}

To arrive at the wave equation, we take apply the gradient twice to calculate the Laplacian.  First vector gradient is:

\begin{equation}\label{eqn:gaMax:gradF}
\spacegrad \BF = -\inv{c}\frac{\partial \BF}{\partial t} + \left(\rho - \frac{\BJ}{c}\right).
\end{equation}

Second application gives:

\begin{equation*}
\spacegrad^2 \BF = -\inv{c}\spacegrad\frac{\partial \BF}{\partial t} + \spacegrad \left(\rho - \frac{\BJ}{c}\right).
\end{equation*}

Assuming continuity sufficient for mixed partial equality, we can swap the order of spatial and time derivatives, and
substitute \eqnref{eqn:gaMax:gradF} back in.

\begin{equation}\label{eqn:gaMaxwell:120}
\begin{aligned}
\spacegrad^2 \BF
&= -\inv{c}\frac{\partial}{\partial t}\left( -\inv{c}\frac{\partial \BF}{\partial t} + \left(\rho - \frac{\BJ}{c}\right) \right) + \spacegrad \left(\rho - \frac{\BJ}{c}\right) \\
\end{aligned}
\end{equation}

Or,

\begin{equation} \label{eqn:gaMax:wave}
\left(\spacegrad^2 - \inv{c^2}\partial_{tt} \right)\BF = \left(\spacegrad - \inv{c}\partial_{t} \right) \left(\rho - \frac{\BJ}{c}\right)
\end{equation}

Now there are a number of things that can be read out of this equation.  The first is that in a charge and current free region the electromagnetic field is described by an unforced wave equation:

\begin{equation}
\left(\spacegrad^2 - \inv{c^2}\partial_{tt} \right)\BF = 0
\end{equation}

This confirms the meaning that was assigned to \(c\).
It is the speed that an electrodynamic wave propagates in a charge and current free region of space.

\subsection{Charge and current density conservation}
\index{charge conservation}
\index{current density conservation}

Now, lets look at the right hand side of \eqnref{eqn:gaMax:wave} a bit closer:

\begin{equation}\label{eqn:gaMaxwell:140}
\begin{aligned}
\left(\spacegrad -\partial_{ct} \right) \left(\rho - \frac{\BJ}{c}\right)
&= - \inv{c}\left(\frac{\partial \rho}{\partial t} + \spacegrad \cdot \BJ \right) + \spacegrad \rho - \inv{c}\spacegrad \wedge \BJ + \inv{c^2}\frac{\partial \BJ}{\partial t}
\end{aligned}
\end{equation}

Compare this to the left hand side of \eqnref{eqn:gaMax:wave} which has only vector and bivector parts.  This implies that the scalar components of the right hand side are zero.  Specifically:

\begin{equation*}
\frac{\partial \rho}{\partial t} + \spacegrad \cdot \BJ = 0
\end{equation*}

This is a statement of charge conservation, and is more easily interpreted in integral form:
%\partial_t \rho = -\spacegrad \cdot \BJ

\begin{equation}
-\int_{S(\text{closed boundary of V})} \BJ \cdot \ncap dA = \frac{\partial}{\partial t} \int_V \rho dV = \frac{\partial Q_{enc}}{\partial t}
\end{equation}

FIXME: think about signs fully here.

The flux of the current density vector through a closed surface equals the time rate of change of the charge enclosed by that volume (ie: the current).  This could perhaps be viewed as the definition of the current density itself.  This fact would probably be more obvious if I did the math myself to demonstrate exactly how to take Maxwell's equations in integral form and convert those to their differential form.  In lieu of having done that proof myself I can at least determine this as a side effect of a bit of math.

\subsection{Electric and Magnetic field dependence on charge and current density}

Removing the explicit scalar terms from \eqnref{eqn:gaMax:wave} we have:

\begin{equation*}
\left(\spacegrad^2 - \partial_{ct, ct}\right) \BF =
\inv{c } \left(\spacegrad c \rho + \frac{\partial \BJ}{\partial c t} \right)
- \inv{c } \spacegrad \wedge \BJ
\end{equation*}

This shows explicitly how the charge and current forced wave equations for the
electric and magnetic fields is split:

\begin{equation*}
\left(\spacegrad^2 - \partial_{ct, ct}\right) \BE =
\inv{c } \left(\spacegrad c \rho + \frac{\partial \BJ}{\partial c t} \right)
\end{equation*}

\begin{equation*}
\left(\spacegrad^2 - \partial_{ct, ct}\right) \BB = - \inv{c^2 }  \spacegrad \cross \BJ
\end{equation*}

\subsection{Spacetime basis}

Now, if we look back to Maxwell's equation in the form of \eqnref{eqn:gaMax:maxwellvs}, we have a spacetime ``gradient'' with vector and scalar parts, an electrodynamic field with vector and trivector parts, and a charge and current density term with scalar and vector parts.

It is still rather confused, but it all works out, and one can recover the original four vector equations by taking scalar, vector, bivector, and trivector parts.

We want however to put this into a natural orderly fashion, and can do so if we use a normal bivector basis for all the spatial basis vectors, and factor out a basis vector from that for each of the scalar (timelike) factors.

Since bivectors over a Euclidean space have negative square, and this is not what we want for our Euclidean basis, and will have to
pick a bivector basis with a mixed metric.  We will see that this defines a Minkowski metric space.  Amazingly, by the simple
desire that we want to express
Maxwell's equations be written in the most orderly fashion, we arrive at the mixed signature spacetime metric that is the basis
of special relativity.
% (and we can use either signature as convienent so long as we do so consistently.)

Now, perhaps the reasons why to try to factor the spatial basis into a bivector basis are not obvious.  It is worth noting that
we have suggestions of conjugate operations above.  Examples of this are the charge and current terms with alternate signs, and
the alternation in sign in the wave equation itself.  Also worth pointing out is the natural appearance of a complex factor \(i\)
in Maxwell's equation coupled with the time term (that idea is explored more in ../maxwell/maxwell.pdf).  This coupling was
observed long ago and Minkowski's original paper refactors Maxwell's equation using it.  Now we have also seen that complex numbers
are isomorphic with a scalar plus vector representation.  Quaternions, which were originally ``designed'' to fit naturally
in Maxwell's equation and express the inherent structure are exactly this, a scalar and bivector sum.  There is a lot of history
that leads up to this idea, and the ideas here are not too surprising with some reading of the past attempts to put structure to these
equations.

On to the math...

Having chosen to find a bivector representation for our spatial basis vectors we write:

\begin{equation*}
\Be_i
= \gamma_i \wedge \gamma_0
= \gamma_i \gamma_0
= \gamma^0 \wedge \gamma^i
= \gamma^0 \gamma^i
\end{equation*}

For our Euclidean space we want

\begin{equation*}
(\Be_i)^2 = \gamma_i \gamma_0 \gamma_i \gamma_0 = -(\gamma_i)^2 (\gamma_0)^2 = 1
\end{equation*}

This implies the mixed signature:

\begin{equation*}
(\gamma_i)^2 = -(\gamma_0)^2 = \pm 1
\end{equation*}

We are free to pick either \(\gamma_0\) or \(\gamma_i\) to have a negative square, but following GAFP we use:

\begin{equation}\label{eqn:gaMaxwell:160}
\begin{aligned}
(\gamma_0)^2 &= 1 \\
(\gamma_i)^2 &= -1 \\
\gamma^0 &= \gamma_0 \\
\gamma^i &= -\gamma_i
\end{aligned}
\end{equation}

Now, lets translate the other scalar, vector, bivector, and trivector representations to use this alternate basis, and see what we get.  Start with the spacial pseudoscalar that is part of our magnetic field:

\begin{equation}\label{eqn:gaMaxwell:180}
\begin{aligned}
i
&= \Be_{123} \\
&= \gamma_{102030} \\
&= -\gamma_{012030} \\
&= \gamma_{012300} \\
&= \gamma_{0123}
\end{aligned}
\end{equation}

We see that the three dimensional pseudoscalar represented with this four dimensional basis is in fact also a pseudoscalar for that space.  Lets now use this to expand the trivector part of our electromagnetic field in this new basis:

\begin{equation}\label{eqn:gaMaxwell:200}
\begin{aligned}
i \BB
&= \sum i \Be_{i} B^i
&= \sum \gamma_{0123i0} B^i
&=
\gamma_{32} B^1
+\gamma_{13} B^2
+\gamma_{21} B^3
\end{aligned}
\end{equation}
%012310 = 011230 = 11 0230 = (11 00) 23
%012320 = -012230 = 22 0130 = (22 00) 13
%012330 = 33 0120 = (33 00) 21

So we see that our electromagnetic field has a bivector only representation with this mixed signature basis:

\begin{equation}
\BF = \BE + ic\BB = \gamma_{10} E^1 +\gamma_{20} E^2 +\gamma_{30} E^3 +\gamma_{32} c B^1 +\gamma_{13} c B^2 +\gamma_{21} c B^3
\end{equation}

Each of the possible bivector basis vectors is associated with a component of the combined electromagnetic field.
I had the signs wrong initially for the \(\BB\) components, but I think it is right now (and signature independent in fact).  ?  If I did get it wrong the idea is the same ... \(F\) is naturally
viewed as a pure bivector, which fits well with the fact that the tensor formulation is two completely antisymmetric rank two tensors.

Now, lets look at the spacetime gradient terms, first writing the spacial gradient in index form:

\begin{equation}\label{eqn:gaMaxwell:220}
\begin{aligned}
\spacegrad
&= \sum \Be^i \frac{\partial}{\partial x^i} \\
&= \sum \Be_i \frac{\partial}{\partial x^i} \\
&= \sum \gamma_i \gamma_0 \frac{\partial}{\partial x^i} \\
&= \gamma_0 \sum \gamma^i \frac{\partial}{\partial x^i}.
\end{aligned}
\end{equation}

This allows the spacetime gradient to be written in vector form replacing the vector plus scalar formulation:

\begin{equation}\label{eqn:gaMaxwell:240}
\begin{aligned}
\spacegrad + \partial_{ct}
&= \gamma_0 \sum \gamma^i \frac{\partial}{\partial x^i} + \partial_{ct} \\
&= \gamma_0 \left(\sum \gamma^i \frac{\partial}{\partial x^i} + \gamma^0 \partial_{ct} \right) \\
&= \gamma_0 \sum \gamma^{\mu} \frac{\partial}{\partial x^i}
\end{aligned}
\end{equation}

Observe that after writing \(x^0 = ct\) we can factor out the \(\gamma_0\), and write the spacetime gradient in pure vector form, using this mixed signature basis.

Now, let us do the same thing for the charge and current density terms, writing \(\BJ = e_i J^i\):

\begin{equation}\label{eqn:gaMaxwell:260}
\begin{aligned}
\rho - \frac{\BJ}{c}
&= \inv{c} \left( c\rho - \sum \Be_i J^i \right) \\
&= \inv{c} \left( c\rho - \sum \gamma_i \gamma_0 J^i \right) \\
&= \inv{c} \left( c\rho + \gamma_0 \sum \gamma_i J^i \right) \\
&= \gamma_0 \inv{c} \left( \gamma_0 c \rho + \sum \gamma_i J^i \right) \\
\end{aligned}
\end{equation}

Thus after writing \(J^0 = c \rho\), we have:

\begin{equation*}
\rho - \frac{\BJ}{c} = \gamma_0 \inv{c} \sum \gamma_{\mu} J^{\mu}
\end{equation*}

Putting these together and canceling out the leading \(\gamma_0\) terms we have the final result:

\begin{equation}
\sum \gamma^{\mu} \frac{\partial}{\partial x^i} \BF = \inv{c} \sum \gamma_{\mu} J^{\mu}.
\end{equation}

Or with a four-gradient \(\grad = \sum \gamma^{\mu} \frac{\partial}{\partial x^i}\), and four current \(J = \sum \gamma_{\mu} J^{\mu}\), we have Maxwell's equation in their most compact and powerful form:

\begin{equation}\label{eqn:gaMax:maxwell}
\grad \BF = \frac{J}{c}.
\end{equation}

\subsection{Examining the GA form Maxwell equation in more detail}

From \eqnref{eqn:gaMax:maxwell}, the wave equation becomes quite simple to derive.  Lets look at this again from this point of
view.  Applying the gradient we have:

\begin{equation}
\grad^2 \BF = \frac{\grad J}{c}.
\end{equation}

\begin{equation}
\grad^2 = \grad \cdot \grad = \sum (\gamma^{\mu})^2 \partial_{x^{\mu},x^{\mu}} = -\spacegrad^2 + \inv{c^2}\partial_{tt}.
\end{equation}

Thus for a charge and current free region, we still have the wave equation.

Now, lets look at the right hand side, and verify that it meets the expectations:

\begin{equation}\label{eqn:gaMax:waveright}
\inv{c}\grad J = \inv{c}\left(\grad \cdot J + \grad \wedge J\right)
\end{equation}

First thing to observe is that the left hand side is a pure spacetime bivector, which implies that the scalar part of
\eqnref{eqn:gaMax:waveright} is zero as we previously observed.  Lets verify that this is still the charge conservation
condition:

\begin{equation}\label{eqn:gaMaxwell:280}
\begin{aligned}
0
&= \grad \cdot J \\
&= (\sum \gamma^{\mu} \partial_{\mu}) \cdot \sum \gamma_{\nu} J^{\nu} \\
&= \sum \gamma^{\mu} \cdot \gamma_{\nu} \partial_{\mu} J^{\nu} \\
&= \sum \delta^{\mu}_{\nu} \partial_{\mu} J^{\nu} \\
&= \sum \partial_{\mu} J^{\mu} \\
&= \partial_{ct}(c \rho) + \sum \partial_{i} J^{i} \\
\end{aligned}
\end{equation}

This is our previous result:

\begin{equation}
\frac{\partial \rho}{\partial_{t}} + \spacegrad \cdot \BJ = 0
\end{equation}

This allows a slight simplification of the current forced wave equation for an electrodynamic field, by taking just the bivector
parts:

\begin{equation}\label{eqn:gaMax:waveF}
\left(\spacegrad^2 - \inv{c^2}\partial_{tt}\right) \BF = -\grad \wedge \frac{J}{c}
\end{equation}

Now we know how to solve the left hand side of this equation in its homogeneous form, but the four space curl term on the right is
new.
%, and we need some new tools to deal with it.  Next step in the learning process looks like it has to be geometric calculus.

%Perhaps do not need new tools.
This is really a set of six equations, subject to coupled boundary value conditions.  Written this out in components, one for each \(F \cdot (\gamma^{\nu} \wedge \gamma^{\mu})\) term and the corresponding terms of the right hand side one ends up with:

\begin{equation*}
-\grad^2 \BE = \spacegrad \rho/\epsilon_0 + \mu_0 \partial_t{\BJ}
\end{equation*}
\begin{equation*}
-\grad^2 \BB = -\mu_0 \spacegrad \cross \BJ
\end{equation*}

I have not bothered transcribing my notes for how to get this.  One way (messy) was starting with \eqnref{eqn:gaMax:waveF} and dotting with \(\gamma^{\nu \mu}\) to calculate the tensor \(F^{\mu\nu}\) (components of which are \(E\) and \(B\) components).  Doing the same for the spacetime curl term the end result is:

\begin{equation*}
(\grad \wedge J) \cdot (\gamma^{\nu\mu}) = \partial_{\mu}J^{\nu} (\gamma^{\mu})^2 - \partial_{\nu}J^{\mu} (\gamma^{\nu})^2
\end{equation*}

For a spacetime split of indices one gets the \(\spacegrad\rho\), and \(\partial_t \BJ\) term, and for a space-space pair of indices one gets the spacial curl in the \(\BB\) equation.

An easier starting point for this is actually using equations \eqnref{eqn:gaMax:gradE} and \eqnref{eqn:gaMax:gradB} since they are already split into \(\BE\), and \(\BB\) fields.

\subsection{Minkowski metric}

Having observed that a mixed signature bivector basis with a space time mix of underlying basis vectors is what we want to
express Maxwell's equation in its most simple form, now lets step back and look at that in a bit more detail.  In particular
lets examine the dot product of a four vector with such a basis.  Our current density four vector is one such vector:

\begin{equation}\label{eqn:gaMaxwell:300}
\begin{aligned}
J^2 = J \cdot J = \sum (J^{\mu})^2 (\gamma_{\mu})^2 = (c\rho)^2 - \BJ^2
\end{aligned}
\end{equation}

The coordinate vector that is forms the partials of our four gradient is another such vector:

\begin{equation*}
x = (ct, x^1, x^2, x^3) = \sum \gamma_{\mu} x^{\mu}
\end{equation*}

Again, the length applied to this vector is:

\begin{equation}
x^2 = x \cdot x = (ct)^2 - \Bx^2
\end{equation}

As a result of nothing more than a desire to put Maxwell's equations into structured form, we have the special relativity metric
of Minkowski and Einstein.
