%
% Copyright � 2012 Peeter Joot.  All Rights Reserved.
% Licenced as described in the file LICENSE under the root directory of this GIT repository.
%

%
%
\chapter{Gaussian Surface invariance for radial field}
\label{chap:gaussianSurface}
%\date{November 22, 2008.  gaussianSurface.tex}

\section{Flux independence of surface}

\imageFigure{../figures/gabook/surface_flux_element}{Flux through tilted spherical surface element}{fig:surface_flux_element}{0.4}

In \citep{purcell1963eam}, section \(1.10\) is a demonstration that the flux
through any closed surface is the same as that through a sphere.

A similar demonstration of the same is possible using a spherical polar basis \(\{\rcap, \thetacap, \phicap\}\) with an element of surface area that is tilted slightly as illustrated in \cref{fig:surface_flux_element}.

The tangential surface on the sphere at radius \(r\) will have bivector

\begin{equation}\label{eqn:gaussianSurface:20}
\begin{aligned}
d\BA_r = r^2 d\theta d\phi \thetacap\phicap
\end{aligned}
\end{equation}

where \(d\theta\), and \(d\phi\) are the subtended angles (should have put them in the figure).

Now, as in the figure we want to compute the bivector for the tilted surface at radius \(R\).  The vector \(\Bu\) in the figure is required.
This is \(\rcap R + R d\theta \thetacap - \rcap(R + dr)\), so the bivector for that area element is

\begin{equation}\label{eqn:gaussianSurface:40}
\begin{aligned}
\left(R \rcap + R d\theta \thetacap - (R + dr) \rcap \right) \wedge {R d\theta \phicap}
&= \left(R d\theta \thetacap - dr \rcap \right) \wedge {R d\phi \phicap} \\
\end{aligned}
\end{equation}

For
\begin{equation}\label{eqn:gaussianSurface:60}
\begin{aligned}
d\BA_R = R^2 d\theta d\phi \thetacap \phicap - R dr d\phi \rcap \phicap
\end{aligned}
\end{equation}

Now normal area elements can be calculated by multiplication with a \R{3} pseudoscalar such as \(I = \rcap \thetacap \phicap\).

\begin{equation}\label{eqn:gaussianSurface:80}
\begin{aligned}
\ncap_r \Abs{d\BA_r}
&= r^2 d\theta d\phi \rcap \thetacap \phicap \thetacap\phicap \\
&= -r^2 d\theta d\phi \rcap \\
\end{aligned}
\end{equation}

And

\begin{equation}\label{eqn:gaussianSurface:100}
\begin{aligned}
\ncap_R \Abs{d\BA_R}
&= \rcap \thetacap \phicap \left( R^2 d\theta d\phi \thetacap \phicap - R dr d\phi \rcap \phicap \right) \\
&= - R^2 d\theta d\phi \rcap - R dr d\phi \thetacap
\end{aligned}
\end{equation}

Calculating \(\BE \cdot \ncap dA\) for the spherical surface element at radius \(r\) we have

\begin{equation}\label{eqn:gaussianSurface:120}
\begin{aligned}
\BE(r) \cdot \ncap_r \Abs{d\BA_r}
&= \inv{4 \pi \epsilon_0 r^2} q \rcap \cdot (-r^2 d\theta d\phi \rcap) \\
&= \frac{-d\theta d\phi q}{4 \pi \epsilon_0}
\end{aligned}
\end{equation}

and for the tilted surface at \(R\)

\begin{equation}\label{eqn:gaussianSurface:140}
\begin{aligned}
\BE(R) \cdot \ncap_R \Abs{d\BA_R}
&= \frac{q}{4 \pi \epsilon_0 R^2} \rcap \cdot \left(- R^2 d\theta d\phi \rcap - R dr d\phi \thetacap \right) \\
&= \frac{-d\theta d\phi q}{4 \pi \epsilon_0}
\end{aligned}
\end{equation}

The \(\thetacap\) component of the surface normal has no contribution to the flux since it is perpendicular to the outwards (\(\rcap\) facing) field.  Here the particular normal to the surface happened to be inwards facing due to choice of the pseudoscalar, but because the normals chosen in each case had the same orientation this does not make a difference to the equivalence result.

\subsection{Suggests dual form of Gauss's law can be natural}
\index{Gauss's law}

The fact that the bivector area elements work well to describe the surface
can also be used to write Gauss's law in an alternate form.  Let \(\ncap dA = -I d\BA\)

\begin{equation}\label{eqn:gaussianSurface:160}
\begin{aligned}
\BE \cdot \ncap dA
&= -\BE \cdot (I d\BA) \\
&= \frac{-1}{2} ( \BE I d\BA + I d\BA \BE ) \\
&= \frac{-I}{2} ( \BE d\BA + d\BA \BE ) \\
&= -I ( \BE \wedge d\BA )
\end{aligned}
\end{equation}

So for

\begin{equation}\label{eqn:gaussianSurface:180}
\begin{aligned}
\int \BE \cdot \ncap dA
&= \int \frac{\rho}{\epsilon_0} dV
\end{aligned}
\end{equation}

with \(d\BV = I dV\), we have Gauss's law in dual form:

\begin{equation}\label{eqn:gaussianSurface:200}
\begin{aligned}
\int \BE \wedge d\BA &= \int \frac{\rho}{\epsilon_0} d\BV
\end{aligned}
\end{equation}

Writing Gauss's law in this form it becomes almost obvious that we can
deform the surface without changing the flux, since all the non-tangential
surface elements will have an \(\rcap\) factor and thus produce a zero
once wedged with the radial field.

