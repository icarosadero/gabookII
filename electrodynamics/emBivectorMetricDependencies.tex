%
% Copyright � 2012 Peeter Joot.  All Rights Reserved.
% Licenced as described in the file LICENSE under the root directory of this GIT repository.
%

%
%
\mychapter{Metric signature dependencies}
\index{metric}
\label{chap:emBivectorMetricDependencies}
%\date{Sept 5, 2008.  emBivectorMetricDependencies.tex}

\section{Motivation}

Doran/Lasenby use a \(+,-,-,-\) signature, and I had gotten used to that.  On first seeing the alternate signature used by
\href{http://www.av8n.com/physics/maxwell-ga.pdf}{ John Denker's excellent GA explanatory paper },
I found myself disoriented.  How many of the identities that I was used to were metric dependent?   Here are some notes that explore some of the
metric dependencies of STA, in particular observing which identities are metric dependent and which are not.

In the end this exploration turned into a big meandering examination and comparison of the bivector and tensor forms of Maxwell's equation.  That part has been split into a different writeup.

\section{The guts}

\subsection{Spatial basis}

Our spatial (bivector) basis:
%
\begin{equation*}
\sigma_i = \gamma_i \wedge \gamma_0 = \gamma_{i0},
\end{equation*}
%
that behaves like Euclidean vectors (positive square) still behave as desired, regardless of the signature:
%
\begin{equation}\label{eqn:emBivectorMetricDependencies:20}
\begin{aligned}
\sigma_i \cdot \sigma_j
&= \gpgradezero{\gamma_{i0j0}}  \\
&= - \gpgradezero{\gamma_{ij}} (\gamma_{0})^2  \\
&= -\delta_{ij} (\gamma_i)^2 (\gamma_{0})^2
\end{aligned}
\end{equation}
%
Regardless of the signature the pair of products \((\gamma_i)^2 (\gamma_{0})^2 = -1\), so our spatial bivectors are metric invariant.

\subsection{How about commutation?}
%
Commutation with
\begin{equation*}
i \gamma_{\mu} = \gamma_{0123\mu} = \gamma_{\mu0123}
\end{equation*}
%
\(\mu\) has to "pass" three indices regardless of metric, so anticommutes for any \(\mu\).
%
\begin{equation*}
\sigma_k \gamma_{\mu} = \gamma_{k0\mu}
\end{equation*}
%
If \(k = \mu\), or \(0 = \mu\), then we get a sign inversion, and otherwise commute (pass two indices).  This is also metric invariant.

\subsection{Spatial and time component selection}

With a positive time metric (Doran/Lasenby) selection of the \(x^0\) component of a vector \(x\) requires a dot product:
%
\begin{equation*}
x = x^0 \gamma_0 + x^i \gamma_i
\end{equation*}
%
\begin{equation*}
x \cdot \gamma_0 = x^0 (\gamma_0)^2
\end{equation*}
%
Obviously this is a metric dependent operation.  To generalize it appropriately, we need to dot with \(\gamma^0\) instead:
%
\begin{equation*}
x \cdot \gamma^0 = x^0
\end{equation*}
%
Now, what do we get when wedging with this upper index quantity instead.
%
\begin{equation}\label{eqn:emBivectorMetricDependencies:40}
\begin{aligned}
x \wedge \gamma^0
&= \left(x^0 \gamma_0 + x^i \gamma_i\right) \wedge \gamma^0 \\
&= x^i \gamma_i \wedge \gamma^0 \\
&= x^i \gamma_{i0} (\gamma^0)^2 \\
&= x^i \sigma_i (\gamma^0)^2 \\
&= \Bx \left(\gamma^0\right)^2
\end{aligned}
\end{equation}
%
Not quite the usual expression we are used to, but it still behaves as a Euclidean vector (positive square), regardless of the metric:
%
\begin{equation*}
(x \wedge \gamma^0)^2 = \left(\pm \Bx\right)^2 = \Bx^2
\end{equation*}
%
This suggests that we should define our spatial projection vector as \(x \wedge \gamma^0\) instead of \(x \wedge \gamma_0\) as done in
Doran/Lasenby (where a positive time metric is used).

\subsubsection{Velocity}

Variation of a event path with some parameter we have:
%
\begin{equation}\label{eqn:emBivectorMetricDependencies:60}
\begin{aligned}
\frac{ d x }{ d \lambda }
&= \frac{ d x^{\mu} }{ d \lambda } \gamma_{\mu} = c \frac{dt}{d\lambda} \gamma_0 + \frac{d x^i }{d\lambda} \gamma_i \\
&= \frac{d t}{d \lambda} \left( c \gamma_0 + \frac{d x^i }{dt} \gamma_i \right)
\end{aligned}
\end{equation}
%
The square of this is:
%becomes metric dependent:
\begin{equation}\label{eqn:emBivectorMetricDependencies:80}
\begin{aligned}
\inv{c^2} \left(\frac{ d x }{ d \lambda } \right)^2
&= \left(\frac{dt }{d\lambda}\right)^2 (\gamma_0)^2 \left( 1 + \inv{c^2}\left(\frac{d x^i }{dt}\right)^2 (\gamma_i)^2 (\gamma_0)^2 \right) \\
&= \left(\frac{d t}{d \lambda}\right)^2 (\gamma_0)^2 \left( 1 - (\Bv/c)^2 \right) \\
\frac{ (\gamma_0)^2 }{c^2} \left(\frac{ d x }{ d \lambda } \right)^2 &= \left(\frac{d t}{d \lambda}\right)^2 \left( 1 - (\Bv/c)^2 \right) \\
\end{aligned}
\end{equation}
%
We define the proper time \(\tau\) as that particular parametrization \(c \tau = \lambda\) such that the LHS equals 1.  This is implicitly defined
via the integral
%
\begin{equation*}
\tau = \int \sqrt{ 1 - (\Bv/c)^2 } dt = \int \sqrt{ 1 - \left(\inv{c} \frac{dx^i }{d \alpha} \right)^2 } d\alpha
\end{equation*}
%
Regardless of this parametrization \(\alpha = \alpha(t)\), this velocity scaled 4D arc length is the same.  This is a bit of a digression from the
ideas of metric dependence investigation.  There is however a metric dependence in the first steps arriving at this result.

with proper velocity defined in terms of proper time \(v = dx/d\tau\), we also have:
%
\begin{equation}\label{eqn:emBiMetDep:gamma}
\gamma = \frac{dt}{d\tau} = \inv{ \sqrt{ 1 - (\Bv/c)^2 } }
\end{equation}
\begin{equation}
v = \gamma \left(c \gamma_0 + \frac{d x^i }{d t} \gamma_i \right)
\end{equation}
%
Therefore we can select this quantity \(\gamma\), and our spatial velocity components, from our proper velocity:
%
\begin{equation*}
c \gamma = v \cdot \gamma^0
\end{equation*}
%
In \eqnref{eqn:emBiMetDep:gamma} we did not define \(\Bv\), only implicitly requiring that its square was \(\sum (dx^i/dt)^2\), as we require for correspondence with Euclidean meaning.  This can be made more exact by
taking wedge products to weed out the time component:
%
\begin{equation*}
v \wedge \gamma^0 = \gamma \frac{d x^i }{d t} \gamma_i \wedge \gamma^0
\end{equation*}
%
With a definition of \(\Bv = \frac{d x^i }{d t} \gamma_i \wedge \gamma^0\) (which has the desired positive square), we therefore have:
%
\begin{equation}\label{eqn:emBivectorMetricDependencies:100}
\begin{aligned}
\Bv
&= \frac{v \wedge \gamma^0 }{\gamma} \\
&= \frac{v \wedge \gamma^0 }{ v/c \cdot \gamma^0 } \\
\end{aligned}
\end{equation}
%
Or,
\begin{equation}
\Bv/c = \frac{v/c \wedge \gamma^0 }{ v/c \cdot \gamma^0 }
\end{equation}
%
All the lead up to this allows for expression of the spatial component of the proper velocity in a metric independent fashion.

\subsection{Reciprocal Vectors}
\index{reciprocal frame}

By reciprocal frame we mean the set of vectors \(\{u^{\alpha}\}\) associated with a basis
for some linear subspace \(\{u_{\alpha}\}\) such that:
%
\begin{equation*}
u_{\alpha} \cdot u^{\beta} = \delta_{\alpha}^\beta
\end{equation*}
%
In the special case of orthonormal vectors \(u_{\alpha} \cdot u_{\beta} = \pm \delta_{\alpha\beta}\) the reciprocal frame vectors
are just the inverses (literally reciprocals), which can be verified by taking dot products:
%
\begin{equation}\label{eqn:emBivectorMetricDependencies:120}
\begin{aligned}
\inv{u_{\alpha}} \cdot {u_{\alpha}}
&= \gpgradezero{ \inv{u_{\alpha}} {u_{\alpha}} } \\
&= \gpgradezero{ \inv{u_{\alpha}} \frac{u_{\alpha}}{u_{\alpha}} {u_{\alpha}} } \\
&= \gpgradezero{ \frac{(u_{\alpha})^2}{(u_{\alpha})^2} } \\
&= 1
\end{aligned}
\end{equation}
%
Written out explicitly for our positive "orthonormal" time metric:
%
\begin{equation}\label{eqn:emBivectorMetricDependencies:140}
\begin{aligned}
(\gamma_0)^2 &= 1 \\
(\gamma_i)^2 &= -1,
\end{aligned}
\end{equation}
%
we have the reciprocal vectors:
\begin{equation}\label{eqn:emBivectorMetricDependencies:160}
\begin{aligned}
\gamma_0 &= \gamma^0 \\
\gamma_i &= -\gamma^i \\
\end{aligned}
\end{equation}
%
Note that this last statement is consistent with \((\gamma_i)^2 = -1\), since \((\gamma_i)^2 = \gamma_i (-\gamma^i) = -\delta_i^i = -1\)

Contrast this with a positive spatial metric:
%
\begin{equation}\label{eqn:emBivectorMetricDependencies:180}
\begin{aligned}
(\gamma_0)^2 &= -1 \\
(\gamma_i)^2 &= 1,
\end{aligned}
\end{equation}
%
with reciprocal vectors:
\begin{equation}\label{eqn:emBivectorMetricDependencies:200}
\begin{aligned}
\gamma_0 &= -\gamma^0 \\
\gamma_i &= \gamma^i \\
\end{aligned}
\end{equation}
%
where we have the opposite.

\subsection{Reciprocal Bivectors}

Now, let us examine the bivector reciprocals.  Given our orthonormal vector basis, let us invert the bivector and verify that is what we want:
%
\begin{equation}\label{eqn:emBivectorMetricDependencies:220}
\begin{aligned}
\inv{\gamma_{\mu\nu}}
&= \inv{\gamma_{\mu\nu}} \frac{ \gamma_{\nu\mu} }{ \gamma_{\nu\mu} } \\
&= \inv{\gamma_{\mu\nu}} \inv{ \gamma_{\nu\mu} }{ \gamma_{\nu\mu} } \\
&= \inv{\gamma_{\mu\nu\nu\mu} }{ \gamma_{\nu\mu}} \\
&= \inv{ (\gamma_{\mu})^2 (\gamma_{\nu})^2 } { \gamma_{\nu\mu}} \\
\end{aligned}
\end{equation}
%
Multiplication with our vector we will get 1 if this has the required reciprocal relationship:
\begin{equation}\label{eqn:emBivectorMetricDependencies:240}
\begin{aligned}
\inv{\gamma_{\mu\nu}} \gamma_{\mu\nu}
&= \inv{ (\gamma_{\mu})^2 (\gamma_{\nu})^2 } { \gamma_{\nu\mu}} \gamma_{\mu\nu} \\
&= \frac{ (\gamma_{\mu})^2 (\gamma_{\nu})^2 }{ (\gamma_{\mu})^2 (\gamma_{\nu})^2 } \\
&= 1
\end{aligned}
\end{equation}
%
Observe that unlike our basis vectors the bivector reciprocals are metric independent.  Let us verify this explicitly:
%
\begin{equation}\label{eqn:emBivectorMetricDependencies:260}
\begin{aligned}
\inv{\gamma_{i0}} &= \inv{ (\gamma_{i})^2 (\gamma_{0})^2 } { \gamma_{0i}} \\
\inv{\gamma_{ij}} &= \inv{ (\gamma_{i})^2 (\gamma_{j})^2 } { \gamma_{ji}} \\
\inv{\gamma_{0i}} &= \inv{ (\gamma_{0})^2 (\gamma_{i})^2 } { \gamma_{i0}} \\
\end{aligned}
\end{equation}
%
With a spacetime mix of indices we have a \(-1\) denominator for either metric.  With a spatial only mix (\(B\) components) we have \(1\) in the denominator \(1^2 = (-1)^2\) for either metric.

Now, perhaps counter to intuition the reciprocal \(\inv{\gamma_{\mu\nu}}\) of \(\gamma_{\mu\nu}\) is not \(\gamma^{\mu\nu}\), but instead \(\gamma^{\nu\mu}\).  Here the shorthand can be deceptive and it is worth verifying this statement explicitly:
%
\begin{equation}\label{eqn:emBivectorMetricDependencies:280}
\begin{aligned}
\gamma_{\mu\nu} \cdot \gamma^{\alpha\beta}
&= (\gamma_{\mu} \wedge \gamma_{\nu}) \cdot (\gamma^{\alpha} \wedge \gamma^{\beta}) \\
&= ((\gamma_{\mu} \wedge \gamma_{\nu}) \cdot \gamma^{\alpha}) \cdot \gamma^{\beta}) \\
&= ( \gamma_{\mu} (\gamma_{\nu} \cdot \gamma^{\alpha}) - \gamma_{\nu} (\gamma_{\mu} \cdot \gamma^{\alpha}) ) \cdot \gamma^{\beta}) \\
&= ( \gamma_{\mu} {\delta_{\nu}}^{\alpha} - \gamma_{\nu} {\delta_{\mu}}^{\alpha} ) \cdot \gamma^{\beta} \\
\end{aligned}
\end{equation}
%
Or,
\begin{equation}
\gamma_{\mu\nu} \cdot \gamma^{\alpha\beta} = {\delta_{\mu}}^{\beta} {\delta_{\nu}}^{\alpha} - {\delta_{\nu}}^{\beta} {\delta_{\mu}}^{\alpha}
\end{equation}
%
In particular for matched pairs of indices we have:
\begin{equation*}
\gamma_{\mu\nu} \cdot \gamma^{\nu\mu} = {\delta_{\mu}}^{\mu} {\delta_{\nu}}^{\nu} - {\delta_{\nu}}^{\mu} {\delta_{\mu}}^{\nu} = 1
\end{equation*}
%
\subsection{Pseudoscalar expressed with reciprocal frame vectors}
\index{pseudoscalar}

With a positive time metric
%
\begin{equation*}
\gamma_{0123} = -\gamma^{0123}
\end{equation*}
%
(three inversions for each of the spatial quantities).  This is metric invariant too since it will match the single negation for the same operation
using a positive spatial metric.

\subsection{Spatial bivector basis commutation with pseudoscalar}
%
I have been used to writing:
\begin{equation*}
\sigma_j = \gamma_{j0}
\end{equation*}
%
as a spatial basis, and having this equivalent to the four-pseudoscalar, but this only works with a time positive metric:
\begin{equation*}
i_3 = \sigma_{123} = \gamma_{102030} = \gamma_{0123} (\gamma_0)^2
\end{equation*}
%
With the spatial positive spacetime metric we therefore have:
%
\begin{equation*}
i_3 = \sigma_{123} = \gamma_{102030} = -i_4
\end{equation*}
%
instead of \(i_3 = i_4\) as is the case with a time positive spacetime metric.  We see that the metric choice can also be interpreted as a choice of handedness.

That choice allowed Doran/Lasenby to initially write the field as a vector plus trivector where \(i\) is the spatial pseudoscalar:
%
\begin{equation}\label{eqn:emBiMetDep:field}
F = \BE + i c \BB,
\end{equation}
%
and then later switch the interpretation of \(i\) to the four space pseudoscalar.  The freedom to do so is metric dependent freedom, but
\eqnref{eqn:emBiMetDep:field} works regardless of metric when \(i\) is uniformly interpreted as the spacetime pseudoscalar.

Regardless of the metric the spacetime pseudoscalar commutes with \(\sigma_j = \gamma_{j0}\), since it anticommutes twice to cross:
%
\begin{equation*}
\sigma_j i = \gamma_{j00123} = \gamma_{00123j} = \gamma_{0123j0} = i \sigma_j
\end{equation*}
%
\subsection{Gradient and Laplacian}
\index{gradient}
\index{Laplacian}

As seen by the Lagrangian based derivation of the (spacetime or spatial) gradient, the form is metric independent and valid even for non-orthonormal frames:
%
\begin{equation*}
\grad = \gamma^{\mu} \PD{x^{\mu}}{}
\end{equation*}
%
\subsubsection{Vector derivative}
\index{vector derivative}

A cute aside, as pointed out in John Denker's paper, for orthonormal frames, this can also be written as:
%
\begin{equation}\label{eqn:emBiMetDep:gradient}
\grad = \inv{\gamma_{\mu}} \PD{x^{\mu}}{}
\end{equation}
%
as a mnemonic for remembering where the signs go, since in that form the upper and lower indices are nicely matched in summation convention fashion.

Now, \(\gamma_{\mu}\) is a constant when we are not working in curvilinear coordinates, and for constants we are used to the freedom to pull them into our
derivatives as in:
%
\begin{equation*}
\inv{c} \PD{t}{} = \PD{(ct)}{}
\end{equation*}
%
Supposing that one had an orthogonal vector decomposition:
%
\begin{equation*}
\Bx = \sum \gamma_i x^i = \sum \Bx_i
\end{equation*}
%
then, we can abuse notation and do the same thing with our unit vectors, rewriting the gradient \eqnref{eqn:emBiMetDep:gradient} as:
%
\begin{equation}\label{eqn:emBiMetDep:gradvec}
\grad = \PD{(\gamma_{\mu} x^{\mu})}{} = \sum \PD{\Bx_i}{}
\end{equation}
%
Is there anything to this that is not just abuse of notation?  I think so, and I am guessing the notational freedom to do this is closely related to
what Hestenes calls geometric calculus.

Expanding out the gradient in the form of \eqnref{eqn:emBiMetDep:gradvec} as a limit statement this becomes, rather loosely:
%
\begin{equation*}
\grad = \sum_i \lim_{d\Bx_i \to 0} \inv{ d \Bx_i } \left(f( \Bx + d\Bx_i ) - f( \Bx )\right)
\end{equation*}
%
If nothing else this justifies the notation for the polar form gradient of a function that is only radially dependent, where the quantity:
%
\begin{equation*}
\spacegrad = \rcap\PD{r}{} = \inv{\rcap}\PD{r}{}
\end{equation*}
%
is sometimes written:
%
\begin{equation*}
\spacegrad = \PD{\Br}{}
\end{equation*}
%
Tong does this for example in his online dynamics paper, although there it appears to be not much more than a fancy shorthand for gradient.

\subsection{Four-Laplacian}
\index{four-Laplacian}

Now, although our gradient is metric invariant, its square the four-Laplacian is not.  There we have:
%
\begin{equation}\label{eqn:emBivectorMetricDependencies:300}
\begin{aligned}
\grad^2
&= \sum (\gamma^{\mu})^2 \PDsQ{x^{\mu}}{} \\
&= (\gamma^0)^2 \left( \PDsQ{x^0}{} + (\gamma^0)^2 (\gamma^i)^2 \PDsQ{x^i}{} \right) \\
&= (\gamma^0)^2 \left( \PDsQ{x^0}{} - \PDsQ{x^i}{} \right)
\end{aligned}
\end{equation}
%
This makes the metric dependency explicit so that we have:
%
\begin{equation*}
\grad^2 = \inv{c^2} \PDsQ{t}{} - \PDsQ{x^i}{} \quad \mbox{if \((\gamma^0)^2 = 1\)}
\end{equation*}
\begin{equation*}
\grad^2 = \PDsQ{x^i}{} - \inv{c^2} \PDsQ{t}{} \quad \mbox{if \((\gamma^0)^2 = -1\)}
\end{equation*}
