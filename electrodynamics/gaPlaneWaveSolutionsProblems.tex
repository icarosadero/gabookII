%
% Copyright � 2013 Peeter Joot.  All Rights Reserved.
% Licenced as described in the file LICENSE under the root directory of this GIT repository.
%
\makeproblem{Electrodynamic plane wave constraints}{ch:gaPlaneWaveSolutions:pr1}{

It was claimed that
\begin{subequations}
\begin{equation}\label{eqn:gaPlaneWaveSolutionsProblems:430a}
\BE_0 =- \kcap \cross v \BB_0
\end{equation}
\begin{equation}\label{eqn:gaPlaneWaveSolutionsProblems:450b}
v \BB_0 = \kcap \cross \BE_0
\end{equation}
\end{subequations}

%equations \eqnref{eqn:gaPlaneWaveSolutionsProblems:430} and \eqnref{eqn:gaPlaneWaveSolutionsProblems:450}
relating the electric and magnetic field of electrodynamic plane waves were dependent.  Show this.
}
\makeanswer{ch:gaPlaneWaveSolutions:pr1}{
This can be shown by crossing \(\kcap\) with \eqnref{eqn:gaPlaneWaveSolutionsProblems:430a} and using the identity

\begin{equation}\label{eqn:gaPlaneWaveSolutionsProblems:490}
\Ba \cross (\Ba \cross \Bb) = - \Ba^2 \Bb + \Ba (\Ba \cdot \Bb).
\end{equation}

This gives
\begin{equation}\label{eqn:gaPlaneWaveSolutionsProblems:490c}
\begin{aligned}
\kcap \cross \BE_0
&= - \kcap \cross (\kcap \cross v \BB_0 ) \\
&= \kcap^2 v \BB_0 - \kcap ( \cancel{\kcap \cdot \BE_0} ) \\
&= v \BB_0.
\end{aligned}
\end{equation}
}
\makeproblem{Proving that the wavevectors are all coplanar}{ch:gaPlaneWaveSolutions:2}{

\citep{griffiths1999introduction} poses the following simple but excellent problem, related to the relationship between the incident, transmission and reflection phasors, which he states has the following form

\begin{equation}\label{eqn:gaPlaneWaveSolutionsProblems:510}
() e^{i (\Bk_i \cdot \Bx - \omega t)}
+ () e^{i (\Bk_r \cdot \Bx - \omega t)}
= () e^{i (\Bk_t \cdot \Bx - \omega t)},
\end{equation}

He poses the problem (9.15)

Suppose \(A e^{i a x} + B e^{i b x} = C e^{i c x}\) for some nonzero constants \(A\), \(B\), \(C\), \(a\), \(b\), \(c\), and for all \(x\).  Prove that \(a = b = c\) and \(A + B = C\).
}

\makeanswer{ch:gaPlaneWaveSolutions:2}{
If this relation holds for all \(x\), then for \(x = 0\), we have \(A + B = C\).  We are left to show that

\begin{equation}\label{eqn:gaPlaneWaveSolutionsProblems:530}
A \left( e^{i a x} - e^{i c x} \right)
+ B \left( e^{i b x} - e^{i c x} \right) = 0.
\end{equation}

Let \(a = c + \delta\) and \(b = c + \epsilon\), so that

\begin{equation}\label{eqn:gaPlaneWaveSolutionsProblems:550}
A \left( e^{i \delta x} - 1 \right)
+ B \left( e^{i \epsilon x} - 1 \right) = 0.
\end{equation}

Now consider some special values of \(x\).  For \(x = 2 \pi/\epsilon\) we have

\begin{equation}\label{eqn:gaPlaneWaveSolutionsProblems:570}
A \left( e^{2 \pi i \delta/\epsilon} - 1 \right) = 0,
\end{equation}

and because \(A \ne 0\), we must conclude that \(\delta/\epsilon\) is an integer.

Similarily, for \(x = 2 \pi/\delta\), we have

\begin{equation}\label{eqn:gaPlaneWaveSolutionsProblems:590}
B \left( e^{2 \pi i \epsilon/\delta} - 1 \right) = 0,
\end{equation}

and this time must conclude that \(\epsilon/\delta\) is an integer.  These ratios must therefore take one of the values \(0, 1, -1\).  Consider the points \(x = 2 n \pi/\epsilon\) or \(x = 2 m \pi/\delta\) we find that \(n \delta/\epsilon\) and \(m \epsilon/\delta\) must be integers for any integers \(m, n\).  This only leaves \(\epsilon = \delta = 0\), or \(a = b = c\) as possibilities.
}
