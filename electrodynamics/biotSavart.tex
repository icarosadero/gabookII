%
% Copyright � 2012 Peeter Joot.  All Rights Reserved.
% Licenced as described in the file LICENSE under the root directory of this GIT repository.
%

%
%
\mychapter{Biot Savart Derivation}
\label{chap:biotSavart}
\index{Biot Savart}
%\date{April 18, 2009.  biotSavart.tex}

\section{Motivation}

Looked at my Biot-Savart derivation in \chapcite{PJelectricFieldEnergy}.  There I was playing with doing this without first dropping down to the
familiar vector relations, and end up with an expression of the Biot Savart law in terms of the complete Faraday bivector.  This is
an excessive approach, albeit interesting (to me).  Let us try this again in terms of just the magnetic field.

\section{Do it}

\subsection{Setup. Ampere-Maxwell equation for steady state}
\index{Ampere-Maxwell equation}

The starting point can still be Maxwell's equation

\begin{equation}\label{eqn:biotSavart:20}
\begin{aligned}
\grad F = J/\epsilon_0 c
\end{aligned}
\end{equation}

and the approach taken will be the more usual consideration of a loop of steady-state (no-time variation) current.

In the steady state we have

\begin{equation}\label{eqn:biotSavart:40}
\begin{aligned}
\grad = \gamma^0 \inv{c} \partial_t + \gamma^k \partial_k = \gamma^k \partial_k
\end{aligned}
\end{equation}

and in particular

\begin{equation}\label{eqn:biotSavart:60}
\begin{aligned}
\gamma_0 \grad F
&= \gamma_0 \gamma^k \partial_k F \\
&= \gamma_k \gamma_0 \partial_k F \\
&= \sigma_k \partial_k F \\
&= \spacegrad (\BE + I c \BB) \\
\end{aligned}
\end{equation}

and for the RHS,

\begin{equation}\label{eqn:biotSavart:80}
\begin{aligned}
\gamma_0 J/\epsilon_0 c
&=
\gamma_0 (c \rho \gamma_0 + J^k \gamma_k)/\epsilon_0 c  \\
&=
(c \rho - J^k \sigma_k)/\epsilon_0 c  \\
&=
(c \rho - \Bj)/\epsilon_0 c  \\
\end{aligned}
\end{equation}

So we have

\begin{equation}\label{eqn:biotSavart:spacetimeSplitSteadyState}
\begin{aligned}
\spacegrad (\BE + I c \BB)
&=
\inv{\epsilon_0}\rho - \frac{\Bj}{\epsilon_0 c}
\end{aligned}
\end{equation}

Selection of the (spatial) vector grades gives

\begin{equation}\label{eqn:biotSavart:100}
\begin{aligned}
I c (\spacegrad \wedge \BB) &= - \frac{\Bj}{\epsilon_0 c}
\end{aligned}
\end{equation}

or with \(\Ba \wedge \Bb = I (\Ba \cross \Bb)\), and \(\epsilon_0 \mu_0 c^2 = 1\), this is the familiar Ampere-Maxwell equation when \(\PDi{t}{\BE} = 0\).

\begin{equation}\label{eqn:biotSavart:AmpereMaxwellSteady}
\begin{aligned}
\spacegrad \cross \BB &= \mu_0 \Bj
\end{aligned}
\end{equation}

\subsection{Three vector potential solution}

With \(\spacegrad \cdot \BB = 0\) (the trivector part of \eqnref{eqn:biotSavart:spacetimeSplitSteadyState}), we can write

\begin{equation}\label{eqn:biotSavart:120}
\begin{aligned}
\BB = \spacegrad \cross \BA
\end{aligned}
\end{equation}

For some vector potential \(\BA\).  In particular, we have in \eqnref{eqn:biotSavart:AmpereMaxwellSteady},

\begin{equation}\label{eqn:biotSavart:140}
\begin{aligned}
\spacegrad \cross \BB
&=
\spacegrad \cross (\spacegrad \cross \BA) \\
&=
-I (\spacegrad \wedge (\spacegrad \cross \BA) ) \\
&=
-\frac{I}{2} (
\spacegrad (\spacegrad \cross \BA)
- (\spacegrad \cross \BA) \spacegrad
) \\
&=
\frac{I^2}{2} (
\spacegrad (\spacegrad \wedge \BA)
- (\spacegrad \wedge \BA) \spacegrad
) \\
&=
- \spacegrad \cdot (\spacegrad \wedge \BA)
\\
\end{aligned}
\end{equation}

Therefore the three vector potential equation for the magnetic field is

\begin{equation}\label{eqn:biotSavart:vecAwithJ}
\begin{aligned}
\spacegrad (\spacegrad \cdot \BA) - \spacegrad^2 \BA &= \mu_0 \Bj
\end{aligned}
\end{equation}

\subsection{Gauge freedom}
\index{gauge freedom}

We have the freedom to set \(\spacegrad \cdot \BA = 0\), in \eqnref{eqn:biotSavart:vecAwithJ}.  To see this suppose that the vector potential is
expressed in terms of some other potential \(\BA'\) that does have zero divergence (\(\spacegrad \cdot \BA' = 0\)) plus a (spatial) gradient

\begin{equation}\label{eqn:biotSavart:160}
\begin{aligned}
\BA = \BA' + \spacegrad \phi
\end{aligned}
\end{equation}

Provided such a construction is possible, then we have

\begin{equation}\label{eqn:biotSavart:180}
\begin{aligned}
\spacegrad (\spacegrad \cdot \BA) - \spacegrad^2 \BA
&=
\spacegrad (\spacegrad \cdot (\BA' + \spacegrad \phi)) - \spacegrad^2 (\BA' + \spacegrad \phi) \\
&=
- \spacegrad^2 \BA'
\end{aligned}
\end{equation}

and can instead solve the simpler equivalent problem

\begin{equation}\label{eqn:biotSavart:blah}
\begin{aligned}
\spacegrad^2 \BA'  &= -\mu_0 \Bj
\end{aligned}
\end{equation}

Addition of the gradient \(\spacegrad \phi\) to \(A'\) will not change the magnetic field \(\BB\) since \(\spacegrad \cross (\spacegrad \phi) = 0\).

FIXME: what was not shown here is that it is possible to express any vector potential \(\BA\) in terms of a divergence free potential and a
gradient.  How would one show this?

\subsection{Solution to the vector Poisson equation}
\index{Poisson equation}

The solution (dropping primes) to the Poisson \eqnref{eqn:biotSavart:blah} is

\begin{equation}\label{eqn:biotSavart:200}
\begin{aligned}
\BA = \frac{\mu_0}{4 \pi} \int \frac{\Bj}{r} dV
\end{aligned}
\end{equation}

(See \citep{schwartz1987pe} for example.)

The magnetic field follows by taking the spatial curl

\begin{equation}\label{eqn:biotSavart:220}
\begin{aligned}
\BB
&= \spacegrad \cross \BA \\
&= \frac{\mu_0}{4 \pi} \spacegrad \cross \int \frac{\Bj'}{\Abs{\Br-\Br'}} dV'
\end{aligned}
\end{equation}

Pulling the curl into the integral and writing the gradient in terms of radial components

\begin{equation}\label{eqn:biotSavart:240}
\begin{aligned}
\spacegrad &= \frac{\Br - \Br'}{\Abs{\Br - \Br'}} \PD{\Abs{\Br - \Br'}}{}
\end{aligned}
\end{equation}

we have
\begin{equation}\label{eqn:biotSavart:260}
\begin{aligned}
\BB
&= \frac{\mu_0}{4 \pi} \int
\frac{\Br - \Br'}{\Abs{\Br - \Br'}} \cross \Bj' \PD{\Abs{\Br - \Br'}}{}{\inv{ \Abs{\Br-\Br'} }} dV' \\
&= -\frac{\mu_0}{4 \pi} \int
\frac{\Br - \Br'}{\Abs{\Br - \Br'}^3} \cross \Bj' dV' \\
\end{aligned}
\end{equation}

Finally with \(\Bj' dV' = I \jCap' dl'\), we have

\begin{equation}\label{eqn:biotSavart:280}
\begin{aligned}
\BB(\Br) &= \frac{\mu_0}{4 \pi} \int dl' \jCap' \cross \frac{\Br - \Br'}{\Abs{\Br - \Br'}^3}
\end{aligned}
\end{equation}
